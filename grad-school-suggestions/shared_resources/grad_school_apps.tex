%%%%%%%%%%%%%%%%%%%%%%%%%%%%%%%%%%%%%%%%%
\subsection{\hspace{0.1in} Zhiyang's Suggestions for Graduate School Applications}
\label{zygradschapps}

Notes about applying to graduate school: \vspace{-0.3cm}
\begin{enumerate} \itemsep -4pt
\item Unless you go to a good research university in the US, you have to make trade-offs. So, it's a matter of picking a more appropriate trade-off for you, rather than for anybody else. Each person would naturally prefer a different trade-off and perform better in certain types of trade-offs. Hence, your task is to find out what trade-offs do you have to make, and make them so that you can be part of an academic environment that poorly resembles those at good U.S. research universities...  If you have not published good research papers, like some undergrads in the US, you don't have much of a choice. And, you can't effectively formulate this as a multi-objective optimization problem and find the most suitable Pareto-optimal trade-off because of your ignorance, bias, and prejudices.
\item As a general guide, a lot of the NCAA Division I schools (especially research universities that are good at sports) are also good or average research universities. Think about Michigan, UT Austin, University of Florida, University of Illinois at Urbana-Champaign, UCLA, USC, University of North Carolina at Chapel Hill, University of Virginia, University of Massachusetts at Amherst.
\item Personally, I would say pick a program based on its technical excellence. Join the best lab in your preferred research area/topic. If it can provide you with a good broad and deep base via interdisciplinary classes and rare advanced classes in your research topic, go for it. As far as the culture and environment of the campus goes, you can adapt to it. If you are smart enough to do a Ph.D., you can find workarounds or poor substitutes for different problems and challenges that you would face. You may not always or regularly find workarounds or poor substitutes, but you can.
\item Things that you may want to look into: \vspace{-0.3cm}
	\begin{enumerate} \itemsep -2pt
	\item Availability of research lab and professors in area of research interest
	\item Availability of very advanced classes for grad students that are hardly offered in the world. For example, a class on satisfiability modulo theories (SMT), neural implant engineering (alternatively known as neural prostheses design, neuroprosthetics design, or neural prosthetics design), neural image processing, or sequential equivalence checking may only be offered by 1-5 universities in the world. These classes should have 2-4 big projects, a weekly/fortnightly assignment, $\geq$1 midterm (mid-semester exam), and (perhaps) a final (exam). You should look to see if programs have several graduate classes in your research area of interest; good Ph.D. programs have several graduate classes for their Ph.D. students to help them gain technical knowledge in their research interests/topics (via the big projects using industrial tools and tool/work flows for EE/CS grad students). These classes have small numbers of students (e.g., 5-30), and have high class numbers (e.g., EE 290A at Berkeley or EE 681 at USC)
	\item Are there enough advanced classes for grad students to help you with learning additional information and skills (this may not be that helpful in outside science and engineering), especially for subject material that is hard to pick up on your own? What is the quality of the coursework involved like? In terms of classes to be taken, how much is the academic workload for Ph.D. students?	
	\item For the research labs that you want to join, how many of the grad students get funded? Does the professor help her/his students get scholarships, fellowships, and internships? Have any of the professor's students or alumni win best paper awards, best dissertation awards from organizations such as ACM, and prestigious Ph.D. and postdoc fellowships from Microsoft, Intel, IBM, NVIDIA, and Google? How accomplished are the professors that you wanna work with? If you want to be a professor, how many of the professors's former students (including alumni from her/his current research lab) end up as tenured faculty in world-class universities? If you wanna become a research scientist, how many of them are working as research scientists at IBM, Intel, Sun Microsystems, HP, Bell Labs, Google, Yahoo, Microsoft, NVIDIA, or Adobe?
	\item Availability of funding (for the project and your grad student stipend)
	\item Number of professors in research area (preferably 2-6), which is more general than a research topic
	\item Start-up/entrepreneurship and industrial involvement (created, co-founded, or funded start-ups and collaborate with industry on research projects). Pay attention to professors who get funding from big companies or consortiums in your research area. E.g., Google or Intel
	\item Support and encouragement for interdisciplinary research
	\item Does the culture of the research lab suit you? You can guess the culture of the lab by looking at the web pages of the professor(s) and students in the lab. Do they use programming languages that most people have difficulty learning or are too lazy to learn (e.g., Clojure) or software development tools that average students do not use (e.g., GNU Autotools)? Do they use a UNIX-like OS? Is the lab diverse? As in, are they mostly Indian, Persian, or Korean?
	\item Weather of the city/town
	\item Campus safety (Are there mad people running around with guns?) as well as safety in the surrounding communities
	\item Support for student groups, such as international students, women, and minorities. This also includes support groups and staff teams on campus, such as the departments dealing with residential education, support for sexual assault survivors, support for students whom have to deal with deaths of loved ones, and support for the significant others of Ph.D. students.
	\item Amount of stipend for grad students
	\item Student services, including medical, dental, and counseling services
	\item Dining options on and around campus. Trust me, you don't want to eat goop and rice everyday. Also, if you have to stay up till 4 or 6 am to work on a class project or conference paper, can you get food (e.g., a sandwich, burger, or burrito) and coffee/tea or energy drinks (e.g., Red Bull or Rockstar) at 3 am?
	\item Environment of the campus and neighborhood. Does it have a nice arboretum that you can walk in, just like those at Berkeley, Stanford, or UCLA? Psst, it's a nice place to walk around to chill out with a date, to run in, or just to take your mind off research.
	\item Social life on campus and in the town/city. Do you want to be single for several more years?
	\item Student housing. Can you stay in a dorm or residential college, which have various activities to help you develop academically, personally, and professionally? Events at the residential colleges that I have been in include career talks by engineers from Motorola, a visit to the university's supercomputer center and research lab in computer vision, tutorials to provide academic support for students, seminars on writing your resume/CV (there is a difference between the CV and resume in the US), seminars on applying for internships as international students, and many cultural and social events. If you prefer to live in an apartment, can you find affordable apartments on campus or close to campus? Does the campus have excellent social events for students, including those in the arts and music?
	\item The dominant language and culture on campus (a Taiwanese university will issue Chinese names to students who do not have one), and the surrounding neighborhood around campus and in the town/city.
	\item Social, economic, and political stability of the country.
	\item Immigration issues for international students.
	\item Some advisors are crazy, and treat you like their property/slave. Are there any professors in the Ph.D. program who are like that? Can you find out if the professor who you want to work with is supportive of students who desire to get married during their Ph.D. programs, or female students whom get pregnant then, and students whom have to deal with deaths of loved ones? With regards to the first two cases, I am not suggesting that you make an absolute decision about whether you will get married or have a child during the Ph.D. program and stick to it. What I would want to know is that if I decide to get married during my Ph.D. candidature, regardless of my current decision about getting married as a grad student, I would like to know if my advisor would disapprove of it. For female grad students, it would be nice to have an advisor who cares about your personal needs and desires as well as your academic and research progress.
	\end{enumerate}
\item Good research universities will satisfy most of your preferences for the above considerations (in the sub-list, ``Things that you may want to look into''). As you can see, you will miss out on a lot when you are not studying in a world-class research university in the US. Hence, you should find a Ph.D. program that allows you to have as many of these benefits/advantages as possible. I would encourage you to focus on satisfying your considerations for the first few points that are relevant to the classes that you will take and research that you will carry out.
\item In summary, examine your personal and professional goals and desires, skills and knowledge, as well as strengths and weaknesses. Subsequently, determine which Ph.D. program and higher education system (e.g., US or European system) is more suitable for you to become a good researcher in your chosen field (e.g., computer science) and research area/topic (e.g., electronic design automation / logic synthesis). Next, apply to programs that are more suitable for you. When you get your Ph.D., if you are considering a career in academia, you can apply the same method to select research labs that you may want to join as a postdoc.
\item In \proc{Apply-to-Grad-Schools}, when no research area is preferred [lines \ref{apply-grad-sch-no-research-area1}-\ref{apply-grad-sch-no-research-area2}], you may want to apply to the best Ph.D./graduate program that you can get into. This requires you to estimate the rank of the Ph.D. programs under consideration. This means that the ranking that you use would be based on the (biased) opinions of others. You can attempt to rank the universities on your own by obtaining your own data. However, this involves a lot of work, and would be an overkill... [lines \ref{apply-grad-sch-no-research-area1}-\ref{apply-grad-sch-pick-research-area}] A lot of US Ph.D. programs don't require you to work only in the research labs that you mention in your statement of purpose. In Europe, you must convince a professor (who is leading the research group/lab that you want to join) or the admissions committee that a particular research topic interests you by explaining why it interests you... [line \ref{apply-grad-sch-db-ref}] Use reference management software to manage the database of references. Examples of reference management include {\it Mendeley}, {\it JabRef}, and {\it BibDesk}... [line \ref{apply-grad-sch-international-deadlines}] Note that US Ph.D. programs tend to have deadlines in December/January; European Ph.D. programs have different deadlines, depending on the country and the university. Canadian Ph.D. programs have deadlines from December till March, Taiwanese Ph.D. programs have deadlines in March and April, and Ph.D. programs at the Technion in Israel have deadlines on April 15 and Sep 15... [line \ref{apply-grad-sch-types-of}] By applying to reach schools, match schools, followed by safety schools, I would have some room for error with respect to match schools. If I mess up an earlier application, I can learn from my mistakes... [line \ref{apply-grad-sch-LaTeX}] Use comments in your \LaTeX\ documents to comment/uncomment previously written sections that can be reused later... [line \ref{apply-grad-sch-rcs}] Examples of revision control systems include {\it CVS}, {\it Subversion}, {\it Git}, and {\it Mercurial}... [line \ref{apply-grad-sch-rcs}] When you apply to labs in a given research area, some of the projects in these labs may be similar. Hence, you can reuse information from previous SOPs. However, always customize your SOP... 
\end{enumerate}


\begin{codebox}
\Procname{$\proc{Apply-to-Grad-Schools}()$}
\zi	\Comment {\it Input : None}
\zi	\Comment {\it Output : None}
\zi
\li \If no-research-area-preferred	\label{apply-grad-sch-no-research-area1}
\li	\Then Apply to the best Ph.D. program that you can get into	\label{apply-grad-sch-no-research-area2}
\li	exit					\Comment {\it End of procedure}
%\zi \ElseIf you-want-to-work-in-a-research-area-that-you-like
\li \ElseIf you want to work in a research area that you like
\zi	\Then
	\Comment {\it x := number of weeks for literature review}
\li	int x $\gets$ 4;
\zi
\li	\While $x > 0$
	\Do
\li		Read technical magazines to get an idea of research trends
\zi		\Comment {\it E.g., IEEE Computer magazine or Communications of the ACM}
\zi
\li		\If (topic is too difficult/challenging for me)
		\Then
\li			ignore that topic
\li		\ElseIf (topic is interesting and doable)
		\Then
\li			take note of that topic
\li			read journal and conference papers of that topic
\li			for each paper, enter it into your database of references		\label{apply-grad-sch-db-ref}
\zi			
\li			\If (a professor's papers are very interesting)
			\Then
\li				apply to that lab
			\End
		\End
\zi
\li		narrow down list of research topics to 2-3 related topics
\zi		\Comment {\it use the 2/3 topics for grad school applications}
\li		x--;	\label{apply-grad-sch-pick-research-area}
	\End
\zi \End

\li \If (research area has been selected)
	\Then
\li	Decide which higher education system do I prefer (e.g., US/Europe)
\li	Determine the academic requirements and application deadlines		\label{apply-grad-sch-international-deadlines}
\li	Group research labs or Ph.D. programs into match schools, safety schools, \& reach schools
\zi	\Comment {\it Most of your programs that you want to apply to should be match schools}
\zi	\Comment {\it Apply to some reach schools and safety schools}
\zi	\Comment {\it Or group them into reach, target, and reliable schools.}
\zi
\li	Put the deadlines on a digital calendar	\Comment {\it E.g., use Google Calendar or iCal}
\li	Create a prioritized timeline of the deadlines		\Comment {\it A Gantt chart will help}
\li	Apply to reach schools, match schools, followed by safety schools	\label{apply-grad-sch-types-of}
\li	Use LaTeX to write your resume/CV, and statement of purpose (SOP)	\label{apply-grad-sch-LaTeX}
\li	Use a revision control system to keep track of the changes in your \LaTeX\ documents	\label{apply-grad-sch-rcs}
\zi \End
\end{codebox}