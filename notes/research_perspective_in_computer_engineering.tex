%%%%%%%%%%%%%%%%%%%%%%%%%%%%%%%%%%%%%%%%%%%%
\subsection{\hspace{0.1in} Research Perspective in Computer Engineering: November 5, 2010}
\label{computerengrresearch}

Let me begin by quoting the president of USC, C. L. Max Nikias, who has echoed the importance of having good breath and depth in our higher education that many people have. These people include author Thomas Friedman, film director Spike Lee, basketball legend Michael Jordan, and many, many professors and researchers. Many of them would also encourage youths and young adults, like us, to gain living and studying/working experiences in multiple cultural environments (or countries, if you like).  \\

``Intellectual fire is generated by interdisciplinary sparks. The real action, the real invention happens where branches of knowledge converge. We can move backward and forward through the maze of human knowledge, but it is the thread of interdisciplinary work that holds it all together, that helps us find our way, that provides us with real meaning. So, I urge you to break down the barriers between disciplines. Reach out beyond your department. Reach out beyond your school. Seize opportunities to collaborate with your colleagues from other fields.'' \\
``When I was a Ph.D. student, I worked very hard! I read all the papers of my field and those related to my field. I reached a point where I knew the literature better than my advisor. This is what you have to do!'' \\
-- C. L. Max Nikias, ``USC: A Place of Innovation'' Address to Doctoral Students and Postdocs. Available at: \url{http://www.president.usc.edu/speeches/usc-a-place-of-innovation-address-to-doctoral-students-and-postdocs.html}; last accessed on September 5, 2010. \\
  
Also, I am gonna draw from material that I have written in my Facebook notes, ``Resources for and Thoughts on Complex Systems'' and ``An open note to Mushegh on career opportunities in statistics or data mining/analysis''. \\


%%%%%%%%%%%%%%%%%%%%%%%%%%%%%%%%%%%%%%%%%%%%
\subsubsection{\hspace{0.1in} Preliminary Thoughts on Computer Engineering}
\label{computerengpreliminarythoughts}
  
I may be wrong, but I strongly believe that computer engineering involves more than computer architecture. It would also include operating systems, compiler design, computer networking, and parallel computing. \\ %The computer engineering Ph.D. program at USC focus on the hardware aspects of computer engineering, while leaving it up to the discretion of individual Ph.D. students in computer engineering to take classes in areas such as computer engineering to help prepare themselves for careers in computer engineering.

In general, a computer engineering Ph.D. student is expected to be able to design and implement a processor. This includes microarchitectural implementations of the processor at RTL, and at logic level; contemporary EDA tools as well as problems associated with designing, verifying, and testing VLSI systems using nanoscale process technologies render transistor-level circuit design irrelevant for computer engineers. A computer engineering Ph.D. student should understand the hardware/software interface well. After all, by definition, computer engineers are expected to deal with the hardware/software interface, right? So, a computer engineering Ph.D. student should take classes in operating systems, compiler design, computer networking, concurrent programming, and parallel computing (hardware aspects) at the advanced undergraduate or junior graduate level. \\

If you talk to world-renowned experts (top 3-5 people in the field) in computer architecture, operating systems, compiler design, computer networking, and/or parallel computing, you would find that they share this approach. E.g., look at the webcasts of Prof. Saman Amarasinghe's compiler design class at MIT on iTunes U. He did discuss career prospects of students who have taken his class on compiler design; most computer science and engineering students who have taken a class on compiler design may not develop compilers as computer scientists/engineers, but work in areas that allow them to take advantage of their knowledge in compiler design (e.g., microarchitecture design).



%%%%%%%%%%%%%%%%%%%%%%%%%%%%%%%%%%%%%%%%%%%%
\subsubsection{\hspace{0.1in} Contemporary and Emerging Trends in the IT/Software, Semiconductor, and Electronics Industries}
\label{trendsinitseminelectronics}

Now, the IT/software, semiconductor, and electronics industries are at a point of inflection because of problems associated with scaling. It is important to bear in mind that problems associated with scaling occur at two ends of the spectrum: we have difficulties scaling down the size of devices, and we have problems scaling up systems (think system integration). This leads to the following trends, exacerbated by the forces of globalization as well as social, economic, and political trends.: \vspace{-0.3cm}
\begin{enumerate} \itemsep -4pt
\item Moore's Law affects the way we approach our research topics. We can't even have researchers in semiconductor process technology, device engineering, and technology CAD (TCAD) tell us how can we reliably scale devices below 20 nm. Thus, we cannot determine how will architecture design, RTL design, logic design, and transistor-level circuit design and design methodologies be affected by emerging process technologies and semiconductor devices, or rather computational devices (which can be organic, rather than semiconductors).

Researchers in computer architecture, VLSI design, and EDA cannot do much about this, since problems concerning this is outside their research scope. However, they should work with people in process technology, device engineering, and technology CAD to develop design methodologies. Intel does this very well. They bring experts from process technology, device engineering, technology CAD, computer architecture, VLSI design, and EDA together when they start to develop a new process technology node.
\item System integration becomes harder, especially when we want to integrated so many hardware and software components/features into the same electronic system. Dealing with limits in thermal dissipation, power consumption, and performance improvement makes this much harder. Complicating this would be problems associated with process variation (think Design for Manufacturability, DFM). Look at papers from design for variation and reliability, as well as papers from fault tolerance.

An effect of this is the shift towards multi- and many- core processors. Complicating this effect is the emergence of GPGPU computing as a viable alternative to traditional high-performance computing with multi-processors. There exists many interesting research topics covering parallel computing and network-on-chips. For example, USC alum, Prof. Radu Marculescu, and his former Ph.D. student, {\"{U}}mit Y. Ogras, have done very interesting work on network-on-chip design for thousand-core processors under a SRC-funded project with Intel Research. Prof. Radu Marculescu's wife, Prof. Diana Marculescu, is also an electrical and computer engineering professor at Carnegie Mellon University, and a fellow alum of USC.
\item Hardware design and software development are becoming more interrelated. VLSI design is becoming more software oriented. An increasing number of teams in digital VLSI design are using electronic system-level design methodologies, including platform-based design. For example, they may design a processor in SystemC, and also use languages such as C/C++ for verification. This also extends to analog/RF and mixed-signal design and verification; see Verilog-AMS, VHDL-AMS, SystemC-AMS, and SystemC-WMS. E.g., you can design an OFDM receiver based on SystemC-AMS.

Also, better EDA tools, such as high-level synthesis tools, can take you from system-level design at SystemC to layout/GDSII/OASIS. It is even scarier when you realize that automation has made huge impacts in semiconductor manufacturing too. If you have attended Mr. Michael Herring's talk on, ``Automated Manufacturing Technology (AMT) of Intel's semiconductor development and manufacturing facilities,'' you would know that by the late 2000s, Intel has successfully automated semiconductor manufacturing (and possibly also testing and assembly). An Intel foundry may only have 30-50 people monitoring and maintaining Intel's automation technology, rather than have 2 or 3 thousand people engaged in the manufacturing process. This is really scary, and it would fundamentally change the way we view careers in electrical and computer engineering.

Complicating point 3) is point 2). Thanks to Amdahl's law, or rather a generalized version of it, there are limits to how much we can reduce power consumption, improve thermal dissipation, improve fault and error tolerance, and improve reliability via hardware techniques. There is a trend towards using software-oriented approaches to address issues concerning power, fault and error tolerance, and reliability. E.g., energy-aware operating systems or power-aware compilers. This doesn't mean that hardware aspects are no longer important, but that we need both hardware and software techniques.

So, we need to look at things holistically, and carry out multi-objective design space exploration. To excel in this, we would need a good broad base, and sufficient depth in 2-3 research topics. This means that we should have taken advanced classes (500/600-level classes at USC) in 2-3 research areas, and we should have taken most of the intermediate classes (400-level classes at USC for advanced undergraduates and junior graduate students) in electrical and computer engineering, as well as computer science (CS). Note that USC's CS classes at the 400-level do not cover many traditional topics in CS, such as compiler design, formal methods/verification, and topics in theoretical computer science such as computational geometry and computational complexity that many undergraduates in other good CS undergraduate programs have. This is a result of the choice that USC's CS professors have made in designing the undergraduate CS curriculum.

Hence, some interesting topics in computer architecture include, OS-aware architecture design (OS refers to operating systems) or compiler-aware microarchitecture design. Remember how the hardware/software interface involves the ISA, compilers, and operating system?
\item Interesting developments and trends in interdisciplinary fields: \vspace{-0.3cm}
	\begin{enumerate} \itemsep -2pt
	\item There is an increasing awareness of complex systems among lay people. Even visual artists and graphic designers are looking at visual representations of complex systems as abstract art. Musicians are dabbling into complex systems as well. If you read my Facebook note on complex systems, you can find many interesting resources on complex systems, particularly with applications in electrical and computer engineering (ECE) as well as CS. E.g., an interesting complex system would be networks of thousand-core processors. How do you design, verify, and test such networks? How would you like to program this beast? How would you analyze the performance of this monster?
	\item Emergence of ambient intelligence and pervasive/ubiquitous computing, including wearable computing, wireless embedded devices (e.g., smart phones, tablet computers), and home automation.
	\item Nano-Bio-Info technologies (some may prefer Nano-Bio-Info-Cogno technologies). Solving many challenging problems that we have today involve taking an interdisciplinary approach. While researchers have addressed problems with this paradigm before, things are only gonna heat up.
	\item Hybrid and heterogenous systems. This goes beyond VLSI systems with analog/RF, mixed-signal, and digital circuits, or even hardware/software systems (i.e., embedded systems). It also includes systems with mechanical, optical, chemical, and biological components. Think about NEMS, MEMS, bioMEMS, MOEMS, DNA computing, robotics, and synthetic biology. Prof. Giovanni De Micheli and Prof. Jan Rabaey have given keynote speeches on this at the Design Automation Conference (Prof. Rabaey in 2007), International Conference on Computer-Aided Design (Prof. De Micheli in 2008) and Design Automation and Test in Europe conference (Prof. De Micheli again, in 2008). See my Facebook note on complex system, or collection of Facebook links, to access video clips of their talks. This involves integrating approaches in the continuous and discrete domains, such as techniques in symbolic-numeric computation.
	\item Hybrid AI techniques that integrate logical AI and statistical. This can change the way we look at computation/computing.
	\end{enumerate}
\end{enumerate}


%%%%%%%%%%%%%%%%%%%%%%%%%%%%%%%%%%%%%%%%%%%%
\subsubsection{\hspace{0.1in} Globalization, the Economic/Financial Crisis of the Late 2000s, and Research}
\label{globalizationfinancialcrisis2008research}

Now, when we account for how globalization has changed the way we live, things become incredibly exciting and interesting for people who have good breath and depth in their higher education, and who also have good (cross-/inter-) cultural competence. For some other people, this is incredibly daunting, challenging, and scary. Exacerbating this would be the financial/economics crisis of the late 2000s, which some would argue is ongoing. \\

If you had bothered to do good literature reviews of your past and current research topics, you may have noticed that elite teams of researchers exist in places and universities that most people, including myself, have not heard of. E.g., the University of Trento (in Trento, Italy) has developed an SMT solver (solver for Satisfiability Modulo Theories) that is better than those from SRI/Stanford, Berkeley, Microsoft Research, IBM Research, Intel Labs/Research, and Carnegie Mellon University. In terms of incremental, short-term research, UFRGS in Porto Alegre, Brazil and National Taiwan University (in Taipei, Taiwan) have Ph.D. students in EDA that are comparable to those in any good research lab in EDA at good US research universities. Ditto for teams at Universitat Polit{\`{e}}cnica de Catalunya (in Barcelona, Spain) and Johannes Kepler Universit{\"{a}}t (in Linz, Austria). \\

When companies are hiring VLSI design engineers in Armenia with skills comparable to electrical and computer engineering graduate students at USC (or EDA companies doing likewise in Chile), you know that the way we do business and live has fundamentally changed. It's not just about Brazil, Russia, India, and China. It's about the stuff that Thomas Friedman has been talking about. Anybody from anywhere with access to education, technology, and information can make an impact in this world. \\

So, the challenge for us junior researchers is to figure out how to get our research recognized. Faculty search committees in good US research universities want lots of high quality research papers. Ditto for directors and researchers of corporate research labs, like Intel Labs or IBM Research. If you cannot obtain that, what do you prefer, more lower-quality papers or less upper-quality papers? \\
 
Is it unrealistic to shoot for best paper awards, first prize in programming contests or VLSI design contests, and best doctoral/Ph.D. dissertation awards? Yes? No? Maybe???