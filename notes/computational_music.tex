%%%%%%%%%%%%%%%%%%%%%%%%%%%%%%%%%%%%%%%%%%%%
\subsection{\hspace{0.1in} Computational Music (or Music Computation) and Music Synthesis}
\label{computationalmusic}

I was first introduced to evolutionary music when I was looking into evolutionary computation, which is a field in artificial intelligence (AI), a couple of years back. That is, computer scientists are creating music using AI. \\
 
Dr. Peter J. Bentley (\url{http://www.cs.ucl.ac.uk/staff/p.bentley/}) leads the Digital Biology Interest Group (\url{http://www.cs.ucl.ac.uk/staff/P.Bentley/igdigitalbiology/index2.html}) at University College London (UCL). In an interview with BBC for his book, ``Digital Biology,'' he demonstrates how evolutionary algorithms (a technique in AI) can be used to create music (``evolutionary'' jazz music) that can complement a jazz musician, or improvise and lead a jazz band. \\
 
The AI software plays jazz along with musicians, and ``adapts'' to complement the musician during its ``evolution'' by tuning in to the way ``its'' band members play jazz music. Alternatively, the AI software can ``evolve'' and improvise its jazz piece during its ``evolution'', while real jazz musicians (people) can accommodate its style of music (just like musicians would accommodate John Coltrane or Miles Davis when they improvise). \\
 
You can listen to a sample of this demonstration at the end of the interview that is provided in the audio clip at: audio clip for evolutionary jazz music (\url{http://www.peterjbentley.com/digitalbiology3.html}). \\
 
I think this is neat. AI software for evolutionary music allows musicians to play cool stuff when they are alone or when one or two members of a jazz band is missing. What may surprise the musicians is that the evolutionary process of such AI software is stochastic in nature... It generates notes randomly, and selectively adds them to synthesize jazz music based on rules/grammar in music theory for a given genre. \\
 
Other examples and resources of evolutionary music are MusiGenesis (\url{http://www.musigenesis.com/}), Evolutionary Sound System (\url{http://www.mcld.co.uk/evolutionarysound/}), GeneticDrummer (\url{http://dostal.inf.upol.cz/evm.html}), Evolutionary Music Bibliography (\url{http://www.ist.rit.edu/~jab/EvoMusic/EvoMusBib.html}), and ``Evolutionary Computation and its application to art and design'' by Craig Reynolds (\url{http://www.red3d.com/cwr/evolve.html}). \\
 
 
 
Also, see articles from Wikipedia on Algorithmic composition (\url{http://en.wikipedia.org/wiki/Algorithmic_composition}), Pop Music Automation (\url{http://en.wikipedia.org/wiki/Pop_music_automation}), Computer music (\url{http://en.wikipedia.org/wiki/Computer_music}), Generative music (\url{http://en.wikipedia.org/wiki/Generative_music}), and Evolutionary music (\url{http://en.wikipedia.org/wiki/Evolutionary_music}). \\
 
 
 
Around this time, I noticed USC's dominance in research on entertainment technology and computational music (or music computation). \\
 
Later, when I was reading about Joseph Costello (former CEO of Cadence Design Systems), I learned of electronic implementations (as VLSI systems, or hardware if you like; \url{http://en.wikipedia.org/wiki/Sound_synthesis}) of speech (\url{http://en.wikipedia.org/wiki/Speech_synthesis}) or sound synthesis (\url{http://en.wikipedia.org/wiki/Software_synthesizer})... In the June 2010 edition of the IEEE Transactions on Very Large Scale Integration Systems (\url{http://ieeexplore.ieee.org/servlet/opac?punumber=92}), there is a journal paper on a VLSI design of a ``scalable and programmable sound synthesizer'' (\url{http://dx.doi.org/10.1109/TVLSI.2009.2017197}). This is pretty cool! \\
 
Recently, I encountered a reference from the June 25, 2010 edition (\url{http://technews.acm.org/archives.cfm?fo=2010-06-jun/jun-25-2010.html}) of ACM TechNews (\url{http://queue.acm.org/technews.cfm}) about ``a computer program (\url{http://www.csmonitor.com/Innovation/Tech/2010/0617/How-a-computer-program-became-classical-music-s-hot-new-composer}) that composes classical music by following rules of music its [computer] programmer taught it''. Ain't that awesome? An AI software that can compose classical music! \\
 
Some resources for computational music (or music computation) include: USC's Music Computation and Cognition (MuCoaCo) group (\url{http://www-bcf.usc.edu/~mucoaco/}), the MCM Workshop on Computational Music Analysis (\url{http://www.neurogems.org/music/index.html}), and Computational Music for Indian classical music (\url{http://www.computationalmusic.com/}). \\
 
Some resources for music synthesis include: the ``The Amateur Gentleman's Introduction to the Principles of Music Synthesis'' by Beau Sievers (\url{http://beausievers.com/synth/synthbasics/}); Stanford University's Center for Computer Research in Music and Acoustics (\url{https://ccrma.stanford.edu/groups}); CERL Sound Group, Computer-based Education Research Laboratory at the University of Illinois at Urbana-Champaign (\url{http://www.cerlsoundgroup.org/main.html}); and Dr. Lippold Haken's web page (\url{http://www.cerlsoundgroup.org/CSGdesc/LippoldHome.html}). \\
 
Also, see \url{http://www.acs.psu.edu/users/smithsh/synth.html} and \url{http://www.informatics.sussex.ac.uk/users/nc81/courses/cm1/cm1bibliography.html} for further information on computational music (or music computation) and music synthesis. \\


\vspace{1cm}
\ \\
Claim: \\
There is art in engineering and computer science. \\
 \ \\
Proof by existence: \\
Look at the conference proceedings for the track on ``Artificial Life Models for Musical Applications II : Searching for musical creativity'' (\url{http://galileo.cincom.unical.it/esg/Music/ALMMAII/ALMMAII_file/home.html}) at the Workshop of the $8^{th}$ International Conference on the Simulation and Synthesis of Living Systems (ALife VIII), which is held in 2002 at the University of New South Wales in Sydney, Australia. \\
\ \\
Lemma: \\
The aforementioned audio clip found on Dr. Bentley's web page. \\
\ \\
Conjecture: \\
Computer scientists can replace musicians with computers and software... It is preposterous of me to suggest this conjecture, and I apologize for any grievance caused. \\
\vspace{1cm}

Yes!!! Computer scientists have used artificial intelligence to create music - opera, jazz, and classical music. And, also poems, jokes, stories, and visual art (e.g., evolutionary art [\url{http://en.wikipedia.org/wiki/Evolutionary_art}] or algorithmic art [\url{http://en.wikipedia.org/wiki/Algorithmic_art}])... \\


%%%%%%%%%%%%%%%%%%%%%%%%%%%%%%%%%%%%%%%%%%%%
\subsubsection{\hspace{0.1in} Addendum \#1: Computational Music}
\label{compmusicaddendum1}

See \url{http://www.nsf.gov/news/special_reports/science_nation/index.jsp} or \url{http://www.nsf.gov/news/special_reports/science_nation/musicman.jsp} for a project where electrical engineers collaborate with musicians. That is, the musicians will play their musical instruments, while the technologies (developed by the electrical engineers) will transcribe the music and subsequently synthesize that music to accompany the musician(s). \\

 
Prof. Mark Bocko from the University of Rochester's Department of Electrical and Computer Engineering works on: \vspace{-0.3cm}
\begin{enumerate} \itemsep -4pt
\item musical acoustics (physics of wind instruments)
\item automated transcription
\item model-based music synthesis
\item musical sound representations
\item audio watermarking and steganography
\end{enumerate}

Prof. Bocko mentions that the audio/data compression technologies developed in his research projects can be used to enable musicians perform in geographically distributed locations. This requires technologies to synchronize the musicians, so that they can play in harmony. That said, USC had been working on this for several years at least. Check out the ``Distributed Immersive Performance'' (DIP) project at: \url{http://imsc.usc.edu/dip/index.html}. Related projects include ``Network Stream Reliability and Synchronization'' (\url{http://imsc.usc.edu/research/project/netstream/index.html}) and ``Video Streaming Over the Internet'' (\url{http://imsc.usc.edu/research/project/videostream/index.html}). \\
 
 
You can check out other cool research on entertainment technologies at Integrated Media Systems Center: \url{http://imsc.usc.edu/}




%%%%%%%%%%%%%%%%%%%%%%%%%%%%%%%%%%%%%%%%%%%%
\subsubsection{\hspace{0.1in} Addendum \#2: Resources for Computational Music}
\label{compmusicaddendum2}

Resources for computational music: \vspace{-0.3cm}
\begin{enumerate} \itemsep -4pt
\item Texas A\&M University: \vspace{-0.3cm}
	\begin{enumerate} \itemsep -2pt
	\item Philip Galanter. Available online at: \url{http://philipgalanter.com/about/}; last accessed on November 3, 2010.
	\end{enumerate}
\item University of Coimbra: \vspace{-0.3cm}
	\begin{enumerate} \itemsep -2pt
	\item Department of Informatics Engineering: \vspace{-0.2cm}
		\begin{enumerate} \itemsep -2pt
		\item Penousal Machado: \vspace{-0.1cm}
			\begin{enumerate} \itemsep -1pt
			\item \url{http://eden.dei.uc.pt/~machado/}
			\item Evolutionary Art
			\item Computational Creativity
			\item Evolutionary Computation
			\item Artificial Artists
			\item Computer-Aided Creativity
			\end{enumerate}
		\end{enumerate}
	\end{enumerate}
\item Indiana Univesity, Bloomington: Music Informatics Program, \url{http://www.music.informatics.indiana.edu/}
\item \cite{Romero2008}
\item \cite{Miranda2007}
\end{enumerate}



%%%%%%%%%%%%%%%%%%%%%%%%%%%%%%%%%%%%%%%%%%%%
\subsubsection{\hspace{0.1in} Terms in Computational Music}
\label{compmusicterms}

Terms in computational music: \vspace{-0.3cm}
\begin{enumerate} \itemsep -4pt
\item atonal and aleatoric (randomly generated) music
\item computer generated music
\end{enumerate}

