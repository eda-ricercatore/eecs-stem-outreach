%%%%%%%%%%%%%%%%%%%%%%%%%%%%%%%%%%%%%%%%%%%
\subsubsection{\hspace{0.1in} Grad School Information}
\label{gradschinfo}

Things for me to look into when considering graduate and professional programs: \vspace{-0.3cm}
\begin{enumerate} \itemsep -4pt
\item Is the graduate or professional program accredited?
\item Would attending the graduate/professional program help me obtain my career goals?
\item Would attending the graduate/professional program help me obtain my career goals?
\item See \url{http://gradschool.princeton.edu/facts/time_to_degree/naturalsciences/} for statistics of graduates from various programs at Princeton University.
\item Some statistics that you may wanna look into include, average number of Ph.D. students per Ph.D. advisor (tells you about advisor-to-advisee ratios), number of Ph.D. graduates for the Ph.D. program per year (tells you about the size of the department), graduate outcomes (what did alumni do after their Ph.D. programs), drop-out rate, failure rate for Ph.D. preliminary and qualifying examinations, and reasons for dropping out of the Ph.D. program.
\item Some research labs do list the names of alumni members (Ph.D. and Masters students, undergraduates, and postdocs). They may also include their first vocation after their Ph.D. program, and what are they currently doing. You can use this to help you gauge the strengths/weaknesses of the lab/advisor.
\item Accommodation options. Is it hard to find housing in the surrounding area? Is housing in the surrounding area cheap? What are some of the statistics of the local population? LA Times has some statistics of local populations in LA. You can determine the average income, highest level of education, political views, and so on... Try to determine if there is an equivalent of that for the local area.
\item 
\end{enumerate}


Grad school info: \vspace{-0.3cm}
\begin{itemize} \itemsep -4pt
%%%%%%%%%%%%%%%%%%%%%%%%%%%%
\item general information and advice about grad school: \vspace{-0.3cm}
	\begin{enumerate} \itemsep -2pt
	\item IEEE: \vspace{-0.2cm}
		\begin{enumerate} \itemsep -2pt
		\item Susan Karlin, ``How to Choose A Grad School: Figure out what you want and who can give it to you,'' {\it IEEE Spectrum}, September 2005. Available at: \url{http://spectrum.ieee.org/at-work/education/how-to-choose-a-grad-school}; last accessed on August 28, 2010.
		\item IEEE Potentials, Volume 21, Issue 3, Aug/Sep 2002.
		\item IEEE Potentials, Volume 24, Issue 3, Aug/Sep 2005.
		\end{enumerate}
	\item ACM: \vspace{-0.2cm}
		\begin{enumerate} \itemsep -2pt
		\item ACM, {\it ACM Crossroads Student Resources}, ACM, New York, NY, Aug 17, 2005. Available at: \url{http://oldwww.acm.org/crossroads/resources/}; last accessed on August 29, 2010.
		\item ACM, {\it Graduate Educational Resources from ACM Crossroads}, ACM, New York, NY, Aug 7, 2005. Available at: \url{http://oldwww.acm.org/crossroads/resources/graduate.html}; last accessed on August 29, 2010.
% http://oldwww.acm.org/crossroads/resources/graduate.html
		\item {\it ACM Crossroads} articles on grad school application and life in grad school: \cite{Agre1997,desJardins1994,desJardins1995,desJardins2001,desJardins2008,Pottinger1999,Wendler2010}.	
		\end{enumerate}
	\item European Commission: \vspace{-0.2cm}
		\begin{enumerate} \itemsep -2pt
		\item Marie Curie Actions: \url{http://ec.europa.eu/research/mariecurieactions/index.htm}... {\bf \color{blue} SCHOLARSHIPS!!!}
		\item {\it EURAXESS} Research Job Vacancies: \url{http://ec.europa.eu/euraxess/index.cfm/jobs/jvSearch} or \url{http://ec.europa.eu/euraxess/index_en.cfm?l1=13&l2=3&initSearch=1#}... {\bf \color{blue} SCHOLARSHIPS!!!}
		\item European Commission, ``Third European Report on Science \& Technology Indicators 2003,'' Directorate-General for Research, European Commission. Available at: \url{http://cordis.europa.eu/indicators/contacts.htm}; last accessed on September 1, 2010. Report on European universities and research initiatives.
		\item Publications by the Directorate-General for Research, European Commission, about research initiatives and output in Europe: \url{http://cordis.europa.eu/indicators/publications.htm}
		\end{enumerate}
	\item American Psychological Association: \vspace{-0.2cm}
		\begin{enumerate} \itemsep -2pt
		\item American Psychological Association, {\it gradPSYCH} [ magazine ], American Psychological Association, Washington, DC. Available at: \url{http://www.apa.org/gradpsych/}; last accessed on September 1, 2010. [ Read issues from May 2003 till January 2005; {\color{blue} \bf This is an excellence source of information about doing well in graduate school and internships, and seeking research careers in academia and the industry.} ]
		\end{enumerate}
	\item Computing Research Association (CRA): \vspace{-0.2cm}
		\begin{enumerate} \itemsep -2pt
		\item Information for Undergraduate and Graduate Students: \url{http://www.cra.org/for-students/}
		\item Computer Research Association's Committee on the Status of Women in Computing Research (CRA-W), ``Graduate Student Information Guide''. Available at: \url{http://www.cra-w.org/sites/default/files/grad-guide.pdf}; CRA-W $\Rightarrow$ Resources $\Rightarrow$ Publications $\Rightarrow$ link and description to the guide, ``Graduate Student Information Guide''; last accessed on September 3, 2010.
		\end{enumerate}
	\item American Mathematical Society: \vspace{-0.2cm}
		\begin{enumerate} \itemsep -2pt
		\item {\it Applying to Graduate School}. Available at: \url{http://www.ams.org/profession/career-info/grad-school/grad-school}; last accessed on September 2, 2010.
		\end{enumerate}
	\item The Mathematical Association of America: \vspace{-0.2cm}
		\begin{enumerate} \itemsep -2pt
		\item {\it MAA Students}. Available at: \url{http://www.maa.org/students/}; last accessed on September 2, 2010.
		\end{enumerate}
	\item American Institute of Mathematics: \vspace{-0.2cm}
		\begin{enumerate} \itemsep -2pt
		\item Resources for the Math Community: \vspace{-0.1cm}
			\begin{enumerate} \itemsep -1pt
			\item \url{http://www.aimath.org/mathcommunity/}
			\item David W. Farmer, ``The AIM REU: individual projects with a common theme,'' in the {\it Proceedings of the Conference on Promoting Undergraduate Research in Mathematics}, American Mathematical Society, 2006. Available online at: \url{http://www.aimath.org/mathcommunity/farmerREU.pdf}; last accessed on January 9, 2010. [ ``AIM Research Experience for Undergraduates (REU)'' ]
			\item Sally Koutsoliotas and David W. Farmer, ``Preparing students to give talks,'' American Institute of Mathematics. Available online at: \url{http://www.aimath.org/mathcommunity/studenttalks.pdf}; last accessed on January 9, 2010. [ ``Preparing students to give talks'' ]
			\end{enumerate}
		\end{enumerate}
	\item University of California, Berkeley: \vspace{-0.2cm}
		\begin{enumerate} \itemsep -2pt
		\item Matthew Moskewicz, Parallel Computing Laboratory (Par Lab), Department of Electrical Engineering and Computer Sciences: \vspace{-0.1cm}
			\begin{enumerate} \itemsep -1pt
			\item Developed the {\it Chaff} SAT solver with Conor Madigan as undergrads that is 10-100X faster than then existing SAT solvers.
			\item He is named a co-winner of the 2009 CAV Award, along with his co-developers of {\it Chaff} and the developers of the {\it GRASP} SAT solver, for his fundamental contribution to the field of Computer Aided Verification.
			\item \url{http://www.princeton.edu/engineering/eqnews/spring01/feature5.html}
			\item \url{http://parlab.eecs.berkeley.edu/people/matthew-moskewicz}
			\end{enumerate}
		\item Mark Borgschulte, ``Economics Grad School Application Advice,'' Department of Economics, University of California, Berkeley. Available online at: \url{http://sites.google.com/site/markborgschulte/economicsgradschoolapplicationadvice}; last accessed on January 9, 2010.
		\item Mark Borgschulte, ``Info for Berkeley Admits,'' Department of Economics, University of California, Berkeley. Available online at: \url{http://sites.google.com/site/markborgschulte/infoforberkeleyadmits}; last accessed on January 9, 2010.
		\item Mark Borgschulte, ``Reading List,'' Department of Economics, University of California, Berkeley. Available online at: \url{http://sites.google.com/site/markborgschulte/readinglist}; last accessed on January 9, 2010.
		\item {\it Secret Blogging Seminar} is a blog written by recent Ph.D. graduates from Berkeley's Department of Mathematics. Noah Snyder, ``Thoughts on graduate school,'' May 13, 2009. Available at: \url{http://sbseminar.wordpress.com/2009/05/13/thoughts-on-graduate-school/}; last accessed on September 1, 2010.
		\item {\it UC Berkeley Career Center}, ``Graduate School - Letters of Recommendation,'' UC Berkeley. Available at: \url{https://career.berkeley.edu/grad/gradletter.stm}; last accessed on September 5, 2010.
		\end{enumerate}
	\item Carnegie Mellon University, Computer Science Department: \vspace{-0.2cm}
		\begin{enumerate} \itemsep -2pt
		\item Carnegie Mellon University, {\it Ph.D. in Computer Science}, Computer Science Department, Carnegie Mellon University. Available at: \url{http://www.csd.cs.cmu.edu/education/phd/index.html}; last accessed on August 28, 2010.
		\item Mark Leone, {\it Advice on Research and Writing}, Computer Science Department, Carnegie Mellon University. Available at: \url{http://www-2.cs.cmu.edu/afs/cs.cmu.edu/user/mleone/web/how-to.html}; last accessed on August 28, 2010. Also, see \url{http://www.cs.cmu.edu/~mleone/how-to.html} for another copy. [Mark Leone has graduated with a MS CS from CMU.]
		\item Jason I. Hong, {\it Grad School Advice}, Human Computer Interaction Institute, School of Computer Science, Carnegie Mellon University, Sept 20, 2006. Available online at: \url{http://www.cs.cmu.edu/~jasonh/advice.html}; last accessed on December 17, 2010.
		\item women@SCS School of Computer Science: \vspace{-0.1cm}
			\begin{enumerate} \itemsep -1pt
			\item Career Advice: \url{http://women.cs.cmu.edu/Resources/JobsResearch/careeradvice.php}
			\end{enumerate}
		\end{enumerate}
	\item University of California, San Diego: \vspace{-0.2cm}
		\begin{enumerate} \itemsep -2pt
		\item UCSD VLSI CAD Laboratory, {\it Useful tips on how to succeed in graduate school and your subsequent research career}, Department of Computer Science and Engineering \& Department of Electrical and Computer Engineering, University of California, San Diego. Available at: \url{http://vlsicad.ucsd.edu/Research/Advice/index.html}; last accessed on August 28, 2010. \colorbox{yellow}{\bf EXCELLENT!!!}
		\item Mihir Bellare, ``The Ph.D Experience,'' Department of Computer Science \& Engineering, University of California at San Diego. Available at: \url{http://cseweb.ucsd.edu/~mihir/phd.html}; last accessed on September 13, 2010. [ See {\it Educational} material about graduate school, research, and technical writing at: \url{http://cseweb.ucsd.edu/users/mihir/education.html}. ]
		\item Fan Chung Graham, ``A few words on research for graduate students (especially for those potential combinatorialists),'' in {\it Teaching}, Department of Mathematics, University of California, San Diego. Available online at: \url{http://math.ucsd.edu/~fan/teach/gradpol.html}; last accessed on January 9, 2010.
		\end{enumerate}
	\item University of Michigan, Ann Arbor: \vspace{-0.2cm}
		\begin{enumerate} \itemsep -2pt
		\item  Igor Markov, Department of Electrical Engineering and Computer Science: \url{http://www.eecs.umich.edu/~imarkov/i_students.html}. In particular, see his ``Advice for graduate students''.
		\end{enumerate}
	\item University of California, Los Angeles: \vspace{-0.2cm}
		\begin{enumerate} \itemsep -2pt
		\item Philip E. Agre, ``Advice for Undergraduates Considering Graduate School,'' UCLA Department of Information Studies, University of California, Los Angeles, October 1996 (Modified: May 2001). Available at: \url{http://polaris.gseis.ucla.edu/pagre/grad-school.html}; last accessed on August 28, 2010. See \url{http://polaris.gseis.ucla.edu/pagre/grad-school.pdf} for a PDF copy of this article. \colorbox{yellow}{\bf CLASSIC!!!}. See \url{http://polaris.gseis.ucla.edu/pagre/index.html} for more articles.
		\item Terence Tao, {\it Career advice}, Department of Mathematics, University of California, Los Angeles. Available at: \url{http://terrytao.wordpress.com/career-advice/}; last accessed on September 1, 2010. Additional information can be found at: \url{http://www.math.ucla.edu/~tao/}.
		\item Yu Hu: \vspace{-0.1cm}
			\begin{enumerate} \itemsep -1pt
			\item Yu Hu, ``Links: Programming Tools and Tips; and Documentation and Presentation,'' Electrical Engineering Department, University of California, Los Angeles, Aug 27, 2007. Available online at: \url{http://www.ee.ucla.edu/~hu/links.htm}; last accessed on September 18, 2010. [ This web page has resources for computer programming, and creating presentation slides and documentations. ]
			\item Courses @ UCLA, April 22, 2006: \url{http://www.ee.ucla.edu/~hu/course.htm}
			\item Research: \url{http://www.ee.ucla.edu/~hu/project.htm}
			\end{enumerate}
		\end{enumerate}
	\item Stanford University: \vspace{-0.2cm}
		\begin{enumerate} \itemsep -2pt
		\item John Ousterhout, ``My Favorite Sayings,'' Department of Computer Science, Stanford University, September 09, 2009. Available at: \url{http://www.stanford.edu/~ouster/cgi-bin/sayings.php}; last accessed on September 4, 2010. [Also, see {\it Odds \& Ends}: \url{http://www.stanford.edu/~ouster/cgi-bin/misc.php} ]
		\item Jeffrey Michael Heer, Department of Computer Science: \vspace{-0.1cm}
			\begin{enumerate} \itemsep -1pt
			\item The only Ph.D. student to have ever won the Microsoft Graduate Fellowship and the IBM Ph.D. Fellowship concurrently. After he won these fellowships as a Ph.D. student at Berkeley, IBM changed the rules for its Ph.D. fellowship so that nobody else can do this anymore. This prevents other companies from competing with IBM for hiring these fellows as research interns.
			\item \url{http://hci.stanford.edu/jheer/cv/}
			\end{enumerate}
		\item Ravi Vakil, ``For potential students,'' Department of Mathematics, Stanford University. Available at: \url{http://math.stanford.edu/~vakil/potentialstudents.html}; last accessed on September 1, 2010. [ ``Great articles and books'' in mathematics: \url{http://math.stanford.edu/~vakil/greatwriting.html}. Information about getting/writing letters of recommendation: \url{http://math.stanford.edu/~vakil/recommendations.html}. ]
		\item Philip Guo, {\it Academic Home Page}, Department of Computer Science, Stanford University. Available at: \url{http://www.stanford.edu/~pgbovine/academic.htm}; last accessed on September 1, 2010. [ See resources at the bottom of the page. Also, see \url{http://www.stanford.edu/~pgbovine/writings.htm} for his non-academic/research articles. ]
		\item Stanford University, {\it Tomorrow's Professor$^{\rm SM}$ Mailing List Links}, Center for Teaching and Learning, Stanford University. Available at: \url{http://www.stanford.edu/dept/CTL/Tomprof/links.html}; last accessed on September 1, 2010.
		\item Eran Magen, {\it How I Got Into the Stanford Psychology Ph.D. Program}, Department of Psychology, Stanford University. Available at: \url{http://www.howigotintostanford.com/}; last accessed on September 1, 2010.
		\item Stanford University, {\it Tutoring and Academic Support}, [Office of the] Vice Provost for Undergraduate Education, Stanford University. Available at: \url{http://ual.stanford.edu./ARS/index.html}; last accessed on September 1, 2010.
		\item Stanford University, ``Guidelines for Advising Relationships between Faculty Advisors and Graduate Students,'' Office of the Vice Provost for Graduate Education (VPGE), Stanford University, 2009. Available online at: \url{http://vpge.stanford.edu/docs/Advisor_Guidelines.pdf}; last accessed on December 22, 2010.
		\end{enumerate}
	\item University of Washington: \vspace{-0.2cm}
		\begin{enumerate} \itemsep -2pt
		\item University of Washington, {\it 10-Year Review Self-Study}, Department of Computer Science \& Engineering, University of Washington, January 2000. Available at: \url{http://www.cs.washington.edu/homes/lazowska/selfstudy/}; last accessed on September 2, 2010. See other information on Prof. Ed Lazowska's web page: \url{http://www.cs.washington.edu/homes/lazowska/}.
		\item Yuriy Brun, {\it Yuriy Brun's Advice}, Department of Computer Science \& Engineering, University of Washington. Available at: \url{http://www.cs.washington.edu/homes/brun/advice/}; last accessed on August 28, 2010. See \url{http://www.cs.washington.edu/homes/brun/advice/PhDAdvice.pdf} for: Yuriy Brun, ``Getting a Ph.D. at the University of Southern California,'' May 20, 2010.
		\item Michael Ernst, {\it Advice for researchers and students}, Department of Computer Science \& Engineering, University of Washington. Available at: \url{http://www.cs.washington.edu/homes/mernst/advice/}; last accessed on August 28, 2010.
		\item Karin Strauss, {\it For graduate students} [ see the links on the left side of her home page ]. Available at: \url{http://www.cs.washington.edu/homes/kstrauss/}; last accessed on September 3, 2010.
		\item Wanda Pratt, {\it Advice}, Information School \& Division of Biomedical \& Health Informatics / Department of Medical Education and Biomedical Informatics / School of Medicine, University of Washington. Available at: \url{http://faculty.washington.edu/wpratt/advice.htm}; last accessed on September 3, 2010.
		\item William A. Stein, {\it Home Page}, Department of Mathematics, University of Washington. Available at: \url{http://wstein.org/}; last accessed on September 5, 2010. {\bf \color{blue} [ Has GREAT resources for junior faculty application and research grant proposals. He has provided tar balls (or zip files) and PDF files of these material. ]}
		\item University of Washington Graduate School, {\it Re-envisioning the Ph.D. project}, University of Washington Graduate School, University of Washington. Available at: \url{http://www.grad.washington.edu/envision/index.html}; last accessed on August 28, 2010. [This research is about issues concerning Ph.D. programs, such as: how to improve the quality of Ph.D. programs and student outcomes, and the lifestyle (including social life) of Ph.D. students; and funding issues.]
		\end{enumerate}
	\item Duke University: \vspace{-0.2cm}
		\begin{enumerate} \itemsep -2pt
		\item Xiaowei Yang, {\it Advice Collection}, Department of Computer Science, Duke University. Available at: \url{http://www.cs.duke.edu/~xwy/advices.html}; last accessed on August 28, 2010.
		\end{enumerate}
	\item Columbia University: \vspace{-0.2cm}
		\begin{enumerate} \itemsep -2pt
		\item Department of Economics: \vspace{-0.1cm}
			\begin{enumerate} \itemsep -1pt
			\item Donald R. Davis, ``Ph.D. Thesis Research: Where do I Start?'', Department of Economics, Columbia University, February 2001. Available online at: \url{http://www.columbia.edu/~drd28/Thesis%20Research.pdf}; last accessed on January 9, 2010.
			\end{enumerate}
		\end{enumerate}
	\item Princeton University: \vspace{-0.2cm}
		\begin{enumerate} \itemsep -2pt
		\item Boaz Barak, Department of Computer Science: \vspace{-0.1cm}
			\begin{enumerate} \itemsep -1pt
			\item He won the ACM Doctoral Dissertation Award.
			\item \url{http://awards.acm.org/doctoral_dissertation/}
			\item \url{http://www.cs.princeton.edu/~boaz/}
			\end{enumerate}
		\end{enumerate}
	\item The University of Texas at Austin: \vspace{-0.2cm}
		\begin{enumerate} \itemsep -2pt
		\item Department of Computer Science: \vspace{-0.1cm}
			\begin{enumerate} \itemsep -1pt
			\item Mike Dahlin, ``Advice to systems researchers,'' Department of Computer Science, The University of Texas at Austin. Available online at: \url{http://www.cs.utexas.edu/users/dahlin/advice.html}; last accessed on January 9, 2010.
			\end{enumerate}
		\end{enumerate}
	\item University of Pennsylvania: \vspace{-0.2cm}
		\begin{enumerate} \itemsep -2pt
		\item Stephanie Weirich, {\it Advice for Graduate Studies}, Department of Computer and Information Science, School of Engineering and Applied Science, University of Pennsylvania. Available at: \url{http://www.seas.upenn.edu/~sweirich/phdadvice.htm}; last accessed on September 5, 2010.
		\end{enumerate}
	\item Northwestern University: \vspace{-0.2cm}
		\begin{enumerate} \itemsep -2pt
		\item Department of Electrical Engineering and Computer Science; Robert R. McCormick School of Engineering and Applied Science: \vspace{-0.1cm}
			\begin{enumerate} \itemsep -1pt
			\item Lance Fortnow and William Gasarch, ``Graduate Student Guide,'' in their blog {\it Computational Complexity}, Department of Electrical Engineering and Computer Science, Robert R. McCormick School of Engineering and Applied Science, Northwestern University, February 21, 2007. Available at: \url{http://blog.computationalcomplexity.org/2007/02/graduate-student-guide.html}; last accessed on September 14, 2010. [ William Gasarch is from the Department of Computer Science at the University of Maryland, College Park. THIS IS EXCELLENT!!! ]
			\end{enumerate}
		\end{enumerate}
	\item Purdue University: \vspace{-0.2cm}
		\begin{enumerate} \itemsep -2pt
		\item Douglas E. Comer, {\it A few essays about Computer Science}, Department of Computer Science, Purdue University: \vspace{-0.1cm}
			\begin{enumerate} \itemsep -1pt
			\item \url{http://www.cs.purdue.edu/homes/dec/}
			\item Look for the section, ``A few essays about Computer Science''. The essay, ``How to generate a CS research topic,'' is funny: \url{http://www.cs.purdue.edu/homes/dec/essay.topic.generator.html}.
			\item Douglas E. Comer, ``Notes On The PhD Degree,'' Department of Computer Science, Purdue University. Available at: \url{http://www.cs.purdue.edu/homes/dec/essay.phd.html}; last accessed on September 12, 2010.
			\end{enumerate}
		\item Jan Vitek, {\it Home Page: Miscellaneous}, Department of Computer Science, Purdue University. Available online at: \url{http://www.cs.purdue.edu/homes/jv/}; last accessed on September 28, 2010.
		\end{enumerate}
	\item Cornell University: \vspace{-0.2cm}
		\begin{enumerate} \itemsep -2pt
		\item CU-ADVANCE Center (research and resource center concerning diversity and gender equity): \url{http://advance.cornell.edu/}
		\item Department of Computer Science, Faculty of Computing and Information Science (CIS): \vspace{-0.1cm}
			\begin{enumerate} \itemsep -1pt
			\item Charles F. Van Loan: \url{http://www.cs.cornell.edu/cv/default.htm}
			\end{enumerate}
		\end{enumerate}
	\item Rice University: \vspace{-0.2cm}
		\begin{enumerate} \itemsep -2pt
		\item Richard G. Baraniuk, ``Seven Steps to Success in Graduate School (and Beyond),'' Department of Electrical and Computer Engineering, George R. Brown School of Engineering, Rice University. Available online at: \url{http://www.ece.rice.edu/~richb/success.html}; last accessed on January 9, 2010.
		\end{enumerate}
	\item Yale University: \vspace{-0.2cm}
		\begin{enumerate} \itemsep -2pt
		\item Stephen C. Stearns, ``Some Modest Advice for Graduate Students,'' Department of Ecology and Evolutionary Biology, Yale University. Available at: \url{http://www.yale.edu/eeb/stearns/advice.htm}; last accessed on August 28, 2010.
		\item Stephen C. Stearns, ``Designs for Learning,'' Department of Ecology and Evolutionary Biology, Yale University. Available at: \url{http://www.yale.edu/eeb/stearns/designs.htm}; last accessed on August 28, 2010.
		\end{enumerate}
	\item Harvard University: \vspace{-0.2cm}
		\begin{enumerate} \itemsep -2pt
		\item H. T. Kung, ``Useful Things to Know About Ph. D. Thesis Research,'' Harvard School of Engineering and Applied Sciences, Harvard University. (Prepared for ``What is Research'' Immigration Course, Computer Science Department, Carnegie Mellon University, 14 October 1987)
		\item Susan Athey, ``Advice for Applying to Grad School in Economics,'' Department of Economics, Harvard University. Available online at: \url{http://kuznets.fas.harvard.edu/~athey/gradadv.html}; last accessed on January 9, 2010. Prof. Athey has also provided an article, ``Negotiating Senior Job Offers,'' on her web page; this would concern senior faculty job offers.
		\item The Collaborative on Academic Careers in Higher Education, Harvard University Graduate School of Education: \url{http://isites.harvard.edu/icb/icb.do?keyword=coache&tabgroupid=icb.tabgroup104863}
		\end{enumerate}
	\item University of Wisconsin-Madison: \vspace{-0.2cm}
		\begin{enumerate} \itemsep -2pt
		\item Dorothea Salo, {\it A Tale of Graduate School Burnout}. Available at: \url{http://members.terracom.net/~dorothea/gradsch/index.html}; last accessed on August 28, 2010.
		\item Dorothea Salo, {\it Straight Talk about Graduate School}. Available at: \url{http://members.terracom.net/~dorothea/gradsch/straighttalk.html}; last accessed on August 28, 2010.
		\item Dorothea Salo, {\it What to do before applying to graduate school}. Available at: \url{http://members.terracom.net/~dorothea/gradsch/success.html}; last accessed on August 28, 2010.
		\end{enumerate}
	\item Pennsylvania State University: \vspace{-0.2cm}
		\begin{enumerate} \itemsep -2pt
		\item Tao Xie and Yuan Xie, {\it Advice Collection}, Department of Computer Science at North Carolina State University, and Department of Computer Science and Engineering at Pennsylvania State University. Available at: \url{http://www.cse.psu.edu/~yuanxie/advice.htm}; last accessed on August 25, 2010. Also, see \url{http://people.engr.ncsu.edu/txie/advice/index.html} and \url{http://people.engr.ncsu.edu/txie/advice.htm}.
		\item Office of Engineering Diversity; Penn State College of Engineering: \vspace{-0.1cm}
			\begin{enumerate} \itemsep -1pt
			\item Office of Engineering Diversity, ``Tips for Graduate Students,'' Penn State College of Engineering, Pennsylvania State University, 2009. Available online at: \url{http://www.engr.psu.edu/mep/tips.html}; last accessed on December 9, 2010.: \vspace{-0.1cm}
				\begin{itemize} \itemsep -1pt
				\item Getting the most out of the relationship with your research advisor or boss: \vspace{-0.1cm}
					\begin{itemize} \itemsep -1pt
					\item {\bf Meet regularly} - you should insist on meeting once a week or at least every other week because it gives you motivation to make regular progress and it keeps your advisor aware of your work.
					\item {\bf Prepare for your meetings} - come to each meeting with: List of topics to discuss; Plan for what you hope to get out of the meeting; Summary of you have done since your last meeting; List of any upcoming deadlines; \& Notes from your previous meeting
					\item {\bf Email him/her a brief summary of EVERY meeting} - this helps avoid misunderstandings and provides a great record of your research progress. Include (where applicable): Time and plan for next meeting; New summary of what you think you are doing; To do list for yourself; To do list for your advisor; List of related work to read; List of major topics discussed; List of what you agreed on; \& List of advice that you may not follow
					\item {\bf Show your advisor the results of your work as soon as possible} - this will help your advisor understand your research and identify potential points of conflict early in the process. Summaries of related work. Anything you write about your research. Experimental results.
					\item {\bf Communicate clearly} - if you disagree with your advisor, state your objections or concerns clearly and calmly. If you feel something about your relationship is not working well, discuss it with him or her. Whenever possible, suggest steps they could take to address your concerns.
					\item {\bf Take the initiative} - you do not need to clear every activity with your advisor. He/she has a lot of work to do too. You must be responsible for your own research ideas and progress.
					\end{itemize}
				\item Getting the most out of what you read: \vspace{-0.1cm}
					\begin{itemize} \itemsep -1pt
					\item {\bf Be organized}. Keep an electronic bibliography with notes \& pointers to the paper files. Keep and file all the papers you have read or skimmed.
					\item {\bf Be efficient} - only read what you need to. Start by reading only the conclusion, scanning figures \& tables, and looking at their references. Read the other sections only if the paper seems relevant or you think it may help you get a different perspective. Skip the sections that you already understand (often the background and motivation sections).
					\item {\bf Take notes on every paper you find worth reading} - What problem are they trying to solve? What is their approach? How is it different from other approaches?
					\item {\bf Summarize what you have read on each topic} -- after you have read several papers covering some topic, note the: key problems; various formulations of the problem they are addressing; relationship among the various approaches; and alternative approaches
					\item {\bf Read Ph.D. theses} -- even though they are long they can be very helpful in quickly learning about what has been done is some field. Especially focus on: Background sections; Method sections; and Your advisor's thesis (this will give you an idea for what he/she expects from you).
					\end{itemize}
				\item Making continual progress on your research: \vspace{-0.1cm}
					\begin{itemize} \itemsep -1pt
					\item {\bf Keep a journal of your ideas} - write down everything you are thinking about even if you think it is stupid. It will help you keep track of your progress and keep you from going in circles. Do not plan to share it with anyone, so you can write freely.
					\item {\bf Set some reasonable goals with deadlines}: Identify key tasks that need to be completed; Set a reasonable date for completing them (on the order of weeks or months); Share this with your advisor or enlist your advisors help in creating the goals and deadlines; and Set some deadlines that you must keep (e.g., volunteer to give a student seminar on your research, work toward a conference paper submission deadline, etc.)
					\item {\bf Keep a to do list} - Checking off things on a to do list can feel very rewarding when you are working on a long-term project. List the small tasks that can be done in about an hour. Pick at least one that has to be completed each day.
					\item {\bf Continually update your: Problem statement}, Goals, Approach (or a list of possible approaches); One-minute version of your research (aka the elevator ride summary); and Five-minute version of your research
					\item {\bf Discuss your research with anyone who will listen} - use your fellow students, friends, family, etc. to practice discussing your research on various levels. They may have useful insights or you may find that verbalizing your ideas clarifies them for yourself.
					\item {\bf Write about your work}. Early stage: Write short idea papers and share them with your advisor and colleagues. Intermediate stage: Find workshops and conferences for submitting preliminary results. This can also help you set deadlines. Advanced stage: Target relevant journals.
					\item {\bf Avoid distractions} - it is easy to ignore your research in favor of more structured tasks such as taking classes, teaching classes, organizing student activities, creating web pages like this, etc. Minimize these kinds of activities or commitments.
					\item {\bf Confront your fears and weaknesses}: If you are afraid of public speaking, volunteer to give lots of talks. If you are afraid your ideas are stupid, discuss them with someone. If you are afraid of writing, write something about your research every day. One-minute version of your research (aka the elevator ride summary). Five-minute version of your research. List of advice that you may not follow.
					\item {\bf Balance reading, thinking, writing and hacking} - often research needs to be an iterative process across all of those tasks.
					\item {\bf Finding a thesis topic or formulating a research plan}. Pick something you find interesting - if you work on something solely because your advisor wants you to, it will be difficult to stay motivated ... Pick something your advisor finds interesting - if your advisor doesn't find it interesting he/she is unlikely to devote much time to your research. He/she will be even more motivated to help you if your project is on their critical path (although this has down sides too! )... Pick something the research community will find interesting - if you want to make yourself marketable... Make sure it addresses a real problem. Remember that your topic will evolve as you work on it. Pick something that is narrow enough that it can be done in a reasonable time frame. Have realistic expectations (i.e. Don't expect the Nobel Prize ). Don't worry that you will be stuck in this area for the rest of your career. It is very likely that you will be doing very different research after you graduate.
					\end{itemize}
				\item Characteristics to look for in a good advisor, mentor, boss, or committee member: \vspace{-0.1cm}
					\begin{itemize} \itemsep -1pt
					\item It is unreasonable to expect one person to have all of the qualities you desire. You should choose thesis committee members who are strong in the areas where your advisor is weak.
					\item {\bf Willing to meet with you regularly (about 1 hour every week or every other week). You can trust him/her to}: Give you credit for the work you do; Defend your work when you are not around; Tell you when your work is or is not good enough; Help you graduate in a reasonable time frame; and Look out for you professionally and personally
					\item {\bf Is interested in your topic. Has good personal and communication skills}: You can talk freely and easily about research ideas; Tells you when you are doing something stupid; Patient, Never feels threatened by your capabilities; Helps motivate you and keep you unstuck
					\end{itemize}
				\item {\bf Has good technical skills}. Can provide constructive criticism of papers you write or talks you give. Knows if what you are doing is good enough for a good thesis. Can help you figure out what you are not doing well; Can help you improve your skills; Can suggest related articles to read or people to talk to; Can tell you or help you discover if what you are doing has already been done; Can help you set and obtain reasonable goals. Will be around until you finish. Is well respected in his/her field. Has good connections for the type of job you would want when you graduate; Willing and able to provide financial and computing support.
				\item {\bf Avoiding the research blues}: \vspace{-0.1cm}
					\begin{itemize} \itemsep -1pt
					\item When you meet your goals, reward yourself.
					\item Don't compare yourself to senior researchers who have many more years of work and publications.
					\item Don't be afraid to leave part of your research problem for future work.
					\item Exercise.
					\item Use the student counseling services
					\item Occasionally, do something fun without feeling guilty! 
					\end{itemize}
				\end{itemize}
			\end{enumerate}
		\end{enumerate}
	\item Technion - Israel Institute of Technology: \vspace{-0.2cm}
		\begin{enumerate} \itemsep -2pt
		\item Department of Computer Science: \vspace{-0.1cm}
			\begin{enumerate} \itemsep -1pt
			\item 
			\end{enumerate}
		\end{enumerate}
	\item University of California, Santa Cruz: \vspace{-0.2cm}
		\begin{enumerate} \itemsep -2pt
		\item Department of Computer Science; Jack Baskin School of Engineering: \vspace{-0.1cm}
			\begin{enumerate} \itemsep -1pt
			\item CMPS/CMPE 200 - Research and Teaching in Computer Science and Engineering (and Applied Math and Statistics and Technology and Information Management) (Fall 2010) by Prof. Cormac Flanagan and Prof. Jose Renau (\& Prof. Alexandre Brandwajn???). Available online at: \url{http://www.soe.ucsc.edu/classes/cmps200/Fall10/}; last accessed on December 17, 2010. [ Also, see {\it CMPS 200 - Research and Teaching in Computer Science and Engineering}: \url{http://www.cs.ucsc.edu/courses/course?cmps200} ]
			\end{enumerate}
		\end{enumerate}
	\item The University of North Carolina at Chapel Hill: \vspace{-0.2cm}
		\begin{enumerate} \itemsep -2pt
		\item Ronald T. Azuma, {``So long, and thanks for the Ph.D.!'' a.k.a. ``Everything I wanted to know about C.S. graduate school at the beginning but didn't learn until later.'' The 4th guide in the Hitchhiker's guide trilogy (and if that doesn't make sense, you obviously have not read Douglas Adams}, v. 1.08, Department of Computer Science, The University of North Carolina at Chapel Hill, January 2003. Available at: \url{http://www.cs.unc.edu/~azuma/hitch4.html}; last accessed on September 3, 2010. [ Also, see {\it Guides to surviving Computer Science graduate school} at: \url{http://www.cs.unc.edu/~azuma/azuma_guides.html}. ]
		\end{enumerate}
	\item Johns Hopkins University: \vspace{-0.2cm}
		\begin{enumerate} \itemsep -2pt
		\item Adam Ruben, Department of Biological Chemistry, School of Medicine, Johns Hopkins University. Adam Ruben, ``Surviving Your Stupid, Stupid Decision to Go to Grad School,'' Broadway Books, New York, NY, 2010.
		\end{enumerate}
	\item University of British Columbia: \vspace{-0.2cm}
		\begin{enumerate} \itemsep -2pt
		\item UBC Faculty of Graduate Studies, {\it The Graduate Game Plan}, UBC Faculty of Graduate Studies, University of British Columbia. Available at: \url{http://www.grad.ubc.ca/current-students/gps-graduate-pathways-success/graduate-game-plan}; last accessed on August 28, 2010.
		\item UBC Faculty of Graduate Studies, {\it Resources for Achieving Success}, UBC Faculty of Graduate Studies, University of British Columbia. Available at: \url{http://www.grad.ubc.ca/current-students/gps-graduate-pathways-success/resources-achieving-success}; last accessed on August 28, 2010.
		\item UBC Faculty of Graduate Studies, {\it Research on the Lived Experience of Graduate Students}, UBC Faculty of Graduate Studies, University of British Columbia. Available at: \url{http://www.grad.ubc.ca/current-students/gps-graduate-pathways-success/research-lived-experience-graduate-students}; last accessed on August 28, 2010.
		\item UBC Faculty of Graduate Studies, {\it Resources for Graduate Student Career Development}, UBC Faculty of Graduate Studies, University of British Columbia. Available at: \url{http://www.grad.ubc.ca/current-students/gps-graduate-pathways-success/resources-graduate-student-career-development}; last accessed on August 28, 2010.
		\item UBC Faculty of Graduate Studies, {\it Graduate Guides}, UBC Faculty of Graduate Studies, University of British Columbia. Available at: \url{http://www.grad.ubc.ca/current-students/gps-graduate-pathways-success/graduate-guides}; last accessed on August 28, 2010.
		\item UBC Faculty of Graduate Studies, {\it Present and Publish Your Research}, UBC Faculty of Graduate Studies, University of British Columbia. Available at: \url{http://www.grad.ubc.ca/current-students/gps-graduate-pathways-success/present-publish-your-research}; last accessed on August 28, 2010.
		\end{enumerate}
	\item Oxford University: \vspace{-0.2cm}
		\begin{enumerate} \itemsep -2pt
		\item Computing Laboratory (Computer Science department): \vspace{-0.1cm}
			\begin{enumerate} \itemsep -1pt
			\item Computing Laboratory, ``D.Phil. in Computer Science,'' Oxford University. Available online at: \url{http://www.comlab.ox.ac.uk/admissions/dphil/transfer.html}; last accessed on October 29, 2010.
			\item Marta Kwiatkowska, ``How to apply for a Doctorate in the Computing Laboratory,'' Computing Laboratory, MT 2009. Available online at: \url{http://www.comlab.ox.ac.uk/admissions/dphil/howtoapply2009.pdf}; last accessed on December 17, 2010. \vspace{-0.1cm}
				\begin{itemize} \itemsep -1pt
				\item A B.S. or B.A. gives you general education; a Masters is your license to practice; and a Ph.D. is a license to teach, do research, and examine people for their Ph.D. defense.
				\item A Ph.D. ``is a research degree,'' and an ``apprentice in research''. It is ``awarded for a significant and substantial piece of research,'' and ``examined by experts and defended in a viva.''
				\item Research is about finding out about something, and discovering how to do that. It involves taking responsibility for organizing my time and taking charge of the investigation.
				\item Doing a Ph.D. is exciting as I ``carry out investigations into [the] unknown,'' and it is ``enriching to learn and master new techniques'' while doing so.
				\end{itemize}
			\end{enumerate}
		\end{enumerate}
	\item University of California, Irvine; Donald Bren School of Information and Computer Sciences: \vspace{-0.2cm}
		\begin{enumerate} \itemsep -2pt
		\item {\it UCI on iTunes U}, ``Improving your Grad School Application,'' University of California, Irvine: Donald Bren School of Information and Computer Sciences: Bren School Honors Seminar on November 12, 2008. Available at: \url{http://deimos3.apple.com/WebObjects/Core.woa/Browse/uci.edu.1983660442.01983660444.1975757832?i=2046363870}; last accessed on August 28, 2010. Also, see \url{http://www.oit.uci.edu/itunesu/} for {\it UCI on iTunes U}, and \url{http://www.ics.uci.edu/about/videos/index.php} for {\it Bren School iTunes U} content. [I can access this from the main {iTunes} site as follows: Look at the ``Find Educational Provider'' tab on the left panel (it's in the middle), and select ``Universities andColleges'' $\Longrightarrow$ Select ``UC Irvine'' $\Longrightarrow$ Under the Courses panel, select ``Donald Bren School of Information and Computer Sciences'' $\Longrightarrow$ Under the ``Community Outreach'' panel, select ``Improving your grad school application'' $\Longrightarrow$ watch this video. The video clip is about a panel discussion of professors about how to get into a top-tier graduate program in CS. It talks about things that admission officers look for, how to get strong letters of recommendation.]
		\end{enumerate}
	\item University of California, Davis: \vspace{-0.2cm}
		\begin{enumerate} \itemsep -2pt
		\item Galois Group, Department of Mathematics: \vspace{-0.1cm}
			\begin{enumerate} \itemsep -1pt
			\item University of California, Davis, {\it Useful things to know when starting graduate school... {\small ...as contributed by experienced grad students!}}, Department of Mathematics, University of California, Davis. Available at: \url{http://galois.math.ucdavis.edu/UsefulGradInfo/HelpfulAdvice/WishIdKnown}; last accessed on September 1, 2010.
			\item University of California, Davis, {\it LaTeX Tutorial}, Department of Mathematics, University of California, Davis. Available at: \url{http://galois.math.ucdavis.edu/UsefulGradInfo/GettingStarted/LaTeXTutorial}; last accessed on September 1, 2010. [ Has \LaTeX\ template for research proposal that is required for the Ph.D. qualifying exam. ]
			\item University of California, Davis, {\it Writing Your Doctoral Thesis}, Department of Mathematics, University of California, Davis. Available at: \url{http://galois.math.ucdavis.edu/UsefulGradInfo/HelpfulAdvice/WritingYourThesis}; last accessed on September 1, 2010. [ Has \LaTeX\ template for Ph.D. dissertations. ]
			\item ``If you are a UC Davis Math Grad Student, then you are a member of the Galois Group.''
			\end{enumerate}
		\end{enumerate}
	\item University of Chicago: \vspace{-0.2cm}
		\begin{enumerate} \itemsep -2pt
		\item Pedro F. Felzenszwalb, Department of Computer Science: \vspace{-0.2cm}
			\begin{enumerate} \itemsep -2pt
			\item \url{http://people.cs.uchicago.edu/~pff/}
			\item His paper, ``Digipaper: A Versatile Color Document Image Representation,'' has been cited 24 times in about 11 years since publication (as of September 1, 2010). He was an undergraduate then, and probably did this work as a junior or early in his senior year.
			\item His paper, ``Efficient Matching of Pictorial Structures,'' is probably based on his work done as a senior. As of September 1, 2010, this paper as been cited 222 times in about 10 years. From \url{http://cs.uchicago.edu/}, it states the following in its news section on September 1, 2010. ``Pedro Felzenszwalb receives Longuet-Higgins prize. The 2010 Longuet-Higgins award has been given to Pedro Felzenszwalb and Daniel Huttenlocher, for their paper "Efficient Matching of Pictorial Structures", Conference on Computer Vision and Pattern Recognition 2000. This award goes to a paper from 10 years ago that has made a fundamental impact on computer vision. Congratulations, Pedro!''
			\end{enumerate}
		\end{enumerate}
	\item University of Virginia: \vspace{-0.2cm}
		\begin{enumerate} \itemsep -2pt
		\item David Evans, {\it Advice}, Department of Computer Science, University of Virginia. Available at: \url{http://www.cs.virginia.edu/~evans/advice/}; last accessed on September 2, 2010. Also, see {\color{blue} ``advice for prospective research students''}: \url{http://www.cs.virginia.edu/~evans/advice/prospective.html}.
		\end{enumerate}
	\item University of Maryland, Baltimore County: \vspace{-0.2cm}
		\begin{enumerate} \itemsep -2pt
		\item Marie desJardins, Department of Computer Science and Electrical Engineering: \vspace{-0.1cm}
			\begin{enumerate} \itemsep -1pt
			\item \url{http://www.cs.umbc.edu/~mariedj/}
			\item Has information on ``How to Succeed in Graduate School,'' ``how to organize a workshop,'' and ``Presenting your research: Papers, talks and chats''.
			\item E.g., Marie desJardins, ``How to Succeed in Graduate School,'' Department of Computer Science and Electrical Engineering, University of Maryland, Baltimore County: \url{http://www.cs.umbc.edu/~mariedj/papers/advice-summary.html}
			\end{enumerate}
		\end{enumerate}
	\item University of Maryland, College Park: \vspace{-0.2cm}
		\begin{enumerate} \itemsep -2pt
		\item Dianne Prost O'Leary, {\it Graduate Study in the Computer and Mathematical Sciences: A Survival Guide}, Department of Computer Science, University of Maryland, College Park. Available at: \url{http://www.cs.umd.edu/~oleary/gradstudy/gradstudy.html}; last accessed on August 28, 2010. It is also available at: \url{http://www.cs.umd.edu/~oleary/gradstudy/}. See \url{http://www.cs.umd.edu/~oleary/} for more articles about ``the accessibility of computer science,'' ``8 rules for career success,'' and the disparity in gender ratios in STEM fields.
		\end{enumerate}
	\item University of Utah: \vspace{-0.2cm}
		\begin{enumerate} \itemsep -2pt
		\item Matt Might, {\it The illustrated guide to a Ph.D.}, School of Computing, University of Utah. Available at: \url{http://matt.might.net/articles/phd-school-in-pictures/}; last accessed on September 13, 2010. [
See {\it blog.might.net} for more articles about graduate school: \url{http://matt.might.net/articles/}; e.g., read ``10 easy ways to fail a Ph.D.'' at: \url{http://matt.might.net/articles/ways-to-fail-a-phd/}. His web page, \url{http://matt.might.net/}, has a sample of these articles. THIS IS EXCELLENT!!! ]
		\item Department of Electrical and Computer Engineering: \vspace{-0.1cm}
			\begin{enumerate} \itemsep -1pt
			\item Prof. Cynthia Furse: \vspace{-0.1cm}
				\begin{itemize} \itemsep -1pt
				\item \url{http://www.ece.utah.edu/~cfurse/}
				\item Cynthia Furse, {\it Graduate Student Survival 101}, February 2009. Available online at: \url{http://www.ece.utah.edu/~cfurse/Tutorials/UU%20Thesis/How%20to%20Write%20a%20Thesis.htm}; last accessed on December 10, 2010.
%	http://www.ece.utah.edu/~cfurse/Tutorials/UU%20Thesis/How%20to%20Write%20a%20Thesis.htm
				\item Cynthia Furse, {\it Dr. Furse's OnLine Tutorials}, August 2007. Available online at: \url{http://www.ece.utah.edu/~cfurse/Tutorials/tutorialsUofU.htm}; last accessed on December 10, 2010. [ Has a lot of good resources for teaching/lecturing, academic/technical writing (including thesis writing), advice for grad/Ph.D. students, making presentations and giving talks, writing grants and proposals, entrepreneurship, and resources for job hunting. ]
				\end{itemize}
			\end{enumerate}
		\end{enumerate}
	\item Indiana University: \vspace{-0.2cm}
		\begin{enumerate} \itemsep -2pt
		\item Indiana University, {\it What Every New Grad Student Should Know}, School of Informatics and Computing, Indiana University. Available at: \url{http://www.cs.indiana.edu/docproject/grad.stuff.html}; last accessed on September 1, 2010.
		\item David Chapman (Editor), {\it How to do Research At the MIT AI Lab}, AI Working Paper 316, MIT AI Lab, Massachusetts Institute of Technology, October, 1988. Available at: \url{http://www.cs.indiana.edu/docproject/grad.stuff.html}; last accessed on September 1, 2010.
		\item Marie desJardins, {\it How to Be a Good Graduate Student}, SRI International (formerly Stanford Research Institute), March 1994. Available at: \url{http://www.cs.indiana.edu/docproject/grad.stuff.html}; last accessed on September 1, 2010.
		\end{enumerate}
	\item The University of Arizona: \vspace{-0.2cm}
		\begin{enumerate} \itemsep -2pt
		\item Jonathan Sprinkle, {\it Students: So, you want to be my student}, Department of Electrical and Computer Engineering, The University of Arizona. Available at: \url{http://www2.engr.arizona.edu/~sprinkjm/Main/Students}; last accessed on September 5, 2010. ``Choose 2-3 IEEE or AIAA journal or conference publications from my website that interest you. Do not choose technical reports, or student papers. 
Write a critical review of the papers, including why the work is interesting, but most importantly where you think the work should go next. In this review, you are proving to me that you understand the purpose of research, and most importantly that you understand the technical details of the paper and how they relate to research.''
		\end{enumerate}
	\item Dartmouth College: \vspace{-0.2cm}
		\begin{enumerate} \itemsep -2pt
		\item Mark L. Tomforde, ``I've passed my quals, now what? - A guide for Ph.D. candidates in mathematics at Dartmouth College,'' Department of Mathematics, Dartmouth College, August 5, 2002. Available online at: \url{http://www.math.dartmouth.edu/graduate-students/current/guide/GradGuide.pdf}; last accessed on December 22, 2010. [ Also, available at: \url{http://www.math.dartmouth.edu/graduate-students/current/guide/} ]
		\end{enumerate}
	\item Vienna University of Technology (TU Vienna): \vspace{-0.2cm}
		\begin{enumerate} \itemsep -2pt
		\item Silvia Miksch, {\it Tips: How to Do Research}, Faculty of Informatics, Vienna University of Technology. Available at: \url{http://www.ifs.tuwien.ac.at/~silvia/research-tips/}; last accessed on September 1, 2010. It has plenty of resources about: \vspace{-0.1cm}
			\begin{enumerate} \itemsep -1pt
			\item ``How to Do Research''
			\item ``How to Write a Scientific Paper''
			\item ``How to Design a Poster''
			\item ``Tips on Organizing Conferences, Workshops, and Symposia''
			\item ``How to Review''
			\item ``Digitial Libaries''
			\item ``Tips for Writing Correct English''
			\end{enumerate}
		\end{enumerate}
	\item Tufts University: \vspace{-0.2cm}
		\begin{enumerate} \itemsep -2pt
		\item Norman Ramsey, {\it Resources for Students}, Department of Computer Science, Tufts University. Available at: \url{http://www.cs.tufts.edu/~nr/students/}; last accessed on September 2, 2010.
		\item Norman Ramsey, {\it How to get admitted to a PhD program}, Department of Computer Science, Tufts University. Available at: \url{http://www.cs.tufts.edu/~nr/students/admit.html}; last accessed on September 2, 2010.
		\end{enumerate}
	\item Portland State University: \vspace{-0.2cm}
		\begin{enumerate} \itemsep -2pt
		\item Department of Computer Science; Maseeh College of Engineering and Computer Science: \vspace{-0.1cm}
			\begin{enumerate} \itemsep -1pt
			\item PSU CS 569 (MS Students) and CS 669 (PhD students) - Scholarship Skills (Fall 2010) by Prof. Andrew Black and Tim Sheard. Available online at: \url{http://web.cecs.pdx.edu/~black/ScholarshipSkills/}; last accessed on September 29, 2010.
			\end{enumerate}
		\end{enumerate}
	\item State University of New York at Buffalo: \vspace{-0.2cm}
		\begin{enumerate} \itemsep -2pt
		\item William J. Rapaport, {\it Information for Grad Students in Computer Science \& Engineering at UB}, Department of Computer Science and Engineering, Department of Philosophy, and Center for Cognitive Science, State University of New York at Buffalo, Buffalo, NY. Available at: \url{http://www.cse.buffalo.edu/~rapaport/GRAD/}; last accessed on September 2, 2010.
		\end{enumerate}
	\item The Ohio State University: \vspace{-0.2cm}
		\begin{enumerate} \itemsep -2pt
		\item The Ohio Science and Engineering Alliance: \vspace{-0.1cm}
			\begin{enumerate} \itemsep -1pt
			\item Graduate School: \url{http://www.ohiosea.org/academic/academic_grad.html}
			\end{enumerate}
		\end{enumerate}
	\item Swarthmore College: \vspace{-0.2cm}
		\begin{enumerate} \itemsep -2pt
		\item Department of History: \vspace{-0.1cm}
			\begin{enumerate} \itemsep -1pt
			\item Timothy Burke, ``Should You Go to Graduate School?,'' in his blog {\it Easily Distracted: Culture, Politics, Academia and Other Shiny Objects}, Department of History, Swarthmore College. Available at: \url{http://weblogs.swarthmore.edu/burke/permanent-features-advice-on-academia/features/}; last accessed on September 14, 2010. [ Also, see \url{http://www.swarthmore.edu/SocSci/tburke1/gradschool.html}. ]
			\item Timothy Burke, ``From ABD to the Job Market: Advice for the Grad School Endgame,'' in his blog {\it Easily Distracted: Culture, Politics, Academia and Other Shiny Objects}, Department of History, Swarthmore College. Available at: \url{http://weblogs.swarthmore.edu/burke/permanent-features-advice-on-academia/abd/}; last accessed on September 14, 2010.
			\end{enumerate}
		\end{enumerate}
	\item Georgetown University: \vspace{-0.2cm}
		\begin{enumerate} \itemsep -2pt
		\item Department of Economics: \vspace{-0.1cm}
			\begin{enumerate} \itemsep -1pt
			\item Garance Genicot, ``Applying to Grad School in Economics,'' Department of Economics, Georgetown University. Available online at: \url{http://www9.georgetown.edu/faculty/gg58/GradSchool.html}; last accessed on January 9, 2010.
			\end{enumerate}
		\end{enumerate}
	\item Norwegian University of Science and Technology: \vspace{-0.2cm}
		\begin{enumerate} \itemsep -2pt
		\item ``PhD Studies,'' Department of Computer and Information Science; Faculty of Information Technology, Mathematics and Electrical Engineering. Available online at: \url{http://www.idi.ntnu.no/research/phd.php}; last accessed on September 29, 2010. [ Includes an outline of a Ph.D. research proposal that is required for applications to its Ph.D. program in computer science. ]
		\end{enumerate}
	\item North Carolina State University: \vspace{-0.2cm}
		\begin{enumerate} \itemsep -2pt
		\item Richard M. Felder, ``An Engineering Student Survival Guide,'' Department of Chemical and Biomolecular Engineering, North Carolina State University, 1993. Available at: \url{http://www4.ncsu.edu/unity/lockers/users/f/felder/public/Papers/survivalguide.htm}; last accessed on August 27, 2010.
		\item Matthias F. (Matt) Stallmann, ``What CSC Graduates Should Know,'' Department of Computer Science, North Carolina State University, February 9, 1996. Available online at: \url{http://people.engr.ncsu.edu/mfms/Teaching/what-grads-should-know.html}; last accessed on October 6, 2010.
		\end{enumerate}
	\item Michigan State University: \vspace{-0.2cm}
		\begin{enumerate} \itemsep -2pt
		\item The Graduate School: \vspace{-0.1cm}
			\begin{enumerate} \itemsep -1pt
			\item The Graduate School, {\it ``Setting Expectations and Resolving Conflict'' Program: Developing Communication and Conflict Management Skills to Save Time and Enhance Productivity}, The Graduate School, Michigan State University, July 12, 2010. Available online at: \url{http://grad.msu.edu/conflictresolution/}; last accessed on December 22, 2010.
			\end{enumerate}
		\item Collegiate Employment Research Institute: \vspace{-0.1cm}
			\begin{enumerate} \itemsep -1pt
			\item \url{http://www.ceri.msu.edu/}
			\item Recruiting Trends 2010-2011: \url{http://www.ceri.msu.edu/recruiting-trends-2009-2010/}
			\end{enumerate}
		\end{enumerate}
	\item San Francisco State University: \vspace{-0.2cm}
		\begin{enumerate} \itemsep -2pt
		\item Eric Hsu, Department of Mathematics: \vspace{-0.1cm}
			\begin{enumerate} \itemsep -1pt
			\item Eric Hsu, {\it Math Education Job Search Resources}, Department of Mathematics, San Francisco State University. Available at: \url{http://bfc.sfsu.edu/cgi-bin/hsu.pl?Math_Education_Job_Search_Resources}; last accessed on September 1, 2010. Also, accessible at: \url{http://math.sfsu.edu/hsu/jobs.html}.
			\item Has lots of information on applying for positions in academia.
			\item Has lists of postdoc positions and (junior) faculty openings.
			\end{enumerate}
		\end{enumerate}
	\item Colorado School of Mines: \vspace{-0.2cm}
		\begin{enumerate} \itemsep -2pt
		\item Department of Physics: \vspace{-0.1cm}
			\begin{enumerate} \itemsep -1pt
			\item Robert L. Read, ``How to be a Programmer,'' APPLICATI, 2003. Available online at: \url{http://samizdat.mines.edu/howto/}; last accessed on September 28, 2010. [ Another publisher is FEINHOCHBURG ]
			\end{enumerate}
		\end{enumerate}
	\item New Mexico Institute of Mining and Technology (New Mexico Tech): \vspace{-0.2cm}
		\begin{enumerate} \itemsep -2pt
		\item Brian Borchers, ``Recommendation Letters,'' Department of Mathematics, New Mexico Institute of Mining and Technology. Available at: \url{http://infohost.nmt.edu/~borchers/recletters.html}; last accessed on September 2, 2010.
		\end{enumerate}
	\item University of Minnesota Duluth: \vspace{-0.2cm}
		\begin{enumerate} \itemsep -2pt
		\item University of Minnesota Duluth, {\it Is Graduate School Right For You?}, Career Services, University of Minnesota Duluth. Available at: \url{http://www.d.umn.edu/careers/grad_school/right_for_you.html}; last accessed on September 2, 2010.
		\end{enumerate}
	\item The University of Waikato: \vspace{-0.2cm}
		\begin{enumerate} \itemsep -2pt
		\item Sean Oughton, {\it Graduate School Survival Guide}, Department of Mathematics, The University of Waikato. Available at: \url{http://www.math.waikato.ac.nz/~seano/grad-school-advice.html}; last accessed on September 3, 2010. [ Also, see \url{http://www.math.waikato.ac.nz/~seano/} for Sean's web page. ]
		\end{enumerate}
	\item Institute for Operations Research and the Management Sciences (INFORMS): \vspace{-0.2cm}
		\begin{enumerate} \itemsep -2pt
		\item {\it Career Center} (has some information on funding/fellowships, and academic careers): \url{http://www.informs.org/Build-Your-Career/INFORMS-Student-Union/Career-Center}
		\end{enumerate}
	\item The PhD Project: \vspace{-0.2cm}
		\begin{enumerate} \itemsep -2pt
		\item Resources for Potential/Current Doctoral Students: \vspace{-0.1cm}
			\begin{enumerate} \itemsep -1pt
			\item \url{http://www.phdproject.org/resources.html}
			\item Information about good business schools that offer Ph.D. programs, preparation for the GMAT, and the life in graduate school as a Ph.D. student.
			\item Suggested Reading: \vspace{-0.1cm}
				\begin{itemize} \itemsep -1pt
				\item \url{http://www.phdproject.org/reading.html}
				\item Has information life in graduate school as a Ph.D. student, racial diversity/issues in higher education, job searching in academia, and work-life balance for female Ph.D. students.
				\end{itemize}
			\end{enumerate}
		\end{enumerate}
	\item U.S. Department of Education: \vspace{-0.2cm}
		\begin{enumerate} \itemsep -2pt
		\item National Center for Education Statistics (NCES); Institute of Education Sciences: \vspace{-0.1cm}
			\begin{enumerate} \itemsep -1pt
			\item National Center for Education Statistics, {\it Projections of Education Statistics to 2018} [report], National Center for Education Statistics, Institute of Education Sciences, U.S. Department of Education, September 2009. Available online at: \url{http://nces.ed.gov/programs/projections/projections2018/index.asp}; last accessed on January 7, 2010.
			\end{enumerate}
		\end{enumerate}
	\item ABD Solution Company: \vspace{-0.2cm}
		\begin{enumerate} \itemsep -2pt
		\item Dr. Carter�s Educational Group, L.L.C.: \vspace{-0.1cm}
			\begin{enumerate} \itemsep -1pt
			\item Wendy Y. Carter, {\it TA-DA!}\texttrademark\ {\it Thesis and Dissertation Accomplished}, 2011. Available online at: \url{http://www.tadafinallyfinished.com/index.html}; last accessed on January 8, 2010. \vspace{-0.1cm}
				\begin{itemize} \itemsep -1pt
				\item {\it TA-DA!}\texttrademark\ Links and Resources: \url{http://www.tadafinallyfinished.com/links/index.html}
				\end{itemize}
			\item Educational Research Institute: \vspace{-0.1cm}
				\begin{itemize} \itemsep -1pt
				\item \url{http://www.educationalresearchinstitute.org/}
				\end{itemize}
			\end{enumerate}
		\end{enumerate}
	\item About.com: \vspace{-0.2cm}
		\begin{enumerate} \itemsep -2pt
		\item About.com, {\it Graduate School}. Available at: \url{http://gradschool.about.com/}; last accessed on August 25, 2010.
		\item Timothy Dzurilla, {\it Writing Graduate Application Essay: Tips to Writing a Successful Personal Statement}, About.com, Nov 1, 2007. Available at: \url{http://www.suite101.com/content/writing-graduate-application-essay-a34598}; last accessed on September 1, 2010. [ Help with writing a statement of purpose ]
		\item Naomi Rockler-Gladen, {\it How to Choose a Graduate School: Faculty, Fit, Student Culture, and Other Grad Program Considerations}, About.com, Nov 12, 2007. Available at: \url{http://www.suite101.com/content/how-to-choose-a-graduate-school-a35387}; last accessed on September 1, 2010. [ How to select a graduate program ... EXCELLENT ]
		\item Naomi Rockler-Gladen, {\it How to Choose a Graduate Advisor: Finding a Faculty Member to Direct an MA Thesis or PhD Dissertation}, About.com, Oct 30, 2007. Available at: \url{http://www.suite101.com/content/how-to-choose-a-graduate-advisor-a34468}; last accessed on September 1, 2010. [ How to pick an advisor ... EXCELLENT ]
		\end{enumerate}
	\item UndergradEcon.com, {\it Graduate School Economics}. Available online at: \url{http://www.undergradecon.com/grad_school.html}; last accessed on January 9, 2010.
	\item Sumit Gupta, {\it Articles and Information about Graduate School}. Available at: \url{http://www.4bearsonline.com/collections/grad/index.shtml}; last accessed on August 25, 2010. 
	\item William Stallings: \vspace{-0.2cm}
		\begin{enumerate} \itemsep -2pt
		\item {\it Computer Science Student Resource Site: How-To}: \url{http://www.computersciencestudent.com/SS/SS-howto.html}
		\item {\it Computer Science Student Resource Site: Computer Science Careers}: \url{http://www.computersciencestudent.com/SS/SS-career.html}
		\item {\it Computer Science Student Resource Site}: \url{http://www.computersciencestudent.com/}
		\end{enumerate}
	\item Dario Toncich, {\it Key Factors in Postgraduate Research - A Guide for Students}. Available at: \url{http://www.doctortee.net/KeyFactors.html}; last accessed on September 1, 2010.
	\end{enumerate}
%%%%%%%%%%%%%%%%%%%%%%%%%%%%%%%%%%%%%
%%%%%%%%%%%%%%%%%%%%%%%%%%%%%%%%%%%%%
%%%%%%%%%%%%%%%%%%%%%%%%%%%%%%%%%%%%%
\item Grad school admission advice: \vspace{-0.3cm}
	\begin{enumerate} \itemsep -2pt
	\item University of California, Berkeley: \vspace{-0.2cm}
		\begin{enumerate} \itemsep -2pt
		\item Department of Economics, ``Criteria,'' Department of Economics, University of California, Berkeley. Available at: \url{http://www.econ.berkeley.edu/econ/grad/admit-criteria.shtml}; last accessed on August 28, 2010.
		\end{enumerate}
	\item Harvard University: \vspace{-0.2cm}
		\begin{enumerate} \itemsep -2pt
		\item Susan Athey, ``Advice for Applying to Grad School in Economics,'' Department of Economics, Harvard University. Available at: \url{http://kuznets.fas.harvard.edu/~athey/gradadv.html}; last accessed on August 28, 2010.
		\end{enumerate}
	\item Statementofpurpose.com: \url{http://www.statementofpurpose.com/}
	\end{enumerate}
%%%%%%%%%%%%%%%%%%%%%%%%%%%%%%%%%%%%%
%%%%%%%%%%%%%%%%%%%%%%%%%%%%%%%%%%%%%
%%%%%%%%%%%%%%%%%%%%%%%%%%%%%%%%%%%%%
\item Advice concerning research: \vspace{-0.3cm}
	\begin{enumerate} \itemsep -2pt
	\item University of California, Riverside: \vspace{-0.2cm}
		\begin{enumerate} \itemsep -2pt
		\item John Baez, ``Advice for the Young Scientist,'' Department of Mathematics, University of California, Riverside, March 25, 2007. Available at: \url{http://math.ucr.edu/home/baez/advice.html}; last accessed on August 28, 2010.
		\end{enumerate}
	\item Research Information Network, RIN: \vspace{-0.2cm}
		\begin{enumerate} \itemsep -2pt
		\item Researchers resources: \url{http://www.rin.ac.uk/resources/researcher-resources}
		\item Researcher development and skills: \url{http://www.rin.ac.uk/resources/researcher-development-and-skills}
		\item Publishing: \url{http://www.rin.ac.uk/resources/publishing}
		\item Learned and professional society: \url{http://www.rin.ac.uk/resources/learned-and-professional-society}
		\end{enumerate}
	\end{enumerate}
%%%%%%%%%%%%%%%%%%%%%%%%%%%%%%%%%%%%%
%%%%%%%%%%%%%%%%%%%%%%%%%%%%%%%%%%%%%
%%%%%%%%%%%%%%%%%%%%%%%%%%%%%%%%%%%%%
\item advice about giving presentations: \vspace{-0.3cm}
	\begin{enumerate} \itemsep -2pt
	\item Cornell University, Department of Computer Science, Faculty of Computing and Information Science (CIS): \vspace{-0.2cm}
		\begin{enumerate} \itemsep -2pt
		\item Charles F. Van Loan, ``The Short Talk,'' Department of Computer Science, Cornell University. Available at: \url{http://www.cs.cornell.edu/cv/ShortTalk.htm}; last accessed on August 25, 2010.
		\end{enumerate}
	\item University of Wisconsin-Madison, Computer Sciences Department: \vspace{-0.2cm}
		\begin{enumerate} \itemsep -2pt
		\item Mark D. Hill, ``Oral Presentation Advice,'' Computer Sciences Department, University of Wisconsin-Madison, April 1992, Revised January 1997. Available at: \url{http://pages.cs.wisc.edu/~markhill/conference-talk.html}; last accessed on August 25, 2010. It includes a short summary of a presentation on this topic by Prof. David A. Patterson. David A. Patterson, ``How to Give a Bad Talk,'' Computer Science Division, Department of Electrical Engineering and Computer Sciences, University of California-Berkeley, 1983.
		\end{enumerate}
	\item University of California, Los Angeles: \vspace{-0.2cm}
		\begin{enumerate} \itemsep -2pt
		\item Terence Tao, ``Talks are not the same as papers,'' Department of Mathematics, University of California, Los Angeles. Available at: \url{http://terrytao.wordpress.com/career-advice/talks-are-not-the-same-as-papers/}; last accessed on September 1, 2010.
		\end{enumerate}
	\item North Carolina State University, Department of Chemical and Biomolecular Engineering: \vspace{-0.2cm}
		\begin{enumerate} \itemsep -2pt
		\item Richard M. Felder, ``Tips on Talks,'' Department of Chemical and Biomolecular Engineering, North Carolina State University. Available at: \url{http://www4.ncsu.edu/unity/lockers/users/f/felder/public/Papers/speakingtips.htm}; last accessed on August 28, 2010.
		\end{enumerate}
	\item Random information: \vspace{-0.2cm}
		\begin{enumerate} \itemsep -2pt
		\item The number of presentation slides is approximately the same as the number of minutes allocated for the presentation. Therefore, for a 15 minutes presentation, the speaker shall use about 15 slides for her/his presentation.
		\item 
		\end{enumerate}
	\end{enumerate}
%%%%%%%%%%%%%%%%%%%%%%%%%%%%%%%%%%%%%
%%%%%%%%%%%%%%%%%%%%%%%%%%%%%%%%%%%%%
%%%%%%%%%%%%%%%%%%%%%%%%%%%%%%%%%%%%%
\item Advice on studying: \vspace{-0.3cm}
	\begin{enumerate} \itemsep -2pt
	\item State University of New York at Buffalo, Department of Computer Science and Engineering: \vspace{-0.2cm}
		\begin{enumerate} \itemsep -2pt
		\item William J. Rapaport, ``How to Study: A Brief Guide,'' Department of Computer Science and Engineering, Department of Philosophy, and Center for Cognitive Science, State University of New York at Buffalo, Buffalo, NY. Available at: \url{http://www.cse.buffalo.edu/~rapaport/howtostudy.html}; last accessed on August 25, 2010.
		\end{enumerate}
	\item University of Oregon, Teaching and Learning Center: \vspace{-0.2cm}
		\begin{enumerate} \itemsep -2pt
		\item Ronald C. Blue, ``How to Study,'' Teaching and Learning Center, University of Oregon. Available at: \url{http://tep.uoregon.edu/resources/faqs/outsidehelp/study.html}; last accessed on August 25, 2010.
		\end{enumerate}
	\item North Carolina State University, Department of Chemical and Biomolecular Engineering: \vspace{-0.2cm}
		\begin{enumerate} \itemsep -2pt
		\item Richard M. Felder, ``Handouts for Students,'' Department of Chemical and Biomolecular Engineering, North Carolina State University. Available at: \url{http://www4.ncsu.edu/unity/lockers/users/f/felder/public/Student_handouts.html}; last accessed on August 28, 2010.
		\end{enumerate}
	\item Middle Tennessee State University: \vspace{-0.2cm}
		\begin{enumerate} \itemsep -2pt
		\item Carolyn Hopper, ``The Study Skills Help Page: Learning Strategies for Success,'' Middle Tennessee State University. Available at: \url{http://frank.mtsu.edu/~studskl/}; last accessed on August 25, 2010.
		\end{enumerate}
	\item Joseph Frank Landsberger: \vspace{-0.2cm}
		\begin{enumerate} \itemsep -2pt
		\item Joseph Frank Landsberger, {\it Study Guides and Strategies}. Available at: \url{http://www.studygs.net/}; last accessed on August 25, 2010.
		\end{enumerate}
	\end{enumerate}
%%%%%%%%%%%%%%%%%%%%%%%%%%%%%%%%%%%%%
%%%%%%%%%%%%%%%%%%%%%%%%%%%%%%%%%%%%%
%%%%%%%%%%%%%%%%%%%%%%%%%%%%%%%%%%%%%
\item Advice on test preparation and test taking: \vspace{-0.3cm}
	\begin{enumerate} \itemsep -2pt
	\item North Carolina State University: \vspace{-0.2cm}
		\begin{enumerate} \itemsep -2pt
		\item Richard M. Felder, ``Random Thoughts:  Memo,'' {\it Chemical Engineering Education}, Vol. 33, No. 2, pp. 136--137, 1999. Available at: \url{http://www4.ncsu.edu/unity/lockers/users/f/felder/public/Columns/memo.html}; last accessed on August 28, 2010.
		\item Richard M. Felder and James E. Stice, ``Tips on Test Taking,'' Department of Chemical and Biomolecular Engineering, North Carolina State University, and Deptartment of Chemical Engineering, The University of Texas at Austin. Available at: \url{http://www4.ncsu.edu/unity/lockers/users/f/felder/public/Papers/testtaking.htm}; last accessed on August 28, 2010.
		\item Richard M. Felder and Matthias F. (Matt) Stallmann, ``Tips for Test Takers,'' Department of Computer Science, North Carolina State University, February 17, 2005. Available online at: \url{http://people.engr.ncsu.edu/mfms/Teaching/tips-for-test-takers.html}; last accessed on October 6, 2010.
		\end{enumerate}
	\end{enumerate}
%%%%%%%%%%%%%%%%%%%%%%%%%%%%%%%%%%%%%
%%%%%%%%%%%%%%%%%%%%%%%%%%%%%%%%%%%%%
%%%%%%%%%%%%%%%%%%%%%%%%%%%%%%%%%%%%%
\item Advice for engineering students: \vspace{-0.3cm}
	\begin{enumerate} \itemsep -2pt
	\item University of Maryland, Baltimore County; Department of Computer Science and Electrical Engineering: \vspace{-0.2cm}
		\begin{enumerate} \itemsep -2pt
		\item Alan T. Sherman (Alan Theodore Sherman), {\it How To's and Other Generic Course Documents}, Department of Computer Science and Electrical Engineering, University of Maryland, Baltimore County, September 12, 1995. Available at: \url{http://www.csee.umbc.edu/~sherman/Courses/documents/}; last accessed on August 28, 2010. Also, see \url{http://www.csee.umbc.edu/~sherman/Courses/}.
		\item Alan T. Sherman (Alan Theodore Sherman), {\it Teaching Activites}, Department of Computer Science and Electrical Engineering, University of Maryland, Baltimore County. Available at: \url{http://www.csee.umbc.edu/~sherman/mycourses.html}; last accessed on August 28, 2010.
		\end{enumerate}
	\item North Carolina State University, Department of Chemical and Biomolecular Engineering: \vspace{-0.2cm}
		\begin{enumerate} \itemsep -2pt
		\item Richard M. Felder's column, ``Random Thoughts,'' in the journal, {\it Chemical Engineering Education}. Available at: \url{http://www4.ncsu.edu/unity/lockers/users/f/felder/public/Columns.html}; last accessed on August 28, 2010.
		\item Richard M. Felder, ``An Engineering Student Survival Guide,'' Department of Chemical and Biomolecular Engineering, North Carolina State University, 1993. Available at: \url{http://www4.ncsu.edu/unity/lockers/users/f/felder/public/Papers/survivalguide.htm}; last accessed on August 27, 2010. \vspace{-0.2cm}
			\begin{itemize} \itemsep -2pt
			\item Do not expect people to tell me how to solve certain problems, especially implementation details as a researcher.
			\item Learn to find out for myself what I need to know. That is, determine the scope of things that I need to know, the time frame and deadline(s) in which I should acquire knowledge of those skills and knowledge, and create a plan to acquire those skills and knowledge.
			\item I should learn to be more resourceful, and determine where can I get help. Particularly, resources (e.g., publications and online material), individuals, and networks of people that can provide a significant amount of help to people.
			\item Make a serious effort to solve a problem before I approach others for help. Else, they may get annoyed when I did not bother to learn how to solve problems that are actually very simple. Also, bring my (attempted/considered) solutions to the people who I seek help from. Show them my flow charts, schematics, calculations, algorithms, and heuristics. This would convince them that I have done my homework, and am not asking bane questions.
			\item To help me understand how to apply the skills and knowledge that I am acquiring in ``practical, real-world applications,'' I should look at textbooks (including alternative textbooks/books, such as handbooks and encyclopedias) for such examples. I can also look ahead further in the chapter, book, or books of subsequent classes to see how these skills and knowledge will be applied. Note that information that I skip while reading my textbook, manuals, and guides may actually contain the solution that I am looking for. Think of the questions that I asked Anders Franzen about the SMT-LIB manual during my internship at FBK while working on the {\it MathSAT} project. I did not understand the material in the manual, since I lacked a background in compiler design and formal grammar/languages. So, I had to get he to help me interpret what I was reading. This was like when I was learning about UNIX as a freshman/sophomore. I did not understand what I was reading when I looked at the UNIX manual (``man pages''). Thus, I shall learn how to read technical literature better.
			\item By reading technical and semi-technical magazines and newspapers/newsletters, I can learn about ``practical, real-world applications'' of the things that I am learning about. In addition, by talking to others (students further ahead of me in engineering education and professional engineers), I learn to see how can I apply the things that I am learning about in ``practical, real-world applications''. Furthermore, I can tap into the newsletters and technical magazines of professional organizations, such as ACM and IEEE, to find out about research opportunities/projects where I can apply what I am learning about.
			\item If my lecture slides/notes and textbooks(s) do not have adequate worked-out examples (e.g., only trivial examples) to help me understand mathematical theories and formulas, and engineering concepts, I shall seek other resources. E.g., I can look at lecture slides/notes from equivalent/similar classes that are taught at other universities. In addition, I can look at other textbooks/books on this subject. Note that for advanced topics, such as those covered by advanced graduate classes, I may only be able to find 1 or 2 books on this topic. So, I may not always have the luxury of looking at worked examples from other textbooks; e.g., I could find many textbooks for differential equations and vector analysis, but not for antenna analysis or satisfiability modulo theories.
			\item I shall improve my ability to work out problems on my own if I cannot find adequate examples for that problem or similar problems. In addition, I shall document worked solutions digitally, so that I can refer to them during revision for an exam, my prelims/quals, or when I encounter a similar problem during research and development.
			\item I shall also improve the way I revise/relearn concepts, technologies, skills, and knowledge. Documenting resources and prior solutions to problems would help me relearn things. Such documentation requires proper information management, so that I can reuse the previously acquired knowledge and skills. Remember that using \LaTeX\ on a UNIX-like operating system helps me with information management.
			\item If I do not understand how and why things work, determine if knowledge of that is required for solving problems in my research/class project, or assignment. If not, I can move on and address this when I have more free time (e.g., ``slack'' periods during the calendar year). To find out more about how certain things work, I can look into the references in my lecture slides/notes and textbooks(s), or search/google for references online. Note that learning how and why certain things work may require (advanced) knowledge that is outside the scope of my discipline or research area. Hence, it is important to know when to stop delving into (/investigating/probing) a concept/technique. Remember my problems with understanding Prof. Sanjit A. Seshia�s publications on adaptive eager encodings, in which he used ``polyhedral theory'' (I believe in the context of integer programming and combinatorial optimization) to prove a theorem regarding the satisfiability of UTVPI formulas? Ditto for lambda expressions (and lambda calculus) so that I can understand how syntax is represented for
	a given signature in first-order logic.]
			\item I shall improve my ability to convert descriptions of architectures and techniques into hardware/software implementations. It is easier to grasp the concepts in pseudocode, flowcharts, schematics, figures, and demonstrations than to learn the concepts from text and create abstractions of those concepts on my own. I shall improve my ability to create pseudocode, flowcharts, schematics, figures, and demonstrations from what I have read, especially in journal and conference papers.
			\item I shall improve my ability to perform statistical analysis on my experimental data, and analyze the figures/graphs that I have plotted.
			\item Use a book from the {\it Schaum's Outline} series from {\it McGraw-Hill} to help me learn material from introductory and intermediate classes. ``Even if you can't find a reference with exactly the type of coverage that works best for you, just reading about the same topic in two different places usually clarifies the ideas.'' [Remember how I read about the same topic in different advance engineering math textbooks to learn concepts for my classes in differential equations and vector analysis?]
			\item Working with others allows me to overcome obstacles that I may not be able to overcome on my own. While I may give up on learning certain things or overcome specific problems in individual projects, my teammates may be able to come up with solutions to problems in group projects. In addition, working in a diverse group exposes me to solutions that can be more effective and/or efficient... {\it students routinely teach one another in group work -- and as any professor will tell you, teaching something is probably the most effective way to learn it.}
			\item Try to find groups of three to four people to work on a problem. When I work in pairs, I may not expose myself to a sufficient variety of approaches. Similarly, when I work in larger groups (i.e., $> 4$), some individuals may be left out of the ``active problem-solving process''.
			\item I shall endeavor to outline solution on my own first, without being boggled or encumbered by the implementation details. Subsequently, I can work out the complete solutions with my group. If each individual does this, each group member can learn how to get started in solving problems in the project. That is, let's outline solutions to the problem, before we meet to discuss our considered solutions and develop the complete solution together.
			\item ``For group work to be fully effective, every group member should be able to explain in detail every solution obtained in a work session. Having the group members (particularly the weaker ones) go through these explanations before ending the session is a good way to make sure that the session has achieved its objectives.'' This will mitigate the tendency for the more technically challenged and reserved individuals to accept proposed solutions without understanding those solutions.
			\end{itemize}
		\item Richard M. Felder, ``How to Survive Engineering School,'' Department of Chemical and Biomolecular Engineering, North Carolina State University. Available at: \url{http://www4.ncsu.edu/unity/lockers/users/f/felder/public/Columns/Surviving-School.html}; last accessed on August 28, 2010.
		\end{enumerate}
	\end{enumerate}
%%%%%%%%%%%%%%%%%%%%%%%%%%%%%%%%%%%%%
%%%%%%%%%%%%%%%%%%%%%%%%%%%%%%%%%%%%%
%%%%%%%%%%%%%%%%%%%%%%%%%%%%%%%%%%%%%
\item Advice on teaching: \vspace{-0.3cm}
	\begin{enumerate} \itemsep -2pt
	\item Stanford University: \vspace{-0.2cm}
		\begin{enumerate} \itemsep -2pt
		\item Stanford University, {\it Teaching at Stanford}, Center for Teaching and Learning, Stanford University. Available at: \url{http://ctl.stanford.edu/teaching-at-stanford.html}; last accessed on September 1, 2010.
		\item Stanford University, {\it Handouts and Teaching Tips}, Center for Teaching and Learning, Stanford University. Available at: \url{http://ctl.stanford.edu/teachingta/handouts-and-teaching-tips.html}; last accessed on September 1, 2010.
		\item Stanford University, {\it Speaking of Teaching Newsletters}, Center for Teaching and Learning, Stanford University. Available at: \url{http://ctl.stanford.edu/speaking-of-teaching-newsletters.html}; last accessed on September 1, 2010.
		\end{enumerate}
	\item University of California, Riverside; Department of Mathematics: \vspace{-0.2cm}
		\begin{enumerate} \itemsep -2pt
		\item John Baez, ``How to Teach Stuff,'' Department of Mathematics, University of California, Riverside, January 23, 2006. Available at: \url{http://math.ucr.edu/home/baez/teaching.html}; last accessed on August 28, 2010.
		\end{enumerate}
	\item University of Oregon, Teaching and Learning Center: \vspace{-0.2cm}
		\begin{enumerate} \itemsep -2pt
		\item Teaching Effectiveness Program, {\it Teaching Resources}, Teaching and Learning Center, University of Oregon. Available at: \url{http://tep.uoregon.edu/resources/index.html}; last accessed on August 25, 2010. Also, look the ``Teaching FAQ's'': \url{http://tep.uoregon.edu/resources/faqs/}
		\item Teaching Effectiveness Program, Resources for {\it Teaching with Technology}, Teaching and Learning Center, University of Oregon. Available at: \url{http://tep.uoregon.edu/technology/index.html}; last accessed on August 25, 2010.
		\end{enumerate}
	\item North Carolina State University, Department of Chemical and Biomolecular Engineering: \vspace{-0.2cm}
		\begin{enumerate} \itemsep -2pt
		\item Richard M. Felder, {\it Student-centered Teaching and Learning}, Department of Chemical and Biomolecular Engineering, North Carolina State University. Available at: \url{http://www4.ncsu.edu/unity/lockers/users/f/felder/public/Student-Centered.html}; last accessed on August 28, 2010.
		\item Richard M. Felder, {\it Index of Learning Styles}, Department of Chemical and Biomolecular Engineering, North Carolina State University. Available at: \url{http://www4.ncsu.edu/unity/lockers/users/f/felder/public/ILSpage.html}; last accessed on August 28, 2010.
		\item Richard M. Felder, {\it Learning Styles}, Department of Chemical and Biomolecular Engineering, North Carolina State University. Available at: \url{http://www4.ncsu.edu/unity/lockers/users/f/felder/public/Learning_Styles.html}; last accessed on August 28, 2010.
		\item Richard M. Felder, {\it Richard Felder's Education-related Publications}, Department of Chemical and Biomolecular Engineering, North Carolina State University. Available at: \url{http://www4.ncsu.edu/unity/lockers/users/f/felder/public/Papers/Education_Papers.html}; last accessed on August 28, 2010.
		\end{enumerate}
	\item Joseph Frank Landsberger: \vspace{-0.2cm}
		\begin{enumerate} \itemsep -2pt
		\item Joseph Frank Landsberger, {\it Teaching Guides and Strategies}. Available at: \url{http://www.studygs.net/teaching/}; last accessed on August 25, 2010.
		\end{enumerate}
	\end{enumerate}
%%%%%%%%%%%%%%%%%%%%%%%%%%%%%%%%%%%%%
%%%%%%%%%%%%%%%%%%%%%%%%%%%%%%%%%%%%%
%%%%%%%%%%%%%%%%%%%%%%%%%%%%%%%%%%%%%
\item Resources to improve my English skills: \vspace{-0.3cm}
	\begin{enumerate} \itemsep -2pt
	\item {\it Guide to Online Schools}: \vspace{-0.2cm}
		\begin{enumerate} \itemsep -2pt
		\item {\it Guide to Online Schools} [or {\it GuideToOnlineSchools.com}], {\it Resources to Help Improve Your English Pronunciation}. Available at: \url{http://www.guidetoonlineschools.com/tips-and-tools/english-pronunciation}; last accessed on August 25, 2010.
		\end{enumerate}
	\end{enumerate}
%%%%%%%%%%%%%%%%%%%%%%%%%%%%%%%%%%%%%
%%%%%%%%%%%%%%%%%%%%%%%%%%%%%%%%%%%%%
%%%%%%%%%%%%%%%%%%%%%%%%%%%%%%%%%%%%%
\item time management: \vspace{-0.3cm}
	\begin{enumerate} \itemsep -2pt
	\item {\it Guide to Online Schools}: \vspace{-0.2cm}
		\begin{enumerate} \itemsep -2pt
		\item {\it Guide to Online Schools} [or {\it GuideToOnlineSchools.com}], {\it The Best Compilation of Time Management Resources on the Web}. Available at: \url{http://www.guidetoonlineschools.com/tips-and-tools/time-management}; last accessed on August 25, 2010.
		\end{enumerate}
	\end{enumerate}
%%%%%%%%%%%%%%%%%%%%%%%%%%%%%%%%%%%%%
%%%%%%%%%%%%%%%%%%%%%%%%%%%%%%%%%%%%%
%%%%%%%%%%%%%%%%%%%%%%%%%%%%%%%%%%%%%
\item Math and Science revision: \vspace{-0.3cm}
	\begin{enumerate} \itemsep -2pt
	\item Basic high school math: \vspace{-0.2cm}
		\begin{enumerate} \itemsep -2pt
		\item North Carolina State University, Department of Chemical and Biomolecular Engineering: \vspace{-0.1cm}
			\begin{itemize} \itemsep -1pt
			\item Kenny Felder and Gary Felder, ``Kenny's Math and Physics Help,'' 2009. Available at: \url{http://www4.ncsu.edu/unity/lockers/users/f/felder/public/kenny/home.html}; last accessed on August 28, 2010.
			\item Kenny Felder, ``Selected Other Educational Sites on the Web''. Available at: \url{http://www4.ncsu.edu/unity/lockers/users/f/felder/public/kenny/edulinks.html}; last accessed on August 28, 2010.
			\end{itemize}
		\end{enumerate}
	\end{enumerate}
%%%%%%%%%%%%%%%%%%%%%%%%%%%%%%%%%%%%%
%%%%%%%%%%%%%%%%%%%%%%%%%%%%%%%%%%%%%
%%%%%%%%%%%%%%%%%%%%%%%%%%%%%%%%%%%%%
\item good blogs about graduate school: \vspace{-0.3cm}
	\begin{enumerate} \itemsep -2pt
	\item Marc Eaddy, {\it Marc Eaddy: Confessions of an Ex-PhD Student}. Available at: \url{http://marceaddy.blogspot.com/}; last accessed on August 28, 2010.
	\item The Academic Blog Portal: \url{http://academicblogs.org/wiki/index.php/Main_Page}
	\end{enumerate}
%%%%%%%%%%%%%%%%%%%%%%%%%%%%%%%%%%%%%
%%%%%%%%%%%%%%%%%%%%%%%%%%%%%%%%%%%%%
%%%%%%%%%%%%%%%%%%%%%%%%%%%%%%%%%%%%%
\item fun stuff about grad school: \vspace{-0.3cm}
	\begin{enumerate} \itemsep -2pt
	\item Jorge Cham, {\it Piled Higher and Deeper}. Available at: \url{http://www.phdcomics.com/}; last accessed on August 28, 2010. See the latest ``PHD Comics'' at: \url{http://www.phdcomics.com/comics.php}. This comic strip pokes fun at the [fun, harsh, interesting, and absurd] realities of life in grad school.
	\item Jorge Cham, {\it Academica}. Available online at: \url{http://academia.edu/academica}; last accessed on October 3, 2010.
	\item Jorge Cham, {\it The PhD Forums}. Available online at: \url{http://www.phdcomics.com/proceedings/index.php}; last accessed on October 3, 2010. [ Has useful guidelines for surviving graduate school and is a decent resource for some technical support (e.g., with \LaTeX\ or C++). ]
	\end{enumerate}
%%%%%%%%%%%%%%%%%%%%%%%%%%%%%%%%%%%%%
%%%%%%%%%%%%%%%%%%%%%%%%%%%%%%%%%%%%%
%%%%%%%%%%%%%%%%%%%%%%%%%%%%%%%%%%%%%
\item other information about or related to grad school (and higher education): \vspace{-0.3cm}
	\begin{enumerate} \itemsep -2pt
	\item {\it Eurodoc}: union of grad student associations of each European country; see \url{http://en.wikipedia.org/wiki/EURODOC} and \url{http://www.eurodoc.net/}
	\item {\it European Association for Quality Assurance in Higher Education} (ENQA): union of accreditation board(s) of each European country; see \url{http://en.wikipedia.org/wiki/ENQA} and \url{http://www.enqa.eu/}
	\item {\it Innolyst}: \vspace{-0.2cm}
		\begin{enumerate} \itemsep -2pt
		\item Innolyst, {\it ResearchCrossroads}, Innolyst: \vspace{-0.1cm}
			\begin{enumerate} \itemsep -1pt
			\item Ernest Kuh, UC Berkeley: \url{http://www.researchcrossroads.org/Researchers/830167} or \url{http://www.researchcrossroads.org/index.php?option=com_content&view=article&id=49&Itemid=55&user_id=830167}
			\item US-based researchers who receive US government funding for their research have profiles in {\it ResearchCrossroads}. E.g., I can find the amount of public funding that my professors at USC received, and the organization that funds them. I can also read an abstract of the project that they got funded for.
			\item \url{http://www.researchcrossroads.org/}
			\end{enumerate}
		\item \url{http://www.innolyst.com/}
		\end{enumerate}
	\item A. Lee, C. Dennis, and P. Campbell. Nature's guide for mentors. Nature, 447(7146):791--797, June 14, 2007 \cite{Lee2007}.
	\item American Council of Trustees and Alumni (ACTA): \vspace{-0.2cm}
		\begin{enumerate} \itemsep -2pt
		\item The American Council of Trustees and Alumni (ACTA) is an independent, non-profit organization committed to academic freedom, excellence, and accountability at America�s colleges and universities.
		\item Launched in 1995, we are the only organization that works with alumni, donors, trustees, and education leaders across the United States to support liberal arts education, uphold high academic standards, safeguard the free exchange of ideas on campus, and ensure that the next generation receives a philosophically rich, high-quality college education at an affordable price.
		\item ACTA Publications: \url{https://www.goacta.org/publications/}. [ ACTA publications cover many aspects of issues concerning higher education institutions, and serve to provide standards of academic excellence and strategies for achieving these standards. ]
		\end{enumerate}
		\item GOOD, {\it GOOD Education}: \url{http://www.good.is/series/good-education/}
	\item Lumina Foundation for Education, {\it Publications}. Available at: \url{http://www.luminafoundation.org/publications/}; last accessed on September 4, 2010.
	\item National Center for Academic Transformation: \url{http://thencat.org/}
	\item Graduate Software Engineering 2009 (GSwE2009): \url{http://www.gswe2009.org/}
	\item CollegeMeasures.org (a joint endeavor by American Institutes for Research and Matrix Knowledge Group): \url{http://collegemeasures.org/}
	\item Association of American Colleges and Universities (AAC\&U): \vspace{-0.2cm}
		\begin{enumerate} \itemsep -2pt
		\item \url{http://www.aacu.org/}
		\item DiversityWeb: \vspace{-0.1cm}
			\begin{enumerate} \itemsep -1pt
			\item The DiversityWeb project is housed within the Office of Diversity, Equity and Global Initiatives at the Association of American Colleges and Universities (AAC\&U)
			\item \url{http://www.diversityweb.org/index.cfm}
			\end{enumerate}
		\end{enumerate}
	\item Newsweek Education: \url{http://education.newsweek.com/index.html}
	\item GRE: \cite{Green2000,Stewart2003,Kaplan2008,ETS2007,Wells2005,Lurie2002,Wu2007,Curtis2007}
	\item Council for Higher Education Accreditation: \url{http://www.chea.org/}
	\item Council of Graduate Schools (CGS): \vspace{-0.2cm}
		\begin{enumerate} \itemsep -2pt
		\item \url{http://www.cgsnet.org/}
		\item Ph.D. Completion Project: \url{http://www.phdcompletion.org/}
		\end{enumerate}
	\item Unigo: \vspace{-0.2cm}
		\begin{enumerate} \itemsep -2pt
		\item \url{http://www.unigo.com/}
		\item Offers students' perspectives on various aspects of college life, from admissions and dorm/college life to studying abroad and academics (studying skills and selecting a major)
		\end{enumerate}
	\item NAFSA / Association of International Educators (formerly, National Association of Foreign Student Advisers): \vspace{-0.2cm}
		\begin{enumerate} \itemsep -2pt
		\item \url{http://www.nafsa.org/}
		\item For Students: \vspace{-0.1cm}
			\begin{enumerate} \itemsep -1pt
			\item \url{http://www.nafsa.org/students.sec/}
			\item Resources about seeking financial aid for study abroad programs, absentee ballot procedure
			\end{enumerate}
		\item {\it Connecting Our World}: \url{http://www.connectingourworld.org/}
		\end{enumerate}
	\end{enumerate}
\end{itemize}



%%%%%%%%%%%%%%%%%%%%%%%%%%%%%%%%%%%%%%%%%
%\subsubsection{\hspace{0.1in} Zhiyang's Suggestions for Graduate School Applications}
%\label{zygradschapps}
\input{./grad_school_apps}

