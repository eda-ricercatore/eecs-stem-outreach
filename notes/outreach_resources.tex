\documentclass[letter,12pt]{article}
%%%%%%%%%%%%%%%%%%%%%%%%%%%%%%%%%%%%%%%%%%%%%%%%%
\usepackage{graphicx}
\usepackage{amsmath}
\usepackage{array}
\usepackage{amssymb}
\usepackage{setspace}
%\usepackage[margin=1.5cm,vmargin={0pt,1cm},nohead]{geometry}
\usepackage[margin=1in,vmargin={1in,1in}]{geometry}
% Package that has the symbol for ``:=''
\usepackage{txfonts}
% Create fancy headers and footers for this document
\usepackage{fancyhdr}
%\usepackage{cite}
% The ``cite'' package causes the hyperlinks for the in-text references/citations to fail. I believe it is because this package overrides the default package for referencing. Hence, only use the ``cite'' package with the IEEE format.
% Package for ``turnstile'' binary relations, where letters are defined above and below symbols
\usepackage{turnstile}
\usepackage{extarrows}
% Package that provides the cross symbol
\usepackage{ifsym}
\usepackage{marvosym}
% Commands for using the package for hyperlinks - 
\usepackage[pdftex,
	pdftitle={Graphics and Color with LaTeX},
	pdfauthor={Patrick W Daly},
	pdfsubject={Importing images and use of color in LaTeX},
	pdfkeywords={LaTeX, graphics, color},
	pdfpagemode=UseOutlines,bookmarks, bookmarksopen,
	pdfstartview=FitH, colorlinks, linkcolor=blue, citecolor=blue, urlcolor=red,
]{hyperref}
\hypersetup{colorlinks, linkcolor=blue}
% Concatenate references
\usepackage{cite}
%\usepackage{/data/others/notes/clrscode3e}
\usepackage{/Applications/apps/comune/SienaLaTeX/rapporto/clrscode3e}

% definition of new \LaTeX command for the citation: \cite{Cimatti08} and \cite{Barrett09}
% This allows mathematical/logic symbols to be typeset with the font ``Zapf Chancery'' in ``\LaTeX\ math mode''. To typeset symbols in such font, try: \mathpzc{ABCdef123}
\DeclareMathAlphabet{\mathpzc}{OT1}{pzc}{m}{it}

%%%%%%%%%%%%%%%%%%%%%%%%%%%%%%%%%%%%%%%%%%%%%
% Start of document
\begin{document}
\title{Resoures for Outreach Activities}
\date{\today}
\author{Zhiyang Ong
%\thanks{Email correspondence to: \href{mailto:ongz@acm.org}{ongz@acm.org}}
}
\maketitle

\begin{abstract} 
This is a list of resources for my outreach activities, which includes helping people explore careers in science, technology, engineering, and mathematics (STEM). It also includes resources to help parents and teachers of youths prepare youths for college; in addition, it has a list of scholarship resources. Furthermore, it has a list of resources that I use to help me with academic writing. Moreover, it also has resources to help people learn about various markets through publications based on market surveys of industries, such as semiconductors, biotechnology, and green technology. Finally, it has a list of resources to help people learn material from K-12 through advanced topics for graduate students.

\end{abstract}


%%%%%%%%%%%%%%%%%%%%%%%%%%%%%%%%%%%%%%%%%%%%%
%%%%%%%%%%%%%%%%%%%%%%%%%%%%%%%%%%%%%%%%%%%%%
% Create the table of contents
\tableofcontents
%%%%%%%%%%%%%%%%%%%%%%%%%%%%%%%%%%%%%%%%%%%%%
%%%%%%%%%%%%%%%%%%%%%%%%%%%%%%%%%%%%%%%%%%%%%

{\tt This is last updated on September 8, 2010.}

% This is written by Zhiyang Ong for his management of information and tasks.
%
% It includes information on professional development, including membership of professional organizations and networking societies.





%%%%%%%%%%%%%%%%%%%%%%%%%%%%%%%%%%%%%%%%%%%
\section{Heuristic for Locating Outreach Resources}
\label{heuristiclocateoutreach}

\proc{Find}$(\varphi, \tau)$ is a heuristic for locating resources for outreach activities, which includes finding information about the following: \vspace{-0.3cm}
\begin{enumerate} \itemsep -4pt
\item awards
\item career resources (including material for career guidance)
\item competitions and contests
\item educational material (e.g., suggested activities and curricular) for specific areas, such as marine sciences and electrical/computer engineering
\item fellowships
\item internships
\item scholarships
\item summer camps
\item summer programs (or summer schools); here, summer schools refer to short educational programs that last from days (e.g., a weekend for the ACM SIGDA Design Automation Summer School) to about a month (e.g., Santa Fe Institute's Complex Systems Summer Schools)
\end{enumerate}
\ \\

Its input $\tau$ is the deadline by which this search process must terminate. For example, if I have to apply for internships by next week, I would use the date of a week from now as the deadline $\tau$. In line \ref{find-pt-professional-org}, an example of a professional organization is the Institute of Electrical and Electronics Engineers (IEEE). The term ``good'' that is used in line \ref{find-pt-gd-uni} is an arbitrary measure of quality determined by the reader/user. \\

A reading group (in line \ref{find-pt-reading-grp}) is a small group of (graduate) students, which may possibly include professors and postdocs, that meet regularly (e.g., once/twice a week) to discuss papers that they have read since the previous meeting/discussion. Each individual in the reading group can be assigned a paper to read and present at the next meeting. The aim of a reading group is to improve the coverage of papers in our research area that each member has read. This is important for interdisciplinary research, since grad students working in interdisciplinary research areas have so much ground to cover. \\

Line \ref{find-pt-athletics} uses the term ``athletics department'' to refer to an administrative department at an American college or university that is in charge of managing varsity/NCAA sports teams. An example of a profession-specific networking organization (line \ref{find-pt-netwk-org}) is DVClub. In line \ref{find-pt-domain-specific-www}, a domain-specific web page is {\it SAT Live!}. An example of a corporate research laboratory (line \ref{find-pt-corporate-research-labs}) is ``Cadence Research Laboratories'' (\url{http://www.cadence.com/cadence/cadence_labs/pages/default.aspx}), and an example of a research institute (line \ref{find-pt-research-institute}) is Santa Fe Institute.




\begin{codebox}
\Procname{$\proc{Find}(\varphi, \tau)$}
\zi	\Comment {\it Input $\varphi \gets $ Item to find out about}
\zi	\Comment {\it Input $\tau \gets $ Deadline for the search process}
\zi	\Comment {\it Output $\kappa \gets $ List of resources about $\varphi$}
\zi
\li \While ( [ resources about $\varphi$ are inadequate ] AND [ $\tau$ has not yet passed ] )
	\Do
\li	Find out the professional organizations for the field of $\varphi$	\label{find-pt-professional-org}
\li	\For each professional organization in the field
		\Do
\li		Check if it has information about $\varphi$ in its web pages, publications, or mailing list archive
\li		\If (it has information about $\varphi$)
			\Then
\li			Add that information to $\kappa$
			\End
		\End
\zi
\li	\For each good (college OR university)	\label{find-pt-gd-uni}
		\Do
\li		\If ($\varphi == $ summer programs )
			\Then
\li			Search for summer programs in the web pages of departments \& schools/colleges
\li		\ElseIf ($\varphi == $ summer camps )
			\Then
\li			Search for summer camps in the web pages of departments \& schools/colleges
\li			Search for summer camps in the web pages of administrative/athletics departments	\label{find-pt-athletics}
\li		\ElseNoIf
\li			Search for $\varphi$ in the web pages of the department(s), including its news section/archive
\li			Search for $\varphi$ in the web pages of professors, postdoctoral researchers, \& students
\li			Search for $\varphi$ in the web pages of reading groups		\label{find-pt-reading-grp}
\li			Search for $\varphi$ in the web pages of student organizations
\li			Search for $\varphi$ in the mailing list archive of classes \& the department
\li			Search for $\varphi$ in the mailing list archive of research groups/labs and projects
\li			Search for $\varphi$ in the mailing list archive of reading groups
\li			Search for $\varphi$ in the mailing list archive of student organizations
			\End
\zi
\li		\If (it has information about $\varphi$)
			\Then
\li			Add that information to $\kappa$
			\End
		\End
\zi
\li	Search for $\varphi$ in the mailing list archive of open-source projects
\li	Search for $\varphi$ in the mailing list archive of profession-specific networking organizations	\label{find-pt-netwk-org}
\li	Search for $\varphi$ in the web pages of domain-specific web pages	\label{find-pt-domain-specific-www}
\li	Search for $\varphi$ in the web pages of research scientists in corporate research labs	\label{find-pt-corporate-research-labs}
\li	Search for $\varphi$ in the web pages of research scientists in research institutes	\label{find-pt-research-institute}
\li	\If ( [ mailing list archive OR web page ] has information about $\varphi$)
		\Then
\li		Add that information to $\kappa$
		\End
	\End	
\li \Return $\kappa$
\end{codebox}




%%%%%%%%%%%%%%%%%%%%%%%%%%%%%%%%%%%%%%%%%%%
\section{General Outreach Resources}
\label{generaloutreachresources}

General outreach resources: \vspace{-0.3cm}
\begin{enumerate} \itemsep -4pt
\item volunteering opportunities: \vspace{-0.3cm}
	\begin{enumerate} \itemsep -2pt
	\item Engineers Without Borders: \url{http://www.ewb-international.org/}
	\item Australian Volunteers International: \url{http://www.australianvolunteers.com/}
	\item Youth Challenge Australia: \url{http://www.youthchallenge.com.au/}
	\item Go Volunteer: \url{http://www.govolunteer.com.au/}
	\item Volunteer Search: \url{http://www.volunteersearch.gov.au/}
	\item Conservation Volunteers: \url{http://www.conservationvolunteers.com.au/volunteer}
	\item Volunteering Australia: \url{http://www.volunteeringaustralia.org/html/s01_home/home.asp}
	\item Sponsors for Educational Opportunity (SEO): \vspace{-0.2cm}
		\begin{enumerate} \itemsep -2pt
		\item Philanthropy \& Volunteerism Resources, \url{http://www.seo-usa.org/AlumniResources}
		\item Volunteer Leadership Opportunities: \url{http://www.seo-usa.org/Alumni_Volunteer}
		\end{enumerate}
	\item : \url{}
	\end{enumerate}
\item public health and preventive medicine: \vspace{-0.3cm}
	\begin{enumerate} \itemsep -2pt
	\item U.S. Department of Health \& Human Services: \vspace{-0.2cm}
		\begin{enumerate} \itemsep -2pt
		\item Agency for Healthcare Research and Quality (AHRQ): \vspace{-0.1cm}
			\begin{enumerate} \itemsep -1pt
			\item Prevention \& Care Management: Resources and Materials, \url{http://www.ahrq.gov/clinic/ppipix.htm}
			\end{enumerate}
		\end{enumerate}
	\end{enumerate}
\item career resources: \vspace{-0.3cm}
	\begin{enumerate} \itemsep -2pt
	\item CRAC: The Career Development Organisation: \vspace{-0.2cm}
		\begin{enumerate} \itemsep -2pt
		\item {\it icould}: \vspace{-0.1cm}
			\begin{enumerate} \itemsep -1pt
			\item \url{http://icould.com/about/}
			\item Resource for students, people who are commencing their careers or are making changes in their careers, career counselors, parents, educators, human resource staff, and employers.
			\item icould, {\it Stories by Life Theme}, in icould: Watch Career Stories. Available online at: \url{http://icould.com/watch-career-stories/by-life-theme/}; last accessed on December 25, 2010. [ Has articles briefly describing how people pursued their career goals or their career paths as they went through different experiences in life. This includes people who ``blossomed after school,'' changed careers or became entrepreneurs, had no plans, took risks, encountered turning points, faced adversity, have disabilities, went through financial hardship, or got laid off. It also has stories of people who volunteered, took a gap year, or pursued internships. ]
			\item icould, {\it Stories by Job Type}, in icould: Watch Career Stories. Available online at: \url{http://icould.com/watch-career-stories/by-job-type/}; last accessed on December 25, 2010. [ Includes stories of people in automotive retail, customer services, engineering, education, and many other job types. ]
			\end{enumerate}
		\end{enumerate}
	\item Jobs for the Future: \vspace{-0.2cm}
		\begin{enumerate} \itemsep -2pt
		\item \url{http://www.jff.org/}
		\item Current Projects: \url{http://www.jff.org/projects/current}
		\item Publications: \url{http://www.jff.org/publications}
		\item Policy: \url{http://www.jff.org/policy}
		\item Funders (funding agencies/organizations): \url{http://www.jff.org/funders}
		\item Programs: \url{http://www.jff.org/index.php?select=work}
		\end{enumerate}
	\item SkillsUSA: \vspace{-0.2cm}
		\begin{enumerate} \itemsep -2pt
		\item ``SkillsUSA is a partnership of students, teachers and industry working together to ensure America has a skilled work force. SkillsUSA helps each student excel.''
		\item Educators: \vspace{-0.1cm}
			\begin{enumerate} \itemsep -1pt
			\item \url{http://www.skillsusa.org/educators/index.shtml}
			\item Programs and Curricula: \url{http://www.skillsusa.org/educators/programs.shtml}
			\end{enumerate}
		\item Students: \vspace{-0.1cm}
			\begin{enumerate} \itemsep -1pt
			\item \url{http://www.skillsusa.org/students/index.shtml}
			\item Scholarships \& Financial Aid--SkillsUSA-related Scholarships: \url{http://www.skillsusa.org/students/scholarships.shtml}
			\end{enumerate}
		\item SkillsUSA competitions: \url{http://www.skillsusa.org/compete/index.shtml}
		\end{enumerate}
	\item others: \vspace{-0.2cm}
		\begin{enumerate} \itemsep -2pt
		\item public speaking and leadership: \vspace{-0.1cm}
			\begin{enumerate} \itemsep -1pt
			\item {\it Toastmasters International} is a non-profit educational organization that teaches public speaking and leadership skills through a worldwide network of meeting locations. Available online at: \url{http://www.toastmasters.org/}; last accessed on January 7, 2010.
			\end{enumerate}
		\end{enumerate}
	\end{enumerate}
\end{enumerate}




%%%%%%%%%%%%%%%%%%%%%%%%%%%%%%%%%%%%%%%%%%%
\section{Youth Outreach}
\label{youthoutreach}

Resources for youth outreach: \vspace{-0.3cm}
\begin{enumerate} \itemsep -4pt
%%%%%%%%%%%%%%%%%%%%%%%
\item educational (computer) games: \vspace{-0.3cm}
	\begin{enumerate} \itemsep -2pt
	\item Chevron Corporation: \vspace{-0.2cm}
		\begin{enumerate} \itemsep -2pt
		\item Energyville (about issues concerning energy and the environment): \url{http://www.willyoujoinus.com/energyville/}
		\end{enumerate}
	\item {\it Lego Digital Designer (LDD)}: \vspace{-0.2cm}
		\begin{enumerate} \itemsep -2pt
		\item CAD software for building Lego toys on Windows and Mac OS X platforms
		\item Free software, as in free beer
		\item \url{http://designbyme.lego.com/en-us/Default.aspx} and \url{http://ldd.lego.com/}
		\end{enumerate}
	\item Robocode: \vspace{-0.2cm}
		\begin{enumerate} \itemsep -2pt
		\item \url{http://en.wikipedia.org/wiki/Robocode} and \url{http://robocode.sourceforge.net/}
		\item Learn how to develop computer programs that will control a robot
		\end{enumerate}
	\item {\it Skill-Life}: \vspace{-0.2cm}
		\begin{enumerate} \itemsep -2pt
		\item \url{http://skill-life.com/}
		\item Use online games to teach youth life skills concerning financial literacy, nutrition, and citizenship.
		\end{enumerate}
	\item PowerUp (IBM with TryScience/New York Hall of Science): \vspace{-0.2cm}
		\begin{enumerate} \itemsep -2pt
		\item \url{http://www.powerupthegame.org/}
		\item Computer game to teach youths about energy conservation, global warming, renewable energy, and sustainable engineering
		\end{enumerate}
	\item EnergyNet: \vspace{-0.2cm}
		\begin{enumerate} \itemsep -2pt
		\item \url{http://www.energynet.net/games/}
		\item Computer game to teach youths about energy efficiency, and other topics related to energy
		\end{enumerate}
	\end{enumerate}
%%%%%%%%%%%%%%%%%%%%%%%
\item summer camps: \vspace{-0.3cm}
	\begin{enumerate} \itemsep -2pt
	\item United States Naval Academy: \vspace{-0.2cm}
		\begin{enumerate} \itemsep -2pt
		\item Naval Academy Athletic Association: \vspace{-0.1cm}
			\begin{enumerate} \itemsep -1pt
			\item Sports camps: \url{http://www.navysports.com/camps/navy-camps.html}
			\end{enumerate}
		\end{enumerate}
	\end{enumerate}
%%%%%%%%%%%%%%%%%%%%%%%
\item competitions for youths: \vspace{-0.3cm}
	\begin{enumerate} \itemsep -2pt
	\item International Geography Olympiad (for high school students): \url{http://www.geoolympiad.org/}
	\item International Linguistic Olympiad (for high school students): \url{http://en.wikipedia.org/wiki/International_Linguistics_Olympiad}
	\item International Philosophy Olympiad (for high school students): \url{http://www.philosophy-olympiad.org/}
	\item JA Worldwide: Responsible People Business Competition (for students in North and South America, and Europe), \url{http://www.responsible-business.org/}
	\item The Choral Arts Society of Washington: \vspace{-0.2cm}
		\begin{enumerate} \itemsep -2pt
		\item \url{http://www.choralarts.org/MLK-Celebration-Community-Initiative/Writing-Competition.aspx}
		\item ``As part of our MLK Celebration Community Initiative and in celebration of Black History Month, The Choral Arts Society of Washington hosts an annual writing competition for students in grades K-12.''
		\item ``Each year, students are presented with a different writing prompt and are asked to respond in poetic form.''
		\item ``Students are encouraged to be creative in their writing and to use their knowledge of Martin Luther King, Jr.'s life, the Civil Rights Movement, and current events as inspiration for their writing.''
		\end{enumerate}
	\item Vocal Arts DC (or Vocal Arts Society): \vspace{-0.2cm}
		\begin{enumerate} \itemsep -2pt
		\item Young Artists Competition: \vspace{-0.1cm}
			\begin{enumerate} \itemsep -1pt
			\item \url{http://vocalartsdc.org/youngartists.shtml}
			\item ``Each year, Vocal Arts DC holds a vocal competition open to all singers who are residents of the greater DC area, including Baltimore and Annapolis.''
			\item ``Singers are asked to submit a CD for review along with a sample recital program that the singer is prepared to sing in recital. The CDs will be reviewed in a blind audition and finalist will be selected for live auditions.''
			\item ``Two winners are selected from the finalists and are presented in the Art Song Discovery Series in four different venues across the greater DC area.''
			\end{enumerate}
		\end{enumerate}
	\item The John F. Kennedy Center for the Performing Arts: \vspace{-0.2cm}
		\begin{enumerate} \itemsep -2pt
		\item The National Symphony Orchestra (NSO): \vspace{-0.1cm}
			\begin{enumerate} \itemsep -1pt
			\item Young Soloists' Competition (High School Division; Washington metropolitan area): \url{http://www.kennedy-center.org/nso/nsoed/youngsoloists.cfm#concerts}
			\end{enumerate}
		\end{enumerate}
	\item Center for Interactive Learning and Collaboration (CILC): \vspace{-0.2cm}
		\begin{enumerate} \itemsep -2pt
		\item Kids Creating Community Content KC$^{3}$ International Contest (for students in Middle and High School): \vspace{-0.1cm}
			\begin{enumerate} \itemsep -1pt
			\item \url{http://kc3.cilc.org/} and \url{http://kc3.cilc.org/guidelines.htm}
			\item Make a short film to educate others about the uniqueness of your community, geographical region, natural/agricultural resources, local/national treasures, culture/heritage, or country.
			\end{enumerate}
		\end{enumerate}
	\end{enumerate}
%%%%%%%%%%%%%%%%%%%%%%%
\item educational resources: \vspace{-0.3cm}
	\begin{itemize} \itemsep -2pt
	\item Xcel Energy Foundation: \vspace{-0.2cm}
		\begin{enumerate} \itemsep -2pt
		\item Focus Area Grants: \vspace{-0.1cm}
			\begin{enumerate} \itemsep -1pt
			\item \url{http://www.xcelenergy.com/Minnesota/Company/Community/Xcel%20Energy%20Foundation/Pages/Focus_Area_Grants.aspx}
			\item Scope of eligible funding, and details on the grant application process
			\end{enumerate}
		\item Education Initiatives: \vspace{-0.1cm}
			\begin{enumerate} \itemsep -1pt
			\item \url{http://www.xcelenergy.com/Minnesota/Company/Community/Education%20Initiatives/Pages/Education_Initiatives.aspx}
			\item Energy Safety Calendar Program, K-6: \vspace{-0.1cm}
				\begin{itemize} \itemsep -1pt
				\item \url{http://www.xcelenergy.com/New%20Mexico/Company/Community/Education%20Initiatives/Pages/Energy_Safety_Calendar_ProgramK-6.aspx}
				\item ``The Energy Safety Calendar Program offers K-6 students in our service territory a great opportunity to learn about electricity and natural gas safety.''
				\end{itemize}
			\end{enumerate}
		\item Safety World: \vspace{-0.1cm}
			\begin{enumerate} \itemsep -1pt
			\item \url{http://www.xcelenergy.com/New%20Mexico/Company/Community/Education%20Initiatives/Pages/Safety_World.aspx}
			\item e-SMART kid: \vspace{-0.1cm}
				\begin{itemize} \itemsep -1pt
				\item \url{http://www.e-smartonline.net/xcelenergy/}
				\item Help children and youth learn about ``electricity and natural gas and how to use them safely''
				\end{itemize}
			\end{enumerate}
		\item Energy Classroom: \vspace{-0.1cm}
			\begin{enumerate} \itemsep -1pt
			\item \url{http://www.energyclassroom.com/}
			\item \url{http://www.xcelenergy.com/Minnesota/Company/Community/Pages/Energy_Classroom.aspx}
			\item Educational material for students about energy sources, energy conservation, and environmental protection
			\item For Teachers (educational material and suggested class activities): \url{http://www.energyclassroom.com/index.php?id=34&page=For_Teachers}
			\end{enumerate}
		\item Power Plant Tour Information: \url{http://www.xcelenergy.com/New%20Mexico/Company/About_Energy_and_Rates/Power%20Generation/Pages/Power_Plant_Tour_Information.aspx}
		\end{enumerate}
	\item HowStuffWorks, Inc.: \url{http://www.howstuffworks.com/}
	\item Chevron Corporation: \vspace{-0.2cm}
		\begin{enumerate} \itemsep -2pt
		\item {\it Will you join us}: \vspace{-0.1cm}
			\begin{enumerate} \itemsep -1pt
			\item Energy issues: \url{http://www.willyoujoinus.com/energyissues/}
			\item Tools and resources: \vspace{-0.1cm}
				\begin{itemize} \itemsep -1pt
				\item \url{http://www.willyoujoinus.com/toolsresources/}
				\item Helpful links (includes K-12 educational material): \url{http://www.willyoujoinus.com/toolsresources/helpfullinks/}
				\end{itemize}
			\item MPG Optimizer: \url{http://www.willyoujoinus.com/usingenergywisely/mpgoptimizer/}
			\item Energy generator: \url{http://www.willyoujoinus.com/usingenergywisely/energygenerator/}
			\end{enumerate}
		\end{enumerate}
	\item National Energy Foundation: \vspace{-0.2cm}
		\begin{enumerate} \itemsep -2pt
		\item \url{http://www.nef.org.uk/} and \url{http://www.nef1.org/}
		\item Students: \url{http://www.nef1.org/students.html}
		\item Educators: \url{http://www.nef1.org/educators.html}
		\item Schools: \vspace{-0.1cm}
			\begin{enumerate} \itemsep -1pt
			\item \url{http://www.nef.org.uk/communities/schools/index.html}
			\item Helpful links: \url{http://www.nef.org.uk/communities/schools/energylinks.html}
			\item School Resources: \url{http://www.nef.org.uk/communities/schools/resources/index.html}
			\item {\it LogiCity} is a fun interactive computer game with a difference. It's a game set in a 3D virtual city with five main activities where you are set the task of reducing the carbon footprint of an average resident. See \url{http://www.nef.org.uk/communities/schools/logicity.html}.
			\end{enumerate}
		\item Resources: \url{http://www.nef.org.uk/actonCO2/index.asp}
		\item Igniting Creative Energy - A National Student Challenge: \vspace{-0.1cm}
			\begin{enumerate} \itemsep -1pt
			\item \url{http://www.ignitingcreativeenergy.org/}
			\item Students: \url{http://www.ignitingcreativeenergy.org/students.html}
			\end{enumerate}
		\end{enumerate}
	\item StartSpot Mediaworks: \vspace{-0.2cm}
		\begin{enumerate} \itemsep -2pt
		\item StartSpot Network: \vspace{-0.1cm}
			\begin{enumerate} \itemsep -1pt
			\item HomeworkSpot: \vspace{-0.1cm}
				\begin{itemize} \itemsep -1pt
				\item \url{http://www.homeworkspot.com/}
				\item Science Fair Project Center: \url{http://www.homeworkspot.com/sciencefair/}
				\end{itemize}
			\end{enumerate}
		\end{enumerate}
	\item Super Science Fair Projects: \url{http://www.super-science-fair-projects.com/}
	\item All Science Fair Projects: Science Fair Projects with Complete Instructions, \url{http://www.all-science-fair-projects.com/}
	\item The Science Club: \vspace{-0.2cm}
		\begin{enumerate} \itemsep -2pt
		\item \url{http://scienceclub.org/}
		\item Science Fair Idea Exchange: \url{http://scienceclub.org/scifair.html}
		\end{enumerate}
	\item Oracle Education Foundation: \vspace{-0.2cm}
		\begin{enumerate} \itemsep -2pt
		\item \url{http://www.oraclefoundation.org/}
		\item ThinkQuest: \vspace{-0.1cm}
			\begin{enumerate} \itemsep -1pt
			\item \url{http://www.thinkquest.org/en/}
			\item ThinkQuest International Competition: \url{http://www.thinkquest.org/competition/}
			\item Projects: \url{http://thinkquest.org/en/projects/index.html}
			\item Library: \url{http://thinkquest.org/pls/html/think.library}
			\item Example of a computer game developed by students: Crisis! - The Game, \url{http://library.thinkquest.org/20331/game/}
			\end{enumerate}
		\end{enumerate}
	\item University of Minnesota: \vspace{-0.2cm}
		\begin{enumerate} \itemsep -2pt
		\item Institute on Community Integration; College of Education and Human Development: \vspace{-0.1cm}
			\begin{enumerate} \itemsep -1pt
			\item National Center on Secondary Education and Transition (NCSET): \vspace{-0.1cm}
				\begin{itemize} \itemsep -1pt
				\item \url{http://www.ncset.org/}
				\item NCSET Topics: \url{http://www.ncset.org/topics/default.asp}
				\item Web Sites: \url{http://www.ncset.org/websites/default.asp}
				\item The Youthhood!: \url{http://www.youthhood.org/}
				\end{itemize}
			\end{enumerate}
		\end{enumerate}
	\item Jobs for America's Graduates: \vspace{-0.2cm}
		\begin{enumerate} \itemsep -2pt
		\item \url{http://www.jag.org/}
		\item JAG Model program applications: \vspace{-0.1cm}
			\begin{enumerate} \itemsep -1pt
			\item \url{http://www.jag.org/model.htm}
			\item Programs are available for students in middle school and high school, high school dropouts, high school seniors, students in alternative education programs, and college underclassmen
			\end{enumerate}
		\item JAG Career Corner: \url{http://www.jag.org/jag_career_corner.htm}
		\item Students: \url{http://www.jag.org/students.htm}
		\item Resource library: \url{http://www.jag.org/library.htm}
		\item Performance outcomes: \url{http://www.jag.org/outcomes.htm}
		\item Funding: \url{http://www.jag.org/funding.htm}
		\end{enumerate}
	\item Alliance to Save Energy: \vspace{-0.2cm}
		\begin{enumerate} \itemsep -2pt
		\item Energy Hog campaign: \vspace{-0.1cm}
			\begin{enumerate} \itemsep -1pt
			\item \url{http://www.energyhog.org/}
			\item Adults: \url{http://www.energyhog.org/adult/adults.htm}
			\item Children: \url{http://www.energyhog.org/childrens.htm}
			\end{enumerate}
		\end{enumerate}
	\item Learning First Alliance: \vspace{-0.2cm}
		\begin{enumerate} \itemsep -2pt
		\item \url{http://www.learningfirst.org/}
		\item Issues and publications: \url{http://www.learningfirst.org/issues}
		\item Resources: \url{http://www.learningfirst.org/resources}
		\end{enumerate}
	\item NaMaYa: \url{http://www.namaya.com/}
	\item NIXTY: \url{http://nixty.com/}
	\item K12 Open Ed: \url{http://www.k12opened.com/wiki/index.php/Main_Page}
	\item Learning Is For Everyone: \url{http://www.learningis4everyone.org/}
	\item The Smithsonian Commons Prototype: \url{http://www.si.edu/commons/prototype/}
	\item Futurelab: Resources for educators and parents, \url{http://www.futurelab.org.uk/resources}
	\item Innosight Institute: Resources for education, \url{http://www.innosightinstitute.org/practices/education/}
	\item WGBH Educational Foundation: \url{http://www.wgbh.org/}
	\item Discovery Education: \vspace{-0.2cm}
		\begin{enumerate} \itemsep -2pt
		\item Classroom resources: \url{http://school.discoveryeducation.com/}
		\item Home resources: \url{http://school.discoveryeducation.com/homeworkhelp/homework_help_home.html}
		\end{enumerate}
	\item The Gilder Lehrman Institute of American History: \vspace{-0.2cm}
		\begin{enumerate} \itemsep -2pt
		\item \url{http://www.gilderlehrman.org/}
		\item Resources for teachers and schools: \url{http://www.gilderlehrman.org/teachers/}
		\item Civil War Essay Contest (for students in selected K-12 schools): \url{http://www.gilderlehrman.org/affiliate/civil_war.php}
		\end{enumerate}
	\item The GRAMMY Museum: \vspace{-0.2cm}
		\begin{enumerate} \itemsep -2pt
		\item Teacher curriculum and resources. Available online at: \url{http://www.grammymuseum.org/interior.php?section=education&page=teachercurriculum}; last accessed on November 15, 2010.
		\end{enumerate}
	\item Purdue University: \vspace{-0.2cm}
		\begin{enumerate} \itemsep -2pt
		\item Department of Entomology: \vspace{-0.1cm}
			\begin{enumerate} \itemsep -1pt
			\item Genomics Analogy Model for Educators (G.A.M.E.): \url{http://www.entm.purdue.edu/extensiongenomics/GAME/default.html}
			\end{enumerate}
		\end{enumerate}
	\item Verizon Thinkfinity: \url{http://www.thinkfinity.org/about-us}
	\item Oregon Virtual School District (ORVSD): \vspace{-0.2cm}
		\begin{enumerate} \itemsep -2pt
		\item \url{http://orvsd.org/}
		\item ``Oregon Virtual School District (ORVSD) helps integrate technology into Oregon public school classrooms by giving teachers access to free tech tools and resources online.''
		\item ``The Oregon Virtual School District is a program led by the Oregon Department of Education that, in cooperation with a consortium of virtual learning providers throughout the state, seeks to increase access and availability of online learning and teaching resources free of charge to public school teachers of Oregon. Oregon State University is providing hosting and development resources through a partnership with the OSU Open Source Lab and the OSU Business Solutions Group.''
		\end{enumerate}
	\item The Association of Educational Publishers (AEP): \vspace{-0.2cm}
		\begin{enumerate} \itemsep -2pt
		\item The AEP Awards: \vspace{-0.1cm}
			\begin{enumerate} \itemsep -1pt
			\item \url{http://www.aepweb.org/awards/index.htm}
			\item Look at the winners of previous AEP awards to determine some of the good educational resources that are available
			\end{enumerate}
		\end{enumerate}
	\item Educational Dividends: \vspace{-0.2cm}
		\begin{enumerate} \itemsep -2pt
		\item \url{http://www.educationaldividends.com/}
		\item Teachers: \vspace{-0.1cm}
			\begin{enumerate} \itemsep -1pt
			\item \url{http://www.educationaldividends.com/index.asp?menu=Teachers}
			\item Teaching Tools: \url{http://www.educationaldividends.com/teachers/tools.asp}
			\item Reference Desk: \vspace{-0.1cm}
				\begin{itemize} \itemsep -1pt
				\item \url{http://www.educationaldividends.com/teachers/reference.asp}
				\item Standards Reference Desk (resources for education standards in the US at the national, state, and local levels): \url{http://www.educationaldividends.com/teachers/standards_desk.asp}
				\item How We Learn: Learning Styles, \url{http://www.educationaldividends.com/teachers/learning_styles.asp}
				\item How We Learn: Multiple Intelligences, \url{http://www.educationaldividends.com/teachers/multiple_intelligences.asp}
				\item Statistics Desk (statistical information about education in the US): \url{http://www.educationaldividends.com/teachers/statistics_desk.asp}
				\end{itemize}
			\item Information about the teaching profession: \vspace{-0.1cm}
				\begin{itemize} \itemsep -1pt
				\item \url{http://www.educationaldividends.com/teachers/welcome.asp}
				\item Office of Occupational Statistics and Employment Projections, ``Educational Services,'' in {\it Career Guide to Industries}, 2010-11 Edition, U.S. Bureau of Labor Statistics, U.S. Department of Labor, Washington, DC, December 17, 2009. Available online at: \url{http://stats.bls.gov/oco/cg/cgs034.htm}; last accessed on December 8, 2010. [ Suggested citation: Bureau of Labor Statistics, U.S. Department of Labor, {\it Career Guide to Industries, 2010-11 Edition}, Educational Services , on the Internet at \url{http://www.bls.gov/oco/cg/cgs034.htm} (visited December 07, 2010). ]
				\item Experience Teaching: \url{http://www.educationaldividends.com/teachers/experience.asp}
				\item Continuous Improvement: \url{http://www.educationaldividends.com/teachers/toolkit.asp}
				\end{itemize}
			\end{enumerate}
		\item Personality and Career Tests: \url{http://www.educationaldividends.com/teachers/tests.asp}
		\end{enumerate}
	\item Smithsonian Institution: \vspace{-0.2cm}
		\begin{enumerate} \itemsep -2pt
		\item Educators: \url{http://www.si.edu/Educators}
		\item Smithsonian Institution Traveling Exhibition Service (SITES): \vspace{-0.1cm}
			\begin{enumerate} \itemsep -1pt
			\item For Teachers: \url{http://www.sites.si.edu/education/teachers_res2.htm}
			\end{enumerate}
		\item Smithsonian Folkways Recordings (or simply, Smithsonian Folkways): \vspace{-0.1cm}
			\begin{enumerate} \itemsep -1pt
			\item Tools for Teaching: \url{http://www.folkways.si.edu/tools_for_teaching/introduction.aspx}
			\end{enumerate}
		\item Freer Gallery of Art / Arthur M. Sackler Gallery: \vspace{-0.1cm}
			\begin{enumerate} \itemsep -1pt
			\item Resources for Educators: \url{http://www.asia.si.edu/explore/teacherResources.asp}
			\item Explore + Learn: Browse Online Resources by Area: \vspace{-0.1cm}
				\begin{itemize} \itemsep -1pt
				\item \url{http://www.asia.si.edu/explore/default.asp}
				\item Has resources for art in: \vspace{-0.1cm}
					\begin{itemize} \itemsep -1pt
					\item The Americas
					\item Ancient Egypt
					\item Ancient Near East
					\item Islamic world
					\item China
					\item Japan
					\item Korea
					\item South Asia
					\item Himalayas
					\item Southeast Asian
					\item It also has biblical manuscripts and contemporary art
					\end{itemize}
				\end{itemize}
			\item Online Exhibition Features: \url{http://www.asia.si.edu/exhibitions/online.asp}
			\item Collections: \url{http://www.asia.si.edu/collections/default.asp}
			\end{enumerate}
		\item National Museum of American History: \vspace{-0.1cm}
			\begin{enumerate} \itemsep -1pt
			\item Jerome and Dorothy Lemelson Center for the Study of Invention and Innovation: \vspace{-0.1cm}
				\begin{itemize} \itemsep -1pt
				\item Resources: \vspace{-0.1cm}
					\begin{itemize} \itemsep -1pt
					\item \url{http://invention.smithsonian.org/resources/}
					\item \url{http://invention.smithsonian.org/resources/default_sites_weblinks.aspx}
					\item Invention stories - archives, articles, audio, and video: \url{http://invention.smithsonian.org/resources/default_index.aspx}
					\end{itemize}
				\item Educational Materials: \vspace{-0.1cm}
					\begin{itemize} \itemsep -1pt
					\item \url{http://invention.smithsonian.org/resources/menu_edu_materials.aspx}
					\item Experiments: \url{http://invention.smithsonian.org/resources/menu_edu_materials.aspx?MaterialTypeID=3&MaterialTypeDesc=Experiments}
					\item Educational Materials: \url{http://invention.smithsonian.org/resources/menu_edu_materials_f.aspx?MaterialTypeDesc=Features}
					\end{itemize}
				\item Centerpieces: \vspace{-0.1cm}
					\begin{itemize} \itemsep -1pt
					\item \url{http://invention.smithsonian.org/centerpieces/}
					\item \url{http://invention.smithsonian.org/centerpieces/iap-info.aspx}
					\item Electric guitar: \url{http://invention.smithsonian.org/centerpieces/electricguitar/index.htm}
					\item Innovative Lives: \url{http://invention.smithsonian.org/centerpieces/ilives/}
					\item ``Exploring the History of Women Inventors'' by J.E. Bedi (in {\it Innovative Lives}): \url{http://invention.smithsonian.org/centerpieces/ilives/womeninventors.html}
					\item Whole Cloth: \url{http://invention.smithsonian.org/centerpieces/whole_cloth/index.html}
					\item The Quartz Watch: \url{http://invention.smithsonian.org/centerpieces/quartz/index.html}
					\item Edison Invents!: All about Thomas Edison and his invention, \url{http://invention.smithsonian.org/centerpieces/edison/default.asp}
					\end{itemize}
				\item Modern Inventors Documentation Program (MIND): \url{http://invention.smithsonian.org/resources/mind_resources.aspx}
				\item Invention at Play: \vspace{-0.1cm}
					\begin{itemize} \itemsep -1pt
					\item \url{http://inventionatplay.org/}
					\item Resources: \url{http://inventionatplay.org/resources.html}
					\item Invention Playhouse: \url{http://inventionatplay.org/playhouse_main.html}
					\item Inventors' Stories: \url{http://inventionatplay.org/inventors_main.html}
					\item Does play matter? (using play to help children learn and think): \url{http://inventionatplay.org/matter_main.html}
					\end{itemize}
				\item Spark!Lab: \vspace{-0.1cm}
					\begin{itemize} \itemsep -1pt
					\item \url{http://sparklab.si.edu/}
					\item About Spark!Lab (introduce children to the process of innovation via play and fun activities): \url{http://sparklab.si.edu/spark-about.html}
					\item Activities \& Experiments: \url{http://sparklab.si.edu/spark-experiments.html}
					\item Inventor Profiles: \url{http://sparklab.si.edu/spark-inventors.html}
					\item Resources: \url{http://sparklab.si.edu/spark-resources.html}
					\end{itemize}
				\end{itemize}
			\end{enumerate}
		\end{enumerate}
	\item Economic and Social Research Council (ESRC): \vspace{-0.2cm}
		\begin{enumerate} \itemsep -2pt
		\item {\it Social Science for Schools}; Science in Society Team: \vspace{-0.1cm}
			\begin{enumerate} \itemsep -1pt
			\item \url{http://www.esrcsocietytoday.ac.uk/ESRCInfoCentre/ssfs/}
			\item Social science resources: \url{http://www.esrcsocietytoday.ac.uk/ESRCInfoCentre/ssfs/resources/}
			\item Career guides for different disciplines in social science and economics: \url{http://www.esrcsocietytoday.ac.uk/ESRCInfoCentre/ssfs/careers/}
			\item Related online resources: \url{http://www.esrcsocietytoday.ac.uk/ESRCInfoCentre/ssfs/links/}
			\end{enumerate}
		\end{enumerate}
	\end{itemize}
\item National Council for Accreditation of Teacher Education (NCATE): \vspace{-0.3cm}
	\begin{enumerate} \itemsep -2pt
	\item \url{http://www.ncate.org/}
	\item Has resources about degree programs in education and their accreditation, as well as how to become a teacher
	\item State-specific Recognized Programs by NCATE and Specialized Professional Associations (SPAs): \vspace{-0.2cm}
		\begin{enumerate} \itemsep -2pt
		\item \url{http://www.ncate.org/tabid/165/Default.aspx}
		\item Find out about educational programs in: \vspace{-0.1cm}
			\begin{enumerate} \itemsep -1pt
			\item special education
			\item early childhood education
			\item educational leadership
			\item educational technology specialist
			\item elementary education
			\item English
			\item health education
			\item foreign languages
			\item gifted education
			\item mathematics
			\item physical education
			\item science education
			\item school psychology
			\item secondary computer science education
			\item social studies
			\item Teachers of English to Speakers of Other Languages (TESOL)
			\item technology and engineering educators
			\end{enumerate}
		\end{enumerate}
	\item Financial Aid Resources for Teacher Education Students: \url{http://www.ncate.org/Public/CurrentFutureTeachers/FinancialAidResources/tabid/351/Default.aspx}
	\end{enumerate}
%%%%%%%%%%%%%%%%%%%%%%%
\item scholarships: \vspace{-0.3cm}
	\begin{enumerate} \itemsep -2pt
	\item U.S. Department of State: \vspace{-0.2cm}
		\begin{enumerate} \itemsep -2pt
		\item Bureau of Educational and Cultural Affairs: \vspace{-0.1cm}
			\begin{enumerate} \itemsep -1pt
			\item National Security Language Initiative for Youth (NSLI-Y): \vspace{-0.1cm}
				\begin{itemize} \itemsep -1pt
				\item \url{http://exchanges.state.gov/youth/programs/nsli.html}
				\item ``The State Department�s National Security Language Initiative for Youth (NSLI-Y) provides merit-based scholarships to U.S. high school students and recent graduates interested in learning less-commonly studied foreign languages.''
				\end{itemize}
			\end{enumerate}
		\end{enumerate}
	\end{enumerate}
%%%%%%%%%%%%%%%%%%%%%%%
\item underrepresented minorities: \vspace{-0.3cm}
	\begin{enumerate} \itemsep -2pt
	\item The University of North Carolina at Chapel Hill: \vspace{-0.2cm}
		\begin{enumerate} \itemsep -2pt
		\item Gary Bishop, {\it Research}, Department of Computer Science, The University of North Carolina at Chapel Hill. Available at: \url{http://wwwx.cs.unc.edu/~gb/wp/research/}; last accessed on September 3, 2010. [ Has plenty of information and resources (including learning aids and material) to help people who are visually or mobility impaired learn. ]
		\end{enumerate}
	\item Myra Sadker Foundation: \vspace{-0.2cm}
		\begin{enumerate} \itemsep -2pt
		\item $100+$ Ideas to Promote Gender Equity in Schools and Beyond: \url{http://www.sadker.org/100ideas.html}
		\item Gender Equity Activities: \url{http://www.sadker.org/WhatYouCanDo.html}
		\item Gender Equity Activities for Concerned Citizens: \url{http://www.sadker.org/GenderEquity-citizens.html}
		\item Gender Equity Activities for Families: \url{http://www.sadker.org/GenderEquity-family.html}
		\item Gender Equity Activities for Teachers: \vspace{-0.1cm}
			\begin{enumerate} \itemsep -1pt
			\item Early Childhood: \url{http://www.sadker.org/GenderEquity-teacher1.html}
			\item Primary Grades: \url{http://www.sadker.org/GenderEquity-teacher2.html}
			\item Upper Elementary: \url{http://www.sadker.org/GenderEquity-teacher3.html}
			\item Middle and High School: \url{http://www.sadker.org/GenderEquity-teacher4.html}
			\end{enumerate}
		\item Resources for feminism and links to web pages of feminist organizations: \url{http://www.sadker.org/ReadsLinks.html}
		\end{enumerate}
	\item League of United Latin American Citizens (LULAC): \vspace{-0.3cm}
		\begin{enumerate} \itemsep -2pt
		\item LULAC National Educational Service Centers, Inc: \vspace{-0.2cm}
			\begin{enumerate} \itemsep -2pt
			\item \url{http://www.lnesc.org/}
			\item Programs: \vspace{-0.1cm}
				\begin{itemize} \itemsep -1pt
				\item Improving literacy among Latino/Latina youth
				\item Encouraging Latino/Latina youth to pursue careers in science and engineering
				\item Helping Latino/Latina youth acquire leadership skills
				\item Improving college access for Latino/Latina youth by mentoring and summer programs (e.g., Gear-Up, Upward Bound, and the PALMS Initiative)
				\item Helping Latino/Latina families acquire financial success, so that Latino/Latina youth can pursue higher education
				\item Scholarships for Latino/Latina youth
				\item \url{http://lnesc.org/index.asp?Type=B_BASIC&SEC={808B6D04-913C-483F-8A05-5BD44B03ED62}}
				\end{itemize}
			\end{enumerate}
		\end{enumerate}
	\item ASPIRA: \vspace{-0.2cm}
		\begin{enumerate} \itemsep -2pt
		\item ASPIRA Programs for Latino/Latina youth: \url{http://aspira.org/manuals/aspira-programs}
		\end{enumerate}
	\end{enumerate}
%%%%%%%%%%%%%%%%%%%%%%%
\item places to visit: \vspace{-0.3cm}
	\begin{enumerate} \itemsep -2pt
	\item Exploratorium @ The Palace of Fine Arts (San Francisco, CA): \url{http://www.exploratorium.edu/}
	\item Educational Dividends: \vspace{-0.2cm}
		\begin{enumerate} \itemsep -2pt
		\item \url{http://www.educationaldividends.com/}
		\item Suggestions for organizing field trips to explore your interests: \url{http://www.educationaldividends.com/students/student_issues.asp}
		\item Career exploration: \url{http://www.educationaldividends.com/students/career_choices.asp}
		\item Computer skills: \url{http://www.educationaldividends.com/students/technology.asp}
		\item Quizzes to help you find out what is your preferred learning style and to discover more about your personality: \url{http://www.educationaldividends.com/students/learning_quiz.asp}
		\item Resources to help you learn about various topics in science, mathematics, social science, and humanities: \url{http://www.educationaldividends.com/students/resources.asp}
		\end{enumerate}
	\end{enumerate}
%%%%%%%%%%%%%%%%%%%%%%%
\item resources for at-risk youths: \vspace{-0.3cm}
	\begin{enumerate} \itemsep -2pt
	\item At-Risk Youth: \url{http://www.at-risk.org/}
	\item Peace First: \vspace{-0.2cm}
		\begin{enumerate} \itemsep -2pt
		\item \url{http://www.peacefirst.org/site/}
		\item To help youths become ``problem-solvers, rather than witnesses, or victims of their surrounding''
		\item To reduce youth homicide rates
		\item Teach children ``critical conflict resolution skills''
		\item Help teachers improve their ``conflict resolution and classroom management skills''
		\item To encourage youths to help each other, and get them to break up fights
		\item ``The Peace First curriculum is tailored to meet the developmental needs of students in Pre-K through eighth grade. Once a week, young adult volunteers and classroom teachers work together to teach students about friendship, communication, and conflict resolution through the use of experiential activities. First graders learn about communicating their feelings, third graders work on being peacemakers in their classroom, and fifth graders explore how to resolve and deescalate conflicts.''
		\item Has programs for students/youths, teachers, principals, and volunteers.
		\end{enumerate}
	\item Americans for the Arts: \vspace{-0.2cm}
		\begin{enumerate} \itemsep -2pt
		\item YouthARTS: \vspace{-0.1cm}
			\begin{enumerate} \itemsep -1pt
			\item \url{http://www.artsusa.org/youtharts/index.asp}
			\item ``The YouthARTS site is designed to give arts agencies, juvenile justice agencies, social service organizations, and other community-based organizations detailed information about how to plan, run, provide training, and evaluate arts programs for at-risk youth.''
			\end{enumerate}
		\end{enumerate}
	\end{enumerate}
%%%%%%%%%%%%%%%%%%%%%%%
\item general music and arts education: \vspace{-0.3cm}
	\begin{enumerate} \itemsep -2pt
	\item Americans for the Arts: \vspace{-0.2cm}
		\begin{enumerate} \itemsep -2pt
		\item Americans for the Arts, ``Ten Simple Ways Parents Can Get More Art in Their Kids' Lives.'' Available online at: \url{http://www.americansforthearts.org/public_awareness/get_involved/001.asp}; last accessed on November 30, 2010.
		\item YouthARTS: \vspace{-0.1cm}
			\begin{enumerate} \itemsep -1pt
			\item \url{http://www.artsusa.org/youtharts/index.asp}
			\item ``The YouthARTS site is designed to give arts agencies, juvenile justice agencies, social service organizations, and other community-based organizations detailed information about how to plan, run, provide training, and evaluate arts programs for at-risk youth.''
			\end{enumerate}
		\end{enumerate}
	\item The John F. Kennedy Center for the Performing Arts: \vspace{-0.2cm}
		\begin{enumerate} \itemsep -2pt
		\item Kennedy Center Institute for Arts Management: \url{http://artsmanagerfba.artsmanager.org/common/Pages/About.aspx}
		\item {\sc ArtsEdge}: \vspace{-0.1cm}
			\begin{enumerate} \itemsep -1pt
			\item The National Standards for Arts Education for Grades K-4, 5-8, and 9-12: \url{http://artsedge.kennedy-center.org/educators/standards.aspx}
			\item Tips and guides for educators: \url{http://artsedge.kennedy-center.org/educators/how-to.aspx}
			\item Lesson plans for educators: \url{http://artsedge.kennedy-center.org/educators/lessons.aspx}
			\item Information for parents, guardians, foster parents, baby-sitters, and grandparents: \url{http://artsedge.kennedy-center.org/families.aspx}
			\item Information for students: \url{http://artsedge.kennedy-center.org/students.aspx}
			\item Themes for artistic, cultural, academic, and intellectual exploration: \url{http://artsedge.kennedy-center.org/themes.aspx}
			\item Multimedia: \url{http://artsedge.kennedy-center.org/multimedia.aspx}
			\end{enumerate}
		\end{enumerate}
	\end{enumerate}
\item music education: \vspace{-0.3cm}
	\begin{enumerate} \itemsep -2pt
	\item Washington Performing Arts Society (WPAS): \vspace{-0.2cm}
		\begin{enumerate} \itemsep -2pt
		\item WPAS Education \& Community -- Connections through the Arts Education Programs for All Ages: \vspace{-0.1cm}
			\begin{enumerate} \itemsep -1pt
			\item The Capitol Jazz Project: \vspace{-0.1cm}
				\begin{itemize} \itemsep -1pt
				\item \url{http://www.wpas.org/educcomm/programsforyoungpeople/capitoljazzproject.aspx}
				\item ``Washington Performing Arts Society (WPAS) and the D.C. Public Schools, in collaboration with Jazz at Lincoln Center, has launched The Capitol Jazz Project, an important step in supporting music education for all students in the District of Colombia.''
				\item ``Through the Capitol Jazz Project, students hone their listening, performing, improvising, composing, arranging, music reading, and notation skills.''
				\item ``The Capitol Jazz Project is being implemented in 6 D.C. middle schools with a total enrollment of more than 500 music students.''
				\item ``A true collaboration, The Capitol Jazz Project brings the combined resources and expertise of WPAS, Jazz at Lincoln Center, and the D.C. Public Schools to create a model music education program.''
				\end{itemize}
			\item Joseph and Goldie Feder Memorial String Competition: \vspace{-0.1cm}
				\begin{itemize} \itemsep -1pt
				\item \url{http://www.wpas.org/educcomm/programsforyoungpeople/josephandgoldiefedermemorialstringcompetition.aspx}
				\item ``The Feder String Competition inspires and nurtures D.C. area youth in grades 6 through 12 who study violin, viola, cello, and double bass.''
				\item ``Each year, 80 students compete for 30 awards and scholarships.''
				\item ``Held each spring, WPAS awards cash prizes toward private lessons, scholarships for summer study programs, and tickets for top winners and their family members to attend a WPAS concert.''
				\item ``Winners of the competition are also given special performance opportunities such as on the Kennedy Center's Millennium Stage and The Shakespeare Theatre Company's Happenings at the Harman series.''
				\end{itemize}
			\item WPAS Summer Performing Arts Academy summer programs: \vspace{-0.1cm}
				\begin{itemize} \itemsep -1pt
				\item \url{http://www.wpas.org/educcomm/programsforyoungpeople/wpassummerperformingartsacademy.aspx}
				\end{itemize}
			\end{enumerate}
		\end{enumerate}
	\item Young Concert Artists, Inc. \vspace{-0.2cm}
		\begin{enumerate} \itemsep -2pt
		\item Annaliese Soros Educational Residency Program: \url{http://www.yca.org/auditions/}
		\end{enumerate}
	\item The Choral Arts Society of Washington: \vspace{-0.2cm}
		\begin{enumerate} \itemsep -2pt
		\item Classroom Resources: \url{http://www.choralarts.org/Education/Classroom-Resources.aspx}
		\end{enumerate}
	\item League of American Orchestras: \vspace{-0.2cm}
		\begin{enumerate} \itemsep -2pt
		\item Career planning: \vspace{-0.1cm}
			\begin{enumerate} \itemsep -1pt
			\item Resources for pre-college students, college students, and graduate students: \url{http://www.americanorchestras.org/career_center/career_planning.html}
			\item Arts Administration programs: \url{http://www.americanorchestras.org/career_center/arts_admin_programs.html}
			\item Non-profit management, {\bf public policy} and leadership programs: \url{http://www.americanorchestras.org/career_center/resources_non_prof_and.html}
			\end{enumerate}
		\end{enumerate}
	\item The John F. Kennedy Center for the Performing Arts: \vspace{-0.2cm}
		\begin{enumerate} \itemsep -2pt
		\item Betty Carter's Jazz Ahead: \vspace{-0.1cm}
			\begin{enumerate} \itemsep -1pt
			\item \url{http://www.kennedy-center.org/programs/jazz/jazzahead/}
			\item ``Music residency program for young people''
			\item ``The Jazz Ahead program identifies outstanding, emerging jazz artists in their mid-teens to age thirty, and brings them together under the tutelage of experienced artist-instructors who coach and counsel them, helping to polish their performance, composing and arranging skills.''
			\item ``The two week-long residency program includes daily workshops and rehearsals with established jazz artists, and culminate in three concerts on the Kennedy Center Millennium Stage, which will be broadcast live over the internet.''
			\end{enumerate}
		\item The National Symphony Orchestra (NSO): \vspace{-0.1cm}
			\begin{enumerate} \itemsep -1pt
			\item The National Symphony Orchestra's Summer Music Institute (SMI): \vspace{-0.1cm}
				\begin{itemize} \itemsep -1pt
				\item \url{http://www.kennedy-center.org/nso/nsoed/smi/home.cfm}
				\item ``Every summer, approximately 70 students (ages 15-20) from all over the nation meet in Washington, D.C., to attend the National Symphony Orchestra's Summer Music Institute (SMI).''
				\item ``The Institute offers four weeks of private lessons, rehearsals, coaching by National Symphony Orchestra members, classes, and lectures to prepare aspiring musicians for their futures in music.''
				\end{itemize}
			\item Young Associates' Program: \vspace{-0.1cm}
				\begin{itemize} \itemsep -1pt
				\item \url{http://www.kennedy-center.org/nso/nsoed/youngassociates.html}
				\item ``The National Symphony Orchestra (NSO) is sponsoring its Young Associates' Program for high school students in grades 11 and 12 in the Washington, DC, metropolitan area who are interested in pursuing a musical career.''
				\item ``Twenty outstanding instrumentalists (pianists are not included) will be selected to attend rehearsals of the National Symphony Orchestra and take part in seminars with conductors, artists, NSO musicians, and representatives of the arts management field.''
				\item ``Through this program, the Young Associates will acquire an appreciation of the wide range of skills, knowledge, and abilities--managerial as well as musical--that are required to put together a performance by a major symphony orchestra. Selection process is by application.''
				\item ``The core of the program involves attendance at rehearsals of the National Symphony Orchestra at the Kennedy Center and observation of various guest artists. In addition to attending NSO rehearsals, students participate in workshops to explore careers in management, music education, publicity, music library, and other professions that are essential to the life of every successful orchestra.''
				\item ``Students do not play their instruments as part of the program. Students learn through listening, observation, and asking questions of professionals.''
				\end{itemize}
			\end{enumerate}
		\end{enumerate}
	\end{enumerate}
\item dance education: \vspace{-0.3cm}
	\begin{enumerate} \itemsep -2pt
	\item The Washington Ballet: \vspace{-0.2cm}
		\begin{enumerate} \itemsep -2pt
		\item The Washington School of Ballet (TWSB): \vspace{-0.1cm}
			\begin{enumerate} \itemsep -1pt
			\item Summer Intensive program (requires an audition): \url{http://www.washingtonballet.org/the-school/summer-intensive/}
			\end{enumerate}
		\item TWB's EXCEL! scholarship program (for DanceDC students): \vspace{-0.1cm}
			\begin{enumerate} \itemsep -1pt
			\item \url{http://www.washingtonballet.org/community-engagement/default.htm}
			\item \url{http://www.washingtonballet.org/community-engagement/other-programs/}
			\item Also, has need-based scholarships
			\end{enumerate}
		\end{enumerate}
	\item The John F. Kennedy Center for the Performing Arts: \vspace{-0.2cm}
		\begin{enumerate} \itemsep -2pt
		\item Exploring Ballet With Suzanne Farrell: A Three-Week Summer Ballet Intensive for Young Dancers: \vspace{-0.1cm}
			\begin{enumerate} \itemsep -1pt
			\item \url{http://www.kennedy-center.org/education/farrell/}
			\item ``In July and August, students from across the United States and around the world will participate in the eighteenth annual session of the Kennedy Center's ballet training program Exploring Ballet with Suzanne Farrell. The three-week residency for dancers ages 14 to 18 with at least five years of ballet training will be held at the Kennedy Center from August 1 - August 20, 2011.''
			\item ``During the three-week period, students take two ballet technique classes a day, six days a week, with Ms. Farrell. Students also participate in a number of cultural activities to enhance their experience in Washington, D.C., including museum visits, trips to historical landmarks, and attending performances.''
			\end{enumerate}
		\item Dance Theatre of Harlem Residency program: \vspace{-0.1cm}
			\begin{enumerate} \itemsep -1pt
			\item \url{http://www.kennedy-center.org/education/community/programs.html#artistic}
			\item ``Since 1993, the Kennedy Center's Dance Theatre of Harlem Residency program has provided ballet training for male and female students age 8-18 with identified promise in ballet taught by Dance Theatre of Harlem (DTH) instructors or former principal dancers.''
			\item ``Students are selected by audition for a twenty-class series, culminating with a public demonstration and performance on a Kennedy Center main stage.''
			\item ``Classical ballet training is taught in four class levels, from novice to advance.''
			\item ``Students must have at least one year of ballet training to qualify for the program.''
			\end{enumerate}
		\end{enumerate}
	\end{enumerate}
%%%%%%%%%%%%%%%%%%%%%%%
\item JA Worldwide (Junior Achievement): \vspace{-0.3cm}
	\begin{enumerate} \itemsep -2pt
	\item \url{http://www.ja.org/}
	\item Resources for educators: \url{http://www.ja.org/involved/involved_educat.shtml}
	\item Resources for parents: \url{http://www.ja.org/involved/involved_parents.shtml}
	\item Resources for students: \url{http://www.ja.org/involved/involved_students.shtml}
	\end{enumerate}
\item U.S. Department of State: \vspace{-0.3cm}
	\begin{enumerate} \itemsep -2pt
	\item Programs for Americans and non-Americans.
	\item Summer Work Travel - In the summer work travel program: \url{http://exchanges.state.gov/}
	\item Cultural Programs Division: \url{http://exchanges.state.gov/cultural/index.html}
	\item Youth Programs Division: \url{http://exchanges.state.gov/youth/index.html}
	\item EducationUSA: \url{http://educationusa.state.gov/}
	\item International Visitor Leadership Program: \url{http://exchanges.state.gov/ivlp/ivlp.html}
	\item Programs for non-U.S. Citizens: \url{http://exchanges.state.gov/prog-non-us.html}
	\item Programs for U.S. Citizens: \url{http://exchanges.state.gov/prog-us.html}
	\item Resources for Students: \url{http://exchanges.state.gov/student.html}
	\item Bureau of Educational and Cultural Affairs: \vspace{-0.2cm}
		\begin{enumerate} \itemsep -2pt
		\item Future Leaders Exchange (FLEX) Program: \vspace{-0.1cm}
			\begin{enumerate} \itemsep -1pt
			\item \url{http://exchanges.state.gov/youth/programs/flex.html}
			\item ``The Future Leaders Exchange (FLEX) Program gives students (ages 15-17) the chance to live with a host family and attend a U.S. high school for a year.''
			\end{enumerate}
		\item Office of Citizen Exchanges: \vspace{-0.1cm}
			\begin{enumerate} \itemsep -1pt
			\item Youth Programs Division: \vspace{-0.1cm}
				\begin{itemize} \itemsep -1pt
				\item \url{http://exchanges.state.gov/youth/index.html}
				\item Has programs for youths in various parts of the world
				\item ``The Youth Programs Division is committed to empowering the next generation and establishing long-lasting ties between the United States and other countries through exchange programs and institutional partnerships. Programs focus primarily on secondary schools and promote mutual understanding, leadership development, educational transformation and democratic ideals.''
				\end{itemize}
			\item SportsUnited: \vspace{-0.1cm}
				\begin{itemize} \itemsep -1pt
				\item \url{http://exchanges.state.gov/sports/index.html}
				\item SportsUnited is an international sports programming initiative designed to help start a dialogue at the grassroots level with non-elite boys and girls ages 7-17.
				\item The programs aid youth in discovering how success in athletics can be translated into the development of life skills and achievement in the classroom.
				\item Foreign participants are given an opportunity to establish links with U.S. sports professionals and exposure to American life and culture.
				\item Americans learn about foreign cultures and the challenges young people from overseas face today.
				\item The U.S. Department of State has programmed initiatives in: baseball, basketball, football, track and field, soccer, volleyball, wrestling, archery, boxing, swimming, fencing, table tennis, ice skating, weightlifting, water polo and managing sports community centers.
				\item Countries covered by this program are listed on the web page.
				\item Sports Envoy Program: \vspace{-0.1cm}
					\begin{itemize} \itemsep -1pt
					\item \url{http://exchanges.state.gov/sports/envoy1.html}
					\item Working with the national sports leagues and the U.S. Olympic Committee, athletes and coaches in various sports are chosen to serve as envoys or ambassadors of sport in overseas programs that include conducting clinics, visiting schools and speaking to youth.
					\item The American athletes and coaches conduct drills and team building activities, as well as engage the youth in a dialogue on the importance of an education, positive health practices and respect for diversity.
					\end{itemize}
				\item Sports Grant Competition: \vspace{-0.1cm}
					\begin{itemize} \itemsep -1pt
					\item The Bureau of Educational and Cultural Affairs (ECA) has an annual open competition under its International Sports Programming Initiative.
					\item Public and private non-profit organizations, 501(c)(3), may submit proposals to discuss approaches designed to enhance and improve the infrastructure of youth sports programs.
					\item The focus of all programs must be reaching out to non-elite youth ages 7-17 and/or their coaches/administrators.
					\item There are four themes that a proposal can address; Youth Sports Management, Training Sports Coaches, Sport and Disability, and Sport and Health.
					\item The list of eligible countries changes each year.
					\item \url{http://exchanges.state.gov/sports/index/sports-grant-competition.html}
					\end{itemize}
				\item Sports Visitor Program: \vspace{-0.1cm}
					\begin{itemize} \itemsep -1pt
					\item Nominated by our U.S. embassies overseas, selected athletes, managers and coaches are brought to the U.S. for technical sports training, sports management, conflict resolution training and exposure to valuable U.S. sports contacts and then are encouraged to return to conduct in-country clinics for youth with their newly learned skills.
					\item \url{http://exchanges.state.gov/sports/visitors.html}
					\end{itemize}
				\end{itemize}
			\end{enumerate}
		\end{enumerate}
	\end{enumerate}
\item U.S. Department of Labor: \vspace{-0.3cm}
	\begin{enumerate} \itemsep -2pt
	\item Wage and Hour Division: \vspace{-0.2cm}
		\begin{enumerate} \itemsep -2pt
		\item YouthRules!: \vspace{-0.1cm}
			\begin{enumerate} \itemsep -1pt
			\item \url{http://youthrules.dol.gov/}
			\item Has information for youths, parents, educators, and employers on how to let youth work part-time safely
			\item Teens: \url{http://youthrules.dol.gov/teens/default.htm}
			\item Parents: \url{http://youthrules.dol.gov/parents/default.htm}
			\item Educators: \url{http://youthrules.dol.gov/educators/default.htm}
			\item Employers: \url{http://youthrules.dol.gov/employers/default.htm}
			\item Resources: \url{http://youthrules.dol.gov/resources.htm}
			\item Compliance Assistance: \url{http://youthrules.dol.gov/ca.htm}
			\end{enumerate}
		\end{enumerate}
	\end{enumerate}
\item ASCL Educational Services, Inc. (Marc McCulloch): \vspace{-0.3cm}
	\begin{enumerate} \itemsep -2pt
	\item Transitions: Life Skills for Personal Success!: \vspace{-0.2cm}
		\begin{enumerate} \itemsep -2pt
		\item Curriculum \& Materials: \url{http://transitions.ascl.info/infomaterials}
		\item Soft Skills: \url{http://transitions.ascl.info/infoskills}
		\end{enumerate}
	\end{enumerate}
\item Partnership for 21st Century Skills: \vspace{-0.3cm}
	\begin{enumerate} \itemsep -2pt
	\item \url{http://www.p21.org/}
	\item Framework for 21st Century Learning: \url{http://www.p21.org/index.php?option=com_content&task=view&id=254&Itemid=119}
	\item Tools and Resources: \url{http://www.p21.org/index.php?option=com_content&task=view&id=273&Itemid=139}
	\end{enumerate}
\item National Career and Technical Education Foundation (NCTEF): \vspace{-0.3cm}
	\begin{enumerate} \itemsep -2pt
	\item States' Career Clusters Initiative (SCCI): \vspace{-0.2cm}
		\begin{enumerate} \itemsep -2pt
		\item \url{http://www.careerclusters.org/}
		\item The 16 Career Clusters: \url{http://www.careerclusters.org/16clusters.cfm}
		\item Plans of Study: \url{http://www.careerclusters.org/resources/web/pos.cfm}
		\item Knowledge and Skills Charts: \url{http://www.careerclusters.org/resources/web/ks.php}
		\item Crosswalks: \url{http://www.careerclusters.org/crosswalks.cfm}
		\item Publications: \url{http://www.careerclusters.org/publications.php}
		\item Sixteen Career Clusters and Their Pathways: \url{http://www.careerclusters.org/list16clusters.php}
		\item Career Clusters Models: \url{http://www.careerclusters.org/resources/web/16ccall.php?action=models}
		\item Career Clusters Brochure Previews: \url{http://www.careerclusters.org/resources/web/16ccall.php?action=brochures}
		\item Career Clusters Interest Survey: \url{http://www.careerclusters.org/ccinterestsurvey.php}
		\item Related Websites: \url{http://www.careerclusters.org/related.php}
		\end{enumerate}
	\end{enumerate}
\item U. S. Department of Labor: \vspace{-0.3cm}
	\begin{enumerate} \itemsep -2pt
	\item Employment and Training Administration: \vspace{-0.2cm}
		\begin{enumerate} \itemsep -2pt
		\item CareerOneStop: \vspace{-0.1cm}
			\begin{enumerate} \itemsep -1pt
			\item \url{http://www.careeronestop.org/}
			\item Students, parents, and career advisors: \url{http://www.careeronestop.org/studentsandcareeradvisors/studentsandcareeradvisors.aspx}
			\end{enumerate}
		\end{enumerate}
	\end{enumerate}
\item U. S. Department of Defense: \vspace{-0.3cm}
	\begin{enumerate} \itemsep -2pt
	\item ASVAB Career Exploration Program: \vspace{-0.2cm}
		\begin{enumerate} \itemsep -2pt
		\item \url{http://www.asvabprogram.com/}
		\item Learn about yourself: \url{http://www.asvabprogram.com/index.cfm?fuseaction=learn.main}
		\item Explore careers: \url{http://www.asvabprogram.com/index.cfm?fuseaction=explore.main}
		\item Plan for your future: \url{http://www.asvabprogram.com/index.cfm?fuseaction=plan.main}
		\item Information for educators and career counselors: \url{http://www.asvabprogram.com/index.cfm?fuseaction=edu.main}
		\item Information for parents: \url{http://www.asvabprogram.com/index.cfm?fuseaction=parents.main}
		\end{enumerate}
	\end{enumerate}
\end{enumerate}



%%%%%%%%%%%%%%%%%%%%%%%%%%%%%%%%%%%%%%%%%%%
\section{Internship Opportunities}
\label{Internship Opportunities}

Internship opportunities: \vspace{-0.3cm}
\begin{enumerate} \itemsep -4pt
\item Canada: \vspace{-0.3cm}
	\begin{enumerate} \itemsep -2pt
	\item SWAP: \vspace{-0.2cm}
		\begin{enumerate} \itemsep -2pt
		\item \url{http://www.swap.ca/}
		\item For Canadians who want to work abroad: \url{http://www.swap.ca/out_eng/index.aspx}
		\item For citizens of selected countries who want to work in Canada: \url{http://www.swap.ca/in_eng/partner_organizations.aspx}
		\end{enumerate}
	\end{enumerate}
\item Singapore: \vspace{-0.3cm}
	\begin{enumerate} \itemsep -2pt
	\item Speedwing Training (Asia) Pte Ltd: \vspace{-0.2cm}
		\begin{enumerate} \itemsep -2pt
		\item \url{http://www.speedwing.org/}
		\item For Singaporeans who want to work in the United States, Canada, Germany, and New Zealand
		\item For citizens of selected countries who want to work in Singapore
		\end{enumerate}
	\end{enumerate}
\end{enumerate}

%%%%%%%%%%%%%%%%%%%%%%%%%%%%%%%%%%%%%%%%%%%
\subsection{Internship Opportunities in Australia}
\label{internshipaus}

Internship Opportunities in Australia: \vspace{-0.3cm}
\begin{enumerate} \itemsep -4pt
\item The Association of Professional Engineers, Scientists and Managers, Australia: \url{http://www.apesma.asn.au/index.asp} --- Ask for guide to internships in your region/major; free student membership
\item Engineers Australia: \url{http://www.engineersaustralia.org.au/} --- Ask for guide to internships in your region/major; free student membership
\item CPA Australia: \url{http://www.cpaaustralia.com.au/cps/rde/xchg/cpa/hs.xsl/index.html} and \url{http://www.cpaaustralia.com.au/cps/rde/xchg/careers/site/index_ENA_HTML.htm/cps/rde/xchg/SID-3F57FECB-EEFEF50E/careers/site/204_ENA_HTML.htm}
\item Institute of Chartered Accountants in Australia: \url{http://www.charteredaccountants.com.au/}
\item 
\end{enumerate}


%%%%%%%%%%%%%%%%%%%%%%%%%%%%%%%%%%%%%%%%%%%
\subsection{Internship Opportunities in Europe}
\label{internshipeu}

Internship Opportunities in Portugal: \vspace{-0.3cm}
\begin{enumerate} \itemsep -4pt
\item Portugal: \vspace{-0.3cm}
	\begin{enumerate} \itemsep -2pt
	\item IAESTE Portugal (The International Association for the Exchange of Students for Technical Experience): \url{http://www.iaeste.pt/en/foreign-trainees/why-portugal/}
	\end{enumerate}
\item United Kingdom: \vspace{-0.3cm}
	\begin{enumerate} \itemsep -2pt
	\item Graduate Talent Pool: \url{http://graduatetalentpool.direct.gov.uk/}
	\end{enumerate}
\end{enumerate}




%%%%%%%%%%%%%%%%%%%%%%%%%%%%%%%%%%%%%%%%%%%
\subsection{Internship Opportunities in the United States}
\label{internshipsus}

Internship Opportunities in the United States: \vspace{-0.3cm}
\begin{enumerate} \itemsep -4pt
\item Use the Procedure \proc{Find}$(\varphi, \tau)$ in \S\ref{heuristiclocateoutreach} to look up internship opportunities and lists of internship opportunities.

Look at government organizations (e.g., the White House), nonprofit organizations (e.g., Engineers Without Borders), professional organizations (e.g., IEEE and ACM), colleges and universities, and companies (e.g., Intel, Google, and start-ups).

You can start your search by looking at the organizations that provide resources for underrepresented minorities as well as resources for scholarships and fellowships. These information can be found in other sections of this document.

If you do not know where to start, speak to a professor or staff member at the career center of your college/university. Alternatively, you can ask your awesome resident advisors (RAs).

My personal advice is to start your search based on your interests and skill set. You can always narrow the search space based on factors, such as geographical location, later on.

Competitive internships, especially research internships in electrical and computer engineering or computer science, weed out many students from applying via demanding job requirements. For example, if you want to apply for research internships with electronic design automation (EDA) companies and corporate research labs, you would need to have significant experience designing integrated circuits and developing EDA software. The stringent job requirements also mean that students need to plan in advance (say, about a year) about the internships that they would like to seek, and plan to acquire the necessary skill set and experiences before the application deadlines (which can be several months before the start of your internship).

Taking as many challenging classes as you can possibly cope, especially in electrical and computer engineering or computer science, would provide you with a skill set that allows you to apply for competitive internships in many fields. Apart from taking challenging classes as well as engaging in research and/or open source projects, you can try to acquire additional skills and experience in your free time to boost the competitiveness of your internship application. Certain skills and experiences, such as compiler design, are hard to acquire in your free time, so it would be ``easier'' to take classes that would help you acquire those skills and experiences.

Note that you may want to look into creating your own entrepreneurial venture, say an EDA start-up or organization in social entrepreneurship, rather than to seek an internship. Also, seeking an internship abroad is always a good addition to your resume/CV.
\item National Science Foundation: \vspace{-0.3cm}
	\begin{enumerate} \itemsep -2pt
	\item Research Experiences for Undergraduates (REU): \vspace{-0.2cm}
		\begin{enumerate} \itemsep -2pt
		\item \url{http://www.nsf.gov/crssprgm/reu/reu_search.cfm}
		\item Academic fields: \vspace{-0.1cm}
			\begin{enumerate} \itemsep -1pt
			\item Astronomical Sciences
			\item Atmospheric and Geospace Sciences
			\item Biological Sciences
			\item Chemistry
			\item Computer and Information Science and Engineering
			\item Cyberinfrastructure
			\item Department of Defense (DoD)
			\item Earth Sciences
			\item Education and Human Resources
			\item Engineering
			\item Ethics and Values Studies
			\item International Science and Engineering
			\item Materials Research
			\item Mathematical Sciences
			\item Ocean Sciences
			\item Physics
			\item Polar Programs
			\item Social, Behavioral, and Economic Sciences
			\end{enumerate}
		\end{enumerate}
	\end{enumerate}
\item Society for Industrial and Applied Mathematics: \vspace{-0.3cm}
	\begin{enumerate} \itemsep -2pt
	\item Internship and Career Information in Industry, Research Institutions, and Government Labs: \url{http://www.siam.org/careers/internships.php}
	\end{enumerate}
\item American Institute of Physics (AIP): \vspace{-0.3cm}
	\begin{enumerate} \itemsep -2pt
	\item Society of Physics Students (SPS): \vspace{-0.2cm}
		\begin{enumerate} \itemsep -2pt
		\item SPS Internships: \url{http://www.spsnational.org/programs/internships/}
		\item Research Opportunities: \url{http://www.spsnational.org/programs/research/}
		\end{enumerate}
	\end{enumerate}
%%%%%%%%%%%%%%%%%%%%%%%%%%%%%%%%%%%%%%
%%%%%%%%%%%%%%%%%%%%%%%%%%%%%%%%%%%%%%
%%%%%%%%%%%%%%%%%%%%%%%%%%%%%%%%%%%%%%
\item United States Office of Personnel Management: \vspace{-0.3cm}
	\begin{enumerate} \itemsep -2pt
	\item USAJOBS: \vspace{-0.2cm}
		\begin{enumerate} \itemsep -2pt
		\item Student Jobs: \url{http://www.usajobs.gov/studentjobs/}
		\end{enumerate}
	\end{enumerate}
%%%%%%%%%%%%%%%%%%%%%%%%%%%%%%%%%%%%%%
%%%%%%%%%%%%%%%%%%%%%%%%%%%%%%%%%%%%%%
%%%%%%%%%%%%%%%%%%%%%%%%%%%%%%%%%%%%%%
\item Americans for the Arts: \vspace{-0.3cm}
	\begin{enumerate} \itemsep -2pt
	\item Internship Program: \url{http://www.americansforthearts.org/about_us/internships.asp}
	\end{enumerate}
\item New York Women's Foundation: \vspace{-0.3cm}
	\begin{enumerate} \itemsep -2pt
	\item Internship Opportunities: \url{http://www.nywf.org/internship.html}
	\item Volunteer Opportunities: \url{http://www.nywf.org/volunteer.html}
	\end{enumerate}
\item Council on International Educational Exchange (CIEE): \url{http://www.ciee.org/hire/index.aspx}
\item The John F. Kennedy Center for the Performing Arts: \vspace{-0.3cm}
	\begin{enumerate} \itemsep -2pt
	\item Kennedy Center Arts Management Internships: \url{http://www.kennedy-center.org/education/artsmanagement/internships/}
	\end{enumerate}
\item Washington Performing Arts Society (WPAS): \vspace{-0.3cm}
	\begin{enumerate} \itemsep -2pt
	\item Internships with WPAS: \vspace{-0.2cm}
		\begin{enumerate} \itemsep -2pt
		\item \url{http://www.wpas.org/aboutwpas/opportunities/intern.aspx}
		\item ``WPAS offers internships throughout the year. Applicants should be highly motivated, creative and hard-working individuals with an interest in all aspects of arts management. It is required that applicants have previous office experience.''
		\item In addition, applicants should possess: \vspace{-0.1cm}
			\begin{enumerate} \itemsep -1pt
			\item Interest/background in music, dance or performance art
			\item Strong organizational skills
			\item Effective writing and communication skills
			\item Ability to learn quickly, handle multiple tasks, take initiative, and work independently with little supervision
			\item High energy level and ability to work well in deadline and/or pressure situations
			\item Computer literacy
			\end{enumerate}
		\item ``WPAS interns leave our offices with a better understanding of arts management, knowledge of artists in a variety of fields (classical music, world music, dance and performance art), contacts in theaters throughout the D.C. metro area, practical experience and a portfolio of work. The internship is unpaid, however stipends are occasionally granted during the performance year (September - May). Interns are also invited to attend many WPAS performances on a complimentary basis.''
		\item Types of internships: \vspace{-0.1cm}
			\begin{enumerate} \itemsep -1pt
			\item Accounting Internship
			\item Development Internship
			\item Education Internship
			\item Marketing/Public Relations Internship
			\item Office Administration Internship
			\item Programming Internship
			\end{enumerate}
		\end{enumerate}
	\end{enumerate}
\item The Washington Ballet: Internships, \url{http://www.washingtonballet.org/about-twb/auditions-employment/#internships}
\item The Choral Arts Society of Washington: \vspace{-0.3cm}
	\begin{enumerate} \itemsep -2pt
	\item Internship and Apprenticeship Program: \url{http://www.choralarts.org/About-Us/Internships-and-Apprenticeships.aspx}
	\end{enumerate}
\item League of American Orchestras: Internships, \url{http://www.americanorchestras.org/career_center/internships.html}
%%%%%%%%%%%%%%%%%%%%%%%%%%%%%%%%%%%%%%
%%%%%%%%%%%%%%%%%%%%%%%%%%%%%%%%%%%%%%
%%%%%%%%%%%%%%%%%%%%%%%%%%%%%%%%%%%%%%
\item Congressional Hispanic Caucus Institute (CHCI): \vspace{-0.3cm}
	\begin{enumerate} \itemsep -2pt
	\item CHCI United Health Foundation Scholars: \vspace{-0.2cm}
		\begin{enumerate} \itemsep -2pt
		\item \url{http://www.chci.org/scholarships/page/chci-united-health-foundation-scholars-}
		\item In addition to providing scholarship opportunities for Latino youth, the United Health Foundation decided to partner with CHCI to create a six-month internship program for students interested in the medical field.
		\item Seventeen participants enrolled in either a full-time undergraduate or graduate course of study at an accredited two- or four-year college, university, vocational or technical school were selected.
		\end{enumerate}
	\item CHCI Congressional Internship: \vspace{-0.2cm}
		\begin{enumerate} \itemsep -2pt
		\item The purpose of the Congressional Internship Program (CIP) is to expose young Latinos to the legislative process and to strengthen their professional and leadership skills, ultimately promoting the presence of Latinos on Capitol Hill.
		\item The Congressional Internship Program provides college students with a paid Congressional work placement on Capitol Hill for a period of twelve weeks (Spring/Fall) or eight weeks (Summer). This unmatched experience allows students to learn first hand about our nation's legislative process.
		\end{enumerate}
	\end{enumerate}
\item Mexican American Legal Defense and Educational Fund (MALDEF): Law Clerk Summer Internship program, \url{http://maldef.org/about/jobs/index.html}
\item Hispanic Association of Colleges and Universities (HACU): \vspace{-0.3cm}
	\begin{enumerate} \itemsep -2pt
	\item HACU National Internship Program (HNIP): \url{http://www.hacu.net/hacu/HNIP_EN.asp}
	\end{enumerate}
%%%%%%%%%%%%%%%%%%%%%%%%%%%%
\item Smithsonian Institution: \vspace{-0.3cm}
	\begin{enumerate} \itemsep -2pt
	\item Smithsonian Institution Traveling Exhibition Service (SITES): \vspace{-0.2cm}
		\begin{enumerate} \itemsep -2pt
		\item Internship programs: \url{http://www.sites.si.edu/interns/internships.htm}
		\item ``The Smithsonian Institution Traveling Exhibition Service internship programs allows people with diverse interests, strengths, and goals to experience an educational environment where they can work and learn from professionals in the museum field.''
		\item ``SITES offers internship opportunities in a variety of different areas: public relations, development (fund raising), research, and project design.''
		\end{enumerate}
	\item Smithsonian Folkways Recordings (or simply, Smithsonian Folkways): \vspace{-0.2cm}
		\begin{enumerate} \itemsep -2pt
		\item Internships: \url{http://www.folkways.si.edu/about_us/jobs.aspx}
		\end{enumerate}
	\item Freer Gallery of Art / Arthur M. Sackler Gallery: \vspace{-0.2cm}
		\begin{enumerate} \itemsep -2pt
		\item Internships: \url{http://www.asia.si.edu/research/internships.asp}
		\end{enumerate}
	\item National Museum of American History: \vspace{-0.2cm}
		\begin{enumerate} \itemsep -2pt
		\item Jerome and Dorothy Lemelson Center for the Study of Invention and Innovation: \vspace{-0.1cm}
			\begin{enumerate} \itemsep -1pt
			\item Archival Internships: \url{http://invention.smithsonian.org/resources/research_interns.aspx}
			\end{enumerate}
		\end{enumerate}
	\end{enumerate}
%%%%%%%%%%%%%%%%%%%%%%%%%%%%
\item Council on International Educational Exchange (CIEE): \vspace{-0.3cm}
	\begin{enumerate} \itemsep -2pt
	\item CIEE's Trainee Program: \vspace{-0.2cm}
		\begin{enumerate} \itemsep -2pt
		\item part of the J-1 visa category of the US government�s Exchange Visitor Program
		\item \url{http://www.ciee.org/trainee/}
		\end{enumerate}
	\item CIEE Work \& Travel USA; and Internship USA: \vspace{-0.2cm}
		\begin{enumerate} \itemsep -2pt
		\item \url{http://www.ciee.org/hire/}
		\item \url{http://www.ciee.org/wat/}
		\end{enumerate}
	\end{enumerate}
\item American Institute For Foreign Study (AIFS): \vspace{-0.3cm}
	\begin{enumerate} \itemsep -2pt
	\item Camp America Counselors and Summer Staff: \url{http://www.aifs.com/work_travel.asp}
	\item Au Pair Placement: \url{http://www.aifs.com/au_pair.asp}
	\end{enumerate}
\item U.S. Department of State: \vspace{-0.3cm}
	\begin{enumerate} \itemsep -2pt
	\item Bureau of Educational and Cultural Affairs: \vspace{-0.2cm}
		\begin{enumerate} \itemsep -2pt
		\item International cultural programs: \url{http://exchanges.state.gov/cultural/related-cultural-programs.html}
		\item Office of Global Educational Programs: \vspace{-0.1cm}
			\begin{enumerate} \itemsep -1pt
			\item Camp Counselor: \vspace{-0.1cm}
				\begin{itemize} \itemsep -1pt
				\item \url{http://exchanges.state.gov/jexchanges/programs/camp.html}
				\item Camp counselors interact with groups of American youth by overseeing their camp activities during the U.S. summer.
				\item Through the Camp Counselor program, American campers have the chance to gain knowledge of foreign cultures, while foreign participants increase their knowledge of American culture.
				\item Participants must be at least 18 years of age and may work as counselors in U.S. summer camps for up to four months. Extensions are not allowed. They receive a combination a pay and benefits equal to Americans who work in the same position.
				\end{itemize}
			\end{enumerate}
		\item Private Sector Exchange office: \vspace{-0.1cm}
			\begin{enumerate} \itemsep -1pt
			\item \url{http://exchanges.state.gov/jexchanges/index.html}
			\item The Private Sector Exchange office designates, monitors and partners with U.S. organizations, including government agencies, academic institutions, educational and cultural organizations, and corporations, that administer the Exchange Visitor Program.
			\item Au Pair program: \vspace{-0.1cm}
				\begin{itemize} \itemsep -1pt
				\item Through the Au Pair program, foreign nationals between 18 and 26 years of age participate in the home life of a host family. Au pairs provide limited childcare services for up to 12 months. An extension of 6, 9, or 12 months may be granted in certain cases.
				\item \url{http://exchanges.state.gov/jexchanges/programs/aupair.html}
				\end{itemize}
			\item Internships: \vspace{-0.1cm}
				\begin{itemize} \itemsep -1pt
				\item \url{http://exchanges.state.gov/jexchanges/programs/intern.html}
				\item Internship programs are designed to allow foreign professionals to come to the United States to gain exposure to U.S. culture and to receive training in U.S. business practices in their chosen occupational field.
				\item The maximum duration of an internship in any occupational field is 12 months.
				\item Upon completion of their exchange programs, participants are expected to return to their home countries.
				\item The State Department allows internships in the following occupational categories: \vspace{-0.1cm}
					\begin{itemize} \itemsep -1pt
					\item Agriculture, Forestry, and Fishing
					\item Arts and Culture
					\item Construction and Building Trades
					\item Education, Social Sciences, Library Science, Counseling and Social Services
					\item Health Related Occupations
					\item Hospitality and Tourism
					\item Information Media and Communications
					\item Management, Business, Commerce and Finance
					\item Public Administration and Law
					\item The Sciences, Engineering, Architecture, Mathematics, and Industrial Occupations.
					\end{itemize}
				\item An Intern must be a foreign national: \vspace{-0.1cm}
					\begin{itemize} \itemsep -1pt
					\item Who is currently enrolled in and pursuing studies at a foreign degree- or certificate-granting post-secondary academic institution outside the United States, or
					\item Who has graduated from such an institution no more than 12 months prior to his or her exchange visitor program start date.
					\end{itemize}
				\item Interns cannot work in unskilled or casual labor positions, in positions that require or involve child care or elder care, or in any kind of position that involves medical patient care or contact. Nor can interns work in positions that require more than 20 per cent clerical or office support work.
				\end{itemize}
			\item The Summer Work Travel Program: \vspace{-0.1cm}
				\begin{itemize} \itemsep -1pt
				\item \url{http://exchanges.state.gov/jexchanges/programs/swt.html}
				\item In the summer work travel program, post-secondary students may enter the United States to work and travel during their summer vacation.
				\item Participants can be admitted to the program more than once.
				\item The maximum length of the program is four months.
				\item Most of the time, participants work in unskilled service positions at resorts, hotels, restaurants, and amusement parks. However, they may also work in other types of organizations.
				\item For example, they could work in architectural firms, scientific research organizations, graphic art/publishing and other media communication businesses, advertising agencies, computer software and electronics firms, legal offices, etc.
				\item The program may not exceed four-months and must be finished during the student's summer vacation.
				\item Participants receive pay and benefits equal to an American working in the same or similar position.
				\end{itemize}
			\item Training programs: \vspace{-0.1cm}
				\begin{itemize} \itemsep -1pt
				\item \url{http://exchanges.state.gov/jexchanges/programs/trainee.html}
				\item Training programs are designed to allow foreign professionals to come to the United States to gain exposure to U.S. culture and to receive training in U.S. business practices in their chosen occupational field.
				\item Foreign nationals have had the opportunity to train with some of the finest employers in the U.S., gaining real time experience in their chosen career fields.
				\item Upon completion of their exchange programs, participants are expected to return to their home countries to utilize their newly learned skills and knowledge to advance their careers and share their experiences with their communities.
				\item The State Department allows training programs in the following occupational categories: \vspace{-0.1cm}
					\begin{itemize} \itemsep -1pt
					\item Agriculture, Forestry, and Fishing
					\item Arts and Culture
					\item Construction and Building Trades
					\item Education, Social Sciences, Library Science, Counseling and Social Services
					\item Health Related Occupations
					\item Hospitality and Tourism
					\item Information Media and Communications
					\item Management, Business, Commerce and Finance
					\item Public Administration and Law
					\item The Sciences, Engineering, Architecture, Mathematics, and Industrial Occupations.
					\end{itemize}
				\item A trainee must be a foreign national who has: \vspace{-0.1cm}
					\begin{itemize} \itemsep -1pt
					\item A degree or professional certificate from a foreign post-secondary academic institution and at least one year of prior related work experience in his or her occupational field outside the United States, or
					\item Five years of work experience outside the United States in the occupational field in which they are seeking training.
					\end{itemize}
				\end{itemize}
			\item Specialists: \vspace{-0.1cm}
				\begin{itemize} \itemsep -1pt
				\item \url{http://exchanges.state.gov/jexchanges/programs/specialist.html}
				\item This category is for a participant who is an expert in a field of specialized knowledge or skill who will demonstrate such skills in the United States. Such exchanges are to provide opportunities to increase the exchange knowledge and ideas between American and foreign specialists. The maximum duration of this program is one year.
				\item This category is for foreign nationals who are experts in a field of specialized knowledge or skill, coming to the United States for observing, consulting, or demonstrating their special skills, except: Professors and Research Scholars, Short-Term Scholars, and Alien Physicians.
				\item Individuals participating in the specialist program are: \vspace{-0.1cm}
					\begin{itemize} \itemsep -1pt
					\item Experts in a field of specialized knowledge or skill;
					\item Seeks to travel to the United States for the purpose of observing, consulting, or demonstrating their special knowledge or skills;
					\item Does not fill a permanent or long-term position of employment while in the U.S.
					\end{itemize}
				\end{itemize}
			\item International Visitor: \vspace{-0.1cm}
				\begin{itemize} \itemsep -1pt
				\item \url{http://exchanges.state.gov/jexchanges/programs/intl_visitor.html}
				\item The international visitor category enables visitors to better understand American culture and enhanced American knowledge of foreign cultures.
				\item This category is for individuals who are recognized as potential leaders in their own country, selected by the Department of State to participate in observation tours, discussions, consultation, professional meetings, conferences, workshops and travel.
				\item The maximum duration of the program is one year.
				\end{itemize}
			\item Alien Physician: \vspace{-0.1cm}
				\begin{itemize} \itemsep -1pt
				\item \url{http://exchanges.state.gov/jexchanges/programs/physician.html}
				\item The Alien Physician program is for foreign national physicians seeking entry into U.S. graduate medical education programs or training at accredited U.S. schools of medicine or other U.S. institutions.
				\item There are generally two types of exchange programs in which a foreign national physician (also referred to as a foreign/international medical graduate) participates: \vspace{-0.1cm}
					\begin{itemize} \itemsep -1pt
					\item Clinical training in the �alien physician� category
					\item Non-Clinical training in the �research scholar� category
					\end{itemize}
				\end{itemize}
			\item FORTUNE/U.S. State Department Global Women's Mentoring Partnership: \vspace{-0.1cm}
				\begin{itemize} \itemsep -1pt
				\item \url{http://exchanges.state.gov/citizens/professionals/fortunepartnership.html}
				\item This public-private partnership places talented, emerging women leaders from all over the world in mentoring programs with FORTUNE's Most Powerful Women Leaders.
				\item For three weeks, American and international participants work together in mentoring relationships to share the skills and experiences necessary for strengthening women�s leadership.
				\end{itemize}
			\item American Council of Young Political Leaders (ACYPL): \vspace{-0.1cm}
				\begin{itemize} \itemsep -1pt
				\item \url{http://exchanges.state.gov/citizens/profs/acypl.html}
				\item \url{http://www.acypl.org/}
				\item For 44 years, the American Council of Young Political Leaders (ACYPL) has designed, organized and managed unique international educational exchanges for young political leaders (ages 25-40) worldwide.
				\item ACYPL programs are designed to promote mutual understanding, respect, and friendship and to cultivate long-lasting relationships among young people who are poised to become tomorrow's global leaders and policy makers.
				\item American participants are nominated by members of Congress, governors, political party leaders, and ACYPL alumni, while international delegates are selected from countries where ACYPL is currently conducting programs by international program partners with the U.S. Embassy input.
				\end{itemize}
			\item Edward R. Murrow Program for Journalists: \vspace{-0.1cm}
				\begin{itemize} \itemsep -1pt
				\item \url{http://exchanges.state.gov/ivlp/murrow.html}
				\item The Edward R. Murrow Program for Journalists invites rising international journalists to travel to the United States and examine journalistic principles and practices.
				\end{itemize}
			\end{enumerate}
		\item Office of Citizen Exchanges: \vspace{-0.1cm}
			\begin{enumerate} \itemsep -1pt
			\item Youth Programs Division: \vspace{-0.1cm}
				\begin{itemize} \itemsep -1pt
				\item \url{http://exchanges.state.gov/youth/index.html}
				\item The Youth Programs Division is committed to empowering the successor generation and establishing long-lasting ties between the United States and other countries through exchange programs and institutional partnerships.
				\item Programs focus primarily on secondary schools and promote mutual understanding, leadership development, educational transformation, and democratic ideals.
				\item Year-Long Programs, Short Term Programs, and Virtual Partnerships: \url{http://exchanges.state.gov/youth/programs-by-type.html}
				\item Programs for Young Americans, and Programs for International Students and Teachers: \url{http://exchanges.state.gov/youth/programs-by-participants.html}
				\item Opportunities for American Hosts: Families and Schools, \url{http://exchanges.state.gov/youth/opps-for-am-hosts.html}
				\item Programs for High School Students: \url{http://exchanges.state.gov/youth/programs.html}
				\end{itemize}
			\item Professional Exchanges Division: \vspace{-0.1cm}
				\begin{itemize} \itemsep -1pt
				\item \url{http://exchanges.state.gov/citizens/profs.html}
				\item The Professional Exchanges division provides grants to U.S. nonprofit organizations to carry out exchange programs that support the professional development of foreign participants. The purpose of each exchange program is to engage with foreign leaders in critical professions, to demonstrate respect for foreign cultures, and to promote mutual understanding between the people of the United States and other countries.
				\item Professional exchanges typically last several years and include internships, study tours or workshops in the United States and in the host country. Participants come from a variety of professions including education administrators, public servants, journalists, labor union officials, entrepreneurs, environmental leaders, jurists, lawyers, and civic leaders.
				\item ECA grant opportunities: \vspace{-0.1cm}
					\begin{itemize} \itemsep -1pt
					\item Open Funding Opportunities: Requests For Grant Proposals (RFGPs), \url{http://exchanges.state.gov/grants/open2.html}
					\item Grants.gov: \url{http://www.grants.gov/}
					\end{itemize}
				\item Grants by Region: \vspace{-0.1cm}
					\begin{itemize} \itemsep -1pt
					\item \url{http://exchanges.state.gov/citizens/professionals/grant-region.html}
					\item Africa 
					\item East Asia and the Pacific 
					\item Europe and Eurasia 
					\item North Africa and the Middle East 
					\item South and Central Asia 
					\item Western Hemisphere 
					\item Multi-regional
					\end{itemize}
				\end{itemize}
			\end{enumerate}
		\end{enumerate}
	\end{enumerate}
\end{enumerate}






%%%%%%%%%%%%%%%%%%%%%%%%%%%%%%%%%%%%%%%%%%%
\section{Resources on Studying Abroad}
\label{resourcesonstudyingabroad}

Resources on studying abroad: \vspace{-0.3cm}
\begin{enumerate} \itemsep -4pt
\item Council on International Educational Exchange (CIEE): \vspace{-0.3cm}
	\begin{enumerate} \itemsep -2pt
	\item Study abroad programs for high school students from the United States: \vspace{-0.2cm}
		\begin{enumerate} \itemsep -2pt
		\item \url{http://www.ciee.org/hsabroad/index.html}
		\item \url{http://www.ciee.org/hsabroad/high-school-study-abroad/index.html}
		\item These programs include:: \vspace{-0.1cm}
			\begin{enumerate} \itemsep -1pt
			\item High School Abroad programs (for U.S. high school students)
			\item Summer High School Abroad programs (for U.S. high school students)
			\item Gap Year Abroad programs (for recent U.S. high school graduates)
			\end{enumerate}
		\end{enumerate}
	\end{enumerate}
\item U.S. Department of State: \vspace{-0.3cm}
	\begin{enumerate} \itemsep -2pt
	\item Bureau of Educational and Cultural Affairs: \vspace{-0.2cm}
		\begin{enumerate} \itemsep -2pt
		\item Office of Global Educational Programs: \vspace{-0.1cm}
			\begin{enumerate} \itemsep -1pt
			\item EducationUSA: \vspace{-0.1cm}
				\begin{itemize} \itemsep -1pt
				\item EducationUSA is a network of more than 400 student advising centers, which offer accurate, comprehensive, objective and timely information about educational opportunities in the United States and guidance to qualified individuals on how best to access those opportunities. This includes information about application procedures, standardized test requirements, student visas, financial aid, and the full range of accredited U.S. higher education institutions.
				\item \url{http://exchanges.state.gov/globalexchanges/index/educationusa.html}
				\item \url{http://www.educationusa.state.gov/} and \url{http://www.educationusa.info/centers.php}
				\end{itemize}
			\item Open Doors: \vspace{-0.1cm}
				\begin{itemize} \itemsep -1pt
				\item The Educational Information and Resources Branch funds Open Doors, a census of foreign students and scholars in the U.S. and of U.S. students studying abroad published annually by the Institute for International Education.
				\item Open Doors data is used by U.S. embassies, the Departments of State, Commerce, and Education, and U.S. colleges and universities to inform policy decisions about educational exchanges, trade in educational services, and study abroad activity.
				\item \url{http://exchanges.state.gov/globalexchanges/index/open_doors.html}
				\item \url{http://www.opendoors.iienetwork.org/}
				\end{itemize}
			\end{enumerate}
		\item EducationUSA: \vspace{-0.1cm}
			\begin{enumerate} \itemsep -1pt
			\item \url{http://educationusa.state.gov/}
			\item For U.S. (college) students who want to study/work abroad: \url{http://www.educationusa.info/pages/students/forus.php}
			\end{enumerate}
		\end{enumerate}
	\end{enumerate}
\item IES Abroad (formerly Institute of European Studies / Institute for the International Education of Students): \vspace{-0.3cm}
	\begin{enumerate} \itemsep -2pt
	\item \url{https://www.iesabroad.org/} and \url{https://www.iesabroad.org/IES/home.html}
	\end{enumerate}
\item Global Learning Semesters, Inc.: \vspace{-0.3cm}
	\begin{enumerate} \itemsep -2pt
	\item Summer in the Mediterranean: \vspace{-0.2cm}
		\begin{enumerate} \itemsep -2pt
		\item \url{http://www.globalsemesters.com/Mediterranean.html}
		\item Has programs in the following areas: \vspace{-0.1cm}
			\begin{enumerate} \itemsep -1pt
			\item Art \& Photography
			\item Early Christianity
			\item Greek Heritage
			\item International Marketing
			\item Music
			\end{enumerate}
		\end{enumerate}
	\end{enumerate}
\item American Institute For Foreign Study (AIFS): \vspace{-0.3cm}
	\begin{enumerate} \itemsep -2pt
	\item \url{http://www.aifs.com/}
	\item College Study Abroad: \url{http://www.aifsabroad.com/}
	\item For high school students: \vspace{-0.2cm}
		\begin{enumerate} \itemsep -2pt
		\item Gifted Education: \url{http://www.aifs.com/gifted_education.asp}
		\item High School Study and Travel: \url{http://www.aifs.com/highschool_study_travel.asp}
		\item Academic Year in America (AYA): \url{http://www.academicyear.org/?source=AIFS}
		\end{enumerate}
	\end{enumerate}
\end{enumerate}





%%%%%%%%%%%%%%%%%%%%%%%%%%%%%%%%%%%%%%%%%%%
\section{College Preparation}
\label{collegepreparation}

College preparation: \vspace{-0.3cm}
\begin{enumerate} \itemsep -4pt
\item {\it Guide to Online Schools} [or {\it GuideToOnlineSchools.com}], {\it The Top 53 College Preparation Resources for Students}. Available at: \url{http://www.guidetoonlineschools.com/tips-and-tools/college-prep-resources}; last accessed on August 25, 2010.
\item U.S. Department of Education's resources for parents to help their children learn: \url{http://www2.ed.gov/parents/academic/help/hyc.html} and \url{http://www2.ed.gov/parents/academic/help/homework/index.html}
\item The College Board: \vspace{-0.3cm}
	\begin{enumerate} \itemsep -2pt
	\item Information about SATs, college preparation, and financial aid
	\item {\it Trends in Higher Education} series 201X: \url{http://trends.collegeboard.org/}
	\item \url{http://www.collegeboard.com/}
	\end{enumerate}
\item {\it Accreditation.org}: \vspace{-0.3cm}
	\begin{enumerate} \itemsep -2pt
	\item Information about the accreditation of engineering degree programs around the world
	\item \url{http://www.accreditation.org/}
	\end{enumerate}
\item {\it New York Times}: \vspace{-0.3cm}
	\begin{enumerate} \itemsep -2pt
	\item The Learning Network: \url{http://learning.blogs.nytimes.com/category/test-yourself/}
	\item New York Times Magazine: \vspace{-0.2cm}
		\begin{enumerate} \itemsep -2pt
		\item The Sep 20, 2010 issue has many articles covering how technology can be used to improve education in K-12 programs. Available online at: \url{http://www.nytimes.com/indexes/2010/09/19/magazine/index.html?ref=magazine}; last accessed on September 20, 2010.
		\item ``New York Times Magazine Features Technology in Education,'' in {\it CCC Blog}, Computing Community Consortium (CCC), Computing Research Association (CRA), Sep 20, 2010. Available online at: \url{http://www.cccblog.org/2010/09/20/new-york-times-magazine-features-technology-in-education/}; last accessed on September 20, 2010.
		\item Articles in this issue discuss: \vspace{-0.1cm}
			\begin{enumerate} \itemsep -1pt
			\item How journalists can make use of technology to automate certain tasks, and improve their productivity and effectiveness in covering news stories
			\item How children can create computer games that introduces them to careers in computing and helps them to develop skills in computational thinking
			\item How to learn things without a lot of rote learning, to have fun while learning, and to use technology to make learning more fun
			\end{enumerate}
		\end{enumerate}
	\end{enumerate}
\item University of Southern California, USC: \vspace{-0.3cm}
	\begin{enumerate} \itemsep -2pt
	\item USC Office of Continuing Education and Summer Programs: \vspace{-0.2cm}
		\begin{enumerate} \itemsep -2pt
		\item \url{http://cesp.usc.edu/}
		\item These programs allow students in K-12 to earn credit at USC, and exposes them to different majors/professions, like medicine, engineering, creative writing, or film making.
		\item Students can benefit from these programs, and learn about different academic disciplines before applying to college. This would help them in their college applications.
		\item Underrepresented minority students can get scholarships to attend these programs. So, if parents have financial difficulty paying for the programs, they can seek financial aid for this.
		\item Also, current undergraduates can also sign up for programs to learn about marketing, finance, and entrepreneurship. They can also do summer research with USC researchers.
		\end{enumerate}
	\item Summer sports programs for youths: \vspace{-0.2cm}
		\begin{enumerate} \itemsep -2pt
		\item SC Futbol Academy (USC Soccer Camps): \url{http://www.usctrojans.com/sports/w-soccer/spec-rel/021610aaa.html}
		\item Mick Haley's USC Girls Volleyball Camp: \url{http://www.usctrojans.com/sports/w-volley/spec-rel/volley-camp.html}
		\item Salo Swim Camp: \url{http://www.saloswimcamp.com/on-line/default.asp}
		\item USC NYSP Trojan KidSCamp: \url{http://sait.usc.edu/recsports/site_content/youth_sports/nysptk.html}
		\item After School Sports Connection, ASSC (operates in fall, spring, and summer): \url{http://sait.usc.edu/recsports/site_content/youth_sports/assc.html}
		\end{enumerate}
	\end{enumerate}
\item Telluride Association: \vspace{-0.3cm}
	\begin{enumerate} \itemsep -2pt
	\item Telluride Association Summer Program (TASP) [ for high school students ]: \url{http://www.tellurideassociation.org/programs/high_school_students/tasp/tasp_general_info.html}
	\item Telluride Association Sophomore Seminar (TASS) [ for high school students ]: \url{http://www.tellurideassociation.org/programs/high_school_students/tass/tass_general_info.html}
	\item Resources for high school educators to nominate summer program applicants: \url{http://www.tellurideassociation.org/programs/high_school_students/hs_resources/hs_resources_general_information.html}
	\end{enumerate}
\item MathNerds: \vspace{-0.3cm}
	\begin{enumerate} \itemsep -2pt
	\item \url{http://www.mathnerds.com/}
	\item ``Provides free, discovery-based, mathematical guidance via an international, volunteer network of mathematicians.''
	\item If you have a mathematical problem to solve, you can ask mathematicans at {\it MathNerds} for help.
	\item They would require you to discuss your attempted approaches/solutions.
	\item If you have not made attempts to solve the problem, they will not give you much guidance.
	\item In addition, they cannot solve problems for you.
	\item They provide guidance for mathematical problems from K-12 material through undergraduate mathematics and statistics classes.
	\item They also provide help for selected topics in advanced mathematics classes (for graduate students).
	\item Other resources: \url{http://www.mathnerds.com/links/links.aspx}
	\end{enumerate}
\item Hobsons: \vspace{-0.3cm}
	\begin{enumerate} \itemsep -2pt
	\item CollegeView (Hobsons' college recruiting services): \url{http://www.collegeview.com/index.jsp}
	\end{enumerate}
\item Sponsors for Educational Opportunity (SEO): \vspace{-0.3cm}
	\begin{enumerate} \itemsep -2pt
	\item Resources: \url{http://www.seo-usa.org/ScholarsResources}
	\end{enumerate}
\item U.S. Department of Education: \vspace{-0.3cm}
	\begin{enumerate} \itemsep -2pt
	\item Students.gov: \url{http://www.students.gov/STUGOVWebApp/index.jsp}
	\item college.gov: \url{http://www.college.gov/wps/portal}
	\end{enumerate}
\item U.S. Department of State: \vspace{-0.3cm}
	\begin{enumerate} \itemsep -2pt
	\item Bureau of Educational and Cultural Affairs: \vspace{-0.2cm}
		\begin{enumerate} \itemsep -2pt
		\item EducationUSA: \vspace{-0.1cm}
			\begin{enumerate} \itemsep -1pt
			\item Information for international students: \url{http://www.educationusa.info/students.php}
			\end{enumerate}
		\end{enumerate}
	\end{enumerate}
\item Congressional Hispanic Caucus Institute (CHCI): \vspace{-0.3cm}
	\begin{enumerate} \itemsep -2pt
	\item CHCI Education Center: \vspace{-0.2cm}
		\begin{enumerate} \itemsep -2pt
		\item \url{http://www.chci.org/education_center/}
		\item Has resources on college planning, financial aid, scholarships, college internships, and housing.
		\item For Parents: \url{http://www.chci.org/education_center/page/for-parents}
		\item For Students: \url{http://www.chci.org/education_center/page/for-students}
		\end{enumerate}
	\end{enumerate}
\item My College Options: \vspace{-0.3cm}
	\begin{enumerate} \itemsep -2pt
	\item \url{http://www.mycollegeoptions.org/}
	\item ``My College Options is a FREE college planning service, offering assistance to students, parents, high schools, counselors, and teachers nationwide.''
	\item ``It is designed to assist high school students in exploring a wide range of post-secondary opportunities, with special emphasis on the college search process.''
	\end{enumerate}
\end{enumerate}

Resources for financial aid: \vspace{-0.3cm}
\begin{enumerate} \itemsep -4pt
\item {\it Guide to Online Schools} [or {\it GuideToOnlineSchools.com}], {\it Financial Aid}. Available at: \url{http://www.guidetoonlineschools.com/financial-aid}; last accessed on August 25, 2010.
\item The Institute for College Access \& Success, {\it Links} [ Resources that provide information about student loans and student debt ]. Available at: \url{http://projectonstudentdebt.org/links.vp.html}; last accessed on September 4, 2010. [ Also, see \url{http://projectonstudentdebt.org/advice.vp.html} and \url{http://ticas.org/about.vp.html}. ]
\end{enumerate}


Information about colleges and universities: \vspace{-0.3cm}
\begin{enumerate} \itemsep -4pt
\item The Institute for College Access \& Success, {\it College InSight}. Available at: \url{http://college-insight.org/}; last accessed on September 4, 2010.
\item 
\end{enumerate}



%%%%%%%%%%%%%%%%%%%%%%%%%%%%%%%%%%%%%%%%%%%
\section{Outreach for Students in Colleges and Universities}
\label{outreachcollege}

Resources to reach out to students in colleges and universities: \vspace{-0.3cm}
\begin{enumerate} \itemsep -4pt
%%%%%%%%%%%%%%%%%%%%%%%%%%%%%
\item Film contests: \vspace{-0.3cm}
	\begin{enumerate} \itemsep -2pt
	\item Ed Wood Film Festival [@ USC]: \vspace{-0.2cm}
		\begin{enumerate} \itemsep -2pt
		\item Celebrating independent filmmaking at USC and named for the famous director, the Ed Wood Film Festival is put on by a committee of Residential Education staff members at New Residential College, chaired by the Cinema Floor RA's.
		\item Teams of students come together to obtain the year's secret theme in which to write, shoot, and edit their very own short film within 24 hours. A week later, the films are shown at USC's Norris Cinema and a panel of judges selects the Festival winners in a variety of categories.
		\item \url{http://sait.usc.edu/resed/Programs.aspx}
		\end{enumerate}
	\item Reel LA: Parkside International Film Festival [or USC Reel LA Film Festival at USC]; see \url{http://www-scf.usc.edu/~pirc/areagov/government.php}
	\end{enumerate}
%%%%%%%%%%%%%%%%%%%%%%%%%%%%%
\item residential education: \vspace{-0.3cm}
	\begin{enumerate} \itemsep -2pt
	\item Telluride Association: \vspace{-0.2cm}
		\begin{enumerate} \itemsep -2pt
		\item Information about how to reside at the Cornell Branch (also known as Telluride House or CBTA) and the Michigan Branch of Telluride Association, which are ``residential colleges'': \url{http://www.tellurideassociation.org/programs/university_students.html}
		\item Awards for residents at the Cornell or Michigan Branch: \url{http://www.tellurideassociation.org/programs/university_students/us_awards.html}
		\end{enumerate}
	\end{enumerate}
%%%%%%%%%%%%%%%%%%%%%%%%%%%%%
\item MathNerds: \vspace{-0.3cm}
	\begin{enumerate} \itemsep -2pt
	\item \url{http://www.mathnerds.com/}
	\item ``Provides free, discovery-based, mathematical guidance via an international, volunteer network of mathematicians.''
	\item If you have a mathematical problem to solve, you can ask mathematicans at {\it MathNerds} for help.
	\item They would require you to discuss your attempted approaches/solutions.
	\item If you have not made attempts to solve the problem, they will not give you much guidance.
	\item In addition, they cannot solve problems for you.
	\item They provide guidance for mathematical problems from K-12 material through undergraduate mathematics and statistics classes.
	\item They also provide help for selected topics in advanced mathematics classes (for graduate students).
	\end{enumerate}
%%%%%%%%%%%%%%%%%%%%%%%%%%%%%
\item Invent Now: \vspace{-0.3cm}
	\begin{enumerate} \itemsep -2pt
	\item 
	\end{enumerate}
\item Journal of Young Investigators (JYI): \vspace{-0.3cm}
	\begin{enumerate} \itemsep -2pt
	\item \url{http://www.jyi.org/}
	\item ``peer-reviewed journal for undergraduates''
	\item ``JYI's web journal (which is also called JYI) is dedicated to the presentation of undergraduate research in science, mathematics, and engineering. It publishes the best submissions from undergraduates, with an emphasis on both the quality of research and the manner in which it is communicated. The journal, JYI, also allows students to experience the other side of the scientific publication process: the review process. Students working with their faculty advisors review the work of their peers and determine whether that work is acceptable for publication in JYI.''
	\end{enumerate}
\item The Recording Academy: \vspace{-0.3cm}
	\begin{enumerate} \itemsep -2pt
	\item GRAMMY U: \vspace{-0.2cm}
		\begin{enumerate} \itemsep -2pt
		\item \url{http://www.grammy365.com/grammy-u}
		\item GRAMMY U is a unique and fast-growing community of full-time college students, primarily between the ages of 17 and 25,  who are pursuing a career in the recording industry.
		\item The Recording Academy created GRAMMY U to help prepare college students for their careers in the music industry through networking, educational programs and performance opportunities.
		\item GRAMMY U is designed to enhance students' current academic curriculum with access to recording industry professionals to give an ``out of classroom'' perspective on the recording industry.
		\end{enumerate}
	\end{enumerate}
%%%%%%%%%%%%%%%%%%%%%%%%%%%%%
\item --- --- --- --- --- --- --- --- --- --- --- --- --- --- --- --- --- --- --- --- --- --- --- --- --- --- --- --- --- --- ---
\item \colorbox{blue}{\bf Help for Underrepresented Minorities}
% Help for Underrepresented Minorities
\item INROADS, Inc.: \vspace{-0.3cm}
	\begin{enumerate} \itemsep -2pt
	\item Internships: \url{http://www.inroads.org/interns/internWhatItTakes.jsp}
	\end{enumerate}
\item The PhD Project: \vspace{-0.3cm}
	\begin{enumerate} \itemsep -2pt
	\item \url{http://www.phdproject.org/index.html}
	\item Program and informational network to encourage ``African-Americans, Hispanic-Americans and Native Americans'' to pursue Ph.D. programs in business and seek careers in academia.
	\item Annual PhD Project Conference: \vspace{-0.2cm}
		\begin{enumerate} \itemsep -2pt
		\item Conference: \vspace{-0.1cm}
			\begin{enumerate} \itemsep -1pt
			\item \url{http://www.phdproject.org/conference.html}
			\item \url{http://www.phdproject.org/conference_application.html}
			\item For prospective Ph.D. students in business to learn more about Ph.D. programs in business, the Ph.D. application process, and life in graduate school.
			\item Registration Policy: \vspace{-0.1cm}
				\begin{itemize} \itemsep -1pt
				\item If you are selected to attend the conference you will be required to pay a \$200 registration fee which can be processed via credit card during the registration process. All travel and conferences expenses will paid by The PhD Project (total conference expenses for hotel, meals, materials, and transportation are valued at approximately \$1,500 per invited attendee.) Your investment of the \$200 registration fee will be refunded if you enter a full-time, AACSB accredited business doctoral program within 3 years of attending the conference. 
				\item If you previously attended a PhD Project Conference, you may submit an application to be reviewed, however if you are selected to attend, The PhD Project will only cover hotel costs (shared room with another participant). You will be required to pay the registration and travel costs
				\end{itemize}
			\end{enumerate}
		\item Resources for Potential/Current Doctoral Students: \vspace{-0.1cm}
			\begin{enumerate} \itemsep -1pt
			\item \url{http://www.phdproject.org/resources.html}
			\item Information about good business schools that offer Ph.D. programs, preparation for the GMAT, and the life in graduate school as a Ph.D. student.
			\item Suggested Reading: \vspace{-0.1cm}
				\begin{itemize} \itemsep -1pt
				\item \url{http://www.phdproject.org/reading.html}
				\item Has information life in graduate school as a Ph.D. student, racial diversity/issues in higher education, job searching in academia, and work-life balance for female Ph.D. students.
				\end{itemize}
			\end{enumerate}
		\item The PhD Project Doctoral Student Association (DSA): \vspace{-0.1cm}
			\begin{enumerate} \itemsep -1pt
			\item The PhD Project network: \vspace{-0.1cm}
				\begin{itemize} \itemsep -1pt
				\item \url{http://www.myphdnetwork.org/}
				\item ``There are 5 discipline specific associations covering the major areas of business education: Accounting, Finance, Information Systems, Management, Marketing.''
				\end{itemize}
			\end{enumerate}
		\end{enumerate}
	\end{enumerate}
\item MS-to-Ph.D. program for underrepresented minorities at Fisk and Vanderbilt in certain areas of
science (including astronomy, material science, and physics)
\item Outreach programs for underrepresented minorities to help them get into medical (and/or graduate) schools. Search for ``PREP (Post-baccalaureate Research Education Programs),'' which have stipends. E.g., Georgetown University School of Medicine, and George Washington University's medical school
\item New York University: \vspace{-0.3cm}
	\begin{enumerate} \itemsep -2pt
	\item Leonard N. Stern School of Business: \vspace{-0.2cm}
		\begin{enumerate} \itemsep -2pt
		\item Stern Pre-Doctoral program: \url{http://www.stern.nyu.edu/AcademicPrograms/PhD/Pre-Doctoral/index.htm}
		\end{enumerate}
	\end{enumerate}
\end{enumerate}



%%%%%%%%%%%%%%%%%%%%%%%%%%%%%%%%%%%%%%%%%%%
\section{Science \& Engineering Outreach}
\label{stemoutreach}

%%%%%%%%%%%%%%%%%%%%%%%%%%%%%%%%%%%%%%%%%%%
\subsection{Precollege Science \& Engineering Outreach}
\label{stemoutreachk12}

Science and engineering outreach to high-school (and middle-school) students, and their parents, teachers, and career counselors: \vspace{-0.3cm}
\begin{enumerate} \itemsep -4pt
\item {\it MentorNet}: \vspace{-0.3cm}
	\begin{enumerate} \itemsep -2pt
	\item \url{http://www.mentornet.net/}
	\item Enables people to network with scientists, engineers, and professors in Science, Technology, Engineering, and Mathematics (STEM)
	\item Is very supportive of minorities, so that more minorities (particularly underrepresented minorities) can be attracted to STEM careers
	\end{enumerate}
\item International Science Olympiad (for high school students): \vspace{-0.3cm}
	\begin{enumerate} \itemsep -2pt
	\item International Olympiad in Informatics: \url{http://en.wikipedia.org/wiki/International_Olympiad_in_Informatics} and \url{http://www.ioinformatics.org/index.shtml}
	\item International Mathematical Olympiad: \url{http://www.imo-official.org/}
	\item International Physics Olympiad: \url{http://www.jyu.fi/tdk/kastdk/olympiads/}
	\item International Chemistry Olympiad: \url{http://www.icho.sk/}
	\item International Biology Olympiad: \url{http://www.ibo-info.org/}
	\item \url{http://scienceolympiads.org/}
	\end{enumerate}
\item International Astronomy Olympiad: \url{http://www.issp.ac.ru/iao/}
\item International Earth Science Olympiad: \url{http://en.wikipedia.org/wiki/International_Earth_Science_Olympiad}
\item International Junior Science Olympiad (for students younger than 15 years old): \url{http://www.ijso-official.org/home}
\item Teen Leadership Institute Science, Technology, Engineering, and Math (STEM) programs @ YWCA Greater Pittsburgh; see \url{http://www.ywcapgh.org/STEM_Programs.asp}
\item For Inspiration and Recognition of Science and Technology (FIRST): \url{http://www.usfirst.org/} (including resources and guides to mentoring); scholarships @ \url{http://www.usfirst.org/aboutus/content.aspx?id=508}; and robotics programs @ \url{http://www.usfirst.org/roboticsprograms/frc/default.aspx?=966}
\item Mac Hyman, ``Good Choices for Great Careers in the Mathematical Sciences,'' talk given at 2008 SIAM Annual Meeting. Available at: \url{http://client.blueskybroadcast.com/siam08/hyman/index.html}; last accessed on August 25, 2010. Also, see \url{http://meetings.siam.org/program.cfm?CONFCODE=AN08}, \url{http://www.siam.org/meetings/an08/program.php}, and \url{http://www.siam.org/meetings/an08/}.
\item {\it RoboCup}\texttrademark\ competitions: \vspace{-0.2cm}
	\begin{enumerate} \itemsep -2pt
	\item Junior category for K-12 students involves contests the these areas of challenges: \vspace{-0.1cm}
		\begin{enumerate} \itemsep -1pt
		\item soccer
		\item dance
		\item rescue operations
		\end{enumerate}
	\item \url{http://www.robocup.org/}
	\end{enumerate}
\item {\it Curriki}, which is an online educational resource for teachers, students, and parents in K-12: \url{http://www.curriki.org/xwiki/bin/view/Main/About}
%%%%%%%%%%%%%%%%%%%%%%%%%%%%%%%%%%%%%%%%
%%%%%%%%%%%%%%%%%%%%%%%%%%%%%%%%%%%%%%%%
\item Electrical and computer engineering and/or computer science: \vspace{-0.2cm}
	\begin{enumerate} \itemsep -2pt
	\item {\it TopCoder} coding and design contests: \vspace{-0.2cm}
		\begin{enumerate} \itemsep -2pt
		\item High School category
		\item \url{http://www.topcoder.com/}
		\end{enumerate}
	\item Student Cluster Competition (SCC): \vspace{-0.2cm}
		\begin{enumerate} \itemsep -2pt
		\item SCC is held at each (annual) SC conference, which is the International Conference for High Performance Computing, Networking, Storage, and Analysis. IEEE Computer Society and the Association for Computing Machinery are the sponsors for this conference.
		\item During SC10, teams consisting of six students, undergraduate and/or high school, will showcase the amazing power of clusters and the ability to utilize open source software to solve interesting and important problems. They will compete in real-time on the exhibit floor to run a workload of real-world applications on clusters of their own design while never exceeding the dictated power limit.
		\item During SC10 in New Orleans, teams will assemble, test and tune their machines and run the HPCC benchmarks until the starting bell rings on Monday night at the Exhibit Opening Gala where they will be given the competition data sets. In full view of conference attendees, teams will execute the prescribed workload while showing progress and science visualization output on large high-resolution displays in their areas. Teams race to correctly complete the greatest number of application runs during the competition period until the close of the exhibit floor on Wednesday evening.
		\item \url{http://sc10.supercomputing.org/?pg=studentcluster.html}
		\end{enumerate}
	\item Institute of Electrical and Electronics Engineers, IEEE: \vspace{-0.3cm}
		\begin{enumerate} \itemsep -2pt
		\item {\it IEEE Educational Activities} recommended resources: \url{http://www.ieee.org/education_careers/education/preuniversity/resources/index.html}
		\item Engineering Projects in Community Service (EPICS) in IEEE: \vspace{-0.2cm}
			\begin{enumerate} \itemsep -2pt
			\item High school students collaborate with college students in engineering projects to benefit the community
			\item \url{http://www.ieee.org/education_careers/education/preuniversity/epics_high.html}
			\end{enumerate}
		\item Talk given by John Cohn at the IEEE International Symposium on Circuits and Systems (ISCAS), May 18-21, 2008. The talk is titled, ``Kids these days. How we can inspire the next generation of Engineers and Scientists?'' See \url{http://ewh.ieee.org/soc/icss/IEEE-ISCAS-08-Tue-Keynote-JC/IEEE-ISCAS-08-Tue-Keynote-JC.HTML}. [ Alternatively, go to: IEEE Circuits and Systems Society, \url{http://www.ieee-cas.org/}: Select the ``Resources'' tab in the menu bar, and select the ``ISCAS Keynote Videos'' option. Click on the video link with the appropriate title. ]
		\end{enumerate}
	\item Association for Computing Machinery (ACM): \vspace{-0.2cm}
		\begin{enumerate} \itemsep -2pt
		\item Sanjeev Arora, Boaz Barak, and Luca Trevisan, ``Survey Papers and Essays,'' in {\it Theory Matters Wiki: Theoretical Computer Science (TCS) Advocacy Wiki}, SIGACT Committee for the Advancement of Theoretical Computer Science, ACM Special Interest Group on Algorithms and Computation Theory (SIGACT), Association for Computing Machinery, February 25, 2010. Available at: \url{http://theorymatters.org/pmwiki/pmwiki.php?n=Main.SurveyCollection}; last accessed on September 14, 2010.
		\end{enumerate}
	\item WGBH Educational Foundation: \vspace{-0.2cm}
		\begin{enumerate} \itemsep -2pt
		\item Dot Diva / New Image for Computing (NIC) initiative: \vspace{-0.1cm}
			\begin{enumerate} \itemsep -1pt
			\item \url{http://dotdiva.org/}
			\item Resource for parents and teachers: \url{http://dotdiva.org/parents.html}
			\end{enumerate}
		\end{enumerate}
	\item Silicon Valley StRUT: \vspace{-0.2cm}
		\begin{enumerate} \itemsep -2pt
		\item Students Recycling Used Technology, StRUT, Competition; StRUT Competition consists of: \vspace{-0.1cm}
			\begin{enumerate} \itemsep -1pt
			\item Disassemble and Reassemble A Computer 
			\item Create and Present a Powerpoint Presentation 
			\item Computer Parts Identification and Challenge Test  
			\item Team Quiz Bowl on Computer Technology and Related Subjects
			\item \url{http://www.svstrut.org/cms/content/section/1/5/}
			\item Teacher Resources: \url{http://www.svstrut.org/cms/component/option,com_weblinks/catid,11/Itemid,10/}
			\item [ Resources to Support ] Curriculum for Engineering and Computer Technology Education: \url{http://www.svstrut.org/cms/content/view/8/18/}
			\end{enumerate}
		\item \url{http://www.svstrut.org/cms/}
		\end{enumerate}
	\item Google Code Jam (programming contest): \url{http://code.google.com/codejam/} and \url{http://en.wikipedia.org/wiki/Google_Code_Jam}
	\item University of Illinois at Urbana-Champaign (UIUC): \vspace{-0.2cm}
		\begin{enumerate} \itemsep -2pt
		\item College of Engineering; Department of Computer Science: \vspace{-0.1cm}
			\begin{enumerate} \itemsep -1pt
			\item Outreach \& Diversity: \url{http://cs.illinois.edu/outreach}
			\item ChicTech: \url{http://cs.illinois.edu/outreach/chictech}
			\item Technical Ambassadors: \url{http://cs.illinois.edu/outreach/tac}
			\item Games4Girls: \url{http://cs.illinois.edu/outreach/games4girls}
			\item Workshops \& Camps: \url{http://cs.illinois.edu/outreach/k12}
			\item \url{http://cs.illinois.edu/outreach}
			\end{enumerate}
		\end{enumerate}
	\item Carnegie Mellon University: \vspace{-0.2cm}
		\begin{enumerate} \itemsep -2pt
		\item women@SCS School of Computer Science, Carnegie Mellon University: \vspace{-0.1cm}
			\begin{enumerate} \itemsep -1pt
			\item Papers: \url{http://women.cs.cmu.edu/Resources/Papers/}
			\item Alumnae Interviews / Profiles: \url{http://women.cs.cmu.edu/Who/Alumnae/alumInterviews.php}
			\item Job and Research Opportunities: \url{http://www.women.cs.cmu.edu/Resources/JobsResearch/}
			\item Career Advice: \url{http://women.cs.cmu.edu/Resources/JobsResearch/careeradvice.php}
			\item Other Sites: \url{http://www.women.cs.cmu.edu/Miscellaneous/Other/}
			\end{enumerate}
		\end{enumerate}
	\item {\it Quora}: \vspace{-0.2cm}
		\begin{enumerate} \itemsep -2pt
		\item ``If a 10-year-old wanted to start programming today, what language path would be the most valuable moving forward?'' Available online at: \url{http://www.quora.com/If-a-10-year-old-wanted-to-start-programming-today-what-language-path-would-be-the-most-valuable-moving-forward}; last accessed on November 23, 2010.
		\end{enumerate}
	\end{enumerate}
%%%%%%%%%%%%%%%%%%%%%%%%%%%%%%%%%%%%%%%%
%%%%%%%%%%%%%%%%%%%%%%%%%%%%%%%%%%%%%%%%
\item Engineering Education Service Center (EESC): \vspace{-0.3cm}
	\begin{enumerate} \itemsep -2pt
	\item Has lists of: \vspace{-0.2cm}
		\begin{enumerate} \itemsep -2pt
		\item Educational material: \vspace{-0.1cm}
			\begin{enumerate} \itemsep -1pt
			\item books
			\item DVDs
			\item resource kits for teachers
			\end{enumerate}
		\item engineering camps (for the summer in the United States): \url{http://www.engineeringedu.com/camps/}
		\item {\it Women in Engineering} programs at US engineering schools: \url{http://www.engineeringedu.com/wie.html}
		\item US engineering schools: \url{http://www.engineeringedu.com/engrschools.htm}
		\item competitions for youths, including high school students: \url{http://www.engineeringedu.com/competitions.html}
		\item online resources
		\item list of professional organizations in engineering (or engineering societies): \url{http://www.engineeringedu.com/soc1.html}
		\item scholarships: \url{http://www.engineeringedu.com/scholars.html}
		\end{enumerate}
	\item It has resources for K-12 students, and their teachers and parents. It also has information for girls who are seeking careers in engineering; in addition, it provides their parents and teachers with information to guide the girls.
	\item It runs a workshop (in the US) for mother-daughter pairs to encourage girls to pursue careers in engineering.
	\item \url{http://www.engineeringedu.com/}
	\end{enumerate}
\item TryNano.org: \vspace{-0.3cm}
	\begin{enumerate} \itemsep -2pt
	\item Information about educational opportunities and careers in nanotechnology and nanoscience
	\item \url{TryNano.org}
	\end{enumerate}
\item {\it Mathematical Association of America} (MAA): \vspace{-0.3cm}
	\begin{enumerate} \itemsep -2pt
	\item Middle/High School Students: \url{http://www.maa.org/students/middle_high/}
	\item Parents: \url{http://www.maa.org/students/Parents.html}
	\item MAA American Mathematics Competitions: \vspace{-0.2cm}
		\begin{enumerate} \itemsep -2pt
		\item {\it Students} [resources]. Available at: \url{http://amc.maa.org/a-activities/a4-for-students/s-index.shtml}; last accessed on September 2, 2010.
		\item It includes tips to help students do well in math contests and Olympiads, a reading list for students interested in mathematics, problems from past math contests and Olympiads, and other resources from the World Wide Web.
		\end{enumerate}
	\item {\it Fun Math Sites}. Available at: \url{http://www.maa.org/students/funsites.html}; last accessed on September 2, 2010.
	\item Special Interest Group on Mathematics and the Arts (SIGMAA-ARTS): Resources, see \url{http://myweb.cwpost.liu.edu/aburns/sigmaa-arts/resources.html}.
	\item Special Interest Group of the MAA on Quantitative Literacy (SIGMAA QL): \url{http://sigmaa.maa.org/ql/}
	\end{enumerate}
\item eGFI (Engineering, Go For It!): \vspace{-0.3cm}
	\begin{enumerate} \itemsep -2pt
	\item Provides information for students, parents, and teachers about educational pathways and careers in engineering
	\item \url{http://egfi-k12.org/}
	\end{enumerate}
\item {\it Sloan Career Cornerstone Center}: \vspace{-0.3cm}
	\begin{enumerate} \itemsep -2pt
	\item Career exploration resources in STEM (science, technology, engineering, mathematics, computing, and healthcare)
	\item \url{http://www.careercornerstone.org/}
	\end{enumerate}
\item {\it TryEngineering}: \vspace{-0.3cm}
	\begin{enumerate} \itemsep -2pt
	\item Career exploration resources for engineering
	\item \url{http://www.tryengineering.org/}
	\end{enumerate}
\item {\it Junior Engineering Technical Society, JETS}: \vspace{-0.3cm}
	\begin{enumerate} \itemsep -2pt
	\item Career exploration resources for engineering
	\item \url{http://www.jets.org/}
	\end{enumerate}
\item {\it American Society of Mechanical Engineers, ASME}: \vspace{-0.3cm}
	\begin{enumerate} \itemsep -2pt
	\item K-12 Student Resources: \url{http://www.asme.org/Communities/Students/K12/} and \url{http://www.asme.org/Education/PreCollege/EngineeringResources/}
	\item Engineering Camps: \url{http://www.asme.org/Communities/Students/K12/Camps.cfm}
	\end{enumerate}
\item BESTRobotics, Inc.: \vspace{-0.3cm}
	\begin{enumerate} \itemsep -2pt
	\item BEST (Boosting Engineering, Science, and Technology) competition: \vspace{-0.2cm}
		\begin{enumerate} \itemsep -2pt
		\item \url{http://best.eng.auburn.edu/}
		\item Hosted at Auburn University's Samuel Ginn College of Engineering
		\item BEST World Championship: \url{http://best.eng.auburn.edu/world-championship/}
		\end{enumerate}
	\end{enumerate}
\item {\it American Society of Civil Engineers, ASCE}: \vspace{-0.3cm}
	\begin{enumerate} \itemsep -2pt
	\item Outreach resource for K-12 students, and their parents and teachers
	\item \url{http://content.asce.org/asceville/index.html}
	\end{enumerate}
\item {\it Science.gov} (USA.gov for Science): Internship and Fellowship Opportunities in Science (for high school students); see \url{http://www.science.gov/internships/k-12.html}
\item {\it iTunes U}: \vspace{-0.3cm}
	\begin{enumerate} \itemsep -2pt
	\item {\it iTunes} is required to listen to or watch these lectures, talks, and presentations.
	\item Access {\it iTunes U} at: \url{http://deimos3.apple.com/indigo/main/main.html?v0=WWW-AMUS-ITUNESU070521-N48LX}
	\item WGBH's Teachers' Domain -- Boston's PBS Station: Video presentation on ``Engineering for the Red Planet''; see \url{http://deimos3.apple.com/WebObjects/Core.woa/Browse/wgbh.org.1416254059.01416254061.1416793683?i=1951581658}. Also, check out its video clip on ``Carbon Fiber Car of the Future''.
	\item {\it iTunes U} is a set of webcast and podcasts, where you can easily find audio and video clips for lectures, seminars, announcements, virtual tours, and so on. For example, some professors from schools like MIT or Berkeley will provide audio/video clips of their lectures on {\it iTunes U}.
	\item This can help in exploring different majors during the college application process, or before a college student declares her/his major(s). If a student is not sure if she/he wants to double major in deaf studies and linguistics, this student can check out some linguistics lectures from her/his (preferred) college/university, if it uses {\it iTunes U}, or those from other universities.
	\end{enumerate}
\item High School Ace's College Prep Guide: \url{http://highschoolace.com/ace/colleges.cfm}
\item {\it Dr. Sally Ride} (America�s first woman in space): \vspace{-0.3cm}
	\begin{enumerate} \itemsep -2pt
	\item {\it Sally Ride Science}'s resources for educators: \url{https://www.sallyridescience.com/for_educators}
	\item Sally Ride Science Educator Institutes (to educate K-12 teachers about science): \url{https://www.sallyridescience.com/for_educators/institutes}
	\item {\it Sally Ride Science Academy} helps teachers to increase their students' interest in science: \url{https://www.sallyridescience.com/academy}
	\item {\it Sally Ride Science}'s resources for teachers: \url{https://www.sallyridescience.com/resources}
	\item {\it Sally Ride Science Festivals} are events for girls from the $5^{th}$ grade to the $8^{th}$ grade: \url{https://www.sallyridescience.com/festivals}
	\item {\it Sally Ride Science Camps} are summer camps for girls from the $4^{th}$ grade to the $9^{th}$ grade: \url{http://www.sallyridecamps.com/}
	\item GRAIL MoonKAM: \vspace{-0.2cm}
		\begin{enumerate} \itemsep -2pt
		\item ``GRAIL MoonKAM (Moon Knowledge Acquired by Middle school students) is GRAIL's signature education and public outreach program.''
		\item ``GRAIL MoonKAM will engage middle schools across the country in the GRAIL mission and lunar exploration.''
		\item \url{https://www.grailmoonkam.com/}
		\end{enumerate}
	\item EarthKAM: \vspace{-0.2cm}
		\begin{enumerate} \itemsep -2pt
		\item EarthKAM (Earth Knowledge Acquired by Middle school students) is a NASA educational outreach program enabling students, teachers and the public to learn about Earth from the unique perspective of space.
		\item \url{https://earthkam.ucsd.edu/}
		\end{enumerate}
	\end{enumerate}
\item Andrew Rader Studios: \vspace{-0.3cm}
	\begin{enumerate} \itemsep -2pt
	\item Chem4Kids.com: \url{http://www.chem4kids.com/}
	\end{enumerate}
\item {\it American Association for the Advancement of Science, AAAS}: \vspace{-0.3cm}
	\begin{enumerate} \itemsep -2pt
	\item ENTRY POINT! for Students With Disabilities (in STEM): \url{http://www.aaas.org/careercenter/fellowships/} and \url{http://ehrweb.aaas.org/entrypoint/}
	\item AAAS Mass Media Science \& Engineering Fellows Program (for STEM grad students to intern in mass media companies): \url{http://www.aaas.org/programs/education/MassMedia/}
	\item Diversity Issues: \url{http://sciencecareers.sciencemag.org/career_magazine/diversity_issues/}
	\item Internships involving science and journalism, human rights, scientific freedom, responsibility, or law: \url{http://www.aaas.org/careercenter/} and \url{http://www.aaas.org/careercenter/internships/scienceminority.shtml} (AAAS Minority Science Writers Internship)
	\item Kinetic City: \url{http://www.kineticcity.com/}
	\end{enumerate}
\item {\it NASA} resources for students: \url{http://www.nasa.gov/audience/forstudents/index.html} and \url{http://www.nasa.gov/offices/education/programs/national/summer/education_resources/index.html} (NASA Summer of Innovation)
\item National Academy of Engineering, NAE: \vspace{-0.3cm}
	\begin{enumerate} \itemsep -2pt
	\item NAE Grand Challenges: \vspace{-0.2cm}
		\begin{enumerate} \itemsep -2pt
		\item Includes a list of NAE Grand Challenges, which indicate some of the challenges faced by people on a global scale that can be partially solved by engineers. This can be used to get children and youths to be excited about engineering.
		\item NAE Grand Challenges: \vspace{-0.1cm}
			\begin{enumerate} \itemsep -1pt
			\item Make solar energy economical
			\item Provide energy from fusion
			\item Develop carbon sequestration methods
			\item Manage the nitrogen cycle
			\item Provide access to clean water
			\item Restore and improve urban infrastructure
			\item Advance health informatics
			\item Engineer better medicines
			\item Reverse-engineer the brain
			\item Prevent nuclear terror
			\item Secure cyberspace
			\item Enhance virtual reality
			\item Advance personalized learning
			\item Engineer the tools of scientific discovery
			\end{enumerate}
		\item \url{http://www.engineeringchallenges.org/}
		\item NAE Grand Challenge K12 Partners Program: \vspace{-0.1cm}
			\begin{enumerate} \itemsep -1pt
			\item \url{http://www.grandchallengek12.org/about}
			\item 5-Part Make it Happen Plan: \url{http://www.grandchallengek12.org/5-part-plan}
			\end{enumerate}
		\end{enumerate}
	\item {\it National Academy of Engineering}'s technological literacy program for people (students, parents, and educators) to learn more about technology: \url{http://www.nae.edu/nae/techlithome.nsf}
	\item Greatest Engineering Achievements: \url{http://www.greatachievements.org/}
	\end{enumerate}
\item National Science Foundation: \vspace{-0.3cm}
	\begin{enumerate} \itemsep -2pt
	\item Broadening Participation in Computing (BPC): \vspace{-0.2cm}
		\begin{enumerate} \itemsep -2pt
		\item \url{http://www.bpcportal.org/}
		\item \url{http://www.bpcportal.org/bpc/shared/home.jhtml;jsessionid=0MIUYDR5U4ARXABAVRSSFEQ?_requestid=9445}
		\item \url{http://www.nsf.gov/funding/pgm_summ.jsp?pims_id=13510}
		\item \url{http://www.nsf.gov/funding/pgm_summ.jsp?pims_id=13510&org=NSF&sel_org=NSF&from=fund}
		\item ``Broadening Participation in Computing (BPC) is a NSF sponsored program with the goal of significantly increasing the number of underrepresented graduates in the computing disciplines, with an emphasis on women, persons with disabilities, and minorities (African Americans, Hispanics, American Indians, Alaska Natives, Native Hawaiians, and Pacific Islanders).''
		\item Broadening Participation in Computing Digital Library: \vspace{-0.1cm}
			\begin{enumerate} \itemsep -1pt
			\item \url{http://www.bpcportal.org/bpc/interdiscipline/dl_index.jhtml;jsessionid=ROYEHJV1UQYWNABAVRSSFEQ?comm=BPC}
			\item Includes resources for different target populations: \vspace{-0.1cm}
				\begin{itemize} \itemsep -1pt
				\item Women
				\item African Americans
				\item Hispanic Americans, or Latinas and Latinos
				\item People with disabilities
				\item Native Americans
				\end{itemize}
			\item It also includes resources for different topics, such as mentoring, recruitment, retention, and work-life balance.
			\end{enumerate}
		\item Alliances (other professional organizations): \url{http://www.bpcportal.org/bpc/comm/projects.jhtml}
		\end{enumerate}
	\item The National Science Digital Library (NSDL): \vspace{-0.2cm}
		\begin{enumerate} \itemsep -2pt
		\item \url{http://www.nsdl.org/} and \url{http://www.nsdl.org/browse/}
		\item ``The National Science Digital Library is a national network dedicated to advancing STEM teaching and learning for all learners, in both formal and informal settings, and the locus of activity for the National Science Foundation's National STEM Distributed Learning program.''
		\item Outreach materials: \vspace{-0.1cm}
			\begin{enumerate} \itemsep -1pt
			\item \url{http://www.nsdl.org/pd/?pager=materials}
			\item Has outreach materials for educators in K-12 and higher educational institutions.
			\end{enumerate}
		\item Resources for K-12 Teachers: \url{http://nsdl.org/resources_for/k12_teachers/}
		\item Resources for Librarians: \url{http://nsdl.org/resources_for/librarians/}
		\item Billingual Resources: \url{http://www.nsdlnetwork.org/collections/billingual-resources}
		\item NSDL on {\it iTunes U}: \url{http://www.nsdl.org/iTunesU/}
		\item Collections: \url{http://www.nsdl.org/browse/?subject=All}
		\item NSDL Pathways: \vspace{-0.1cm}
			\begin{enumerate} \itemsep -1pt
			\item \url{http://nsdl.org/about/?pager=pathways}
			\item ``Pathways are large projects that are aggregators and stewards of resources and services to broad categories of users---either discipline-focused (e.g. chemistry), or audience-focused (e.g. middle school educators), or resources of a specific type or format (e.g. multimedia content).''
			\item ``They are digital library portals developed and managed in partnership with organizations and institutions that have a history and expertise in serving their portal's target audiences.''
			\item ``They contribute metadata (descriptive information) about their resources to NSDL to make their resources searchable and discoverable via the NSDL.org portal, in addition to their own portals.''
			\end{enumerate}
		\item {\bf NSDL Science Literacy Maps}: \vspace{-0.1cm}
			\begin{enumerate} \itemsep -1pt
			\item \url{http://strandmaps.nsdl.org/}
			\item ``{\it NSDL Science Literacy Maps} are a tool for teachers and students to find resources that relate to specific science and math concepts. The maps illustrate connections between concepts as well as how concepts build upon one another across grade levels.''
			\end{enumerate}
		\item NSDL Professional Development: \url{http://www.nsdl.org/pd/}
		\item NSDL Technical Network Services: \vspace{-0.1cm}
			\begin{enumerate} \itemsep -1pt
			\item \url{http://www.nsdl.org/about/?pager=tns}
			\item \url{http://nsdlnetwork.org/}
			\item \url{http://nsdlnetwork.org/content/book/page/953/about-nsdl-technical-network-services}
			\end{enumerate}
		\item NSDL Resource Center: \url{http://nsdlnetwork.org/content/book/951/page/954/about-nsdl-resource-center}
		\end{enumerate}
	\end{enumerate}
\item {\it American Chemical Society} Science for Kids program (for students in K-12): \url{http://portal.acs.org/portal/acs/corg/content?_nfpb=true&_pageLabel=PP_TRANSITIONMAIN&node_id=878&use_sec=false&sec_url_var=region1&__uuid=984d4ee7-4214-4d35-9899-bc2f91dee58b}
\item {\it California Digital Educator Consortium}, ``Digital Educator,'' Digital Learning Center: \url{http://www.digitaleducator.com/}
\item Kenny Felder, ``Selected Other Educational Sites on the Web''. Available at: \url{http://www4.ncsu.edu/unity/lockers/users/f/felder/public/kenny/edulinks.html}; last accessed on August 28, 2010.
\item FHSST (Free High School Science Texts); free textbooks for grades 10-12 in Physics, Chemistry, and Mathematics. Available at: \url{http://www.fhsst.org/}; last accessed on August 28, 2010.
\item John Baez, {\it Usenet Physics FAQ}, Department of Mathematics, University of California, Riverside, September 2009. Available at: \url{http://math.ucr.edu/home/baez/physics/}; last accessed on August 28, 2010.
\item {\it American Society for Engineering Education}: \vspace{-0.3cm}
	\begin{enumerate} \itemsep -2pt
	\item Science and Engineering Apprenticeship Program (SEAP): \vspace{-0.2cm}
		\begin{enumerate} \itemsep -2pt
		\item ``The Science and Engineering Apprenticeship Program (SEAP) provides an opportunity for students to participate in research at a Department of Navy (DoN) laboratory during the summer.''
		\item ``The goals of SEAP are to encourage participating students to pursue science and engineering careers, to further their education via mentoring by laboratory personnel and their participation in research, and to make them aware of DoN Research and technology efforts, which can lead to employment within the DoN.''
		\item ``High school students who have completed at least Grade 9. A graduating senior is eligible to apply.''
		\item ``Must be 16 years of age for most laboratories. Some laboratories may accept a 15 year old applicant. Please check individual lab description for more details.''
		\item ``Applicants must be US citizens and participation by Permanent Resident Aliens is limited. Please check individual lab descriptions for participation of Permanent Resident Aliens.''
		\item \url{http://seap.asee.org/}
		\end{enumerate}
	\end{enumerate}
\item robots.net, {\it Robot Competitions} (list of robot competitions and contests) : \url{http://robots.net/rcfaq.html}
\item International Council on Systems Engineering (INCOSE): \vspace{-0.3cm}
	\begin{enumerate} \itemsep -2pt
	\item Careers in Systems Engineering: \url{http://www.incose.org/educationcareers/careersinsystemseng.aspx}
	\item Frequently Asked Questions for Students [about Systems Engineering]: \url{http://www.incose.org/educationcareers/faqsforstudents.aspx}
	\item What is Systems Engineering?: \url{http://www.incose.org/practice/whatissystemseng.aspx}
	\end{enumerate}
\item {\it National Society of Professional Engineers}: \vspace{-0.3cm}
	\begin{enumerate} \itemsep -2pt
	\item A Sightseer's Guide to Engineering: \url{http://www.engineeringsights.org/}
	\end{enumerate}
\item {\it Engineers Dedicated to a Better Tomorrow (a.k.a., DedicatedEngineers)}: \vspace{-0.3cm}
	\begin{enumerate} \itemsep -2pt
	\item The ``K-12 Crowd'' (Students, Teachers, Guidance Counselors and Parents): \url{http://www.dedicatedengineers.org/intro_for_K-12.htm}
	\item \url{http://www.dedicatedengineers.org/}
	\end{enumerate}
\item National Engineers Week Foundation: \vspace{-0.3cm}
	\begin{enumerate} \itemsep -2pt
	\item Discover Engineering: \url{http://www.discoverengineering.org/}
	\item Introduce A Girl to Engineering: \url{http://www.eweek.org/EngineersWeek/IntroduceAGirl.aspx}
	\item All About Engineering: \url{http://www.eweek.org/AboutEngineering/AboutEngineering.aspx}
	\end{enumerate}
\item University of California: \vspace{-0.3cm}
	\begin{enumerate} \itemsep -2pt
	\item The Coalition For Science After School: \vspace{-0.2cm}
		\begin{enumerate} \itemsep -2pt
		\item \url{http://afterschoolscience.org/}
		\item ``Promoting high-quality afterschool science'' ... ``The Coalition for Science After School envisions the day when young people from all backgrounds have access to high-quality science, technology, engineering and mathematics (STEM) learning beyond the classroom.''
		\item Tools for advocates--Championing afterschool science: \url{http://afterschoolscience.org/tools/}
		\item Program resources--Enhancing the quality of afterschool opportunities: \url{http://afterschoolscience.org/resources/}
		\item The National After School Science Directory: \vspace{-0.1cm}
			\begin{enumerate} \itemsep -1pt
			\item \url{http://afterschoolscience.org/directory/}
			\item ``The National After School Science Directory is a searchable database designed to increase access to high-quality science, technology, engineering and math (STEM) education beyond the classroom for youth and families across the nation. The Directory houses thousands of STEM opportunities, submitted by science centers, museums, schools and other youth-serving organizations. Search our Directory to view opportunities to connect the America's youth to high-quality STEM learning experiences.''
			\end{enumerate}
		\item Become an advocate: \url{http://afterschoolscience.org/tools/advocate.php}
		\item Funders (funding organizations/agencies): \url{http://afterschoolscience.org/tools/funders.php}
		\end{enumerate}
	\end{enumerate}
\item Harvey Mudd College: \vspace{-0.3cm}
	\begin{enumerate} \itemsep -2pt
	\item Francis Edward Su, {\it Math Fun Facts!}, Department of Mathematics, Harvey Mudd College: \url{http://www.math.hmc.edu/funfacts/}
	\end{enumerate}
\item Clay Mathematics Institute: \vspace{-0.3cm}
	\begin{enumerate} \itemsep -2pt
	\item Program in Mathematics for Young Scientists, PROMYS: \vspace{-0.2cm}
		\begin{enumerate} \itemsep -2pt
		\item \url{http://www.claymath.org/programs/outreach/PROMYS/}
		\item \url{http://math.bu.edu/people/promys/}
		\item \url{http://www.promys.org/}
		\end{enumerate}
	\item Ross Program (for pre-college students): \vspace{-0.2cm}
		\begin{enumerate} \itemsep -2pt
		\item \url{http://www.claymath.org/programs/outreach/ross/}
		\item \url{http://www.math.ohio-state.edu/ross/}
		\end{enumerate}
	\item CMI Summer Schools: \url{http://www.claymath.org/programs/summer_school/}
	\end{enumerate}
\item Consortium for Ocean Leadership: \vspace{-0.3cm}
	\begin{enumerate} \itemsep -2pt
	\item Oceans of Opportunity (for African American students in K-12, and colleges and universities -- includes undergraduates and grad students): \url{http://www.oceanleadership.org/education/diversity/oceans-of-opportunity/}
	\item The JOIDES Resolution (The JR) scientific research vessel [ Deep Earth Academy ]: \vspace{-0.2cm}
		\begin{enumerate} \itemsep -2pt
		\item Fun \& Games: \url{http://joidesresolution.org/node/53}
		\item Discovery Center: \url{http://joidesresolution.org/node/44}
		\item Just for Kids Blog: \url{http://joidesresolution.org/node/366}
		\end{enumerate}
	\item National Ocean Sciences Bowl (high school academic competition that provides a forum for talented students to test their knowledge of the marine sciences including biology, chemistry, physics, and geology): \vspace{-0.2cm}
		\begin{enumerate} \itemsep -2pt
		\item \url{http://www.nosb.org/}
		\item Career Resources: \url{http://www.nosb.org/ocean-careers/career-resources/}
		\end{enumerate}
	\item Integrated Ocean Drilling Program (IODP), IODP United States Implementing Organization (IODP-USIO): \vspace{-0.2cm}
		\begin{enumerate} \itemsep -2pt
		\item U.S.-sponsored Teacher at Sea Program (for US teachers to participate in seagoing research experiences aboard the JOIDES Resolution): \url{http://www.iodp-usio.org/Education/TAS.html}
		\end{enumerate}
	\item Careers: \url{http://www.oceanleadership.org/education/deep-earth-academy/students/careers/}
	\end{enumerate}
\item The Oceanography Society: \vspace{-0.3cm}
	\begin{enumerate} \itemsep -2pt
	\item Careers in Oceanography: Profiles, \url{http://www.tos.org/resources/career_profiles.html}
	\item Links [includes links to educational material for students in K-12]: \url{http://www.tos.org/resources/links.html}
	\end{enumerate}
\item American Geophysical Union: \vspace{-0.3cm}
	\begin{enumerate} \itemsep -2pt
	\item Bright Students Training as Research Scientists (Bright STaRS): \vspace{-0.2cm}
		\begin{enumerate} \itemsep -2pt
		\item \url{http://www.agu.org/education/diversity_programs/bstars.shtml}
		\item ``High school students participating in after-school and summer research experiences in the Earth and space sciences are invited to participate in the AGU Bright STaRS program. The Bright STaRS program provides a dedicated forum for $\sim$50 students to present their own research results to the scientific community and learn about exciting research, education, and career opportunities in the geosciences.''
		\end{enumerate}
	\end{enumerate}
\item American Geological Institute, AGI: \vspace{-0.3cm}
	\begin{enumerate} \itemsep -2pt
	\item AGI Education Department: \url{http://www.agiweb.org/geoeducation.html}
	\end{enumerate}
\item Society for Science \& the Public (SSP): \vspace{-0.3cm}
	\begin{enumerate} \itemsep -2pt
	\item Intel International Science \& Engineering Fair (Intel ISEF), which is a pre-college science competition: \url{http://www.societyforscience.org/isef/}
	\item Broadcom MASTERS\texttrademark\ competition (which stands for Broadcom Math, Applied Science, Technology and Engineering for Rising Stars): \vspace{-0.2cm}
		\begin{enumerate} \itemsep -2pt
		\item Is a U.S. ``national science, technology, engineering, and math competition for America's $6^{th}$, $7^{th}$, and $8^{th}$ graders.''
		\item \url{http://www.societyforscience.org/masters} or \url{http://www.broadcomfoundation.org/masters/}
		\end{enumerate} 
	\item Science resources: \url{http://www.societyforscience.org/resources}
	\item Science News: \url{http://www.sciencenews.org/}
	\item Science News for Kids (for ``children of ages 9-14, their teachers and their parents''): \url{http://www.societyforscience.org/sciencenewsforkids} and \url{http://www.sciencenewsforkids.org/}
	\end{enumerate}
\item Institute for Operations Research and the Management Sciences (INFORMS): \vspace{-0.3cm}
	\begin{enumerate} \itemsep -2pt
	\item Operations Research: The Science of Better, \url{http://www.scienceofbetter.org/}
	\end{enumerate}
\item Technion - Israel Institute of Technology: \vspace{-0.3cm}
	\begin{enumerate} \itemsep -2pt
	\item SciTech - the summer camp for talented students ($11^{th}$ and $12^{th}$ graders from all over the world): \url{http://www.scitech.technion.ac.il/}
	\end{enumerate}
\item USA Science \& Engineering Festival: \url{http://www.usasciencefestival.org/}
\item Girl Scouts: \vspace{-0.3cm}
	\begin{enumerate} \itemsep -2pt
	\item Girl Scouts of Western New York: \vspace{-0.2cm}
		\begin{enumerate} \itemsep -2pt
		\item STEM Resource Guide: \url{http://www.gswny.org/Data/Documents/STEM%2520Resource%2520Guide%25202010-Oct-11.pdf}
		\item Also, see \url{http://www.gswny.org/Programs/Awards/Gold/}; scroll to the bottom of the page and look under the subsection heading, ``Tell Us About Your Gold Award Project''
		\end{enumerate}
	\item Science, Technology, Engineering and Math (STEM): \url{http://www.girlscouts.org/program/program_opportunities/science/}
	\end{enumerate}
\item American Museum of Science and Energy (AMSE): \vspace{-0.3cm}
	\begin{enumerate} \itemsep -2pt
	\item \url{http://www.amse.org/}
	\item Owned by the US Department of Energy, and managed under Oak Ridge National Laboratory
	\item Educators: \url{http://www.amse.org/content.aspx?article=1140&parent=30}
	\item Educational Programs: \url{http://www.amse.org/content.aspx?article=1139&parent=30}
	\item Home school programs: \url{http://www.amse.org/content.aspx?article=1169&parent=30}
	\item Online resources: \url{http://www.amse.org/content.aspx?article=1170&parent=30}
	\end{enumerate}
\item Center for Energy Workforce Development (CEWD): \vspace{-0.3cm}
	\begin{enumerate} \itemsep -2pt
	\item Teachers and guidance counselors: \vspace{-0.2cm}
		\begin{enumerate} \itemsep -2pt
		\item \url{http://www.cewd.org/educators_index.asp}
		\item Lesson plans for teachers: \url{http://www.cewd.org/educators_lessonplans.asp}
		\end{enumerate}
	\item Parents: \url{http://www.cewd.org/parents_index.asp}
	\end{enumerate}
\item TryScience: \url{http://tryscience.net/tryscinetmain.nsf/Welcome?OpenPage}
\item The Dana Foundation: \vspace{-0.3cm}
	\begin{enumerate} \itemsep -2pt
	\item Brainy Kids: \vspace{-0.2cm}
		\begin{enumerate} \itemsep -2pt
		\item \url{http://www.dana.org/resources/brainykids/}
		\item Fun: \vspace{-0.1cm}
			\begin{enumerate} \itemsep -1pt
			\item \url{http://dana.org/resources/brainykids/detail.aspx?folder_id=104}
			\item Has interactive online games, activities, and fun quizzes on: \vspace{-0.1cm}
				\begin{itemize} \itemsep -1pt
				\item biology
				\item health
				\item neuroscience
				\item astronomy
				\item chemistry
				\item ecology
				\end{itemize}
			\end{enumerate}
		\item The Lab: \vspace{-0.1cm}
			\begin{enumerate} \itemsep -1pt
			\item \url{http://dana.org/resources/brainykids/detail.aspx?folder_id=106}
			\item Has maps of the brain, virtual dissections, resources for science fairs, and virtual microscopes
			\end{enumerate}
		\item Lesson Plans: \vspace{-0.1cm}
			\begin{enumerate} \itemsep -1pt
			\item \url{http://dana.org/resources/brainykids/detail.aspx?folder_id=108}
			\item Includes resources that cover the history of science and technology, lesson plans for K-12 science teachers, and science news for youths.
			\end{enumerate}
		\item The Mindboggling Workbook: \vspace{-0.1cm}
			\begin{enumerate} \itemsep -1pt
			\item \url{http://www.dana.org/uploadedFiles/The_Dana_Alliances/mindboggling_workbook.pdf}
			\item ``A fun-filled activity book about the brain for children in grades K-3 (ages 5-9). Provides an introduction to how the brain works, what the brain does, its importance, and how to take care of it.''
			\end{enumerate}
		\end{enumerate}
	\end{enumerate}
\item University of New Mexico: \vspace{-0.3cm}
	\begin{enumerate} \itemsep -2pt
	\item Department of Mathematics and Statistics: \vspace{-0.2cm}
		\begin{enumerate} \itemsep -2pt
		\item UNM - PNM Statewide Mathematics Contest (sponsored by the PNM Foundation): \url{http://mathcontest.unm.edu/}
		\end{enumerate}
	\end{enumerate}
\item Center for Energy Workforce (CEWD): \vspace{-0.3cm}
	\begin{enumerate} \itemsep -2pt
	\item Get Into Energy: \vspace{-0.2cm}
		\begin{enumerate} \itemsep -2pt
		\item \url{http://www.getintoenergy.com/index.asp} and \url{http://www.getintoenergy.com/careers.asp}
		\item Fun educational resources for students: \url{http://www.getintoenergy.com/students.asp}
		\item Career Quiz: \vspace{-0.1cm}
			\begin{enumerate} \itemsep -1pt
			\item \url{http://www.getintoenergy.com/search/careerquizj.asp}
			\item Help you find out more about career options in the energy field
			\end{enumerate}
		\item Career Resources: \vspace{-0.1cm}
			\begin{enumerate} \itemsep -1pt
			\item \url{http://www.getintoenergy.com/careerresources.asp}
			\item Has information on: \vspace{-0.1cm}
				\begin{itemize} \itemsep -1pt
				\item Training Programs (technical schools and colleges)
				\item Work-based Programs (apprenticeships and internships)
				\item Featured Employers
				\end{itemize}
			\end{enumerate}
		\item Skills Needed in the Energy Field: \vspace{-0.1cm}
			\begin{enumerate} \itemsep -1pt
			\item \url{http://www.getintoenergy.com/skills.asp}
			\item List skills for different kinds of jobs in the energy field
			\end{enumerate}
		\item Information for parents: \url{http://www.getintoenergy.com/Parents.asp}
		\item Information for teachers and guidance counselors: \url{http://www.getintoenergy.com/Educators.asp}
		\end{enumerate}
	\end{enumerate}
\item University of Utah: \vspace{-0.3cm}
	\begin{enumerate} \itemsep -2pt
	\item Department of Electrical and Computer Engineering: \vspace{-0.2cm}
		\begin{enumerate} \itemsep -2pt
		\item Prof. Cynthia Furse: \vspace{-0.1cm}
			\begin{enumerate} \itemsep -1pt
			\item Cynthia Furse, {\it K-12 Engineering Outreach}, August 2007. Available online at: \url{http://www.ece.utah.edu/~cfurse/K12.html}; last accessed on December 10, 2010.
			\item Cynthia Furse, {\it U Dream. U Design. U Create.}, Department of Electrical and Computer Engineering, University of Utah. Available online at: \url{http://www.ece.utah.edu/~cfurse/NSF/}; last accessed on December 10, 2010.
			\end{enumerate}
		\end{enumerate}
	\end{enumerate}
\item Society for Industrial and Applied Mathematics: \vspace{-0.3cm}
	\begin{enumerate} \itemsep -2pt
	\item Public Awareness: \vspace{-0.2cm}
		\begin{enumerate} \itemsep -2pt
		\item Math Competitions, \url{http://www.siam.org/publicawareness/competitions.php}
		\item Moody's Mega Math Challenge (M3 Challenge) is an applied mathematics competition for high school students. Available online at: \url{http://m3challenge.siam.org/}; last accessed on December 13, 2010.
		\item {\it Math Matters, Apply It!}: \url{http://www.siam.org/careers/matters.php}
		\item Nuggets: \url{http://www.siam.org/publicawareness/nuggets.php}
		\end{enumerate}
	\item Society for Industrial and Applied Mathematics, ``Unveiling Why Do Math,'' May 27, 2010. Available online at: \url{http://www.siam.org/about/news-siam.php?id=1741}; last accessed on December 13, 2010.
	\end{enumerate}
\item International Federation of Operational Research Societies (IFORS): \vspace{-0.3cm}
	\begin{enumerate} \itemsep -2pt
	\item Association of European Operational Research Societies (EURO): \vspace{-0.2cm}
		\begin{enumerate} \itemsep -2pt
		\item {\it What is Operational Research?}: \url{http://www.euro-online.org/display.php?pageid=197&}
		\item Applications of OR in music, literature, and aesthetics: \url{http://www.euro-online.org/display.php?pageid=211&}
		\item 24 Hours Operations Research: \url{http://www.24hor.org/}
		\item Branding OR: \url{http://www.euro-online.org/display.php?pageid=198&}
		\end{enumerate}
	\end{enumerate}
\item American Institute of Aeronautics and Astronautics (AIAA): \vspace{-0.3cm}
	\begin{enumerate} \itemsep -2pt
	\item Students \& Educators: \url{http://www.aiaa.org/content.cfm?pageid=5}
	\item Ask An Engineer: \url{http://www.aiaa.org/content.cfm?pageid=214}
	\item Kid's Place: \vspace{-0.2cm}
		\begin{enumerate} \itemsep -2pt
		\item \url{http://www.aiaa.org/content.cfm?pageid=473}
		\item Enjoy games, puzzles, fun experiments, teen-recommended books and movies, and more.
		\end{enumerate}
	\item History of Flight Timeline: \url{http://www.aiaa.org/content.cfm?pageid=260}
	\item Ask Polaris: \vspace{-0.2cm}
		\begin{enumerate} \itemsep -2pt
		\item \url{http://www.askpolaris.org/}
		\item Resource for career exploration in aerospace engineering and related fields
		\end{enumerate}
	\end{enumerate}
\item Massachusetts Institute of Technology: \vspace{-0.3cm}
	\begin{enumerate} \itemsep -2pt
	\item MIT School of Engineering: \vspace{-0.2cm}
		\begin{enumerate} \itemsep -2pt
		\item Lemelson-MIT Program: \vspace{-0.1cm}
			\begin{enumerate} \itemsep -1pt
			\item \url{http://web.mit.edu/invent/}
			\item Inventor's Handbook: \url{http://web.mit.edu/invent/h-main.html}
			\item Games \& Trivia; \url{http://web.mit.edu/invent/g-main.html}
			\item Links \& Resources: \url{http://web.mit.edu/invent/r-main.html}
			\end{enumerate}
		\end{enumerate}
	\end{enumerate}
\item BT Group plc: \vspace{-0.3cm}
	\begin{enumerate} \itemsep -2pt
	\item British Telecommunications plc (BT): \vspace{-0.2cm}
		\begin{enumerate} \itemsep -2pt
		\item BT Young Scientist \& Technology Exhibition: \vspace{-0.1cm}
			\begin{enumerate} \itemsep -1pt
			\item \url{http://www.btyoungscientist.com/}
			\item \url{http://www.btyoungscientist.com/all-you-need-to-know/}
			\item Science and technology fair for high/secondary school students in Ireland
			\end{enumerate}
		\end{enumerate}
	\end{enumerate}
\item NHS Medical Careers: \vspace{-0.3cm}
	\begin{enumerate} \itemsep -2pt
	\item \url{http://www.medicalcareers.nhs.uk/Default.aspx}
	\item Provides information about careers in medicine for prospective medical students, medical students, medical school graduates (or young medical professionals), (medical speciality) trainers, and medical specialists.
	\end{enumerate}
\item British Science Association: \vspace{-0.3cm}
	\begin{enumerate} \itemsep -2pt
	\item British Science Festival: \vspace{-0.2cm}
		\begin{enumerate} \itemsep -2pt
		\item \url{http://www.britishscienceassociation.org/web/BritishScienceFestival/AboutFestival/index.htm}
		\item Festival Student Bursaries: \url{http://www.britishscienceassociation.org/web/BritishScienceFestival/StudentBursaries/index.htm}
		\end{enumerate}
	\item National Science \& Engineering Week: \url{http://www.britishscienceassociation.org/web/NSEW/index.htm}
	\item Clubs, CREST Awards and Fairs (programs and activities for children and youth, 5-19 years of age): \url{http://www.britishscienceassociation.org/web/ccaf/index.htm}
	\item National Science \& Engineering Competition: \url{http://www.britishscienceassociation.org/web/NSEC/index.htm} and \url{http://www.thebigbangfair.co.uk/nsec/}
	\end{enumerate}
\item Research Councils UK (RCUK): \vspace{-0.3cm}
	\begin{enumerate} \itemsep -2pt
	\item \url{http://www.rcuk.ac.uk/per/Pages/Schools.aspx}
	\item Schoolscience: \vspace{-0.2cm}
		\begin{enumerate} \itemsep -2pt
		\item \url{http://www.schoolscience.co.uk/}
		\item For students and educators in K-12 to enrich the learning experiences of science topics, and help students connect classroom material to the real world.
		\item Teacher Zone - professional resources for teachers: \url{http://www.schoolscience.co.uk/teacher_zone.cfm}
		\item Interactive Learning Resources: \url{http://www.schoolscience.co.uk/interactives.cfm}
		\item Free Resources: \url{http://www.schoolscience.co.uk/freebies.cfm}
		\item Competitions: \url{http://www.schoolscience.co.uk/competitions.cfm}
		\item Research focus: \url{http://www.schoolscience.co.uk/research_focus.cfm}
		\item Resources on the World Wide Web: \url{http://www.schoolscience.co.uk/sciencelink.cfm}
		\end{enumerate}
	\item Researchers in Residence (RinR): \vspace{-0.2cm}
		\begin{enumerate} \itemsep -2pt
		\item \url{http://www.researchersinresidence.ac.uk/cms/schools-colleges/}
		\item For students in middle and high schools to job shadow (observe first-hand) a Ph.D. student or postdoctoral researcher in her/his research activities for up to a week, so that students can learn what doing research in her/his research area is like. In addition, the researcher would explain in laypeople's terms what her/his research is about. It can be considered as an externship program.
		\end{enumerate}
	\item Nuffield Bursaries: \vspace{-0.2cm}
		\begin{enumerate} \itemsep -2pt
		\item \url{http://www.nuffieldfoundation.org/capacity-building}
		\item \url{http://www.nuffieldfoundation.org/science-bursaries-schools-and-colleges}
		\item For high school juniors/seniors to pursue a research internship in science and engineering.
		\end{enumerate}
	\item CREST (Creativity in Science and Technology): \vspace{-0.2cm}
		\begin{enumerate} \itemsep -2pt
		\item \url{http://www.britishscienceassociation.org/web/ccaf/CREST/index.htm}
		\item Program to help students get engaged in a science or engineering project, where they learn how to solve real problems in science or engineering.
		\end{enumerate}
	\end{enumerate}
\item Nuffield Foundation: \vspace{-0.3cm}
	\begin{enumerate} \itemsep -2pt
	\item Science bursaries for schools and colleges: \url{http://www.nuffieldfoundation.org/science-bursaries-schools-and-colleges}
	\item Students: \url{http://www.nuffieldfoundation.org/students}
	\item Twenty First Century Science: \vspace{-0.2cm}
		\begin{enumerate} \itemsep -2pt
		\item \url{http://www.21stcenturyscience.org/}
		\item ``Twenty First Century Science is a set of GCSE science courses giving all 14-16-year-olds a worthwhile and inspiring experience of science. The strength of the programme is that it meets the needs, through flexible options, of those who will go on to be professional scientists and of those who will not.''
		\item The Courses: \url{http://www.21stcenturyscience.org/the-courses/}
		\item Assessment overview: \url{http://www.21stcenturyscience.org/assess/}
		\item Teaching resources: \url{http://www.21stcenturyscience.org/resources/}
		\end{enumerate}
	\item Science in Society: \vspace{-0.2cm}
		\begin{enumerate} \itemsep -2pt
		\item \url{http://www.scienceinsocietyadvanced.org/}
		\item ``Science in Society is an interesting and topical GCE advanced level course. It aims to develop the knowledge and skills that are needed for students to understand how science works, analyse contemporary issues involving science and technology and communicate their scientific appreciation and understanding to others.''
		\end{enumerate}
	\item Parents: \url{http://www.nuffieldfoundation.org/parents}
	\item Education: \url{http://www.nuffieldfoundation.org/education}
	\item Teachers (has excellent resources for science and mathematics): \url{http://www.nuffieldfoundation.org/teachers}
	\item Capacity building: \url{http://www.nuffieldfoundation.org/capacity-building}
	\end{enumerate}
\item The Story of Stuff Project (by Annie Leonard): \vspace{-0.3cm}
	\begin{enumerate} \itemsep -2pt
	\item \url{http://www.storyofstuff.com/}
	\item ``The Story of Stuff Project was created by Annie Leonard to leverage and extend the film's impact. We amplify public discourse on a series of environmental, social and economic concerns and facilitate the growing Story of Stuff community's involvement in strategic efforts to build a more sustainable and just world.''
	\item Resources: \vspace{-0.2cm}
		\begin{enumerate} \itemsep -2pt
		\item \url{http://www.storyofstuff.com/resources.php}
		\item The Story of Stuff Project PDFs: \url{http://www.storyofstuff.com/dl-pdfs.php}
		\item Teaching Tools: \url{http://www.storyofstuff.com/teach.php}
		\item More About Stuff: \url{http://www.storyofstuff.com/aboutstuff.php}
		\item Recommended Reading \& Bibliography: \url{http://www.storyofstuff.com/reading.php}
		\item Get Involved: \url{http://www.storyofstuff.com/getinvolved.php}
		\item Curricula: \url{http://storyofstuff.org/curricula.php}
		\end{enumerate}
	\end{enumerate}
\item Facing the Future: \vspace{-0.3cm}
	\begin{enumerate} \itemsep -2pt
	\item \url{http://www.facingthefuture.org/}
	\item ``{\it Facing the Future} engages students in learning by making academics relevant to their lives. We empower students to think critically, develop a global perspective, and participate in positive solutions for a sustainable future.''
	\item Curriculum Alignment with Education Standards: \url{http://www.facingthefuture.org/Curriculum/AlignmentwithEducationStandards/tabid/116/Default.aspx}
	\item Global Sustainability Curriculum Finder: \url{http://www.facingthefuture.org/Curriculum/FindCurriculumthatisRightforYou/tabid/68/Default.aspx}
	\item Download FREE Global Issues and Sustainability Curriculum: \url{http://www.facingthefuture.org/Curriculum/DownloadFreeCurriculum/tabid/114/Default.aspx}
	\item Classroom Examples: How Engaging Curriculum Can Help Address Classroom Challenges, \url{http://www.facingthefuture.org/ForEducators/ClassroomExamples/tabid/213/Default.aspx}
	\item Our Impact on Student Achievement: \url{http://www.facingthefuture.org/ForEducators/OurImpactonStudentAchievement/tabid/73/Default.aspx}
	\item Action Project Database: \url{http://www.facingthefuture.org/ServiceLearning/ActionProjectDatabase/tabid/94/Default.aspx}
	\item Service Learning Examples: \url{http://www.facingthefuture.org/ServiceLearning/ExamplesofStudentsTakingAction/tabid/147/Default.aspx}
	\item Curriculum: \url{http://www.facingthefuture.org/Curriculum/CurriculumHome/tabid/113/Default.aspx}
	\end{enumerate}
\item U.S. Department of Energy: \vspace{-0.3cm}
	\begin{enumerate} \itemsep -2pt
	\item Office of Science: \vspace{-0.2cm}
		\begin{enumerate} \itemsep -2pt
		\item U.S. Department of Energy (DOE) National Science Bowl\textregistered: \vspace{-0.1cm}
			\begin{enumerate} \itemsep -1pt
			\item \url{http://www.scied.science.doe.gov/nsb/default.htm}
			\item ``The U.S. Department of Energy (DOE) National Science Bowl\textregistered\ is a nationwide academic competition that tests students' knowledge in all areas of science. High school and middle school students are quizzed in a fast paced question-and-answer format similar to Jeopardy. Competing teams from diverse backgrounds are comprised of four students, one alternate, and a teacher who serves as an advisor and coach.''
			\end{enumerate}
		\item Argonne National Laboratory: \vspace{-0.1cm}
			\begin{enumerate} \itemsep -1pt
			\item Division of Educational Programs: \vspace{-0.1cm}
				\begin{itemize} \itemsep -1pt
				\item Newton BBS Ask A Scientist: \url{http://www.newton.dep.anl.gov/aas.htm}
				\end{itemize}
			\end{enumerate}
		\end{enumerate}
	\item Office of Energy Efficiency and Renewable Energy (EERE): \vspace{-0.2cm}
		\begin{enumerate} \itemsep -2pt
		\item Kids Saving Energy: \vspace{-0.1cm}
			\begin{enumerate} \itemsep -1pt
			\item \url{http://www.eere.energy.gov/kids/index.html}
			\item K-12 Lesson Plans \& Activities: \url{http://www1.eere.energy.gov/education/lessonplans/}
			\item Energy Savers: \url{http://www.energysavers.gov/}
			\item Games and activities: \url{http://www.eere.energy.gov/kids/games.html}
			\item Smart home: \url{http://www.eere.energy.gov/kids/smart_home.html}
			\item About renewable energy: \url{http://www.eere.energy.gov/kids/renergy.html}
			\end{enumerate}
		\end{enumerate}
	\item Contest \& Competitions: \url{http://www.energy.gov/contests&competitions.htm}
	\end{enumerate}
\item United States Department of Defense (DoD): \vspace{-0.3cm}
	\begin{enumerate} \itemsep -2pt
	\item National Defense Education Program; Defense Advanced Research Projects Agency (DARPA): \vspace{-0.2cm}
		\begin{enumerate} \itemsep -2pt
		\item Resource for Students: \url{http://www.ndep.us/GetInvoStu.aspx}
		\item Resource for Educators: \url{http://www.ndep.us/GetInvoTea.aspx}
		\end{enumerate}
	\end{enumerate}
\item Project Lead The Way: \vspace{-0.3cm}
	\begin{enumerate} \itemsep -2pt
	\item \url{http://www.pltw.org/}
	\item Getting started: \url{http://www.pltw.org/getting-started/getting-started}
	\item Program support: \url{http://www.pltw.org/program-support/program-support}
	\item Grants available to schools and teachers: \url{http://www.pltw.org/pltw-in-the-news/grants-available-schools-teachers-and-classrooms}
	\item Students: \url{http://www.pltw.org/students/students}
	\item Educators and Administrators: \url{http://www.pltw.org/educators-administrators/educators-administrators-overview}
	\item Parents: \url{http://www.pltw.org/parents/parents}
	\end{enumerate}
\item National Science Teachers Association: \vspace{-0.3cm}
	\begin{enumerate} \itemsep -2pt
	\item \url{http://www.exploravision.org/}
	\item Science competition for K-12 students
	\end{enumerate}
\item American Mathematical Society: \vspace{-0.3cm}
	\begin{enumerate} \itemsep -2pt
	\item Some career resources for mathematics: \url{http://e-math.ams.org/samplings/samplings}
	\end{enumerate}
\item American Institute of Physics (AIP): \vspace{-0.3cm}
	\begin{enumerate} \itemsep -2pt
	\item Physics Success Stories: \url{http://www.aip.org/success/}
	\item Physics is for you; Career Services Division: \vspace{-0.2cm}
		\begin{enumerate} \itemsep -2pt
		\item \url{http://www.aip.org/careersvc/pify/}
		\item Physicists at work: \url{http://www.aip.org/careersvc/pify/yellow.html}
		\end{enumerate}
	\item Society of Physics Students (SPS): \vspace{-0.2cm}
		\begin{enumerate} \itemsep -2pt
		\item Careers Using Physics (CUP): \vspace{-0.1cm}
			\begin{enumerate} \itemsep -1pt
			\item \url{http://www.spsnational.org/cup/}
			\item Advice: \url{http://www.spsnational.org/cup/advice/index.html}
			\item Resources: \url{http://www.spsnational.org/cup/resources.html}
			\item Preparing to Teach: \url{http://www.spsnational.org/cup/teach/index.html}
			\end{enumerate}
		\end{enumerate}
	\item ComPADRE Digital Library: \vspace{-0.2cm}
		\begin{enumerate} \itemsep -2pt
		\item \url{http://www.compadre.org/}
		\item The Physics Career Resource: \url{http://www.compadre.org/careers/}
		\end{enumerate}
	\item Career guidance for high school and undergraduate students: \url{http://www.aip.org/statistics/trends/career.html}
	\item Gayle A. Buck, Jack G. Hehn, and Diandra L. Leslie-Pelecky (Editors), ``The Role of Physics Departments in Preparing K-12 Teachers,'' American Institute of Physics. Available online at: \url{http://www.aip.org/education/teacherprep/}; last accessed on January 9, 2010.
	\item American Geophysical Union: \vspace{-0.2cm}
		\begin{enumerate} \itemsep -2pt
		\item Students \& Teachers: \url{http://www.agu.org/education/students_teachers.shtml}
		\item Diversity Programs: \url{http://www.agu.org/education/diversity_programs/}
		\end{enumerate}
	\end{enumerate}
\item Institute for Operations Research and the Management Sciences (INFORMS): \vspace{-0.3cm}
	\begin{enumerate} \itemsep -2pt
	\item Career FAQ's: \url{http://www.informs.org/Build-Your-Career/INFORMS-Student-Union/Career-Center/Career-FAQ-s}
	\end{enumerate}
\item American Institute of Mathematics: \vspace{-0.3cm}
	\begin{enumerate} \itemsep -2pt
	\item Math Teachers' Circle Network: \vspace{-0.2cm}
		\begin{enumerate} \itemsep -2pt
		\item Classroom Materials: \url{http://www.mathteacherscircle.org/resources/classroommaterials.html}
		\item Helpful Resources: \url{http://www.mathteacherscircle.org/resources/general.html}
		\end{enumerate}
	\item Resources for the Math Community: \vspace{-0.2cm}
		\begin{enumerate} \itemsep -2pt
		\item \url{http://www.aimath.org/mathcommunity/}
		\item David W. Farmer, ``The AIM REU: individual projects with a common theme,'' in the {\it Proceedings of the Conference on Promoting Undergraduate Research in Mathematics}, American Mathematical Society, 2006. Available online at: \url{http://www.aimath.org/mathcommunity/farmerREU.pdf}; last accessed on January 9, 2010. [ ``AIM Research Experience for Undergraduates (REU)'' ]
		\item Sally Koutsoliotas and David W. Farmer, ``Preparing students to give talks,'' American Institute of Mathematics. Available online at: \url{http://www.aimath.org/mathcommunity/studenttalks.pdf}; last accessed on January 9, 2010. [ ``Preparing students to give talks'' ]
		\end{enumerate}
	\end{enumerate}
\item Invent Now: \vspace{-0.3cm}
	\begin{enumerate} \itemsep -2pt
	\item Camp Invention: \vspace{-0.2cm}
		\begin{enumerate} \itemsep -2pt
		\item ``Summer enrichment program for children entering grades one through six.''
		\item ``The Camp Invention program instills vital 21st century life skills such as problem-solving and teamwork through hands-on fun!''
		\item Parents: \url{http://www.invent.org/camp/parents.aspx}
		\item Teachers: \url{http://www.invent.org/camp/teachers.aspx}
		\end{enumerate}
	\end{enumerate}
\item Massachusetts Institute of Technology: \vspace{-0.3cm}
	\begin{enumerate} \itemsep -2pt
	\item MIT School of Engineering: \vspace{-0.2cm}
		\begin{enumerate} \itemsep -2pt
		\item Lemelson-MIT Program: \vspace{-0.1cm}
			\begin{enumerate} \itemsep -1pt
			\item \url{http://web.mit.edu/invent/}
			\item Invention Dimension (for children): \url{http://web.mit.edu/invent/invent-main.html}
			\end{enumerate}
		\end{enumerate}
	\end{enumerate}
\item The Lemelson Foundation: \vspace{-0.3cm}
	\begin{enumerate} \itemsep -2pt
	\item \url{http://web.mit.edu/invent/w-foundation.html}
	\item Programs \& Grants: \url{http://www.lemelson.org/programs-grants}
	\item Grantmaking: \url{http://www.lemelson.org/grantmaking}
	\end{enumerate}
\item Smithsonian Institution: \vspace{-0.3cm}
	\begin{enumerate} \itemsep -2pt
	\item Smithsonian Kids: \url{http://www.si.edu/Kids}
	\item National Museum of American History: \vspace{-0.2cm}
		\begin{enumerate} \itemsep -2pt
		\item Lemelson Center for the Study of Invention and Innovation: \vspace{-0.1cm}
			\begin{enumerate} \itemsep -1pt
			\item \url{http://inventionatplay.org/index.html}
			\item Resources: \url{http://inventionatplay.org/resources.html}
			\end{enumerate}
		\end{enumerate}
	\end{enumerate}
%%%%%%%%%%%%%%%%%%%%%%%%%%%%%%%%%%%%%%%%
%%%%%%%%%%%%%%%%%%%%%%%%%%%%%%%%%%%%%%%%
\item Scholarships: \vspace{-0.3cm}
	\begin{enumerate} \itemsep -2pt
	\item IEEE Presidents' Scholarship: \url{http://www.ieee.org/education_careers/education/preuniversity/scholarship.html}
	\item ACM/SIGDA {\it P. O. Pistilli scholarship}: \vspace{-0.1cm}
		\begin{enumerate} \itemsep -1pt
		\item Supported by the Design Automation Conference which ACM/SIGDA sponsors, the objective of the P. O. Pistilli Scholarship is to increase the pool of professionals in Electrical Engineering and Computer Science from underrepresented groups (Women, African American, Hispanic, American Indian, and Disabled).
		\item Scholarships of \$4000 per year, renewable for up to 5 years, are awarded annually to 2-7 high school seniors from the above mentioned under represented groups who have a 3.00 GPA or better (on a 4.00 scale), have demonstrated high achievement in math and science courses, have expressed a strong desire to pursue careers in electrical engineering, computer engineering, or computer science, and who have demonstrated substantial financial need.
		\item U.S. citizenship is not required, but applicants must be U.S. residents when they apply and must plan to attend an accredited US college or university.
		\item \url{http://www.sigda.org/pistilli.html}
		\end{enumerate}
	\item Engineering Education Service Center (EESC): \url{http://www.engineeringedu.com/scholars.html}
	\item ASME-ASME Auxiliary FIRST Clarke Scholarships: \url{http://www.asme.org/Education/College/FinancialAid/High_School_Seniors.cfm} and \url{http://www.asme.org/Education/College/FinancialAid/Auxiliary_FIRST_Clarke.cfm}
	\item International Petroleum Institute�s High School Scholarships (for individuals entering a college program in engineering): \url{http://www.asme-ipti.org/public/pagscholarshipprograms.aspx}
	\item American Institute of Chemical Engineers (AIChE): \vspace{-0.2cm}
		\begin{enumerate} \itemsep -2pt
		\item Fuels and Petrochemicals Division Scholarship (for high school students entering undergraduate programs in engineering or science that are related to fuels and petrochemicals): \url{http://www.aiche.org/Students/Awards/F_PDScholarship.aspx}
		\item Minority Scholarship Awards for Incoming College Freshmen (for underrepresented minorities entering an undergraduate chemical engineering program): \url{http://www.aiche.org/Students/Awards/MinorityScholarshipAwardsIncomingFreshmen.aspx}
		\end{enumerate}
	\item Sallie Mae Fund: \vspace{-0.3cm}
		\begin{enumerate} \itemsep -2pt
		\item \url{http://www.thesalliemaefund.org/smfnew/index.html}
		\item List of scholarship resources: \url{http://www.thesalliemaefund.org/smfnew/sections/search.html}
		\item Top 10 Tips for Planning and Paying for College: \url{http://www.thesalliemaefund.org/smfnew/fin_aid/index.html}
		\item Scholarships: \url{http://www.thesalliemaefund.org/smfnew/scholarship/index.html} and \url{http://www.thesalliemaefund.org/smfnew/sections/apply.html}
		\item Important information for parents about saving for college and getting financial aid: \vspace{-0.2cm}
			\begin{enumerate} \itemsep -2pt
			\item \url{http://www.thesalliemaefund.org/smfnew/sections/download.html}
			\item This information is also available in Spanish. Summaries are also available in other languages such as: \vspace{-0.1cm}
				\begin{itemize} \itemsep -1pt
				\item French
				\item German
				\item Italian
				\item Korean
				\item Russian
				\item Simplified and Traditional Chinese
				\item Tagalog
				\item Vietnamese
				\end{itemize}
			\item Top 10 Tips for Planning and Paying for College: \url{http://www.thesalliemaefund.org/smfnew/fin_aid/index.html}
			\end{enumerate}
		\item Kids2College program: \url{http://www.thesalliemaefund.org/smfnew/initiatives/kidscollege.html}
		\item For African-American individuals entering college: \vspace{-0.2cm}
			\begin{enumerate} \itemsep -2pt
			\item Black College Dollars: \url{http://www.thesalliemaefund.org/smfnew/scholarship_directory/index.html}
			\item \url{http://www.thesalliemaefund.org/smfnew/initiatives/aa.html}
			\end{enumerate}
		\item For Hispanic Americans, or Latinos/Latinas: \vspace{-0.2cm}
			\begin{enumerate} \itemsep -2pt
			\item \url{http://www.thesalliemaefund.org/smfnew/pdf/Scholarship_Directory.pdf}
			\item Latino College Dollars: \url{http://www.latinocollegedollars.org/}
			\end{enumerate}
		\end{enumerate}
	\item {\it American Chemical Society}: \vspace{-0.3cm}
		\begin{enumerate} \itemsep -2pt
		\item ACS Scholars Program (for underrepresented minorities in, or entering, an undergraduate program in chemistry, biochemistry, or chemical engineering): \url{http://portal.acs.org/portal/acs/corg/content?_nfpb=true&_pageLabel=PP_SUPERARTICLE&node_id=1650&use_sec=false&sec_url_var=region1&__uuid=b3b583cf-18ae-4fb0-9375-33f75a0ccf49}
		\item Project SEED Scholarships (for high school seniors who have worked at least one summer at a science institute under the Project SEED program): \url{http://portal.acs.org/portal/acs/corg/content?_nfpb=true&_pageLabel=PP_SUPERARTICLE&node_id=2031&use_sec=false&sec_url_var=region1&__uuid=99bc6a62-3e78-4b2a-be3f-50b28f7ff265}
		\end{enumerate}
	\item The Posse Foundation: \url{http://www.possefoundation.org/}
	\item Hispanic Scholarship Fund (HSF) scholarship programs for high school students: \url{http://www.hsf.net/innerContent.aspx?id=426}
	\item Asian \& Pacific Islander American Scholarship Fund (APIASF): scholarships for individuals entering college as freshmen; see \url{http://www.apiasf.org/scholarship_apiasf.html}
	\item Nationally Coveted College Scholarships, Graduate School Fellowships \& Postdoctoral Awards: \url{http://scholarships.fatomei.com/}
	\item {\it SPIE} Scholarship Program (for high school students entering college to study optics, photonics, imaging, optoelectronics, or related program): \url{http://spie.org//x1733.xml?WT.svl=mddm14}
	\item Susan G. Komen for the Cure\textregistered: The Komen College Scholarship Program, \url{http://ww5.komen.org/ResearchGrants/CollegeScholarshipAward.html}
	\item National Society of Professional Engineers's list of scholarships for high school students: \url{http://www.nspe.org/Students/Scholarships/index.html}
	\item AWM Essay Contest: Biographies of Contemporary Women in Mathematics; see \url{http://www.awm-math.org/biographies/contest.html}
	\item National Engineers Week Future City Competition (students from $6^{th}$--$8^{th}$ grades): \url{http://www.futurecity.org/}
	\item National Ocean Sciences Bowl: \vspace{-0.2cm}
		\begin{enumerate} \itemsep -2pt
		\item \url{http://www.nosb.org/ocean-careers/}
		\item National Ocean Scholar Program (for high school seniors who are current/past participants of the Bowl, and are seeking a career in the ocean sciences or a marine-related field): \url{http://www.nosb.org/ocean-careers/national-ocean-scholar-program/}
		\end{enumerate}
	\item National Center for Women \& Information Technology (NCWIT): \vspace{-0.2cm}
		\begin{enumerate} \itemsep -2pt
		\item NCWIT Award for Aspirations in Computing (for young women in high school): \url{http://www.ncwit.org/work.awards.aspiration.html}
		\end{enumerate}
	\end{enumerate}
%%%%%%%%%%%%%%%%%%%%%%%%%%%%%%%%%%%%%%%%
%%%%%%%%%%%%%%%%%%%%%%%%%%%%%%%%%%%%%%%%
\item Resources for teachers/educators: \vspace{-0.3cm}
	\begin{enumerate} \itemsep -2pt
	\item Google: \vspace{-0.2cm}
		\begin{enumerate} \itemsep -2pt
		\item Google Teacher Academy (for teachers to learn how to use Google technologies to facilitate teaching): \url{http://www.google.com/educators/gta.html}
		\item Classroom activities (suggestions): \url{http://www.google.com/educators/activities.html}
		\end{enumerate}
	\item IEEE Teacher In-Service Program (TISP): \vspace{-0.2cm}
		\begin{enumerate} \itemsep -2pt
		\item \url{http://www.ieee.org/education_careers/education/preuniversity/tispt/index.html}
		\item Lesson Plans for Pre-university Instructors: \url{http://www.ieee.org/education_careers/education/preuniversity/resources/index.html}
		\end{enumerate}
	\item Global Challenge Award: \url{http://www.globalchallengeaward.org/display/public/Home}
	\item Teachers' Domain (to teach students about science, engineering, and the arts): \url{http://www.teachersdomain.org/}
	\item {\it TeachEngineering} digital library: \vspace{-0.2cm}
		\begin{enumerate} \itemsep -2pt
		\item The {\it TeachEngineering} digital library provides teacher-tested, standards-based engineering content for K-12 teachers engineering content for K12 teachers to use in science and math classrooms. Engineering lessons connect real-world experiences with curricular content already taught in K-12 classrooms. Mapped to educational content standards, {\it TeachEngineering}'s comprehensive curricula are hands-on, free, and relevant to children's daily lives.
		\item \url{http://www.teachengineering.com/index.php}
		\end{enumerate}
	\item Engineering Pathway: \url{http://www.engineeringpathway.com/ep/index.jhtml}
	\item {\it American Society of Mechanical Engineers, ASME}: \url{http://www.asme.org/Education/PreCollege/TeacherResources/}
	\item {\it National Science Foundation} resources for the K-12 classroom: \url{http://nsf.gov/news/classroom/engineering.jsp}
	\item {\it NASA}: \url{http://www.nasa.gov/audience/foreducators/index.html}
	\item The Mathematical Association of America: \vspace{-0.2cm}
		\begin{enumerate} \itemsep -2pt
		\item Pre-College Programs: \url{http://www.maa.org/funding/pre_college.html}. Also, see \url{http://www.maa.org/funding/undergraduate.html}.
		\item Special Interest Group of the Mathematical Association of America on the use of the World-Wide Web in Undergraduate Mathematics Instruction (Web SIGMAA). Available at: \url{http://math.chapman.edu/websigmaa/index.php/Main_Page}; last accessed on September 2, 2010.
		\item SIGMAA TAHSM (Teaching Advanced High School Mathematics). Available at: \url{http://sigmaa.maa.org/tahsm/}; last accessed on September 2, 2010.
		\item Special Interest Group on Statistics Education: \url{http://sigmaa.maa.org/stat-ed/}
		\end{enumerate}
	\item Math for America: \vspace{-0.2cm}
		\begin{enumerate} \itemsep -2pt
		\item M$f$A Master Teacher Fellowship program: \vspace{-0.1cm}
			\begin{enumerate} \itemsep -1pt
			\item The Math for America Master Teacher Fellowship program rewards exceptional public secondary school math teachers with a four-year Fellowship.
			\item M$f$A Master Teacher Fellowships are currently available in Berkeley, Boston and New York City.
			\item \url{http://www.mathforamerica.org/web/guest/master-teachers}
			\end{enumerate}
		\item M$f$A Early Career Fellows: \vspace{-0.1cm}
			\begin{enumerate} \itemsep -1pt
			\item The Math for America Early Career Fellowship is awarded to public secondary school math teachers early in their careers.
			\item The M$f$A Early Career Fellowship requires a commitment of four years and is available in New York City. 
			\item \url{http://www.mathforamerica.org/early-career-fellows}
			\end{enumerate}
		\item M$f$A Fellows: \vspace{-0.1cm}
			\begin{enumerate} \itemsep -1pt
			\item \url{http://www.mathforamerica.org/web/guest/mfa-fellows}
			\end{enumerate}
		\item Teachers resources: \url{http://www.mathforamerica.org/web/guest/teacher-resources} and \url{http://www.mathforamerica.org/teacher-resources/classroom} (classroom resources)
		\item Resources for professional development (teachers): \url{http://www.mathforamerica.org/teacher-resources/professional}
		\item \url{http://www.mathforamerica.org/home}
		\end{enumerate}
	\item Association for Symbolic Logic (ASL): \vspace{-0.2cm}
		\begin{enumerate} \itemsep -2pt
		\item Guidelines on Logic Education: \url{http://www.ucalgary.ca/aslcle/guidelines}
		\item Educational Logic Software: \url{http://www.ucalgary.ca/aslcle/logic-courseware}
		\end{enumerate}
	\item Consortium for Ocean Leadership: \vspace{-0.2cm}
		\begin{enumerate} \itemsep -2pt
		\item Educational Resources: \url{http://www.oceanleadership.org/gulf-oil-spill/educational-resources/}
		\item The JOIDES Resolution (The JR) scientific research vessel [ Deep Earth Academy ]: \vspace{-0.1cm}
			\begin{enumerate} \itemsep -1pt
			\item Teacher Resources (to teach students about geology and physical geography): \url{http://joidesresolution.org/node/46}
			\item Teachers at Sea/On-board Education Officer (for teachers to go on scientific expeditions on board): \url{http://joidesresolution.org/node/453}
			\end{enumerate}
		\item Integrated Ocean Drilling Program (IODP) -- IODP United States Implementing Organization (IODP-USIO): \vspace{-0.1cm}
			\begin{enumerate} \itemsep -1pt
			\item Teaching Materials: \url{http://www.iodp-usio.org/Education/educ.html}
			\end{enumerate}
		\item Deep Earth Academy (includes suggested ``curriculum and classroom activities for kindergarten through college level''): \vspace{-0.1cm}
			\begin{enumerate} \itemsep -1pt
			\item \url{http://www.oceanleadership.org/education/deep-earth-academy/}
			\item For Educators: \url{http://www.oceanleadership.org/education/deep-earth-academy/educators/}
			\end{enumerate}
		\end{enumerate}
	\item Virginia Institute of Marine Science (College of William and Mary): \vspace{-0.2cm}
		\begin{enumerate} \itemsep -2pt
		\item Bridge Ocean Education Teacher Resource Center: \url{http://web.vims.edu/bridge/?svr=www#}
		\end{enumerate}
	\item American Geological Institute: \vspace{-0.2cm}
		\begin{enumerate} \itemsep -2pt
		\item Awards for teachers: \url{http://www.agiweb.org/education/awards/index.html}
		\item Edward C. Roy, Jr. Award For Excellence in K-8 Earth Science Teaching (for middle school teachers in the US who are teaching earth science): \url{http://www.agiweb.org/education/awards/ed-roy/}
		\item Presidential Awards for Excellence in Mathematics \& Science Teaching, PAEMST (for kindergarten and K-12 teachers in the US who are teaching students about STEM fields): \url{http://www.agiweb.org/education/awards/paemst.html}
		\item National Association of Geoscience Teachers (NAGT) Outstanding Earth Science Teacher Award: \url{http://www.agiweb.org/education/awards/nagt.html}
		\item American Association of Petroleum Geologists' (AAPG) National Earth Science Teacher of the Year Award: \url{http://www.agiweb.org/education/awards/aapg.html}
		\item Curriculum Materials and Activities: \url{http://www.agiweb.org/education/curriculum/index.html}
		\item K-12 Professional Development Programs: \url{http://www.agiweb.org/education/pd/index.html}
		\item Educational Resources: \url{http://www.agiweb.org/education/resource/index.html}
		\end{enumerate}
	\item Institute for Broadening Participation: \vspace{-0.2cm}
		\begin{enumerate} \itemsep -2pt
		\item PathwaysToScience.org: \vspace{-0.1cm}
			\begin{enumerate} \itemsep -1pt
			\item For K-12 teachers (resources to encourage students to be interested in STEM): \url{http://www.pathwaystoscience.org/Teachers.asp}
			\end{enumerate}
		\end{enumerate}
	\item National Science Foundation: \vspace{-0.2cm}
		\begin{enumerate} \itemsep -2pt
		\item The National Science Digital Library (NSDL): \vspace{-0.1cm}
			\begin{enumerate} \itemsep -1pt
			\item Resources for K-12 Teachers: \url{http://nsdl.org/resources_for/k12_teachers/}
			\end{enumerate}
		\end{enumerate}
	\item National Academy of Engineering, NAE: \vspace{-0.2cm}
		\begin{enumerate} \itemsep -2pt
		\item NAE Grand Challenges: \vspace{-0.1cm}
			\begin{enumerate} \itemsep -1pt
			\item Includes a list of NAE Grand Challenges, which indicate some of the challenges faced by people on a global scale that can be partially solved by engineers. This can be used to get children and youths to be excited about engineering. 
			\item NAE Grand Challenges: \vspace{-0.1cm}
				\begin{itemize} \itemsep -1pt
				\item Make solar energy economical
				\item Provide energy from fusion
				\item Develop carbon sequestration methods
				\item Manage the nitrogen cycle
				\item Provide access to clean water
				\item Restore and improve urban infrastructure
				\item Advance health informatics
				\item Engineer better medicines
				\item Reverse-engineer the brain
				\item Prevent nuclear terror
				\item Secure cyberspace
				\item Enhance virtual reality
				\item Advance personalized learning
				\item Engineer the tools of scientific discovery
				\end{itemize}
			\item \url{http://www.engineeringchallenges.org/}
			\end{enumerate}
		\item NAE Grand Challenge K12 Partners Program: \vspace{-0.1cm}
			\begin{enumerate} \itemsep -1pt
			\item Can be used by schools/teachers to raise awareness of global challenges among students and to encourage students to plan career paths to tackle these challenges
			\item 5-Part Make it Happen Plan (includes suggested activities for students in elementary school to learn about basic science and engineering concepts that are relevant to solve the NAE grand challenges): \url{http://www.grandchallengek12.org/5-part-plan}
			\item \url{http://www.grandchallengek12.org/about}
			\end{enumerate}
		\item {\it National Academy of Engineering}'s technological literacy program for people (students, parents, and educators) to learn more about technology: \url{http://www.nae.edu/nae/techlithome.nsf}
		\end{enumerate}
	\item Women in Technology (WIT): \vspace{-0.2cm}
		\begin{enumerate} \itemsep -2pt
		\item Girls In Technology (GIT): \vspace{-0.1cm}
			\begin{enumerate} \itemsep -1pt
			\item Get Involved: \vspace{-0.1cm}
				\begin{itemize} \itemsep -1pt
				\item \url{http://www.girlsintechnology.org/getinvolved.cfm}
				\item Teacher: teach girls about IT as an after-school activity or in a summer camp session
				\item Assistant Teacher: Assist instructors in GIT sessions, after-school activities, or summer camp sessions
				\item Develop Curriculum: Develop a curriculum for a supported GIT educational program
				\item Mentor: Mentor a girl in one of [GIT's] supported programs
				\item Job Shadow: ``Let a girl shadow you at work''
				\item Guest Speaker: ``Speak to a group of girls on a topic both you and they enjoy, such as computers, technology, education, how to take apart computers, how to build a web site, etc.''
				\end{itemize}
			\end{enumerate}
		\end{enumerate}
	\item Organization for Economic Co-operation and Development (OECD): \vspace{-0.2cm}
		\begin{enumerate} \itemsep -2pt
		\item Programme for International Student Assessment (PISA): \vspace{-0.1cm}
			\begin{enumerate} \itemsep -1pt
			\item {\it PISA 2009 Results}. Available online at: \url{http://www.oecd.org/document/61/0,3343,en_32252351_32235731_46567613_1_1_1_1,00.html}; last accessed on December 10, 2010. [ Includes suggestions to improve learning outcomes, as well as education policies and practices. ]
			\end{enumerate}
		\end{enumerate}
	\item American Institute of Aeronautics and Astronautics (AIAA): \vspace{-0.2cm}
		\begin{enumerate} \itemsep -2pt
		\item K-12 Educators: \url{http://www.aiaa.org/content.cfm?pageid=208}
		\end{enumerate}
	\item Research Councils UK (RCUK): \vspace{-0.2cm}
		\begin{enumerate} \itemsep -2pt
		\item Biotechnology and Biological Sciences Research Council (BBSRC): \vspace{-0.1cm}
			\begin{enumerate} \itemsep -1pt
			\item Resources for schools and young people: \url{http://www.bbsrc.ac.uk/society/schools/schools-index.aspx}
			\item Teaching resources: publications and web-based activities: \vspace{-0.1cm}
				\begin{itemize} \itemsep -1pt
				\item Primary (ages 5-12) resources: \url{http://www.bbsrc.ac.uk/society/schools/primary/primary-index.aspx}
				\item Secondary (ages 12-16) and post-16 resources: \url{http://www.bbsrc.ac.uk/society/schools/secondary/secondary-index.aspx}
				\end{itemize}
			\end{enumerate}
		\end{enumerate}
	\item Nuffield Foundation: \vspace{-0.2cm}
		\begin{enumerate} \itemsep -2pt
		\item Education: \url{http://www.nuffieldfoundation.org/education}
		\item Teachers: \vspace{-0.1cm}
			\begin{enumerate} \itemsep -1pt
			\item (Excellent) resources in science and mathematics: \url{http://www.nuffieldfoundation.org/teachers}
			\item \url{http://www.nuffieldfoundation.org/teachers-0}
			\end{enumerate}
		\end{enumerate}
	\item Wellcome Trust: \vspace{-0.2cm}
		\begin{enumerate} \itemsep -2pt
		\item Education resources: \url{http://www.wellcome.ac.uk/Education-resources/index.htm}
		\item {\it yourgenome.org}: \vspace{-0.1cm}
			\begin{enumerate} \itemsep -1pt
			\item \url{http://www.yourgenome.org/}
			\item Resources for teachers about genomics: \url{http://www.yourgenome.org/landing_teachers.shtml}
			\end{enumerate}
		\item Network of Science Learning Centers (Science Learning Centers): \vspace{-0.1cm}
			\begin{enumerate} \itemsep -1pt
			\item \url{https://www.sciencelearningcentres.org.uk/}
			\item Awards and Bursaries: \vspace{-0.1cm}
				\begin{itemize} \itemsep -1pt
				\item \url{https://www.sciencelearningcentres.org.uk/centres/national/awards-and-bursaries}
				\item \url{https://www.sciencelearningcentres.org.uk/about/impact-awards}
				\end{itemize}
			\item Resource collections: \url{https://www.sciencelearningcentres.org.uk/resources}
			\item Curriculum resources for primary, secondary, and tertiary education: \url{https://www.sciencelearningcentres.org.uk/curriculum}
			\end{enumerate}
		\end{enumerate}
	\end{enumerate}
%%%%%%%%%%%%%%%%%%%%%%%%%%%%%%%%%%%%%%%%
%%%%%%%%%%%%%%%%%%%%%%%%%%%%%%%%%%%%%%%%
\item Underrepresented minorities: \vspace{-0.3cm}
	\begin{enumerate} \itemsep -2pt
	\item University of Washington: \vspace{-0.2cm}
		\begin{enumerate} \itemsep -2pt
		\item Department of Computer Science and Engineering: \vspace{-0.1cm}
			\begin{enumerate} \itemsep -1pt
			\item {\it AccessComputing}: \vspace{-0.1cm}
				\begin{itemize} \itemsep -1pt
				\item \url{http://www.washington.edu/accesscomputing/}
				\item Has resources to help students with disabilities to pursue ``undergraduate and graduate degrees and careers in computing fields''.
				\item It runs the ``Summer Academy for Advancing Deaf \& Hard of Hearing in Computing'' for youths who are hearing impaired: \url{http://www.washington.edu/accesscomputing/dhh/academy/index.html}
				\end{itemize}
			\end{enumerate}
		\end{enumerate}
	%%%%%%%%%%%%%%%%%%%%%%%%%
	\item Engineer Girl: \vspace{-0.2cm}
		\begin{enumerate} \itemsep -2pt
		\item Resources for students, parents, and teachers to encourage girls to explore careers and educational opportunities in engineering
		\item Created by the National Academy of Sciences and The National Academy of Engineering
		\item Contests for K-12 students: \url{http://www.engineergirl.org/?id=4436}
		\item \url{http://www.engineergirl.org/}
		\end{enumerate}
	\item Engineering Your Life: \url{http://www.engineeryourlife.org/}
	\item GirlGeeks: \url{http://www.girlgeeks.org/}
	\item {\it Women in Science, Technology, Engineering, and Mathematics ON THE AIR!}: \vspace{-0.2cm}
		\begin{enumerate} \itemsep -2pt
		\item Audio resources that describe stories about women in science, technology, engineering, and mathematics (STEM) fields
		\item \url{http://www.womeninscience.org/}
		\end{enumerate}
	\item {\it Women Scientists in History}: \url{http://www.hypatiamaze.org/}
	\item Association for Women in Mathematics (AWM): \vspace{-0.2cm}
		\begin{enumerate} \itemsep -2pt
		\item \url{http://www.awm-math.org/}
		\item Education: \vspace{-0.1cm}
			\begin{enumerate} \itemsep -1pt
			\item \url{http://sites.google.com/site/awmmath/awm-resources/education}
			\item Includes information for students in middle school, high school, college and university (including graduate students). It also includes information for parents and teachers/educators.
			\end{enumerate}
		\item Women in Math, Science, and Society: \url{http://sites.google.com/site/awmmath/women-in-math-science-and-society}
		\item Essay contest on biographies of contemporary women in mathematics: \url{http://sites.google.com/site/awmmath/programs/essay-contest}
		\end{enumerate}
	\item Women in Technology (WIT): \vspace{-0.2cm}
		\begin{enumerate} \itemsep -2pt
		\item Girls in Technology: \vspace{-0.1cm}
			\begin{enumerate} \itemsep -1pt
			\item \url{http://www.girlsintechnology.org/}
			\item WIT Education Foundation: provides educational programs for girls in technology
			\item TeamBusiness Fundraiser: ``A combined fundraiser and program for girls in Grades 9-12 across the Metro DC area. Each year, up to forty girls participate with mentors and WIT volunteers in a full-day business simulation workshop conducted by TeamBusiness USA. The teams competed as companies, learning how to run a technology company in a fun and exciting simulation environment.''
			\item Hispanic Youth Foundation: ``In 2005, GIT established a partnership with the Hispanic Youth Foundation (HYF) and provided a grant to fund HYF�s innovative Laptops for Learning Dollars program, providing laptops and Internet connections for elementary and middle school students and their families in Arlington County and the City of Manassas.''
			\item Empower Girls -- CLCP Clubs: ``Empower Girls after-school programs were held at Hybla Valley Elementary School and Sacramento Community Center. GIT/WITEF provided funding to run these programs in conjunction with the Fairfax County Computer Learning Center Partnership (CLCP). The selected centers serve economically challenged communities in Fairfax County.''
			\end{enumerate}
		\end{enumerate}
	%%%%%%%%%%%%%%%%%%%%%%%%%
	\item National Society of Black Engineers (NSBE) competitions for high school/K-12 students: \url{http://www.nsbe.org/Programs/Competitions/NSBE-Jr-.aspx}
	\item The Society of Mexican American Engineers and Scientists (MAES): MAES PreCollege Outreach Programs, \url{http://www.maes-natl.org/index.php?module=ContentExpress&func=display&ceid=16&meid=236}
	\item {\it Center for the Advancement of Hispanics in Science and Engineering Education} (CAHSEE): \vspace{-0.2cm}
		\begin{enumerate} \itemsep -2pt
		\item STEM - The Science, Technology, Engineering \& Mathematics Institute (for students from grades 5 through 11): \url{http://www.cahsee.org/2programs/stem.asp.htm}
		\item YEP - Young Educators Program (fellows would learn how to train students in the aforementioned STEM Institute): \url{http://www.cahsee.org/2programs/yep.asp.htm}
		\item CAYSA - Central American Young Scholar Awards: \url{http://www.cahsee.org/2programs/caysa.asp.htm}. ``The CAYSA ceremonies honor more than 60 Washington, D.C. area high school seniors of Central American descent who have demonstrated remarkable success throughout all four years of high school. Students must be of Central American descent and have at least a 3.0 gpa.''
		\item Scholarships: \url{http://www.cahsee.org/6resources/scholarships.asp.htm}
		\item \url{http://www.cahsee.org/about/about.asp.htm}
		\end{enumerate}
	%%%%%%%%%%%%%%%%%%%%%%%%%
	\item International Computer Science Institute (UC Berkeley): \vspace{-0.2cm}
		\begin{enumerate} \itemsep -2pt
		\item Berkeley Foundation for Opportunities in Information Technology, BFOIT: \vspace{-0.1cm}
			\begin{enumerate} \itemsep -1pt
			\item BFOIT Programs for women and underrepresented minorities (African Americans and Chicanos/Latinos) in middle/high school who are interested in electrical/computer engineering and computer science careers: \url{http://www.bfoit.org/programs.html}
			\end{enumerate}
		\end{enumerate}
	\item Institute for Broadening Participation: \vspace{-0.2cm}
		\begin{enumerate} \itemsep -2pt
		\item PathwaysToScience.org: \vspace{-0.1cm}
			\begin{enumerate} \itemsep -1pt
			\item PathwaysToScience.org is a portal website supporting pathways to the STEM fields: science, technology, engineering, and mathematics.
			\item Particular emphasis is placed on connecting traditionally underrepresented groups with STEM programs and resources, including funding and mentoring opportunities. 
			\item For K-12 students: \url{http://www.pathwaystoscience.org/K12.asp}
			\item STEM Resources by Institution (colleges, universities, and US national research laboratories): \url{http://www.pathwaystoscience.org/Institution.asp}
			\item profiles of people and programs in STEM: \vspace{-0.3cm}
				\begin{itemize} \itemsep -2pt
				\item \url{http://www.pathwaystoscience.org/Profiles.asp}
				\item Find out about the career paths of underrepresented minorities in STEM
				\item Find out about programs that are offered by institutions for underrepresented minorities in STEM
				\end{itemize}
			\item Directory of partners (organizations that cooperate with or support the Institute for Broadening Participation): \url{http://www.pathwaystoscience.org/Partners.asp}
			\item Additional resources: \url{http://www.pathwaystoscience.org/Ideaexchange.asp}
			\end{enumerate}
		\item Maine Pathways to STEM (Science, Technology, Engineering \& Mathematics): \vspace{-0.1cm}
			\begin{enumerate} \itemsep -1pt
			\item \url{http://www.mainestem.org/}
			\item K-12 Teachers \& University Faculty: \url{http://www.mainestem.org/METeachersFaculty.asp}
			\item K-12 STEM Resources: \url{http://www.mainestem.org/MEK12.asp}
			\end{enumerate}
		\end{enumerate}
	\item Building Engineering and Science Talent, BEST: \vspace{-0.2cm}
		\begin{enumerate} \itemsep -2pt
		\item \url{http://www.bestworkforce.org/}
		\item Publications: \url{http://www.bestworkforce.org/publications.htm}
		\item List of programs to help underrepresented minority students in K-12 schools explore careers in STEM: \url{http://www.bestworkforce.org/links.htm}
		\end{enumerate}
	\item American Indian Science and Engineering Society (AISES): \vspace{-0.2cm}
		\begin{enumerate} \itemsep -2pt
		\item Pre-college programs: \vspace{-0.1cm}
			\begin{enumerate} \itemsep -1pt
			\item \url{http://www.aises.org/Programs}
			\item Resources: \url{http://www.aises.org/Programs/Resources}
			\end{enumerate}
		\end{enumerate}
	\end{enumerate}
\end{enumerate}







%%%%%%%%%%%%%%%%%%%%%%%%%%%%%%%%%%%%%%%%%%%
\subsection{Science \& Engineering Outreach for Undergraduates, Grad Students, \& Postdocs}
\label{stemoutreachcollegegradsch}


Science, mathematics, and engineering outreach to undergraduates, graduate students, and postdocs: \vspace{-0.3cm}
\begin{enumerate} \itemsep -4pt
\item Mac Hyman, ``Good Choices for Great Careers in the Mathematical Sciences,'' talk given at 2008 SIAM Annual Meeting. Available at: \url{http://client.blueskybroadcast.com/siam08/hyman/index.html}; last accessed on August 25, 2010. Also, see \url{http://meetings.siam.org/program.cfm?CONFCODE=AN08}, \url{http://www.siam.org/meetings/an08/program.php}, and \url{http://www.siam.org/meetings/an08/}.
\item {\it Accreditation.org}: \vspace{-0.3cm}
	\begin{enumerate} \itemsep -2pt
	\item Information about the accreditation of engineering degree programs around the world
	\item \url{http://www.accreditation.org/}
	\end{enumerate}
\item John Baez, ``How to Learn Math and Physics,'' Department of Mathematics, University of California, Riverside, December 24, 2007. Available at: \url{http://math.ucr.edu/home/baez/books.html}; last accessed on August 28, 2010.
\item {\it MentorNet}: \vspace{-0.3cm}
	\begin{enumerate} \itemsep -2pt
	\item \url{http://www.mentornet.net/}
	\item Enables people to network with scientists, engineers, and professors in Science, Technology, Engineering, and Mathematics (STEM)
	\item Is very supportive of minorities, so that more minorities (particularly underrepresented minorities) can be attracted to STEM careers
	\end{enumerate}
\item {\it The Indus Entrepreneurs (TiE)} for networking among high-tech entrepreneurs, start-up co-founders, venture capitalists, and angel investors: \url{http://www.tie.org/}
\item National Academy of Engineering, NAE: \vspace{-0.3cm}
	\begin{enumerate} \itemsep -2pt
	\item Includes a list of NAE Grand Challenges, which can provide some suggestions for research trajectories
	\item Summit Series on the Grand Challenges: Includes the National Grand Challenges Summits
	\item \url{http://www.engineeringchallenges.org/}
	\end{enumerate}
\item {\it National Society of Professional Engineers}: \vspace{-0.3cm}
	\begin{enumerate} \itemsep -2pt
	\item Student Resources: \vspace{-0.2cm}
		\begin{enumerate} \itemsep -2pt
		\item \url{http://www.nspe.org/Students/Resources/index.html}
		\item An Employment Guidelines Checklist for the Engineer Job Applicant: \url{http://www.nspe.org/Students/Resources/checklist.html}
		\end{enumerate}
	\item Career Center: \url{http://www.nspe.org/CareerCenter/index.html}
	\item A Sightseer's Guide to Engineering: \url{http://www.engineeringsights.org/}
	\end{enumerate}
\item {\it JustGarciaHill} ``Study Skills for Budding Scientists'': \url{http://www.justgarciahill.org/index.php/science-study-skills.html}
\item {\it NASA} resources for students: \vspace{-0.3cm}
	\begin{enumerate} \itemsep -2pt
	\item \url{http://www.nasa.gov/audience/forstudents/index.html}
	\item NASA University Student Launch Initiative, or USLI: \url{http://www.nasa.gov/offices/education/programs/descriptions/University_Student_Launch_Initiative.html}
	\end{enumerate}
\item {\it iTunes U}: \vspace{-0.3cm}
	\begin{enumerate} \itemsep -2pt
	\item {\it iTunes} is required to listen to or watch these lectures, talks, and presentations.
	\item Access {\it iTunes U} at: \url{http://www.apple.com/education/itunes-u/} or \url{http://deimos3.apple.com/indigo/main/main.html?v0=WWW-AMUS-ITUNESU070521-N48LX}
	\item {\it iTunes U} is a set of webcast and podcasts, where you can easily find audio and video clips for lectures, seminars, announcements, virtual tours, and so on. For example, some professors from schools like MIT or Berkeley will provide audio/video clips of their lectures on {\it iTunes U}.
	\item This can help in exploring different majors before a college student declares her/his major(s). If a student is not sure if she/he wants to double major in deaf studies and linguistics, this student can check out some linguistics lectures from her/his (preferred) college/university, if it uses {\it iTunes U}, or those from other universities.
	\end{enumerate}
\item Harvey Mudd College: \vspace{-0.3cm}
	\begin{enumerate} \itemsep -2pt
	\item Francis Edward Su, {\it Math Fun Facts!}, Department of Mathematics, Harvey Mudd College: \url{http://www.math.hmc.edu/funfacts/}
	\end{enumerate}
\item Engineering Pathway: \url{http://www.engineeringpathway.com/ep/index.jhtml}
\item Rochester Institute of Technology, ``Biology \& Biotechnology Paid Co-op/Internships for 2011,'' Department of Biological Sciences, Rochester Institute of Technology: \url{http://people.rit.edu/gtfsbi/Symp/summer.htm}
\item {\it Mathematical Association of America (MAA)} information on educational pathways and career opportunities: \vspace{-0.3cm}
	\begin{enumerate} \itemsep -2pt
	\item Undergraduate Students: \url{http://www.maa.org/students/undergrad/}
	\item Graduate Students: \url{http://www.maa.org/students/grad/}
	\item Underrepresented Groups: \url{http://www.maa.org/programs/underrep.html}
	\item Mathematical Association of America (MAA) MathFest (for students in mathematics): \url{http://www.maa.org/mathfest/}
	\item MAA Online Columns: \url{http://www.maa.org/news/columns.html}
	\end{enumerate}
\item New Zealand Institute of Mathematics and its Applications (NZIMA): \vspace{-0.3cm}
	\begin{enumerate} \itemsep -2pt
	\item {\it MathsReach}: Careers (information about careers based on a higher education in mathematics or related field): \url{http://www.mathsreach.org/Careers}
	\end{enumerate}
\item {\it Engineers Dedicated to a Better Tomorrow (a.k.a., DedicatedEngineers)}: \vspace{-0.3cm}
	\begin{enumerate} \itemsep -2pt
	\item [Resources for] College Students and Faculty/Staff Members: \url{http://www.dedicatedengineers.org/intro_for_college.htm}
	\item \url{http://www.dedicatedengineers.org/}
	\end{enumerate}
\item American Institute of Physics: \vspace{-0.3cm}
	\begin{enumerate} \itemsep -2pt
	\item GradschoolShopper.com: \vspace{-0.2cm}
		\begin{enumerate} \itemsep -2pt
		\item \url{http://www.gradschoolshopper.com/}
		\item ``Find information on graduate programs in physics, astronomy, and other physical sciences''
		\end{enumerate}
	\item Career guidance for high school and undergraduate students: \url{http://www.aip.org/statistics/trends/career.html}
	\item American Geophysical Union: \vspace{-0.2cm}
		\begin{enumerate} \itemsep -2pt
		\item Diversity Programs: \url{http://www.agu.org/education/diversity_programs/}
		\end{enumerate}
	\end{enumerate}
\item {\it icademic.org} resources for the life sciences and engineering: \url{http://www.icademic.org/}
\item The Oceanography Society: \vspace{-0.3cm}
	\begin{enumerate} \itemsep -2pt
	\item Hands-On Oceanography: peer-reviewed activities appropriate for undergraduate and/or graduate classes in oceanography, \url{http://www.tos.org/hands-on/index.html}
	\end{enumerate}
%%%%%%%%%%%%%%%%%%%%%%%%%%%%%%%%%%%%%%%
%%%%%%%%%%%%%%%%%%%%%%%%%%%%%%%%%%%%%%%
\item outreach activities (including mentoring) to students in K-12: \vspace{-0.3cm}
	\begin{enumerate} \itemsep -2pt
	\item Research Councils UK (RCUK): \vspace{-0.2cm}
		\begin{enumerate} \itemsep -2pt
		\item Researchers in Residence (RinR): \vspace{-0.1cm}
			\begin{enumerate} \itemsep -1pt
			\item \url{http://www.researchersinresidence.ac.uk/cms/}
			\item \url{http://www.researchersinresidence.ac.uk/cms/researchers/}
			\item Mentor middle and high school students who are job shadowing (observing you first-hand) in your research activities for up to a week, so that they can learn what doing research in your research area is like. You should explain in laypeople's terms what your research is about. That is, be a mentor for the externships of middle and high school students.
			\end{enumerate}
		\end{enumerate}
	\end{enumerate}
%%%%%%%%%%%%%%%%%%%%%%%%%%%%%%%%%%%%%%%
%%%%%%%%%%%%%%%%%%%%%%%%%%%%%%%%%%%%%%%
\item competitions: \vspace{-0.3cm}
	\begin{enumerate} \itemsep -2pt
	\item Invent Now, Inc.: \vspace{-0.2cm}
		\begin{enumerate} \itemsep -2pt
		\item Collegiate Inventors Competition: \url{http://www.invent.org/collegiate/} [ Resources for {\color{blue} Patent Search Strategy} are available. \colorbox{blue}{\bf This is the ultimate competition for US students in science and engineering.} ]
		\end{enumerate}
	\item INFORMS Doing Good with Good OR - Student Competition: \vspace{-0.2cm}
		\begin{enumerate} \itemsep -2pt
		\item Doing Good with Good OR-Student Competition is held each year to identify and honor outstanding projects in the field of operations research and the management sciences conducted by a student or student group that have a significant societal impact.
		\item \url{http://www.informs.org/Recognize-Excellence/INFORMS-Prizes-Awards/Doing-Good-with-Good-OR}
		\end{enumerate}
	\item AWM Essay Contest: Biographies of Contemporary Women in Mathematics; see \url{http://www.awm-math.org/biographies/contest.html}
	\item American Society of Mechanical Engineers (ASME): \vspace{-0.2cm}
		\begin{enumerate} \itemsep -2pt
		\item Student Design Competition: \url{http://www.asme.org/Events/Contests/DesignContest/Student_Design_Competition.cfm}
		\item ASME Student Mechanism and Robot Design Competition: \url{http://www.asme.org/Events/Contests/Student_Mechanism_Robot_2.cfm}
		\end{enumerate}
	\item American Institute of Chemical Engineers (AIChE) competitions: \url{http://www.aiche.org/Students/Awards/index.aspx}
	\item Association for Unmanned Vehicle Systems International (AUVSI): \vspace{-0.2cm}
		\begin{enumerate} \itemsep -2pt
		\item AUVSI Student Competitions: \vspace{-0.1cm}
			\begin{enumerate} \itemsep -1pt
			\item \url{http://www.auvsi.org/AUVSI/AUVSI/Home/Default.aspx}, or \url{http://www.auvsi.org/}
			\item Annual Intelligent Ground Vehicle Competition (IGVC): \url{http://www.igvc.org/}
			\item Annual Student Unmanned Air System (SUAS) Competition: \url{http://65.210.16.57/studentcomp2010/default.html}
			\item International Aerial Robotics Competition (IARC): \url{http://iarc.angel-strike.com/}
			\item AUVSI and ONR's International Autonomous Surface Vehicle (ASV) Competition [ASVC]
			\item AUVSI Foundation and ONR's (U.S. Office of Naval Research) 4th International RoboBoats Competition: \url{http://www.auvsifoundation.org/AUVSI/FOUNDATION/Competitions/ASVCompetition/Default.aspx?C=00000000-0000-0000-0000-000000000000}
			\item AUVSI Foundation and ONR's (U.S. Office of Naval Research) International RoboSub Competition (or AUVSI and ONR's International Autonomous Underwater Vehicle Competition): \url{http://www.auvsifoundation.org/AUVSI/FOUNDATION/Competitions/AUVCompetition/Default.aspx}
			\item ONR: U.S. Office of Naval Research
			\end{enumerate}
		\end{enumerate}
	\item American Institute of Aeronautics and Astronautics (AIAA): \vspace{-0.2cm}
		\begin{enumerate} \itemsep -2pt
		\item Design Competitions: \url{http://www.aiaa.org/content.cfm?pageid=210}
		\end{enumerate}
	\item National Aeronautics and Space Administration: \vspace{-0.2cm}
		\begin{enumerate} \itemsep -2pt
		\item NASA's Langley Research Center: \vspace{-0.1cm}
			\begin{enumerate} \itemsep -1pt
			\item SpaceTech Engineering Design Challenge: \url{http://spacetech.larc.nasa.gov}
			\end{enumerate}
		\end{enumerate}
	\item American Concrete Institute (ACI): \vspace{-0.2cm}
		\begin{enumerate} \itemsep -2pt
		\item Competitions: \url{http://www.concrete.org/STUDENTS/st_competitions.htm}
		\end{enumerate}
	\end{enumerate}
%%%%%%%%%%%%%%%%%%%%%%%%%%%%%%%%%%%%%%%
%%%%%%%%%%%%%%%%%%%%%%%%%%%%%%%%%%%%%%%
\item underrepresented minorities: \vspace{-0.3cm}
	\begin{enumerate} \itemsep -2pt
	\item The Society of Women Engineers: \url{http://societyofwomenengineers.swe.org/}
	\item Association for Women in Science (AWIS): \url{http://www.awis.org/} and \url{http://www.awis.affiniscape.com/displaycommon.cfm?an=1&subarticlenbr=19}
	\item Association for Women in Mathematics (AWM): \vspace{-0.2cm}
		\begin{enumerate} \itemsep -2pt
		\item \url{http://www.awm-math.org/}
		\item Education: \vspace{-0.1cm}
			\begin{enumerate} \itemsep -1pt
			\item \url{http://sites.google.com/site/awmmath/awm-resources/education}
			\item Includes information for students in middle school, high school, college and university (including graduate students). It also includes information for parents and teachers/educators.
			\end{enumerate}
		\item Career advice and opportunities: \url{http://sites.google.com/site/awmmath/awm-resources/career}
		\item Women in Math, Science, and Society: \url{http://sites.google.com/site/awmmath/women-in-math-science-and-society}
		\item Essay contest on biographies of contemporary women in mathematics: \url{http://sites.google.com/site/awmmath/programs/essay-contest}
		\end{enumerate}
	\item Sigma Delta Epsilon-Graduate Women in Science (GWIS): \url{http://www.gwis.org/}
	\item Society of Hispanic Professional Engineers (SHPE): \vspace{-0.2cm}
		\begin{enumerate} \itemsep -2pt
		\item Advancing Hispanic Excellence in Technology, Engineering, Math and Science (AHETEMS) Foundation: \url{http://www.ahetems.org/}
		\item AHETEMS Scholarship Program: \url{http://www.ahetems.org/scholarships/}
		\item Graduate \& Young Professional Fellowship Program (encourage young professionals to engage in {\bf public policy}): \url{http://www.ahetems.org/graduate/graduate-young-professional-fellowship-program/}
		\item SHPE/GEM Fellowship (for graduate students in STEM at GEM Member Universities): \url{http://www.ahetems.org/graduate/shpe-gem-graduate-award/}. See \url{http://www.gemfellowship.org/gem-universities/university-members} for a list of GEM member universities.
		\item Internship opportunities: \url{http://www.ahetems.org/scholar-internships/}
		\item \url{http://oneshpe.shpe.org/wps/portal/national}
		\end{enumerate}
	\item National Society of Black Engineers (NSBE): \vspace{-0.2cm}
		\begin{enumerate} \itemsep -2pt
		\item Scholarships: \url{http://www.nsbe.org/Programs/Scholarships.aspx}
		\item Competitions for undergraduates and graduate students: \url{http://www.nsbe.org/Programs/Competitions/Collegiate.aspx}
		\item \url{http://www.nsbe.org/}
		\end{enumerate}
	\item The Society of Mexican American Engineers and Scientists (MAES): \vspace{-0.2cm}
		\begin{enumerate} \itemsep -2pt
		\item MAES Undergraduate and Graduate Outreach Programs (including ``GRE/Graduate Application Fee Waivers''): \url{http://www.maes-natl.org/index.php?module=ContentExpress&func=display&ceid=90&meid=238}
		\item Scholarships \& Awards: \url{http://www.maes-natl.org/index.php?meid=328}
		\item MAES Scholarship Program: \url{http://www.maes-natl.org/index.php?module=ContentExpress&func=display&ceid=518&meid=241}
		\end{enumerate}
	\item SACNAS (Society for Advancement of Chicanos and Native Americans in Science): \vspace{-0.2cm}
		\begin{enumerate} \itemsep -2pt
		\item Scholarships: \url{http://www.sacnas.org/webadindex.cfm?webadcategory_id=7}
		\item Fellowships: \url{http://www.sacnas.org/webadIndex.cfm?webadcategory_id=5}
		\end{enumerate}
	\item {\it Center for the Advancement of Hispanics in Science and Engineering Education} (CAHSEE): \vspace{-0.2cm}
		\begin{enumerate} \itemsep -2pt
		\item YESP - Young Engineers \& Scientists Program: \url{http://www.cahsee.org/2programs/yesp.asp.htm}. ``This program places talented Hispanic college students in the research labs of government agencies.''
		\item Scholarships: \url{http://www.cahsee.org/6resources/scholarships.asp.htm}
		\end{enumerate}
	\item American Geophysical Union: \vspace{-0.2cm}
		\begin{enumerate} \itemsep -2pt
		\item Has a list of organizations for specific underrepresented ethnic-minority groups in the geosciences and physics: \vspace{-0.1cm}
			\begin{enumerate} \itemsep -1pt
			\item \url{http://www.agu.org/education/diversity_programs/}
			\item These organizations may have information about scholarships, fellowships, and educational material for K-12 and college students.
			\end{enumerate}
		\end{enumerate}
	\item Institute for Broadening Participation: \vspace{-0.2cm}
		\begin{enumerate} \itemsep -2pt
		\item Minorities Striving and Pursuing Higher Degrees of Success in Earth System Science (MS PHD'S\textregistered) initiative: \vspace{-0.1cm}
			\begin{enumerate} \itemsep -1pt
			\item \url{http://www.msphds.org/}
			\item Prospective Students/Mentees: \url{http://www.msphds.org/prospective.asp}
			\item For MS PHD'S Students: \url{http://www.msphds.org/students.asp}
			\end{enumerate}
		\item PathwaysToScience.org: \vspace{-0.1cm}
			\begin{enumerate} \itemsep -1pt
			\item Resources for undergraduate students: \url{http://www.pathwaystoscience.org/Undergrads.asp}
			\item Resources for graduate students: \url{http://www.pathwaystoscience.org/Grad.asp}
			\item Resources for postdocs: \url{http://www.pathwaystoscience.org/Postdocs_portal.asp}
			\item STEM Resources by Institution (colleges, universities, and US national research laboratories): \url{http://www.pathwaystoscience.org/Institution.asp}
			\item Additional resources: \url{http://www.pathwaystoscience.org/Ideaexchange.asp}
			\end{enumerate}
		\item National Alliance for Doctoral Studies in the Mathematical Sciences: \vspace{-0.1cm}
			\begin{enumerate} \itemsep -1pt
			\item \url{http://www.mathalliance.org/}
			\item Student/Alliance Scholars: \url{http://www.mathalliance.org/scholars.asp}
			\item Alliance Mentors / Alliance Undergraduate Mentors: \url{http://www.mathalliance.org/mentors.asp}
			\item Alliance Programs: \url{http://www.mathalliance.org/programs.asp}
			\end{enumerate}
		\item Alliances for Graduate Education and the Professoriate (AGEP): \vspace{-0.1cm}
			\begin{enumerate} \itemsep -1pt
			\item \url{http://www.agep.us/}
			\end{enumerate}
		\item Maine Pathways to STEM (Science, Technology, Engineering \& Mathematics): \vspace{-0.1cm}
			\begin{enumerate} \itemsep -1pt
			\item \url{http://www.mainestem.org/}
			\item K-12 Teachers \& University Faculty: \url{http://www.mainestem.org/METeachersFaculty.asp}
			\item Graduate \& Undergraduate Students: \url{http://www.mainestem.org/MEUndergradGrad.asp}
			\end{enumerate}
		\end{enumerate}
	\item ARTSI (Advancing Robotics Technology for Societal Impact) Alliance: \vspace{-0.2cm}
		\begin{enumerate} \itemsep -2pt
		\item \url{http://artsialliance.org/}
		\item ``The ARTSI (Advancing Robotics Technology for Societal Impact) Alliance is a collaborative education and research project centered around robotics for healthcare, the arts, and entrepreneurship.  Spelman College, a historically black college (HBCU) for women is leading the alliance in partnership with several other HBCUs and Research I (R1) institutions.''
		\item Summer REU (Research Experience for Undergraduates) program: \url{http://artsialliance.org/Summer-REU-Program}
		\end{enumerate}
	\item Women in Technology (WIT): \vspace{-0.2cm}
		\begin{enumerate} \itemsep -2pt
		\item \url{http://www.womenintechnology.org/index.asp}
		\item WIT Mentor-Prot{\'{e}}g{\'{e}} Program: \url{http://www.womenintechnology.org/content.asp?contentid=59}
		\item {\bf \color{blue} WIT Career Transition Resource Guide}: \url{http://www.womenintechnology.org/content.asp?contentid=146}
		\item Girls In Technology (GIT): \vspace{-0.1cm}
			\begin{enumerate} \itemsep -1pt
			\item Get Involved: \vspace{-0.1cm}
				\begin{itemize} \itemsep -1pt
				\item \url{http://www.girlsintechnology.org/getinvolved.cfm}
				\item Teacher: teach girls about IT as an after-school activity or in a summer camp session
				\item Assistant Teacher: Assist instructors in GIT sessions, after-school activities, or summer camp sessions
				\item Develop Curriculum: Develop a curriculum for a supported GIT educational program
				\item Mentor: Mentor a girl in one of [GIT's] supported programs
				\item Job Shadow: ``Let a girl shadow you at work''
				\item Guest Speaker: ``Speak to a group of girls on a topic both you and they enjoy, such as computers, technology, education, how to take apart computers, how to build a web site, etc.''
				\end{itemize}
			\end{enumerate}
		\end{enumerate}
	\item Arizona State University: \vspace{-0.2cm}
		\begin{enumerate} \itemsep -2pt
		\item {\it Career}WISE: \vspace{-0.1cm}
			\begin{enumerate} \itemsep -1pt
			\item \url{http://careerwise.asu.edu/}
			\item Helpful resources for female graduate/Ph.D. students in science and engineering.
			\end{enumerate}
		\end{enumerate}
	\item American Indian Science and Engineering Society (AISES): \vspace{-0.2cm}
		\begin{enumerate} \itemsep -2pt
		\item Programs for undergraduates and grad students (including scholarships and internships): \vspace{-0.1cm}
			\begin{enumerate} \itemsep -1pt
			\item \url{http://www.aises.org/Programs}
			\item Resources: \url{http://www.aises.org/Programs/Resources}
			\end{enumerate}
		\end{enumerate}
	\end{enumerate}
\end{enumerate}




%%%%%%%%%%%%%%%%%%%%%%%%%%%%%%%%%%%%%%%%%%%
\subsection{Other Science and Engineering Outreach}
\label{otherstemoutreach}

Other Science and Engineering Outreach: \vspace{-0.3cm}
\begin{enumerate} \itemsep -4pt
\item Frontiers of Engineering (networking event for mid-career engineers): \url{http://www.naefrontiers.org/}
\item Consortium for Ocean Leadership: \vspace{-0.3cm}
	\begin{enumerate} \itemsep -2pt
	\item Resources for scientists in the marine sciences to use in outreach activities: \url{http://www.oceanleadership.org/education/deep-earth-academy/scientists/}
	\end{enumerate}
\item The Oceanography Society: \vspace{-0.3cm}
	\begin{enumerate} \itemsep -2pt
	\item Education and Public Outreach (EPO): A Guide for Scientists [material that scientists and professors can use for outreach activities], \url{http://www.tos.org/epo_guide/index.html}
	\end{enumerate}
\item The Joy McCann Foundation: \vspace{-0.3cm}
	\begin{enumerate} \itemsep -2pt
	\item McCann Scholar (for professors in medicine, science, and nursing): \url{http://www.mccannfoundation.org/scholars.htm}
	\item The Joy McCann Professorship for Women in Medicine: \url{http://www.mccannfoundation.org/medicine.htm}
	\end{enumerate}
\item U.S. National Academies: \vspace{-0.3cm}
	\begin{enumerate} \itemsep -2pt
	\item International Activities of the U.S. National Academies -- Science, Engineering \& Medicine: Working toward a better world: \vspace{-0.2cm}
		\begin{enumerate} \itemsep -2pt
		\item \url{http://sites.nationalacademies.org/International/}
		\item Solving the grand challenges: \vspace{-0.1cm}
			\begin{enumerate} \itemsep -1pt
			\item Energy and the Environment
			\item Global Health
			\item Water Resources
			\item Agriculture and Food Security
			\item International Security
			\item Population
			\end{enumerate}
		\item Help other countries build/improve their capacities: \vspace{-0.1cm}
			\begin{enumerate} \itemsep -1pt
			\item Cooperative Program with Pakistan 
			\item African Science Academies 
			\item Visiting Math Lecturer Program in Cambodia 
			\item Humanitarian Relief Efforts
			\item Improved Road Safety
			\item Science-based Decision Making for Sustainability
			\item Science Academies' Input to G8 Summits
			\end{enumerate}
		\item Scientific Cooperation: \vspace{-0.1cm}
			\begin{enumerate} \itemsep -1pt
			\item Building Bridges in the Middle East
			\item Cooperation with Iran
			\item Human Rights
			\item Frontiers of Science and Engineering Symposia
			\item Travel Grants
			\item International Conference on Women's Issues in Transportation
			\end{enumerate}
		\item Advising the U.S. Government: \vspace{-0.1cm}
			\begin{enumerate} \itemsep -1pt
			\item Science \& Technology in Foreign Policy
			\item Health 
			\item Science and Security
			\end{enumerate}
		\end{enumerate}
	\end{enumerate}
\item National Academy of Engineering: \vspace{-0.3cm}
	\begin{enumerate} \itemsep -2pt
	\item The Charles Stark Draper Prize (``to recognize innovative engineering achievements and their reduction to practice in ways that have led to important benefits and significant improvement in the well being and freedom of humanity''): \url{http://www.draperprize.org/}
	\item NAE Grand Challenge Scholars Program: \url{http://www.grandchallengescholars.org/}
	\end{enumerate}
\item United States Department of Defense (DoD): \vspace{-0.3cm}
	\begin{enumerate} \itemsep -2pt
	\item National Defense Education Program; Defense Advanced Research Projects Agency (DARPA): \vspace{-0.2cm}
		\begin{enumerate} \itemsep -2pt
		\item Resource for scientists and engineers to mentor youths, so that they would look into pursuing careers in science and engineering: \url{http://www.ndep.us/GetInvoSci.aspx}
		\item STEM Learning Modules (SLM): \vspace{-0.1cm}
			\begin{enumerate} \itemsep -1pt
			\item \url{http://www.ndep.us/ProgSLM.aspx}
			\item Help educators develop programs in science and engineering in K-12 institutions, so that youths would be encouraged to explore careers in science and engineering
			\end{enumerate}
		\end{enumerate}
	\end{enumerate}
\item Hewlett-Packard Development Company: \vspace{-0.3cm}
	\begin{enumerate} \itemsep -2pt
	\item HP Catalyst Initiative (grants for STEM education in colleges and universities): \url{http://www.hp.com/hpinfo/socialinnovation/catalyst.html}
	\item HP EdTech Innovators Award (for higher educational institutions that integrate IT into the curricular): \url{http://www.hp.com/hpinfo/socialinnovation/edtech.html}
	\end{enumerate}
\item The William and Flora Hewlett Foundation (Hewlett Foundation): \vspace{-0.3cm}
	\begin{enumerate} \itemsep -2pt
	\item Funding Programs: \url{http://www.hewlett.org/programs}
	\item Grantseekers: \url{http://www.hewlett.org/grants/grantseekers}
	\end{enumerate}
\item The Sloan Consortium (Sloan-C): \vspace{-0.3cm}
	\begin{enumerate} \itemsep -2pt
	\item Sloan-C Awards (for recognizing outstanding work in the field of online education) and Sloan-C Fellows: \url{http://sloanconsortium.org/aboutus/awards}
	\item Mayadas Leadership Award in Online Education: \url{http://sloanconsortium.org/mayadas_award}
	\end{enumerate}
\item W.K. Kellogg Foundation: \vspace{-0.3cm}
	\begin{enumerate} \itemsep -2pt
	\item Grant database: \url{http://www.wkkf.org/grants/grants-database.aspx}
	\end{enumerate}
\item Hewlett-Packard Company: \vspace{-0.3cm}
	\begin{enumerate} \itemsep -2pt
	\item HP community investment for education, economic development, and the environment: \url{http://www.hp.com/hpinfo/socialinnovation/focus.html}
	\item Entrepreneurship education: \vspace{-0.2cm}
		\begin{enumerate} \itemsep -2pt
		\item \url{http://www.hp.com/hpinfo/globalcitizenship/society/social/entrepreneurship.html}
		\item HP Graduate Entrepreneurship Training through IT (GET-IT)
		\item HP Entrepreneurship Learning Program (HELP)
		\end{enumerate}
	\item HP Innovations in Education grants: \url{http://www.hp.com/hpinfo/globalcitizenship/society/social/innovations.html}
	\end{enumerate}
\item General Electric Company: \vspace{-0.3cm}
	\begin{enumerate} \itemsep -2pt
	\item GE Foundation: \vspace{-0.2cm}
		\begin{enumerate} \itemsep -2pt
		\item Developing Futures\texttrademark\ in Education program (which encompasses the GE College Bound Program): \url{http://www.ge.com/foundation/developing_futures_in_education/index.jsp}
		\item Environment, health and safety, and health industry training programs (outside the US): \url{http://www.ge.com/foundation/international_programs/training.jsp}
		\item Student, education and scholarship initiatives: \url{http://www.ge.com/foundation/international_programs/education_initiatives.jsp}
		\end{enumerate}
	\end{enumerate}
\item The GRAMMY Foundation: \vspace{-0.3cm}
	\begin{enumerate} \itemsep -2pt
	\item GRAMMY Foundation Grants: \vspace{-0.2cm}
		\begin{enumerate} \itemsep -2pt
		\item \url{http://www2.grammy.com/GRAMMY_Foundation/Grants/}
		\item It funds {\bf Scientific Research Projects} as well as {\it Archiving And Preservation Projects}.
		\item Concerning scientific research projects: ``The GRAMMY Foundation Grant Program awards grants to organizations and individuals to support research on the impact of music on the human condition. Examples might include the study of the effects of music on mood, cognition and healing, as well as the medical and occupational well-being of music professionals and the creative process underlying music.'' [ E.g., look at music therapy as a possible research topic/area. ]
		\end{enumerate}
	\end{enumerate}
\item The Dana Foundation: \vspace{-0.3cm}
	\begin{enumerate} \itemsep -2pt
	\item \url{http://www.dana.org/grants/}
	\item Has grants for: \vspace{-0.2cm}
		\begin{enumerate} \itemsep -2pt
		\item Brain and Immuno-Imaging
		\item Clinical Neuroscience
		\item Human Immunology
		\item Neuroimmunology of Brain Infections and Cancers
		\end{enumerate}
		\item Deadlines and Requests for Proposals (RFP): \url{http://www.dana.org/grants/deadlines.aspx}
	\end{enumerate}
%%%%%%%%%%%%%%%%%%%%%%%%%%%%%%%%%%%%%%%
% underrepresented minorities
\item Institute for Broadening Participation: \vspace{-0.3cm}
	\begin{enumerate} \itemsep -2pt
	\item PathwaysToScience.org: \vspace{-0.2cm}
		\begin{enumerate} \itemsep -2pt
		\item Resources for faculty and administrators (to facilitate STEM outreach activities as well as the recruitment of underrepresented minorities to the student body and faculty): \url{http://www.pathwaystoscience.org/Faculty.asp}
		\end{enumerate}
	\end{enumerate}
\item National Center for Women \& Information Technology (NCWIT): \vspace{-0.3cm}
	\begin{enumerate} \itemsep -2pt
	\item NCWIT Academic Alliance Seed Fund (for developing and implementing initiatives in colleges and universities to recruit and retain women in computing and information technology): \url{http://www.ncwit.org/work.awards.seed.html}
	\item NCWIT Symons Innovator Award (for outstanding women who have successfully built and funded an IT business): \url{http://www.ncwit.org/work.awards.innovator.html}
	\end{enumerate}
\item Women in Technology (WIT): \vspace{-0.3cm}
	\begin{enumerate} \itemsep -2pt
	\item Girls In Technology (GIT): \vspace{-0.2cm}
		\begin{enumerate} \itemsep -2pt
		\item Get Involved: \vspace{-0.1cm}
			\begin{itemize} \itemsep -1pt
			\item \url{http://www.girlsintechnology.org/getinvolved.cfm}
			\item Teacher: teach girls about IT as an after-school activity or in a summer camp session
			\item Assistant Teacher: Assist instructors in GIT sessions, after-school activities, or summer camp sessions
			\item Develop Curriculum: Develop a curriculum for a supported GIT educational program
			\item Mentor: Mentor a girl in one of [GIT's] supported programs
			\item Job Shadow: ``Let a girl shadow you at work''
			\item Guest Speaker: ``Speak to a group of girls on a topic both you and they enjoy, such as computers, technology, education, how to take apart computers, how to build a web site, etc.''
			\end{itemize}
		\end{enumerate}
	\end{enumerate}
\item European Platform of Women Scientists (EPWS): \vspace{-0.3cm}
	\begin{enumerate} \itemsep -2pt
	\item \url{http://www.epws.org/}
	\item Members: \url{http://www.epws.org/index.php?option=com_content&task=blogcategory&id=134&Itemid=4652}
	\end{enumerate}
\end{enumerate}





Commercializing academic research into products and services via start-ups: \vspace{-0.3cm}
\begin{enumerate} \itemsep -4pt
\item Ben Franklin Technology Partners (BFTP): \vspace{-0.3cm}
	\begin{enumerate} \itemsep -2pt
	\item Innovation Works (IW): \vspace{-0.2cm}
		\begin{enumerate} \itemsep -2pt
		\item For universities in the Pittsburgh metropolitan area
		\item University Innovation Grants (UIGs) / University Grants: \vspace{-0.1cm}
			\begin{enumerate} \itemsep -1pt
			\item For technology validation, market research, prototype development, and intellectual property evaluation
			\item Available online at: \url{http://www.innovationworks.org/OurPrograms/UniversityGrants/tabid/115/Default.aspx}; last accessed on November 14, 2010.
			\end{enumerate}
		\end{enumerate}
	\end{enumerate}
\end{enumerate}









%%%%%%%%%%%%%%%%%%%%%%%%%%%%%%%%%%%%%%%%%%%
\subsection{Electrical and Computer Engineering \& Computer Science Outreach}
\label{ececsoutreach}

Electrical and computer engineering, and computer science outreach: \vspace{-0.3cm}
\begin{enumerate} \itemsep -4pt
\item IEEE: \vspace{-0.3cm}
	\begin{enumerate} \itemsep -2pt
	\item {\it IEEE-USA Salary Service} provides a survey of jobs in electrical and computer engineering: \url{http://www.ieeeusa.org/careers/salary/}
	\item {\it IEEE Santa Clara Valley Section PACE}: Professional Activities Committee for Engineers (PACE); see \url{http://www.ewh.ieee.org/r6/scv/PACE/}
	\item {\it IEEE Santa Clara Valley Section}: \url{http://ewh.ieee.org/r6/scv/} and \url{http://www.ieee.org/scv}
	\item 
	\end{enumerate}
\item Association for Computing Machinery, ACM: \vspace{-0.3cm}
	\begin{enumerate} \itemsep -2pt
	\item Sanjeev Arora, Boaz Barak, and Luca Trevisan, ``Survey Papers and Essays,'' in {\it Theory Matters Wiki: Theoretical Computer Science (TCS) Advocacy Wiki}, SIGACT Committee for the Advancement of Theoretical Computer Science, ACM Special Interest Group on Algorithms and Computation Theory (SIGACT), Association for Computing Machinery, February 25, 2010. Available at: \url{http://theorymatters.org/pmwiki/pmwiki.php?n=Main.SurveyCollection}; last accessed on September 14, 2010.
	\item Online Resources for Graduating Students: \url{http://www.acm.org/membership/student/resources-for-grads}
	\end{enumerate}
\item VLSI design and verification: \vspace{-0.3cm}
	\begin{enumerate} \itemsep -2pt
	\item {\it DVClub} for individuals interested in VLSI verification: \url{http://www.dvclub.org/}
	\item {\it DeepChip.com}: \url{http://www.deepchip.com}
	\end{enumerate}
%%%%%%%%%%%%%%%%%%%%%%%%%%%%%%%
\item undergraduates: \vspace{-0.3cm}
	\begin{enumerate} \itemsep -2pt
	\item {\it Humanitarian FOSS Project}: \vspace{-0.2cm}
		\begin{enumerate} \itemsep -2pt
		\item Where FOSS refers to Free and Open Source Software
		\item For computer science and engineering students
		\item \url{http://www.hfoss.org/}
		\end{enumerate}
	\item {\it SIGDA Design Automation Summer School}: \vspace{-0.2cm}
		\begin{enumerate} \itemsep -2pt
		\item {\it NSF�SRC�SIGDA�DAC Design Automation Summer School}
		\item \url{http://www.sigda.org/dass.html}
		\item Travel grants are provided to defray travel and accommodation expenses
		\end{enumerate}
	\item {\it Young Student Support Program at DAC}: \vspace{-0.2cm}
		\begin{enumerate} \itemsep -2pt
		\item Also known as {\it DAC Young Student Support Program}
		\item \url{http://www.sigda.org/youngstudent.html}
		\item Travel grants are provided to defray travel and accommodation expenses
		\end{enumerate}
	\item {\it ACM Student Research Competition at Design Automation Conference}: \vspace{-0.2cm}
		\begin{enumerate} \itemsep -2pt
		\item Sponsored by {\it Microsoft Research}
		\item \url{http://www.sigda.org/studentcomp.html}
		\item Also, see {\it ACM Student Research Competition} @ \url{http://src.acm.org/}.
		\end{enumerate}
	\item Job database for positions in the Video Game, Animation, VFX, and Software/Technology industries: \url{http://www.creativeheads.net/}
	\end{enumerate}
%%%%%%%%%%%%%%%%%%%%%%%%%%%%%%
\item graduate students: \vspace{-0.3cm}
	\begin{enumerate} \itemsep -2pt
	\item {\it SIGDA Design Automation Summer School}: \vspace{-0.2cm}
		\begin{enumerate} \itemsep -2pt
		\item {\it NSF�SRC�SIGDA�DAC Design Automation Summer School}
		\item \url{http://www.sigda.org/dass.html}
		\item Travel grants are provided to defray travel and accommodation expenses
		\end{enumerate}
	\item {\it Young Student Support Program at DAC}: \vspace{-0.2cm}
		\begin{enumerate} \itemsep -2pt
		\item Also known as {\it DAC Young Student Support Program}
		\item \url{http://www.sigda.org/youngstudent.html}
		\item Travel grants are provided to defray travel and accommodation expenses
		\end{enumerate}
	\item {\it ACM Student Research Competition at Design Automation Conference}: \vspace{-0.2cm}
		\begin{enumerate} \itemsep -2pt
		\item Sponsored by {\it Microsoft Research}
		\item \url{http://www.sigda.org/studentcomp.html}
		\item Also, see {\it ACM Student Research Competition} @ \url{http://src.acm.org/}.
		\end{enumerate}
	\item {\it SIGDA University Booth at DAC}: \vspace{-0.2cm}
		\begin{enumerate} \itemsep -2pt
		\item Or, {\it SIGDA/DAC University Booth}
		\item \url{http://www.sigda.org/ubooth.html}
		\end{enumerate}
	\item {\it SIGDA Ph.D. Forum at DAC}: \vspace{-0.2cm}
		\begin{enumerate} \itemsep -2pt
		\item \url{http://www.sigda.org/phdforum.html}
		\item \url{http://www.sigda.org/daforum/}
		\end{enumerate}
	\item {\it DAC Graduate Scholarship}: \vspace{-0.2cm}
		\begin{enumerate} \itemsep -2pt
		\item {\it A. Richard Newton Graduate Scholarships} to Support Graduate Research and Study
		\item \url{http://www.sigda.org/gradscholarship.html}
		\end{enumerate}
	\end{enumerate}
%%%%%%%%%%%%%%%%%%%%%%%%%%%%%%
\item competitions, and programming contests and challenges: \vspace{-0.3cm}
	\begin{itemize} \itemsep -2pt
	\item {\it SIGDA CADathlon at ICCAD}: \vspace{-0.2cm}
		\begin{enumerate} \itemsep -2pt
		\item \url{http://www.sigda.org/programs/cadathlon/}
		\item \url{http://www.sigda.org/cadathlon.html}
		\item Travel grants are provided to defray travel and accommodation expenses
		\end{enumerate}
	\item ISPD Programming Contest: \url{http://www.ispd.cc/contests/}
	\item ACM International Workshop on Timing Issues in the Specification and Synthesis of Digital Systems (TAU Workshop): \vspace{-0.2cm}
		\begin{enumerate} \itemsep -2pt
		\item Power Grid Simulation Contest: \url{http://www.tauworkshop.com/PREVIOUS/contest_2011.html}
		\end{enumerate}
	\item IEEE Computer Society Simulator Design competition: \url{http://www.computer.org/portal/web/competition}
	\item {\it DAC/ISSCC Student Design Contest}: \vspace{-0.2cm}
		\begin{enumerate} \itemsep -2pt
		\item \url{http://www.dac.com}
		\end{enumerate}
	\item {\it ACM/IEEE International Conference on Formal Methods and Models for Codesign -- Design Contest}: \vspace{-0.2cm}
		\begin{enumerate} \itemsep -2pt
		\item MEMOCODE Hardware/Software Co-Design Contest (MEMOCODE HW/SW co-design contest)
		\item \url{http://www-memocode2010.imag.fr/}
		\item \url{http://memocode2010.csail.mit.edu/redmine/wiki/memocode2010/Results}
		\end{enumerate}
	\item {\it International Low Power Design Contest}: \vspace{-0.2cm}
		\begin{enumerate} \itemsep -2pt
		\item ACM/IEEE International Symposium on Low Power Electronics and Design (ISLPED) -- Design Contest
		\item The International Symposium on Low Power Electronics and Design is holding the International Low Power Design Contest to provide a forum for universities and research organizations to showcase original ``power-aware'' designs and to highlight the innovations and design choices targeted at low power.
		\item The goal is to encourage and highlight design-oriented approaches to power reduction.
		\item \url{http://www.islped.org/2010/index.html}
		\end{enumerate}
	\item {\it University LSI Design Contest @ ASP-DAC}: \vspace{-0.2cm}
		\begin{enumerate} \itemsep -2pt
		\item Application areas or types of circuits of the original LSI circuit designs include (but are not limited to): \vspace{-0.1cm}
			\begin{enumerate} \itemsep -1pt
			\item Analog, RF and Mixed-Signal Circuits
			\item Digital Signal Processing
			\item Microprocessors
			\item Custom ASIC
			\end{enumerate} 
		\item Methods or technology used for implementation include: \vspace{-0.1cm}
			\begin{enumerate} \itemsep -1pt
			\item Full Custom and Cell-Based LSIs
			\item Gate Arrays
			\item FPGA/PLDs.
			\end{enumerate}
		\item \url{http://www.aspdac.com/aspdac2011/cfd/}
		\end{enumerate}
	\item IEEE Programming Challenge at IWLS: \url{http://www.iwls.org/challenge/}
	\item IEEE Asian Solid-State Circuits Conference (A-SSCC) Student Design Contest: \url{http://a-sscc2010.a-sscc.org/contest.html}
	\item {\it VLSI Conference 2011 - Design Contest}: \vspace{-0.2cm}
		\begin{enumerate} \itemsep -2pt
		\item Design/project fields include (but not limited to): \vspace{-0.1cm}
			\begin{enumerate} \itemsep -1pt
			\item Digital Integrated Circuits
			\item Analog Integrated Circuits
			\item FPGA based designs
			\item Computer Architectures/ Processors
			\item Reconfigurable Computing Systems
			\item SoC / Platform-based designs
			\item Embedded Systems
			\item MEMS/Optics/Bio-Chips
			\item Innovative Design Methodologies and Verification Techniques.
			\end{enumerate}
		\item \url{http://vlsiconference.com/vlsi2011/submissions_design_contest.html}
		\end{enumerate}
	\item {\it Satisfiability Modulo Theories Competition} (SMT-COMP): \vspace{-0.2cm}
		\begin{enumerate} \itemsep -2pt
		\item Competition for SMT solvers
		\item \url{http://www.smtcomp.org/2010/}
		\end{enumerate}
	\item {\it SAT Competition 201X}, where $X > 0$ \& $X {\it mod} 2 = 1$: \vspace{-0.2cm}
		\begin{enumerate} \itemsep -2pt
		\item The purpose of the competition is to identify new challenging benchmarks and to promote new solvers for the propositional satisfiability problem (SAT) as well as to compare them with state-of-the-art solvers.
		\item \url{http://www.satcompetition.org/}
		\end{enumerate}
	\item {\it SAT-Race 201X}, where $X > 0$ \& $X {\it mod} 2 = 0$: \vspace{-0.2cm}
		\begin{enumerate} \itemsep -2pt
		\item SAT-Race 201X is a competitive event for solvers of the Boolean Satisfiability (SAT) problem. 
		\item In contrast to the SAT Competitions, the focus of SAT-Race is on application benchmarks only.
		\item \url{http://baldur.iti.uka.de/sat-race-2010/}
		\end{enumerate}
	\item Hardware Model Checking Competition (HWMCC): \url{http://fmv.jku.at/hwmcc10/}
	\item {\it CADE ATP System Competition} (CASC): \vspace{-0.2cm}
		\begin{enumerate} \itemsep -2pt
		\item It is a yearly competition of fully automated theorem provers for classical first order logic.
		\item \url{http://www.cs.miami.edu/~tptp/CASC/}
		\end{enumerate}
	\item Apple Design Awards: \url{http://developer.apple.com/wwdc/ada/index.html}
	\item {\it International Constraint Solver Competition}: \vspace{-0.2cm}
		\begin{enumerate} \itemsep -2pt
		\item Also known as: \vspace{-0.2cm}
			\begin{enumerate} \itemsep -2pt
			\item International Constraint Solver Competition (CSP, Max-CSP and Weighted-CSP competition)
			\item International CSP Solver Competition (CSP, Max-CSP and Weighted-CSP competition)
			\end{enumerate}
		\item The Fourth International Constraint Solver Competition (CSC'2009) is organized to improve our knowledge of what is behind the efficiency of constraint satisfaction algorithms, heuristics, solving strategies, and constraint systems.
		\item \url{http://cpai.ucc.ie/}
		\end{enumerate}
	\item International Conference on Field-Programmable Technology (FPT 201X): \vspace{-0.2cm}
		\begin{enumerate} \itemsep -2pt
		\item FPT Design Competition: \url{http://cas.ee.ic.ac.uk/people/as999/FPTDesignComp/}
		\end{enumerate}
	\item International Microwave Symposium: Student Design Competitions -- Jan (includes AMS circuit simulation, and AMS/RF EDA); \url{http://ims2011.org/Technical_Program/Student_Design_Competitions.html}
	\item {\it QBFEVAL'1X}: \vspace{-0.2cm}
		\begin{enumerate} \itemsep -2pt
		\item QBF Solver competition for solvers to determine Quantified Boolean Formula (QBF) satisfiability.
		\item QBFLIB is a collection of instances, solvers, and tools related to Quantified Boolean Formula (QBF) satisfiability. See \url{http://www.qbflib.org/}.
		\item \url{http://www.qbflib.org/index_eval.php}
		\end{enumerate}
	\item {\it Pseudo-Boolean Competition 201X}: \vspace{-0.2cm}
		\begin{enumerate} \itemsep -2pt
		\item Competition for pseudo-Boolean solvers.
		\item \url{http://www.cril.univ-artois.fr/PB10/}
		\end{enumerate}
	\item {\it Answer Set Programming System Competition}: \vspace{-0.2cm}
		\begin{enumerate} \itemsep -2pt
		\item \url{http://dtai.cs.kuleuven.be/events/ASP-competition/}
		\end{enumerate}
	\item {\it Max-SAT Evaluation, Max-SAT 201X}: \vspace{-0.2cm}
		\begin{enumerate} \itemsep -2pt
		\item Competition for Max-SAT solvers
		\item \url{http://www.maxsat.udl.cat/}
		\item \url{http://www.maxsat.udl.cat/09/}
		\end{enumerate}
	\item {\it IEEEXtreme 24 Hour Programming Challenge}: \vspace{-0.2cm}
		\begin{enumerate} \itemsep -2pt
		\item Programming contest for college students
		\item \url{http://portal.ieee.org/web/membership/students/scholarshipsawardscontests/ieeextreme.html}
		\end{enumerate}
	\item {\it ACM International Collegiate Programming Contest} (ACM-ICPC or ICPC): \vspace{-0.2cm}
		\begin{enumerate} \itemsep -2pt
		\item Programming contest for college students
		\item Official web page: \url{http://cm.baylor.edu/welcome.icpc}
		\item Other web resources: \vspace{-0.1cm}
			\begin{enumerate} \itemsep -1pt
			\item {\it Wikipedia}: \url{http://en.wikipedia.org/wiki/ACM_International_Collegiate_Programming_Contest}
			\item {\it }: \url{}
			\item {\it }: \url{}
			\item {\it Valladolid Online Judge Site}: \url{http://acm.uva.es/}
			\item {\it ACMSolver :: Art of Programming Contest, Tips and Tricks for C, C++, Java}: \url{http://www.acmsolver.org/}
			\end{enumerate}
		\item 
		\end{enumerate}
	\item {\it TopCoder} coding and design contests: \vspace{-0.2cm}
		\begin{enumerate} \itemsep -2pt
		\item The contests cover various fields, such as: \vspace{-0.1cm}
			\begin{enumerate} \itemsep -1pt
			\item Algorithm
			\item Conceptualization
			\item Specification
			\item Architecture
			\item Component Design
			\item Component Development
			\item Assembly
			\item Test Scenarios
			\item Test Suites
			\item UI Prototype
			\item Rich Internet Application (RIA) Build
			\item Bug Race
			\item Marathon Match
			\item High School (for high school students)
			\item Copilot Opportunities
			\end{enumerate}
		\item \url{http://www.topcoder.com/}
		\end{enumerate}
	\item IEEE Presidents' Change the World competition: \vspace{-0.2cm}
		\begin{enumerate} \itemsep -2pt
		\item The IEEE Presidents� Change the World Competition recognizes students who develop unique solutions to real-world problems using engineering, science, computing and leadership skills to benefit their community, the world at large, or both. 
		\item \url{http://www.ieeechangetheworld.org/}
		\end{enumerate}
	\item Google Code Jam (programming contest): \url{http://code.google.com/codejam/} and \url{http://en.wikipedia.org/wiki/Google_Code_Jam}
	\item {\it RoboCup}\texttrademark\ competitions: \vspace{-0.2cm}
		\begin{enumerate} \itemsep -2pt
		\item Has different categories, including soccer, rescue operations, and home applications.
		\item \url{http://www.robocup.org/}
		\end{enumerate}
	\item ICFP Programming Contest (ICFP refers to International Conference on Functional Programming): \url{http://icfpcontest.org/}
	\item Student Cluster Competition (SCC): \vspace{-0.2cm}
		\begin{enumerate} \itemsep -2pt
		\item SCC is held at each (annual) SC conference, which is the International Conference for High Performance Computing, Networking, Storage, and Analysis. IEEE Computer Society and the Association for Computing Machinery are the sponsors for this conference.
		\item During SC10, teams consisting of six students, undergraduate and/or high school, will showcase the amazing power of clusters and the ability to utilize open source software to solve interesting and important problems. They will compete in real-time on the exhibit floor to run a workload of real-world applications on clusters of their own design while never exceeding the dictated power limit.
		\item During SC10 in New Orleans, teams will assemble, test and tune their machines and run the HPCC benchmarks until the starting bell rings on Monday night at the Exhibit Opening Gala where they will be given the competition data sets. In full view of conference attendees, teams will execute the prescribed workload while showing progress and science visualization output on large high-resolution displays in their areas. Teams race to correctly complete the greatest number of application runs during the competition period until the close of the exhibit floor on Wednesday evening.
		\item \url{http://sc10.supercomputing.org/?pg=studentcluster.html}
		\end{enumerate}
	\item Cypress Semiconductor Corporation: \vspace{-0.2cm}
		\begin{enumerate} \itemsep -2pt
		\item ARM Cortex-M3 PSoC\textregistered\ 5 Design Challenge: \url{http://www.cypress.com/?id=3271}
		\end{enumerate}
	\item Mentor Graphics: \vspace{-0.2cm}
		\begin{enumerate} \itemsep -2pt
		\item PCB Technology Leadership Awards (PCB design contest): \url{http://www.mentor.com/products/pcb-system-design/tla/index.cfm?v=mentorgraphics&p=handout:tla&a=print_card&g=sdd&s=1x1&c=ocid_2203&cmpid=3911}, or \url{http://www.mentor.com/go/tla}
		\end{enumerate}
	\item INFORMS Data Mining Contest: \vspace{-0.2cm}
		\begin{enumerate} \itemsep -2pt
		\item \url{http://ifors.org/web/call-for-participation-informs-data-mining-contest-2010/}
		\item \url{http://kaggle.com/informs2010}
		\end{enumerate}
	\item INFORMS Doing Good with Good OR - Student Competition: \vspace{-0.2cm}
		\begin{enumerate} \itemsep -2pt
		\item Doing Good with Good OR-Student Competition is held each year to identify and honor outstanding projects in the field of operations research and the management sciences conducted by a student or student group that have a significant societal impact.
		\item \url{http://www.informs.org/Recognize-Excellence/INFORMS-Prizes-Awards/Doing-Good-with-Good-OR}
		\end{enumerate}
	\item HPC Challenge Award Competition: \url{http://www.hpcchallenge.org/}
	\item Sphere Online Judge, SPOJ (programming contest): \url{http://www.spoj.pl/}
	\item High Performance and Scientific Computing Contest (Argonne National Laboratory, U.S. Department of Energy, DOE): \url{https://wiki.alcf.anl.gov/index.php/HPSC_Contest_Information}
	\item Argonne National Laboratory, ANL; Mathematics and Computer Science Division: \vspace{-0.2cm}
		\begin{enumerate} \itemsep -2pt
		\item J. H. Wilkinson Prize for Numerical Software (for developers of numerical software): \url{http://www.mcs.anl.gov/research/opportunities/wilkinsonprize/index.php}
		\end{enumerate}
	\item Society for Industrial and Applied Mathematics, SIAM: \vspace{-0.2cm}
		\begin{enumerate} \itemsep -2pt
		\item SIAM/ACM Prize in Computational Science and Engineering: \url{http://www.siam.org/prizes/sponsored/cse.php}. [ For developers of mathematical and computational tools and methods for the solution of science and engineering. Or, for developers of computational science and engineering software. ]
		\end{enumerate}
	\end{itemize}
	\item Sun HPC Software Programming Challenge (Oracle Corporation): \url{http://wikis.sun.com/display/HPCContest/Home}
%%%%%%%%%%%%%%%%%%%%%%%%%%%%%%
\item News media: \vspace{-0.3cm}
	\begin{itemize} \itemsep -2pt
	\item --- --- --- --- --- --- --- --- --- --- --- --- --- --- --- --- --- --- --- --- --- --- --- --- --- --- --- --- --- --- ---
	\item \colorbox{blue}{\bf News media for Electronic Design Automation}
	% News media for Electronic Design Automation
	\item {\it EDACafe}: \url{http://www.edacafe.com/}
	\item {\it SIGDA E-Newsletter} (SIGDA Electronic Newsletter): \url{http://www.sigda.org/newsletter/}
	\item {\it DeepChip.com}: \url{http://www.deepchip.com}
	\item --- --- --- --- --- --- --- --- --- --- --- --- --- --- --- --- --- --- --- --- --- --- --- --- --- --- --- --- --- --- ---
	\item \colorbox{blue}{\bf News media for Electrical and Computer Engineering}
	% News media for Electrical and Computer Engineering
	\item {\it EE Times} (Electronic Engineering Times): \url{http://www.eetimes.com/}
	\item {\it EDN} (Electrical Design News): \url{http://www.edn.com/}
	\item {\it IEEE Spectrum}: \url{http://spectrum.ieee.org/}
	\item {\it The Institute} (from IEEE): \url{http://www.theinstitute.ieee.org}
	\item {\it IEEE-USA Today's Engineer}: \url{http://www.todaysengineer.org/}
	\item {\it DeepChip.com}: \url{http://www.deepchip.com}
	\item --- --- --- --- --- --- --- --- --- --- --- --- --- --- --- --- --- --- --- --- --- --- --- --- --- --- --- --- --- --- ---
	\item \colorbox{blue}{\bf News media for Computer Science and Engineering, Information Systems, and IT}
	% News media for Computer Science and Engineering, Information Systems, and IT
	\item {\it ACM TechNews}: \url{http://technews.acm.org/}
	\item {\it TechCareers}: \url{http://www.techcareers.com/}
	\item {\it }: \url{}
	\item {\it }: \url{}
	\item {\it }: \url{}
	\item {\it }: \url{}
	\item {\it }: \url{}
	\item {\it }: \url{}
	\item {\it }: \url{}
	\item --- --- --- --- --- --- --- --- --- --- --- --- --- --- --- --- --- --- --- --- --- --- --- --- --- --- --- --- --- --- ---
	\item \colorbox{blue}{\bf Other News Media}
	% Other News Media
	\item {\it iTunes U}
	\item {\it YouTube EDU}
	\end{itemize}
%%%%%%%%%%%%%%%%%%%%%%%%%%%%%%
\item underrepresented minorities: \vspace{-0.3cm}
	\begin{enumerate} \itemsep -2pt
	\item women: \vspace{-0.2cm}
		\begin{enumerate} \itemsep -2pt
		\item IEEE Women in Engineering (WIE): \url{http://www.ieee.org/membership_services/membership/women/index.html?WT.mc_id=WIE_nav1}
		\item ACM-W: \url{http://women.acm.org/}
		\item Computer Research Association's Committee on the Status of Women in Computing Research (CRA-W): \vspace{-0.1cm}
			\begin{enumerate} \itemsep -1pt
			\item \url{http://www.cra-w.org/}
			\item Computing Research Association's Committee on the Status of Women (CRA-W) and the Coalition to Diversify Computing (CDC), {\it CompArch Summer School on Parallel Programming and Architectures}. Available at: \url{http://www.princeton.edu/~archss/}; last accessed on September 3, 2010.
			\end{enumerate}
		\item National Center for Women \& Information Technology: \url{http://www.ncwit.org/}
		\item African-American Women in Technology organization (AAWIT): \url{http://www.aawit.net/09/index.cfm}
		\item Grace Hopper Celebration of Women in Computing (conference for female IT students, professors, and professionals): \url{http://gracehopper.org/} or \url{http://gracehopper.org/2010/}
		\item Anita Borg Institute for Women and Technology: \vspace{-0.1cm}
			\begin{enumerate} \itemsep -1pt
			\item Has many programs for female students and professionals: \url{http://anitaborg.org/}
			\end{enumerate}
		\end{enumerate}
	\end{enumerate}
\end{enumerate}







%%%%%%%%%%%%%%%%%%%%%%%%%%%%%%%%%%%%%%%%%%%
\section{Scholarships, Fellowships, Awards, and Financial Aid}
\label{scholarshipsfinaidawards}

Resources for scholarships, fellowships, and financial aid: \vspace{-0.3cm}
\begin{enumerate} \itemsep -4pt
\item --- --- --- --- --- --- --- --- --- --- --- --- --- --- --- --- --- --- --- --- --- --- --- --- --- --- --- --- --- --- ---
\item \colorbox{blue}{\bf Lists of Scholarships and Fellowships}
% Lists of Scholarships and Fellowships
\item List of scholarships: \vspace{-0.3cm}
	\begin{enumerate} \itemsep -2pt
	\item Engineering Education Service Center, EESC (Engineering): \url{http://www.engineeringedu.com/scholars.html}
	\item High Performance and Embedded Architecture and Compilation, HiPEAC (Computer Science and Engineering): \url{http://www.hipeac.net/all_jobs_op}
	\item Office of Doctoral Programs at USC Viterbi School of Engineering, {\bf University of Southern California}. External Fellowships and other support: \url{http://viterbi.usc.edu/students/phd/fellowships-and-other-support/external-fellowships.htm}. USC Fellowships: \url{http://viterbi.usc.edu/students/phd/fellowships-and-other-support/usc-fellowships.htm}
	\item Columbia College, {\bf Columbia University} in the City of New York: \url{http://www.college.columbia.edu/students/fellowships/catalog}
	\item {\bf New York University} School of Law: \url{http://www.law.nyu.edu/financialaid/supplementalaid/fellowships/index.htm}
	\item Swedish Institute: \vspace{-0.2cm}
		\begin{enumerate} \itemsep -2pt
		\item The Swedish Institute, a government agency, administers over 500 scholarships each year for students and researchers coming to Sweden to pursue their objectives at a Swedish university.
		\item Study in Sweden: scholarships, \url{http://www.studyinsweden.se/Scholarships/}
		\item Swedish Institute (SI): \url{http://www.si.se/English/Navigation/Scholarships-and-exchanges/} [ Has special programs for Pakistanis and Turkish citizens ]
		\end{enumerate}
	\item The Swedish Foundation for International Cooperation in Research and Higher Education (STINT): \vspace{-0.2cm}
		\begin{enumerate} \itemsep -2pt
		\item \url{http://www.stint.se/en}
		\item Scholarships and grants: \url{http://www.stint.se/en/scholarships_and_grants}
		\end{enumerate}
	\item Center for the Advancement of Hispanics in Science and Engineering Education (CAHSEE): \url{http://www.cahsee.org/6resources/scholarships.asp.htm}
	\item University of Wisconsin-Madison: \vspace{-0.2cm}
		\begin{enumerate} \itemsep -2pt
		\item Grants Information Collection: A Cooperating Collection of the Foundation Center Library Network, \url{http://grants.library.wisc.edu/}
		\end{enumerate}
	\item {\it Find A PhD}: \url{http://www.findaphd.com/}
	\item QS World Grad School Tour Scholarships (QS Quacquarelli Symonds Limited): \url{http://graduateschool.topuniversities.com/world-grad-school-tour/scholarships}
	\item GlobalGrant (requires paid access to the list of scholarships and fellowships): \url{http://www.globalgrant.com/en/stipendier.html} and \url{http://www.globalgrant.com/}
	\item Stockholm University: \vspace{-0.2cm}
		\begin{enumerate} \itemsep -2pt
		\item \url{http://www.su.se/pub/jsp/polopoly.jsp?d=777&a=1770}
		\item \url{http://www.su.se/pub/jsp/polopoly.jsp?d=797}
		\item \url{http://www.su.se/pub/jsp/polopoly.jsp?d=788}
		\item \url{http://www.su.se/pub/jsp/polopoly.jsp?d=777&a=1769}
		\end{enumerate}
	\item NordForsk (in Norwegian): \url{http://www.nordforsk.org/index.cfm}
	\item Wallenberg Scholars (in Swedish): \url{http://www.wallenberg.com/default.aspx} or \url{http://www.wallenberg.com/in-english.aspx}
	\item Royal Institute of Technology (in Swedish): \url{http://www.kth.se/aktuellt/stipendier/stipendier-och-anslag-1.2024}
	\item European Commission: \vspace{-0.2cm}
		\begin{enumerate} \itemsep -2pt
		\item Marie Curie Fellowships: \vspace{-0.1cm}
			\begin{enumerate} \itemsep -1pt
			\item \url{http://cordis.europa.eu/fp7/people/home_en.html}
			\item \url{http://ec.europa.eu/research/mariecurieactions/}
			\item \url{http://ec.europa.eu/research/fp6/mariecurie-actions/action/fellow_en.html}
			\item \url{http://www.mariecurie.org/}
			\end{enumerate}
		\item Euraxess: \url{http://ec.europa.eu/euraxess/}
		\item \url{http://ec.europa.eu/index_en.htm}
		\end{enumerate}
	\item Science Please (for research positions in life sciences in The Netherlands and Belgium, including Ph.D. and postdoc positions): \url{http://www.scienceplease.com/} or \url{http://www.scienceplease.com/about-us}
	\item University of Gothenburg: \vspace{-0.2cm}
		\begin{enumerate} \itemsep -2pt
		\item ResearchResearch: \url{http://www.researchresearch.com/} or \url{http://www.gu.se/english/research/scholarships/ResearchResearch/}
		\item Scholarship links: \url{http://www.gu.se/english/research/scholarships/scholarship_links/}
		\item Scholarships at University of Gothenburg: \url{http://www.gu.se/english/research/scholarships/gu/}
		\end{enumerate}
	\item Princeton University; The Graduate School: \url{http://gradschool.princeton.edu/financial/}
	\item National Association for Bilingual Education: \vspace{-0.2cm}
		\begin{enumerate} \itemsep -2pt
		\item List of Scholarships: \url{http://www.nabe.org/scholarship.html}
		\end{enumerate}
	\item {\bf Pennsylvania State University}: \vspace{-0.2cm}
		\begin{enumerate} \itemsep -2pt
		\item Office of Engineering Diversity; Penn State College of Engineering: \vspace{-0.1cm}
			\begin{enumerate} \itemsep -1pt
			\item Undergraduate Student Scholarships: \url{http://www.engr.psu.edu/oed/UnderScholarships.html}
			\item Graduate Student Scholarships: \url{http://www.engr.psu.edu/oed/GradScholarships.html}
			\item High School Student Scholarships: \url{http://www.engr.psu.edu/oed/HighSchoolScholarships.html}
			\item Disabled Student Scholarships: \url{http://www.engr.psu.edu/oed/DisabScholarships.html}
			\item Corporate Office of Engineering Diversity (OED) Scholarships: \url{http://www.engr.psu.edu/oed/OEDScholarships.html}
			\end{enumerate}
		\item University Fellowships Office: \vspace{-0.1cm}
			\begin{enumerate} \itemsep -1pt
			\item \url{http://sites.google.com/site/psuufo/}
			\item Prestigious Scholarships: \url{http://sites.google.com/site/psuufo/prestigious}
			\item Penn State Scholarships: \url{http://sites.google.com/site/psuufo/internal-scholarships}
			\item Other resources: \url{http://sites.google.com/site/psuufo/resources}
			\end{enumerate}
		\end{enumerate}
	\item {\bf Peterson's} college search: \vspace{-0.2cm}
		\begin{enumerate} \itemsep -2pt
		\item {\it College Scholarship Search}: \url{http://www.petersons.com/college-search/scholarship-search.aspx}
		\end{enumerate}
	\item Society for Industrial and Applied Mathematics (SIAM): \vspace{-0.2cm}
		\begin{enumerate} \itemsep -2pt
		\item Fellowship \& Research Opportunities: \url{http://www.siam.org/students/resources/fellowship.php}
		\end{enumerate}
	\item Institute of International Education (IIE): \vspace{-0.2cm}
		\begin{enumerate} \itemsep -2pt
		\item {\it Funding for US Study Online}: \vspace{-0.1cm}
			\begin{enumerate} \itemsep -1pt
			\item \url{http://www.fundingusstudy.org/}
			\end{enumerate}
		\end{enumerate}
	\end{enumerate}
\item --- --- --- --- --- --- --- --- --- --- --- --- --- --- --- --- --- --- --- --- --- --- --- --- --- --- --- --- --- --- ---
\item \colorbox{blue}{\bf Scholarships and Fellowships in Electrical and Computer Engineering}
% Scholarships and Fellowships in Electrical and Computer Engineering
\item IEEE: \vspace{-0.3cm}
	\begin{enumerate} \itemsep -2pt
	\item IEEE Awards, Competitions, and Scholarships: \url{http://www.ieee.org/membership_services/membership/students/awards/index.html}
	\item IEEE Circuits and Systems Society Pre-Doctoral Scholarships: Announced via email from IEEE Circuits and Systems Society
	\item IEEE Power \& Energy Society: \vspace{-0.2cm}
		\begin{enumerate} \itemsep -2pt
		\item G. Ray Ekenstam Memorial Scholarship: \vspace{-0.1cm}
			\begin{enumerate} \itemsep -1pt
			\item \url{http://www.ieee-pes.org/g-ray-ekenstam-memorial-scholarship}
			\item ``The Scholarship Fund awards, on an annual basis, a scholarship to a qualified undergraduate student who seeks an electrical engineering degree in the field of power or a related discipline, from an accredited US university or college.''
			\end{enumerate}
		\end{enumerate}
	\item IEEE Reliability Society: \vspace{-0.2cm}
		\begin{enumerate} \itemsep -2pt
		\item IEEE Reliability Society Scholarship: \url{http://www.ieee.org/portal/cms_docs_relsoc/relsoc/newsflipper/RS_Scholarship_Application.pdf} [ Look under the tab/option on ``Useful Information'' in the panel on the left. ]
		\end{enumerate}
	\end{enumerate}
\item The George Michael Memorial HPC Fellowship Program: \vspace{-0.3cm}
	\begin{enumerate} \itemsep -2pt
	\item The Association of Computing Machinery (ACM), IEEE Computer Society and SC Conference series have established the High Performance Computing (HPC) Ph.D. Fellowship Program. The SC conference is the International Conference for High Performance Computing, Networking, Storage, and Analysis. IEEE Computer Society and the Association for Computing Machinery are the sponsors for this conference.
	\item Every year, up to three fellowship recipients will each receive a stipend of at least \$5,000 (U.S.) for one academic year, plus travel support to attend the SC conference.
	\item See \url{http://sc10.supercomputing.org/?searchterm=fellowship&pg=GeorgeMichaelMemorial.html}
	\end{enumerate}
\item Intel: \vspace{-0.3cm}
	\begin{enumerate} \itemsep -2pt
	\item Intel Foundation Fellowship: \vspace{-0.2cm}
		\begin{enumerate} \itemsep -2pt
		\item Intel Foundation Ph.D. Fellowship % \url{http://intelscholarships.intel.com/}
		\item \url{http://www.intel.com/education/highered/studentprograms/fellowship.htm}
		\item This awards two-year fellowships to Ph.D. candidates pursuing leading-edge work in fields related to Intel's business and research interests.
		\item Fellowships are available at select U.S. universities, by invitation only, and focus on Ph.D. students who have completed at least one year of study.
		\item The fellowship includes a cash award (tuition/fees/stipend), an Intel mentor, and the opportunity to participate in an internship at Intel.
		\end{enumerate}
	\end{enumerate}
\item IBM: \vspace{-0.3cm}
	\begin{enumerate} \itemsep -2pt
	\item \url{http://www-304.ibm.com/jct01005c/university/scholars/phdfellowship}
	\item IBM Ph.D. Fellowship Award
	\item The IBM Ph.D. Fellowship Awards is an intensely competitive program which honors exceptional Ph.D. students in many academic disciplines and areas of study, for example: computer science and engineering, electrical and mechanical engineering , physical sciences (including chemistry, material sciences, and physics), mathematical sciences (including optimization), business sciences (including financial services, communication, and learning/knowledge), and service sciences, management, and engineering.
	\item IBM Herman Goldstine Postdoctoral Fellowship in Mathematical Sciences: \url{http://domino.research.ibm.com/comm/research_projects.nsf/pages/goldstine.index.html}
	\item Josef Raviv Memorial Postdoctoral Fellowship; see \url{http://domino.research.ibm.com/comm/research.nsf/pages/d.compsci.josef.raviv.general.info.html}, \url{http://domino.research.ibm.com/comm/research.nsf/pages/d.compsci.raviv.winner.html}, and \url{http://domino.research.ibm.com/comm/research.nsf/pages/d.compsci.raviv.winner2008.html}
	\end{enumerate}
\item AMD: Ph.D. fellowship, \url{http://developer.amd.com/programs/fellowship/Pages/default.aspx}
\item Qualcomm, {\it Qualcomm Innovation Fellowship} for Ph.D. students in Electrical Engineering and Computer Science at Stanford, UC Berkeley, UCLA, UCSD, and USC: \url{http://www.qualcomm.com/innovation/research/university_relations/innovation_fellowship/qinf10.html}
\item NVIDIA: \vspace{-0.3cm}
	\begin{enumerate} \itemsep -2pt
	\item NVIDIA Fellowship Program; see \url{http://www.nvidia.com/page/fellowship_programs.html}
	\end{enumerate}
\item Automatic RF Techniques Group (ARFTG): \vspace{-0.3cm}
	\begin{enumerate} \itemsep -2pt
	\item Microwave Measurement Student Fellowship (for ``graduate students who show promise and interest in pursuing research related to improvement of radio frequency and microwave measurement techniques''): \url{http://www.arftg.org/student_fellowship.html}
	\end{enumerate}
\item Gallium Arsenide Applications Symposium (GAAS) Association: \vspace{-0.3cm}
	\begin{enumerate} \itemsep -2pt
	\item GAAS PhD Student Fellowship (for Ph.D. students who have accepted papers at the European Microwave Integrated Circuits Conference, EuMIC): \url{http://www.gaas-symposium.org/english/awards_fellowship.htm} and \url{http://www.eumweek.com/2010/EuMIC.asp?id=c}
	\end{enumerate}
\item The Institution of Engineering and Technology, IET: \vspace{-0.3cm}
	\begin{enumerate} \itemsep -2pt
	\item Hudswell International Research Scholarship: \url{http://www.theiet.org/about/scholarships-awards/ambition/postgraduate1/hudswell-what.cfm}
	\item IET Postgraduate Scholarship: \url{http://www.theiet.org/about/scholarships-awards/ambition/postgraduate1/postgrad-what.cfm}
	\end{enumerate}
\item --- --- --- --- --- --- --- --- --- --- --- --- --- --- --- --- --- --- --- --- --- --- --- --- --- --- --- --- --- --- ---
\item \colorbox{blue}{\bf Scholarships and Fellowships in Computer Science}
% Scholarships and Fellowships in Computer Science
\item ACM Special Interest Group on Symbolic and Algebraic Manipulation (SIGSAM): List of Ph.D. positions in computer algebra and symbolic computation, as listed by SIGSAM; see \url{http://www.sigsam.org/opportunities.phtml?searchterm=fellowship}
\item Carnegie Mellon University: \vspace{-0.3cm}
	\begin{enumerate} \itemsep -2pt
	\item women@SCS School of Computer Science: \vspace{-0.2cm}
		\begin{enumerate} \itemsep -2pt
		\item Individuals, Corporations \& Organizations: \url{http://women.cs.cmu.edu/Resources/Funding/}
		\end{enumerate}
	\end{enumerate}
\item IBM: \vspace{-0.3cm}
	\begin{enumerate} \itemsep -2pt
	\item \url{http://www-304.ibm.com/jct01005c/university/scholars/phdfellowship}
	\item IBM Ph.D. Fellowship Award
	\item The IBM Ph.D. Fellowship Awards is an intensely competitive program which honors exceptional Ph.D. students in many academic disciplines and areas of study, for example: computer science and engineering, electrical and mechanical engineering , physical sciences (including chemistry, material sciences, and physics), mathematical sciences (including optimization), business sciences (including financial services, communication, and learning/knowledge), and service sciences, management, and engineering.
	\item IBM Herman Goldstine Postdoctoral Fellowship in Mathematical Sciences: \url{http://domino.research.ibm.com/comm/research_projects.nsf/pages/goldstine.index.html}
	\item Josef Raviv Memorial Postdoctoral Fellowship; see \url{http://domino.research.ibm.com/comm/research.nsf/pages/d.compsci.josef.raviv.general.info.html}, \url{http://domino.research.ibm.com/comm/research.nsf/pages/d.compsci.raviv.winner.html}, and \url{http://domino.research.ibm.com/comm/research.nsf/pages/d.compsci.raviv.winner2008.html}
	\end{enumerate}
\item Computing Innovation Fellows (CIFellows); post my profile on \url{http://cifellows.org/profiles/}; also see \url{http://www.cifellows.org/}
\item Microsoft: \vspace{-0.3cm}
	\begin{enumerate} \itemsep -2pt
	\item Microsoft Research Graduate Women's Scholarship: \url{http://research.microsoft.com/en-us/collaboration/awards/fellows-women.aspx}
	\item Microsoft Research PhD Fellowship: \url{http://research.microsoft.com/en-us/collaboration/awards/apply-us.aspx}
	\end{enumerate}
\item Google: \vspace{-0.3cm}
	\begin{enumerate} \itemsep -2pt
	\item Google Fellowship Program; see \url{http://googleblog.blogspot.com/2009/05/best-and-brightest.html}
	\end{enumerate}
\item NVIDIA: \vspace{-0.3cm}
	\begin{enumerate} \itemsep -2pt
	\item NVIDIA Fellowship Program; see \url{http://www.nvidia.com/page/fellowship_programs.html}
	\end{enumerate}
\item Facebook Ph.D. Fellowship: \url{http://www.facebook.com/careers/fellowship.php}
\item Yahoo! Labs: Yahoo! Key Scientific Challenges Program, \url{http://labs.yahoo.com/ksc}
\item Qualcomm, {\it Qualcomm Innovation Fellowship} for Ph.D. students in Electrical Engineering and Computer Science at Stanford, UC Berkeley, UCLA, UCSD, and USC: \url{http://www.qualcomm.com/innovation/research/university_relations/innovation_fellowship/qinf10.html} and \url{http://www.qualcomm.com/innovation/research/university_relations/innovation_fellowship/}
\item Computing Research Association (CRA): Outstanding Undergraduate Researchers, \url{http://www.cra.org/awards/undergrad-current/}
\item {\color{blue} European Research Consortium for Informatics and Mathematics (ERCIM)}: \vspace{-0.3cm}
	\begin{enumerate} \itemsep -2pt
	\item ERCIM Alain Bensoussan Fellowship Programme (for Ph.D. degree holders in selected research areas): \url{http://fellowship.ercim.eu/} and \url{http://www.ercim.eu/news/283-fellowship-programme}; research areas are listed at: \url{http://fellowship.ercim.eu/home/topic}. Deadlines are on April 30 and September 30 annually.
	\end{enumerate}
\item {\it Theory Matters Wiki}; Theoretical Computer Science (TCS) Advocacy Wiki: \vspace{-0.3cm}
	\begin{enumerate} \itemsep -2pt
	\item Funding Opportunities and Tips: \url{http://theorymatters.org/pmwiki/pmwiki.php?n=Main.FundingOpportunities}
	\end{enumerate}
\item Kurt G{\"{o}}del Research Prize Fellowship: \vspace{-0.3cm}
	\begin{enumerate} \itemsep -2pt
	\item 2 Ph.D. (pre-doctoral) fellowships
	\item 2 post-doctoral fellowships
	\item 1 unrestricted fellowship
	\item $[$Scope of the$]$ original fellowship proposals [includes] the areas of: \vspace{-0.2cm}
		\begin{enumerate} \itemsep -2pt
		\item set theory
		\item recursion theory
		\item proof theory/intuitionism
		\item model theory
		\item computer assisted reasoning
		\item philosophy of mathematics 
		\end{enumerate}
	\item All fellowship proposals, regardless of subject area, will be judged according to: \vspace{-0.2cm}
		\begin{enumerate} \itemsep -2pt
		\item the relevance and resemblance of the research (finished and proposed) to the great insights and originality of Kurt G{\"{o}}del
		\item its general interest and clarity of motivation
		\item its rigorous scientific quality and depth. 
		\end{enumerate}
	\item \url{http://fellowship.logic.at/}
	\end{enumerate}
\item Hewlett-Packard Company: \vspace{-0.3cm}
	\begin{enumerate} \itemsep -2pt
	\item Hewlett-Packard Labs India (Bengaluru / Bangalore): \vspace{-0.2cm}
		\begin{enumerate} \itemsep -2pt
		\item {\it BITS - HP Labs India Ph.D. Fellowship} for Research related to Information Technologies: \vspace{-0.1cm}
			\begin{enumerate} \itemsep -1pt
			\item \url{http://www.hpl.hp.com/india/bits-hplindia_phd/index.html} or \url{http://www.hpl.hp.com/india/bits-hplindia_phd/}
			\item \url{http://www.hpl.hp.com/india/bits-hplindia_phd/iiitbphd.html}
			\item BITS, Pilani and HP Labs India jointly offer a unique PhD fellowship for research in Information and Communication Technologies (ICT) relevant to fast-growing markets like India.
			\item HP Labs India currently has ongoing Ph.D. Fellowships with BITS Pilani and IIIT, Bangalore: \url{http://www.hpl.hp.com/india/bits-hplindia_phd/university.html}
			\end{enumerate}
		\item Open Innovation Office: \vspace{-0.1cm}
			\begin{enumerate} \itemsep -1pt
			\item \url{http://www.hpl.hp.com/open_innovation/}
			\item HP Labs Innovation Research Program (IRP): \url{http://www.hpl.hp.com/open_innovation/irp/index.html}
			\end{enumerate}
		\end{enumerate}
	\end{enumerate}
\item Code for America (CfA): \vspace{-0.3cm}
	\begin{enumerate} \itemsep -2pt
	\item CfA Fellowship (develop web applications for local governments in the US): \url{http://codeforamerica.org/fellows/}
	\end{enumerate}
\item University of Minnesota, Twin Cities: \vspace{-0.3cm}
	\begin{enumerate} \itemsep -2pt
	\item College of Science and Engineering: \vspace{-0.2cm}
		\begin{enumerate} \itemsep -2pt
		\item Charles Babbage Institute: \vspace{-0.1cm}
			\begin{enumerate} \itemsep -1pt
			\item Adelle and Erwin Tomash Graduate Fellowship (for Ph.D. candidates doing research in the history of IT/computing - all but dissertation Ph.D. students only): \url{http://www.cbi.umn.edu/research/tfellowship.html}
			\item Arthur L. Norberg Travel Fund (short-term grants-in-aid to help scholars with travel expenses to use archival collections at the Charles Babbage Institute): \url{http://www.cbi.umn.edu/research/ntravelfund.html}
			\end{enumerate}
		\end{enumerate}
	\end{enumerate}
\item --- --- --- --- --- --- --- --- --- --- --- --- --- --- --- --- --- --- --- --- --- --- --- --- --- --- --- --- --- --- ---
\item \colorbox{blue}{\bf Scholarships and Fellowships in Biomedical Engineering}
% Scholarships and Fellowships in Biomedical Engineering
\item Whitaker International Fellows and Scholars Program: \vspace{-0.3cm}
	\begin{enumerate} \itemsep -2pt
	\item For graduate/Ph.D. students and postdocs in biomedical engineering
	\item \url{http://www.whitaker.org/home}
	\end{enumerate}
\item --- --- --- --- --- --- --- --- --- --- --- --- --- --- --- --- --- --- --- --- --- --- --- --- --- --- --- --- --- --- ---
\item \colorbox{blue}{\bf Scholarships and Fellowships in Optical Engineering}
% Scholarships and Fellowships in Optical Engineering
\item {\it SPIE} -- The International Society for Optical Engineering: \vspace{-0.3cm}
	\begin{enumerate} \itemsep -2pt
	\item ``SPIE Scholarship Program'' for undergraduates or graduate students studying optics, photonics, imaging, or optoelectronics program or related discipline (e.g., physics, electrical engineering): \url{http://spie.org//x1733.xml?WT.svl=mddm14}
	\item Other scholarships (including scholarships for students doing research in nanolithography techniques and lasers): \url{http://spie.org/x1736.xml}
	\end{enumerate}
\item {\it Kidger Optics Associates} Michael Kidger Memorial Scholarship (to a college freshman, or sophomore of optical design): \url{http://www.kidger.com/mkms_requirements.html}
\item --- --- --- --- --- --- --- --- --- --- --- --- --- --- --- --- --- --- --- --- --- --- --- --- --- --- --- --- --- --- ---
\item \colorbox{blue}{\bf Scholarships and Fellowships in Mechanical Engineering}
% Scholarships and Fellowships in Mechanical Engineering
\item American Society of Mechanical Engineers (ASME): \vspace{-0.3cm}
	\begin{enumerate} \itemsep -2pt
	\item Graduate Teaching Fellowships (for Ph.D. students in mechanical engineering): \url{http://www.asme.org/Education/College/FinancialAid/Graduate_Teaching_Fellowships.cfm}
	\item ASME Scholarships: \vspace{-0.2cm}
		\begin{enumerate} \itemsep -2pt
		\item \url{http://www.asme.org/Education/College/FinancialAid/Scholarships.cfm}
		\item US Undergraduates: \url{http://www.asme.org/Education/College/FinancialAid/US_Undergraduates.cfm}
		\item Graduate Students: \url{http://www.asme.org/Education/College/FinancialAid/Graduate_Students.cfm}
		\item International Students: \url{http://www.asme.org/Education/College/FinancialAid/International_Undergraduates.cfm}
		\end{enumerate}
	\item Auxiliary Scholarships: \vspace{-0.2cm}
		\begin{enumerate} \itemsep -2pt
		\item \url{http://www.asme.org/Education/College/FinancialAid/Auxiliary_Scholarships.cfm}
		\item Undergraduate Scholarships: \url{http://www.asme.org/Education/College/FinancialAid/Undergraduate_Scholarships.cfm}
		\item Graduate Scholarships: \url{http://www.asme.org/Education/College/FinancialAid/Graduate_Scholarships.cfm}
		\item Rice-Cullimore Scholarship (for international graduate students in the US): \url{http://www.asme.org/Education/College/FinancialAid/RiceCullimore_Scholarship.cfm}
		\end{enumerate}
	\item International Petroleum Institute�s College Scholarships (for undergraduates): \url{http://www.asme-ipti.org/public/pagscholarshipprograms.aspx}
	\item International Petroleum Institute�s Graduate Fellowship (for individuals entering a graduate program in mechanical engineering, and has an interest in the petroleum industry): \url{http://www.asme-ipti.org/public/pagscholarshipprograms.aspx} and \url{http://www.asme.org/Communities/Students/Grad/Fellowships.cfm}
	\end{enumerate}
\item --- --- --- --- --- --- --- --- --- --- --- --- --- --- --- --- --- --- --- --- --- --- --- --- --- --- --- --- --- --- ---
\item \colorbox{blue}{\bf Scholarships and Fellowships in Civil Engineering}
% Scholarships and Fellowships in Civil Engineering
\item American Society of Civil Engineers (ASCE): \vspace{-0.3cm}
	\begin{enumerate} \itemsep -2pt
	\item Jack E. Leisch Memorial National Graduate Fellowship (for graduate students in transportation/traffic engineering): \url{http://www.asce.org/Content.aspx?id=25021}
	\item Scholarships \& Fellowships (for undergraduates and graduate students): \url{http://www.asce.org/Content.aspx?id=18337}
	\end{enumerate}
\item American Concrete Institute (ACI): \vspace{-0.3cm}
	\begin{enumerate} \itemsep -2pt
	\item ACI Foundation Fellowships \& Scholarships: \url{http://www.concrete.org/STUDENTS/ST_SCHOLARSHIPS.HTM}
	\end{enumerate}
\item Institute of Transportation Engineers: \vspace{-0.3cm}
	\begin{enumerate} \itemsep -2pt
	\item Burton W. Marsh Fellowship for Graduate Study in Traffic and Transportation Engineering: \url{http://www.ite.org/education/Burton_W_MarshFellowship.asp}
	\end{enumerate}
\item --- --- --- --- --- --- --- --- --- --- --- --- --- --- --- --- --- --- --- --- --- --- --- --- --- --- --- --- --- --- ---
\item \colorbox{blue}{\bf Scholarships and Fellowships in Chemical Engineering}
% Scholarships and Fellowships in Chemical Engineering
\item American Institute of Chemical Engineers (AIChE) scholarships (includes scholarships for underrepresented minorities): \url{http://www.aiche.org/Students/Scholarships/index.aspx}
\item --- --- --- --- --- --- --- --- --- --- --- --- --- --- --- --- --- --- --- --- --- --- --- --- --- --- --- --- --- --- ---
\item \colorbox{blue}{\bf Scholarships and Fellowships in Aerospace Engineering}
% Scholarships and Fellowships in Aerospace Engineering
\item American Institute of Aeronautics and Astronautics (AIAA): \vspace{-0.3cm}
	\begin{enumerate} \itemsep -2pt
	\item AIAA Foundation Scholarships: \vspace{-0.2cm}
		\begin{enumerate} \itemsep -2pt
		\item \url{http://www.aiaa.org/content.cfm?pageid=211}
		\item For undergraduates and graduate students
		\item Named scholarships for undergraduates are: \vspace{-0.1cm}
			\begin{enumerate} \itemsep -1pt
			\item \url{http://www.aiaa.org/content.cfm?pageid=226}
			\item A. Thomas Young Scholarship
			\item L. S. ``Skip'' Fletcher Scholarship 
			\item Sam F. Iacobellis Scholarship
			\item Robert L. Crippen Scholarship
			\item E. C. ``Pete'' Aldridge Scholarship
			\item Liquid Propulsion Technical Committee Scholarship
			\item Space Transportation Technical Committee Scholarship
			\item Digital Avionics Technical Committee Scholarship (4)
			\item Next Century of Flight Scholarship (2)
			\item Leatrice Gregory Pendray Scholarship
			\end{enumerate}
		\item Awards for graduate students: \vspace{-0.1cm}
			\begin{enumerate} \itemsep -1pt
			\item \url{http://www.aiaa.org/content.cfm?pageid=227}
			\item Martin Summerfield Propellants and Combustion Graduate Award
			\item Guidance, Navigation, And Control Graduate Award
			\item Gordon C. Oates Air Breathing Propulsion Graduate Award
			\item William T. Piper, Sr. General Aviation Systems Graduate Award
			\item Orville and Wilbur Wright Graduate Award
			\item John Leland Atwood Graduate Award
			\item Open Topic Graduate Award
			\end{enumerate}
		\end{enumerate}
	\item Student Design Competition Award: \url{http://www.aiaa.org/content.cfm?pageid=401}
	\end{enumerate}
\item --- --- --- --- --- --- --- --- --- --- --- --- --- --- --- --- --- --- --- --- --- --- --- --- --- --- --- --- --- --- ---
\item \colorbox{blue}{\bf Scholarships and Fellowships in Mathematics}
% Scholarships and Fellowships in Mathematics
\item Association for Women in Mathematics (AWM): \vspace{-0.3cm}
	\begin{enumerate} \itemsep -2pt
	\item Travel grants: \url{http://sites.google.com/site/awmmath/programs/travel-grants}
	\item Alice T. Schafer Mathematics Prize for excellence in mathematics by an undergraduate woman: \url{http://sites.google.com/site/awmmath/programs/schafer-prize}
	\item The {\it Ruth I. Michler Memorial Prize} of the AWM is awarded annually to a woman recently promoted to Associate Professor or an equivalent position in the mathematical sciences: \url{http://sites.google.com/site/awmmath/programs/michler-prize}
	\end{enumerate}
\item Seth Bonder Scholarship for Applied Operations Research in Health Services: \url{http://www.informs.org/Recognize-Excellence/INFORMS-Community-Prizes-and-Awards/Seth-Bonder-Scholarship-for-Applied-Operations-Research-in-Health-Services}
\item Oberwolfach Foundation: \vspace{-0.3cm}
	\begin{enumerate} \itemsep -2pt
	\item Oberwolfach Prize (for young European mathematicians): \url{http://www.mfo.de/programme/prize/}
	\item John Todd Fellowship (or John Todd Award) [for young excellent mathematicians working in numerical analysis]: \url{http://www.mfo.de/programme/todd/}
	\end{enumerate}
\item Clay Mathematics Institute: Clay Research Award, \url{http://www.claymath.org/research_award/}
\item --- --- --- --- --- --- --- --- --- --- --- --- --- --- --- --- --- --- --- --- --- --- --- --- --- --- --- --- --- --- ---
\item \colorbox{blue}{\bf Scholarships and Fellowships in Science}
% Scholarships and Fellowships in Science
\item {\it Science.gov} (USA.gov for Science): \vspace{-0.3cm}
	\begin{enumerate} \itemsep -2pt
	\item Internship and Fellowship Opportunities in Science for Undergraduate Students: \url{http://www.science.gov/internships/undergrad.html}
	\item Graduate Students/Postdoctoral Fellowships: \url{http://www.science.gov/internships/graduate.html}
	\end{enumerate}
\item Heinz Family Philanthropies: \vspace{-0.3cm}
	\begin{enumerate} \itemsep -2pt
	\item Teresa Heinz Scholars for Environmental Research program (for Ph.D./MS students working on their thesis in environmental science/engineering) at selected universities: \url{http://www.heinzfamily.org/programs/environmentalscholars.html}
	\item \url{http://www.heinzfamily.org/}
	\end{enumerate}
\item Mayo Clinic: \vspace{-0.3cm}
	\begin{enumerate} \itemsep -2pt
	\item Postbaccalaureate Research Education Program (PREP): \url{http://www.mayo.edu/mgs/postbac-program.html}
	\end{enumerate}
\item {\it American Chemical Society (ACS)}: \vspace{-0.3cm}
	\begin{enumerate} \itemsep -2pt
	\item ACS-Hach Land Grant Undergraduate Scholarship (for chemistry undergraduates at a partner institution of ACS, and who plan to become chemistry teachers in US high schools): \url{http://portal.acs.org/portal/acs/corg/content?_nfpb=true&_pageLabel=PP_SUPERARTICLE&node_id=2243&use_sec=false&sec_url_var=region1&__uuid=eb054647-53e0-4594-81e8-8ef49159f3f4}
	\item ACS-Hach Second Career Teacher Scholarship (for graduates in chemistry or related areas who are entering an education masters program or teacher certification program): \url{http://portal.acs.org/portal/acs/corg/content?_nfpb=true&_pageLabel=PP_SUPERARTICLE&node_id=2244&use_sec=false&sec_url_var=region1&__uuid=4c27333f-4aad-481e-aaa4-f1db045d4eb4}
	\item ACS Scholars Program (for undergraduate underrepresented minorities majoring in chemistry, biochemistry, or chemical engineering): \url{http://portal.acs.org/portal/acs/corg/content?_nfpb=true&_pageLabel=PP_SUPERARTICLE&node_id=1650&use_sec=false&sec_url_var=region1&__uuid=b3b583cf-18ae-4fb0-9375-33f75a0ccf49}
	\item Scholarships: \url{http://portal.acs.org/portal/acs/corg/content?_nfpb=true&_pageLabel=PP_TRANSITIONMAIN&node_id=630&use_sec=false&sec_url_var=region1&__uuid=98e85c05-be75-4283-a97c-7a63ab4c3178}
	\end{enumerate}
\item European Molecular Biology Organization: \vspace{-0.3cm}
	\begin{enumerate} \itemsep -2pt
	\item EMBO Short-Term Fellowships (for junior researchers, including Ph.D. students): \url{http://www.embo.org/programmes/fellowships/short-term.html}
	\item EMBO Long-Term Fellowships (for junior researchers/postdocs): \url{http://www.embo.org/programmes/fellowships/long-term.html}
	\end{enumerate}
\item L'OR{\'{E}}AL: \vspace{-0.3cm}
	\begin{enumerate} \itemsep -2pt
	\item ``For Women in Science'' program: \url{http://www.lorealusa.com/forwomeninscience} or \url{http://www.lorealusa.com/_en/_us/index.aspx?direct1=00008&direct2=00008/00001}
	\item Alternatively, go to \url{http://www.lorealusa.com/_en/_us/} and select the ``For Women in Science'' tab.
	\item Check out the ``L'Or{\'{e}}al USA Fellowships for Women in Science'' (US postdocs), ``UNESCO-L'Or{\'{e}}al Fellowships for Women in Science'' (for female Ph.D. students and postdocs in the life sciences), and the ``L'Or{\'{e}}al-UNESCO Awardss for Women in Science'' (for distinguished female scientists)
	\end{enumerate}
\item American Institute of Physics (AIP): \vspace{-0.3cm}
	\begin{enumerate} \itemsep -2pt
	\item AIP and Member Society Government Science Fellowships: \vspace{-0.2cm}
		\begin{enumerate} \itemsep -2pt
		\item \url{http://www.aip.org/gov/fellowships.html}
		\item American Institute of Physics State Department Science Fellowship: \url{http://www.aip.org/gov/fellowships/sdf.html}
		\item American Institute of Physics Congressional Science Fellowship: \url{http://www.aip.org/gov/fellowships/cf.html}
		\item American Physical Society Congressional Science Fellowship: \url{http://www.aps.org/policy/fellowships/congressional.cfm}
		\item American Geophysical Union Congressional Science Fellowship: \url{http://www.agu.org/sci_pol/cong_fellowship/}
		\item Optical Society of America Congressional Science Fellowships: \url{http://www.osa.org/about_osa/public_policy/congressional_fellowships/default.aspx}
		\item For US citizens with good track records in research
		\end{enumerate}
	\item American Geophysical Union: \vspace{-0.2cm}
		\begin{enumerate} \itemsep -2pt
		\item Research Grants and Awards: \url{http://www.agu.org/about/honors/research_grants/}
		\item Student Travel Grants: \url{http://www.agu.org/education/grants/travel.shtml}
		\item Research Grants \& Awards: \url{http://www.agu.org/education/grants/research.shtml}
		\item Mass Media Fellowship: \url{http://www.agu.org/news/mass_media_fellowship/}
		\end{enumerate}
	\item Society of Physics Students (SPS): \vspace{-0.2cm}
		\begin{enumerate} \itemsep -2pt
		\item SPS Scholarships: \url{http://www.spsnational.org/programs/scholarships/}
		\item SPS Awards: \url{http://www.spsnational.org/programs/awards/}
		\end{enumerate}
	\end{enumerate}
\item Consortium for Ocean Leadership: \vspace{-0.3cm}
	\begin{enumerate} \itemsep -2pt
	\item Employment, Internships, and Opportunities [ includes funding opportunities for researchers (professors, postdocs, and grad students) ]: \url{http://www.oceanleadership.org/about-ocean-leadership/ocean-of-opportunities/}
	\item HBCU Fellowship: Ocean Leadership/IODP-USIO for Students of Historically Black Colleges and Universities, \url{http://www.oceanleadership.org/education/diversity/hbcu-fellowship/}
	\item HBCU Educator at Sea: \url{http://www.oceanleadership.org/education/diversity/hbcu-educator/}
	\item MS PHD's Professional Development Program: The Minorities Striving and Pursuing Higher Degrees of Success in the Earth System Sciences (MS PHD'S) Professional Development Program, \url{http://www.oceanleadership.org/education/diversity/ms-phds-professional-development-program/}
	\item Schlanger Ocean Drilling Fellowship Program (merit-based awards for outstanding graduate students to conduct research related to the Integrated Ocean Drilling Program): \url{http://www.oceanleadership.org/programs-and-partnerships/usssp/schlanger-fellowship/}
	\end{enumerate}
\item American Geological Institute Foundation: \vspace{-0.3cm}
	\begin{enumerate} \itemsep -2pt
	\item William L. Fisher Congressional Geoscience Fellowship (for young geoscientists to get engaged in {\bf public policy}): \url{http://www.agifoundation.org/govtaffairs.html} and \url{http://www.agifoundation.org/endowments.html}
	\item AGI Minority Participation Program: Minority Participation Program Geoscience Student Scholarships for ``underrepresented ethnic-minority (undergraduate or graduate) students in the geosciences'', \url{http://www.agiweb.org/mpp/index.html}
	\end{enumerate}
\item Lady Davis Institute/Jewish General Hospital: \vspace{-0.3cm}
	\begin{enumerate} \itemsep -2pt
	\item Awards for ``graduate students (in biomedical science) and post-doctoral fellows/clinical fellows'': \url{http://www.ladydavis.ca/en/awards}
	\end{enumerate}
\item Adolph C. and Mary Sprague Miller Institute for Basic Research in Science: \vspace{-0.3cm}
	\begin{enumerate} \itemsep -2pt
	\item Miller Fellowships (for outstanding recent Ph.D.s / postdoctoral fellowship): \url{http://millerinstitute.berkeley.edu/topage.php?nav=11&to=1} or \url{http://millerinstitute.berkeley.edu/page.php?nav=11}
	\item Visiting Miller Research Professorships (for professors and research scientists): \url{http://millerinstitute.berkeley.edu/topage.php?nav=24&to=1} or \url{http://millerinstitute.berkeley.edu/page.php?nav=24}
	\item Miller Research Professorships (for professors in the UC system): \url{http://millerinstitute.berkeley.edu/topage.php?nav=15&to=1} or \url{http://millerinstitute.berkeley.edu/page.php?nav=15}
	\item Miller Senior Fellowships (Nominations are solicited by invitation only; Senior Fellow appointments are made to tenured UC Berkeley faculty for five years, possibly renewable for a subsequent five years, but no longer.): \url{http://millerinstitute.berkeley.edu/topage.php?nav=126&to=1}
	\end{enumerate}
\item Funda{\c{c}}{\~{a}}o para a Ci{\^{e}}ncia e a Tecnologia (FCT); Minist{\'{e}}rio da Ci{\^{e}}ncia, Technologia e Ensino Superior (MCTES): International Prize Fernando Gil in Philosophy of Science, \url{http://alfa.fct.mctes.pt/apoios/premios/fernando_gil/index.phtml.pt}
\item Wellcome Trust: \vspace{-0.3cm}
	\begin{enumerate} \itemsep -2pt
	\item Wellcome Trust Sanger Institute: \vspace{-0.2cm}
		\begin{enumerate} \itemsep -2pt
		\item \url{http://www.sanger.ac.uk/workstudy/}
		\item Postdoctoral fellows (for research in genomics): \url{http://www.sanger.ac.uk/workstudy/career/postdocs/}
		\item Graduate program (for research in genomics): \url{http://www.sanger.ac.uk/workstudy/phd/}
		\item Student placements and work experience (for research in genomics): \url{http://www.sanger.ac.uk/workstudy/placements/}
		\end{enumerate}
	\end{enumerate}
\item Paul B. Beeson Career Development Awards in Aging Research Program (formerly the Beeson Physician Faculty Scholars Program): \vspace{-0.3cm}
	\begin{enumerate} \itemsep -2pt
	\item \url{http://www.beeson.org/}
	\item ``Today, the Beeson program continues to foster the independent research careers of clinically trained investigators -- a growing cadre of talented physician-scientists -- whose research and leadership are enhancing the health and quality of life of Americans, particularly older people.''
	\item About the Program: \url{http://www.beeson.org/program_hx.cfm}
	\end{enumerate}
\item American Mathematical Society: \vspace{-0.3cm}
	\begin{enumerate} \itemsep -2pt
	\item AMS Fellowships and Scholarships: \vspace{-0.2cm}
		\begin{enumerate} \itemsep -2pt
		\item \url{http://e-math.ams.org/programs/ams-fellowships/ams-fellowships}
		\item AMS Centennial Research Fellowship Program: \url{http://e-math.ams.org/programs/ams-fellowships/centennial-fellow/emp-centflyer}
		\item Waldemar J. Trijitzinsky Memorial Awards: \url{http://e-math.ams.org/programs/ams-fellowships/trjitzinsky/trjitzinsky-award}
		\item Other Sources of Funding: \url{http://e-math.ams.org/programs/funding/funding}
		\end{enumerate}
	\end{enumerate}
\item --- --- --- --- --- --- --- --- --- --- --- --- --- --- --- --- --- --- --- --- --- --- --- --- --- --- --- --- --- --- ---
\item \colorbox{blue}{\bf Scholarships and Fellowships in Medicine}
% Scholarships and Fellowships in Medicine
\item Sarnoff Medical Student Research Fellowship Program (for US medical students interested in cardiovascular research): \url{http://www.sarnoffendowment.org/}
\item Mayo Clinic: \vspace{-0.3cm}
	\begin{enumerate} \itemsep -2pt
	\item Postbaccalaureate Research Education Program (PREP): \url{http://www.mayo.edu/mgs/postbac-program.html}
	\end{enumerate}
\item Paul B. Beeson Career Development Awards in Aging Research Program (formerly the Beeson Physician Faculty Scholars Program): \vspace{-0.3cm}
	\begin{enumerate} \itemsep -2pt
	\item \url{http://www.beeson.org/}
	\item ``Today, the Beeson program continues to foster the independent research careers of clinically trained investigators -- a growing cadre of talented physician-scientists -- whose research and leadership are enhancing the health and quality of life of Americans, particularly older people.''
	\item About the Program: \url{http://www.beeson.org/program_hx.cfm}
	\end{enumerate}
\item --- --- --- --- --- --- --- --- --- --- --- --- --- --- --- --- --- --- --- --- --- --- --- --- --- --- --- --- --- --- ---
\item \colorbox{blue}{\bf Scholarships and Fellowships in Science and Engineering}
% Scholarships and Fellowships in Science and Engineering
\item National Academies: \vspace{-0.3cm}
	\begin{enumerate} \itemsep -2pt
	\item Research Associateship Programs (graduate, postdoctoral, and senior level research opportunities): \url{http://sites.nationalacademies.org/pga/rap/}
	\item Ford Foundation Fellowship Programs (predoctoral, dissertation or postdoctoral fellowships for individuals seeking academic careers in science and engineering): \url{http://sites.nationalacademies.org/PGA/FordFellowships/index.htm}
	\item \url{http://nationalacademies.org/grantprograms.html}
	\item \url{http://sites.nationalacademies.org/pga/fellowships/}
	\item List of Fellowship, Scholarship, and Grant Databases: \url{http://sites.nationalacademies.org/PGA/Fellowships/PGA_046300}
	\item List of Outside Fellowships, Scholarships, and Grants Websites: \url{http://sites.nationalacademies.org/PGA/Fellowships/PGA_046301}
	\item Awards for junior and mid-career researchers: \url{http://www.nasonline.org/site/PageServer?pagename=AWARDS_main}
	\item National Academy of Engineering, NAE: \vspace{-0.2cm}
		\begin{enumerate} \itemsep -2pt
		\item NAE Grand Challenges Scholars Program: \url{http://www.grandchallengescholars.org/}
		\end{enumerate}
	\item National Science Foundation: \vspace{-0.2cm}
		\begin{enumerate} \itemsep -2pt
		\item International Research Fellowship Program (IRFP) for junior scientists and engineers: \url{http://www.nsf.gov/funding/pgm_summ.jsp?pims_id=5179}
		\item Integrative Graduate Education and Research Traineeship Program (IGERT) for undergraduates and graduate students in STEM: \url{http://www.nsf.gov/funding/pgm_summ.jsp?pims_id=12759}
		\item National Science Foundation's Graduate Research Fellowship Program (GRFP) for students seeking research degrees in STEM: \url{http://www.nsfgrfp.org/}
		\item NSF Alliances for Graduate Education and the Professoriate (AGEP) program (to help underrepresented minorities obtain graduate degrees in STEM and prepare them for faculty positions in academia): \url{http://www.nsfagep.org/}
		\item National Science Foundation's (NSF) East Asia and Pacific Summer Institutes (EAPSI) program: \vspace{-0.1cm}
			\begin{enumerate} \itemsep -1pt
			\item \url{http://www.nsf.gov/funding/pgm_summ.jsp?pims_id=5284}
			\item The East Asia and Pacific Summer Institutes (EAPSI) provide U.S. graduate students in science and engineering: \vspace{-0.1cm}
				\begin{itemize} \itemsep -1pt
				\item first-hand research experiences in Australia, China, Japan, Korea, New Zealand, Singapore or Taiwan
				\item an introduction to the science, science policy, and scientific infrastructure of the respective location
				\item an orientation to the society, culture and language.
				\end{itemize}
			\item ``The primary goals of EAPSI are to introduce students to East Asia and Pacific science and engineering in the context of a research setting, and to help students initiate scientific relationships that will better enable future collaboration with foreign counterparts.''
			\item ``All institutes, except Japan, last approximately eight weeks from June to August. Japan lasts approximately ten weeks from June to August (specific dates are available and updated at \url{http://www.nsfsi.org/}).''
			\item Example of Ph.D. student, Jakub Szefer, from Prof. Ruby Lee's lab at Princeton University, who interned with Prof. Cheng Chen-Mou from National Taiwan University: \url{http://www.nsf.gov/discoveries/disc_summ.jsp?cntn_id=118116&org=NSF}
			\end{enumerate}
		\end{enumerate}
	\end{enumerate}
\item United States Department of Defense (DoD): \vspace{-0.3cm}
	\begin{enumerate} \itemsep -2pt
	\item National Defense Education Program; Defense Advanced Research Projects Agency (DARPA): \vspace{-0.2cm}
		\begin{enumerate} \itemsep -2pt
		\item Science, Mathematics, and Research for Transformation (SMART) scholarship program: \vspace{-0.1cm}
			\begin{itemize} \itemsep -1pt
			\item \url{http://smart.asee.org/}
			\item Co-organized by the American Society for Engineering Education
			\end{itemize}
		\item National Security Science and Engineering Faculty Fellowships (NSSEFF): \url{http://www.ndep.us/ProgNSSEFF.aspx}
		\end{enumerate}
	\end{enumerate}
\item National Society of Professional Engineers: \vspace{-0.3cm}
	\begin{enumerate} \itemsep -2pt
	\item Scholarships for undergraduates and graduate students: \url{http://www.nspe.org/Students/Scholarships/index.html}
	\item NSPE-PEC George B. Hightower, P.E. Fellowship (for an outstanding engineering graduate student): \url{http://www.nspe.org/InterestGroups/PEC/Resources/Awards/hightower_fellowship.html}
	\item PEG Management Fellowship: \vspace{-0.2cm}
		\begin{enumerate} \itemsep -2pt
		\item \url{http://www.nspe.org/InterestGroups/PEG/Resources/AwardsAndScholarships/peg_fellowship.html}
		\item ``This scholarship is designed for graduate students who are pursuing an MBA, a master's degree in engineering management, or a master's degree in public administration.''
		\end{enumerate}
	\end{enumerate}
\item Technion -- Israel Institute of Technology: \vspace{-0.3cm}
	\begin{enumerate} \itemsep -2pt
	\item Department of Mathematics: Anna and Paul Erdos postdoctoral Fellowship, \url{http://www.math.technion.ac.il/Site/people/positions.html}
	\item Lady Davis Postdoctoral Fellowship
	\item Department of Electrical Engineering: \vspace{-0.2cm}
		\begin{enumerate} \itemsep -2pt
		\item The Andrew and Erna Finci Viterbi Fellowship Program (for graduate and post-doctoral fellows), \url{http://webee.technion.ac.il/Research/Fellowship-Programs}
		\item Lady Davis Fellowship Trust: Technion Fellowships (for visiting professors, post-doctoral researchers, as well as Masters and Ph.D. students), \url{http://ldft.huji.ac.il/upload/info/}
		\item \url{http://webee.technion.ac.il/Research/Fellowship-Programs}
		\end{enumerate}
	\end{enumerate}
\item Hebrew University: \vspace{-0.3cm}
	\begin{enumerate} \itemsep -2pt
	\item Lady Davis Fellowship Trust: Technion Fellowships (for visiting professors, post-doctoral researchers, as well as Masters and Ph.D. students), \url{http://ldft.huji.ac.il/upload/info/infoHUa.html}
	\end{enumerate}
\item Hertz Foundation: \vspace{-0.3cm}
	\begin{enumerate} \itemsep -2pt
	\item The Graduate Fellowship Award: \url{http://www.hertzfoundation.org/dx/Fellowships/award.aspx}
	\item Thesis Prize: \url{http://www.hertzfoundation.org/dx/Fellowships/thesis_winners.aspx}
	\end{enumerate}
\item Krell Institute, Inc.: \vspace{-0.3cm}
	\begin{enumerate} \itemsep -2pt
	\item DOE Computational Science Graduate Fellowship: \url{http://www.krellinst.org/csgf/index.shtml}
	\end{enumerate}
\item The Winston Churchill Foundation of the United States: \vspace{-0.3cm}
	\begin{enumerate} \itemsep -2pt
	\item The Churchill Scholarship: \url{http://winstonchurchillfoundation.org/index.php?hide=1&section=eligibility}
	\end{enumerate}
\item American Society for Engineering Education: \vspace{-0.3cm}
	\begin{enumerate} \itemsep -2pt
	\item \url{http://blogs.asee.org/fellowships/}
	\item Fellowship programs: \url{http://www.asee.org/fellowship-programs}
	\item Awards: \url{http://www.asee.org/member-resources/awards/full-list-of-awards}
	\item DuPont Minorities in Engineering Award: \vspace{-0.2cm}
		\begin{enumerate} \itemsep -2pt
		\item \url{http://www.asee.org/member-resources/awards/full-list-of-awards/national-awards/special#DuPont_Minorities_in_Engineering_Award}
		\item {\bf \color{blue} ``The DuPont Minorities in Engineering Award is conferred for outstanding achievements by an engineering or engineering technology educator in increasing student diversity within engineering and engineering technology programs.''}
		\end{enumerate}
	\end{enumerate}
\item Alexander von Humboldt-Stiftung/Foundation: \vspace{-0.3cm}
	\begin{enumerate} \itemsep -2pt
	\item Feodor Lynen Research Fellowship for Postdoctoral Researchers (junior postdocs): \url{http://www.humboldt-foundation.de/web/feodor-lynen-fellowship-postdoc.html}
	\item Friedrich Wilhelm Bessel Research Award (mid-career researchers): \url{http://www.humboldt-foundation.de/web/bessel-award.html}
	\item Georg Forster Research Fellowship for Postdoctoral Researchers (for non-German junior postdocs ``with above average qualifications''): \url{http://www.humboldt-foundation.de/web/georg-forster-fellowship-postdoc.html}
	\item Humboldt Research Fellowship for Postdoctoral Researchers (junior postdocs): \url{http://www.humboldt-foundation.de/web/771.html}
	\item Sofja Kovalevskaja Award (junior postdocs): \url{http://www.humboldt-foundation.de/web/kovalevskaja-award.html}
	\item Fraunhofer-Bessel Research Award: \url{http://www.humboldt-foundation.de/web/fraunhofer-bessel-award.html}
	\item \url{http://www.humboldt-foundation.de/web/home.html}
	\end{enumerate}
\item Santa Fe Institute: Omidyar Postdoctoral Fellowship; see \url{http://www.santafe.edu/education/fellowships/omidyar-postdoctoral/}
\item Applied Materials: Applied Materials Graduate Fellowship
\item American Society of Naval Engineers (ASNE): \vspace{-0.3cm}
	\begin{enumerate} \itemsep -2pt
	\item (Undergraduate and Graduate) Scholarships: \url{http://www.navalengineers.org/awards/scholarships/Pages/ASNELandingPage.aspx}
	\end{enumerate}
\item Lindau Meeting of Nobel Laureates and Students in Lindau (Oak Ridge Associated Universities, ORAU): \vspace{-0.3cm}
	\begin{enumerate} \itemsep -2pt
	\item Graduate Student Award program: \vspace{-0.2cm}
		\begin{enumerate} \itemsep -2pt
		\item \url{http://www.orau.org/lindau/}
		\item A student nominated to participate in this program must: \vspace{-0.1cm}
			\begin{enumerate} \itemsep -1pt
			\item Be a U.S. citizen
			\item Be currently enrolled as a full-time graduate student
			\item Be currently sponsored by, or working on, and supported by projects sponsored by, the agency to which the nomination is made, such as the U.S. Department of Energy Office of Science, the National Institutes of Health or other federal agency
			\item Have completed by June 2011 two years (but not more than four years) of study toward a doctoral degree in medicine or physiology, or in a related discipline, including the basic biomedical (or life) sciences
			\end{enumerate}
		\end{enumerate}
	\end{enumerate}
\item Research Councils UK (RCUK): \vspace{-0.3cm}
	\begin{enumerate} \itemsep -2pt
	\item RCUK Academic Fellowships: \vspace{-0.2cm}
		\begin{enumerate} \itemsep -2pt
		\item \url{http://www.rcuk.ac.uk/ResearchCareers/fellowships/Pages/home.aspx}
		\item \url{http://www.rcuk.ac.uk/ResearchCareers/fellowships/Pages/about.aspx}
		\item Dorothy Hodgkin Postgraduate Awards: \vspace{-0.1cm}
			\begin{enumerate} \itemsep -1pt
			\item \url{http://www.rcuk.ac.uk/ResearchCareers/dhpa/Pages/home.aspx}
			\item ``Dorothy Hodgkin Postgraduate Awards (DHPA) is a UK scheme to bring outstanding students from India, China, Hong Kong, South Africa, Brazil, Russia and the developing world to come and study for PhDs in top rated UK research facilities.''
			\end{enumerate}
		\end{enumerate}
	\item International Funding Opportunities: \vspace{-0.2cm}
		\begin{enumerate} \itemsep -2pt
		\item \url{http://www.rcuk.ac.uk/international/funding/FundingOpps/Pages/home.aspx}
		\item Early Career Researchers: \url{http://www.rcuk.ac.uk/international/funding/FundingOpps/Pages/EarlyCareer.aspx}
		\end{enumerate}
	\item Engineering and Physical Sciences Research Council: \vspace{-0.2cm}
		\begin{enumerate} \itemsep -2pt
		\item Programs: \vspace{-0.1cm}
			\begin{enumerate} \itemsep -1pt
			\item Physical sciences: \vspace{-0.1cm}
				\begin{itemize} \itemsep -1pt
				\item Organic synthetic chemistry studentships: \url{http://www.epsrc.ac.uk/about/progs/physsci/Pages/organicstudentships.aspx}
				\item Analytical science studentships: \url{http://www.epsrc.ac.uk/about/progs/physsci/Pages/analyticalstudentships.aspx}
				\end{itemize}
			\item Mathematical sciences: \vspace{-0.1cm}
				\begin{itemize} \itemsep -1pt
				\item Fellowships (for postdoctoral research): \url{http://www.epsrc.ac.uk/about/progs/maths/Pages/fellowships.aspx}
				\end{itemize}
			\end{enumerate}
		\item Funding: \vspace{-0.1cm}
			\begin{enumerate} \itemsep -1pt
			\item \url{http://www.epsrc.ac.uk/funding/Pages/default.aspx}
			\item Grants available [has funds for (new/junior) professors and to support international collaboration]: \url{http://www.epsrc.ac.uk/funding/grants/Pages/default.aspx}
			\item Calls for proposals (open/current funding calls for applications and future/proposed calls): \url{http://www.epsrc.ac.uk/funding/calls/Pages/default.aspx}
			\item Studentships (training grants for Ph.D. and Masters students, including international students): \url{http://www.epsrc.ac.uk/funding/students/Pages/default.aspx}
			\item Fellowships (from junior scientists and engineers engaged in postdoctoral research to senior researchers): \url{http://www.epsrc.ac.uk/funding/fellows/Pages/default.aspx}
			\end{enumerate}
		\end{enumerate}
	\item Biotechnology and Biological Sciences Research Council (BBSRC): \vspace{-0.2cm}
		\begin{enumerate} \itemsep -2pt
		\item ``The UK's leading funding agency for academic research and training in the non-clinical life sciences''
		\item Funding research: \vspace{-0.1cm}
			\begin{enumerate} \itemsep -1pt
			\item \url{http://www.bbsrc.ac.uk/funding/funding-index.aspx}
			\item Fellowships (for early career scientists, for supporting individuals seeking a change in research directions or scientists who are returning to research, and senior researchers): \url{http://www.bbsrc.ac.uk/funding/fellowships/fellowships-index.aspx}
			\item Studentships (Doctoral training grants, Masters training grants, postgraduate awards, and undergraduate research grants): \url{http://www.bbsrc.ac.uk/funding/studentships/studentships-index.aspx}
			\item Special opportunities (current calls for funding): \url{http://www.bbsrc.ac.uk/funding/opportunities/opportunities-index.aspx}
			\item Apply for funding (information about the process of applying for research funds): \url{http://www.bbsrc.ac.uk/funding/apply/apply-index.aspx}
			\end{enumerate}
		\end{enumerate}
	\item Science and Technology Facilities Council: \vspace{-0.2cm}
		\begin{enumerate} \itemsep -2pt
		\item STFC Grants and Awards: \vspace{-0.1cm}
			\begin{enumerate} \itemsep -1pt
			\item \url{http://www.stfc.ac.uk/Funding+and+Grants/501.aspx}
			\item ``The Science and Technology Facilities Council offers grants and support in Particle Physics, Astronomy, Nuclear Physics and Facility Development. It also provides support for research infrastructure, training, knowledge exchange and public engagement activities through a variety of funding schemes and activities.''
			\item STFC Funding Opportunities: \url{http://www.stfc.ac.uk/Funding%20and%20Grants/598.aspx}
			\item Postgraduate Studentships: \url{http://www.stfc.ac.uk/Funding+and+Grants/637.aspx} or \url{http://www.stfc.ac.uk/Funding%20and%20Grants/636.aspx}
			\end{enumerate}
		\item Fellowship opportunities: \vspace{-0.1cm}
			\begin{enumerate} \itemsep -1pt
			\item \url{http://www.stfc.ac.uk/Funding%20and%20Grants/508.aspx}
			\item ``Fellowship opportunities in Astronomy, Solar and Planetary Science, Particle Physics, Particle Astrophysics, Nuclear Physics, Development of STFC Neutron, Laser and Synchrotron Facilities within the UK.''
			\item There are postdoctoral and advanced research fellowships.
			\end{enumerate}
		\item Innovations Partnership Schemes (IPS and mini-IPS): \url{http://www.stfc.ac.uk/19213.aspx}
		\item IPS Fellowships: \vspace{-0.1cm}
			\begin{enumerate} \itemsep -1pt
			\item \url{http://www.stfc.ac.uk/19226.aspx}
			\item The IPS fellowship is a scheme designed to support a role to develop the commercial exploitation of technologies. This is not a research orientated fellowship.
			\end{enumerate}
		\item Follow-on-Funding: \vspace{-0.1cm}
			\begin{enumerate} \itemsep -1pt
			\item \url{http://www.stfc.ac.uk/19207.aspx}
			\item ``Follow on Funding is intended to provide financial support at the very early or pre-seed stage of turning research outputs into a commercial proposition. Unlike the other research councils, in STFC, industry partners are not allowed. If you have an industry partner, please use the mini-IPS or IPS scheme.''
			\item ``STFC staff, grant funded academics and researchers at CERN and ESO are eligible to apply for follow-on-funds (see the research grants handbook for CERN and ESO eligibility). STFC staff should first investigate whether they can be funded through proof of concept funding.''
			\end{enumerate}
		\end{enumerate}
	\item Natural Environment Research Council: \vspace{-0.2cm}
		\begin{enumerate} \itemsep -2pt
		\item Grants and studentships on the web: \vspace{-0.1cm}
			\begin{enumerate} \itemsep -1pt
			\item \url{http://www.nerc.ac.uk/research/gotw.asp}
			\item Grants on the web: \url{http://gotw.nerc.ac.uk/goti.asp?c=1}
			\end{enumerate}
		\item Funding: \vspace{-0.1cm}
			\begin{enumerate} \itemsep -1pt
			\item \url{http://www.nerc.ac.uk/funding/}
			\item Postgraduate training: \vspace{-0.1cm}
				\begin{itemize} \itemsep -1pt
				\item Postgraduate eligibility (requires UK/EU citizenship): \url{http://www.nerc.ac.uk/funding/available/postgrad/eligibility.asp}
				\end{itemize}
			\item Research Fellowship Scheme [for all nationalities]: \url{http://www.nerc.ac.uk/funding/available/fellowships/}
			\item Research Experience Placements (REP) scheme [for undergraduates]: \url{http://www.nerc.ac.uk/funding/available/rep.asp}
			\item Research Grants: \vspace{-0.1cm}
				\begin{itemize} \itemsep -1pt
				\item Eligibility: \url{http://www.nerc.ac.uk/funding/available/researchgrants/eligibility.asp}
				\end{itemize}
			\end{enumerate}
		\item {\bf Other potential sources of funding}: \vspace{-0.1cm}
			\begin{enumerate} \itemsep -1pt
			\item \url{http://www.nerc.ac.uk/funding/otherfunding.asp}
			\item Look at the ``Higher Education Funding Councils'' for each country (England, Wales, Northern Ireland, and Scotland)
			\end{enumerate}
		\end{enumerate}
	\end{enumerate}
\item Nuffield Foundation: \vspace{-0.3cm}
	\begin{enumerate} \itemsep -2pt
	\item Undergraduate research bursaries in science: \url{http://www.nuffieldfoundation.org/undergraduate-research-bursaries-0}
	\item Funding for social policy projects in the UK: \vspace{-0.2cm}
		\begin{enumerate} \itemsep -2pt
		\item \url{http://www.nuffieldfoundation.org/social-policy}
		\item \url{http://www.nuffieldfoundation.org/children-and-families-law-society-education-and-open-door}
		\end{enumerate}
	\item Apply for funding: \url{http://www.nuffieldfoundation.org/apply-for-funding}
	\item Africa program: \url{http://www.nuffieldfoundation.org/africa-programme-0}
	\item Nuffield Farming Scholarships Trust: \vspace{-0.2cm}
		\begin{enumerate} \itemsep -2pt
		\item Nuffield Farming Scholarships: \url{http://www.nuffieldscholar.org/}
		\end{enumerate}
	\item The Nuffield Trust (or, The Nuffield Trust for Research and Policy Studies in Health Services): \vspace{-0.2cm}
		\begin{enumerate} \itemsep -2pt
		\item Fellowships: \vspace{-0.1cm}
			\begin{enumerate} \itemsep -1pt
			\item \url{http://www.nuffieldtrust.org.uk/fellowships/index.aspx?id=43}
			\item Rock Carling fellowship (for senior researchers in public health): \url{http://www.nuffieldtrust.org.uk/fellowships/index.aspx?id=112}
			\item John Fry Fellowship (for senior researchers in public health): \url{http://www.nuffieldtrust.org.uk/fellowships/index.aspx?id=109}
			\item Harkness Fellowships in Health Care Policy: \vspace{-0.1cm}
				\begin{itemize} \itemsep -1pt
				\item ``Since September 2009 The Nuffield Trust have been the proud co-sponsors of the prestigious Harkness Fellowships programme with The Commonwealth Fund.''
				\item ``These offer an unparalleled opportunity for the health policy analysts of the future to conduct original research and learn about healthcare in North America.''
				\item ``Mid-career health policy researchers and practitioners (including doctors, health services managers, journalists and government officials) are supported to spend 9 to 12 months in the United States conducting a policy-oriented research project and working with leading U.S. health policy experts.''
				\end{itemize}
			\end{enumerate}
		\end{enumerate}
	\end{enumerate}
\item U.S. Department of Homeland Security (DHS): \vspace{-0.3cm}
	\begin{enumerate} \itemsep -2pt
	\item DHS Scholarship and Fellowship Program: \url{http://www.orau.gov/dhsed/}
	\end{enumerate}
\item ACT, Inc.: \vspace{-0.3cm}
	\begin{enumerate} \itemsep -2pt
	\item Barry M. Goldwater Scholarship and Excellence in Education Program (for US residents who will be college upperclassmen in STEM fields in the following academic year): \url{http://www.act.org/goldwater/}
	\end{enumerate}
\item Massachusetts Institute of Technology: \vspace{-0.3cm}
	\begin{enumerate} \itemsep -2pt
	\item MIT School of Engineering: \vspace{-0.2cm}
		\begin{enumerate} \itemsep -2pt
		\item Lemelson-MIT Program: \vspace{-0.1cm}
			\begin{enumerate} \itemsep -1pt
			\item \url{http://web.mit.edu/invent/}
			\item Lemelson-MIT Awards for Invention and Innovation: \url{http://web.mit.edu/invent/a-main.html}
			\end{enumerate}
		\end{enumerate}
	\end{enumerate}
\item --- --- --- --- --- --- --- --- --- --- --- --- --- --- --- --- --- --- --- --- --- --- --- --- --- --- --- --- --- --- ---
\item \colorbox{blue}{\bf Scholarships and Fellowships in Various Fields (Including Creative Arts, Teaching, and Sports)}
% Scholarships and Fellowships in Various Fields (Including Creative Arts, Teaching, and Sports)
\item U.S. Department of Education: \vspace{-0.3cm}
	\begin{enumerate} \itemsep -2pt
	\item Robert C. Byrd Honors Scholarship Program: \vspace{-0.2cm}
		\begin{enumerate} \itemsep -2pt
		\item High school graduates who have been accepted for enrollment at institutions of higher education (IHEs), have demonstrated outstanding academic achievement, and show promise of continued academic excellence may apply to states in which they are residents.
		\item \url{http://www2.ed.gov/programs/iduesbyrd/index.html}
		\end{enumerate}
	\item \colorbox{yellow}{\bf Jacob K. Javits Fellowships Program}: \vspace{-0.1cm}
		\begin{enumerate} \itemsep -1pt
		\item This program provides fellowships to students of superior academic ability -- selected on the basis of demonstrated achievement, financial need, and exceptional promise -- to undertake study at the doctoral and Master of Fine Arts level in selected fields of arts, humanities, and social sciences.
		\item \url{http://www2.ed.gov/programs/jacobjavits/index.html}
		\end{enumerate}
	\item Close Up Fellowship Program: \vspace{-0.2cm}
		\begin{enumerate} \itemsep -2pt
		\item This program provides financial aid to enable low-income students, their teachers, and recent immigrants to come to Washington, D.C., to study the operations of the three branches of the federal government.
		\item \url{http://www2.ed.gov/programs/closeup/index.html}
		\end{enumerate}
	\item {\bf \color{blue} B.J. Stupak Olympic Scholarships}: \vspace{-0.2cm}
		\begin{enumerate} \itemsep -2pt
		\item This program provides financial assistance to athletes who are training at the U.S. Olympic Education Center or one of the U.S. Olympic training centers and who are pursuing a postsecondary education at institutions of higher education (IHEs).
		\item \url{http://www2.ed.gov/programs/olympic/index.html}
		\end{enumerate}
	\item {\bf \color{blue} Teacher Education Assistance for College and Higher Education (TEACH) Grant Program}: \vspace{-0.2cm}
		\begin{enumerate} \itemsep -2pt
		\item Through the College Cost Reduction and Access Act of 2007, Congress created the Teacher Education Assistance for College and Higher Education (TEACH) Grant Program that provides grants of up to \$4,000 per year to students who intend to teach in a public or private elementary or secondary school that serves students from low-income families.
		\item \url{http://studentaid.ed.gov/PORTALSWebApp/students/english/TEACH.jsp}
		\end{enumerate}
	\item Scholarship search engine: \url{https://studentaid2.ed.gov/getmoney/scholarship/}
	\item Financial Aid: \vspace{-0.2cm}
		\begin{enumerate} \itemsep -2pt
		\item \url{http://www2.ed.gov/finaid/landing.jhtml?src=rt}
		\item \url{http://studentaid.ed.gov/PORTALSWebApp/students/english/funding.jsp}
		\item Paying for college: \url{http://www.college.gov}
		\item Student Aid (has information for students at all levels and parents): \url{http://studentaid.ed.gov/}
		\item Student Aid Eligibility: \url{http://studentaid.ed.gov/PORTALSWebApp/students/english/aideligibility.jsp?tab=funding}
		\item Federal Student Aid: \url{http://federalstudentaid.ed.gov/}
		\item Academic Competitiveness Grant: The Academic Competitiveness Grant provides up to \$750 for the first year of undergraduate study and up to \$1,300 for the second year of undergraduate study. See \url{http://studentaid.ed.gov/PORTALSWebApp/students/english/NewPrograms.jsp}.
		\end{enumerate}
	\item Free Application for Federal Student Aid (FAFSA): \vspace{-0.2cm}
		\begin{enumerate} \itemsep -2pt
		\item Financial Aid Estimator Tool (FAFSA4caster): \url{http://www.fafsa4caster.ed.gov/F4CApp/index/index.jsf}
		\item \url{http://www.fafsa.ed.gov/}
		\end{enumerate}
	\item Federal Pell Grant Program: \url{http://www2.ed.gov/programs/fpg/index.html}
	\end{enumerate}
\item European Commission: \vspace{-0.3cm}
	\begin{enumerate} \itemsep -2pt
	\item Erasmus Programme (for Europeans): \url{http://ec.europa.eu/education/lifelong-learning-programme/doc80_en.htm}
	\item Erasmus Mundus (for non-Europeans): \url{http://ec.europa.eu/education/external-relation-programmes/doc72_en.htm}
	\end{enumerate}
\item Woodrow Wilson Foundation: \vspace{-0.3cm}
	\begin{enumerate} \itemsep -2pt
	\item {\bf \color{blue} The Woodrow Wilson-Rockefeller Brothers Fund Fellowships for Aspiring Teachers of Color (for underrepresented minorities seeking a career as a K-12 public school teacher in the US)}: \url{http://www.woodrow.org/teaching-fellowships/wwrbf/index.php}
	\item {\bf \color{blue} Woodrow Wilson Teaching Fellowship (for a MS program in teacher education, who would teach at high-need urban and rural schools or $\ge$ 3 years)}: \url{http://www.wwteachingfellowship.org/}
	\item {\bf \color{blue} Leonore Annenberg Teaching Fellowship (for a MS program in teacher education, who would teach at high-need urban and rural schools or $\ge$ 3 years)}: \url{http://www.woodrow.org/teaching-fellowships/annenberg/index.php}
	\item MMUF Travel \& Research Grants (for graduate students who participated in the Mellon Mays Undergraduate Fellowship Program): \url{http://www.woodrow.org/higher-education-fellowships/opportunity/research/index.php}
	\item MMUF Dissertation Grants (for graduate students who participated in the Mellon Mays Undergraduate Fellowship Program): \url{http://www.woodrow.org/higher-education-fellowships/opportunity/dissertation/index.php}
	\item Charlotte W. Newcombe Doctoral Dissertation Fellowship (for Ph.D. students writing their theses on ethical or religious values in all fields of the humanities and social sciences): \url{http://www.woodrow.org/higher-education-fellowships/religion_ethics/index.php}
	\item {\bf \color{blue} Woodrow Wilson Dissertation Fellowship in Women�s Studies}: \url{http://www.woodrow.org/higher-education-fellowships/women_gender/index.php}
	\item Doris Duke Conservation Fellowship program (Masters students seeking careers as practicing conservationists): \url{http://www.woodrow.org/higher-education-fellowships/conservation/index.php}
	\item Thomas R. Pickering Graduate Foreign Affairs Fellowship: \vspace{-0.2cm}
		\begin{enumerate} \itemsep -2pt
		\item Prior to joining the United States Department of State Foreign Service, this fellowship supports students entering a Masters program in the following fields: \vspace{-0.1cm}
			\begin{enumerate} \itemsep -1pt
			\item {\bf public policy}
			\item international affairs
			\item public administration
			\item academic fields such as: \vspace{-0.1cm}
				\begin{itemize} \itemsep -1pt
				\item business
				\item economics
				\item political science
				\item sociology
				\item foreign languages
				\end{itemize}
			\end{enumerate}
		\item \url{http://www.woodrow.org/higher-education-fellowships/foreign_affairs/pickering_grad/index.php}
		\end{enumerate}
	\item Thomas R. Pickering Undergraduate Foreign Affairs Fellowship (for undergraduates seeking to join the United States Department of State Foreign Service): \url{http://www.woodrow.org/higher-education-fellowships/foreign_affairs/pickering_undergrad/index.php}
	\end{enumerate}
\item Burroughs Wellcome Fund: \vspace{-0.3cm}
	\begin{enumerate} \itemsep -2pt
	\item Career Awards for Medical Scientists (post-Ph.D.): \url{http://www.bwfund.org/pages/188/Career-Awards-for-Medical-Scientists/}
	\item {\bf \color{blue} Career Award for Science and Mathematics Teachers (science or mathematics K-12 teachers in North Carolina public schools)}: \url{http://www.bwfund.org/pages/379/Career-Awards-for-Science-and-Mathematics-Teachers/}
	\end{enumerate}
\item Susan G. Komen for the Cure\textregistered: The Komen College Scholarship Program, \url{http://ww5.komen.org/ResearchGrants/CollegeScholarshipAward.html}
\item University of Kansas Madison \& Lila Self Graduate Fellowship (Ph.D. fellowships for business, economics, and STEM): \url{http://www2.ku.edu/~selfpro/}
\item Nationally Coveted College Scholarships, Graduate School Fellowships \& Postdoctoral Awards: \url{http://scholarships.fatomei.com/}
\item The Andrew W. Mellon Foundation: \vspace{-0.3cm}
	\begin{enumerate} \itemsep -2pt
	\item Fellowships \& Scholarships for undergraduates: \url{http://www.mmuf.org/undergraduates/explore-your-opportunities/fellowships-scholorships}
	\end{enumerate}
\item Siebel Scholars Foundation: \vspace{-0.3cm}
	\begin{enumerate} \itemsep -2pt
	\item For students in selected business, bioengineering, and computer science graduate programs
	\item Only available for students at selected universities.
	\item \url{http://www.siebelscholars.com/scholars}
	\item \url{http://www.siebelscholars.com/}
	\end{enumerate}
\item Aspen Institute (for leaders, e.g. in business, education, community service, and politics): \vspace{-0.3cm}
	\begin{enumerate} \itemsep -2pt
	\item Catto Fellowship Program: \url{http://www.aspeninstitute.org/leadership-programs/catto-fellowship-program}
	\item Rodel Fellowship Program: \url{http://www.aspeninstitute.org/leadership-programs/aspen-institute-rodel-fellowships-public-le-/about-rodel-fellowship-program}
	\item Henry Crown Fellowship Program: \url{http://www.aspeninstitute.org/leadership-programs/henry-crown-fellowship-program}
	\end{enumerate}
\item Smithsonian Institution: \vspace{-0.3cm}
	\begin{enumerate} \itemsep -2pt
	\item Postdoctoral Fellowships, Predoctoral Fellowships, and Graduate Student Fellowships: \vspace{-0.2cm}
		\begin{enumerate} \itemsep -2pt
		\item \url{http://www.si.edu/ofg/infotoapply.htm}
		\item \url{http://www.si.edu/ofg/fell.htm}
		\item \url{http://www.si.edu/ofg/ofgapp.htm}
		\item fields of research and study: \vspace{-0.1cm}
			\begin{enumerate} \itemsep -1pt
			\item {\bf \color{blue} American History, American Material and Folk Culture, and the History of Music and Musical Instruments}
			\item History of Science and Technology
			\item {\bf \color{blue} History of Art, Design, Crafts, and the Decorative Arts}
			\item Anthropology, Archaeology, Linguistics, and Ethnic Studies
			\item Evolutionary, Systematic, Behavioral, Environmental, and Conservation Biology
			\item Earth, Mineral, and Planetary Science
			\item Materials Characterization and Conservation
			\end{enumerate}
		\end{enumerate}
	\item Internship opportunities: \url{http://www.si.edu/ofg/internopp.htm}
	\item Research centers: \url{http://www.si.edu/research/}. [ It also has lots of information for K-12 teachers. It has resources, funding, and internship opportunities for undergraduates and graduate students pursing research in various aspects of humanities, social science, and natural science. ]
	\item Freer Gallery of Art / Arthur M. Sackler Gallery: \vspace{-0.2cm}
		\begin{enumerate} \itemsep -2pt
		\item Fellowships: \url{http://www.asia.si.edu/research/fellowships.asp}
		\end{enumerate}
	\item National Museum of American History: \vspace{-0.2cm}
		\begin{enumerate} \itemsep -2pt
		\item Jerome and Dorothy Lemelson Center for the Study of Invention and Innovation: \vspace{-0.1cm}
			\begin{enumerate} \itemsep -1pt
			\item The Lemelson Center Fellows Program (for Ph.D. students and postdocs): \url{http://invention.smithsonian.org/resources/research_fellowships.aspx}
			\end{enumerate}
		\end{enumerate}
	\end{enumerate}
\item Intercollegiate Studies Institute (ISI): \vspace{-0.3cm}
	\begin{enumerate} \itemsep -2pt
	\item William E. Simon Fellowship for Noble Purpose (for American undergraduates who are planning to use the fellowship grant for serving humanity -- in their own ways): \url{http://www.isi.org/programs/fellowships/simon.html}
	\item {\bf \color{blue} Richard M. Weaver Fellowship (for Americans who are attending a graduate program and are intending to pursue a career in academia/teaching)}: \url{http://www.isi.org/programs/fellowships/richard_weaver.html}
	\item Western Civilization Fellowships (for Americans who are attending a graduate program about Western culture/civilization): \url{http://www.isi.org/programs/fellowships/western_civilization.html}
	\item Salvatori Fellowship (for Americans who are attending a graduate program about early American history): \url{http://www.isi.org/programs/fellowships/salvatori.html}
	\item Bache Renshaw Fellowship for Doctoral Study in Education (for Americans who plan to attend doctoral programs in education): \url{http://www.isi.org/programs/fellowships/bache_renshaw.html}
	\item \url{http://www.isi.org/programs/fellowships/fellowships.html}
	\end{enumerate}
\item Le Fonds qu{\'{e}}b{\'{e}}cois de la recherche sur la nature et les technologies (The Quebec Research Fund on nature and technology): \vspace{-0.3cm}
	\begin{enumerate} \itemsep -2pt
	\item Scholarships: \url{http://www.fqrnt.gouv.qc.ca/en/bourses/index.htm}
	\end{enumerate}
\item Horatio Alger Association of Distinguished Americans, Inc.: \vspace{-0.3cm}
	\begin{enumerate} \itemsep -2pt
	\item Scholarship Programs (for US high school seniors who have faced and overcome great obstacles in their young lives): \url{https://www.horatioalger.org/scholarships/sp.cfm}
	\item Awards: \vspace{-0.2cm}
		\begin{enumerate} \itemsep -2pt
		\item \url{http://www.horatioalger.org/aboutus.cfm}
		\item Horatio Alger Award: ``dedicated community leaders who demonstrate individual initiative and a commitment to excellence; as exemplified by remarkable achievements accomplished through honesty, hard work, self-reliance and perseverance over adversity''
		\item International Horatio Alger Award: ``recipients of this award must have overcome humble beginnings and/or adversity to achieve success. They serve as outstanding role models to the international community and are committed to the Association's mission of encouraging and educating today's young people.''
		\item Norman Vincent Peale Award: ``a Member who has made exceptional humanitarian contributions to society, who has been an active participant in the Association, and who continues to exhibit courage, tenacity and integrity in the face of great challenges. ''
		\end{enumerate}
	\end{enumerate}
\item The W. Garfield Weston Foundation: \vspace{-0.3cm}
	\begin{enumerate} \itemsep -2pt
	\item Entrance Awards \& Upper Year Garfield Weston Awards (for students pursuing college or CEGEP studies in Canada): \url{http://www.garfieldwestonawards.ca/en/about}
	\end{enumerate}
\item Canadian Merit Scholarship Foundation (\url{http://www.cmsf.ca/}): Loran Award (undergraduate funding for Canadian citizens and permanent residents), \url{http://www.loranaward.ca/}
\item StartingBloc: \vspace{-0.3cm}
	\begin{enumerate} \itemsep -2pt
	\item StartingBloc Fellowship: \vspace{-0.2cm}
		\begin{enumerate} \itemsep -2pt
		\item \url{http://www.startingbloc.org/fellowship}
		\item For people who believe that economic value creation and social value creation are complementary... For people who believe in making money and doing good, and creating social and economic impact... 
		\item The Institute for Social Innovation is a ``conference'' to learn about global issues, ``corporate social responsibility, social entrepreneurship, cross sector partnerships and sustainability. Sessions are led by top academics, corporate innovators, social entrepreneurs, activists and government officials.'' 
		\end{enumerate}
	\end{enumerate}
\item The John D. and Catherine T. MacArthur Foundation: \vspace{-0.3cm}
	\begin{enumerate} \itemsep -2pt
	\item Applying for Grants: \url{http://www.macfound.org/site/c.lkLXJ8MQKrH/b.913959/k.E1BE/Applying_for_Grants.htm}
	\item Financial \& Grant Information: \url{http://www.macfound.org/site/c.lkLXJ8MQKrH/b.938093/k.9E4C/Financial__Grant_Information.htm}
	\item MacArthur Fellows Program: \url{http://www.macfound.org/site/c.lkLXJ8MQKrH/b.959463/k.9D7D/Fellows_Program.htm}
	\end{enumerate}
\item Wenner-Gren Foundations (The Wenner-Gren Center Foundation for Scientific Research, The Axel Wenner-Gren Foundation for International Exchange of Scientists and The Foundation Wenner-Grenska Samfundet): Fellowships (for Swedish postdocs), \url{http://www.swgc.org/stipendier.aspx}
\item {\'{E}}gide: \vspace{-0.3cm}
	\begin{enumerate} \itemsep -2pt
	\item EGIDE Latitudes: \url{http://www.egidelatitudes.fr/jahia/Jahia/site/egidelatitudes}
	\item Call for applications to scholarship opportunities (including a scholarship for French citizens to study abroad): \url{http://www.egide.asso.fr/jahia/Jahia/accueil/appels}
	\item Eiffel excellence scholarship programme (organized by the French Ministry of Foreign and European Affairs): \vspace{-0.2cm}
		\begin{enumerate} \itemsep -2pt
		\item \url{http://www.egide.asso.fr/jahia/Jahia/appels/eiffel}
		\item For non-French citizens pursuing advanced degrees.
		\end{enumerate}
	\end{enumerate}
\item Gottlieb Daimler and Karl Benz Foundation: \vspace{-0.3cm}
	\begin{enumerate} \itemsep -2pt
	\item {\bf \color{blue} Ph.D. fellowship for international students to study in Germany}; see \url{http://www.daimler-benz-stiftung.de/home/fellowship/en/start.html}
	\end{enumerate}
\item The San Diego Foundation: \vspace{-0.3cm}
	\begin{enumerate} \itemsep -2pt
	\item San Diego Foundation Community Scholarship Program: \vspace{-0.2cm}
		\begin{enumerate} \itemsep -2pt
		\item \url{http://www.sdfoundation.org/GrantsScholarships/Scholarships.aspx}
		\item Available scholarships: \url{http://www.sdfoundation.org/GrantsScholarships/Scholarships/ForStudents/AvailableScholarships.aspx}. Also, see \url{http://www.sdfoundation.org/GrantsScholarships/Scholarships/ForStudents/AvailableScholarships/CommonApplicationScholarships.aspx#twomey}
		\item It has scholarships for: \vspace{-0.1cm}
			\begin{enumerate} \itemsep -1pt
			\item graduating high school seniors
			\item current undergraduates
			\item non-traditional college students: \vspace{-0.1cm}
				\begin{itemize} \itemsep -1pt
				\item mature-age students
				\item mature student
				\item adult learner
				\item adult student
				\item adults who are returning to college
				\end{itemize}
			\item people pursuing teaching certificates
			\item students attending grad school
			\item students attending trade/vocational school
			\item foster youth
			\item students in various ethnic groups
			\item students in different geographical locations
			\item {\bf \color{blue} students pursuing education in certain fields, such as engineering, nursing, music, and arts and humanities}
			\end{enumerate}
		\item Separate Scholarships: \url{http://www.sdfoundation.org/GrantsScholarships/Scholarships/ForStudents/AvailableScholarships/SeparateScholarships.aspx}
		\item Other Scholarships and Financial Aid Resources: \url{http://www.sdfoundation.org/GrantsScholarships/Scholarships/ForStudents/AvailableScholarships/OtherScholarshipsandFinancialAidResources.aspx}
		\item Financial Aid Information: \url{http://www.sdfoundation.org/GrantsScholarships/Scholarships/ForStudents/Resources/FinancialAidInformation.aspx}
		\end{enumerate}
	\item Grant Opportunities (for non-profit organizations): \url{http://www.sdfoundation.org/GrantsScholarships/ForNonprofits/GrantOpportunities.aspx}
	\end{enumerate}
\item Ewing Marion Kauffman Foundation: \vspace{-0.3cm}
	\begin{enumerate} \itemsep -2pt
	\item Kauffman Dissertation Fellowship Program (for ``Ph.D., D.B.A., or other doctoral students at accredited U.S. universities to support dissertations in the area of entrepreneurship''): \url{http://www.kauffman.org/research-and-policy/kauffman-dissertation-fellowship-program.aspx}
	\item Kauffman Junior Faculty Fellowship in Entrepreneurship Research: \vspace{-0.2cm}
		\begin{enumerate} \itemsep -2pt
		\item \url{http://www.kauffman.org/research-and-policy/kauffman-junior-faculty-fellowship-in-entrepreneurship.aspx}
		\item ``to recognize tenured or tenure-track junior faculty members at accredited U.S. universities who are beginning to establish a record of scholarship and exhibit the potential to make significant contributions to the body of research in the field of entrepreneurship''
		\end{enumerate}
	\item Ewing Marion Kauffman Prize Medal for Distinguished Research in Entrepreneurship (for promising young scholars in the field of entrepreneurship): \url{http://www.kauffman.org/research-and-policy/kauffman-prize-medal-for-entrepreneurship-research.aspx}
	\item Kauffman Legal Fellowship Program (for post-J.D. research fellowship): \url{http://www.kauffman.org/research-and-policy/kauffman-legal-fellowship-program.aspx}
	\item Kauffman Global Scholars Program (for non-American top young entrepreneurs): \url{http://www.kauffman.org/entrepreneurship/kauffman-global-scholars-program.aspx}
	\item Entrepreneur Fellows program (for M.D.s and Ph.D.s who want to become high-tech start-up entrepreneurs): \url{http://www.kauffman.org/entrepreneurship/entrepreneur-fellows-program.aspx}
	\item Entrepreneur Postdoctoral Fellows program (for postdocs who want to become high-tech start-up entrepreneurs): \url{http://www.kauffman.org/entrepreneurship/entrepreneur-postdoctoral-fellows-program.aspx}
	\item Kauffman Fellows Program (``to educate and train future venture capitalists and future leaders of high-growth companies''): \url{http://www.kauffman.org/entrepreneurship/kauffman-fellows.aspx}
	\item Kauffman Foundation Outstanding Postdoctoral Entrepreneur Award: \url{http://www.kauffman.org/entrepreneurship/outstanding-postdoctoral-entrepreneur-award.aspx}
	\end{enumerate}
\item Killam Fellowships Program: \vspace{-0.3cm}
	\begin{enumerate} \itemsep -2pt
	\item \url{http://www.killamfellowships.com/}
	\item The Killam Fellowships Program allows undergraduate students from Canada and the United States to participate in a program of binational residential exchange.
	\item Killam Fellows spend either one semester or a full academic year as an exchange student in the host country.
	\end{enumerate}
\item Canada Council for the Arts: \vspace{-0.3cm}
	\begin{enumerate} \itemsep -2pt
	\item Killam Research Fellowship: \vspace{-0.2cm}
		\begin{enumerate} \itemsep -2pt
		\item \url{http://killam.canadacouncil.ca/welcome.asp}
		\item For researchers in the following fields, and interdisciplinary fields between these fields: \vspace{-0.1cm}
			\begin{enumerate} \itemsep -1pt
			\item humanities
			\item social sciences
			\item natural sciences
			\item health sciences
			\item engineering
			\end{enumerate}
		\item For outstanding researchers who are Canadian citizens or permanent residents
		\end{enumerate}
	\item Killam Prizes (and Killam Research Fellowships): \url{http://www.canadacouncil.ca/prizes/killam}
	\end{enumerate}
\item Killam Trusts: \vspace{-0.3cm}
	\begin{enumerate} \itemsep -2pt
	\item Killam Scholarship and Prize Programs (multiple fields in selected Canadian universities): \url{http://www.killamtrusts.ca/index.asp}
	\item Killam Award winners: \url{http://www.killamtrusts.ca/awardwinners.asp}
	\item Killam Scholarship and Prize Programs at various institutions (including universities): \url{http://www.killamtrusts.ca/uofAlberta.asp}
	\end{enumerate}
\item U.S. Department of State: \vspace{-0.3cm}
	\begin{enumerate} \itemsep -2pt
	\item Bureau of Educational and Cultural Affairs: \vspace{-0.2cm}
		\begin{enumerate} \itemsep -2pt
		\item Institute of International Education (administrator of program): \vspace{-0.1cm}
			\begin{enumerate} \itemsep -1pt
			\item Council for International Exchange of Scholars: \vspace{-0.1cm}
				\begin{itemize} \itemsep -1pt
				\item Fulbright Programs (for U.S. and non-U.S. Scholars): \url{http://www.cies.org/Fulbright_programs.htm}; \url{http://www.cies.org/about_fulb.htm}; \url{http://us.fulbrightonline.org/about.html}; \url{http://foreign.fulbrightonline.org/}; \url{http://exchanges.state.gov/academicexchanges/index/fulbright-program.html}; and \url{http://fulbright.state.gov/}
				\item Hubert H. Humphrey Fellowship Program: \vspace{-0.1cm}
					\begin{itemize} \itemsep -1pt
					\item For mid-career professionals in the following fields: economic development/finance and banking, agricultural and rural development, natural resources, environmental policy, and climate change, human resource management, communications/journalism, teaching of English as a foreign language, educational administration, planning, and policy, substance abuse education, treatment, and prevention, HIV/AIDS policy and prevention, public health policy and management, {\bf public policy} analysis and public administration, law and human rights, urban and regional planning, trafficking in persons - policy and prevention, technology policy and management, and higher education administration
					\item \url{http://www.humphreyfellowship.org/}
					\item \url{http://exchanges.state.gov/globalexchanges/humphrey-fellowship.html}
					\end{itemize}
				\end{itemize}
			\item International programs for scholars (search under each continent): \url{http://www.iie.org/en/Our-Global-Reach}
			\end{enumerate}
		\item International Documentary Filmmakers Fellowship: \vspace{-0.1cm}
			\begin{enumerate} \itemsep -1pt
			\item \url{http://exchanges.state.gov/cultural/docfilmmakers.html}
			\item \url{http://smpa.gwu.edu/doccenter/fellowship.php}
			\item For ``emerging or mid-career documentary filmmakers''
			\item Intensive six-week program at the Documentary Center, The George Washington University
			\end{enumerate}
		\item Office of English Language Programs: \vspace{-0.1cm}
			\begin{enumerate} \itemsep -1pt
			\item English Language Fellow Program (for ``highly qualified U.S. educators in the field of Teaching English to Speakers of Other Languages, TESOL''): \url{http://exchanges.state.gov/englishteaching/el-fellow.html}
			\item English Language Specialist Program: \vspace{-0.1cm}
				\begin{itemize} \itemsep -1pt
				\item \url{http://exchanges.state.gov/englishteaching/el-specialist.html}
				\item U.S. academics in the fields of Teaching English as a Foreign Language (TEFL) / Teaching English as a Second Language (TESL) and Applied Linguistics
				\end{itemize}
			\item E-Teacher Scholarship Program (for English teaching professionals living outside of the United States): \url{http://exchanges.state.gov/englishteaching/eteacher.html}
			\item English Access Microscholarship Program (Access): \vspace{-0.1cm}
				\begin{itemize} \itemsep -1pt
				\item \url{http://exchanges.state.gov/englishteaching/eam.html}
				\item The English Access Microscholarship Program (Access) provides a foundation of English language skills to non-elite, 14 - 18 year old students through afterschool classes and intensive summer learning activities.
				\end{itemize}
			\item \url{http://exchanges.state.gov/englishteaching/index.html}
			\end{enumerate}
		\item Office of Global Educational Programs: \vspace{-0.1cm}
			\begin{enumerate} \itemsep -1pt
			\item Community College Initiative: \vspace{-0.1cm}
				\begin{itemize} \itemsep -1pt
				\item For ``individuals from Brazil, Egypt, Ghana, Indonesia, Pakistan, South Africa, Turkey, and selected countries in Central America to spend one year studying at community colleges in the United States and earn a vocational certificate.''
				\item ``The program provides academic instruction in selected fields including agriculture, applied engineering, business management and administration, health professions, information technology, media, and tourism and hospitality management, while also immersing participants in U.S. society and cultural life.''
				\item ``Participants are recruited from historically underserved populations and may not have had opportunities for formal job training or higher education. Most participants are in their early- to mid-twenties and many already have work experience.''
				\item \url{http://exchanges.state.gov/globalexchanges/community-colleges-initiative.html}
				\end{itemize}
			\item {\bf \color{blue} Benjamin A. Gilman International Scholarship Program}: \vspace{-0.1cm}
				\begin{itemize} \itemsep -1pt
				\item ``The Benjamin A. Gilman International Scholarship Program provides scholarships to U.S. undergraduates with financial need for study abroad, including students from diverse backgrounds and students going to non-traditional study abroad destinations.''
				\item ``The applicant must be receiving a Federal Pell Grant or provide proof that he/she will be receiving a Pell Grant at the time of application or during the term of his/her study abroad.''
				\item \url{http://exchanges.state.gov/globalexchanges/gilman-scholarship-program.html}
				\end{itemize}
			\item Global Undergraduate Exchange Program (Global UGRAD Program): \vspace{-0.1cm}
				\begin{itemize} \itemsep -1pt
				\item \url{http://exchanges.state.gov/academicexchanges/guep.html}
				\item The Global Undergraduate Exchange Program (also known as the Global UGRAD Program) provides one semester and academic year scholarships to outstanding undergraduate students from underrepresented sectors in East Asia, Eurasia and Central Asia, the Near East and South Asia and the Western Hemisphere for non-degree full-time study combined with community service, internships and cultural enrichment.
				\end{itemize}
			\item Professors and Research Scholars: \url{http://exchanges.state.gov/jexchanges/programs/professor.html}
			\item Short-Term Scholar: \url{http://exchanges.state.gov/jexchanges/programs/shortterm.html}
			\item Student, College/University: \vspace{-0.1cm}
				\begin{itemize} \itemsep -1pt
				\item \url{http://exchanges.state.gov/jexchanges/programs/ucstudent.html}
				\item The College/University Student Program gives foreign students the opportunity to study at an American degree-granting post-secondary accredited educational institution, including colleges and universities. Students may participate in degree and non-degree programs. They must pursue a full-time course of study and maintain satisfactory advancement toward the completion of their academic program.
				\end{itemize}
			\item Study of the United States Institutes for Scholars: \vspace{-0.1cm}
				\begin{itemize} \itemsep -1pt
				\item Study of the United States Institutes for Scholars  are designed to strengthen curricula and improve the quality of teaching about the United States in academic institutions overseas.
				\item Foreign university faculty, secondary educators and other scholars spend approximately four weeks at host universities where they take part in a series of lectures, seminar discussions and site visits related to each institute's theme.
				\item They learn about American educational philosophies, explore new teaching methods and pursue related research interests.
				\item Interests of these institutes: \vspace{-0.1cm}
					\begin{itemize} \itemsep -1pt
					\item American Politics and Political Thought
					\item Contemporary American Literature
					\item Journalism and Media
					\item Religious Pluralism in the United States
					\item Secondary School Educators
					\item U.S. Culture and Society
					\item U.S. Foreign Policy
					\item U.S. National Security
					\end{itemize}
				\item \url{http://exchanges.state.gov/academicexchanges/scholars.html}
				\end{itemize}
			\item Study of the United States Institutes for Student Leaders: \vspace{-0.1cm}
				\begin{itemize} \itemsep -1pt
				\item Study of the United States Institutes for Student Leaders are five-to-six-week academic programs for foreign undergraduate leaders.
				\item Hosted by U.S. academic institutions throughout the United States, the Student Leader Institutes include an intensive academic component, an educational tour of other regions of the country, local community service activities and a unique opportunity for participants to get to know their American peers.
				\item \url{http://exchanges.state.gov/academicexchanges/students.html}
				\item Interests of the institutes: \vspace{-0.1cm}
					\begin{itemize} \itemsep -1pt
					\item Comparative {\bf Public Policy} for Pakistani Student Leaders
					\item Energy and the Environment
					\item Global Environmental Issues
					\item New Media
					\item Religious Pluralism in the U.S.
					\item Social Entrepreneurship
					\item U.S. Foreign Policy for East Asian Student Leaders
					\item Western Hemisphere Student Leaders 
					\item Women's Leadership
					\end{itemize}
				\end{itemize}
			\item Edmund S. Muskie Graduate Fellowship: \vspace{-0.1cm}
				\begin{itemize} \itemsep -1pt
				\item \url{http://exchanges.state.gov/academicexchanges/muskie.html}
				\item The Edmund S. Muskie Graduate Fellowship Program (Muskie) confers fellowships for Master's degree-level study in the U.S. in the fields of business administration, economics, education, environmental policy and management, international affairs, journalism/mass communications, law, library and information science, public administration, public health and {\bf public policy} for students and professionals from Eurasia.
				\item Candidates are recruited through a merit-based competition administered by the International Research \& Exchanges Board (IREX).
				\item U.S. host campuses are also selected through a competition process and generally provide tuition waivers of fifty percent.
				\item Approximately 145 fellowships are awarded each academic year.
				\end{itemize}
			\item Critical Language Scholarship Program: \vspace{-0.1cm}
				\begin{itemize} \itemsep -1pt
				\item \url{http://exchanges.state.gov/academicexchanges/sli2.html}
				\item The Critical Language Scholarship (CLS) Program provides overseas foreign language instruction and cultural enrichment experiences in 13 critical need languages for U.S. students in higher education.
				\item The CLS Program is part of a U.S. government effort to expand dramatically the number of Americans studying and mastering critical need foreign languages.
				\item Undergraduate, master's and doctoral-level students of diverse disciplines and majors are encouraged to apply for the seven-to-10-week-long programs.
				\item Participants are expected to continue their language study beyond the scholarship period, and later apply their critical language skills in their future professional careers.
				\end{itemize}
			\item Critical Language Enhancement Award (CLEA): \vspace{-0.1cm}
				\begin{itemize} \itemsep -1pt
				\item \url{http://exchanges.state.gov/academicexchanges/clea2.html}
				\item The Critical Language Enhancement Award (CLEA) provides funding to eligible Fulbright U.S. Student Program Grantees who intend to use one of the following languages for their Fulbright project: \vspace{-0.1cm}
					\begin{itemize} \itemsep -1pt
					\item Arabic (all dialiects)
					\item Azeri
					\item Bangla/Bengali
					\item Bhasa Indonesia
					\item Chinese (Mandarin Only)
					\item Farsi
					\item Gujarati
					\item Hindi
					\item Korean
					\item Marathi
					\item Pashto
					\item Punjabi
					\item Russian
					\item Turkish
					\item Urdu
					\end{itemize}
				\end{itemize}
			\end{enumerate}
		\item Office of International Visitors: \vspace{-0.1cm}
			\begin{enumerate} \itemsep -1pt
			\item International Visitor Leadership Program (IVLP): \vspace{-0.1cm}
				\begin{itemize} \itemsep -1pt
				\item \url{http://exchanges.state.gov/ivlp/index.html}
				\item \url{http://exchanges.state.gov/ivlp/ivlp.html}
				\item The Office of International Visitors manages and funds the International Visitor Leadership Program (IVLP).
				\item Launched in 1940, the IVLP is a professional exchange program that seeks to build mutual understanding between the U.S. and other nations through carefully designed short-term visits to the U.S. for current and emerging foreign leaders.
				\item These visits reflect the International Visitors' professional interests and support the foreign policy goals of the United States.
				\end{itemize}
			\end{enumerate}
		\item Program Search (find international exchange programs sponsored by the Bureau of Educational and Cultural Affairs): \url{http://exchanges.state.gov/index/search.html}
		\end{enumerate}
	\end{enumerate}
\item Mexican American Legal Defense and Educational Fund (MALDEF): \vspace{-0.3cm}
	\begin{enumerate} \itemsep -2pt
	\item Scholarship Resources: \url{http://maldef.org/leadership/scholarships/}
	\item MALDEF Law School Scholarship Program: \vspace{-0.2cm}
		\begin{enumerate} \itemsep -2pt
		\item MALDEF's Law School Scholarship Program provides several scholarships in varying amounts to deserving law students with a commitment to advancing the civil rights of Latinos.
		\item MALDEF's Law School Scholarship Program is open to all law students who will be enrolled full-time in an American-accredited law school in 2010-2011.
		\item Scholarships are awarded to students based on their commitment to serve the Latino community through law; their past achievement and potential for achievement; and their financial need.
		\item \url{http://maldef.org/leadership/scholarships/law_school_scholarship_program/index.html}
		\end{enumerate}
	\item Undergraduate Scholarship Resource Guide: \url{http://maldef.org/leadership/scholarships/resources/index.html}
	\end{enumerate}
\item Ashoka: \vspace{-0.3cm}
	\begin{enumerate} \itemsep -2pt
	\item Ashoka Fellows (to promote and support social entrepreneurship): \url{http://www.ashoka.org/fellows}
	\end{enumerate}
\item Heinz Family Foundation: \vspace{-0.3cm}
	\begin{enumerate} \itemsep -2pt
	\item Heinz Award Criteria: \vspace{-0.2cm}
		\begin{enumerate} \itemsep -2pt
		\item \url{http://heinzawards.net/awards/criteria}
		\item The Heinz Endowments
		\item Attributes and qualities of awardees: \vspace{-0.1cm}
			\begin{enumerate} \itemsep -1pt
			\item an enormous capacity to love
			\item smile
			\item take risks
			\item question
			\item work hard
			\item believe in the power of the individual to improve the lives of others
			\end{enumerate}
		\item ``Candidates [should] possess a remarkable mix of vision, optimism, creativity and hard work which, when combined, produce tangible achievements of lasting good.''
		\item Nominees must exhibit the following personal characteristics: \vspace{-0.1cm}
			\begin{enumerate} \itemsep -1pt
			\item A passion for excellence that goes beyond intellectual curiosity;
			\item A concern for humanity rooted in a deep sensitivity for the well-being of others; 
			\item A knowledge of self which acknowledges weaknesses but relies on individual strengths;
			\item A gritty determination that will see a job through to completion despite the inevitable setbacks;
			\item A broad vision which extends far beyond the particular and embraces something universal.
			\end{enumerate}
		\item Work of the candidates for a Heinz Award must meet the following criteria: \vspace{-0.1cm}
			\begin{enumerate} \itemsep -1pt
			\item Be significant and not a ``quick fix.''
			\item Have an enduring and meaningful impact.
			\item Be creative and innovative, and
			\item Be sufficiently tangible to serve as a model for replication elsewhere.
			\end{enumerate}
		\item ``In addition, candidates should be actively working in the field in which they are nominated with the hope that, in receiving this award, their potential for future societal contribution will be enhanced.''
		\end{enumerate}
	\item Categories: \vspace{-0.2cm}
		\begin{enumerate} \itemsep -2pt
		\item Arts \& Humanities
		\item Environment
		\item Human Condition
		\item {\bf Public Policy}
		\item Technology, Economy, + Employment
		\end{enumerate}
	\end{enumerate}
\item Echoing Green: \vspace{-0.3cm}
	\begin{enumerate} \itemsep -2pt
	\item Echoing Green Fellowship: \vspace{-0.2cm}
		\begin{enumerate} \itemsep -2pt
		\item \url{http://www.echoinggreen.org/fellowship}
		\item Has information on eligibility, the benefits of the fellowship, and application cycle and dates.
		\end{enumerate}
	\item Echoing Green Fellows: \url{http://www.echoinggreen.org/fellows}
	\end{enumerate}
\item Ben Franklin Technology Partners (BFTP): \vspace{-0.3cm}
	\begin{enumerate} \itemsep -2pt
	\item Innovation Works (IW): \vspace{-0.2cm}
		\begin{enumerate} \itemsep -2pt
		\item AlphaLab: \vspace{-0.1cm}
			\begin{enumerate} \itemsep -1pt
			\item ``An immersive environment where entrepreneurs can tap IW's onsite experts for business and market advice and exchange ideas with other entrepreneurs launching in similar markets''
			\end{enumerate}
		\end{enumerate}
	\end{enumerate}
\item Carnegie Corporation of New York: \vspace{-0.3cm}
	\begin{enumerate} \itemsep -2pt
	\item Carnegie Scholars Program (not available in 2010): \url{http://carnegie.org/programs/carnegie-scholars/}
	\end{enumerate}
\item New York Women's Foundation: \vspace{-0.3cm}
	\begin{enumerate} \itemsep -2pt
	\item Finch Scholar Program (with the Finch College Alumnae Association): \vspace{-0.2cm}
		\begin{enumerate} \itemsep -2pt
		\item \url{http://www.nywf.org/internship.html} and \url{http://www.finchcollege.org/}
		\item ``Our partnership with the Finch Scholar Program allows us to provide practical community service experience to an outstanding local student enrolled in college. The internship affords the Finch Scholar opportunities to work in meaningful ways in a nonprofit organization with exposure to social change philanthropy, participatory grantmaking, advocacy and {\bf public policy}. Generally, we offer one scholarship per year with a stipend.''
		\item \url{http://www.finchcollege.org/newFinchScholarPrgm.html}
		\item \url{http://www.finchcollege.org/newscholarships.html}
		\end{enumerate}
	\end{enumerate}
\item The Rockefeller Foundation: \vspace{-0.3cm}
	\begin{enumerate} \itemsep -2pt
	\item The Bellagio Center: \vspace{-0.2cm}
		\begin{enumerate} \itemsep -2pt
		\item \url{http://www.rockefellerfoundation.org/bellagio-center}
		\item Residency Programs: \vspace{-0.1cm}
			\begin{enumerate} \itemsep -1pt
			\item \url{http://www.rockefellerfoundation.org/bellagio-center/residency-programs}
			\item ``The Bellagio Residency program offers scholars, artists, thought leaders, policymakers and practitioners a serene setting conducive to focused, goal-oriented work, and the unparalleled opportunity to establish new connections with fellow residents, across a stimulating array of disciplines and geographies.  The Bellagio Center community generates new knowledge to solve some of the most complex problems facing our world and creates art that inspires reflection, understanding, and imagination.''
			\item Scholarly Residencies: \vspace{-0.1cm}
				\begin{itemize} \itemsep -1pt
				\item ``Researchers in the humanities, natural sciences, social sciences and other academic disciplines''
				\item ``The Center typically offers one-month residencies for no more than 12 scholars and scientists at a time. Individuals in any discipline and from any part of the world are welcome to apply. The Center maintains a core focus on projects consistent with the Foundation's mission to expand opportunities for poor or vulnerable people and to help see that the benefits of globalization are shared more widely. It also seeks to include beyond that core a wide variety of projects from all academic disciplines.''
				\item \url{http://www.rockefellerfoundation.org/bellagio-center/residency-programs/scholarly-residencies}
				\end{itemize}
			\item Creative Artist Residencies: \vspace{-0.1cm}
				\begin{itemize} \itemsep -1pt
				\item ``Artists, composers, writers''
				\item ``Bellagio creative artist residencies for composers, novelists, playwrights, poets, video/filmmakers and visual artists provide time for disciplined work, individual reflection, and collegial engagement, uninterrupted by the usual professional and personal demands. The Center typically offers one-month stays for no more than three to five creative artists at a time. Artists of significant achievement from any country are welcome to apply.''
				\item \url{http://www.rockefellerfoundation.org/bellagio-center/residency-programs/creative-artist-residencies}
				\end{itemize}
			\item Practitioner Residencies: \vspace{-0.1cm}
				\begin{itemize} \itemsep -1pt
				\item ``Policymakers, nonprofit leaders, journalists and public advocates''
				\item ``The Center offers residencies to professionals in fields relevant to the Rockefeller Foundation's issue areas. The Center maintains a core focus on projects consistent with our mission, to expand opportunities for poor or vulnerable people and to help see that the benefits of globalization are shared more widely.   We seek practitioner applicants with demonstrated leadership qualities and the capacity to contribute to the intellectual life at the Center.''
				\item \url{http://www.rockefellerfoundation.org/bellagio-center/residency-programs/practitioner-residencies}
				\end{itemize}
			\end{enumerate}
		\item {\bf \color{blue} Creative Arts Fellowships}: \vspace{-0.1cm}
			\begin{enumerate} \itemsep -1pt
			\item ``This high-profile program hosts visual artists at the Bellagio Center for three-month residencies that inspire creativity and promote interaction between the arts and other fields. Creative Arts Fellows, like other participants in Bellagio residency programs, have the time and space to work independently during the day. They also enjoy and benefit from a lively community of scholars, writers, policymakers and other artists who gather in the evening for dinner and occasional presentations.  The combination of private work space, an extended stay, a generous stipend and a unique group of fellow residents makes a Creative Arts Fellowship at the Bellagio Center a remarkable opportunity.''
			\item \url{http://www.rockefellerfoundation.org/bellagio-center/creative-arts-fellowships}
			\end{enumerate}
		\end{enumerate}
	\end{enumerate}
\item Wellcome Trust: \vspace{-0.3cm}
	\begin{enumerate} \itemsep -2pt
	\item Wellcome Trust Book Prize: \vspace{-0.2cm}
		\begin{enumerate} \itemsep -2pt
		\item \url{http://www.wellcomebookprize.org/About-the-prize/index.htm}
		\item ``The Wellcome Trust Book Prize celebrates the best of medicine in literature by awarding 25 000 each year for the finest fiction or non-fiction book centered around medicine.''
		\end{enumerate}
	\end{enumerate}
\item The Kennedy Memorial Trust: \vspace{-0.3cm}
	\begin{enumerate} \itemsep -2pt
	\item \url{http://www.kennedytrust.org.uk/}
	\item Kennedy Scholarship: \url{http://www.kennedytrust.org.uk/display.aspx?Id=1165&pid=0}
	\item Frank Knox Fellowships: \url{http://www.kennedytrust.org.uk/display.aspx?Id=1175&pid=0}
	\end{enumerate}
\item Foreign \& Commonwealth Office / United Kingdom: \vspace{-0.3cm}
	\begin{enumerate} \itemsep -2pt
	\item Chevening scholarships: \vspace{-0.2cm}
		\begin{enumerate} \itemsep -2pt
		\item \url{http://www.fco.gov.uk/en/about-us/what-we-do/scholarships/}
		\item ``The Chevening programme, has, over 26 years, provided more than 30,000 Scholarships at Higher Education Institutions (HEIs) in the UK for postgraduate students or researchers from countries across the world.''
		\end{enumerate}
	\item {\bf Marshall Scholarships} finance young Americans of high ability to study for a graduate degree in the United Kingdom: \url{http://www.marshallscholarship.org/}
	\end{enumerate}
\item Ministry of Education, Culture, Sports, Science and Technology (MEXT) / Japan: \vspace{-0.3cm}
	\begin{enumerate} \itemsep -2pt
	\item \url{http://www.mext.go.jp/english/}
	\item Monbukagakusho Scholarship: \vspace{-0.2cm}
		\begin{enumerate} \itemsep -2pt
		\item \url{http://en.wikipedia.org/wiki/Monbukagakusho_Scholarship}
		\item \url{http://project.monbusho.org/old/} and \url{http://www.monbusho.org/}
		\end{enumerate}
	\end{enumerate}
\item Institute of International Education (IIE): \vspace{-0.3cm}
	\begin{enumerate} \itemsep -2pt
	\item GE Foundation Scholar-Leaders Program: \vspace{-0.2cm}
		\begin{enumerate} \itemsep -2pt
		\item \url{http://www.iie.org/en/Programs/GE-Foundation-Scholar-Leaders-Program}
		\item ``The GE Foundation Scholar-Leaders Program began in 1987 in Mexico and now supports outstanding students in higher education in fourteen countries around the world. The program initially provided traditional financial support for university education, but has developed into an exciting Leadership Development Program to complement the student's academic curriculum.''
		\item Eligibility: ``Students in their first year of study in engineering, technology, business, finance, management, or economics attending a participating university. GE Foundation Scholar-Leaders qualification requirements vary by region.''
		\end{enumerate}
	\end{enumerate}
\item British Council: \vspace{-0.3cm}
	\begin{enumerate} \itemsep -2pt
	\item Shine! 2011: International Student Awards: \vspace{-0.2cm}
		\begin{enumerate} \itemsep -2pt
		\item \url{http://www.educationuk.org/shine}
		\item For international students in the United Kingdom
		\end{enumerate}
	\item Funding your studies: \vspace{-0.2cm}
		\begin{enumerate} \itemsep -2pt
		\item \url{http://www.britishcouncil.org/learning-funding-your-studies.htm}
		\item Education UK: \url{http://www.educationuk.org/pls/hot_bc/page_pls_user_advice?x=&y=&a=0&d=4460}
		\item 9/11 Scholarship Fund: \vspace{-0.1cm}
			\begin{enumerate} \itemsep -1pt
			\item \url{http://www.britishcouncil.org/911scholarships.htm}
			\item ``The 9/11 Scholarship Fund supports international students who were directly affected by the 2001 terrorist events in the US. Find out more how each scholarship offers the opportunity to study at a UK college or university every year.''
			\end{enumerate}
		\end{enumerate}
	\item {\it Youth in Action} European program: \url{http://www.britishcouncil.org/youthinaction}
	\item British Council Arts Group: \vspace{-0.2cm}
		\begin{enumerate} \itemsep -2pt
		\item Support and funding overview: \url{http://www.britishcouncil.org/arts-support-and-funding-overview.htm}
		\item Visual arts support and funding: \url{http://www.britishcouncil.org/arts-visual-arts-funding.htm}
		\item Drama and dance support and funding: \url{http://www.britishcouncil.org/arts-performing-arts-funding.htm}
		\item Literature support and funding: \url{http://www.britishcouncil.org/arts-literature-support-and-funding.htm}
		\item Film support and funding: \url{http://www.britishcouncil.org/arts-film-funding.htm}
		\item Music support and funding: \url{http://www.britishcouncil.org/arts-music-funding.htm}
		\item Architecture, design, fashion support and funding: \url{http://www.britishcouncil.org/arts-adf-funding.htm}
		\item International Short Film Festival Support Scheme: \url{http://www.britishcouncil.org/arts-film-short-films-scheme.htm}
		\end{enumerate}
	\end{enumerate}
\item Alfred P. Sloan Foundation: \vspace{-0.3cm}
	\begin{enumerate} \itemsep -2pt
	\item Sloan Research Fellowships: \vspace{-0.2cm}
		\begin{enumerate} \itemsep -2pt
		\item \url{http://www.sloan.org/fellowships}
		\item Hold a Ph.D. (or equivalent) in chemistry, physics, mathematics, computer science, economics, neuroscience or computational and evolutionary molecular biology, or in a related interdisciplinary field;
		\item Be members of the regular faculty (i.e., tenure track) of a degree-granting college or university in the United States or Canada; and
		\item Normally, be no more than six years from completion of the most recent Ph.D. or equivalent as of the year of their nomination.
		\end{enumerate}
	\end{enumerate}
\item --- --- --- --- --- --- --- --- --- --- --- --- --- --- --- --- --- --- --- --- --- --- --- --- --- --- --- --- --- --- ---
\item \colorbox{blue}{\bf Scholarships and Fellowships in Business (including Finance, Entrepreneurship, and Accounting)}
% Scholarships and Fellowships in Business (including Finance, Entrepreneurship, and Accounting)
\item IREX: \vspace{-0.3cm}
	\begin{enumerate} \itemsep -2pt
	\item Opportunities ``for individuals, organizations, universities, and alumni'': \url{http://www.irex.org/apply}
	\item Edmund S. Muskie Graduate Fellowship Program: \vspace{-0.2cm}
		\begin{enumerate} \itemsep -2pt
		\item : \url{http://www.irex.org/application/edmund-s-muskie-graduate-fellowship-program-application}
		\item ``The Muskie Program is open to graduate students and professionals from Armenia, Azerbaijan, Belarus, Georgia, Kazakhstan, Kyrgyzstan, Moldova, Russia, Tajikistan, Turkmenistan, Ukraine and Uzbekistan for one-year non-degree, one-year degree, or two-year degree study in the United States.''
		\item ``Eligible fields of study for the Muskie Program are: business administration, economics, education, environmental management, international affairs, journalism and mass communication, law, library and information science, public administration, public health, and {\bf public policy}.''
		\end{enumerate}
	\end{enumerate}
\item Sponsors for Educational Opportunity (SEO): \vspace{-0.3cm}
	\begin{enumerate} \itemsep -2pt
	\item Alternative Investment Fellowship Program: \vspace{-0.2cm}
		\begin{enumerate} \itemsep -2pt
		\item \url{http://www.seo-usa.org/Fellowship}
		\item Eligibility: \vspace{-0.1cm}
			\begin{enumerate} \itemsep -1pt
			\item \url{http://www.seo-usa.org/FellowshipEligibility}
			\item The program is open to professionals traditionally underrepresented in alternative investments who are in the first year (or second year with a third-year offer) of an analyst program at an investment bank.
			\item Corporate finance, M\&A, leveraged finance and structured finance analysts are preferred.
			\item Management consultants will also be considered.
			\end{enumerate}
		\end{enumerate}
	\item The SEO Scholars Program: \vspace{-0.2cm}
		\begin{enumerate} \itemsep -2pt
		\item \url{http://www.seo-usa.org/Scholars}
		\item The SEO Scholars Program is a rigorous out-of-school academic enrichment program that prepares motivated New York City public high school students of color to gain admission to and succeed at competitive colleges and universities throughout the country.  Numerous studies confirm that rigorous academics are the single most important factor for low-income and minority students in gaining college admission and earning a degree.  However, U.S. Department of Education research shows that ``A'' work in low-income schools equals ``C'' work in affluent schools.
		\item Admissions: \url{http://www.seo-usa.org/ScholarsAdmissions}
		\item Roadmap To Success: \url{http://www.seo-usa.org/ScholarsRoadmapToSuccess}
		\item Enrichment Programs: \url{http://www.seo-usa.org/ScholarsEnrichmentPrograms}
		\item Volunteering: \url{http://www.seo-usa.org/ScholarsVolunteering}
		\item Andrew Golkin Fund: \vspace{-0.1cm}
			\begin{enumerate} \itemsep -1pt
			\item \url{http://www.seo-usa.org/ScholarsAndrewGolkinFund}
			\item \url{http://www.seo-usa.org/andrewgolkinfund/index.html}
			\end{enumerate}
		\item Franklin H. and Shirley B. Williams Scholarship Fund: \url{http://www.seo-usa.org/ScholarsFHSBW}
		\item The Advantages of Attending a Competitive College: \url{http://www.seo-usa.org/ScholarsAdvantages}
		\end{enumerate}
	\item Career program: \vspace{-0.2cm}
		\begin{enumerate} \itemsep -2pt
		\item \url{http://www.seo-usa.org/Career}
		\item The SEO Career Program places students of color interested in finance, philanthropy, business and corporate law in internships with competitive pay, rigorous training, support through mentors, and broad access to industry professionals. 
		\item Sponsors for Educational Opportunity (SEO) is the nation's premiere summer internship program for talented underrepresented students of color that can lead to full-time job offers.
		\item SEO offers internship opportunities in the following areas: \vspace{-0.1cm}
			\begin{enumerate} \itemsep -1pt
			\item Corporate Financial Leadership: \url{http://www.seo-usa.org/Career/Corporate_Financial_Leadership}
			\item Banking/Asset Management Areas: \vspace{-0.1cm}
				\begin{itemize} \itemsep -1pt
				\item Investment Banking: \url{http://www.seo-usa.org/Career/Investment_Banking}
				\item Sales \& Trading: \url{http://www.seo-usa.org/Career/Sales_&_Trading}
				\item Investment Research: \url{http://www.seo-usa.org/Career/Investment_Research}
				\item Transaction Services: \url{http://www.seo-usa.org/Career/Transaction_Services}
				\item Asset Management: \url{http://www.seo-usa.org/Career/Asset_Management}
				\item Accounting/Finance: \url{http://www.seo-usa.org/Career/Accounting/Finance}
				\item Information Technology: \url{http://www.seo-usa.org/Career/Information_Technology}
				\end{itemize}
			\item Corporate Law: \url{http://www.seo-usa.org/Career/Corporate_Law}
			\item Nonprofit: \url{http://www.seo-usa.org/Career/Nonprofit}
			\item SEO-U: Freshmen and Sophomore Training: \url{http://www.seo-usa.org/Career/SEO-U:Freshmen_&_Sophomore_Training}
			\end{enumerate}
		\item Application Deadlines: \url{http://www.seo-usa.org/CareerApplicationDeadlines}
		\item Eligibility Information: \url{http://www.seo-usa.org/CareerEligibilityInfo}
		\item Application Tips: \url{http://www.seo-usa.org/CareerApplicationTips}
		\item Interview Tips: \url{http://www.seo-usa.org/CareerInterviewTips}
		\end{enumerate}
	\end{enumerate}
\item --- --- --- --- --- --- --- --- --- --- --- --- --- --- --- --- --- --- --- --- --- --- --- --- --- --- --- --- --- --- ---
\item \colorbox{blue}{\bf Scholarships for Studying Abroad}
% Scholarships for Studying Abroad
\item U.S. Department of State: \vspace{-0.3cm}
	\begin{enumerate} \itemsep -2pt
	\item Bureau of Educational and Cultural Affairs: \vspace{-0.2cm}
		\begin{enumerate} \itemsep -2pt
		\item Benjamin A. Gilman International Scholarship: \vspace{-0.1cm}
			\begin{enumerate} \itemsep -1pt
			\item \url{http://exchanges.state.gov/globalexchanges/gilman-scholarship-program.html}
			\item ``The Benjamin A. Gilman International Scholarship Program provides scholarships to U.S. undergraduates with financial need for study abroad, including students from diverse backgrounds and students going to non-traditional study abroad destinations.  Established under the International Academic Opportunity Act of 2000, Gilman Scholarships provide up to \$5,000 for American students to pursue overseas study for college credit.''
			\item Critical Need Languages: Students studying critical need languages are eligible for up to \$3,000 in additional funding as part of the Gilman Critical Need Language Supplement program. Those critical need languages include: \vspace{-0.1cm}
				\begin{itemize} \itemsep -1pt
				\item Arabic
				\item Chinese
				\item Korean
				\item Russian
				\item Turkic (Azerbaijani, Kazakh, Kyrgyz, Turkish, Turkmen, Uzbek)
				\item Persian (Farsi, Dari, Kurdish, Pashto, Tajiki)
				\item Indic (Hindi, Urdu, Nepali, Sinhala, Bengali, Punjabi, Marathi, Gujurati, Sindhi)
				\end{itemize}
			\item \url{http://www.iie.org/en/Programs/Gilman-Scholarship-Program}
			\item \url{http://www.iie.org/en/Programs/Gilman-Scholarship-Program/About-the-Program}
			\end{enumerate}
		\end{enumerate}
	\end{enumerate}
\item Council on International Educational Exchange (CIEE): \vspace{-0.3cm}
	\begin{enumerate} \itemsep -2pt
	\item CIEE Scholarships: \url{http://www.ciee.org/study/scholarships/index.aspx}
	\end{enumerate}
\item IES Abroad (formerly Institute of European Studies / Institute for the International Education of Students): \vspace{-0.3cm}
	\begin{enumerate} \itemsep -2pt
	\item Scholarships and Financial Aid: \url{https://www.iesabroad.org/IES/Scholarships_and_Aid/financialAid.html}
	\item IES Abroad Need-Based Financial Aid: \url{https://www.iesabroad.org/IES/Scholarships_and_Aid/Need-Based/needBasedFinancialAid.html}
	\item IES Abroad Merit-Based Scholarships: \url{https://www.iesabroad.org/IES/Scholarships_and_Aid/Merit_Based/meritBasedFinancialAid.html}
	\item IES Abroad Public University Grants: \url{https://www.iesabroad.org/IES/Scholarships_and_Aid/publicScholarship.html}
	\end{enumerate}
\item American Institute For Foreign Study (AIFS): \vspace{-0.3cm}
	\begin{enumerate} \itemsep -2pt
	\item AIFS Study Abroad Programs: \vspace{-0.2cm}
		\begin{enumerate} \itemsep -2pt
		\item \url{http://www.aifsabroad.com/programs.asp}
		\item AIFS Study Abroad Scholarships: \url{http://www.aifsabroad.com/scholarships.asp}
		\end{enumerate}
	\end{enumerate}
\item --- --- --- --- --- --- --- --- --- --- --- --- --- --- --- --- --- --- --- --- --- --- --- --- --- --- --- --- --- --- ---
\item \colorbox{blue}{\bf Scholarships and Fellowships in Public Policy and Public Health}
% Scholarships and Fellowships in Public Policy and Public Health
\item The Commonwealth Fund: \vspace{-0.3cm}
	\begin{enumerate} \itemsep -2pt
	\item Commonwealth Fund fellowship programs: \vspace{-0.2cm}
		\begin{enumerate} \itemsep -2pt
		\item \url{http://www.commonwealthfund.org/Fellowships.aspx}
		\item ``Commonwealth Fund fellowship programs are designed to give promising young researchers the opportunity for in-depth study of various health care policy topics, working with investigators, policy analysts, government officials, and others in a number of U.S. and international settings.''
		\item The Commonwealth Fund/Harvard University Fellowship in Minority Health Policy: \url{http://www.commonwealthfund.org/Fellowships/Minority-Health-Policy-Fellowship.aspx}
		\item Harkness Fellowships in Health Care Policy and Practice: \url{http://www.commonwealthfund.org/Fellowships/Harkness-Fellowships.aspx}
		\item Australian-American Health Policy Fellowship: \url{http://www.commonwealthfund.org/Fellowships/Australian-American-Health-Policy-Fellowships.aspx}
		\item Ian Axford (New Zealand) Fellowships in Public Policy: \url{http://www.commonwealthfund.org/Fellowships/Ian-Axford-Fellowships.aspx}
		\end{enumerate}
	\end{enumerate}
\item American Institute of Aeronautics and Astronautics (AIAA): \vspace{-0.3cm}
	\begin{enumerate} \itemsep -2pt
	\item Federal Government Fellows Program: \vspace{-0.2cm}
		\begin{enumerate} \itemsep -2pt
		\item \url{http://www.aiaa.org/content.cfm?pageid=731}
		\item Shaping U.S. {\bf public policy} concerning aerospace research and the aerospace industry
		\end{enumerate}
	\end{enumerate}
\item IEEE-USA: \vspace{-0.3cm}
	\begin{enumerate} \itemsep -2pt
	\item Congressional Fellowship
	\item Engineering \& Diplomacy (State Department) Fellowship
	\item For IEEE-USA members to support the creation and modification of technology-related public policies
	\item \url{http://ieeeusa.org/policy/govfel/default.asp}
	\end{enumerate}
\item American Mathematical Society: \vspace{-0.3cm}
	\begin{enumerate} \itemsep -2pt
	\item Fellowships and Awards (Policy and Advocacy: Government Relations \& Programs): \vspace{-0.2cm}
		\begin{enumerate} \itemsep -2pt
		\item \url{http://e-math.ams.org/policy/government/fellow-awards/fellow-awards}
		\item Mass Media Fellowships: \url{http://e-math.ams.org/programs/ams-fellowships/media-fellow/massmediafellow}
		\item AMS-AAAS Congressional Fellowship: \url{http://e-math.ams.org/programs/ams-fellowships/ams-aaas/ams-aaas-congressional-fellowship}
		\end{enumerate}
	\end{enumerate}
\item American Association for the Advancement of Science: \vspace{-0.3cm}
	\begin{enumerate} \itemsep -2pt
	\item AAAS Science \& Technology Policy Fellowships: \url{http://fellowships.aaas.org/index.shtml}
	\end{enumerate}
\item --- --- --- --- --- --- --- --- --- --- --- --- --- --- --- --- --- --- --- --- --- --- --- --- --- --- --- --- --- --- ---
\item \colorbox{blue}{\bf Scholarships and Fellowships in Social Science and Humanities}
% Scholarships and Fellowships in Social Science and Humanities
\item United States Institute of Peace (USIP): \vspace{-0.3cm}
	\begin{enumerate} \itemsep -2pt
	\item Jennings Randolph Peace Scholarship Dissertation Program (for Ph.D. students working on topics related to peace, conflict, and international security): \url{http://www.usip.org/grants-fellowships/jennings-randolph-peace-scholarship-dissertation-program}
	\end{enumerate}
\item Library of Congress: \vspace{-0.3cm}
	\begin{enumerate} \itemsep -2pt
	\item Kluge Fellowships: \vspace{-0.2cm}
		\begin{enumerate} \itemsep -2pt
		\item Research in the humanities and social sciences, especially interdisciplinary, cross-cultural or multilingual
		\item Open to scholars worldwide with a Ph.D. or other terminal advanced degree conferred within seven years of the July 15 deadline
		\item \url{http://www.loc.gov/loc/kluge/fellowships/kluge.html}
		\end{enumerate}
	\item J. Franklin Jameson Fellowship Research in American History (junior postdocs): \url{http://www.loc.gov/loc/kluge/fellowships/jameson.html}
	\item Kislak Short Term Fellowship Opportunities in American Studies (students, postdocs, and faculty): \url{http://www.loc.gov/loc/kluge/fellowships/kislakshort.html}
	\item Kislak Fellowship in American Studies (Ph.D. requirement): \url{http://www.loc.gov/loc/kluge/fellowships/kislak.html}
	\end{enumerate}
\item American Historical Association (AHA): \vspace{-0.3cm}
	\begin{enumerate} \itemsep -2pt
	\item AHA Research Grants: \url{http://www.historians.org/prizes/Grants.htm}
	\item Fellowships: \url{http://www.historians.org/prizes/Fellowships.htm}
	\end{enumerate}
\item American Sociological Association: \vspace{-0.3cm}
	\begin{enumerate} \itemsep -2pt
	\item ASA Dissertation Award: \url{http://www.asanet.org/about/awards/dissertation.cfm}
	\end{enumerate}
\item American Psychological Association: \vspace{-0.3cm}
	\begin{enumerate} \itemsep -2pt
	\item Scholarships, Grants, and Awards: \url{http://www.apa.org/about/awards/index.aspx}
	\end{enumerate}
\item American Anthropological Association (AAA): \vspace{-0.3cm}
	\begin{enumerate} \itemsep -2pt
	\item AAA Minority Dissertation Fellowship Program (for minority Ph.D. candidates in anthropology): \url{http://www.aaanet.org/cmtes/minority/Minfellow.cfm}
	\item Margaret Mead Award (for young scholars in anthropology): \url{http://www.aaanet.org/about/Prizes-Awards/AAA-Margaret-Mead-Award.cfm}
	\item COSWA Award: \vspace{-0.2cm}
		\begin{enumerate} \itemsep -2pt
		\item The COSWA Award (formerly the Squeaky Wheel Award), sponsored by the Committee on the Status of Women in Anthropology (COSWA), recognizes individuals who have demonstrated the courage to bring to light and investigate practices in anthropology that are potentially discriminatory to women, or have acted to improve the status of women in anthropology through activities that raise awareness of women's contribution to anthropology or identify barriers to full participation by women in anthropology.
		\item \url{http://www.aaanet.org/about/Prizes-Awards/COSWA-Award.cfm}
		\end{enumerate}
	\item David M. Schneider Award (for Ph.D. students in anthropology): \url{http://www.aaanet.org/about/Prizes-Awards/David-Schneider-Award.cfm}
	\item Links to ``Section Prizes \& Awards'': \url{http://www.aaanet.org/about/Prizes-Awards/section_awards.cfm}
	\item List of national (US) and international ``Grants and Fellowships'': \url{http://www.aaanet.org/profdev/fellowships/}
	\item \url{http://www.aaanet.org/}
	\end{enumerate}
\item National Academy of Social Insurance: \vspace{-0.3cm}
	\begin{enumerate} \itemsep -2pt
	\item John Heinz Dissertation Award (Ph.D. students writing their thesis on the planning and implementation of social insurance): \url{http://www.nasi.org/studentopps/heinz}
	\end{enumerate}
\item National Endowment for the Humanities's Division of Research Programs, grants and fellowship opportunities: \url{http://www.neh.gov/grants/}
\item {\it The Henry Luce Foundation}'s Luce Scholars Program to help US graduates learn more about Asia and Asian culture(s): \url{http://www.hluce.org/lsprogram.aspx}
\item Institute for Humane Studies at George Mason University: \vspace{-0.3cm}
	\begin{enumerate} \itemsep -2pt
	\item Humane Studies Fellowships: \vspace{-0.2cm}
		\begin{enumerate} \itemsep -2pt
		\item \url{http://www.theihs.org/programs/humane-studies-fellowships}
		\item Humane Studies Fellowships are awarded to graduate students and outstanding undergraduates planning academic careers with liberty-advancing research interests.
		\item The fellowships are open to students in a range of fields, such as economics, philosophy, law, political science, anthropology, and literature.
		\end{enumerate}
	\end{enumerate}
\item The Gilder Lehrman Institute of American History: Gilder Lehrman History Scholars \& Gilder Lehrman One-Week Scholars (for sophomores or juniors majoring in American history or American Studies), \url{http://www.gilderlehrman.org/education/hs_program_details.php}
\item Myra Sadker Foundation: \vspace{-0.3cm}
	\begin{enumerate} \itemsep -2pt
	\item \url{http://www.sadker.org/awards.html}
	\item Teacher Award: Designed to promote and support teacher projects (K-12) that help students learn about and respect group differences, promote fairness, and in other ways build upon the values and contributions of Myra Sadker's work. Each project should have a gender dimension.
	\item Student Award: Designed to encourage student ideas, activities and projects (K-12) that promote respect for group differences, fairness, and in other ways build upon the values and contributions of Myra Sadker's work. Each project should have a gender dimension. 
	\item Doctoral Dissertation Award: Designed to promote and support graduate students engaged in educational equity research. Doctoral level dissertations that explore or promote educational equity and fairness based on gender, race, ethnicity, religion, class, sexual orientation, or other such variables will be considered for support. Each dissertation should have a gender dimension.
	\end{enumerate}
\item IREX: \vspace{-0.3cm}
	\begin{enumerate} \itemsep -2pt
	\item Opportunities ``for individuals, organizations, universities, and alumni'': \url{http://www.irex.org/apply}
	\item Edmund S. Muskie Graduate Fellowship Program: \vspace{-0.2cm}
		\begin{enumerate} \itemsep -2pt
		\item : \url{http://www.irex.org/application/edmund-s-muskie-graduate-fellowship-program-application}
		\item ``The Muskie Program is open to graduate students and professionals from Armenia, Azerbaijan, Belarus, Georgia, Kazakhstan, Kyrgyzstan, Moldova, Russia, Tajikistan, Turkmenistan, Ukraine and Uzbekistan for one-year non-degree, one-year degree, or two-year degree study in the United States.''
		\item ``Eligible fields of study for the Muskie Program are: business administration, economics, education, environmental management, international affairs, journalism and mass communication, law, library and information science, public administration, public health, and {\bf public policy}.''
		\end{enumerate}
	\item Legal Education and Development (LEAD) Fellowship: \vspace{-0.2cm}
		\begin{enumerate} \itemsep -2pt
		\item \url{http://www.irex.org/application/legal-education-and-development-lead-fellowship-application}
		\item Legal Education and Development Fellowship Program (LEAD) in Tajikistan
		\item Eligibility: \vspace{-0.1cm}
			\begin{enumerate} \itemsep -1pt
			\item Is a citizen, national, or permanent resident qualified to hold a valid passport issued by Tajikistan;
			\item Is the recipient of an undergraduate degree in law (four- or five-year study) by the time of the application;
			\item Is able to begin the academic exchange program in the United States in the summer of 2011;
			\item Is able to receive and maintain a United States J-1 visa.
			\end{enumerate}
		\end{enumerate}
	\item Community Solutions Program: \vspace{-0.2cm}
		\begin{enumerate} \itemsep -2pt
		\item \url{http://www.irex.org/application/community-solutions-information-applicants}
		\item ``a professional development program for the best and brightest global community leaders working in Transparency \& Accountability, Tolerance/Conflict Resolution, Environmental Issues, and Women's Issues''
		\item ``Competition for the Community Solutions Program is merit-based and open to community leaders, ages 25-38 at the time of application''
		\end{enumerate}
	\item Crimea Undergraduate Exchange Program (Crimea UGRAD) Application: \vspace{-0.2cm}
		\begin{enumerate} \itemsep -2pt
		\item \url{http://www.irex.org/application/crimea-undergraduate-exchange-program-crimea-ugrad-application}
		\item ``The Crimea UGRAD Program is open to undergraduate students from the Autonomous Republic of Crimea for one academic year of non-degree study in a US university or community college.''
		\end{enumerate}
	\end{enumerate}
\item {\it Demos}: \vspace{-0.3cm}
	\begin{enumerate} \itemsep -2pt
	\item The Ed Baker Fellowship in Democratic Values: \vspace{-0.2cm}
		\begin{enumerate} \itemsep -2pt
		\item \url{http://www.demos.org/edbakerfellowship.cfm}
		\item ``Based in our New York offices, Ed Baker Fellows will give voice to strong democratic values within a wide range of potential issues, including voting rights, citizen engagement, immigration policy and civic inclusion, campaign finance reform and money in politics, and media reform, among others.''
		\end{enumerate}
	\item Fellows Program: \vspace{-0.2cm}
		\begin{enumerate} \itemsep -2pt
		\item \url{http://www.demos.org/fellowsapp.cfm}
		\item \url{http://www.demos.org/program.cfm?currentprogramID=5A196E48-3FF4-6C82-50CBCA5825B661BA}
		\item ``The Fellows Program at Demos provides support and community for writers and thinkers with well-defined projects that aim to advance the values at the core of Demos' programs and mission: a robust and inclusive democracy; shared prosperity; strong \& effective public governance; and global interdependence.''
		\end{enumerate}
	\end{enumerate}
\item Research Councils UK (RCUK): \vspace{-0.3cm}
	\begin{enumerate} \itemsep -2pt
	\item Economic and Social Research Council (ESRC): \vspace{-0.2cm}
		\begin{enumerate} \itemsep -2pt
		\item Academic (funding opportunities for students, postdocs, and professors): \url{http://www.esrcsocietytoday.ac.uk/ESRCInfoCentre/index_academic.aspx}
		\item Professorial Fellowships (for leading senior social scientists): \url{http://www.esrcsocietytoday.ac.uk/ESRCInfoCentre/opportunities/professorial/}
		\item Funding opportunities: \vspace{-0.1cm}
			\begin{enumerate} \itemsep -1pt
			\item \url{http://www.esrcsocietytoday.ac.uk/ESRCInfoCentre/index_government.aspx}
			\item \url{http://www.esrcsocietytoday.ac.uk/ESRCInfoCentre/opportunities/}
			\item ESRC Research Funding Guide / ESRC's Funding Rules: \url{http://www.esrcsocietytoday.ac.uk/ESRCInfoCentre/opportunities/research_funding}
			\item Eligibility for Research Council Funding: \url{http://www.esrcsocietytoday.ac.uk/ESRCInfoCentre/opportunities/eligibility}
			\item Current Funding Opportunities: \url{http://www.esrcsocietytoday.ac.uk/ESRCInfoCentre/opportunities/current_funding_opportunities/}
			\item Forthcoming funding opportunities: \url{http://www.esrcsocietytoday.ac.uk/ESRCInfoCentre/opportunities/forthcoming_opportunities/}
			\item Placement Fellows Scheme: \url{http://www.esrcsocietytoday.ac.uk/ESRCInfoCentre/opportunities/placement/}
			\item Professorial Fellowships: \url{http://www.esrcsocietytoday.ac.uk/ESRCInfoCentre/opportunities/professorial/}
			\item Early Career Researchers (including Postdoctoral Fellowships, International Training, and Networking Opportunities): \url{http://www.esrcsocietytoday.ac.uk/ESRCInfoCentre/opportunities/earlycareer/}
			\item Postgraduate and Career Development Opportunities: \url{http://www.esrcsocietytoday.ac.uk/ESRCInfoCentre/opportunities/postgraduate/}
			\item International Funding Opportunities: \url{http://www.esrcsocietytoday.ac.uk/ESRCInfoCentre/opportunities/international/}
			\item Joint Funding Opportunities: \url{http://www.esrcsocietytoday.ac.uk/ESRCInfoCentre/opportunities/jointfunding/}
			\item Annual competitions: \url{http://www.esrcsocietytoday.ac.uk/ESRCInfoCentre/opportunities/annual/index.aspx#3}
			\end{enumerate}
		\end{enumerate}
	\item Arts and Humanities Research Council (AHRC): \vspace{-0.2cm}
		\begin{enumerate} \itemsep -2pt
		\item Funding Opportunities: \vspace{-0.1cm}
			\begin{enumerate} \itemsep -1pt
			\item \url{http://www.ahrc.ac.uk/FundingOpportunities/Pages/default.aspx}
			\item Fellowships: \url{http://www.ahrc.ac.uk/FundingOpportunities/Pages/Fellowships.aspx}
			\item Fellowships - route for early career researchers: \url{http://www.ahrc.ac.uk/FundingOpportunities/Pages/Fellowshipserc.aspx}
			\item Placement Fellowship based in the Department for Culture, Media and Sport (DCMS) - Climate Change: \url{http://www.ahrc.ac.uk/FundingOpportunities/Pages/PlacementFellowshipDCMS-Climatechange.aspx}
			\item Placement Fellowship based in the Department for Culture, Media and Sport (DCMS) - Health and Wellbeing: \url{http://www.ahrc.ac.uk/FundingOpportunities/Pages/PlacementFellowshipDCMShealthandwellbeing.aspx}
			\item Research Grants - route for early career researchers: \url{http://www.ahrc.ac.uk/FundingOpportunities/Pages/RG-EarlyCareers.aspx}
			\item Research Grants - Speculative Research: \url{http://www.ahrc.ac.uk/FundingOpportunities/Pages/RG-SpeculativeResearch.aspx}
			\item Research Grants - Standard Route: \url{http://www.ahrc.ac.uk/FundingOpportunities/Pages/RG-StandardRoute.aspx}
			\item Postgraduate Funding (for Masters and Ph.D. students): \url{http://www.ahrc.ac.uk/FundingOpportunities/Pages/summaryinformationforprospectivepostgraduatestudents.aspx}
			\item Browse Funding Opportunities: \url{http://www.ahrc.ac.uk/FundingOpportunities/Pages/BrowseOpportunities.aspx}
			\end{enumerate}
		\end{enumerate}
	\end{enumerate}
\item World Bank Institute (WBI): \vspace{-0.3cm}
	\begin{enumerate} \itemsep -2pt
	\item Or The World Bank Group
	\item Scholarships: \url{http://wbi.worldbank.org/wbi/scholarships} or \url{http://www.worldbank.org/wbi/scholarships/home.html}
	\end{enumerate}
\item --- --- --- --- --- --- --- --- --- --- --- --- --- --- --- --- --- --- --- --- --- --- --- --- --- --- --- --- --- --- ---
\item \colorbox{blue}{\bf Fellowships in Art and Music}
% Fellowships in Art and Music
\item The Kresge Foundation: \vspace{-0.3cm}
	\begin{enumerate} \itemsep -2pt
	\item \url{http://www.kresge.org/index.php/what/detroit_program/kresge_arts_in_detroit/}
	\item Kresge Artist Fellowships: \vspace{-0.2cm}
		\begin{enumerate} \itemsep -2pt
		\item ``Kresge Artist Fellowships seek to advance the art forms and professional careers of artists from the visual, performing and literary arts as well as elevate the profile of the artistic community and encourage creative expression in the region. Each year, Kresge will provide funding for 18 fellowships of \$25,000 each, which are awarded to artists living and working in metropolitan Detroit.''
		\item ``The fellowships recognize creative vision and commitment to excellence within a wide range of artistic disciplines, including artists who have been classically and academically trained, self taught artists and artists whose art forms have been passed down through cultural and traditional heritage.''
		\item ``Kresge Arts in Detroit is committed to supporting artists from diverse cultural backgrounds at all stages of their professional careers.''
		\item \url{http://kresge.collegeforcreativestudies.edu/}
		\item \url{http://kresge.collegeforcreativestudies.edu/kaf_guidelines.html}
		\item Information Sessions: \url{http://kresge.collegeforcreativestudies.edu/kaf_sessions.html}
		\end{enumerate}
	\item Kresge Eminent Artist Award: \vspace{-0.2cm}
		\begin{enumerate} \itemsep -2pt
		\item ``Kresge Eminent Artist Award recognizes an exceptional artist for his or her professional achievements and contributions to the cultural community, and encourages that individual's pursuit of a chosen art form as well as an ongoing commitment to metropolitan Detroit. Each year, one highly accomplished individual will be presented with the award which includes a \$50,000 prize.''
		\item \url{http://kresge.collegeforcreativestudies.edu/eminent-artist-award.html}
		\end{enumerate}
	\end{enumerate}
\item Guggenheim Fellowships from the {\it John Simon Guggenheim Memorial Foundation}: \url{http://www.gf.org/applicants}
\item The John F. Kennedy Center for the Performing Arts: \vspace{-0.3cm}
	\begin{enumerate} \itemsep -2pt
	\item DeVos Institute of Arts Management at the Kennedy Center: \vspace{-0.2cm}
		\begin{enumerate} \itemsep -2pt
		\item DeVos Institute Programs: \vspace{-0.1cm}
			\begin{enumerate} \itemsep -1pt
			\item Kennedy Center Fellowship Program: \vspace{-0.1cm}
				\begin{itemize} \itemsep -1pt
				\item \url{http://www.kennedy-center.org/education/artsmanagement/fellowships.cfm}
				\item \url{http://www.kennedy-center.org/education/artsmanagement/fellowships/home.html}
				\item ``The Kennedy Center Fellowship Program began in 2001, and provides comprehensive study to 10 arts managers at the Kennedy Center with coursework in strategic planning, marketing, and development; three practical work rotations in Center departments; and a series of professional development seminars. The paid fellowships are full-time and last nine months from September through May.''
				\end{itemize}
			\item DeVos Institute Summer International Fellowship Program at the Kennedy Center: \vspace{-0.1cm}
				\begin{itemize} \itemsep -1pt
				\item \url{http://www.kennedy-center.org/education/artsmanagement/fellowships.cfm}
				\item \url{http://www.kennedy-center.org/education/artsmanagement/international_faq.cfm}
				\item ``The Summer International Fellowship Program provides practical experience to 15 mid-to-high level arts leaders currently working in international nonprofit performing arts organizations. This full-time, four-week intensive program takes place at the Kennedy Center each July; Fellows attend each summer for three consecutive years. While at the Center, the fellows take classes and refine strategic plans for their home organizations.''
				\end{itemize}
			\item U.S. Department of State International Exchange Programs: \vspace{-0.1cm}
				\begin{itemize} \itemsep -1pt
				\item \url{http://www.kennedy-center.org/education/state/}
				\item ``The U.S. Department of State and The Kennedy Center have teamed to produce international exchange opportunities through the Performing Artists Cultural Visitors Program and International Cultural Fellows Mentoring Program.''
				\item Performing Artists Cultural Visitors Program: \url{http://www.kennedy-center.org/education/state/cultural/}
				\item International Cultural Fellows Mentoring Program: \url{http://www.kennedy-center.org/education/state/fellows/}
				\item ``Visitors, comprised of modern and hip-hop dancers, theater technicians/designers/actors, as well as classical and jazz musicians, engage with American colleagues in the creation and performance of their discipline in Washington, D.C. and in another American city.''
				\item ``The Fellows, comprised of arts managers and presenters from outside the United States, attend arts management seminars led by Kennedy Center staff, travel to another American city to study with a mentor organization, and visit New York City to meet with experts in their field.''
				\end{itemize}
			\end{enumerate}
		\end{enumerate}
	\item The National Symphony Orchestra (NSO): \vspace{-0.2cm}
		\begin{enumerate} \itemsep -2pt
		\item National Symphony Orchestra Youth Fellowship Program: \vspace{-0.1cm}
			\begin{itemize} \itemsep -1pt
			\item \url{http://www.kennedy-center.org/nso/nsoed/youthfellowship.cfm}
			\item \url{http://www.kennedy-center.org/explorer/artists/?entity_id=10811&source_type=B}
			\item ``Now in its 30th season, the National Symphony Orchestra Youth Fellowship Program is an orchestral training project for high school musicians.''
			\item ``From its inception in 1980-81 to the present, the program provides Washington metropolitan area high school students with scholarships to study privately with NSO members, as well as opportunities to observe NSO rehearsals; attend concerts; and to participate in seminars, discussions, and master classes with musicians, conductors, and NSO and Kennedy Center management.''
			\item ``There are 20 students in the NSO Youth Fellowship Program for 2009-10.''
			\item ``Participation by ethnic minorities is encouraged.''
			\item ``Priority is given to students entering 10th grade in order to provide as sustained a training as possible.''
			\end{itemize}
		\end{enumerate}
	\end{enumerate}
\item League of American Orchestras: \vspace{-0.3cm}
	\begin{enumerate} \itemsep -2pt
	\item Fellowships: \vspace{-0.2cm}
		\begin{enumerate} \itemsep -2pt
		\item \url{http://www.americanorchestras.org/learning_and_leadership/fellowships.html}
		\item Orchestra Management Fellowship Program: \vspace{-0.1cm}
			\begin{enumerate} \itemsep -1pt
			\item \url{http://www.americanorchestras.org/learning_and_leadership/omfp.html}
			\item ``This year-long, highly competitive program is designed to launch executive careers in orchestra management.''
			\item ``Along with an intense course of study, fellows undertake a series of residencies with orchestras of various sizes across the U.S. receiving invaluable work experience and the support of host orchestra staff, in particular that of the orchestra�s executive director.''
			\item ``Fellows also participate in other League leadership seminars throughout the year and receive a comprehensive overview of the classical music industry.''
			\end{enumerate}
		\item ``The League's Fellowship programs identify and prepare the future leaders of tomorrow, today.''
		\item ``Long-term curricula, developed for conductors, executive directors, and managers looking to advance, provide intensive education, hands-on learning, and valuable networking opportunities.''
		\end{enumerate}
	\end{enumerate}
\item Americans for the Arts: \vspace{-0.3cm}
	\begin{enumerate} \itemsep -2pt
	\item Event scholarships (scholarships to attend events): \url{http://www.artsusa.org/events/scholarships.asp}
	\item \url{http://www.artsusa.org/news/annual_awards/default.asp}
	\item Alene Valkanas State Arts Advocacy Award\url{http://www.artsusa.org/news/annual_awards/alene_valkanas/default.asp}
	\item Arts Education Award (awarded to institutions): \url{http://www.artsusa.org/news/annual_awards/arts_education/default.asp}
	\item Emerging Leader Award: \url{http://www.artsusa.org/news/annual_awards/emerging_leader/default.asp}
	\item Michael Newton Award for United Arts Funds Leadership (management and fundraising): \url{http://www.artsusa.org/news/annual_awards/michael_newton/default.asp}
	\item Selina Roberts Ottum Award (contributions to the field of the arts): \url{http://www.artsusa.org/news/annual_awards/selina_roberts_ottum/default.asp}
	\item United States Urban Arts Federation (USUAF): \vspace{-0.2cm}
		\begin{enumerate} \itemsep -2pt
		\item Ray Hanley Innovation Award: \url{http://www.artsusa.org/networks/usuaf/hanley.asp}
		\end{enumerate}
	\end{enumerate}
\item NEA National Heritage Fellowship (for master folk and traditional artists): \url{http://www.nea.gov/honors/heritage/index.html}
\item NEA Jazz Masters Fellowship (jazz artists): \url{http://www.arts.gov/honors/jazz/index.html}
\item Fellowships for Creative Writers [or NEA Literature Fellowships: Creative Writing]: \url{http://www.nea.gov/grants/apply/Lit/index.html} or \url{http://www.arts.gov/grants/apply/Lit/index.html}
\item Carnegie Investment Bank: Carnegie Art Award (for distinguished artists born or living in the Nordic countries), \url{http://www.carnegie.se/sv/ArtAward/About-Carnegie-Art-Award/}, \url{http://www.carnegie.se/artaward/}, and \url{http://www.carnegie.se/en/about/Operations/Carnegie-Art-Award/}
\item Robert McCann Foundation: \vspace{-0.3cm}
	\begin{enumerate} \itemsep -2pt
	\item Funding for artists and designers ``from all Scottish colleges and art schools'' to: \vspace{-0.2cm}
		\begin{enumerate} \itemsep -2pt
		\item extend their training in an area of specialization; OR
		\item finance a project ``in the craft industries associated with film and television''
		\end{enumerate}
	\item \url{http://robertmccannfoundation.com/how.html}
	\end{enumerate}
\item Alexander von Humboldt-Stiftung/Foundation: \vspace{-0.3cm}
	\begin{enumerate} \itemsep -2pt
	\item Hezekiah Wardwell Fellowship (for musicians or musicologists from Spain): \url{http://www.humboldt-foundation.de/web/wardwell-en.html}
	\end{enumerate}
\item Canada Council for the Arts: \vspace{-0.3cm}
	\begin{enumerate} \itemsep -2pt
	\item Endowments and Prizes: \vspace{-0.2cm}
		\begin{enumerate} \itemsep -2pt
		\item \url{http://www.canadacouncil.ca/prizes/}
		\item Prizes and fellowships for Canadian artists and scholars to recognize their contributions to the arts, humanities, and sciences
		\item Categories of prizes and fellowships: \vspace{-0.1cm}
			\begin{enumerate} \itemsep -1pt
			\item dance
			\item inter-arts
			\item media arts
			\item music
			\item theatre
			\item visual arts
			\item writing and publishing
			\end{enumerate}
		\end{enumerate}
	\item Grant Programs: \url{http://www.canadacouncil.ca/grants/}
	\end{enumerate}
\item Institute for Humane Studies at George Mason University: \vspace{-0.3cm}
	\begin{enumerate} \itemsep -2pt
	\item Film \& Fiction Scholarships: \vspace{-0.2cm}
		\begin{enumerate} \itemsep -2pt
		\item Students pursuing MFAs in a variety of areas are eligible: film directing, production, screenwriting, playwriting, fiction, and literary-nonfiction writing
		\item \url{http://www.theihs.org/node/448}
		\end{enumerate}
	\end{enumerate}
\item --- --- --- --- --- --- --- --- --- --- --- --- --- --- --- --- --- --- --- --- --- --- --- --- --- --- --- --- --- --- ---
\item \colorbox{blue}{\bf Scholarships and Fellowships for Underrepresented Minorities}
% Scholarships and Fellowships for Underrepresented Minorities
\item Lists of scholarships and fellowships for underrepresented minorities: \vspace{-0.3cm}
	\begin{enumerate} \itemsep -2pt
	\item Chris Enstrom, ``Cashing in on Diversity Grants and Scholarships,'' in Graduating Engineer \& Computer Careers. Available at: \url{http://www.graduatingengineer.com/higher-education/20061129/Cashing-in-on-Diversity-Grants-and-Scholarships-}; last accessed on August 25, 2010.
	\end{enumerate}
\item Gates Millennium Scholars (GMS) scholarship (for underrepresented minorities in the US): \url{http://www.gmsp.org/}
\item Society of Women Engineers (SWE): SWE Scholarships and other scholarships, \url{http://societyofwomenengineers.swe.org/index.php?option=com_content&task=view&id=222&Itemid=111}
\item Coalition to Diversify Computing: \url{http://www.cdc-computing.org/scholarships/}
\item IES Abroad (formerly Institute of European Studies / Institute for the International Education of Students): \vspace{-0.3cm}
	\begin{enumerate} \itemsep -2pt
	\item Diversity Abroad: \vspace{-0.2cm}
		\begin{enumerate} \itemsep -2pt
		\item \url{https://www.iesabroad.org/IES/Diversity/diversity.html}
		\item Programs to improve student diversity in study abroad programs
		\item IES Abroad Diversity Scholarships: \vspace{-0.1cm}
			\begin{enumerate} \itemsep -1pt
			\item IES Abroad Merit-Based Scholarship for Under-represented Students: \url{https://www.iesabroad.org/IES/Scholarships_and_Aid/Diversity_Scholarships/diversityScholarship.html}
			\item IES Abroad Merit-Based David Porter Diversity Scholarship (Up to \$5,000!): \url{https://www.iesabroad.org/IES/Scholarships_and_Aid/Merit_Based/davidPorterScholarship.html}
			\item HBCU Scholarships: \url{https://www.iesabroad.org/IES/Scholarships_and_Aid/Diversity_Scholarships/hbcuScholarship.html}
			\item HACU-IES Abroad Merit/Need-Based Scholarship: \url{https://www.iesabroad.org/IES/Scholarships_and_Aid/Diversity_Scholarships/HACUScholarship.html}
			\end{enumerate}
		\end{enumerate}
	\end{enumerate}
\item MassMutual Scholars Program: \vspace{-0.3cm}
	\begin{enumerate} \itemsep -2pt
	\item Applicants must be undergraduates of African American/Black, Asian/Pacific Islander or Hispanic decent in the US.
	\item Reside or plan to attend an institution in one of the following metropolitan areas: \vspace{-0.2cm}
		\begin{enumerate} \itemsep -2pt
		\item Atlanta, GA
		\item Chicago, IL
		\item Central New Jersey
		\item Denver, CO
		\item Houston, TX
		\item Miami, FL
		\item Los Angeles, CA
		\item San Antonio, TX
		\item San Francisco, CA
		\end{enumerate}
	\item Be majoring in business, economics, finance, financial planning, management, marketing or sales.
	\item \url{http://www.hsf.net/massmutual.aspx}
	\item \url{http://www.apiasf.org/scholarship_apiasf_massmutual.html}
	\end{enumerate}
\item {\it NASA}'s Minority University Research and Education Program (MUREP): \vspace{-0.3cm}
	\begin{enumerate} \itemsep -2pt
	\item \url{http://www.nasa.gov/offices/education/programs/national/murep/home/index.html}
	\item \url{http://www.nasa.gov/offices/education/about/murep_overview.html}
	\item Jenkins Pre-doctoral Fellowship Project, JPFP: \url{http://www.nasa.gov/offices/education/programs/descriptions/Jenkins_Predoctoral_Fellowship_Project.html}
	\end{enumerate}
\item UNCF: \vspace{-0.3cm}
	\begin{enumerate} \itemsep -2pt
	\item UNCF Special Programs Corporation: \vspace{-0.2cm}
		\begin{enumerate} \itemsep -2pt
		\item Harriett G. Jenkins Pre-doctoral Fellowship Program (JPFP) for underrepresented minorities pursuing graduate degrees in STEM: \url{http://www.uncfsp.org/spknowledge/default.aspx?page=program.view&areaid=1&contentid=177&typeid=jpfp}
		\item NASA Science and Technology Institute (NSTI) Summer Scholars Program (10-week summer research scholarship): \url{http://www.uncfsp.org/spknowledge/default.aspx?page=program.view&areaid=1&contentid=172&typeid=nstiinternship}
		\item Motivating Undergraduates in Science and Technology (MUST) Program for undergraduates in STEM: \url{http://www.uncfsp.org/spknowledge/default.aspx?page=program.view&areaid=1&contentid=346&typeid=must}
		\item Institute for International {\bf Public Policy} Fellows Program: \url{http://www.uncfsp.org/IIPP}
		\item \url{http://www.uncfsp.org/spknowledge/default.aspx?page=home.default}
		\end{enumerate}
	\item UNCF scholarship resources: \url{http://www.uncf.org/forstudents/scholarship.asp}
	\item UNCF $\cdot$ Merck Science Initiative: scholarships and fellowships: \url{http://umsi.uncf.org/ScholarshipsInternshipsFellowships/tabid/151/Default.aspx}
	\end{enumerate}
\item Hispanic College Fund: \vspace{-0.3cm}
	\begin{enumerate} \itemsep -2pt
	\item Scholarships: \url{http://www.hispanicfund.org/scholarships/} and \url{http://scholarships.hispanicfund.org/applications/}
	\item NASA MUST Scholarship Program: \url{http://www.hispanicfund.org/nasa-must/}
	\item Hispanic Youth Symposium (scholarships are awarded to winners of the art competition, talent competition, and speech competition): \url{http://www.hispanicyouth.org/about-the-program}
	\item \url{http://www.hispanicfund.org/}
	\end{enumerate}
\item Hispanic Heritage Foundation (HHF): \vspace{-0.3cm}
	\begin{enumerate} \itemsep -2pt
	\item Scholarships and Resources: \url{http://www.hispanicheritage.org/youth_int.php?sec=80}
	\item \url{http://www.hispanicheritage.org/}
	\end{enumerate}
\item Hispanic Scholarship Fund (HSF): \vspace{-0.3cm}
	\begin{enumerate} \itemsep -2pt
	\item Scholarship programs for: \vspace{-0.2cm}
		\begin{enumerate} \itemsep -2pt
		\item college students
		\item community college transfer students
		\item high school students
		\item Gates Millennium Scholars
		\item See \url{http://www.hsf.net/innercontent.aspx?id=34}
		\end{enumerate}
	\item \url{http://www.hsf.net/}
	\end{enumerate}
\item League of United Latin American Citizens (LULAC): \vspace{-0.3cm}
	\begin{enumerate} \itemsep -2pt
	\item LULAC National Educational Service Centers, Inc: \vspace{-0.2cm}
		\begin{enumerate} \itemsep -2pt
		\item \url{http://www.lnesc.org/}
		\item LULAC National Scholarship Fund (LNSF): \vspace{-0.1cm}
			\begin{enumerate} \itemsep -1pt
			\item \url{http://www.lulac.org/programs/education/scholarships/}
			\item \url{http://lnesc.org/index.asp?Type=B_BASIC&SEC={3AEDB506-F425-4E58-B9F6-44867E2FD943}}
%http://lnesc.org/index.asp?Type=B_BASIC&SEC={3AEDB506-F425-4E58-B9F6-44867E2FD943}
			\item Applicants must meet the following criteria to be considered for a scholarship: \vspace{-0.1cm}
				\begin{itemize} \itemsep -1pt
				\item Must be a U.S. citizen or legal resident
				\item Must have applied to or be enrolled in a   college, university, or graduate school, including 2-year colleges, or vocational schools that lead to an associate�s degree
				\item A student will not be eligible for a scholarship if he/she is related to a scholarship committee member, the Council President, or an individual contributor to the local funds of the Council
				\end{itemize}
			\item National Scholastic Achievement Awards (for high school seniors entering college, university, or vocational school)
			\item Honors Awards (for high school seniors entering college, university, or vocational school)
			\item General Awards (Need, community involvement, and leadership activities will also be considered)
			\item General Electric Foundation/ LULAC Scholarship program: for underrepresented minorities (US freshmen) entering their sophomore year as majors in Business or Engineering with a cumulative college G.P.A. $\leq$ 3.25/4.0; these students must be enrolled in a 4-year undergraduate program.
			\end{enumerate}
		\end{enumerate}
	\end{enumerate}
\item Hispanic Association of Colleges and Universities (HACU): \vspace{-0.3cm}
	\begin{enumerate} \itemsep -2pt
	\item HACU Student Programs Overview: \vspace{-0.2cm}
		\begin{enumerate} \itemsep -2pt
		\item \url{http://www.hacu.net/hacu/HACU_Student_Programs_EN.asp?SnID=1942709283}
		\item HACU Scholarship Programs: \vspace{-0.1cm}
			\begin{enumerate} \itemsep -1pt
			\item \url{http://www.hacu.net/hacu/Scholarships_EN.asp?SnID=1942709283}
			\item Includes scholarships for students in: \vspace{-0.1cm}
				\begin{itemize} \itemsep -1pt
				\item Accounting
				\item Behavioral Health
				\item Business
				\item Clinical Psychology
				\item Computer Engineering
				\item Computer Science
				\item Dental Technician
				\item Electrical Engineering
				\item Engineering
				\item Food Merchandising
				\item Information Technology
				\item International Business
				\item Management
				\item Marketing
				\item Mass Media
				\item Mental Health
				\item Merchandising
				\item Nursing
				\item Physician Assistant
				\item (Pre) Optometry
				\item (Pre) Dental
				\item (Pre) Medicine
				\item (Pre) Pharmacy
				\item Public Health
				\item Public Relations
				\item Retail Management
				\item Sports Marketing
				\item Technology
				\end{itemize}
			\end{enumerate}
		\item ``D{\'{a}}ndole Alas a Tu {\'{E}}xito/Giving Flight to Your Success'' travel award program (Southwest Airlines' Travel Award Program): \vspace{-0.1cm}
			\begin{enumerate} \itemsep -1pt
			\item For students with financial need who have to across the United States to participate in their undergraduate or graduate degree programs
			\item \url{http://www.hacu.net/hacu/Lanzate_EN.asp?SnID=1942709283}
			\item \url{http://www.hacu.net/hacu/Lanzate1_EN.asp?SnID=1808826658}
			\end{enumerate}
		\item HACU Study Abroad Scholarship Programs: \vspace{-0.1cm}
			\begin{enumerate} \itemsep -1pt
			\item \url{http://www.hacu.net/hacu/Study_Abroad_EN.asp?SnID=1808826658}
			\item HACU-Global Learning Semesters (GLS) Program: Hispanic Study Abroad Scholars: \url{http://www.studyabroadscholars.org/index.html}
			\item HACU-American Institute for Foreign Study (AIFS) Scholarship Program: \url{http://www.aifsabroad.com/scholarships.asp#hacu}
			\item HACU-Institute for the International Education of Students (IES) Scholarship Program: \url{https://www.iesabroad.org/IES/home.html}
			\item Hispanic Study Abroad Scholars program: \url{http://www.studyabroadscholars.org/index.html}
			\end{enumerate}
		\item Scholarship Resource List: \url{http://www.hacu.net/hacu/Scholarship_Resource_List_EN.asp?SnID=1109551622}
%		\item Scholarship Resource List: \url{http://www.hacu.net/hacu/Scholarship_Resource_List_EN.asp?SnID=1942709283}		-- Redundant
		\end{enumerate}
	\end{enumerate}
\item Congressional Hispanic Caucus Institute (CHCI): \vspace{-0.3cm}
	\begin{enumerate} \itemsep -2pt
	\item CHCI Scholarship: \vspace{-0.2cm}
		\begin{enumerate} \itemsep -2pt
		\item \url{http://www.chci.org/scholarships/}
		\item CHCI's scholarship opportunities are afforded to Latino students in the United States who have a history of performing public service-oriented activities in their communities and who demonstrate a desire to continue their civic engagement in the future. There is no GPA or academic major requirement. Students with excellent leadership potential are encouraged to apply.
		\item Scholarship awards are intended to provide assistance with tuition, room and board, textbooks, and other educational expenses associated with college enrollment.
		\item Students continue to receive annual disbursements as long as they maintain good academic standing.
		\item CHCI scholarships provide recipients with a one time scholarship of: \vspace{-0.1cm}
			\begin{enumerate} \itemsep -1pt
			\item \$1,000 community college or AA/AS granting institution
			\item \$2,500 4-year academic institution
			\item \$5,000 graduate-level institution
			\end{enumerate}
		\item Eligibility Criteria: \vspace{-0.1cm}
			\begin{enumerate} \itemsep -1pt
			\item Full-time enrollment in a United States Department of Education accredited community college, four-year university, or graduate/professional program during the period for which scholarship is requested
			\item Demonstrated financial need
			\item Consistent, active participation in public and/or community service activities
			\item Strong writing skills
			\item U.S. citizenship or legal permanent residency
			\end{enumerate}
		\end{enumerate}
	\item CHCI Fellowships: \vspace{-0.2cm}
		\begin{enumerate} \itemsep -2pt
		\item \url{http://www.chci.org/fellowships/}
		\item CHCI {\bf Public Policy} Fellowship: \vspace{-0.1cm}
			\begin{enumerate} \itemsep -1pt
			\item This is a paid Fellowship Program that offers talented Latinos, who have earned a bachelor's degree within two years of the program start date, the opportunity to gain hands-on experience at the national level in public policy.
			\item Fellows have the opportunity to work in congressional offices and federal agencies, depending on their area of interest.  Some past focus areas have included international affairs, economic development, health and education policy, housing, or local government.
			\item Program Dates: August to May (10-month internship)
			\item \url{http://www.chci.org/fellowships/page/chci-public-policy-fellowship}
			\end{enumerate}
		\item CHCI Graduate Fellowship Program: \vspace{-0.1cm}
			\begin{enumerate} \itemsep -1pt
			\item The CHCI Graduate Fellowship Program seeks to enhance participants' leadership abilities, strengthen professional skills and ultimately produce more competent and competitive Latino professionals in underserved {\bf public policy} issue areas.
			\item This paid Fellowship Program offers exceptional Latinos who have earned a graduate degree or higher related to a chosen policy issue area within three years of program start date unparalleled exposure to hands-on experience in public policy.
			\item This program focuses specifically on the areas of: \vspace{-0.1cm}
				\begin{itemize} \itemsep -1pt
				\item Higher Education: CHCI Graduate Higher Education Fellowship, \url{http://www.chci.org/fellowships/page/chci-graduate-higher-education-fellowship}
				\item Secondary Education: CHCI Graduate Secondary Education Fellowship, \url{http://www.chci.org/fellowships/page/chci-graduate-secondary-education-fellowship}
				\item Health: CHCI Graduate Health Fellowship, \url{http://www.chci.org/fellowships/page/chci-graduate-health-fellowship}
				\item Housing: CHCI Graduate Housing Fellowship, \url{http://www.chci.org/fellowships/page/chci-graduate-housing-fellowship}
				\item International Affairs (includes last three months abroad in Mexico): CHCI Graduate International Affairs Fellowship, \url{http://www.chci.org/fellowships/page/chci-graduate-international-affairs-fellowship}
				\item Law: CHCI Graduate Law Fellowship, \url{http://www.chci.org/fellowships/page/chci-graduate-law-fellowship}
				\item STEM (Science, Technology, Engineering and Math): CHCI Graduate STEM Fellowship, \url{http://www.chci.org/fellowships/page/chci-graduate-stem-fellowship}
				\end{itemize}
			\item Program Dates: August to May (10-month internship)
			\item \url{http://www.chci.org/fellowships/page/chci-graduate-fellowship-program}
			\end{enumerate}
		\end{enumerate}
	\end{enumerate}
\item American Indian Graduate Center (AIGC): \vspace{-0.3cm}
	\begin{enumerate} \itemsep -2pt
	\item AIGC scholarships and fellowships: \vspace{-0.2cm}
		\begin{enumerate} \itemsep -2pt
		\item for advanced degree students in art, music, environmental studies, journalism, communications, medicine, dentistry, public health, nursing, or other health-related fields
		\item for members of Wisconsin, New Mexico or Arizona tribes.
		\item \url{http://www.aigc.com/02scholarships/scholarships.htm}
		\item AIGC Fellowship (Graduate) for Native Americans and their descendants seeking advanced degrees: \url{http://www.aigc.com/02scholarships/aigc/fellowship.htm}
		\item Rainer Scholarship (for grad students): \url{http://www.aigc.com/02scholarships/rainer.htm}
		\end{enumerate}
	\item List of resources about scholarships and fellowships: \vspace{-0.2cm}
		\begin{enumerate} \itemsep -2pt
		\item \url{http://www.aigc.com/08otherscholarship/otherscholarships.html}
		\item Scholarships: \url{http://www.aigc.com/08otherscholarship/scholarships.htm}
		\item Fellowships: \url{http://www.aigc.com/08otherscholarship/fellowships.htm}
		\end{enumerate}
	\item Gates Millennium Scholar Program (for individuals seeking basic and advanced degrees): \url{http://www.aigc.com/03gms/gms.htm}
	\end{enumerate}
\item Asian \& Pacific Islander American Scholarship Fund (APIASF) scholarship resources: \url{http://www.apiasf.org/scholarships.html}
\item American Association of University Women: \vspace{-0.3cm}
	\begin{enumerate} \itemsep -2pt
	\item \url{http://www.aauw.org/learn/fellowships_grants/index.cfm}
	\end{enumerate}
\item Sigma Delta Epsilon-Graduate Women in Science (GWIS): \url{http://www.gwis.org/programs.html}
\item Society of Hispanic Professional Engineers (SHPE): \vspace{-0.3cm}
	\begin{enumerate} \itemsep -2pt
	\item Advancing Hispanic Excellence in Technology, Engineering, Math and Science (AHETEMS) Foundation: \url{http://www.ahetems.org/}
	\item AHETEMS Scholarship Program: \url{http://www.ahetems.org/scholarships/}
	\item Graduate \& Young Professional Fellowship Program (encourage young professionals to engage in {\bf public policy}): \url{http://www.ahetems.org/graduate/graduate-young-professional-fellowship-program/}
	\item SHPE/GEM Fellowship (for graduate students in STEM at GEM Member Universities): \url{http://www.ahetems.org/graduate/shpe-gem-graduate-award/}. See \url{http://www.gemfellowship.org/gem-universities/university-members} for a list of GEM member universities.
	\end{enumerate}
\item National Society of Black Engineers (NSBE): \vspace{-0.3cm}
	\begin{enumerate} \itemsep -2pt
	\item Scholarships: \url{http://www.nsbe.org/Programs/Scholarships.aspx}
	\end{enumerate}
\item The Society of Mexican American Engineers and Scientists (MAES): \vspace{-0.3cm}
	\begin{enumerate} \itemsep -2pt
	\item Scholarships \& Awards: \url{http://www.maes-natl.org/index.php?meid=328}
	\item MAES Scholarship Program: \url{http://www.maes-natl.org/index.php?module=ContentExpress&func=display&ceid=518&meid=241}
	\end{enumerate}
\item SACNAS (Society for Advancement of Chicanos and Native Americans in Science): \vspace{-0.3cm}
	\begin{enumerate} \itemsep -2pt
	\item Scholarships: \url{http://www.sacnas.org/webadindex.cfm?webadcategory_id=7}
	\item Fellowships: \url{http://www.sacnas.org/webadIndex.cfm?webadcategory_id=5}
	\end{enumerate}
\item {\it Center for the Advancement of Hispanics in Science and Engineering Education} (CAHSEE): \vspace{-0.3cm}
	\begin{enumerate} \itemsep -2pt
	\item Scholarships: \url{http://www.cahsee.org/6resources/scholarships.asp.htm}
	\end{enumerate}
\item National Consortium for Graduate Degrees for Minorities in Engineering and Science, Inc.: \vspace{-0.3cm}
	\begin{enumerate} \itemsep -2pt
	\item National GEM Consortium: GEM Fellowship, \url{http://www.gemfellowship.org/gem-fellowship/application-requirements}
	\end{enumerate}
\item National Physical Science Consortium (NPSC): \vspace{-0.3cm}
	\begin{enumerate} \itemsep -2pt
	\item NPSC Graduate Fellowship: \url{http://www.npsc.org/}
	\end{enumerate}
\item Finch College Alumnae Association: \vspace{-0.3cm}
	\begin{enumerate} \itemsep -2pt
	\item The Finch College Alumnae Foundation Education Grant: \vspace{-0.2cm}
		\begin{enumerate} \itemsep -2pt
		\item \url{http://www.finchcollege.org/newscholarships.html}
		\item \url{http://www.finchcollege.org/newFinchGrantQandA.html}
		\item ``THE FINCH GRANT, an annual program where four community college women entering a four year college are awarded a grant of \$1500 which can be used toward any needs to completing college.  The selection is determined by a panel of college professors.''
		\end{enumerate}
	\end{enumerate}
\item : \url{}
\item : \url{}
\item : \url{}
\item : \url{}
\item : \url{}
\item \S\ref{phdandpostdocfellowships} has more information concerning scholarships and fellowships in the following areas: \vspace{-0.3cm}
	\begin{enumerate} \itemsep -2pt
	\item electronic design automation (EDA), and related areas of design automation: \vspace{-0.2cm}
		\begin{enumerate} \itemsep -2pt
		\item bio design automation (BDA)
		\item Lab-on-chip design (LoC) automation
		\item MEMS/NEMS design automation
		\end{enumerate}
	\item digital VLSI design
	\item analog and mixed-signal (AMS) VLSI design
	\item computer architecture
	\item parallel computing
	\item concurrent programming
	\item data mining
	\item theoretical computer science
	\end{enumerate}
\item Ph.D. dissertation awards: \vspace{-0.3cm}
	\begin{enumerate} \itemsep -2pt
	\item --- --- --- --- --- --- --- --- --- --- --- --- --- --- --- --- --- --- --- --- --- --- --- --- --- --- --- --- --- --- ---
	\item \colorbox{blue}{\bf Ph.D. Dissertation Awards for Computer Science}
	% Ph.D. Dissertation Awards for Computer Science
	\item ACM Doctoral Dissertation Award: \url{http://awards.acm.org/doctoral_dissertation/}
	\item ACM Outstanding Ph.D. Dissertation Award in Electronic Design Automation: \url{http://www.sigda.org/opda.html}
	\item EDAA Outstanding Dissertation Award (European Design and Automation Association, EDAA): \url{http://www.edaa.com/dissertation_award.html} and \url{http://www.esat.kuleuven.be/micas/EDAA-Award/index.php}
	\item EuroSys Roger Needham PhD Award (in the systems area): \vspace{-0.2cm}
		\begin{enumerate} \itemsep -2pt
		\item Areas in systems include: \vspace{-0.1cm}
			\begin{enumerate} \itemsep -1pt
			\item operating systems
			\item distributed systems
			\item real-time systems
			\item systems aspects of databases
			\item language runtimes
			\item \colorbox{yellow}{\bf embedded systems}
			\item computer networks
			\end{enumerate}
		\item \url{http://www.eurosys.org/phdprize/index.php}
		\end{enumerate}
	\item ACM SIGPLAN Outstanding Doctoral Dissertation Award: \url{http://www.sigplan.org/award-dissertation.htm}
	\item ACM SIGKDD Doctoral Disseration Award (in data mining and knowledge discovery): \url{http://www.sigkdd.org/awards_dissertation.php}
	\item ACM SIGMOD Jim Gray Doctoral Dissertation Award (in the database field): \url{http://www.sigmod.org/sigmod-awards/doctoral-dissertation-award}
	\item Special Interest Group of the ACM on Management Information Systems (SIGMIS): \vspace{-0.2cm}
		\begin{enumerate} \itemsep -2pt
		\item ACM SIGMIS Doctoral Dissertation Award Competition (at the International Conference on Information Systems, ICIS): \url{http://ai.arizona.edu/icis2009/program/dissertation.html} and \url{http://icis2010.aisnet.org/dissertation_award.htm}
		\end{enumerate}
	\item Association for Symbolic Logic: \vspace{-0.2cm}
		\begin{enumerate} \itemsep -2pt
		\item ``The Sacks Prize is awarded for the most outstanding doctoral dissertation in mathematical logic''.
		\item \url{http://www.aslonline.org/Sacks_nominations.html} and \url{http://www.aslonline.org/info-prizes.html}
		\end{enumerate}
	\item European Association for Computer Science Logic (EACSL): \vspace{-0.2cm}
		\begin{enumerate} \itemsep -2pt
		\item Ackermann Award (for outstanding dissertations in Logic in Computer Science): \url{http://www.eacsl.org/} and \url{http://www.eacsl.org/award.html}
		\end{enumerate}
	\item European Coordinating Committee for Artificial Intelligence (ECCAI): \vspace{-0.2cm}
		\begin{enumerate} \itemsep -2pt
		\item 201X Artificial Intelligence Dissertation Award: \url{http://www.eccai.org/diss-award/current.shtml}
		\end{enumerate}
	\item European Conference on Wireless Sensor Networks (EWSN 201X, \url{http://www.nes.uni-due.de/ewsn2011}) and CONET, the Cooperating Objects Network of Excellence: Ph.D. Thesis Award Competition, \url{http://www.cooperating-objects.eu/}
	\item --- --- --- --- --- --- --- --- --- --- --- --- --- --- --- --- --- --- --- --- --- --- --- --- --- --- --- --- --- --- ---
	\item \colorbox{blue}{\bf Ph.D. Dissertation Awards for Mathematics}
	% Ph.D. Dissertation Awards for Mathematics
	\item International Center for Scientific Research (CIRS): \vspace{-0.2cm}
		\begin{enumerate} \itemsep -2pt
		\item E. W. Beth Dissertation Prize (for outstanding dissertations in the fields of Logic, Language and Information): \url{http://www.cirs.net/prix/awards.php?id=481}
		\end{enumerate}
	\item The Association for Operations Management, APICS (Advancing Productivity, Innovation, and Competitive Success): \vspace{-0.2cm}
		\begin{enumerate} \itemsep -2pt
		\item Plossl Doctoral Dissertation Competition: The APICS Educational and Research Foundation, will annually grant one award of \$2,500 for a doctoral dissertation dealing with any topic in operations management. Sample topics include operations strategy, operations planning and control systems, supply chain management, quality management, Six Sigma, facility location, forecasting, just-in-time/lean production systems, and project management. Entrants must be candidates for the doctorate in operations management. The dissertation must be approved by the primary thesis advisor and not more than 50\% completed at time of submission. See \url{http://www.apics.org/Education/ERFoundation/Competitions/plossl.htm}.
		\end{enumerate}
	\item SIAM Richard C. DiPrima Prize: \vspace{-0.2cm}
		\begin{enumerate} \itemsep -2pt
		\item The Richard C. DiPrima Prize is awarded every two years to a junior scientist, based on an outstanding doctoral dissertation in applied mathematics.
		\item \url{http://www.siam.org/prizes/nominations/nom_diprima.php}
		\item \url{http://www.siam.org/prizes/sponsored/diprima.php}
		\end{enumerate}
	\item MOS A.W. Tucker Prize: \vspace{-0.2cm}
		\begin{enumerate} \itemsep -2pt
		\item It is awarded for an outstanding doctoral thesis in any aspect of mathematical optimization.
		\item \url{http://www.mathprog.org/?nav=tucker}
		\end{enumerate}
	\item --- --- --- --- --- --- --- --- --- --- --- --- --- --- --- --- --- --- --- --- --- --- --- --- --- --- --- --- --- --- ---
	\item \colorbox{blue}{\bf Other Ph.D. Dissertation Awards}
	% Other Ph.D. Dissertation Awards
	\item Institute for Operations Research and the Management Sciences (INFORMS): \vspace{-0.2cm}
		\begin{enumerate} \itemsep -2pt
		\item INFORMS George B. Dantzig Dissertation Award: \url{http://www.informs.org/Recognize-Excellence/INFORMS-Prizes-Awards/George-B.-Dantzig-Dissertation-Award}
		\item Best Dissertation Award (Technology Management Section, for Ph.D. theses in technology management): \url{http://www.informs.org/Recognize-Excellence/INFORMS-Community-Prizes-and-Awards2/Technology-Management-Section/Best-Dissertation-Award}
		\item TSL Dissertation Prize (Transportation Science and Logistics Section, for doctoral dissertations in the transportation science and logistics area): \url{http://www.informs.org/Recognize-Excellence/INFORMS-Community-Prizes-and-Awards2/Transportation-Science-and-Logistics-Section/TSL-Dissertation-Prize}
		\item Best Dissertation Award (Telecommunications Section, for Ph.D. theses in telecommunications): \url{http://www.informs.org/Recognize-Excellence/INFORMS-Community-Prizes-and-Awards2/Telecommunications-Section/Best-Dissertation-Award}
		\item Frank M. Bass Dissertation Paper Award (Society for Marketing Science, for the best marketing paper derived from a Ph.D. thesis published in an INFORMS-sponsored journal): \url{http://www.informs.org/Recognize-Excellence/INFORMS-Community-Prizes-and-Awards2/Society-for-Marketing-Science/Frank-M.-Bass-Dissertation-Paper-Award}
		\item SOLA - Air Products Bi-Annual Dissertation Award (Section on Location Analysis, for Ph.D. theses on location related research): \url{http://www.informs.org/Recognize-Excellence/INFORMS-Community-Prizes-and-Awards2/Section-on-Location-Analysis/SOLA-Air-Products-Bi-Annual-Dissertation-Award}
		\end{enumerate}
	\item EURO Doctoral Dissertation Award (EDDA) (in operations research): \url{http://www.euro-online.org/display.php?page=edda1}
	\end{enumerate}
\item Other awards: \vspace{-0.3cm}
	\begin{itemize} \itemsep -2pt
	\item --- --- --- --- --- --- --- --- --- --- --- --- --- --- --- --- --- --- --- --- --- --- --- --- --- --- --- --- --- --- ---
	\item \colorbox{blue}{\bf Awards for Computer Science}
	% Awards for Computer Science
	\item ACM SIGMOD Undergraduate Award: \url{http://www.sigmod.org/sigmod-awards/sigmod-awards#undergraduate}
	\item European Association of Theoretical Computer Science (EATCS): Presburger Award (for young researchers in theoretical computer science), \url{http://www.eatcs.org/index.php/presburger}.
	\item Computer Research Association: \vspace{-0.2cm}
		\begin{enumerate} \itemsep -2pt
		\item Committee on the Status of Women in Computing Research (CRA-W): \vspace{-0.1cm}
			\begin{enumerate} \itemsep -1pt
			\item Borg Early Career Award (BECA): \url{http://www.cra-w.org/borg}
			\end{enumerate}
		\end{enumerate}
	\item European Conference on Wireless Sensor Networks (EWSN 201X, \url{http://www.nes.uni-due.de/ewsn2011}) and CONET, the Cooperating Objects Network of Excellence: Ph.D. Thesis Award Competition, \url{http://www.cooperating-objects.eu/}. ``Cooperating Objects combine the strong functional aspects of embedded systems, pervasive computing and wireless sensor networks. Cooperating objects entities federate themselves into dynamic and loose networks in order to reach a common goal. This common goal will typically be related to sensing or control.''
	\item --- --- --- --- --- --- --- --- --- --- --- --- --- --- --- --- --- --- --- --- --- --- --- --- --- --- --- --- --- --- ---
	\item \colorbox{blue}{\bf Awards for Biomedical Engineering}
	% Awards for Biomedical Engineering
	\item Biomedical Engineering Society (BMES): \vspace{-0.2cm}
		\begin{enumerate} \itemsep -2pt
		\item Rita Schaffer Young Investigator Award (for junior researchers in biomedical engineering): \url{http://www.bmes.org/aws/BMES/pt/sp/awards_investigator}
		\item Graduate and Undergraduate Student Awards: \url{http://www.bmes.org/aws/BMES/pt/sp/awards_student}
		\end{enumerate}
	\item --- --- --- --- --- --- --- --- --- --- --- --- --- --- --- --- --- --- --- --- --- --- --- --- --- --- --- --- --- --- ---
	\item \colorbox{blue}{\bf Awards for Mechanical Engineering}
	% Awards for Mechanical Engineering
	\item American Society of Mechanical Engineers (ASME): \vspace{-0.2cm}
		\begin{enumerate} \itemsep -2pt
		\item Henry Hess Award (authors of research papers who are below 31 years old): \url{http://www.asme.org/Governance/Honors/SocietyAwards/Henry_Hess_Award.cfm}
		\item Pi Tau Sigma Gold Medal (outstanding junior engineers): \url{http://www.asme.org/Governance/Honors/SocietyAwards/Pi_Tau_Sigma_Gold_Medal.cfm}
		\item Marshall B. Peterson Award (researchers in tribology who are below 30 years old): \url{http://www.asme.org/Governance/Honors/SocietyAwards/Marshall_B_Peterson_Award.cfm}
		\item Y.C. Fung Young Investigator Award (for young researchers in bioengineering): \url{http://www.asme.org/Governance/Honors/SocietyAwards/YC_Fung_Young_Investigator.cfm}
		\end{enumerate}
	\item --- --- --- --- --- --- --- --- --- --- --- --- --- --- --- --- --- --- --- --- --- --- --- --- --- --- --- --- --- --- ---
	\item \colorbox{blue}{\bf Awards for Civil Engineering}
	% Awards for Civil Engineering
	\item American Society of Civil Engineers (ASCE): \vspace{-0.3cm}
		\begin{enumerate} \itemsep -2pt
		\item Edmund Friedman Young Engineer Award for Professional Achievement (for junior engineers under the age of 36): \url{http://www.asce.org/AwardsContent.aspx?id=16776}
		\item Committee on Younger Members (CYM) Awards (for junior engineers): \url{http://www.asce.org/Content.aspx?id=11311}
		\item Collingwood Prize (for civil engineering researchers under the age of 35): \url{http://www.asce.org/AwardsContent.aspx?id=15352}
		\end{enumerate}
	\item --- --- --- --- --- --- --- --- --- --- --- --- --- --- --- --- --- --- --- --- --- --- --- --- --- --- --- --- --- --- ---
	\item \colorbox{blue}{\bf Awards for Chemical Engineering}
	% Awards for Chemical Engineering
	\item American Institute of Chemical Engineers (AIChE) awards: \url{http://www.aiche.org/Students/Awards/index.aspx}
	\item --- --- --- --- --- --- --- --- --- --- --- --- --- --- --- --- --- --- --- --- --- --- --- --- --- --- --- --- --- --- ---
	\item \colorbox{blue}{\bf Awards for Systems Engineering}
	% Awards for Systems Engineering
	\item International Council on Systems Engineering (INCOSE) Stevens Doctoral Award (for Promising Research in Systems Engineering and Integration; A.B.D.s / Ph.D. candidates): \url{http://www.incose.org/about/foundation/doctoralaward.aspx}
	\item --- --- --- --- --- --- --- --- --- --- --- --- --- --- --- --- --- --- --- --- --- --- --- --- --- --- --- --- --- --- ---
	\item \colorbox{blue}{\bf Awards for Mathematics, Operations Research, \& Management Sciences}
	% Awards for Mathematics, Operations Research, and Management Sciences
	\item Institute for Operations Research and the Management Sciences (INFORMS): \vspace{-0.2cm}
		\begin{enumerate} \itemsep -2pt
		\item INFORMS Undergraduate Operations Research Prize: \url{http://www.informs.org/Recognize-Excellence/INFORMS-Prizes-Awards/INFORMS-Undergraduate-Operations-Research-Prize}
		\item Optimization Prize for Young Researchers: \url{http://www.informs.org/Recognize-Excellence/INFORMS-Community-Prizes-and-Awards2/Optimization-Society/Optimization-Prize-for-Young-Researchers}
		\item Underrepresented Minorities and Women Honoraria: \url{http://www.informs.org/Recognize-Excellence/INFORMS-Community-Prizes-and-Awards2/Simulation-Society/Underrepresented-Minorities-and-Women-Honoraria}
		\item Best Dissertation Proposal Competition (College on Organization Science, for Ph.D. proposals in organizational science): \url{http://www.informs.org/Recognize-Excellence/INFORMS-Community-Prizes-and-Awards2/College-on-Organization-Science/Best-Dissertation-Proposal-Competition}
		\item ISMS Doctoral Dissertation Proposal Competition (Society for Marketing Science, for Ph.D. proposals in marketing): \url{http://www.informs.org/Recognize-Excellence/INFORMS-Community-Prizes-and-Awards2/Society-for-Marketing-Science/ISMS-Doctoral-Dissertation-Proposal-Competition}
		\end{enumerate}
	\item Alice T. Schafer Mathematics Prize For Excellence in Mathematics by an Undergraduate Woman: \url{http://www.awm-math.org/schaferprize.html}
	\item European Prize in Combinatorics: \vspace{-0.2cm}
		\begin{enumerate} \itemsep -2pt
		\item The prize is established to recognize excellent contributions in Combinatorics by young European researchers (eligibility of EU) not older than 35. 
		\item \url{http://www.math.tu-berlin.de/EuroComb05/prize.html}
		\end{enumerate}
	\item The AMS-MAA-SIAM Frank and Brennie Morgan Prize for Outstanding Research in Mathematics by an Undergraduate Student: \url{http://www.maa.org/awards/morgan.html}; \url{http://www.ams.org/profession/prizes-awards/ams-prizes/morgan-prize}; and \url{http://www.siam.org/prizes/sponsored/morgan.php}
	\item --- --- --- --- --- --- --- --- --- --- --- --- --- --- --- --- --- --- --- --- --- --- --- --- --- --- --- --- --- --- ---
	% Lists of awards
	\item \colorbox{blue}{\bf Lists of awards}: \vspace{-0.2cm}
		\begin{enumerate} \itemsep -2pt
		\item Association for Women in Science: \url{http://www.awis.org/displaycommon.cfm?an=1&subarticlenbr=69}
		\item International Center for Scientific Research (CIRS): \url{http://www.cirs.net/indexenglish.htm}
		\end{enumerate}
	\end{itemize}
\end{enumerate}














%%%%%%%%%%%%%%%%%%%%%%%%%%%%%%%%%%%%%%%%%%%
\section{Funding Nonprofit Organizations}
\label{fundingnonprofitorg}

Funding nonprofit organizations (including colleges and universities): \vspace{-0.3cm}
\begin{enumerate} \itemsep -4pt
\item Alfred P. Sloan Foundation: \vspace{-0.3cm}
	\begin{enumerate} \itemsep -2pt
	\item Major Program Areas: \url{http://www.sloan.org/program/1}
	\item Apply for Grants: \url{http://www.sloan.org/apply}
	\end{enumerate}
\item The Commonwealth Fund: \vspace{-0.3cm}
	\begin{enumerate} \itemsep -2pt
	\item Grants \& Programs: \vspace{-0.2cm}
		\begin{enumerate} \itemsep -2pt
		\item \url{http://www.commonwealthfund.org/Grants-and-Programs.aspx}
		\item ``The Fund supports independent research on health and social issues and makes grants to improve health care practice and policy. We are dedicated to helping people become more informed about their health care and improving care for vulnerable populations such as children, the elderly, low-income families, minorities, and the uninsured.''
		\end{enumerate}
	\end{enumerate}
\item The Heinz Endowments (Howard Heinz Endowment \& Vira I. Heinz Endowment): \vspace{-0.3cm}
	\begin{enumerate} \itemsep -2pt
	\item \url{http://www.heinz.org/grants.aspx}
	\item grant-making programs (for non-profit organizations): \vspace{-0.2cm}
		\begin{enumerate} \itemsep -2pt
		\item Arts \& Culture
		\item Children, Youth \& Families
		\item Education
		\item Environment
		\item Innovation Economy
		\end{enumerate}
	\end{enumerate}
\item Ford Foundation: \vspace{-0.3cm}
	\begin{enumerate} \itemsep -2pt
	\item Grants: \vspace{-0.2cm}
		\begin{enumerate} \itemsep -2pt
		\item \url{http://www.fordfoundation.org/grants/}
		\item Individuals Seeking Fellowships: \vspace{-0.1cm}
			\begin{enumerate} \itemsep -1pt
			\item \url{http://www.fordfoundation.org/grants/individuals-seeking-fellowships}
			\item Ford Foundation Fellowship Programs: \url{http://sites.nationalacademies.org/PGA/FordFellowships/index.htm}
			\item Ford Foundation International Fellowships Program: \url{http://www.fordifp.net/}
			\end{enumerate}
		\item Organizations Seeking Grants: \url{http://www.fordfoundation.org/grants/organizations-seeking-grants}
		\item Other Philanthropic Resources: \url{http://www.fordfoundation.org/grants/other-philanthropic-resources}
		\item Grant Search Results (list of grants): \url{http://www.fordfoundation.org/grants/search}
		\end{enumerate}
	\end{enumerate}
\item The Rockefeller Foundation: \vspace{-0.3cm}
	\begin{enumerate} \itemsep -2pt
	\item Grants \& Grantees: \vspace{-0.2cm}
		\begin{enumerate} \itemsep -2pt
		\item \url{http://www.rockefellerfoundation.org/grants}
		\item What We Fund: \url{http://www.rockefellerfoundation.org/grants/what-we-fund}
		\item Resources for Grantseekers: Links to other Philanthropic Sources, \url{http://www.rockefellerfoundation.org/grants/resources-grantseekers}
		\end{enumerate}
	\end{enumerate}
\item Carnegie Corporation of New York: \vspace{-0.3cm}
	\begin{enumerate} \itemsep -2pt
	\item Grantseekers: \vspace{-0.2cm}
		\begin{enumerate} \itemsep -2pt
		\item \url{http://carnegie.org/grants/grantseekers/}
		\item What we fund: \url{http://carnegie.org/grants/grantseekers/what-we-fund/}
		\item What we don't fund: \url{http://carnegie.org/grants/grantseekers/what-we-dont-fund/}
		\end{enumerate}
		\item Grants database: \url{http://carnegie.org/grants/grants-database/} and \url{http://carnegie.org/grants/}
		\item (Past) individual foundation grants: \url{http://carnegie.org/publications/carnegie-reporter/single/view/article/item/221/}
	\end{enumerate}
\item The Kresge Foundation: \vspace{-0.3cm}
	\begin{enumerate} \itemsep -2pt
	\item fields of interest: \vspace{-0.2cm}
		\begin{enumerate} \itemsep -2pt
		\item health,
		\item the environment,
		\item community development,
		\item arts and culture,
		\item education, and
		\item human services
		\end{enumerate}
	\item Values Criteria (for grantmaking): \url{http://www.kresge.org/index.php/who/our_values_criteria/}
	\item funding methods: \vspace{-0.2cm}
		\begin{enumerate} \itemsep -2pt
		\item \url{http://www.kresge.org/index.php/how/index/}
		\item \url{http://www.kresge.org/index.php/our_funding_methods/index/}
		\end{enumerate}
	\item Challenge Grant: \vspace{-0.2cm}
		\begin{enumerate} \itemsep -2pt
		\item \url{http://www.kresge.org/index.php/our_funding_methods/challenge_grant_program/}
		\item ``The Kresge Foundation awards facilities capital as a challenge grant to help nonprofit organizations build their base of private financial support as they conduct capital campaigns to build or renovate their facilities.''
		\item ``Facilities capital challenge grants are awarded to organizations that cater specifically to the needs of poor, disadvantaged and disenfranchised in six program areas: Health Program, the Environment Program, Arts and Culture Program, Education Program, Human Services Program, and Community Development / Detroit Program.''
		\item ``Most challenge grant awards are made to U.S.-based organizations. On rare occasions, we award challenge grants to international organizations undertaking exceptional projects that align with the strategic objectives of a given program and advance Kresge's values.''
		\end{enumerate}
	\item Detroit Program: \vspace{-0.2cm}
		\begin{enumerate} \itemsep -2pt
		\item Kresge Arts Support: \url{http://www.kresge.org/index.php/what/detroit_program/kresge_arts_support/}
		\item Kresge Arts in Detroit: \url{http://www.kresge.org/index.php/what/detroit_program/kresge_arts_in_detroit/}
		\end{enumerate}
	\item Our Grants: \vspace{-0.2cm}
		\begin{enumerate} \itemsep -2pt
		\item \url{http://www.kresge.org/index.php/our_grants/index/}
		\item grants database: \url{http://maps.foundationcenter.org/grantmakers/index.php?gmkey=KRES002}
		\item Arts and Community Building: \vspace{-0.1cm}
			\begin{enumerate} \itemsep -1pt
			\item \url{http://www.kresge.org/index.php/what/arts_and_culture/arts_and_community_building#Community Arts}
			\item ``Cultural institutions and artists animate our communities, bring disparate people together to share common experiences, and help us imagine a better future. As the demographics of our communities become more diverse, artists and cultural institutions help us bridge differences and build cross-cultural understanding. As our economy struggles, creative enterprises and creative sector leaders offer hope for community renewal and new job development.''
			\item two pilot initiatives: College/Arts initiative, and the Community Arts initiative
			\item ``The pilot cities [for the Community Arts initiative] include Baltimore, Maryland; Birmingham, Alabama; Detroit, Michigan; St. Louis, Missouri; and Tucson, Arizona.''
			\item ``Grants for Arts and Community Building are by invitation only.''
			\end{enumerate}
		\end{enumerate}
	\end{enumerate}
\item New York Women's Foundation: \vspace{-0.3cm}
	\begin{enumerate} \itemsep -2pt
	\item Grant Information and Application: \vspace{-0.2cm}
		\begin{enumerate} \itemsep -2pt
		\item \url{http://www.nywf.org/grant.html}
		\item focus areas: \vspace{-0.1cm}
			\begin{enumerate} \itemsep -1pt
			\item Anti-Violence and Safety
			\item Economic Security
			\item Health, Sexual Rights and Reproductive Justice
			\end{enumerate}
		\item ``Grants usually range from \$50,000 to a maximum of \$70,000 [that last for a year, and can be renewed up to 5 years].''
		\end{enumerate}
	\end{enumerate}
\item The Foundation Center: \vspace{-0.3cm}
	\begin{enumerate} \itemsep -2pt
	\item Grantseekers: \url{http://foundationcenter.org/getstarted/}
	\item Find funders: \url{http://foundationcenter.org/findfunders/}
	\item GrantSpace$^{\rm SM}$: \vspace{-0.2cm}
		\begin{enumerate} \itemsep -2pt
		\item \url{http://grantspace.org/}
		\item ``GrantSpace$^{\rm SM}$ will help you gain the knowledge and skills you need to get grants, manage your nonprofit, and improve your community.''
		\item ``Established in 1956 and today supported by close to 550 foundations, the Foundation Center is a national nonprofit service organization recognized as the nation�s leading authority on organized philanthropy, connecting nonprofits and the grantmakers supporting them to tools they can use and information they can trust. Its audiences include grantseekers, grantmakers, researchers, policymakers, the media, and the general public. The Center maintains the most comprehensive database on U.S. grantmakers and their grants; issues a wide variety of print, electronic, and online information resources; conducts and publishes research on trends in foundation growth, giving, and practice; and offers an array of free and affordable educational programs.''
		\item Resources for Non-U.S. Grantseekers: \url{http://grantspace.org/Tools/Knowledge-Base/Resources-for-Non-U.S.-Grantseekers}
		\item Resources for Individual Grantseekers: \vspace{-0.1cm}
			\begin{enumerate} \itemsep -1pt
			\item \url{http://grantspace.org/Tools/Knowledge-Base/Individual-Grantseekers}
			\item \url{http://gtionline.foundationcenter.org/}
			\item General: \url{http://grantspace.org/Tools/Knowledge-Base/Individual-Grantseekers/General}
			\item Artists: \url{http://grantspace.org/Tools/Knowledge-Base/Individual-Grantseekers/Artists}
			\item Students: \url{http://grantspace.org/Tools/Knowledge-Base/Individual-Grantseekers/Students}
			\item Fiscal Sponsorship: \url{http://grantspace.org/Tools/Knowledge-Base/Individual-Grantseekers/Fiscal-Sponsorship}
			\item For-Profit Enterprises: \url{http://grantspace.org/Tools/Knowledge-Base/Individual-Grantseekers/For-Profit-Enterprises}
			\end{enumerate}
		\end{enumerate}
	\end{enumerate}
\item The Lemelson Foundation: \vspace{-0.3cm}
	\begin{enumerate} \itemsep -2pt
	\item \url{http://web.mit.edu/invent/w-foundation.html}
	\item Programs \& Grants: \url{http://www.lemelson.org/programs-grants}
	\item Grantmaking: \url{http://www.lemelson.org/grantmaking}
	\end{enumerate}
\item Partnership for Higher Education in Africa (PHEA): \vspace{-0.3cm}
	\begin{enumerate} \itemsep -2pt
	\item \url{http://www.foundation-partnership.org/} and \url{http://www.foundation-partnership.org/index.php?id=1}
	\item Grants Database: \url{http://www.foundation-partnership.org/index.php?id=2}
	\item Partnership Publications: \url{http://www.foundation-partnership.org/index.php?id=3}
	\end{enumerate}
\item Smithsonian Institution: \vspace{-0.3cm}
	\begin{enumerate} \itemsep -2pt
	\item Smithsonian Institution Traveling Exhibition Service (SITES): \vspace{-0.2cm}
		\begin{enumerate} \itemsep -2pt
		\item Smithsonian Community Grant program (supported by MetLife Foundation): \vspace{-0.1cm}
			\begin{enumerate} \itemsep -1pt
			\item \url{http://www.sites.si.edu/funding/grant2.htm}
			\item ``This program seeks to deepen connections between SITES' host venues and their communities by encouraging exhibitors to engage their local audiences in new and exciting ways while creating broader access to our exhibitions.''
			\item ``Under this new program, eligible SITES exhibitors may apply for up to \$5,000 for expenses related to public, educational programming produced in conjunction with a SITES exhibit. Exhibitors may choose to enhance current program offerings or to create a new program especially suited to the topic of the exhibition.''
			\end{enumerate}
		\end{enumerate}
	\end{enumerate}
\end{enumerate}
















%%%%%%%%%%%%%%%%%%%%%%%%%%%%%%%%%%%%%%%%%%%
\section{Technology-Related Public Policy}
\label{techpublicpolicy}

Resources for engagement in creating technology-related public policy: \vspace{-0.3cm}
\begin{enumerate} \itemsep -4pt
\item Yale Journal of Law \& Technology (YJOLT): \vspace{-0.3cm}
	\begin{enumerate} \itemsep -2pt
	\item \url{http://www.yjolt.org/}
	\item \url{http://wingenroth.org/}
	\end{enumerate}
\item ACM Public Policy Office: \vspace{-0.3cm}
	\begin{enumerate} \itemsep -2pt
	\item It represents ACM and its US Public Policy Council (USACM) on information technology policy issues that impact the computing field.
	\item It seeks to educate policymakers and the public about policies that will that foster innovations in computing and related disciplines in ways that benefit society.
	\item It also informs ACM's members and the public about policy developments through its weblog, Washington Update newsletter and articles in ACM publications.
	\item ACM US Public Policy Council (USACM): \url{http://usacm.acm.org/}
	\item ACM Committee on Computers and Public Policy (CCPP): \url{http://www.acm.org/public-policy/acm-committee-on-computers-and-public-policy}
	\item \url{http://www.acm.org/public-policy}
	\end{enumerate}
\item IEEE: \vspace{-0.3cm}
	\begin{enumerate} \itemsep -2pt
	\item IEEE-USA: \url{http://www.ieeeusa.org/policy/default.asp}
	\item Smart Grids: \url{http://smartgrid.ieee.org/public-policy}
	\end{enumerate}
\item Computing Community Consortium (CCC): \url{http://www.cra.org/ccc/}
\item Computing Research Association (CRA): \vspace{-0.3cm}
	\begin{enumerate} \itemsep -2pt
	\item \url{http://www.cra.org/}
	\item CRA Government Affairs: \url{http://www.cra.org/govaffairs/index.php}
	\end{enumerate}
\item EngineeringPolicy.org: \url{http://www.engineeringpolicy.org/}
\item Congressional Bi-Partisan Robotics Caucus: \url{http://www.roboticscaucus.org/}
\item Advisory Committee for the Congressional Research and Development $[$R\&D$]$ Caucus: \url{http://www.researchcaucus.org/}
\item {\it National Academies Press} (NAP), from the (US) {\it National Academies}: \url{http://www.nap.edu/}
\item {\it Coalition to Diversify Computing}: \url{http://www.cdc-computing.org/}
\item American Institute of Aeronautics and Astronautics (AIAA): \vspace{-0.3cm}
	\begin{enumerate} \itemsep -2pt
	\item \url{http://www.aiaa.org/content.cfm?pageid=7}
	\end{enumerate}
\item : \url{}
\item : \url{}
\item : \url{}
\item : \url{}
\item : \url{}
\item : \url{}
\item : \url{}
\item : \url{}
\end{enumerate}






%%%%%%%%%%%%%%%%%%%%%%%%%%%%%%%%%%%%%%%%%%%
\section{Feminist Outreach}
\label{feministoutreach}

Feminist outreach: \vspace{-0.3cm}
\begin{enumerate} \itemsep -4pt
\item Myra Sadker Foundation: \vspace{-0.3cm}
	\begin{enumerate} \itemsep -2pt
	\item $100+$ Ideas to Promote Gender Equity in Schools and Beyond: \url{http://www.sadker.org/100ideas.html}
	\item Gender Equity Activities: \url{http://www.sadker.org/WhatYouCanDo.html}
	\item Gender Equity Activities for Concerned Citizens: \url{http://www.sadker.org/GenderEquity-citizens.html}
	\item Gender Equity Activities for Families: \url{http://www.sadker.org/GenderEquity-family.html}
	\item Gender Equity Activities for Teachers: \vspace{-0.2cm}
		\begin{enumerate} \itemsep -2pt
		\item Early Childhood: \url{http://www.sadker.org/GenderEquity-teacher1.html}
		\item Primary Grades: \url{http://www.sadker.org/GenderEquity-teacher2.html}
		\item Upper Elementary: \url{http://www.sadker.org/GenderEquity-teacher3.html}
		\item Middle and High School: \url{http://www.sadker.org/GenderEquity-teacher4.html}
		\end{enumerate}
	\item Resources for feminism and links to web pages of feminist organizations: \url{http://www.sadker.org/ReadsLinks.html}
	\end{enumerate}
\item Feminist student organizations at colleges and universities: \vspace{-0.3cm}
	\begin{enumerate} \itemsep -2pt
	\item For example, at the University of Southern California, the organizations associated with feminist causes are: \vspace{-0.2cm}
		\begin{enumerate} \itemsep -2pt
		\item {\it USC Center for Women \& Men}: \url{http://www.usc.edu/student-affairs/cwm/links.html}
		\item {\it USC Women's Student Assembly}: \url{http://www-scf.usc.edu/~wsausc/Welcome.html}
		\end{enumerate}
	\end{enumerate}
\item International Women's Day: \url{http://www.internationalwomensday.com/}
\item Gender Across Borders: \vspace{-0.3cm}
	\begin{enumerate} \itemsep -2pt
	\item Feminism Resources: \url{http://www.genderacrossborders.com/feminist-resources/}
	\end{enumerate}
\item {\it V-Day}: \vspace{-0.3cm}
	\begin{enumerate} \itemsep -2pt
	\item \url{http://www.vday.org/}
	\item Organization that helps women plan and organize events to bring awareness about sexual assault, and what we can do to reduce sexual assault.
	\end{enumerate}
\item {\it Take Back The Night}: \vspace{-0.3cm}
	\begin{enumerate} \itemsep -2pt
	\item \url{http://www.takebackthenight.org/}
	\item Organization that helps women plan and organize events to bring awareness about sexual assault, and what we can do to reduce sexual assault. It also encourages sexual assault survivors to speak out about their sexual assaults, so that they would shame their perpetrators and let other women (and men) know that they is nothing to be ashamed of as sexual assault survivors. This is because the faults lie 100\% with the perpetrators, and not with the survivors.
	\end{enumerate}
\item {\it United Nations Development Fund for Women} (UNIFEM): \vspace{-0.3cm}
	\begin{enumerate} \itemsep -2pt
	\item \url{http://www.unifem.org/}
	\item Organization that addresses many challenges faced by girls and women.
	\end{enumerate}
\item {\it National Organization for Women}: \vspace{-0.3cm}
	\begin{enumerate} \itemsep -2pt
	\item \url{http://www.now.org/}
	\item Feminist organization in the US.
	\end{enumerate}
\item {\it A Woman's Nation}: \vspace{-0.3cm}
	\begin{enumerate} \itemsep -2pt
	\item \url{http://www.shriverreport.com/awn/}
	\item \url{http://awomansnation.com} or \url{http://www.shriverreport.com/}
	\end{enumerate}
\item {\it Peace Over Violence} is a non-profit, feminist, multicultural, volunteer organization dedicated to a building healthy relationships, families and communities free from sexual, domestic and interpersonal violence: \url{http://peaceoverviolence.org/}
\item SoulSpeakOut: \url{http://www.soulspeakout.org/resources/}
\item {\it Haven Hills}: \url{http://havenhills.org/}
%\item MaleSurvivor: \url{http://www.malesurvivor.org/}
\end{enumerate}












%%%%%%%%%%%%%%%%%%%%%%%%%%%%%%%%%%%%%%%%%%%
\section{Outreach: Professional Organizations}
\label{outreachproorgs}

Professional organizations: \vspace{-0.3cm}
\begin{enumerate} \itemsep -4pt
\item --- --- --- --- --- --- --- --- --- --- --- --- --- --- --- --- --- --- --- --- --- --- --- --- --- --- --- --- --- --- ---
\item \colorbox{blue}{\bf Professional Organizations for the Performance, Literary, and Visual Arts}
% Professional Organizations for the Performance, Literary, and Visual Arts
\item Americans for the Arts: \vspace{-0.3cm}
	\begin{enumerate} \itemsep -2pt
	\item \url{http://www.americansforthearts.org/get_involved/membership/default.asp}
	\item \url{http://www.artsusa.org/get_involved/membership/default.asp}
	\item Provides membership for organizations and individuals
	\item Individual membership are available for: \vspace{-0.2cm}
		\begin{enumerate} \itemsep -2pt
		\item Students
		\item Entrepreneurs (e.g., people in art management)
		\item Innovators
		\item Colleagues (artists)
		\end{enumerate}
	\item Americans for the Arts {\bf Emerging Leader Program}: \vspace{-0.2cm}
		\begin{enumerate} \itemsep -2pt
		\item \url{http://www.artsusa.org/networks/emerging_leaders/resources/default.asp}
		\item Has various resources for professional development, including mentoring
		\end{enumerate}
	\item Advocacy ({\bf public policy}): \url{http://www.artsusa.org/get_involved/advocate.asp}
	\end{enumerate}
\item --- --- --- --- --- --- --- --- --- --- --- --- --- --- --- --- --- --- --- --- --- --- --- --- --- --- --- --- --- --- ---
\item \colorbox{blue}{\bf Professional Organizations for the Musical Artists}
% Professional Organizations for the Musical Artists
\item The Recording Academy: \url{http://www.grammy365.com/join/membership-types}
\end{enumerate}













%%%%%%%%%%%%%%%%%%%%%%%%%%%%%%%%%%%%%%%%%%%
\section{Other Outreach}
\label{otheroutreach}

Other outreach: \vspace{-0.3cm}
\begin{enumerate} \itemsep -4pt
\item The Joy McCann Foundation: \vspace{-0.3cm}
	\begin{enumerate} \itemsep -2pt
	\item The Joy McCann Professorships in Law: \url{http://www.mccannfoundation.org/law.htm}
	\end{enumerate}
\item National Academy of Sciences: \vspace{-0.3cm}
	\begin{enumerate} \itemsep -2pt
	\item {\it Science \& Entertainment Exchange} program: \vspace{-0.2cm}
		\begin{enumerate} \itemsep -2pt
		\item \url{http://www.scienceandentertainmentexchange.org/}
		\item Provide science and engineering knowledge to help professionals in the entertainment industry create engaging storylines involving science and technology.
		\end{enumerate}
	\end{enumerate}
\item U.S. Department of State: \vspace{-0.3cm}
	\begin{enumerate} \itemsep -2pt
	\item Bureau of Educational and Cultural Affairs: \vspace{-0.2cm}
		\begin{enumerate} \itemsep -2pt
		\item Programs: \url{http://exchanges.state.gov/jexchanges/programs.html}
		\item Fulbright Classroom Teacher Exchange Program: \vspace{-0.1cm}
			\begin{enumerate} \itemsep -1pt
			\item \url{http://exchanges.state.gov/globalexchanges/fulbright-teacher-exchange-program.html}
			\item ``The Fulbright Classroom Teacher Exchange provides opportunities for primary and secondary teachers to exchange positions with colleagues in other countries. The participants contribute to mutual understanding by bringing international knowledge and perspectives to the U.S. and foreign classrooms, schools and communities. Full-time U.S. teachers can take part in either a year-long or semester-long direct exchange with a counterpart in another country.''
			\end{enumerate}
		\item FORTUNE/U.S. State Department Global Women's Mentoring Partnership: \vspace{-0.1cm}
			\begin{enumerate} \itemsep -1pt
			\item \url{http://exchanges.state.gov/citizens/professionals/fortunepartnership.html}
			\item ``This public-private partnership places talented, emerging women leaders from all over the world in mentoring programs with FORTUNE's Most Powerful Women Leaders.''
			\end{enumerate}
		\item Edward R. Murrow Program for Journalists: \vspace{-0.1cm}
			\begin{enumerate} \itemsep -1pt
			\item \url{http://exchanges.state.gov/ivlp/murrow.html}
			\item ``The Edward R. Murrow Program for Journalists invites rising international journalists to travel to the United States and examine journalistic principles and practices.''
			\end{enumerate}
		\item International Visitor Leadership Program: \vspace{-0.1cm}
			\begin{enumerate} \itemsep -1pt
			\item \url{http://exchanges.state.gov/ivlp/ivlp.html}
			\item ``These visits reflect the International Visitors' professional interests and support the foreign policy goals of the United States.''
			\item ``International Visitors are current or emerging leaders in government, politics, the media, education, the arts, business and other key fields.''
			\item ``International Visitors travel to the U.S. for carefully designed programs that reflect their professional interests and U.S. foreign policy goals. They travel in a variety of thematic programs, either individually or in groups, for up to three weeks. While in the U.S., International Visitors typically visit Washington, DC and three additional towns or cities that highlight the tremendous diversity of the U.S. They attend professional appointments with their American counterparts, learn about the U.S. system of government at the national, state and local levels, visit American schools, and experience American culture and social life.''
			\item ``There is no application for this program. International Visitors are selected and nominated annually by American Foreign Service Officers at U.S. Embassies around the world.''
			\end{enumerate}
		\item Au Pair: \vspace{-0.1cm}
			\begin{enumerate} \itemsep -1pt
			\item \url{http://exchanges.state.gov/jexchanges/programs/aupair.html}
			\item ``Through the Au Pair program, foreign nationals between 18 and 26 years of age participate in the home life of a host family. Au pairs provide limited childcare services for up to 12 months. An extension of 6, 9, or 12 months may be granted in certain cases.''
			\end{enumerate}
		\item Summer Work Travel: \vspace{-0.1cm}
			\begin{enumerate} \itemsep -1pt
			\item \url{http://exchanges.state.gov/jexchanges/programs/swt.html}
			\item ``In the summer work travel program, post-secondary students may enter the United States to work and travel during their summer vacation. Participants can be admitted to the program more than once. The maximum length of the program is four months.''
			\end{enumerate}
		\item Internship: \vspace{-0.1cm}
			\begin{enumerate} \itemsep -1pt
			\item \url{http://exchanges.state.gov/jexchanges/programs/intern.html}
			\item ``Internship programs are designed to allow foreign professionals to come to the United States to gain exposure to U.S. culture and to receive training in U.S. business practices in their chosen occupational field.  The maximum duration of an internship in any occupational field is 12 months. Upon completion of their exchange programs, participants are expected to return to their home countries.''
			\end{enumerate}
		\item Professional Exchanges Division: \vspace{-0.1cm}
			\begin{enumerate} \itemsep -1pt
			\item \url{http://exchanges.state.gov/citizens/profs.html}
			\item ``The Professional Exchanges division provides grants to U.S. nonprofit organizations to carry out exchange programs that support the professional development of foreign participants. The purpose of each exchange program is to engage with foreign leaders in critical professions, to demonstrate respect for foreign cultures, and to promote mutual understanding between the people of the United States and other countries.''
			\item ``Professional exchanges typically last several years and include internships, study tours or workshops in the United States and in the host country. Participants come from a variety of professions including education administrators, public servants, journalists, labor union officials, entrepreneurs, environmental leaders, jurists, lawyers, and civic leaders.''
			\end{enumerate}
		\end{enumerate}
	\end{enumerate}
\item Teach For All: \url{http://teachforallnetwork.org/}
\item --- --- --- --- --- --- --- --- --- --- --- --- --- --- --- --- --- --- --- --- --- --- --- --- --- --- --- --- --- --- ---
\item \colorbox{blue}{\bf Resources for Artists and Musicians}
% Resources for Artists and Musicians
\item League of American Orchestras and the Association of Performing Arts Presenters: \vspace{-0.3cm}
	\begin{enumerate} \itemsep -2pt
	\item {\it ArtistsfromAbroad.org}: \vspace{-0.2cm}
		\begin{enumerate} \itemsep -2pt
		\item \url{http://www.artistsfromabroad.org/}
		\item ``{\it ArtistsfromAbroad.org} features complete and up-to-date guidance on the visa process and tax treatment for foreign guest artists.''
		\end{enumerate}
	\end{enumerate}
\item Young Concert Artists, Inc. \vspace{-0.3cm}
	\begin{enumerate} \itemsep -2pt
	\item Composer Program (for American composers between 20 and 26 years of age): \url{http://www.yca.org/auditions/}
	\end{enumerate}
\item The John F. Kennedy Center for the Performing Arts: \vspace{-0.3cm}
	\begin{enumerate} \itemsep -2pt
	\item Mary Lou Williams Women in Jazz Emerging Artist Workshop: \vspace{-0.2cm}
		\begin{enumerate} \itemsep -2pt
		\item \url{http://www.kennedy-center.org/programs/jazz/womeninjazz/competition.html}
		\item ``The workshop provides female jazz artists ages 18 to 35 with an opportunity to explore and develop their artistry under the guidance of leading jazz artists and instructors. Each year, the workshop will focus on a specific instrument.''
		\item ``The 2011 Mary Lou Williams Women in Jazz Emerging Artist Workshop is open to advanced female jazz pianists who plan to pursue jazz performance as a career. Eligibility is exclusive to pianists who will be 18-35 years old on May 18, 2011 and have never recorded or been contracted to record as a leader or co-leader on a major label at the time of application. All applicants must be proficient in English.''
		\end{enumerate}
	\end{enumerate}
\item Grantmakers in the Arts (GIA): \vspace{-0.3cm}
	\begin{enumerate} \itemsep -2pt
	\item ``The mission of Grantmakers in the Arts (GIA) is to provide leadership and service to advance the use of philanthropic resources on behalf of arts and culture.''
	\item Arts Funding Topics: \url{http://www.giarts.org/arts-funding-topics}
	\end{enumerate}
\item The Dana Foundation: \vspace{-0.3cm}
	\begin{enumerate} \itemsep -2pt
	\item Arts Education program: \vspace{-0.2cm}
		\begin{enumerate} \itemsep -2pt
		\item Arts Education Grants: \vspace{-0.1cm}
			\begin{enumerate} \itemsep -1pt
			\item \url{http://www.dana.org/grants/BrowseArtsGrants.aspx}
			\item ``In 2001, The Dana Foundation created the Arts Education program with a sole focus of providing grants to support professional development for teaching artists and in-school arts specialists. The first several years of grants were to  programs in New York City, Washington, DC, Los Angeles and to organizations with a 50 mile radius of the three.''
			\item ``The Rural Initiative launched in 2006 with 6 grants awarded to organizations providing professional development in rural areas of the United States.''
			\end{enumerate}
		\end{enumerate}
	\end{enumerate}
\item writing/poetry contests: \vspace{-0.3cm}
	\begin{enumerate} \itemsep -2pt
	\item International 3-Day Novel Contest: \url{http://www.3daynovel.com/about/?contest}
	\end{enumerate}
\end{enumerate}




%%%%%%%%%%%%%%%%%%%%%%%%%%%%%%%%%%%%%%%%%%%
\section{Christian Colleges and Universities}
\label{christianunis}

Christian colleges and universities: \vspace{-0.3cm}
\begin{enumerate} \itemsep -4pt
\item List of Christian colleges and universities: \vspace{-0.3cm}
	\begin{enumerate} \itemsep -2pt
	\item Council for Christian Colleges and Universities (CCCU): \vspace{-0.2cm}
		\begin{enumerate} \itemsep -2pt
		\item \url{http://en.wikipedia.org/wiki/Council_for_Christian_Colleges_and_Universities}
		\item \url{http://www.cccu.org/}
		\end{enumerate}
	\item Christian College Consortium: \vspace{-0.2cm}
		\begin{enumerate} \itemsep -2pt
		\item \url{http://en.wikipedia.org/wiki/Christian_College_Consortium}
		\item \url{http://www.ccconsortium.org/}
		\end{enumerate}
	\end{enumerate}
\item California Baptist University, Riverside
\item Messiah College (Grantham, PA): \vspace{-0.3cm}
	\begin{enumerate} \itemsep -2pt
	\item Department of Engineering: \vspace{-0.2cm}
		\begin{enumerate} \itemsep -2pt
		\item \url{http://www.messiah.edu/departments/engineering/}
		\item B.S. programs in: \vspace{-0.1cm}
			\begin{enumerate} \itemsep -1pt
			\item Biomedical Engineering
			\item Computer Engineering
			\item Electrical Engineering
			\end{enumerate}
		\end{enumerate}
	\item Department of Information and Mathematical Sciences: \vspace{-0.2cm}
		\begin{enumerate} \itemsep -2pt
		\item \url{http://www.messiah.edu/departments/mathsci/index.html}
		\item Offers a B.A. Computer Science program
		\end{enumerate}
	\end{enumerate}
\end{enumerate}
























%%%%%%%%%%%%%%%%%%%%%%%%%%%%%%%%%%%%%%%%%
% Thoughts and Resources for Specific Areas and Topics
%	% This is written by Zhiyang Ong for his management of information and tasks.
%
% It includes information on professional development, including membership of professional organizations and networking societies.





%%%%%%%%%%%%%%%%%%%%%%%%%%%%%%%%%%%%%%%%%%%
\section{Heuristic for Locating Outreach Resources}
\label{heuristiclocateoutreach}

\proc{Find}$(\varphi, \tau)$ is a heuristic for locating resources for outreach activities, which includes finding information about the following: \vspace{-0.3cm}
\begin{enumerate} \itemsep -4pt
\item awards
\item career resources (including material for career guidance)
\item competitions and contests
\item educational material (e.g., suggested activities and curricular) for specific areas, such as marine sciences and electrical/computer engineering
\item fellowships
\item internships
\item scholarships
\item summer camps
\item summer programs (or summer schools); here, summer schools refer to short educational programs that last from days (e.g., a weekend for the ACM SIGDA Design Automation Summer School) to about a month (e.g., Santa Fe Institute's Complex Systems Summer Schools)
\end{enumerate}
\ \\

Its input $\tau$ is the deadline by which this search process must terminate. For example, if I have to apply for internships by next week, I would use the date of a week from now as the deadline $\tau$. In line \ref{find-pt-professional-org}, an example of a professional organization is the Institute of Electrical and Electronics Engineers (IEEE). The term ``good'' that is used in line \ref{find-pt-gd-uni} is an arbitrary measure of quality determined by the reader/user. \\

A reading group (in line \ref{find-pt-reading-grp}) is a small group of (graduate) students, which may possibly include professors and postdocs, that meet regularly (e.g., once/twice a week) to discuss papers that they have read since the previous meeting/discussion. Each individual in the reading group can be assigned a paper to read and present at the next meeting. The aim of a reading group is to improve the coverage of papers in our research area that each member has read. This is important for interdisciplinary research, since grad students working in interdisciplinary research areas have so much ground to cover. \\

Line \ref{find-pt-athletics} uses the term ``athletics department'' to refer to an administrative department at an American college or university that is in charge of managing varsity/NCAA sports teams. An example of a profession-specific networking organization (line \ref{find-pt-netwk-org}) is DVClub. In line \ref{find-pt-domain-specific-www}, a domain-specific web page is {\it SAT Live!}. An example of a corporate research laboratory (line \ref{find-pt-corporate-research-labs}) is ``Cadence Research Laboratories'' (\url{http://www.cadence.com/cadence/cadence_labs/pages/default.aspx}), and an example of a research institute (line \ref{find-pt-research-institute}) is Santa Fe Institute.




\begin{codebox}
\Procname{$\proc{Find}(\varphi, \tau)$}
\zi	\Comment {\it Input $\varphi \gets $ Item to find out about}
\zi	\Comment {\it Input $\tau \gets $ Deadline for the search process}
\zi	\Comment {\it Output $\kappa \gets $ List of resources about $\varphi$}
\zi
\li \While ( [ resources about $\varphi$ are inadequate ] AND [ $\tau$ has not yet passed ] )
	\Do
\li	Find out the professional organizations for the field of $\varphi$	\label{find-pt-professional-org}
\li	\For each professional organization in the field
		\Do
\li		Check if it has information about $\varphi$ in its web pages, publications, or mailing list archive
\li		\If (it has information about $\varphi$)
			\Then
\li			Add that information to $\kappa$
			\End
		\End
\zi
\li	\For each good (college OR university)	\label{find-pt-gd-uni}
		\Do
\li		\If ($\varphi == $ summer programs )
			\Then
\li			Search for summer programs in the web pages of departments \& schools/colleges
\li		\ElseIf ($\varphi == $ summer camps )
			\Then
\li			Search for summer camps in the web pages of departments \& schools/colleges
\li			Search for summer camps in the web pages of administrative/athletics departments	\label{find-pt-athletics}
\li		\ElseNoIf
\li			Search for $\varphi$ in the web pages of the department(s), including its news section/archive
\li			Search for $\varphi$ in the web pages of professors, postdoctoral researchers, \& students
\li			Search for $\varphi$ in the web pages of reading groups		\label{find-pt-reading-grp}
\li			Search for $\varphi$ in the web pages of student organizations
\li			Search for $\varphi$ in the mailing list archive of classes \& the department
\li			Search for $\varphi$ in the mailing list archive of research groups/labs and projects
\li			Search for $\varphi$ in the mailing list archive of reading groups
\li			Search for $\varphi$ in the mailing list archive of student organizations
			\End
\zi
\li		\If (it has information about $\varphi$)
			\Then
\li			Add that information to $\kappa$
			\End
		\End
\zi
\li	Search for $\varphi$ in the mailing list archive of open-source projects
\li	Search for $\varphi$ in the mailing list archive of profession-specific networking organizations	\label{find-pt-netwk-org}
\li	Search for $\varphi$ in the web pages of domain-specific web pages	\label{find-pt-domain-specific-www}
\li	Search for $\varphi$ in the web pages of research scientists in corporate research labs	\label{find-pt-corporate-research-labs}
\li	Search for $\varphi$ in the web pages of research scientists in research institutes	\label{find-pt-research-institute}
\li	\If ( [ mailing list archive OR web page ] has information about $\varphi$)
		\Then
\li		Add that information to $\kappa$
		\End
	\End	
\li \Return $\kappa$
\end{codebox}




%%%%%%%%%%%%%%%%%%%%%%%%%%%%%%%%%%%%%%%%%%%
\section{General Outreach Resources}
\label{generaloutreachresources}

General outreach resources: \vspace{-0.3cm}
\begin{enumerate} \itemsep -4pt
\item volunteering opportunities: \vspace{-0.3cm}
	\begin{enumerate} \itemsep -2pt
	\item Engineers Without Borders: \url{http://www.ewb-international.org/}
	\item Australian Volunteers International: \url{http://www.australianvolunteers.com/}
	\item Youth Challenge Australia: \url{http://www.youthchallenge.com.au/}
	\item Go Volunteer: \url{http://www.govolunteer.com.au/}
	\item Volunteer Search: \url{http://www.volunteersearch.gov.au/}
	\item Conservation Volunteers: \url{http://www.conservationvolunteers.com.au/volunteer}
	\item Volunteering Australia: \url{http://www.volunteeringaustralia.org/html/s01_home/home.asp}
	\item Sponsors for Educational Opportunity (SEO): \vspace{-0.2cm}
		\begin{enumerate} \itemsep -2pt
		\item Philanthropy \& Volunteerism Resources, \url{http://www.seo-usa.org/AlumniResources}
		\item Volunteer Leadership Opportunities: \url{http://www.seo-usa.org/Alumni_Volunteer}
		\end{enumerate}
	\item : \url{}
	\end{enumerate}
\item public health and preventive medicine: \vspace{-0.3cm}
	\begin{enumerate} \itemsep -2pt
	\item U.S. Department of Health \& Human Services: \vspace{-0.2cm}
		\begin{enumerate} \itemsep -2pt
		\item Agency for Healthcare Research and Quality (AHRQ): \vspace{-0.1cm}
			\begin{enumerate} \itemsep -1pt
			\item Prevention \& Care Management: Resources and Materials, \url{http://www.ahrq.gov/clinic/ppipix.htm}
			\end{enumerate}
		\end{enumerate}
	\end{enumerate}
\item career resources: \vspace{-0.3cm}
	\begin{enumerate} \itemsep -2pt
	\item CRAC: The Career Development Organisation: \vspace{-0.2cm}
		\begin{enumerate} \itemsep -2pt
		\item {\it icould}: \vspace{-0.1cm}
			\begin{enumerate} \itemsep -1pt
			\item \url{http://icould.com/about/}
			\item Resource for students, people who are commencing their careers or are making changes in their careers, career counselors, parents, educators, human resource staff, and employers.
			\item icould, {\it Stories by Life Theme}, in icould: Watch Career Stories. Available online at: \url{http://icould.com/watch-career-stories/by-life-theme/}; last accessed on December 25, 2010. [ Has articles briefly describing how people pursued their career goals or their career paths as they went through different experiences in life. This includes people who ``blossomed after school,'' changed careers or became entrepreneurs, had no plans, took risks, encountered turning points, faced adversity, have disabilities, went through financial hardship, or got laid off. It also has stories of people who volunteered, took a gap year, or pursued internships. ]
			\item icould, {\it Stories by Job Type}, in icould: Watch Career Stories. Available online at: \url{http://icould.com/watch-career-stories/by-job-type/}; last accessed on December 25, 2010. [ Includes stories of people in automotive retail, customer services, engineering, education, and many other job types. ]
			\end{enumerate}
		\end{enumerate}
	\item Jobs for the Future: \vspace{-0.2cm}
		\begin{enumerate} \itemsep -2pt
		\item \url{http://www.jff.org/}
		\item Current Projects: \url{http://www.jff.org/projects/current}
		\item Publications: \url{http://www.jff.org/publications}
		\item Policy: \url{http://www.jff.org/policy}
		\item Funders (funding agencies/organizations): \url{http://www.jff.org/funders}
		\item Programs: \url{http://www.jff.org/index.php?select=work}
		\end{enumerate}
	\item SkillsUSA: \vspace{-0.2cm}
		\begin{enumerate} \itemsep -2pt
		\item ``SkillsUSA is a partnership of students, teachers and industry working together to ensure America has a skilled work force. SkillsUSA helps each student excel.''
		\item Educators: \vspace{-0.1cm}
			\begin{enumerate} \itemsep -1pt
			\item \url{http://www.skillsusa.org/educators/index.shtml}
			\item Programs and Curricula: \url{http://www.skillsusa.org/educators/programs.shtml}
			\end{enumerate}
		\item Students: \vspace{-0.1cm}
			\begin{enumerate} \itemsep -1pt
			\item \url{http://www.skillsusa.org/students/index.shtml}
			\item Scholarships \& Financial Aid--SkillsUSA-related Scholarships: \url{http://www.skillsusa.org/students/scholarships.shtml}
			\end{enumerate}
		\item SkillsUSA competitions: \url{http://www.skillsusa.org/compete/index.shtml}
		\end{enumerate}
	\item others: \vspace{-0.2cm}
		\begin{enumerate} \itemsep -2pt
		\item public speaking and leadership: \vspace{-0.1cm}
			\begin{enumerate} \itemsep -1pt
			\item {\it Toastmasters International} is a non-profit educational organization that teaches public speaking and leadership skills through a worldwide network of meeting locations. Available online at: \url{http://www.toastmasters.org/}; last accessed on January 7, 2010.
			\end{enumerate}
		\end{enumerate}
	\end{enumerate}
\end{enumerate}




%%%%%%%%%%%%%%%%%%%%%%%%%%%%%%%%%%%%%%%%%%%
\section{Youth Outreach}
\label{youthoutreach}

Resources for youth outreach: \vspace{-0.3cm}
\begin{enumerate} \itemsep -4pt
%%%%%%%%%%%%%%%%%%%%%%%
\item educational (computer) games: \vspace{-0.3cm}
	\begin{enumerate} \itemsep -2pt
	\item Chevron Corporation: \vspace{-0.2cm}
		\begin{enumerate} \itemsep -2pt
		\item Energyville (about issues concerning energy and the environment): \url{http://www.willyoujoinus.com/energyville/}
		\end{enumerate}
	\item {\it Lego Digital Designer (LDD)}: \vspace{-0.2cm}
		\begin{enumerate} \itemsep -2pt
		\item CAD software for building Lego toys on Windows and Mac OS X platforms
		\item Free software, as in free beer
		\item \url{http://designbyme.lego.com/en-us/Default.aspx} and \url{http://ldd.lego.com/}
		\end{enumerate}
	\item Robocode: \vspace{-0.2cm}
		\begin{enumerate} \itemsep -2pt
		\item \url{http://en.wikipedia.org/wiki/Robocode} and \url{http://robocode.sourceforge.net/}
		\item Learn how to develop computer programs that will control a robot
		\end{enumerate}
	\item {\it Skill-Life}: \vspace{-0.2cm}
		\begin{enumerate} \itemsep -2pt
		\item \url{http://skill-life.com/}
		\item Use online games to teach youth life skills concerning financial literacy, nutrition, and citizenship.
		\end{enumerate}
	\item PowerUp (IBM with TryScience/New York Hall of Science): \vspace{-0.2cm}
		\begin{enumerate} \itemsep -2pt
		\item \url{http://www.powerupthegame.org/}
		\item Computer game to teach youths about energy conservation, global warming, renewable energy, and sustainable engineering
		\end{enumerate}
	\item EnergyNet: \vspace{-0.2cm}
		\begin{enumerate} \itemsep -2pt
		\item \url{http://www.energynet.net/games/}
		\item Computer game to teach youths about energy efficiency, and other topics related to energy
		\end{enumerate}
	\end{enumerate}
%%%%%%%%%%%%%%%%%%%%%%%
\item summer camps: \vspace{-0.3cm}
	\begin{enumerate} \itemsep -2pt
	\item United States Naval Academy: \vspace{-0.2cm}
		\begin{enumerate} \itemsep -2pt
		\item Naval Academy Athletic Association: \vspace{-0.1cm}
			\begin{enumerate} \itemsep -1pt
			\item Sports camps: \url{http://www.navysports.com/camps/navy-camps.html}
			\end{enumerate}
		\end{enumerate}
	\end{enumerate}
%%%%%%%%%%%%%%%%%%%%%%%
\item competitions for youths: \vspace{-0.3cm}
	\begin{enumerate} \itemsep -2pt
	\item International Geography Olympiad (for high school students): \url{http://www.geoolympiad.org/}
	\item International Linguistic Olympiad (for high school students): \url{http://en.wikipedia.org/wiki/International_Linguistics_Olympiad}
	\item International Philosophy Olympiad (for high school students): \url{http://www.philosophy-olympiad.org/}
	\item JA Worldwide: Responsible People Business Competition (for students in North and South America, and Europe), \url{http://www.responsible-business.org/}
	\item The Choral Arts Society of Washington: \vspace{-0.2cm}
		\begin{enumerate} \itemsep -2pt
		\item \url{http://www.choralarts.org/MLK-Celebration-Community-Initiative/Writing-Competition.aspx}
		\item ``As part of our MLK Celebration Community Initiative and in celebration of Black History Month, The Choral Arts Society of Washington hosts an annual writing competition for students in grades K-12.''
		\item ``Each year, students are presented with a different writing prompt and are asked to respond in poetic form.''
		\item ``Students are encouraged to be creative in their writing and to use their knowledge of Martin Luther King, Jr.'s life, the Civil Rights Movement, and current events as inspiration for their writing.''
		\end{enumerate}
	\item Vocal Arts DC (or Vocal Arts Society): \vspace{-0.2cm}
		\begin{enumerate} \itemsep -2pt
		\item Young Artists Competition: \vspace{-0.1cm}
			\begin{enumerate} \itemsep -1pt
			\item \url{http://vocalartsdc.org/youngartists.shtml}
			\item ``Each year, Vocal Arts DC holds a vocal competition open to all singers who are residents of the greater DC area, including Baltimore and Annapolis.''
			\item ``Singers are asked to submit a CD for review along with a sample recital program that the singer is prepared to sing in recital. The CDs will be reviewed in a blind audition and finalist will be selected for live auditions.''
			\item ``Two winners are selected from the finalists and are presented in the Art Song Discovery Series in four different venues across the greater DC area.''
			\end{enumerate}
		\end{enumerate}
	\item The John F. Kennedy Center for the Performing Arts: \vspace{-0.2cm}
		\begin{enumerate} \itemsep -2pt
		\item The National Symphony Orchestra (NSO): \vspace{-0.1cm}
			\begin{enumerate} \itemsep -1pt
			\item Young Soloists' Competition (High School Division; Washington metropolitan area): \url{http://www.kennedy-center.org/nso/nsoed/youngsoloists.cfm#concerts}
			\end{enumerate}
		\end{enumerate}
	\item Center for Interactive Learning and Collaboration (CILC): \vspace{-0.2cm}
		\begin{enumerate} \itemsep -2pt
		\item Kids Creating Community Content KC$^{3}$ International Contest (for students in Middle and High School): \vspace{-0.1cm}
			\begin{enumerate} \itemsep -1pt
			\item \url{http://kc3.cilc.org/} and \url{http://kc3.cilc.org/guidelines.htm}
			\item Make a short film to educate others about the uniqueness of your community, geographical region, natural/agricultural resources, local/national treasures, culture/heritage, or country.
			\end{enumerate}
		\end{enumerate}
	\end{enumerate}
%%%%%%%%%%%%%%%%%%%%%%%
\item educational resources: \vspace{-0.3cm}
	\begin{itemize} \itemsep -2pt
	\item Xcel Energy Foundation: \vspace{-0.2cm}
		\begin{enumerate} \itemsep -2pt
		\item Focus Area Grants: \vspace{-0.1cm}
			\begin{enumerate} \itemsep -1pt
			\item \url{http://www.xcelenergy.com/Minnesota/Company/Community/Xcel%20Energy%20Foundation/Pages/Focus_Area_Grants.aspx}
			\item Scope of eligible funding, and details on the grant application process
			\end{enumerate}
		\item Education Initiatives: \vspace{-0.1cm}
			\begin{enumerate} \itemsep -1pt
			\item \url{http://www.xcelenergy.com/Minnesota/Company/Community/Education%20Initiatives/Pages/Education_Initiatives.aspx}
			\item Energy Safety Calendar Program, K-6: \vspace{-0.1cm}
				\begin{itemize} \itemsep -1pt
				\item \url{http://www.xcelenergy.com/New%20Mexico/Company/Community/Education%20Initiatives/Pages/Energy_Safety_Calendar_ProgramK-6.aspx}
				\item ``The Energy Safety Calendar Program offers K-6 students in our service territory a great opportunity to learn about electricity and natural gas safety.''
				\end{itemize}
			\end{enumerate}
		\item Safety World: \vspace{-0.1cm}
			\begin{enumerate} \itemsep -1pt
			\item \url{http://www.xcelenergy.com/New%20Mexico/Company/Community/Education%20Initiatives/Pages/Safety_World.aspx}
			\item e-SMART kid: \vspace{-0.1cm}
				\begin{itemize} \itemsep -1pt
				\item \url{http://www.e-smartonline.net/xcelenergy/}
				\item Help children and youth learn about ``electricity and natural gas and how to use them safely''
				\end{itemize}
			\end{enumerate}
		\item Energy Classroom: \vspace{-0.1cm}
			\begin{enumerate} \itemsep -1pt
			\item \url{http://www.energyclassroom.com/}
			\item \url{http://www.xcelenergy.com/Minnesota/Company/Community/Pages/Energy_Classroom.aspx}
			\item Educational material for students about energy sources, energy conservation, and environmental protection
			\item For Teachers (educational material and suggested class activities): \url{http://www.energyclassroom.com/index.php?id=34&page=For_Teachers}
			\end{enumerate}
		\item Power Plant Tour Information: \url{http://www.xcelenergy.com/New%20Mexico/Company/About_Energy_and_Rates/Power%20Generation/Pages/Power_Plant_Tour_Information.aspx}
		\end{enumerate}
	\item HowStuffWorks, Inc.: \url{http://www.howstuffworks.com/}
	\item Chevron Corporation: \vspace{-0.2cm}
		\begin{enumerate} \itemsep -2pt
		\item {\it Will you join us}: \vspace{-0.1cm}
			\begin{enumerate} \itemsep -1pt
			\item Energy issues: \url{http://www.willyoujoinus.com/energyissues/}
			\item Tools and resources: \vspace{-0.1cm}
				\begin{itemize} \itemsep -1pt
				\item \url{http://www.willyoujoinus.com/toolsresources/}
				\item Helpful links (includes K-12 educational material): \url{http://www.willyoujoinus.com/toolsresources/helpfullinks/}
				\end{itemize}
			\item MPG Optimizer: \url{http://www.willyoujoinus.com/usingenergywisely/mpgoptimizer/}
			\item Energy generator: \url{http://www.willyoujoinus.com/usingenergywisely/energygenerator/}
			\end{enumerate}
		\end{enumerate}
	\item National Energy Foundation: \vspace{-0.2cm}
		\begin{enumerate} \itemsep -2pt
		\item \url{http://www.nef.org.uk/} and \url{http://www.nef1.org/}
		\item Students: \url{http://www.nef1.org/students.html}
		\item Educators: \url{http://www.nef1.org/educators.html}
		\item Schools: \vspace{-0.1cm}
			\begin{enumerate} \itemsep -1pt
			\item \url{http://www.nef.org.uk/communities/schools/index.html}
			\item Helpful links: \url{http://www.nef.org.uk/communities/schools/energylinks.html}
			\item School Resources: \url{http://www.nef.org.uk/communities/schools/resources/index.html}
			\item {\it LogiCity} is a fun interactive computer game with a difference. It's a game set in a 3D virtual city with five main activities where you are set the task of reducing the carbon footprint of an average resident. See \url{http://www.nef.org.uk/communities/schools/logicity.html}.
			\end{enumerate}
		\item Resources: \url{http://www.nef.org.uk/actonCO2/index.asp}
		\item Igniting Creative Energy - A National Student Challenge: \vspace{-0.1cm}
			\begin{enumerate} \itemsep -1pt
			\item \url{http://www.ignitingcreativeenergy.org/}
			\item Students: \url{http://www.ignitingcreativeenergy.org/students.html}
			\end{enumerate}
		\end{enumerate}
	\item StartSpot Mediaworks: \vspace{-0.2cm}
		\begin{enumerate} \itemsep -2pt
		\item StartSpot Network: \vspace{-0.1cm}
			\begin{enumerate} \itemsep -1pt
			\item HomeworkSpot: \vspace{-0.1cm}
				\begin{itemize} \itemsep -1pt
				\item \url{http://www.homeworkspot.com/}
				\item Science Fair Project Center: \url{http://www.homeworkspot.com/sciencefair/}
				\end{itemize}
			\end{enumerate}
		\end{enumerate}
	\item Super Science Fair Projects: \url{http://www.super-science-fair-projects.com/}
	\item All Science Fair Projects: Science Fair Projects with Complete Instructions, \url{http://www.all-science-fair-projects.com/}
	\item The Science Club: \vspace{-0.2cm}
		\begin{enumerate} \itemsep -2pt
		\item \url{http://scienceclub.org/}
		\item Science Fair Idea Exchange: \url{http://scienceclub.org/scifair.html}
		\end{enumerate}
	\item Oracle Education Foundation: \vspace{-0.2cm}
		\begin{enumerate} \itemsep -2pt
		\item \url{http://www.oraclefoundation.org/}
		\item ThinkQuest: \vspace{-0.1cm}
			\begin{enumerate} \itemsep -1pt
			\item \url{http://www.thinkquest.org/en/}
			\item ThinkQuest International Competition: \url{http://www.thinkquest.org/competition/}
			\item Projects: \url{http://thinkquest.org/en/projects/index.html}
			\item Library: \url{http://thinkquest.org/pls/html/think.library}
			\item Example of a computer game developed by students: Crisis! - The Game, \url{http://library.thinkquest.org/20331/game/}
			\end{enumerate}
		\end{enumerate}
	\item University of Minnesota: \vspace{-0.2cm}
		\begin{enumerate} \itemsep -2pt
		\item Institute on Community Integration; College of Education and Human Development: \vspace{-0.1cm}
			\begin{enumerate} \itemsep -1pt
			\item National Center on Secondary Education and Transition (NCSET): \vspace{-0.1cm}
				\begin{itemize} \itemsep -1pt
				\item \url{http://www.ncset.org/}
				\item NCSET Topics: \url{http://www.ncset.org/topics/default.asp}
				\item Web Sites: \url{http://www.ncset.org/websites/default.asp}
				\item The Youthhood!: \url{http://www.youthhood.org/}
				\end{itemize}
			\end{enumerate}
		\end{enumerate}
	\item Jobs for America's Graduates: \vspace{-0.2cm}
		\begin{enumerate} \itemsep -2pt
		\item \url{http://www.jag.org/}
		\item JAG Model program applications: \vspace{-0.1cm}
			\begin{enumerate} \itemsep -1pt
			\item \url{http://www.jag.org/model.htm}
			\item Programs are available for students in middle school and high school, high school dropouts, high school seniors, students in alternative education programs, and college underclassmen
			\end{enumerate}
		\item JAG Career Corner: \url{http://www.jag.org/jag_career_corner.htm}
		\item Students: \url{http://www.jag.org/students.htm}
		\item Resource library: \url{http://www.jag.org/library.htm}
		\item Performance outcomes: \url{http://www.jag.org/outcomes.htm}
		\item Funding: \url{http://www.jag.org/funding.htm}
		\end{enumerate}
	\item Alliance to Save Energy: \vspace{-0.2cm}
		\begin{enumerate} \itemsep -2pt
		\item Energy Hog campaign: \vspace{-0.1cm}
			\begin{enumerate} \itemsep -1pt
			\item \url{http://www.energyhog.org/}
			\item Adults: \url{http://www.energyhog.org/adult/adults.htm}
			\item Children: \url{http://www.energyhog.org/childrens.htm}
			\end{enumerate}
		\end{enumerate}
	\item Learning First Alliance: \vspace{-0.2cm}
		\begin{enumerate} \itemsep -2pt
		\item \url{http://www.learningfirst.org/}
		\item Issues and publications: \url{http://www.learningfirst.org/issues}
		\item Resources: \url{http://www.learningfirst.org/resources}
		\end{enumerate}
	\item NaMaYa: \url{http://www.namaya.com/}
	\item NIXTY: \url{http://nixty.com/}
	\item K12 Open Ed: \url{http://www.k12opened.com/wiki/index.php/Main_Page}
	\item Learning Is For Everyone: \url{http://www.learningis4everyone.org/}
	\item The Smithsonian Commons Prototype: \url{http://www.si.edu/commons/prototype/}
	\item Futurelab: Resources for educators and parents, \url{http://www.futurelab.org.uk/resources}
	\item Innosight Institute: Resources for education, \url{http://www.innosightinstitute.org/practices/education/}
	\item WGBH Educational Foundation: \url{http://www.wgbh.org/}
	\item Discovery Education: \vspace{-0.2cm}
		\begin{enumerate} \itemsep -2pt
		\item Classroom resources: \url{http://school.discoveryeducation.com/}
		\item Home resources: \url{http://school.discoveryeducation.com/homeworkhelp/homework_help_home.html}
		\end{enumerate}
	\item The Gilder Lehrman Institute of American History: \vspace{-0.2cm}
		\begin{enumerate} \itemsep -2pt
		\item \url{http://www.gilderlehrman.org/}
		\item Resources for teachers and schools: \url{http://www.gilderlehrman.org/teachers/}
		\item Civil War Essay Contest (for students in selected K-12 schools): \url{http://www.gilderlehrman.org/affiliate/civil_war.php}
		\end{enumerate}
	\item The GRAMMY Museum: \vspace{-0.2cm}
		\begin{enumerate} \itemsep -2pt
		\item Teacher curriculum and resources. Available online at: \url{http://www.grammymuseum.org/interior.php?section=education&page=teachercurriculum}; last accessed on November 15, 2010.
		\end{enumerate}
	\item Purdue University: \vspace{-0.2cm}
		\begin{enumerate} \itemsep -2pt
		\item Department of Entomology: \vspace{-0.1cm}
			\begin{enumerate} \itemsep -1pt
			\item Genomics Analogy Model for Educators (G.A.M.E.): \url{http://www.entm.purdue.edu/extensiongenomics/GAME/default.html}
			\end{enumerate}
		\end{enumerate}
	\item Verizon Thinkfinity: \url{http://www.thinkfinity.org/about-us}
	\item Oregon Virtual School District (ORVSD): \vspace{-0.2cm}
		\begin{enumerate} \itemsep -2pt
		\item \url{http://orvsd.org/}
		\item ``Oregon Virtual School District (ORVSD) helps integrate technology into Oregon public school classrooms by giving teachers access to free tech tools and resources online.''
		\item ``The Oregon Virtual School District is a program led by the Oregon Department of Education that, in cooperation with a consortium of virtual learning providers throughout the state, seeks to increase access and availability of online learning and teaching resources free of charge to public school teachers of Oregon. Oregon State University is providing hosting and development resources through a partnership with the OSU Open Source Lab and the OSU Business Solutions Group.''
		\end{enumerate}
	\item The Association of Educational Publishers (AEP): \vspace{-0.2cm}
		\begin{enumerate} \itemsep -2pt
		\item The AEP Awards: \vspace{-0.1cm}
			\begin{enumerate} \itemsep -1pt
			\item \url{http://www.aepweb.org/awards/index.htm}
			\item Look at the winners of previous AEP awards to determine some of the good educational resources that are available
			\end{enumerate}
		\end{enumerate}
	\item Educational Dividends: \vspace{-0.2cm}
		\begin{enumerate} \itemsep -2pt
		\item \url{http://www.educationaldividends.com/}
		\item Teachers: \vspace{-0.1cm}
			\begin{enumerate} \itemsep -1pt
			\item \url{http://www.educationaldividends.com/index.asp?menu=Teachers}
			\item Teaching Tools: \url{http://www.educationaldividends.com/teachers/tools.asp}
			\item Reference Desk: \vspace{-0.1cm}
				\begin{itemize} \itemsep -1pt
				\item \url{http://www.educationaldividends.com/teachers/reference.asp}
				\item Standards Reference Desk (resources for education standards in the US at the national, state, and local levels): \url{http://www.educationaldividends.com/teachers/standards_desk.asp}
				\item How We Learn: Learning Styles, \url{http://www.educationaldividends.com/teachers/learning_styles.asp}
				\item How We Learn: Multiple Intelligences, \url{http://www.educationaldividends.com/teachers/multiple_intelligences.asp}
				\item Statistics Desk (statistical information about education in the US): \url{http://www.educationaldividends.com/teachers/statistics_desk.asp}
				\end{itemize}
			\item Information about the teaching profession: \vspace{-0.1cm}
				\begin{itemize} \itemsep -1pt
				\item \url{http://www.educationaldividends.com/teachers/welcome.asp}
				\item Office of Occupational Statistics and Employment Projections, ``Educational Services,'' in {\it Career Guide to Industries}, 2010-11 Edition, U.S. Bureau of Labor Statistics, U.S. Department of Labor, Washington, DC, December 17, 2009. Available online at: \url{http://stats.bls.gov/oco/cg/cgs034.htm}; last accessed on December 8, 2010. [ Suggested citation: Bureau of Labor Statistics, U.S. Department of Labor, {\it Career Guide to Industries, 2010-11 Edition}, Educational Services , on the Internet at \url{http://www.bls.gov/oco/cg/cgs034.htm} (visited December 07, 2010). ]
				\item Experience Teaching: \url{http://www.educationaldividends.com/teachers/experience.asp}
				\item Continuous Improvement: \url{http://www.educationaldividends.com/teachers/toolkit.asp}
				\end{itemize}
			\end{enumerate}
		\item Personality and Career Tests: \url{http://www.educationaldividends.com/teachers/tests.asp}
		\end{enumerate}
	\item Smithsonian Institution: \vspace{-0.2cm}
		\begin{enumerate} \itemsep -2pt
		\item Educators: \url{http://www.si.edu/Educators}
		\item Smithsonian Institution Traveling Exhibition Service (SITES): \vspace{-0.1cm}
			\begin{enumerate} \itemsep -1pt
			\item For Teachers: \url{http://www.sites.si.edu/education/teachers_res2.htm}
			\end{enumerate}
		\item Smithsonian Folkways Recordings (or simply, Smithsonian Folkways): \vspace{-0.1cm}
			\begin{enumerate} \itemsep -1pt
			\item Tools for Teaching: \url{http://www.folkways.si.edu/tools_for_teaching/introduction.aspx}
			\end{enumerate}
		\item Freer Gallery of Art / Arthur M. Sackler Gallery: \vspace{-0.1cm}
			\begin{enumerate} \itemsep -1pt
			\item Resources for Educators: \url{http://www.asia.si.edu/explore/teacherResources.asp}
			\item Explore + Learn: Browse Online Resources by Area: \vspace{-0.1cm}
				\begin{itemize} \itemsep -1pt
				\item \url{http://www.asia.si.edu/explore/default.asp}
				\item Has resources for art in: \vspace{-0.1cm}
					\begin{itemize} \itemsep -1pt
					\item The Americas
					\item Ancient Egypt
					\item Ancient Near East
					\item Islamic world
					\item China
					\item Japan
					\item Korea
					\item South Asia
					\item Himalayas
					\item Southeast Asian
					\item It also has biblical manuscripts and contemporary art
					\end{itemize}
				\end{itemize}
			\item Online Exhibition Features: \url{http://www.asia.si.edu/exhibitions/online.asp}
			\item Collections: \url{http://www.asia.si.edu/collections/default.asp}
			\end{enumerate}
		\item National Museum of American History: \vspace{-0.1cm}
			\begin{enumerate} \itemsep -1pt
			\item Jerome and Dorothy Lemelson Center for the Study of Invention and Innovation: \vspace{-0.1cm}
				\begin{itemize} \itemsep -1pt
				\item Resources: \vspace{-0.1cm}
					\begin{itemize} \itemsep -1pt
					\item \url{http://invention.smithsonian.org/resources/}
					\item \url{http://invention.smithsonian.org/resources/default_sites_weblinks.aspx}
					\item Invention stories - archives, articles, audio, and video: \url{http://invention.smithsonian.org/resources/default_index.aspx}
					\end{itemize}
				\item Educational Materials: \vspace{-0.1cm}
					\begin{itemize} \itemsep -1pt
					\item \url{http://invention.smithsonian.org/resources/menu_edu_materials.aspx}
					\item Experiments: \url{http://invention.smithsonian.org/resources/menu_edu_materials.aspx?MaterialTypeID=3&MaterialTypeDesc=Experiments}
					\item Educational Materials: \url{http://invention.smithsonian.org/resources/menu_edu_materials_f.aspx?MaterialTypeDesc=Features}
					\end{itemize}
				\item Centerpieces: \vspace{-0.1cm}
					\begin{itemize} \itemsep -1pt
					\item \url{http://invention.smithsonian.org/centerpieces/}
					\item \url{http://invention.smithsonian.org/centerpieces/iap-info.aspx}
					\item Electric guitar: \url{http://invention.smithsonian.org/centerpieces/electricguitar/index.htm}
					\item Innovative Lives: \url{http://invention.smithsonian.org/centerpieces/ilives/}
					\item ``Exploring the History of Women Inventors'' by J.E. Bedi (in {\it Innovative Lives}): \url{http://invention.smithsonian.org/centerpieces/ilives/womeninventors.html}
					\item Whole Cloth: \url{http://invention.smithsonian.org/centerpieces/whole_cloth/index.html}
					\item The Quartz Watch: \url{http://invention.smithsonian.org/centerpieces/quartz/index.html}
					\item Edison Invents!: All about Thomas Edison and his invention, \url{http://invention.smithsonian.org/centerpieces/edison/default.asp}
					\end{itemize}
				\item Modern Inventors Documentation Program (MIND): \url{http://invention.smithsonian.org/resources/mind_resources.aspx}
				\item Invention at Play: \vspace{-0.1cm}
					\begin{itemize} \itemsep -1pt
					\item \url{http://inventionatplay.org/}
					\item Resources: \url{http://inventionatplay.org/resources.html}
					\item Invention Playhouse: \url{http://inventionatplay.org/playhouse_main.html}
					\item Inventors' Stories: \url{http://inventionatplay.org/inventors_main.html}
					\item Does play matter? (using play to help children learn and think): \url{http://inventionatplay.org/matter_main.html}
					\end{itemize}
				\item Spark!Lab: \vspace{-0.1cm}
					\begin{itemize} \itemsep -1pt
					\item \url{http://sparklab.si.edu/}
					\item About Spark!Lab (introduce children to the process of innovation via play and fun activities): \url{http://sparklab.si.edu/spark-about.html}
					\item Activities \& Experiments: \url{http://sparklab.si.edu/spark-experiments.html}
					\item Inventor Profiles: \url{http://sparklab.si.edu/spark-inventors.html}
					\item Resources: \url{http://sparklab.si.edu/spark-resources.html}
					\end{itemize}
				\end{itemize}
			\end{enumerate}
		\end{enumerate}
	\item Economic and Social Research Council (ESRC): \vspace{-0.2cm}
		\begin{enumerate} \itemsep -2pt
		\item {\it Social Science for Schools}; Science in Society Team: \vspace{-0.1cm}
			\begin{enumerate} \itemsep -1pt
			\item \url{http://www.esrcsocietytoday.ac.uk/ESRCInfoCentre/ssfs/}
			\item Social science resources: \url{http://www.esrcsocietytoday.ac.uk/ESRCInfoCentre/ssfs/resources/}
			\item Career guides for different disciplines in social science and economics: \url{http://www.esrcsocietytoday.ac.uk/ESRCInfoCentre/ssfs/careers/}
			\item Related online resources: \url{http://www.esrcsocietytoday.ac.uk/ESRCInfoCentre/ssfs/links/}
			\end{enumerate}
		\end{enumerate}
	\end{itemize}
\item National Council for Accreditation of Teacher Education (NCATE): \vspace{-0.3cm}
	\begin{enumerate} \itemsep -2pt
	\item \url{http://www.ncate.org/}
	\item Has resources about degree programs in education and their accreditation, as well as how to become a teacher
	\item State-specific Recognized Programs by NCATE and Specialized Professional Associations (SPAs): \vspace{-0.2cm}
		\begin{enumerate} \itemsep -2pt
		\item \url{http://www.ncate.org/tabid/165/Default.aspx}
		\item Find out about educational programs in: \vspace{-0.1cm}
			\begin{enumerate} \itemsep -1pt
			\item special education
			\item early childhood education
			\item educational leadership
			\item educational technology specialist
			\item elementary education
			\item English
			\item health education
			\item foreign languages
			\item gifted education
			\item mathematics
			\item physical education
			\item science education
			\item school psychology
			\item secondary computer science education
			\item social studies
			\item Teachers of English to Speakers of Other Languages (TESOL)
			\item technology and engineering educators
			\end{enumerate}
		\end{enumerate}
	\item Financial Aid Resources for Teacher Education Students: \url{http://www.ncate.org/Public/CurrentFutureTeachers/FinancialAidResources/tabid/351/Default.aspx}
	\end{enumerate}
%%%%%%%%%%%%%%%%%%%%%%%
\item scholarships: \vspace{-0.3cm}
	\begin{enumerate} \itemsep -2pt
	\item U.S. Department of State: \vspace{-0.2cm}
		\begin{enumerate} \itemsep -2pt
		\item Bureau of Educational and Cultural Affairs: \vspace{-0.1cm}
			\begin{enumerate} \itemsep -1pt
			\item National Security Language Initiative for Youth (NSLI-Y): \vspace{-0.1cm}
				\begin{itemize} \itemsep -1pt
				\item \url{http://exchanges.state.gov/youth/programs/nsli.html}
				\item ``The State Department�s National Security Language Initiative for Youth (NSLI-Y) provides merit-based scholarships to U.S. high school students and recent graduates interested in learning less-commonly studied foreign languages.''
				\end{itemize}
			\end{enumerate}
		\end{enumerate}
	\end{enumerate}
%%%%%%%%%%%%%%%%%%%%%%%
\item underrepresented minorities: \vspace{-0.3cm}
	\begin{enumerate} \itemsep -2pt
	\item The University of North Carolina at Chapel Hill: \vspace{-0.2cm}
		\begin{enumerate} \itemsep -2pt
		\item Gary Bishop, {\it Research}, Department of Computer Science, The University of North Carolina at Chapel Hill. Available at: \url{http://wwwx.cs.unc.edu/~gb/wp/research/}; last accessed on September 3, 2010. [ Has plenty of information and resources (including learning aids and material) to help people who are visually or mobility impaired learn. ]
		\end{enumerate}
	\item Myra Sadker Foundation: \vspace{-0.2cm}
		\begin{enumerate} \itemsep -2pt
		\item $100+$ Ideas to Promote Gender Equity in Schools and Beyond: \url{http://www.sadker.org/100ideas.html}
		\item Gender Equity Activities: \url{http://www.sadker.org/WhatYouCanDo.html}
		\item Gender Equity Activities for Concerned Citizens: \url{http://www.sadker.org/GenderEquity-citizens.html}
		\item Gender Equity Activities for Families: \url{http://www.sadker.org/GenderEquity-family.html}
		\item Gender Equity Activities for Teachers: \vspace{-0.1cm}
			\begin{enumerate} \itemsep -1pt
			\item Early Childhood: \url{http://www.sadker.org/GenderEquity-teacher1.html}
			\item Primary Grades: \url{http://www.sadker.org/GenderEquity-teacher2.html}
			\item Upper Elementary: \url{http://www.sadker.org/GenderEquity-teacher3.html}
			\item Middle and High School: \url{http://www.sadker.org/GenderEquity-teacher4.html}
			\end{enumerate}
		\item Resources for feminism and links to web pages of feminist organizations: \url{http://www.sadker.org/ReadsLinks.html}
		\end{enumerate}
	\item League of United Latin American Citizens (LULAC): \vspace{-0.3cm}
		\begin{enumerate} \itemsep -2pt
		\item LULAC National Educational Service Centers, Inc: \vspace{-0.2cm}
			\begin{enumerate} \itemsep -2pt
			\item \url{http://www.lnesc.org/}
			\item Programs: \vspace{-0.1cm}
				\begin{itemize} \itemsep -1pt
				\item Improving literacy among Latino/Latina youth
				\item Encouraging Latino/Latina youth to pursue careers in science and engineering
				\item Helping Latino/Latina youth acquire leadership skills
				\item Improving college access for Latino/Latina youth by mentoring and summer programs (e.g., Gear-Up, Upward Bound, and the PALMS Initiative)
				\item Helping Latino/Latina families acquire financial success, so that Latino/Latina youth can pursue higher education
				\item Scholarships for Latino/Latina youth
				\item \url{http://lnesc.org/index.asp?Type=B_BASIC&SEC={808B6D04-913C-483F-8A05-5BD44B03ED62}}
				\end{itemize}
			\end{enumerate}
		\end{enumerate}
	\item ASPIRA: \vspace{-0.2cm}
		\begin{enumerate} \itemsep -2pt
		\item ASPIRA Programs for Latino/Latina youth: \url{http://aspira.org/manuals/aspira-programs}
		\end{enumerate}
	\end{enumerate}
%%%%%%%%%%%%%%%%%%%%%%%
\item places to visit: \vspace{-0.3cm}
	\begin{enumerate} \itemsep -2pt
	\item Exploratorium @ The Palace of Fine Arts (San Francisco, CA): \url{http://www.exploratorium.edu/}
	\item Educational Dividends: \vspace{-0.2cm}
		\begin{enumerate} \itemsep -2pt
		\item \url{http://www.educationaldividends.com/}
		\item Suggestions for organizing field trips to explore your interests: \url{http://www.educationaldividends.com/students/student_issues.asp}
		\item Career exploration: \url{http://www.educationaldividends.com/students/career_choices.asp}
		\item Computer skills: \url{http://www.educationaldividends.com/students/technology.asp}
		\item Quizzes to help you find out what is your preferred learning style and to discover more about your personality: \url{http://www.educationaldividends.com/students/learning_quiz.asp}
		\item Resources to help you learn about various topics in science, mathematics, social science, and humanities: \url{http://www.educationaldividends.com/students/resources.asp}
		\end{enumerate}
	\end{enumerate}
%%%%%%%%%%%%%%%%%%%%%%%
\item resources for at-risk youths: \vspace{-0.3cm}
	\begin{enumerate} \itemsep -2pt
	\item At-Risk Youth: \url{http://www.at-risk.org/}
	\item Peace First: \vspace{-0.2cm}
		\begin{enumerate} \itemsep -2pt
		\item \url{http://www.peacefirst.org/site/}
		\item To help youths become ``problem-solvers, rather than witnesses, or victims of their surrounding''
		\item To reduce youth homicide rates
		\item Teach children ``critical conflict resolution skills''
		\item Help teachers improve their ``conflict resolution and classroom management skills''
		\item To encourage youths to help each other, and get them to break up fights
		\item ``The Peace First curriculum is tailored to meet the developmental needs of students in Pre-K through eighth grade. Once a week, young adult volunteers and classroom teachers work together to teach students about friendship, communication, and conflict resolution through the use of experiential activities. First graders learn about communicating their feelings, third graders work on being peacemakers in their classroom, and fifth graders explore how to resolve and deescalate conflicts.''
		\item Has programs for students/youths, teachers, principals, and volunteers.
		\end{enumerate}
	\item Americans for the Arts: \vspace{-0.2cm}
		\begin{enumerate} \itemsep -2pt
		\item YouthARTS: \vspace{-0.1cm}
			\begin{enumerate} \itemsep -1pt
			\item \url{http://www.artsusa.org/youtharts/index.asp}
			\item ``The YouthARTS site is designed to give arts agencies, juvenile justice agencies, social service organizations, and other community-based organizations detailed information about how to plan, run, provide training, and evaluate arts programs for at-risk youth.''
			\end{enumerate}
		\end{enumerate}
	\end{enumerate}
%%%%%%%%%%%%%%%%%%%%%%%
\item general music and arts education: \vspace{-0.3cm}
	\begin{enumerate} \itemsep -2pt
	\item Americans for the Arts: \vspace{-0.2cm}
		\begin{enumerate} \itemsep -2pt
		\item Americans for the Arts, ``Ten Simple Ways Parents Can Get More Art in Their Kids' Lives.'' Available online at: \url{http://www.americansforthearts.org/public_awareness/get_involved/001.asp}; last accessed on November 30, 2010.
		\item YouthARTS: \vspace{-0.1cm}
			\begin{enumerate} \itemsep -1pt
			\item \url{http://www.artsusa.org/youtharts/index.asp}
			\item ``The YouthARTS site is designed to give arts agencies, juvenile justice agencies, social service organizations, and other community-based organizations detailed information about how to plan, run, provide training, and evaluate arts programs for at-risk youth.''
			\end{enumerate}
		\end{enumerate}
	\item The John F. Kennedy Center for the Performing Arts: \vspace{-0.2cm}
		\begin{enumerate} \itemsep -2pt
		\item Kennedy Center Institute for Arts Management: \url{http://artsmanagerfba.artsmanager.org/common/Pages/About.aspx}
		\item {\sc ArtsEdge}: \vspace{-0.1cm}
			\begin{enumerate} \itemsep -1pt
			\item The National Standards for Arts Education for Grades K-4, 5-8, and 9-12: \url{http://artsedge.kennedy-center.org/educators/standards.aspx}
			\item Tips and guides for educators: \url{http://artsedge.kennedy-center.org/educators/how-to.aspx}
			\item Lesson plans for educators: \url{http://artsedge.kennedy-center.org/educators/lessons.aspx}
			\item Information for parents, guardians, foster parents, baby-sitters, and grandparents: \url{http://artsedge.kennedy-center.org/families.aspx}
			\item Information for students: \url{http://artsedge.kennedy-center.org/students.aspx}
			\item Themes for artistic, cultural, academic, and intellectual exploration: \url{http://artsedge.kennedy-center.org/themes.aspx}
			\item Multimedia: \url{http://artsedge.kennedy-center.org/multimedia.aspx}
			\end{enumerate}
		\end{enumerate}
	\end{enumerate}
\item music education: \vspace{-0.3cm}
	\begin{enumerate} \itemsep -2pt
	\item Washington Performing Arts Society (WPAS): \vspace{-0.2cm}
		\begin{enumerate} \itemsep -2pt
		\item WPAS Education \& Community -- Connections through the Arts Education Programs for All Ages: \vspace{-0.1cm}
			\begin{enumerate} \itemsep -1pt
			\item The Capitol Jazz Project: \vspace{-0.1cm}
				\begin{itemize} \itemsep -1pt
				\item \url{http://www.wpas.org/educcomm/programsforyoungpeople/capitoljazzproject.aspx}
				\item ``Washington Performing Arts Society (WPAS) and the D.C. Public Schools, in collaboration with Jazz at Lincoln Center, has launched The Capitol Jazz Project, an important step in supporting music education for all students in the District of Colombia.''
				\item ``Through the Capitol Jazz Project, students hone their listening, performing, improvising, composing, arranging, music reading, and notation skills.''
				\item ``The Capitol Jazz Project is being implemented in 6 D.C. middle schools with a total enrollment of more than 500 music students.''
				\item ``A true collaboration, The Capitol Jazz Project brings the combined resources and expertise of WPAS, Jazz at Lincoln Center, and the D.C. Public Schools to create a model music education program.''
				\end{itemize}
			\item Joseph and Goldie Feder Memorial String Competition: \vspace{-0.1cm}
				\begin{itemize} \itemsep -1pt
				\item \url{http://www.wpas.org/educcomm/programsforyoungpeople/josephandgoldiefedermemorialstringcompetition.aspx}
				\item ``The Feder String Competition inspires and nurtures D.C. area youth in grades 6 through 12 who study violin, viola, cello, and double bass.''
				\item ``Each year, 80 students compete for 30 awards and scholarships.''
				\item ``Held each spring, WPAS awards cash prizes toward private lessons, scholarships for summer study programs, and tickets for top winners and their family members to attend a WPAS concert.''
				\item ``Winners of the competition are also given special performance opportunities such as on the Kennedy Center's Millennium Stage and The Shakespeare Theatre Company's Happenings at the Harman series.''
				\end{itemize}
			\item WPAS Summer Performing Arts Academy summer programs: \vspace{-0.1cm}
				\begin{itemize} \itemsep -1pt
				\item \url{http://www.wpas.org/educcomm/programsforyoungpeople/wpassummerperformingartsacademy.aspx}
				\end{itemize}
			\end{enumerate}
		\end{enumerate}
	\item Young Concert Artists, Inc. \vspace{-0.2cm}
		\begin{enumerate} \itemsep -2pt
		\item Annaliese Soros Educational Residency Program: \url{http://www.yca.org/auditions/}
		\end{enumerate}
	\item The Choral Arts Society of Washington: \vspace{-0.2cm}
		\begin{enumerate} \itemsep -2pt
		\item Classroom Resources: \url{http://www.choralarts.org/Education/Classroom-Resources.aspx}
		\end{enumerate}
	\item League of American Orchestras: \vspace{-0.2cm}
		\begin{enumerate} \itemsep -2pt
		\item Career planning: \vspace{-0.1cm}
			\begin{enumerate} \itemsep -1pt
			\item Resources for pre-college students, college students, and graduate students: \url{http://www.americanorchestras.org/career_center/career_planning.html}
			\item Arts Administration programs: \url{http://www.americanorchestras.org/career_center/arts_admin_programs.html}
			\item Non-profit management, {\bf public policy} and leadership programs: \url{http://www.americanorchestras.org/career_center/resources_non_prof_and.html}
			\end{enumerate}
		\end{enumerate}
	\item The John F. Kennedy Center for the Performing Arts: \vspace{-0.2cm}
		\begin{enumerate} \itemsep -2pt
		\item Betty Carter's Jazz Ahead: \vspace{-0.1cm}
			\begin{enumerate} \itemsep -1pt
			\item \url{http://www.kennedy-center.org/programs/jazz/jazzahead/}
			\item ``Music residency program for young people''
			\item ``The Jazz Ahead program identifies outstanding, emerging jazz artists in their mid-teens to age thirty, and brings them together under the tutelage of experienced artist-instructors who coach and counsel them, helping to polish their performance, composing and arranging skills.''
			\item ``The two week-long residency program includes daily workshops and rehearsals with established jazz artists, and culminate in three concerts on the Kennedy Center Millennium Stage, which will be broadcast live over the internet.''
			\end{enumerate}
		\item The National Symphony Orchestra (NSO): \vspace{-0.1cm}
			\begin{enumerate} \itemsep -1pt
			\item The National Symphony Orchestra's Summer Music Institute (SMI): \vspace{-0.1cm}
				\begin{itemize} \itemsep -1pt
				\item \url{http://www.kennedy-center.org/nso/nsoed/smi/home.cfm}
				\item ``Every summer, approximately 70 students (ages 15-20) from all over the nation meet in Washington, D.C., to attend the National Symphony Orchestra's Summer Music Institute (SMI).''
				\item ``The Institute offers four weeks of private lessons, rehearsals, coaching by National Symphony Orchestra members, classes, and lectures to prepare aspiring musicians for their futures in music.''
				\end{itemize}
			\item Young Associates' Program: \vspace{-0.1cm}
				\begin{itemize} \itemsep -1pt
				\item \url{http://www.kennedy-center.org/nso/nsoed/youngassociates.html}
				\item ``The National Symphony Orchestra (NSO) is sponsoring its Young Associates' Program for high school students in grades 11 and 12 in the Washington, DC, metropolitan area who are interested in pursuing a musical career.''
				\item ``Twenty outstanding instrumentalists (pianists are not included) will be selected to attend rehearsals of the National Symphony Orchestra and take part in seminars with conductors, artists, NSO musicians, and representatives of the arts management field.''
				\item ``Through this program, the Young Associates will acquire an appreciation of the wide range of skills, knowledge, and abilities--managerial as well as musical--that are required to put together a performance by a major symphony orchestra. Selection process is by application.''
				\item ``The core of the program involves attendance at rehearsals of the National Symphony Orchestra at the Kennedy Center and observation of various guest artists. In addition to attending NSO rehearsals, students participate in workshops to explore careers in management, music education, publicity, music library, and other professions that are essential to the life of every successful orchestra.''
				\item ``Students do not play their instruments as part of the program. Students learn through listening, observation, and asking questions of professionals.''
				\end{itemize}
			\end{enumerate}
		\end{enumerate}
	\end{enumerate}
\item dance education: \vspace{-0.3cm}
	\begin{enumerate} \itemsep -2pt
	\item The Washington Ballet: \vspace{-0.2cm}
		\begin{enumerate} \itemsep -2pt
		\item The Washington School of Ballet (TWSB): \vspace{-0.1cm}
			\begin{enumerate} \itemsep -1pt
			\item Summer Intensive program (requires an audition): \url{http://www.washingtonballet.org/the-school/summer-intensive/}
			\end{enumerate}
		\item TWB's EXCEL! scholarship program (for DanceDC students): \vspace{-0.1cm}
			\begin{enumerate} \itemsep -1pt
			\item \url{http://www.washingtonballet.org/community-engagement/default.htm}
			\item \url{http://www.washingtonballet.org/community-engagement/other-programs/}
			\item Also, has need-based scholarships
			\end{enumerate}
		\end{enumerate}
	\item The John F. Kennedy Center for the Performing Arts: \vspace{-0.2cm}
		\begin{enumerate} \itemsep -2pt
		\item Exploring Ballet With Suzanne Farrell: A Three-Week Summer Ballet Intensive for Young Dancers: \vspace{-0.1cm}
			\begin{enumerate} \itemsep -1pt
			\item \url{http://www.kennedy-center.org/education/farrell/}
			\item ``In July and August, students from across the United States and around the world will participate in the eighteenth annual session of the Kennedy Center's ballet training program Exploring Ballet with Suzanne Farrell. The three-week residency for dancers ages 14 to 18 with at least five years of ballet training will be held at the Kennedy Center from August 1 - August 20, 2011.''
			\item ``During the three-week period, students take two ballet technique classes a day, six days a week, with Ms. Farrell. Students also participate in a number of cultural activities to enhance their experience in Washington, D.C., including museum visits, trips to historical landmarks, and attending performances.''
			\end{enumerate}
		\item Dance Theatre of Harlem Residency program: \vspace{-0.1cm}
			\begin{enumerate} \itemsep -1pt
			\item \url{http://www.kennedy-center.org/education/community/programs.html#artistic}
			\item ``Since 1993, the Kennedy Center's Dance Theatre of Harlem Residency program has provided ballet training for male and female students age 8-18 with identified promise in ballet taught by Dance Theatre of Harlem (DTH) instructors or former principal dancers.''
			\item ``Students are selected by audition for a twenty-class series, culminating with a public demonstration and performance on a Kennedy Center main stage.''
			\item ``Classical ballet training is taught in four class levels, from novice to advance.''
			\item ``Students must have at least one year of ballet training to qualify for the program.''
			\end{enumerate}
		\end{enumerate}
	\end{enumerate}
%%%%%%%%%%%%%%%%%%%%%%%
\item JA Worldwide (Junior Achievement): \vspace{-0.3cm}
	\begin{enumerate} \itemsep -2pt
	\item \url{http://www.ja.org/}
	\item Resources for educators: \url{http://www.ja.org/involved/involved_educat.shtml}
	\item Resources for parents: \url{http://www.ja.org/involved/involved_parents.shtml}
	\item Resources for students: \url{http://www.ja.org/involved/involved_students.shtml}
	\end{enumerate}
\item U.S. Department of State: \vspace{-0.3cm}
	\begin{enumerate} \itemsep -2pt
	\item Programs for Americans and non-Americans.
	\item Summer Work Travel - In the summer work travel program: \url{http://exchanges.state.gov/}
	\item Cultural Programs Division: \url{http://exchanges.state.gov/cultural/index.html}
	\item Youth Programs Division: \url{http://exchanges.state.gov/youth/index.html}
	\item EducationUSA: \url{http://educationusa.state.gov/}
	\item International Visitor Leadership Program: \url{http://exchanges.state.gov/ivlp/ivlp.html}
	\item Programs for non-U.S. Citizens: \url{http://exchanges.state.gov/prog-non-us.html}
	\item Programs for U.S. Citizens: \url{http://exchanges.state.gov/prog-us.html}
	\item Resources for Students: \url{http://exchanges.state.gov/student.html}
	\item Bureau of Educational and Cultural Affairs: \vspace{-0.2cm}
		\begin{enumerate} \itemsep -2pt
		\item Future Leaders Exchange (FLEX) Program: \vspace{-0.1cm}
			\begin{enumerate} \itemsep -1pt
			\item \url{http://exchanges.state.gov/youth/programs/flex.html}
			\item ``The Future Leaders Exchange (FLEX) Program gives students (ages 15-17) the chance to live with a host family and attend a U.S. high school for a year.''
			\end{enumerate}
		\item Office of Citizen Exchanges: \vspace{-0.1cm}
			\begin{enumerate} \itemsep -1pt
			\item Youth Programs Division: \vspace{-0.1cm}
				\begin{itemize} \itemsep -1pt
				\item \url{http://exchanges.state.gov/youth/index.html}
				\item Has programs for youths in various parts of the world
				\item ``The Youth Programs Division is committed to empowering the next generation and establishing long-lasting ties between the United States and other countries through exchange programs and institutional partnerships. Programs focus primarily on secondary schools and promote mutual understanding, leadership development, educational transformation and democratic ideals.''
				\end{itemize}
			\item SportsUnited: \vspace{-0.1cm}
				\begin{itemize} \itemsep -1pt
				\item \url{http://exchanges.state.gov/sports/index.html}
				\item SportsUnited is an international sports programming initiative designed to help start a dialogue at the grassroots level with non-elite boys and girls ages 7-17.
				\item The programs aid youth in discovering how success in athletics can be translated into the development of life skills and achievement in the classroom.
				\item Foreign participants are given an opportunity to establish links with U.S. sports professionals and exposure to American life and culture.
				\item Americans learn about foreign cultures and the challenges young people from overseas face today.
				\item The U.S. Department of State has programmed initiatives in: baseball, basketball, football, track and field, soccer, volleyball, wrestling, archery, boxing, swimming, fencing, table tennis, ice skating, weightlifting, water polo and managing sports community centers.
				\item Countries covered by this program are listed on the web page.
				\item Sports Envoy Program: \vspace{-0.1cm}
					\begin{itemize} \itemsep -1pt
					\item \url{http://exchanges.state.gov/sports/envoy1.html}
					\item Working with the national sports leagues and the U.S. Olympic Committee, athletes and coaches in various sports are chosen to serve as envoys or ambassadors of sport in overseas programs that include conducting clinics, visiting schools and speaking to youth.
					\item The American athletes and coaches conduct drills and team building activities, as well as engage the youth in a dialogue on the importance of an education, positive health practices and respect for diversity.
					\end{itemize}
				\item Sports Grant Competition: \vspace{-0.1cm}
					\begin{itemize} \itemsep -1pt
					\item The Bureau of Educational and Cultural Affairs (ECA) has an annual open competition under its International Sports Programming Initiative.
					\item Public and private non-profit organizations, 501(c)(3), may submit proposals to discuss approaches designed to enhance and improve the infrastructure of youth sports programs.
					\item The focus of all programs must be reaching out to non-elite youth ages 7-17 and/or their coaches/administrators.
					\item There are four themes that a proposal can address; Youth Sports Management, Training Sports Coaches, Sport and Disability, and Sport and Health.
					\item The list of eligible countries changes each year.
					\item \url{http://exchanges.state.gov/sports/index/sports-grant-competition.html}
					\end{itemize}
				\item Sports Visitor Program: \vspace{-0.1cm}
					\begin{itemize} \itemsep -1pt
					\item Nominated by our U.S. embassies overseas, selected athletes, managers and coaches are brought to the U.S. for technical sports training, sports management, conflict resolution training and exposure to valuable U.S. sports contacts and then are encouraged to return to conduct in-country clinics for youth with their newly learned skills.
					\item \url{http://exchanges.state.gov/sports/visitors.html}
					\end{itemize}
				\end{itemize}
			\end{enumerate}
		\end{enumerate}
	\end{enumerate}
\item U.S. Department of Labor: \vspace{-0.3cm}
	\begin{enumerate} \itemsep -2pt
	\item Wage and Hour Division: \vspace{-0.2cm}
		\begin{enumerate} \itemsep -2pt
		\item YouthRules!: \vspace{-0.1cm}
			\begin{enumerate} \itemsep -1pt
			\item \url{http://youthrules.dol.gov/}
			\item Has information for youths, parents, educators, and employers on how to let youth work part-time safely
			\item Teens: \url{http://youthrules.dol.gov/teens/default.htm}
			\item Parents: \url{http://youthrules.dol.gov/parents/default.htm}
			\item Educators: \url{http://youthrules.dol.gov/educators/default.htm}
			\item Employers: \url{http://youthrules.dol.gov/employers/default.htm}
			\item Resources: \url{http://youthrules.dol.gov/resources.htm}
			\item Compliance Assistance: \url{http://youthrules.dol.gov/ca.htm}
			\end{enumerate}
		\end{enumerate}
	\end{enumerate}
\item ASCL Educational Services, Inc. (Marc McCulloch): \vspace{-0.3cm}
	\begin{enumerate} \itemsep -2pt
	\item Transitions: Life Skills for Personal Success!: \vspace{-0.2cm}
		\begin{enumerate} \itemsep -2pt
		\item Curriculum \& Materials: \url{http://transitions.ascl.info/infomaterials}
		\item Soft Skills: \url{http://transitions.ascl.info/infoskills}
		\end{enumerate}
	\end{enumerate}
\item Partnership for 21st Century Skills: \vspace{-0.3cm}
	\begin{enumerate} \itemsep -2pt
	\item \url{http://www.p21.org/}
	\item Framework for 21st Century Learning: \url{http://www.p21.org/index.php?option=com_content&task=view&id=254&Itemid=119}
	\item Tools and Resources: \url{http://www.p21.org/index.php?option=com_content&task=view&id=273&Itemid=139}
	\end{enumerate}
\item National Career and Technical Education Foundation (NCTEF): \vspace{-0.3cm}
	\begin{enumerate} \itemsep -2pt
	\item States' Career Clusters Initiative (SCCI): \vspace{-0.2cm}
		\begin{enumerate} \itemsep -2pt
		\item \url{http://www.careerclusters.org/}
		\item The 16 Career Clusters: \url{http://www.careerclusters.org/16clusters.cfm}
		\item Plans of Study: \url{http://www.careerclusters.org/resources/web/pos.cfm}
		\item Knowledge and Skills Charts: \url{http://www.careerclusters.org/resources/web/ks.php}
		\item Crosswalks: \url{http://www.careerclusters.org/crosswalks.cfm}
		\item Publications: \url{http://www.careerclusters.org/publications.php}
		\item Sixteen Career Clusters and Their Pathways: \url{http://www.careerclusters.org/list16clusters.php}
		\item Career Clusters Models: \url{http://www.careerclusters.org/resources/web/16ccall.php?action=models}
		\item Career Clusters Brochure Previews: \url{http://www.careerclusters.org/resources/web/16ccall.php?action=brochures}
		\item Career Clusters Interest Survey: \url{http://www.careerclusters.org/ccinterestsurvey.php}
		\item Related Websites: \url{http://www.careerclusters.org/related.php}
		\end{enumerate}
	\end{enumerate}
\item U. S. Department of Labor: \vspace{-0.3cm}
	\begin{enumerate} \itemsep -2pt
	\item Employment and Training Administration: \vspace{-0.2cm}
		\begin{enumerate} \itemsep -2pt
		\item CareerOneStop: \vspace{-0.1cm}
			\begin{enumerate} \itemsep -1pt
			\item \url{http://www.careeronestop.org/}
			\item Students, parents, and career advisors: \url{http://www.careeronestop.org/studentsandcareeradvisors/studentsandcareeradvisors.aspx}
			\end{enumerate}
		\end{enumerate}
	\end{enumerate}
\item U. S. Department of Defense: \vspace{-0.3cm}
	\begin{enumerate} \itemsep -2pt
	\item ASVAB Career Exploration Program: \vspace{-0.2cm}
		\begin{enumerate} \itemsep -2pt
		\item \url{http://www.asvabprogram.com/}
		\item Learn about yourself: \url{http://www.asvabprogram.com/index.cfm?fuseaction=learn.main}
		\item Explore careers: \url{http://www.asvabprogram.com/index.cfm?fuseaction=explore.main}
		\item Plan for your future: \url{http://www.asvabprogram.com/index.cfm?fuseaction=plan.main}
		\item Information for educators and career counselors: \url{http://www.asvabprogram.com/index.cfm?fuseaction=edu.main}
		\item Information for parents: \url{http://www.asvabprogram.com/index.cfm?fuseaction=parents.main}
		\end{enumerate}
	\end{enumerate}
\end{enumerate}



%%%%%%%%%%%%%%%%%%%%%%%%%%%%%%%%%%%%%%%%%%%
\section{Internship Opportunities}
\label{Internship Opportunities}

Internship opportunities: \vspace{-0.3cm}
\begin{enumerate} \itemsep -4pt
\item Canada: \vspace{-0.3cm}
	\begin{enumerate} \itemsep -2pt
	\item SWAP: \vspace{-0.2cm}
		\begin{enumerate} \itemsep -2pt
		\item \url{http://www.swap.ca/}
		\item For Canadians who want to work abroad: \url{http://www.swap.ca/out_eng/index.aspx}
		\item For citizens of selected countries who want to work in Canada: \url{http://www.swap.ca/in_eng/partner_organizations.aspx}
		\end{enumerate}
	\end{enumerate}
\item Singapore: \vspace{-0.3cm}
	\begin{enumerate} \itemsep -2pt
	\item Speedwing Training (Asia) Pte Ltd: \vspace{-0.2cm}
		\begin{enumerate} \itemsep -2pt
		\item \url{http://www.speedwing.org/}
		\item For Singaporeans who want to work in the United States, Canada, Germany, and New Zealand
		\item For citizens of selected countries who want to work in Singapore
		\end{enumerate}
	\end{enumerate}
\end{enumerate}

%%%%%%%%%%%%%%%%%%%%%%%%%%%%%%%%%%%%%%%%%%%
\subsection{Internship Opportunities in Australia}
\label{internshipaus}

Internship Opportunities in Australia: \vspace{-0.3cm}
\begin{enumerate} \itemsep -4pt
\item The Association of Professional Engineers, Scientists and Managers, Australia: \url{http://www.apesma.asn.au/index.asp} --- Ask for guide to internships in your region/major; free student membership
\item Engineers Australia: \url{http://www.engineersaustralia.org.au/} --- Ask for guide to internships in your region/major; free student membership
\item CPA Australia: \url{http://www.cpaaustralia.com.au/cps/rde/xchg/cpa/hs.xsl/index.html} and \url{http://www.cpaaustralia.com.au/cps/rde/xchg/careers/site/index_ENA_HTML.htm/cps/rde/xchg/SID-3F57FECB-EEFEF50E/careers/site/204_ENA_HTML.htm}
\item Institute of Chartered Accountants in Australia: \url{http://www.charteredaccountants.com.au/}
\item 
\end{enumerate}


%%%%%%%%%%%%%%%%%%%%%%%%%%%%%%%%%%%%%%%%%%%
\subsection{Internship Opportunities in Europe}
\label{internshipeu}

Internship Opportunities in Portugal: \vspace{-0.3cm}
\begin{enumerate} \itemsep -4pt
\item Portugal: \vspace{-0.3cm}
	\begin{enumerate} \itemsep -2pt
	\item IAESTE Portugal (The International Association for the Exchange of Students for Technical Experience): \url{http://www.iaeste.pt/en/foreign-trainees/why-portugal/}
	\end{enumerate}
\item United Kingdom: \vspace{-0.3cm}
	\begin{enumerate} \itemsep -2pt
	\item Graduate Talent Pool: \url{http://graduatetalentpool.direct.gov.uk/}
	\end{enumerate}
\end{enumerate}




%%%%%%%%%%%%%%%%%%%%%%%%%%%%%%%%%%%%%%%%%%%
\subsection{Internship Opportunities in the United States}
\label{internshipsus}

Internship Opportunities in the United States: \vspace{-0.3cm}
\begin{enumerate} \itemsep -4pt
\item Use the Procedure \proc{Find}$(\varphi, \tau)$ in \S\ref{heuristiclocateoutreach} to look up internship opportunities and lists of internship opportunities.

Look at government organizations (e.g., the White House), nonprofit organizations (e.g., Engineers Without Borders), professional organizations (e.g., IEEE and ACM), colleges and universities, and companies (e.g., Intel, Google, and start-ups).

You can start your search by looking at the organizations that provide resources for underrepresented minorities as well as resources for scholarships and fellowships. These information can be found in other sections of this document.

If you do not know where to start, speak to a professor or staff member at the career center of your college/university. Alternatively, you can ask your awesome resident advisors (RAs).

My personal advice is to start your search based on your interests and skill set. You can always narrow the search space based on factors, such as geographical location, later on.

Competitive internships, especially research internships in electrical and computer engineering or computer science, weed out many students from applying via demanding job requirements. For example, if you want to apply for research internships with electronic design automation (EDA) companies and corporate research labs, you would need to have significant experience designing integrated circuits and developing EDA software. The stringent job requirements also mean that students need to plan in advance (say, about a year) about the internships that they would like to seek, and plan to acquire the necessary skill set and experiences before the application deadlines (which can be several months before the start of your internship).

Taking as many challenging classes as you can possibly cope, especially in electrical and computer engineering or computer science, would provide you with a skill set that allows you to apply for competitive internships in many fields. Apart from taking challenging classes as well as engaging in research and/or open source projects, you can try to acquire additional skills and experience in your free time to boost the competitiveness of your internship application. Certain skills and experiences, such as compiler design, are hard to acquire in your free time, so it would be ``easier'' to take classes that would help you acquire those skills and experiences.

Note that you may want to look into creating your own entrepreneurial venture, say an EDA start-up or organization in social entrepreneurship, rather than to seek an internship. Also, seeking an internship abroad is always a good addition to your resume/CV.
\item National Science Foundation: \vspace{-0.3cm}
	\begin{enumerate} \itemsep -2pt
	\item Research Experiences for Undergraduates (REU): \vspace{-0.2cm}
		\begin{enumerate} \itemsep -2pt
		\item \url{http://www.nsf.gov/crssprgm/reu/reu_search.cfm}
		\item Academic fields: \vspace{-0.1cm}
			\begin{enumerate} \itemsep -1pt
			\item Astronomical Sciences
			\item Atmospheric and Geospace Sciences
			\item Biological Sciences
			\item Chemistry
			\item Computer and Information Science and Engineering
			\item Cyberinfrastructure
			\item Department of Defense (DoD)
			\item Earth Sciences
			\item Education and Human Resources
			\item Engineering
			\item Ethics and Values Studies
			\item International Science and Engineering
			\item Materials Research
			\item Mathematical Sciences
			\item Ocean Sciences
			\item Physics
			\item Polar Programs
			\item Social, Behavioral, and Economic Sciences
			\end{enumerate}
		\end{enumerate}
	\end{enumerate}
\item Society for Industrial and Applied Mathematics: \vspace{-0.3cm}
	\begin{enumerate} \itemsep -2pt
	\item Internship and Career Information in Industry, Research Institutions, and Government Labs: \url{http://www.siam.org/careers/internships.php}
	\end{enumerate}
\item American Institute of Physics (AIP): \vspace{-0.3cm}
	\begin{enumerate} \itemsep -2pt
	\item Society of Physics Students (SPS): \vspace{-0.2cm}
		\begin{enumerate} \itemsep -2pt
		\item SPS Internships: \url{http://www.spsnational.org/programs/internships/}
		\item Research Opportunities: \url{http://www.spsnational.org/programs/research/}
		\end{enumerate}
	\end{enumerate}
%%%%%%%%%%%%%%%%%%%%%%%%%%%%%%%%%%%%%%
%%%%%%%%%%%%%%%%%%%%%%%%%%%%%%%%%%%%%%
%%%%%%%%%%%%%%%%%%%%%%%%%%%%%%%%%%%%%%
\item United States Office of Personnel Management: \vspace{-0.3cm}
	\begin{enumerate} \itemsep -2pt
	\item USAJOBS: \vspace{-0.2cm}
		\begin{enumerate} \itemsep -2pt
		\item Student Jobs: \url{http://www.usajobs.gov/studentjobs/}
		\end{enumerate}
	\end{enumerate}
%%%%%%%%%%%%%%%%%%%%%%%%%%%%%%%%%%%%%%
%%%%%%%%%%%%%%%%%%%%%%%%%%%%%%%%%%%%%%
%%%%%%%%%%%%%%%%%%%%%%%%%%%%%%%%%%%%%%
\item Americans for the Arts: \vspace{-0.3cm}
	\begin{enumerate} \itemsep -2pt
	\item Internship Program: \url{http://www.americansforthearts.org/about_us/internships.asp}
	\end{enumerate}
\item New York Women's Foundation: \vspace{-0.3cm}
	\begin{enumerate} \itemsep -2pt
	\item Internship Opportunities: \url{http://www.nywf.org/internship.html}
	\item Volunteer Opportunities: \url{http://www.nywf.org/volunteer.html}
	\end{enumerate}
\item Council on International Educational Exchange (CIEE): \url{http://www.ciee.org/hire/index.aspx}
\item The John F. Kennedy Center for the Performing Arts: \vspace{-0.3cm}
	\begin{enumerate} \itemsep -2pt
	\item Kennedy Center Arts Management Internships: \url{http://www.kennedy-center.org/education/artsmanagement/internships/}
	\end{enumerate}
\item Washington Performing Arts Society (WPAS): \vspace{-0.3cm}
	\begin{enumerate} \itemsep -2pt
	\item Internships with WPAS: \vspace{-0.2cm}
		\begin{enumerate} \itemsep -2pt
		\item \url{http://www.wpas.org/aboutwpas/opportunities/intern.aspx}
		\item ``WPAS offers internships throughout the year. Applicants should be highly motivated, creative and hard-working individuals with an interest in all aspects of arts management. It is required that applicants have previous office experience.''
		\item In addition, applicants should possess: \vspace{-0.1cm}
			\begin{enumerate} \itemsep -1pt
			\item Interest/background in music, dance or performance art
			\item Strong organizational skills
			\item Effective writing and communication skills
			\item Ability to learn quickly, handle multiple tasks, take initiative, and work independently with little supervision
			\item High energy level and ability to work well in deadline and/or pressure situations
			\item Computer literacy
			\end{enumerate}
		\item ``WPAS interns leave our offices with a better understanding of arts management, knowledge of artists in a variety of fields (classical music, world music, dance and performance art), contacts in theaters throughout the D.C. metro area, practical experience and a portfolio of work. The internship is unpaid, however stipends are occasionally granted during the performance year (September - May). Interns are also invited to attend many WPAS performances on a complimentary basis.''
		\item Types of internships: \vspace{-0.1cm}
			\begin{enumerate} \itemsep -1pt
			\item Accounting Internship
			\item Development Internship
			\item Education Internship
			\item Marketing/Public Relations Internship
			\item Office Administration Internship
			\item Programming Internship
			\end{enumerate}
		\end{enumerate}
	\end{enumerate}
\item The Washington Ballet: Internships, \url{http://www.washingtonballet.org/about-twb/auditions-employment/#internships}
\item The Choral Arts Society of Washington: \vspace{-0.3cm}
	\begin{enumerate} \itemsep -2pt
	\item Internship and Apprenticeship Program: \url{http://www.choralarts.org/About-Us/Internships-and-Apprenticeships.aspx}
	\end{enumerate}
\item League of American Orchestras: Internships, \url{http://www.americanorchestras.org/career_center/internships.html}
%%%%%%%%%%%%%%%%%%%%%%%%%%%%%%%%%%%%%%
%%%%%%%%%%%%%%%%%%%%%%%%%%%%%%%%%%%%%%
%%%%%%%%%%%%%%%%%%%%%%%%%%%%%%%%%%%%%%
\item Congressional Hispanic Caucus Institute (CHCI): \vspace{-0.3cm}
	\begin{enumerate} \itemsep -2pt
	\item CHCI United Health Foundation Scholars: \vspace{-0.2cm}
		\begin{enumerate} \itemsep -2pt
		\item \url{http://www.chci.org/scholarships/page/chci-united-health-foundation-scholars-}
		\item In addition to providing scholarship opportunities for Latino youth, the United Health Foundation decided to partner with CHCI to create a six-month internship program for students interested in the medical field.
		\item Seventeen participants enrolled in either a full-time undergraduate or graduate course of study at an accredited two- or four-year college, university, vocational or technical school were selected.
		\end{enumerate}
	\item CHCI Congressional Internship: \vspace{-0.2cm}
		\begin{enumerate} \itemsep -2pt
		\item The purpose of the Congressional Internship Program (CIP) is to expose young Latinos to the legislative process and to strengthen their professional and leadership skills, ultimately promoting the presence of Latinos on Capitol Hill.
		\item The Congressional Internship Program provides college students with a paid Congressional work placement on Capitol Hill for a period of twelve weeks (Spring/Fall) or eight weeks (Summer). This unmatched experience allows students to learn first hand about our nation's legislative process.
		\end{enumerate}
	\end{enumerate}
\item Mexican American Legal Defense and Educational Fund (MALDEF): Law Clerk Summer Internship program, \url{http://maldef.org/about/jobs/index.html}
\item Hispanic Association of Colleges and Universities (HACU): \vspace{-0.3cm}
	\begin{enumerate} \itemsep -2pt
	\item HACU National Internship Program (HNIP): \url{http://www.hacu.net/hacu/HNIP_EN.asp}
	\end{enumerate}
%%%%%%%%%%%%%%%%%%%%%%%%%%%%
\item Smithsonian Institution: \vspace{-0.3cm}
	\begin{enumerate} \itemsep -2pt
	\item Smithsonian Institution Traveling Exhibition Service (SITES): \vspace{-0.2cm}
		\begin{enumerate} \itemsep -2pt
		\item Internship programs: \url{http://www.sites.si.edu/interns/internships.htm}
		\item ``The Smithsonian Institution Traveling Exhibition Service internship programs allows people with diverse interests, strengths, and goals to experience an educational environment where they can work and learn from professionals in the museum field.''
		\item ``SITES offers internship opportunities in a variety of different areas: public relations, development (fund raising), research, and project design.''
		\end{enumerate}
	\item Smithsonian Folkways Recordings (or simply, Smithsonian Folkways): \vspace{-0.2cm}
		\begin{enumerate} \itemsep -2pt
		\item Internships: \url{http://www.folkways.si.edu/about_us/jobs.aspx}
		\end{enumerate}
	\item Freer Gallery of Art / Arthur M. Sackler Gallery: \vspace{-0.2cm}
		\begin{enumerate} \itemsep -2pt
		\item Internships: \url{http://www.asia.si.edu/research/internships.asp}
		\end{enumerate}
	\item National Museum of American History: \vspace{-0.2cm}
		\begin{enumerate} \itemsep -2pt
		\item Jerome and Dorothy Lemelson Center for the Study of Invention and Innovation: \vspace{-0.1cm}
			\begin{enumerate} \itemsep -1pt
			\item Archival Internships: \url{http://invention.smithsonian.org/resources/research_interns.aspx}
			\end{enumerate}
		\end{enumerate}
	\end{enumerate}
%%%%%%%%%%%%%%%%%%%%%%%%%%%%
\item Council on International Educational Exchange (CIEE): \vspace{-0.3cm}
	\begin{enumerate} \itemsep -2pt
	\item CIEE's Trainee Program: \vspace{-0.2cm}
		\begin{enumerate} \itemsep -2pt
		\item part of the J-1 visa category of the US government�s Exchange Visitor Program
		\item \url{http://www.ciee.org/trainee/}
		\end{enumerate}
	\item CIEE Work \& Travel USA; and Internship USA: \vspace{-0.2cm}
		\begin{enumerate} \itemsep -2pt
		\item \url{http://www.ciee.org/hire/}
		\item \url{http://www.ciee.org/wat/}
		\end{enumerate}
	\end{enumerate}
\item American Institute For Foreign Study (AIFS): \vspace{-0.3cm}
	\begin{enumerate} \itemsep -2pt
	\item Camp America Counselors and Summer Staff: \url{http://www.aifs.com/work_travel.asp}
	\item Au Pair Placement: \url{http://www.aifs.com/au_pair.asp}
	\end{enumerate}
\item U.S. Department of State: \vspace{-0.3cm}
	\begin{enumerate} \itemsep -2pt
	\item Bureau of Educational and Cultural Affairs: \vspace{-0.2cm}
		\begin{enumerate} \itemsep -2pt
		\item International cultural programs: \url{http://exchanges.state.gov/cultural/related-cultural-programs.html}
		\item Office of Global Educational Programs: \vspace{-0.1cm}
			\begin{enumerate} \itemsep -1pt
			\item Camp Counselor: \vspace{-0.1cm}
				\begin{itemize} \itemsep -1pt
				\item \url{http://exchanges.state.gov/jexchanges/programs/camp.html}
				\item Camp counselors interact with groups of American youth by overseeing their camp activities during the U.S. summer.
				\item Through the Camp Counselor program, American campers have the chance to gain knowledge of foreign cultures, while foreign participants increase their knowledge of American culture.
				\item Participants must be at least 18 years of age and may work as counselors in U.S. summer camps for up to four months. Extensions are not allowed. They receive a combination a pay and benefits equal to Americans who work in the same position.
				\end{itemize}
			\end{enumerate}
		\item Private Sector Exchange office: \vspace{-0.1cm}
			\begin{enumerate} \itemsep -1pt
			\item \url{http://exchanges.state.gov/jexchanges/index.html}
			\item The Private Sector Exchange office designates, monitors and partners with U.S. organizations, including government agencies, academic institutions, educational and cultural organizations, and corporations, that administer the Exchange Visitor Program.
			\item Au Pair program: \vspace{-0.1cm}
				\begin{itemize} \itemsep -1pt
				\item Through the Au Pair program, foreign nationals between 18 and 26 years of age participate in the home life of a host family. Au pairs provide limited childcare services for up to 12 months. An extension of 6, 9, or 12 months may be granted in certain cases.
				\item \url{http://exchanges.state.gov/jexchanges/programs/aupair.html}
				\end{itemize}
			\item Internships: \vspace{-0.1cm}
				\begin{itemize} \itemsep -1pt
				\item \url{http://exchanges.state.gov/jexchanges/programs/intern.html}
				\item Internship programs are designed to allow foreign professionals to come to the United States to gain exposure to U.S. culture and to receive training in U.S. business practices in their chosen occupational field.
				\item The maximum duration of an internship in any occupational field is 12 months.
				\item Upon completion of their exchange programs, participants are expected to return to their home countries.
				\item The State Department allows internships in the following occupational categories: \vspace{-0.1cm}
					\begin{itemize} \itemsep -1pt
					\item Agriculture, Forestry, and Fishing
					\item Arts and Culture
					\item Construction and Building Trades
					\item Education, Social Sciences, Library Science, Counseling and Social Services
					\item Health Related Occupations
					\item Hospitality and Tourism
					\item Information Media and Communications
					\item Management, Business, Commerce and Finance
					\item Public Administration and Law
					\item The Sciences, Engineering, Architecture, Mathematics, and Industrial Occupations.
					\end{itemize}
				\item An Intern must be a foreign national: \vspace{-0.1cm}
					\begin{itemize} \itemsep -1pt
					\item Who is currently enrolled in and pursuing studies at a foreign degree- or certificate-granting post-secondary academic institution outside the United States, or
					\item Who has graduated from such an institution no more than 12 months prior to his or her exchange visitor program start date.
					\end{itemize}
				\item Interns cannot work in unskilled or casual labor positions, in positions that require or involve child care or elder care, or in any kind of position that involves medical patient care or contact. Nor can interns work in positions that require more than 20 per cent clerical or office support work.
				\end{itemize}
			\item The Summer Work Travel Program: \vspace{-0.1cm}
				\begin{itemize} \itemsep -1pt
				\item \url{http://exchanges.state.gov/jexchanges/programs/swt.html}
				\item In the summer work travel program, post-secondary students may enter the United States to work and travel during their summer vacation.
				\item Participants can be admitted to the program more than once.
				\item The maximum length of the program is four months.
				\item Most of the time, participants work in unskilled service positions at resorts, hotels, restaurants, and amusement parks. However, they may also work in other types of organizations.
				\item For example, they could work in architectural firms, scientific research organizations, graphic art/publishing and other media communication businesses, advertising agencies, computer software and electronics firms, legal offices, etc.
				\item The program may not exceed four-months and must be finished during the student's summer vacation.
				\item Participants receive pay and benefits equal to an American working in the same or similar position.
				\end{itemize}
			\item Training programs: \vspace{-0.1cm}
				\begin{itemize} \itemsep -1pt
				\item \url{http://exchanges.state.gov/jexchanges/programs/trainee.html}
				\item Training programs are designed to allow foreign professionals to come to the United States to gain exposure to U.S. culture and to receive training in U.S. business practices in their chosen occupational field.
				\item Foreign nationals have had the opportunity to train with some of the finest employers in the U.S., gaining real time experience in their chosen career fields.
				\item Upon completion of their exchange programs, participants are expected to return to their home countries to utilize their newly learned skills and knowledge to advance their careers and share their experiences with their communities.
				\item The State Department allows training programs in the following occupational categories: \vspace{-0.1cm}
					\begin{itemize} \itemsep -1pt
					\item Agriculture, Forestry, and Fishing
					\item Arts and Culture
					\item Construction and Building Trades
					\item Education, Social Sciences, Library Science, Counseling and Social Services
					\item Health Related Occupations
					\item Hospitality and Tourism
					\item Information Media and Communications
					\item Management, Business, Commerce and Finance
					\item Public Administration and Law
					\item The Sciences, Engineering, Architecture, Mathematics, and Industrial Occupations.
					\end{itemize}
				\item A trainee must be a foreign national who has: \vspace{-0.1cm}
					\begin{itemize} \itemsep -1pt
					\item A degree or professional certificate from a foreign post-secondary academic institution and at least one year of prior related work experience in his or her occupational field outside the United States, or
					\item Five years of work experience outside the United States in the occupational field in which they are seeking training.
					\end{itemize}
				\end{itemize}
			\item Specialists: \vspace{-0.1cm}
				\begin{itemize} \itemsep -1pt
				\item \url{http://exchanges.state.gov/jexchanges/programs/specialist.html}
				\item This category is for a participant who is an expert in a field of specialized knowledge or skill who will demonstrate such skills in the United States. Such exchanges are to provide opportunities to increase the exchange knowledge and ideas between American and foreign specialists. The maximum duration of this program is one year.
				\item This category is for foreign nationals who are experts in a field of specialized knowledge or skill, coming to the United States for observing, consulting, or demonstrating their special skills, except: Professors and Research Scholars, Short-Term Scholars, and Alien Physicians.
				\item Individuals participating in the specialist program are: \vspace{-0.1cm}
					\begin{itemize} \itemsep -1pt
					\item Experts in a field of specialized knowledge or skill;
					\item Seeks to travel to the United States for the purpose of observing, consulting, or demonstrating their special knowledge or skills;
					\item Does not fill a permanent or long-term position of employment while in the U.S.
					\end{itemize}
				\end{itemize}
			\item International Visitor: \vspace{-0.1cm}
				\begin{itemize} \itemsep -1pt
				\item \url{http://exchanges.state.gov/jexchanges/programs/intl_visitor.html}
				\item The international visitor category enables visitors to better understand American culture and enhanced American knowledge of foreign cultures.
				\item This category is for individuals who are recognized as potential leaders in their own country, selected by the Department of State to participate in observation tours, discussions, consultation, professional meetings, conferences, workshops and travel.
				\item The maximum duration of the program is one year.
				\end{itemize}
			\item Alien Physician: \vspace{-0.1cm}
				\begin{itemize} \itemsep -1pt
				\item \url{http://exchanges.state.gov/jexchanges/programs/physician.html}
				\item The Alien Physician program is for foreign national physicians seeking entry into U.S. graduate medical education programs or training at accredited U.S. schools of medicine or other U.S. institutions.
				\item There are generally two types of exchange programs in which a foreign national physician (also referred to as a foreign/international medical graduate) participates: \vspace{-0.1cm}
					\begin{itemize} \itemsep -1pt
					\item Clinical training in the �alien physician� category
					\item Non-Clinical training in the �research scholar� category
					\end{itemize}
				\end{itemize}
			\item FORTUNE/U.S. State Department Global Women's Mentoring Partnership: \vspace{-0.1cm}
				\begin{itemize} \itemsep -1pt
				\item \url{http://exchanges.state.gov/citizens/professionals/fortunepartnership.html}
				\item This public-private partnership places talented, emerging women leaders from all over the world in mentoring programs with FORTUNE's Most Powerful Women Leaders.
				\item For three weeks, American and international participants work together in mentoring relationships to share the skills and experiences necessary for strengthening women�s leadership.
				\end{itemize}
			\item American Council of Young Political Leaders (ACYPL): \vspace{-0.1cm}
				\begin{itemize} \itemsep -1pt
				\item \url{http://exchanges.state.gov/citizens/profs/acypl.html}
				\item \url{http://www.acypl.org/}
				\item For 44 years, the American Council of Young Political Leaders (ACYPL) has designed, organized and managed unique international educational exchanges for young political leaders (ages 25-40) worldwide.
				\item ACYPL programs are designed to promote mutual understanding, respect, and friendship and to cultivate long-lasting relationships among young people who are poised to become tomorrow's global leaders and policy makers.
				\item American participants are nominated by members of Congress, governors, political party leaders, and ACYPL alumni, while international delegates are selected from countries where ACYPL is currently conducting programs by international program partners with the U.S. Embassy input.
				\end{itemize}
			\item Edward R. Murrow Program for Journalists: \vspace{-0.1cm}
				\begin{itemize} \itemsep -1pt
				\item \url{http://exchanges.state.gov/ivlp/murrow.html}
				\item The Edward R. Murrow Program for Journalists invites rising international journalists to travel to the United States and examine journalistic principles and practices.
				\end{itemize}
			\end{enumerate}
		\item Office of Citizen Exchanges: \vspace{-0.1cm}
			\begin{enumerate} \itemsep -1pt
			\item Youth Programs Division: \vspace{-0.1cm}
				\begin{itemize} \itemsep -1pt
				\item \url{http://exchanges.state.gov/youth/index.html}
				\item The Youth Programs Division is committed to empowering the successor generation and establishing long-lasting ties between the United States and other countries through exchange programs and institutional partnerships.
				\item Programs focus primarily on secondary schools and promote mutual understanding, leadership development, educational transformation, and democratic ideals.
				\item Year-Long Programs, Short Term Programs, and Virtual Partnerships: \url{http://exchanges.state.gov/youth/programs-by-type.html}
				\item Programs for Young Americans, and Programs for International Students and Teachers: \url{http://exchanges.state.gov/youth/programs-by-participants.html}
				\item Opportunities for American Hosts: Families and Schools, \url{http://exchanges.state.gov/youth/opps-for-am-hosts.html}
				\item Programs for High School Students: \url{http://exchanges.state.gov/youth/programs.html}
				\end{itemize}
			\item Professional Exchanges Division: \vspace{-0.1cm}
				\begin{itemize} \itemsep -1pt
				\item \url{http://exchanges.state.gov/citizens/profs.html}
				\item The Professional Exchanges division provides grants to U.S. nonprofit organizations to carry out exchange programs that support the professional development of foreign participants. The purpose of each exchange program is to engage with foreign leaders in critical professions, to demonstrate respect for foreign cultures, and to promote mutual understanding between the people of the United States and other countries.
				\item Professional exchanges typically last several years and include internships, study tours or workshops in the United States and in the host country. Participants come from a variety of professions including education administrators, public servants, journalists, labor union officials, entrepreneurs, environmental leaders, jurists, lawyers, and civic leaders.
				\item ECA grant opportunities: \vspace{-0.1cm}
					\begin{itemize} \itemsep -1pt
					\item Open Funding Opportunities: Requests For Grant Proposals (RFGPs), \url{http://exchanges.state.gov/grants/open2.html}
					\item Grants.gov: \url{http://www.grants.gov/}
					\end{itemize}
				\item Grants by Region: \vspace{-0.1cm}
					\begin{itemize} \itemsep -1pt
					\item \url{http://exchanges.state.gov/citizens/professionals/grant-region.html}
					\item Africa 
					\item East Asia and the Pacific 
					\item Europe and Eurasia 
					\item North Africa and the Middle East 
					\item South and Central Asia 
					\item Western Hemisphere 
					\item Multi-regional
					\end{itemize}
				\end{itemize}
			\end{enumerate}
		\end{enumerate}
	\end{enumerate}
\end{enumerate}






%%%%%%%%%%%%%%%%%%%%%%%%%%%%%%%%%%%%%%%%%%%
\section{Resources on Studying Abroad}
\label{resourcesonstudyingabroad}

Resources on studying abroad: \vspace{-0.3cm}
\begin{enumerate} \itemsep -4pt
\item Council on International Educational Exchange (CIEE): \vspace{-0.3cm}
	\begin{enumerate} \itemsep -2pt
	\item Study abroad programs for high school students from the United States: \vspace{-0.2cm}
		\begin{enumerate} \itemsep -2pt
		\item \url{http://www.ciee.org/hsabroad/index.html}
		\item \url{http://www.ciee.org/hsabroad/high-school-study-abroad/index.html}
		\item These programs include:: \vspace{-0.1cm}
			\begin{enumerate} \itemsep -1pt
			\item High School Abroad programs (for U.S. high school students)
			\item Summer High School Abroad programs (for U.S. high school students)
			\item Gap Year Abroad programs (for recent U.S. high school graduates)
			\end{enumerate}
		\end{enumerate}
	\end{enumerate}
\item U.S. Department of State: \vspace{-0.3cm}
	\begin{enumerate} \itemsep -2pt
	\item Bureau of Educational and Cultural Affairs: \vspace{-0.2cm}
		\begin{enumerate} \itemsep -2pt
		\item Office of Global Educational Programs: \vspace{-0.1cm}
			\begin{enumerate} \itemsep -1pt
			\item EducationUSA: \vspace{-0.1cm}
				\begin{itemize} \itemsep -1pt
				\item EducationUSA is a network of more than 400 student advising centers, which offer accurate, comprehensive, objective and timely information about educational opportunities in the United States and guidance to qualified individuals on how best to access those opportunities. This includes information about application procedures, standardized test requirements, student visas, financial aid, and the full range of accredited U.S. higher education institutions.
				\item \url{http://exchanges.state.gov/globalexchanges/index/educationusa.html}
				\item \url{http://www.educationusa.state.gov/} and \url{http://www.educationusa.info/centers.php}
				\end{itemize}
			\item Open Doors: \vspace{-0.1cm}
				\begin{itemize} \itemsep -1pt
				\item The Educational Information and Resources Branch funds Open Doors, a census of foreign students and scholars in the U.S. and of U.S. students studying abroad published annually by the Institute for International Education.
				\item Open Doors data is used by U.S. embassies, the Departments of State, Commerce, and Education, and U.S. colleges and universities to inform policy decisions about educational exchanges, trade in educational services, and study abroad activity.
				\item \url{http://exchanges.state.gov/globalexchanges/index/open_doors.html}
				\item \url{http://www.opendoors.iienetwork.org/}
				\end{itemize}
			\end{enumerate}
		\item EducationUSA: \vspace{-0.1cm}
			\begin{enumerate} \itemsep -1pt
			\item \url{http://educationusa.state.gov/}
			\item For U.S. (college) students who want to study/work abroad: \url{http://www.educationusa.info/pages/students/forus.php}
			\end{enumerate}
		\end{enumerate}
	\end{enumerate}
\item IES Abroad (formerly Institute of European Studies / Institute for the International Education of Students): \vspace{-0.3cm}
	\begin{enumerate} \itemsep -2pt
	\item \url{https://www.iesabroad.org/} and \url{https://www.iesabroad.org/IES/home.html}
	\end{enumerate}
\item Global Learning Semesters, Inc.: \vspace{-0.3cm}
	\begin{enumerate} \itemsep -2pt
	\item Summer in the Mediterranean: \vspace{-0.2cm}
		\begin{enumerate} \itemsep -2pt
		\item \url{http://www.globalsemesters.com/Mediterranean.html}
		\item Has programs in the following areas: \vspace{-0.1cm}
			\begin{enumerate} \itemsep -1pt
			\item Art \& Photography
			\item Early Christianity
			\item Greek Heritage
			\item International Marketing
			\item Music
			\end{enumerate}
		\end{enumerate}
	\end{enumerate}
\item American Institute For Foreign Study (AIFS): \vspace{-0.3cm}
	\begin{enumerate} \itemsep -2pt
	\item \url{http://www.aifs.com/}
	\item College Study Abroad: \url{http://www.aifsabroad.com/}
	\item For high school students: \vspace{-0.2cm}
		\begin{enumerate} \itemsep -2pt
		\item Gifted Education: \url{http://www.aifs.com/gifted_education.asp}
		\item High School Study and Travel: \url{http://www.aifs.com/highschool_study_travel.asp}
		\item Academic Year in America (AYA): \url{http://www.academicyear.org/?source=AIFS}
		\end{enumerate}
	\end{enumerate}
\end{enumerate}





%%%%%%%%%%%%%%%%%%%%%%%%%%%%%%%%%%%%%%%%%%%
\section{College Preparation}
\label{collegepreparation}

College preparation: \vspace{-0.3cm}
\begin{enumerate} \itemsep -4pt
\item {\it Guide to Online Schools} [or {\it GuideToOnlineSchools.com}], {\it The Top 53 College Preparation Resources for Students}. Available at: \url{http://www.guidetoonlineschools.com/tips-and-tools/college-prep-resources}; last accessed on August 25, 2010.
\item U.S. Department of Education's resources for parents to help their children learn: \url{http://www2.ed.gov/parents/academic/help/hyc.html} and \url{http://www2.ed.gov/parents/academic/help/homework/index.html}
\item The College Board: \vspace{-0.3cm}
	\begin{enumerate} \itemsep -2pt
	\item Information about SATs, college preparation, and financial aid
	\item {\it Trends in Higher Education} series 201X: \url{http://trends.collegeboard.org/}
	\item \url{http://www.collegeboard.com/}
	\end{enumerate}
\item {\it Accreditation.org}: \vspace{-0.3cm}
	\begin{enumerate} \itemsep -2pt
	\item Information about the accreditation of engineering degree programs around the world
	\item \url{http://www.accreditation.org/}
	\end{enumerate}
\item {\it New York Times}: \vspace{-0.3cm}
	\begin{enumerate} \itemsep -2pt
	\item The Learning Network: \url{http://learning.blogs.nytimes.com/category/test-yourself/}
	\item New York Times Magazine: \vspace{-0.2cm}
		\begin{enumerate} \itemsep -2pt
		\item The Sep 20, 2010 issue has many articles covering how technology can be used to improve education in K-12 programs. Available online at: \url{http://www.nytimes.com/indexes/2010/09/19/magazine/index.html?ref=magazine}; last accessed on September 20, 2010.
		\item ``New York Times Magazine Features Technology in Education,'' in {\it CCC Blog}, Computing Community Consortium (CCC), Computing Research Association (CRA), Sep 20, 2010. Available online at: \url{http://www.cccblog.org/2010/09/20/new-york-times-magazine-features-technology-in-education/}; last accessed on September 20, 2010.
		\item Articles in this issue discuss: \vspace{-0.1cm}
			\begin{enumerate} \itemsep -1pt
			\item How journalists can make use of technology to automate certain tasks, and improve their productivity and effectiveness in covering news stories
			\item How children can create computer games that introduces them to careers in computing and helps them to develop skills in computational thinking
			\item How to learn things without a lot of rote learning, to have fun while learning, and to use technology to make learning more fun
			\end{enumerate}
		\end{enumerate}
	\end{enumerate}
\item University of Southern California, USC: \vspace{-0.3cm}
	\begin{enumerate} \itemsep -2pt
	\item USC Office of Continuing Education and Summer Programs: \vspace{-0.2cm}
		\begin{enumerate} \itemsep -2pt
		\item \url{http://cesp.usc.edu/}
		\item These programs allow students in K-12 to earn credit at USC, and exposes them to different majors/professions, like medicine, engineering, creative writing, or film making.
		\item Students can benefit from these programs, and learn about different academic disciplines before applying to college. This would help them in their college applications.
		\item Underrepresented minority students can get scholarships to attend these programs. So, if parents have financial difficulty paying for the programs, they can seek financial aid for this.
		\item Also, current undergraduates can also sign up for programs to learn about marketing, finance, and entrepreneurship. They can also do summer research with USC researchers.
		\end{enumerate}
	\item Summer sports programs for youths: \vspace{-0.2cm}
		\begin{enumerate} \itemsep -2pt
		\item SC Futbol Academy (USC Soccer Camps): \url{http://www.usctrojans.com/sports/w-soccer/spec-rel/021610aaa.html}
		\item Mick Haley's USC Girls Volleyball Camp: \url{http://www.usctrojans.com/sports/w-volley/spec-rel/volley-camp.html}
		\item Salo Swim Camp: \url{http://www.saloswimcamp.com/on-line/default.asp}
		\item USC NYSP Trojan KidSCamp: \url{http://sait.usc.edu/recsports/site_content/youth_sports/nysptk.html}
		\item After School Sports Connection, ASSC (operates in fall, spring, and summer): \url{http://sait.usc.edu/recsports/site_content/youth_sports/assc.html}
		\end{enumerate}
	\end{enumerate}
\item Telluride Association: \vspace{-0.3cm}
	\begin{enumerate} \itemsep -2pt
	\item Telluride Association Summer Program (TASP) [ for high school students ]: \url{http://www.tellurideassociation.org/programs/high_school_students/tasp/tasp_general_info.html}
	\item Telluride Association Sophomore Seminar (TASS) [ for high school students ]: \url{http://www.tellurideassociation.org/programs/high_school_students/tass/tass_general_info.html}
	\item Resources for high school educators to nominate summer program applicants: \url{http://www.tellurideassociation.org/programs/high_school_students/hs_resources/hs_resources_general_information.html}
	\end{enumerate}
\item MathNerds: \vspace{-0.3cm}
	\begin{enumerate} \itemsep -2pt
	\item \url{http://www.mathnerds.com/}
	\item ``Provides free, discovery-based, mathematical guidance via an international, volunteer network of mathematicians.''
	\item If you have a mathematical problem to solve, you can ask mathematicans at {\it MathNerds} for help.
	\item They would require you to discuss your attempted approaches/solutions.
	\item If you have not made attempts to solve the problem, they will not give you much guidance.
	\item In addition, they cannot solve problems for you.
	\item They provide guidance for mathematical problems from K-12 material through undergraduate mathematics and statistics classes.
	\item They also provide help for selected topics in advanced mathematics classes (for graduate students).
	\item Other resources: \url{http://www.mathnerds.com/links/links.aspx}
	\end{enumerate}
\item Hobsons: \vspace{-0.3cm}
	\begin{enumerate} \itemsep -2pt
	\item CollegeView (Hobsons' college recruiting services): \url{http://www.collegeview.com/index.jsp}
	\end{enumerate}
\item Sponsors for Educational Opportunity (SEO): \vspace{-0.3cm}
	\begin{enumerate} \itemsep -2pt
	\item Resources: \url{http://www.seo-usa.org/ScholarsResources}
	\end{enumerate}
\item U.S. Department of Education: \vspace{-0.3cm}
	\begin{enumerate} \itemsep -2pt
	\item Students.gov: \url{http://www.students.gov/STUGOVWebApp/index.jsp}
	\item college.gov: \url{http://www.college.gov/wps/portal}
	\end{enumerate}
\item U.S. Department of State: \vspace{-0.3cm}
	\begin{enumerate} \itemsep -2pt
	\item Bureau of Educational and Cultural Affairs: \vspace{-0.2cm}
		\begin{enumerate} \itemsep -2pt
		\item EducationUSA: \vspace{-0.1cm}
			\begin{enumerate} \itemsep -1pt
			\item Information for international students: \url{http://www.educationusa.info/students.php}
			\end{enumerate}
		\end{enumerate}
	\end{enumerate}
\item Congressional Hispanic Caucus Institute (CHCI): \vspace{-0.3cm}
	\begin{enumerate} \itemsep -2pt
	\item CHCI Education Center: \vspace{-0.2cm}
		\begin{enumerate} \itemsep -2pt
		\item \url{http://www.chci.org/education_center/}
		\item Has resources on college planning, financial aid, scholarships, college internships, and housing.
		\item For Parents: \url{http://www.chci.org/education_center/page/for-parents}
		\item For Students: \url{http://www.chci.org/education_center/page/for-students}
		\end{enumerate}
	\end{enumerate}
\item My College Options: \vspace{-0.3cm}
	\begin{enumerate} \itemsep -2pt
	\item \url{http://www.mycollegeoptions.org/}
	\item ``My College Options is a FREE college planning service, offering assistance to students, parents, high schools, counselors, and teachers nationwide.''
	\item ``It is designed to assist high school students in exploring a wide range of post-secondary opportunities, with special emphasis on the college search process.''
	\end{enumerate}
\end{enumerate}

Resources for financial aid: \vspace{-0.3cm}
\begin{enumerate} \itemsep -4pt
\item {\it Guide to Online Schools} [or {\it GuideToOnlineSchools.com}], {\it Financial Aid}. Available at: \url{http://www.guidetoonlineschools.com/financial-aid}; last accessed on August 25, 2010.
\item The Institute for College Access \& Success, {\it Links} [ Resources that provide information about student loans and student debt ]. Available at: \url{http://projectonstudentdebt.org/links.vp.html}; last accessed on September 4, 2010. [ Also, see \url{http://projectonstudentdebt.org/advice.vp.html} and \url{http://ticas.org/about.vp.html}. ]
\end{enumerate}


Information about colleges and universities: \vspace{-0.3cm}
\begin{enumerate} \itemsep -4pt
\item The Institute for College Access \& Success, {\it College InSight}. Available at: \url{http://college-insight.org/}; last accessed on September 4, 2010.
\item 
\end{enumerate}



%%%%%%%%%%%%%%%%%%%%%%%%%%%%%%%%%%%%%%%%%%%
\section{Outreach for Students in Colleges and Universities}
\label{outreachcollege}

Resources to reach out to students in colleges and universities: \vspace{-0.3cm}
\begin{enumerate} \itemsep -4pt
%%%%%%%%%%%%%%%%%%%%%%%%%%%%%
\item Film contests: \vspace{-0.3cm}
	\begin{enumerate} \itemsep -2pt
	\item Ed Wood Film Festival [@ USC]: \vspace{-0.2cm}
		\begin{enumerate} \itemsep -2pt
		\item Celebrating independent filmmaking at USC and named for the famous director, the Ed Wood Film Festival is put on by a committee of Residential Education staff members at New Residential College, chaired by the Cinema Floor RA's.
		\item Teams of students come together to obtain the year's secret theme in which to write, shoot, and edit their very own short film within 24 hours. A week later, the films are shown at USC's Norris Cinema and a panel of judges selects the Festival winners in a variety of categories.
		\item \url{http://sait.usc.edu/resed/Programs.aspx}
		\end{enumerate}
	\item Reel LA: Parkside International Film Festival [or USC Reel LA Film Festival at USC]; see \url{http://www-scf.usc.edu/~pirc/areagov/government.php}
	\end{enumerate}
%%%%%%%%%%%%%%%%%%%%%%%%%%%%%
\item residential education: \vspace{-0.3cm}
	\begin{enumerate} \itemsep -2pt
	\item Telluride Association: \vspace{-0.2cm}
		\begin{enumerate} \itemsep -2pt
		\item Information about how to reside at the Cornell Branch (also known as Telluride House or CBTA) and the Michigan Branch of Telluride Association, which are ``residential colleges'': \url{http://www.tellurideassociation.org/programs/university_students.html}
		\item Awards for residents at the Cornell or Michigan Branch: \url{http://www.tellurideassociation.org/programs/university_students/us_awards.html}
		\end{enumerate}
	\end{enumerate}
%%%%%%%%%%%%%%%%%%%%%%%%%%%%%
\item MathNerds: \vspace{-0.3cm}
	\begin{enumerate} \itemsep -2pt
	\item \url{http://www.mathnerds.com/}
	\item ``Provides free, discovery-based, mathematical guidance via an international, volunteer network of mathematicians.''
	\item If you have a mathematical problem to solve, you can ask mathematicans at {\it MathNerds} for help.
	\item They would require you to discuss your attempted approaches/solutions.
	\item If you have not made attempts to solve the problem, they will not give you much guidance.
	\item In addition, they cannot solve problems for you.
	\item They provide guidance for mathematical problems from K-12 material through undergraduate mathematics and statistics classes.
	\item They also provide help for selected topics in advanced mathematics classes (for graduate students).
	\end{enumerate}
%%%%%%%%%%%%%%%%%%%%%%%%%%%%%
\item Invent Now: \vspace{-0.3cm}
	\begin{enumerate} \itemsep -2pt
	\item 
	\end{enumerate}
\item Journal of Young Investigators (JYI): \vspace{-0.3cm}
	\begin{enumerate} \itemsep -2pt
	\item \url{http://www.jyi.org/}
	\item ``peer-reviewed journal for undergraduates''
	\item ``JYI's web journal (which is also called JYI) is dedicated to the presentation of undergraduate research in science, mathematics, and engineering. It publishes the best submissions from undergraduates, with an emphasis on both the quality of research and the manner in which it is communicated. The journal, JYI, also allows students to experience the other side of the scientific publication process: the review process. Students working with their faculty advisors review the work of their peers and determine whether that work is acceptable for publication in JYI.''
	\end{enumerate}
\item The Recording Academy: \vspace{-0.3cm}
	\begin{enumerate} \itemsep -2pt
	\item GRAMMY U: \vspace{-0.2cm}
		\begin{enumerate} \itemsep -2pt
		\item \url{http://www.grammy365.com/grammy-u}
		\item GRAMMY U is a unique and fast-growing community of full-time college students, primarily between the ages of 17 and 25,  who are pursuing a career in the recording industry.
		\item The Recording Academy created GRAMMY U to help prepare college students for their careers in the music industry through networking, educational programs and performance opportunities.
		\item GRAMMY U is designed to enhance students' current academic curriculum with access to recording industry professionals to give an ``out of classroom'' perspective on the recording industry.
		\end{enumerate}
	\end{enumerate}
%%%%%%%%%%%%%%%%%%%%%%%%%%%%%
\item --- --- --- --- --- --- --- --- --- --- --- --- --- --- --- --- --- --- --- --- --- --- --- --- --- --- --- --- --- --- ---
\item \colorbox{blue}{\bf Help for Underrepresented Minorities}
% Help for Underrepresented Minorities
\item INROADS, Inc.: \vspace{-0.3cm}
	\begin{enumerate} \itemsep -2pt
	\item Internships: \url{http://www.inroads.org/interns/internWhatItTakes.jsp}
	\end{enumerate}
\item The PhD Project: \vspace{-0.3cm}
	\begin{enumerate} \itemsep -2pt
	\item \url{http://www.phdproject.org/index.html}
	\item Program and informational network to encourage ``African-Americans, Hispanic-Americans and Native Americans'' to pursue Ph.D. programs in business and seek careers in academia.
	\item Annual PhD Project Conference: \vspace{-0.2cm}
		\begin{enumerate} \itemsep -2pt
		\item Conference: \vspace{-0.1cm}
			\begin{enumerate} \itemsep -1pt
			\item \url{http://www.phdproject.org/conference.html}
			\item \url{http://www.phdproject.org/conference_application.html}
			\item For prospective Ph.D. students in business to learn more about Ph.D. programs in business, the Ph.D. application process, and life in graduate school.
			\item Registration Policy: \vspace{-0.1cm}
				\begin{itemize} \itemsep -1pt
				\item If you are selected to attend the conference you will be required to pay a \$200 registration fee which can be processed via credit card during the registration process. All travel and conferences expenses will paid by The PhD Project (total conference expenses for hotel, meals, materials, and transportation are valued at approximately \$1,500 per invited attendee.) Your investment of the \$200 registration fee will be refunded if you enter a full-time, AACSB accredited business doctoral program within 3 years of attending the conference. 
				\item If you previously attended a PhD Project Conference, you may submit an application to be reviewed, however if you are selected to attend, The PhD Project will only cover hotel costs (shared room with another participant). You will be required to pay the registration and travel costs
				\end{itemize}
			\end{enumerate}
		\item Resources for Potential/Current Doctoral Students: \vspace{-0.1cm}
			\begin{enumerate} \itemsep -1pt
			\item \url{http://www.phdproject.org/resources.html}
			\item Information about good business schools that offer Ph.D. programs, preparation for the GMAT, and the life in graduate school as a Ph.D. student.
			\item Suggested Reading: \vspace{-0.1cm}
				\begin{itemize} \itemsep -1pt
				\item \url{http://www.phdproject.org/reading.html}
				\item Has information life in graduate school as a Ph.D. student, racial diversity/issues in higher education, job searching in academia, and work-life balance for female Ph.D. students.
				\end{itemize}
			\end{enumerate}
		\item The PhD Project Doctoral Student Association (DSA): \vspace{-0.1cm}
			\begin{enumerate} \itemsep -1pt
			\item The PhD Project network: \vspace{-0.1cm}
				\begin{itemize} \itemsep -1pt
				\item \url{http://www.myphdnetwork.org/}
				\item ``There are 5 discipline specific associations covering the major areas of business education: Accounting, Finance, Information Systems, Management, Marketing.''
				\end{itemize}
			\end{enumerate}
		\end{enumerate}
	\end{enumerate}
\item MS-to-Ph.D. program for underrepresented minorities at Fisk and Vanderbilt in certain areas of
science (including astronomy, material science, and physics)
\item Outreach programs for underrepresented minorities to help them get into medical (and/or graduate) schools. Search for ``PREP (Post-baccalaureate Research Education Programs),'' which have stipends. E.g., Georgetown University School of Medicine, and George Washington University's medical school
\item New York University: \vspace{-0.3cm}
	\begin{enumerate} \itemsep -2pt
	\item Leonard N. Stern School of Business: \vspace{-0.2cm}
		\begin{enumerate} \itemsep -2pt
		\item Stern Pre-Doctoral program: \url{http://www.stern.nyu.edu/AcademicPrograms/PhD/Pre-Doctoral/index.htm}
		\end{enumerate}
	\end{enumerate}
\end{enumerate}



%%%%%%%%%%%%%%%%%%%%%%%%%%%%%%%%%%%%%%%%%%%
\section{Science \& Engineering Outreach}
\label{stemoutreach}

%%%%%%%%%%%%%%%%%%%%%%%%%%%%%%%%%%%%%%%%%%%
\subsection{Precollege Science \& Engineering Outreach}
\label{stemoutreachk12}

Science and engineering outreach to high-school (and middle-school) students, and their parents, teachers, and career counselors: \vspace{-0.3cm}
\begin{enumerate} \itemsep -4pt
\item {\it MentorNet}: \vspace{-0.3cm}
	\begin{enumerate} \itemsep -2pt
	\item \url{http://www.mentornet.net/}
	\item Enables people to network with scientists, engineers, and professors in Science, Technology, Engineering, and Mathematics (STEM)
	\item Is very supportive of minorities, so that more minorities (particularly underrepresented minorities) can be attracted to STEM careers
	\end{enumerate}
\item International Science Olympiad (for high school students): \vspace{-0.3cm}
	\begin{enumerate} \itemsep -2pt
	\item International Olympiad in Informatics: \url{http://en.wikipedia.org/wiki/International_Olympiad_in_Informatics} and \url{http://www.ioinformatics.org/index.shtml}
	\item International Mathematical Olympiad: \url{http://www.imo-official.org/}
	\item International Physics Olympiad: \url{http://www.jyu.fi/tdk/kastdk/olympiads/}
	\item International Chemistry Olympiad: \url{http://www.icho.sk/}
	\item International Biology Olympiad: \url{http://www.ibo-info.org/}
	\item \url{http://scienceolympiads.org/}
	\end{enumerate}
\item International Astronomy Olympiad: \url{http://www.issp.ac.ru/iao/}
\item International Earth Science Olympiad: \url{http://en.wikipedia.org/wiki/International_Earth_Science_Olympiad}
\item International Junior Science Olympiad (for students younger than 15 years old): \url{http://www.ijso-official.org/home}
\item Teen Leadership Institute Science, Technology, Engineering, and Math (STEM) programs @ YWCA Greater Pittsburgh; see \url{http://www.ywcapgh.org/STEM_Programs.asp}
\item For Inspiration and Recognition of Science and Technology (FIRST): \url{http://www.usfirst.org/} (including resources and guides to mentoring); scholarships @ \url{http://www.usfirst.org/aboutus/content.aspx?id=508}; and robotics programs @ \url{http://www.usfirst.org/roboticsprograms/frc/default.aspx?=966}
\item Mac Hyman, ``Good Choices for Great Careers in the Mathematical Sciences,'' talk given at 2008 SIAM Annual Meeting. Available at: \url{http://client.blueskybroadcast.com/siam08/hyman/index.html}; last accessed on August 25, 2010. Also, see \url{http://meetings.siam.org/program.cfm?CONFCODE=AN08}, \url{http://www.siam.org/meetings/an08/program.php}, and \url{http://www.siam.org/meetings/an08/}.
\item {\it RoboCup}\texttrademark\ competitions: \vspace{-0.2cm}
	\begin{enumerate} \itemsep -2pt
	\item Junior category for K-12 students involves contests the these areas of challenges: \vspace{-0.1cm}
		\begin{enumerate} \itemsep -1pt
		\item soccer
		\item dance
		\item rescue operations
		\end{enumerate}
	\item \url{http://www.robocup.org/}
	\end{enumerate}
\item {\it Curriki}, which is an online educational resource for teachers, students, and parents in K-12: \url{http://www.curriki.org/xwiki/bin/view/Main/About}
%%%%%%%%%%%%%%%%%%%%%%%%%%%%%%%%%%%%%%%%
%%%%%%%%%%%%%%%%%%%%%%%%%%%%%%%%%%%%%%%%
\item Electrical and computer engineering and/or computer science: \vspace{-0.2cm}
	\begin{enumerate} \itemsep -2pt
	\item {\it TopCoder} coding and design contests: \vspace{-0.2cm}
		\begin{enumerate} \itemsep -2pt
		\item High School category
		\item \url{http://www.topcoder.com/}
		\end{enumerate}
	\item Student Cluster Competition (SCC): \vspace{-0.2cm}
		\begin{enumerate} \itemsep -2pt
		\item SCC is held at each (annual) SC conference, which is the International Conference for High Performance Computing, Networking, Storage, and Analysis. IEEE Computer Society and the Association for Computing Machinery are the sponsors for this conference.
		\item During SC10, teams consisting of six students, undergraduate and/or high school, will showcase the amazing power of clusters and the ability to utilize open source software to solve interesting and important problems. They will compete in real-time on the exhibit floor to run a workload of real-world applications on clusters of their own design while never exceeding the dictated power limit.
		\item During SC10 in New Orleans, teams will assemble, test and tune their machines and run the HPCC benchmarks until the starting bell rings on Monday night at the Exhibit Opening Gala where they will be given the competition data sets. In full view of conference attendees, teams will execute the prescribed workload while showing progress and science visualization output on large high-resolution displays in their areas. Teams race to correctly complete the greatest number of application runs during the competition period until the close of the exhibit floor on Wednesday evening.
		\item \url{http://sc10.supercomputing.org/?pg=studentcluster.html}
		\end{enumerate}
	\item Institute of Electrical and Electronics Engineers, IEEE: \vspace{-0.3cm}
		\begin{enumerate} \itemsep -2pt
		\item {\it IEEE Educational Activities} recommended resources: \url{http://www.ieee.org/education_careers/education/preuniversity/resources/index.html}
		\item Engineering Projects in Community Service (EPICS) in IEEE: \vspace{-0.2cm}
			\begin{enumerate} \itemsep -2pt
			\item High school students collaborate with college students in engineering projects to benefit the community
			\item \url{http://www.ieee.org/education_careers/education/preuniversity/epics_high.html}
			\end{enumerate}
		\item Talk given by John Cohn at the IEEE International Symposium on Circuits and Systems (ISCAS), May 18-21, 2008. The talk is titled, ``Kids these days. How we can inspire the next generation of Engineers and Scientists?'' See \url{http://ewh.ieee.org/soc/icss/IEEE-ISCAS-08-Tue-Keynote-JC/IEEE-ISCAS-08-Tue-Keynote-JC.HTML}. [ Alternatively, go to: IEEE Circuits and Systems Society, \url{http://www.ieee-cas.org/}: Select the ``Resources'' tab in the menu bar, and select the ``ISCAS Keynote Videos'' option. Click on the video link with the appropriate title. ]
		\end{enumerate}
	\item Association for Computing Machinery (ACM): \vspace{-0.2cm}
		\begin{enumerate} \itemsep -2pt
		\item Sanjeev Arora, Boaz Barak, and Luca Trevisan, ``Survey Papers and Essays,'' in {\it Theory Matters Wiki: Theoretical Computer Science (TCS) Advocacy Wiki}, SIGACT Committee for the Advancement of Theoretical Computer Science, ACM Special Interest Group on Algorithms and Computation Theory (SIGACT), Association for Computing Machinery, February 25, 2010. Available at: \url{http://theorymatters.org/pmwiki/pmwiki.php?n=Main.SurveyCollection}; last accessed on September 14, 2010.
		\end{enumerate}
	\item WGBH Educational Foundation: \vspace{-0.2cm}
		\begin{enumerate} \itemsep -2pt
		\item Dot Diva / New Image for Computing (NIC) initiative: \vspace{-0.1cm}
			\begin{enumerate} \itemsep -1pt
			\item \url{http://dotdiva.org/}
			\item Resource for parents and teachers: \url{http://dotdiva.org/parents.html}
			\end{enumerate}
		\end{enumerate}
	\item Silicon Valley StRUT: \vspace{-0.2cm}
		\begin{enumerate} \itemsep -2pt
		\item Students Recycling Used Technology, StRUT, Competition; StRUT Competition consists of: \vspace{-0.1cm}
			\begin{enumerate} \itemsep -1pt
			\item Disassemble and Reassemble A Computer 
			\item Create and Present a Powerpoint Presentation 
			\item Computer Parts Identification and Challenge Test  
			\item Team Quiz Bowl on Computer Technology and Related Subjects
			\item \url{http://www.svstrut.org/cms/content/section/1/5/}
			\item Teacher Resources: \url{http://www.svstrut.org/cms/component/option,com_weblinks/catid,11/Itemid,10/}
			\item [ Resources to Support ] Curriculum for Engineering and Computer Technology Education: \url{http://www.svstrut.org/cms/content/view/8/18/}
			\end{enumerate}
		\item \url{http://www.svstrut.org/cms/}
		\end{enumerate}
	\item Google Code Jam (programming contest): \url{http://code.google.com/codejam/} and \url{http://en.wikipedia.org/wiki/Google_Code_Jam}
	\item University of Illinois at Urbana-Champaign (UIUC): \vspace{-0.2cm}
		\begin{enumerate} \itemsep -2pt
		\item College of Engineering; Department of Computer Science: \vspace{-0.1cm}
			\begin{enumerate} \itemsep -1pt
			\item Outreach \& Diversity: \url{http://cs.illinois.edu/outreach}
			\item ChicTech: \url{http://cs.illinois.edu/outreach/chictech}
			\item Technical Ambassadors: \url{http://cs.illinois.edu/outreach/tac}
			\item Games4Girls: \url{http://cs.illinois.edu/outreach/games4girls}
			\item Workshops \& Camps: \url{http://cs.illinois.edu/outreach/k12}
			\item \url{http://cs.illinois.edu/outreach}
			\end{enumerate}
		\end{enumerate}
	\item Carnegie Mellon University: \vspace{-0.2cm}
		\begin{enumerate} \itemsep -2pt
		\item women@SCS School of Computer Science, Carnegie Mellon University: \vspace{-0.1cm}
			\begin{enumerate} \itemsep -1pt
			\item Papers: \url{http://women.cs.cmu.edu/Resources/Papers/}
			\item Alumnae Interviews / Profiles: \url{http://women.cs.cmu.edu/Who/Alumnae/alumInterviews.php}
			\item Job and Research Opportunities: \url{http://www.women.cs.cmu.edu/Resources/JobsResearch/}
			\item Career Advice: \url{http://women.cs.cmu.edu/Resources/JobsResearch/careeradvice.php}
			\item Other Sites: \url{http://www.women.cs.cmu.edu/Miscellaneous/Other/}
			\end{enumerate}
		\end{enumerate}
	\item {\it Quora}: \vspace{-0.2cm}
		\begin{enumerate} \itemsep -2pt
		\item ``If a 10-year-old wanted to start programming today, what language path would be the most valuable moving forward?'' Available online at: \url{http://www.quora.com/If-a-10-year-old-wanted-to-start-programming-today-what-language-path-would-be-the-most-valuable-moving-forward}; last accessed on November 23, 2010.
		\end{enumerate}
	\end{enumerate}
%%%%%%%%%%%%%%%%%%%%%%%%%%%%%%%%%%%%%%%%
%%%%%%%%%%%%%%%%%%%%%%%%%%%%%%%%%%%%%%%%
\item Engineering Education Service Center (EESC): \vspace{-0.3cm}
	\begin{enumerate} \itemsep -2pt
	\item Has lists of: \vspace{-0.2cm}
		\begin{enumerate} \itemsep -2pt
		\item Educational material: \vspace{-0.1cm}
			\begin{enumerate} \itemsep -1pt
			\item books
			\item DVDs
			\item resource kits for teachers
			\end{enumerate}
		\item engineering camps (for the summer in the United States): \url{http://www.engineeringedu.com/camps/}
		\item {\it Women in Engineering} programs at US engineering schools: \url{http://www.engineeringedu.com/wie.html}
		\item US engineering schools: \url{http://www.engineeringedu.com/engrschools.htm}
		\item competitions for youths, including high school students: \url{http://www.engineeringedu.com/competitions.html}
		\item online resources
		\item list of professional organizations in engineering (or engineering societies): \url{http://www.engineeringedu.com/soc1.html}
		\item scholarships: \url{http://www.engineeringedu.com/scholars.html}
		\end{enumerate}
	\item It has resources for K-12 students, and their teachers and parents. It also has information for girls who are seeking careers in engineering; in addition, it provides their parents and teachers with information to guide the girls.
	\item It runs a workshop (in the US) for mother-daughter pairs to encourage girls to pursue careers in engineering.
	\item \url{http://www.engineeringedu.com/}
	\end{enumerate}
\item TryNano.org: \vspace{-0.3cm}
	\begin{enumerate} \itemsep -2pt
	\item Information about educational opportunities and careers in nanotechnology and nanoscience
	\item \url{TryNano.org}
	\end{enumerate}
\item {\it Mathematical Association of America} (MAA): \vspace{-0.3cm}
	\begin{enumerate} \itemsep -2pt
	\item Middle/High School Students: \url{http://www.maa.org/students/middle_high/}
	\item Parents: \url{http://www.maa.org/students/Parents.html}
	\item MAA American Mathematics Competitions: \vspace{-0.2cm}
		\begin{enumerate} \itemsep -2pt
		\item {\it Students} [resources]. Available at: \url{http://amc.maa.org/a-activities/a4-for-students/s-index.shtml}; last accessed on September 2, 2010.
		\item It includes tips to help students do well in math contests and Olympiads, a reading list for students interested in mathematics, problems from past math contests and Olympiads, and other resources from the World Wide Web.
		\end{enumerate}
	\item {\it Fun Math Sites}. Available at: \url{http://www.maa.org/students/funsites.html}; last accessed on September 2, 2010.
	\item Special Interest Group on Mathematics and the Arts (SIGMAA-ARTS): Resources, see \url{http://myweb.cwpost.liu.edu/aburns/sigmaa-arts/resources.html}.
	\item Special Interest Group of the MAA on Quantitative Literacy (SIGMAA QL): \url{http://sigmaa.maa.org/ql/}
	\end{enumerate}
\item eGFI (Engineering, Go For It!): \vspace{-0.3cm}
	\begin{enumerate} \itemsep -2pt
	\item Provides information for students, parents, and teachers about educational pathways and careers in engineering
	\item \url{http://egfi-k12.org/}
	\end{enumerate}
\item {\it Sloan Career Cornerstone Center}: \vspace{-0.3cm}
	\begin{enumerate} \itemsep -2pt
	\item Career exploration resources in STEM (science, technology, engineering, mathematics, computing, and healthcare)
	\item \url{http://www.careercornerstone.org/}
	\end{enumerate}
\item {\it TryEngineering}: \vspace{-0.3cm}
	\begin{enumerate} \itemsep -2pt
	\item Career exploration resources for engineering
	\item \url{http://www.tryengineering.org/}
	\end{enumerate}
\item {\it Junior Engineering Technical Society, JETS}: \vspace{-0.3cm}
	\begin{enumerate} \itemsep -2pt
	\item Career exploration resources for engineering
	\item \url{http://www.jets.org/}
	\end{enumerate}
\item {\it American Society of Mechanical Engineers, ASME}: \vspace{-0.3cm}
	\begin{enumerate} \itemsep -2pt
	\item K-12 Student Resources: \url{http://www.asme.org/Communities/Students/K12/} and \url{http://www.asme.org/Education/PreCollege/EngineeringResources/}
	\item Engineering Camps: \url{http://www.asme.org/Communities/Students/K12/Camps.cfm}
	\end{enumerate}
\item BESTRobotics, Inc.: \vspace{-0.3cm}
	\begin{enumerate} \itemsep -2pt
	\item BEST (Boosting Engineering, Science, and Technology) competition: \vspace{-0.2cm}
		\begin{enumerate} \itemsep -2pt
		\item \url{http://best.eng.auburn.edu/}
		\item Hosted at Auburn University's Samuel Ginn College of Engineering
		\item BEST World Championship: \url{http://best.eng.auburn.edu/world-championship/}
		\end{enumerate}
	\end{enumerate}
\item {\it American Society of Civil Engineers, ASCE}: \vspace{-0.3cm}
	\begin{enumerate} \itemsep -2pt
	\item Outreach resource for K-12 students, and their parents and teachers
	\item \url{http://content.asce.org/asceville/index.html}
	\end{enumerate}
\item {\it Science.gov} (USA.gov for Science): Internship and Fellowship Opportunities in Science (for high school students); see \url{http://www.science.gov/internships/k-12.html}
\item {\it iTunes U}: \vspace{-0.3cm}
	\begin{enumerate} \itemsep -2pt
	\item {\it iTunes} is required to listen to or watch these lectures, talks, and presentations.
	\item Access {\it iTunes U} at: \url{http://deimos3.apple.com/indigo/main/main.html?v0=WWW-AMUS-ITUNESU070521-N48LX}
	\item WGBH's Teachers' Domain -- Boston's PBS Station: Video presentation on ``Engineering for the Red Planet''; see \url{http://deimos3.apple.com/WebObjects/Core.woa/Browse/wgbh.org.1416254059.01416254061.1416793683?i=1951581658}. Also, check out its video clip on ``Carbon Fiber Car of the Future''.
	\item {\it iTunes U} is a set of webcast and podcasts, where you can easily find audio and video clips for lectures, seminars, announcements, virtual tours, and so on. For example, some professors from schools like MIT or Berkeley will provide audio/video clips of their lectures on {\it iTunes U}.
	\item This can help in exploring different majors during the college application process, or before a college student declares her/his major(s). If a student is not sure if she/he wants to double major in deaf studies and linguistics, this student can check out some linguistics lectures from her/his (preferred) college/university, if it uses {\it iTunes U}, or those from other universities.
	\end{enumerate}
\item High School Ace's College Prep Guide: \url{http://highschoolace.com/ace/colleges.cfm}
\item {\it Dr. Sally Ride} (America�s first woman in space): \vspace{-0.3cm}
	\begin{enumerate} \itemsep -2pt
	\item {\it Sally Ride Science}'s resources for educators: \url{https://www.sallyridescience.com/for_educators}
	\item Sally Ride Science Educator Institutes (to educate K-12 teachers about science): \url{https://www.sallyridescience.com/for_educators/institutes}
	\item {\it Sally Ride Science Academy} helps teachers to increase their students' interest in science: \url{https://www.sallyridescience.com/academy}
	\item {\it Sally Ride Science}'s resources for teachers: \url{https://www.sallyridescience.com/resources}
	\item {\it Sally Ride Science Festivals} are events for girls from the $5^{th}$ grade to the $8^{th}$ grade: \url{https://www.sallyridescience.com/festivals}
	\item {\it Sally Ride Science Camps} are summer camps for girls from the $4^{th}$ grade to the $9^{th}$ grade: \url{http://www.sallyridecamps.com/}
	\item GRAIL MoonKAM: \vspace{-0.2cm}
		\begin{enumerate} \itemsep -2pt
		\item ``GRAIL MoonKAM (Moon Knowledge Acquired by Middle school students) is GRAIL's signature education and public outreach program.''
		\item ``GRAIL MoonKAM will engage middle schools across the country in the GRAIL mission and lunar exploration.''
		\item \url{https://www.grailmoonkam.com/}
		\end{enumerate}
	\item EarthKAM: \vspace{-0.2cm}
		\begin{enumerate} \itemsep -2pt
		\item EarthKAM (Earth Knowledge Acquired by Middle school students) is a NASA educational outreach program enabling students, teachers and the public to learn about Earth from the unique perspective of space.
		\item \url{https://earthkam.ucsd.edu/}
		\end{enumerate}
	\end{enumerate}
\item Andrew Rader Studios: \vspace{-0.3cm}
	\begin{enumerate} \itemsep -2pt
	\item Chem4Kids.com: \url{http://www.chem4kids.com/}
	\end{enumerate}
\item {\it American Association for the Advancement of Science, AAAS}: \vspace{-0.3cm}
	\begin{enumerate} \itemsep -2pt
	\item ENTRY POINT! for Students With Disabilities (in STEM): \url{http://www.aaas.org/careercenter/fellowships/} and \url{http://ehrweb.aaas.org/entrypoint/}
	\item AAAS Mass Media Science \& Engineering Fellows Program (for STEM grad students to intern in mass media companies): \url{http://www.aaas.org/programs/education/MassMedia/}
	\item Diversity Issues: \url{http://sciencecareers.sciencemag.org/career_magazine/diversity_issues/}
	\item Internships involving science and journalism, human rights, scientific freedom, responsibility, or law: \url{http://www.aaas.org/careercenter/} and \url{http://www.aaas.org/careercenter/internships/scienceminority.shtml} (AAAS Minority Science Writers Internship)
	\item Kinetic City: \url{http://www.kineticcity.com/}
	\end{enumerate}
\item {\it NASA} resources for students: \url{http://www.nasa.gov/audience/forstudents/index.html} and \url{http://www.nasa.gov/offices/education/programs/national/summer/education_resources/index.html} (NASA Summer of Innovation)
\item National Academy of Engineering, NAE: \vspace{-0.3cm}
	\begin{enumerate} \itemsep -2pt
	\item NAE Grand Challenges: \vspace{-0.2cm}
		\begin{enumerate} \itemsep -2pt
		\item Includes a list of NAE Grand Challenges, which indicate some of the challenges faced by people on a global scale that can be partially solved by engineers. This can be used to get children and youths to be excited about engineering.
		\item NAE Grand Challenges: \vspace{-0.1cm}
			\begin{enumerate} \itemsep -1pt
			\item Make solar energy economical
			\item Provide energy from fusion
			\item Develop carbon sequestration methods
			\item Manage the nitrogen cycle
			\item Provide access to clean water
			\item Restore and improve urban infrastructure
			\item Advance health informatics
			\item Engineer better medicines
			\item Reverse-engineer the brain
			\item Prevent nuclear terror
			\item Secure cyberspace
			\item Enhance virtual reality
			\item Advance personalized learning
			\item Engineer the tools of scientific discovery
			\end{enumerate}
		\item \url{http://www.engineeringchallenges.org/}
		\item NAE Grand Challenge K12 Partners Program: \vspace{-0.1cm}
			\begin{enumerate} \itemsep -1pt
			\item \url{http://www.grandchallengek12.org/about}
			\item 5-Part Make it Happen Plan: \url{http://www.grandchallengek12.org/5-part-plan}
			\end{enumerate}
		\end{enumerate}
	\item {\it National Academy of Engineering}'s technological literacy program for people (students, parents, and educators) to learn more about technology: \url{http://www.nae.edu/nae/techlithome.nsf}
	\item Greatest Engineering Achievements: \url{http://www.greatachievements.org/}
	\end{enumerate}
\item National Science Foundation: \vspace{-0.3cm}
	\begin{enumerate} \itemsep -2pt
	\item Broadening Participation in Computing (BPC): \vspace{-0.2cm}
		\begin{enumerate} \itemsep -2pt
		\item \url{http://www.bpcportal.org/}
		\item \url{http://www.bpcportal.org/bpc/shared/home.jhtml;jsessionid=0MIUYDR5U4ARXABAVRSSFEQ?_requestid=9445}
		\item \url{http://www.nsf.gov/funding/pgm_summ.jsp?pims_id=13510}
		\item \url{http://www.nsf.gov/funding/pgm_summ.jsp?pims_id=13510&org=NSF&sel_org=NSF&from=fund}
		\item ``Broadening Participation in Computing (BPC) is a NSF sponsored program with the goal of significantly increasing the number of underrepresented graduates in the computing disciplines, with an emphasis on women, persons with disabilities, and minorities (African Americans, Hispanics, American Indians, Alaska Natives, Native Hawaiians, and Pacific Islanders).''
		\item Broadening Participation in Computing Digital Library: \vspace{-0.1cm}
			\begin{enumerate} \itemsep -1pt
			\item \url{http://www.bpcportal.org/bpc/interdiscipline/dl_index.jhtml;jsessionid=ROYEHJV1UQYWNABAVRSSFEQ?comm=BPC}
			\item Includes resources for different target populations: \vspace{-0.1cm}
				\begin{itemize} \itemsep -1pt
				\item Women
				\item African Americans
				\item Hispanic Americans, or Latinas and Latinos
				\item People with disabilities
				\item Native Americans
				\end{itemize}
			\item It also includes resources for different topics, such as mentoring, recruitment, retention, and work-life balance.
			\end{enumerate}
		\item Alliances (other professional organizations): \url{http://www.bpcportal.org/bpc/comm/projects.jhtml}
		\end{enumerate}
	\item The National Science Digital Library (NSDL): \vspace{-0.2cm}
		\begin{enumerate} \itemsep -2pt
		\item \url{http://www.nsdl.org/} and \url{http://www.nsdl.org/browse/}
		\item ``The National Science Digital Library is a national network dedicated to advancing STEM teaching and learning for all learners, in both formal and informal settings, and the locus of activity for the National Science Foundation's National STEM Distributed Learning program.''
		\item Outreach materials: \vspace{-0.1cm}
			\begin{enumerate} \itemsep -1pt
			\item \url{http://www.nsdl.org/pd/?pager=materials}
			\item Has outreach materials for educators in K-12 and higher educational institutions.
			\end{enumerate}
		\item Resources for K-12 Teachers: \url{http://nsdl.org/resources_for/k12_teachers/}
		\item Resources for Librarians: \url{http://nsdl.org/resources_for/librarians/}
		\item Billingual Resources: \url{http://www.nsdlnetwork.org/collections/billingual-resources}
		\item NSDL on {\it iTunes U}: \url{http://www.nsdl.org/iTunesU/}
		\item Collections: \url{http://www.nsdl.org/browse/?subject=All}
		\item NSDL Pathways: \vspace{-0.1cm}
			\begin{enumerate} \itemsep -1pt
			\item \url{http://nsdl.org/about/?pager=pathways}
			\item ``Pathways are large projects that are aggregators and stewards of resources and services to broad categories of users---either discipline-focused (e.g. chemistry), or audience-focused (e.g. middle school educators), or resources of a specific type or format (e.g. multimedia content).''
			\item ``They are digital library portals developed and managed in partnership with organizations and institutions that have a history and expertise in serving their portal's target audiences.''
			\item ``They contribute metadata (descriptive information) about their resources to NSDL to make their resources searchable and discoverable via the NSDL.org portal, in addition to their own portals.''
			\end{enumerate}
		\item {\bf NSDL Science Literacy Maps}: \vspace{-0.1cm}
			\begin{enumerate} \itemsep -1pt
			\item \url{http://strandmaps.nsdl.org/}
			\item ``{\it NSDL Science Literacy Maps} are a tool for teachers and students to find resources that relate to specific science and math concepts. The maps illustrate connections between concepts as well as how concepts build upon one another across grade levels.''
			\end{enumerate}
		\item NSDL Professional Development: \url{http://www.nsdl.org/pd/}
		\item NSDL Technical Network Services: \vspace{-0.1cm}
			\begin{enumerate} \itemsep -1pt
			\item \url{http://www.nsdl.org/about/?pager=tns}
			\item \url{http://nsdlnetwork.org/}
			\item \url{http://nsdlnetwork.org/content/book/page/953/about-nsdl-technical-network-services}
			\end{enumerate}
		\item NSDL Resource Center: \url{http://nsdlnetwork.org/content/book/951/page/954/about-nsdl-resource-center}
		\end{enumerate}
	\end{enumerate}
\item {\it American Chemical Society} Science for Kids program (for students in K-12): \url{http://portal.acs.org/portal/acs/corg/content?_nfpb=true&_pageLabel=PP_TRANSITIONMAIN&node_id=878&use_sec=false&sec_url_var=region1&__uuid=984d4ee7-4214-4d35-9899-bc2f91dee58b}
\item {\it California Digital Educator Consortium}, ``Digital Educator,'' Digital Learning Center: \url{http://www.digitaleducator.com/}
\item Kenny Felder, ``Selected Other Educational Sites on the Web''. Available at: \url{http://www4.ncsu.edu/unity/lockers/users/f/felder/public/kenny/edulinks.html}; last accessed on August 28, 2010.
\item FHSST (Free High School Science Texts); free textbooks for grades 10-12 in Physics, Chemistry, and Mathematics. Available at: \url{http://www.fhsst.org/}; last accessed on August 28, 2010.
\item John Baez, {\it Usenet Physics FAQ}, Department of Mathematics, University of California, Riverside, September 2009. Available at: \url{http://math.ucr.edu/home/baez/physics/}; last accessed on August 28, 2010.
\item {\it American Society for Engineering Education}: \vspace{-0.3cm}
	\begin{enumerate} \itemsep -2pt
	\item Science and Engineering Apprenticeship Program (SEAP): \vspace{-0.2cm}
		\begin{enumerate} \itemsep -2pt
		\item ``The Science and Engineering Apprenticeship Program (SEAP) provides an opportunity for students to participate in research at a Department of Navy (DoN) laboratory during the summer.''
		\item ``The goals of SEAP are to encourage participating students to pursue science and engineering careers, to further their education via mentoring by laboratory personnel and their participation in research, and to make them aware of DoN Research and technology efforts, which can lead to employment within the DoN.''
		\item ``High school students who have completed at least Grade 9. A graduating senior is eligible to apply.''
		\item ``Must be 16 years of age for most laboratories. Some laboratories may accept a 15 year old applicant. Please check individual lab description for more details.''
		\item ``Applicants must be US citizens and participation by Permanent Resident Aliens is limited. Please check individual lab descriptions for participation of Permanent Resident Aliens.''
		\item \url{http://seap.asee.org/}
		\end{enumerate}
	\end{enumerate}
\item robots.net, {\it Robot Competitions} (list of robot competitions and contests) : \url{http://robots.net/rcfaq.html}
\item International Council on Systems Engineering (INCOSE): \vspace{-0.3cm}
	\begin{enumerate} \itemsep -2pt
	\item Careers in Systems Engineering: \url{http://www.incose.org/educationcareers/careersinsystemseng.aspx}
	\item Frequently Asked Questions for Students [about Systems Engineering]: \url{http://www.incose.org/educationcareers/faqsforstudents.aspx}
	\item What is Systems Engineering?: \url{http://www.incose.org/practice/whatissystemseng.aspx}
	\end{enumerate}
\item {\it National Society of Professional Engineers}: \vspace{-0.3cm}
	\begin{enumerate} \itemsep -2pt
	\item A Sightseer's Guide to Engineering: \url{http://www.engineeringsights.org/}
	\end{enumerate}
\item {\it Engineers Dedicated to a Better Tomorrow (a.k.a., DedicatedEngineers)}: \vspace{-0.3cm}
	\begin{enumerate} \itemsep -2pt
	\item The ``K-12 Crowd'' (Students, Teachers, Guidance Counselors and Parents): \url{http://www.dedicatedengineers.org/intro_for_K-12.htm}
	\item \url{http://www.dedicatedengineers.org/}
	\end{enumerate}
\item National Engineers Week Foundation: \vspace{-0.3cm}
	\begin{enumerate} \itemsep -2pt
	\item Discover Engineering: \url{http://www.discoverengineering.org/}
	\item Introduce A Girl to Engineering: \url{http://www.eweek.org/EngineersWeek/IntroduceAGirl.aspx}
	\item All About Engineering: \url{http://www.eweek.org/AboutEngineering/AboutEngineering.aspx}
	\end{enumerate}
\item University of California: \vspace{-0.3cm}
	\begin{enumerate} \itemsep -2pt
	\item The Coalition For Science After School: \vspace{-0.2cm}
		\begin{enumerate} \itemsep -2pt
		\item \url{http://afterschoolscience.org/}
		\item ``Promoting high-quality afterschool science'' ... ``The Coalition for Science After School envisions the day when young people from all backgrounds have access to high-quality science, technology, engineering and mathematics (STEM) learning beyond the classroom.''
		\item Tools for advocates--Championing afterschool science: \url{http://afterschoolscience.org/tools/}
		\item Program resources--Enhancing the quality of afterschool opportunities: \url{http://afterschoolscience.org/resources/}
		\item The National After School Science Directory: \vspace{-0.1cm}
			\begin{enumerate} \itemsep -1pt
			\item \url{http://afterschoolscience.org/directory/}
			\item ``The National After School Science Directory is a searchable database designed to increase access to high-quality science, technology, engineering and math (STEM) education beyond the classroom for youth and families across the nation. The Directory houses thousands of STEM opportunities, submitted by science centers, museums, schools and other youth-serving organizations. Search our Directory to view opportunities to connect the America's youth to high-quality STEM learning experiences.''
			\end{enumerate}
		\item Become an advocate: \url{http://afterschoolscience.org/tools/advocate.php}
		\item Funders (funding organizations/agencies): \url{http://afterschoolscience.org/tools/funders.php}
		\end{enumerate}
	\end{enumerate}
\item Harvey Mudd College: \vspace{-0.3cm}
	\begin{enumerate} \itemsep -2pt
	\item Francis Edward Su, {\it Math Fun Facts!}, Department of Mathematics, Harvey Mudd College: \url{http://www.math.hmc.edu/funfacts/}
	\end{enumerate}
\item Clay Mathematics Institute: \vspace{-0.3cm}
	\begin{enumerate} \itemsep -2pt
	\item Program in Mathematics for Young Scientists, PROMYS: \vspace{-0.2cm}
		\begin{enumerate} \itemsep -2pt
		\item \url{http://www.claymath.org/programs/outreach/PROMYS/}
		\item \url{http://math.bu.edu/people/promys/}
		\item \url{http://www.promys.org/}
		\end{enumerate}
	\item Ross Program (for pre-college students): \vspace{-0.2cm}
		\begin{enumerate} \itemsep -2pt
		\item \url{http://www.claymath.org/programs/outreach/ross/}
		\item \url{http://www.math.ohio-state.edu/ross/}
		\end{enumerate}
	\item CMI Summer Schools: \url{http://www.claymath.org/programs/summer_school/}
	\end{enumerate}
\item Consortium for Ocean Leadership: \vspace{-0.3cm}
	\begin{enumerate} \itemsep -2pt
	\item Oceans of Opportunity (for African American students in K-12, and colleges and universities -- includes undergraduates and grad students): \url{http://www.oceanleadership.org/education/diversity/oceans-of-opportunity/}
	\item The JOIDES Resolution (The JR) scientific research vessel [ Deep Earth Academy ]: \vspace{-0.2cm}
		\begin{enumerate} \itemsep -2pt
		\item Fun \& Games: \url{http://joidesresolution.org/node/53}
		\item Discovery Center: \url{http://joidesresolution.org/node/44}
		\item Just for Kids Blog: \url{http://joidesresolution.org/node/366}
		\end{enumerate}
	\item National Ocean Sciences Bowl (high school academic competition that provides a forum for talented students to test their knowledge of the marine sciences including biology, chemistry, physics, and geology): \vspace{-0.2cm}
		\begin{enumerate} \itemsep -2pt
		\item \url{http://www.nosb.org/}
		\item Career Resources: \url{http://www.nosb.org/ocean-careers/career-resources/}
		\end{enumerate}
	\item Integrated Ocean Drilling Program (IODP), IODP United States Implementing Organization (IODP-USIO): \vspace{-0.2cm}
		\begin{enumerate} \itemsep -2pt
		\item U.S.-sponsored Teacher at Sea Program (for US teachers to participate in seagoing research experiences aboard the JOIDES Resolution): \url{http://www.iodp-usio.org/Education/TAS.html}
		\end{enumerate}
	\item Careers: \url{http://www.oceanleadership.org/education/deep-earth-academy/students/careers/}
	\end{enumerate}
\item The Oceanography Society: \vspace{-0.3cm}
	\begin{enumerate} \itemsep -2pt
	\item Careers in Oceanography: Profiles, \url{http://www.tos.org/resources/career_profiles.html}
	\item Links [includes links to educational material for students in K-12]: \url{http://www.tos.org/resources/links.html}
	\end{enumerate}
\item American Geophysical Union: \vspace{-0.3cm}
	\begin{enumerate} \itemsep -2pt
	\item Bright Students Training as Research Scientists (Bright STaRS): \vspace{-0.2cm}
		\begin{enumerate} \itemsep -2pt
		\item \url{http://www.agu.org/education/diversity_programs/bstars.shtml}
		\item ``High school students participating in after-school and summer research experiences in the Earth and space sciences are invited to participate in the AGU Bright STaRS program. The Bright STaRS program provides a dedicated forum for $\sim$50 students to present their own research results to the scientific community and learn about exciting research, education, and career opportunities in the geosciences.''
		\end{enumerate}
	\end{enumerate}
\item American Geological Institute, AGI: \vspace{-0.3cm}
	\begin{enumerate} \itemsep -2pt
	\item AGI Education Department: \url{http://www.agiweb.org/geoeducation.html}
	\end{enumerate}
\item Society for Science \& the Public (SSP): \vspace{-0.3cm}
	\begin{enumerate} \itemsep -2pt
	\item Intel International Science \& Engineering Fair (Intel ISEF), which is a pre-college science competition: \url{http://www.societyforscience.org/isef/}
	\item Broadcom MASTERS\texttrademark\ competition (which stands for Broadcom Math, Applied Science, Technology and Engineering for Rising Stars): \vspace{-0.2cm}
		\begin{enumerate} \itemsep -2pt
		\item Is a U.S. ``national science, technology, engineering, and math competition for America's $6^{th}$, $7^{th}$, and $8^{th}$ graders.''
		\item \url{http://www.societyforscience.org/masters} or \url{http://www.broadcomfoundation.org/masters/}
		\end{enumerate} 
	\item Science resources: \url{http://www.societyforscience.org/resources}
	\item Science News: \url{http://www.sciencenews.org/}
	\item Science News for Kids (for ``children of ages 9-14, their teachers and their parents''): \url{http://www.societyforscience.org/sciencenewsforkids} and \url{http://www.sciencenewsforkids.org/}
	\end{enumerate}
\item Institute for Operations Research and the Management Sciences (INFORMS): \vspace{-0.3cm}
	\begin{enumerate} \itemsep -2pt
	\item Operations Research: The Science of Better, \url{http://www.scienceofbetter.org/}
	\end{enumerate}
\item Technion - Israel Institute of Technology: \vspace{-0.3cm}
	\begin{enumerate} \itemsep -2pt
	\item SciTech - the summer camp for talented students ($11^{th}$ and $12^{th}$ graders from all over the world): \url{http://www.scitech.technion.ac.il/}
	\end{enumerate}
\item USA Science \& Engineering Festival: \url{http://www.usasciencefestival.org/}
\item Girl Scouts: \vspace{-0.3cm}
	\begin{enumerate} \itemsep -2pt
	\item Girl Scouts of Western New York: \vspace{-0.2cm}
		\begin{enumerate} \itemsep -2pt
		\item STEM Resource Guide: \url{http://www.gswny.org/Data/Documents/STEM%2520Resource%2520Guide%25202010-Oct-11.pdf}
		\item Also, see \url{http://www.gswny.org/Programs/Awards/Gold/}; scroll to the bottom of the page and look under the subsection heading, ``Tell Us About Your Gold Award Project''
		\end{enumerate}
	\item Science, Technology, Engineering and Math (STEM): \url{http://www.girlscouts.org/program/program_opportunities/science/}
	\end{enumerate}
\item American Museum of Science and Energy (AMSE): \vspace{-0.3cm}
	\begin{enumerate} \itemsep -2pt
	\item \url{http://www.amse.org/}
	\item Owned by the US Department of Energy, and managed under Oak Ridge National Laboratory
	\item Educators: \url{http://www.amse.org/content.aspx?article=1140&parent=30}
	\item Educational Programs: \url{http://www.amse.org/content.aspx?article=1139&parent=30}
	\item Home school programs: \url{http://www.amse.org/content.aspx?article=1169&parent=30}
	\item Online resources: \url{http://www.amse.org/content.aspx?article=1170&parent=30}
	\end{enumerate}
\item Center for Energy Workforce Development (CEWD): \vspace{-0.3cm}
	\begin{enumerate} \itemsep -2pt
	\item Teachers and guidance counselors: \vspace{-0.2cm}
		\begin{enumerate} \itemsep -2pt
		\item \url{http://www.cewd.org/educators_index.asp}
		\item Lesson plans for teachers: \url{http://www.cewd.org/educators_lessonplans.asp}
		\end{enumerate}
	\item Parents: \url{http://www.cewd.org/parents_index.asp}
	\end{enumerate}
\item TryScience: \url{http://tryscience.net/tryscinetmain.nsf/Welcome?OpenPage}
\item The Dana Foundation: \vspace{-0.3cm}
	\begin{enumerate} \itemsep -2pt
	\item Brainy Kids: \vspace{-0.2cm}
		\begin{enumerate} \itemsep -2pt
		\item \url{http://www.dana.org/resources/brainykids/}
		\item Fun: \vspace{-0.1cm}
			\begin{enumerate} \itemsep -1pt
			\item \url{http://dana.org/resources/brainykids/detail.aspx?folder_id=104}
			\item Has interactive online games, activities, and fun quizzes on: \vspace{-0.1cm}
				\begin{itemize} \itemsep -1pt
				\item biology
				\item health
				\item neuroscience
				\item astronomy
				\item chemistry
				\item ecology
				\end{itemize}
			\end{enumerate}
		\item The Lab: \vspace{-0.1cm}
			\begin{enumerate} \itemsep -1pt
			\item \url{http://dana.org/resources/brainykids/detail.aspx?folder_id=106}
			\item Has maps of the brain, virtual dissections, resources for science fairs, and virtual microscopes
			\end{enumerate}
		\item Lesson Plans: \vspace{-0.1cm}
			\begin{enumerate} \itemsep -1pt
			\item \url{http://dana.org/resources/brainykids/detail.aspx?folder_id=108}
			\item Includes resources that cover the history of science and technology, lesson plans for K-12 science teachers, and science news for youths.
			\end{enumerate}
		\item The Mindboggling Workbook: \vspace{-0.1cm}
			\begin{enumerate} \itemsep -1pt
			\item \url{http://www.dana.org/uploadedFiles/The_Dana_Alliances/mindboggling_workbook.pdf}
			\item ``A fun-filled activity book about the brain for children in grades K-3 (ages 5-9). Provides an introduction to how the brain works, what the brain does, its importance, and how to take care of it.''
			\end{enumerate}
		\end{enumerate}
	\end{enumerate}
\item University of New Mexico: \vspace{-0.3cm}
	\begin{enumerate} \itemsep -2pt
	\item Department of Mathematics and Statistics: \vspace{-0.2cm}
		\begin{enumerate} \itemsep -2pt
		\item UNM - PNM Statewide Mathematics Contest (sponsored by the PNM Foundation): \url{http://mathcontest.unm.edu/}
		\end{enumerate}
	\end{enumerate}
\item Center for Energy Workforce (CEWD): \vspace{-0.3cm}
	\begin{enumerate} \itemsep -2pt
	\item Get Into Energy: \vspace{-0.2cm}
		\begin{enumerate} \itemsep -2pt
		\item \url{http://www.getintoenergy.com/index.asp} and \url{http://www.getintoenergy.com/careers.asp}
		\item Fun educational resources for students: \url{http://www.getintoenergy.com/students.asp}
		\item Career Quiz: \vspace{-0.1cm}
			\begin{enumerate} \itemsep -1pt
			\item \url{http://www.getintoenergy.com/search/careerquizj.asp}
			\item Help you find out more about career options in the energy field
			\end{enumerate}
		\item Career Resources: \vspace{-0.1cm}
			\begin{enumerate} \itemsep -1pt
			\item \url{http://www.getintoenergy.com/careerresources.asp}
			\item Has information on: \vspace{-0.1cm}
				\begin{itemize} \itemsep -1pt
				\item Training Programs (technical schools and colleges)
				\item Work-based Programs (apprenticeships and internships)
				\item Featured Employers
				\end{itemize}
			\end{enumerate}
		\item Skills Needed in the Energy Field: \vspace{-0.1cm}
			\begin{enumerate} \itemsep -1pt
			\item \url{http://www.getintoenergy.com/skills.asp}
			\item List skills for different kinds of jobs in the energy field
			\end{enumerate}
		\item Information for parents: \url{http://www.getintoenergy.com/Parents.asp}
		\item Information for teachers and guidance counselors: \url{http://www.getintoenergy.com/Educators.asp}
		\end{enumerate}
	\end{enumerate}
\item University of Utah: \vspace{-0.3cm}
	\begin{enumerate} \itemsep -2pt
	\item Department of Electrical and Computer Engineering: \vspace{-0.2cm}
		\begin{enumerate} \itemsep -2pt
		\item Prof. Cynthia Furse: \vspace{-0.1cm}
			\begin{enumerate} \itemsep -1pt
			\item Cynthia Furse, {\it K-12 Engineering Outreach}, August 2007. Available online at: \url{http://www.ece.utah.edu/~cfurse/K12.html}; last accessed on December 10, 2010.
			\item Cynthia Furse, {\it U Dream. U Design. U Create.}, Department of Electrical and Computer Engineering, University of Utah. Available online at: \url{http://www.ece.utah.edu/~cfurse/NSF/}; last accessed on December 10, 2010.
			\end{enumerate}
		\end{enumerate}
	\end{enumerate}
\item Society for Industrial and Applied Mathematics: \vspace{-0.3cm}
	\begin{enumerate} \itemsep -2pt
	\item Public Awareness: \vspace{-0.2cm}
		\begin{enumerate} \itemsep -2pt
		\item Math Competitions, \url{http://www.siam.org/publicawareness/competitions.php}
		\item Moody's Mega Math Challenge (M3 Challenge) is an applied mathematics competition for high school students. Available online at: \url{http://m3challenge.siam.org/}; last accessed on December 13, 2010.
		\item {\it Math Matters, Apply It!}: \url{http://www.siam.org/careers/matters.php}
		\item Nuggets: \url{http://www.siam.org/publicawareness/nuggets.php}
		\end{enumerate}
	\item Society for Industrial and Applied Mathematics, ``Unveiling Why Do Math,'' May 27, 2010. Available online at: \url{http://www.siam.org/about/news-siam.php?id=1741}; last accessed on December 13, 2010.
	\end{enumerate}
\item International Federation of Operational Research Societies (IFORS): \vspace{-0.3cm}
	\begin{enumerate} \itemsep -2pt
	\item Association of European Operational Research Societies (EURO): \vspace{-0.2cm}
		\begin{enumerate} \itemsep -2pt
		\item {\it What is Operational Research?}: \url{http://www.euro-online.org/display.php?pageid=197&}
		\item Applications of OR in music, literature, and aesthetics: \url{http://www.euro-online.org/display.php?pageid=211&}
		\item 24 Hours Operations Research: \url{http://www.24hor.org/}
		\item Branding OR: \url{http://www.euro-online.org/display.php?pageid=198&}
		\end{enumerate}
	\end{enumerate}
\item American Institute of Aeronautics and Astronautics (AIAA): \vspace{-0.3cm}
	\begin{enumerate} \itemsep -2pt
	\item Students \& Educators: \url{http://www.aiaa.org/content.cfm?pageid=5}
	\item Ask An Engineer: \url{http://www.aiaa.org/content.cfm?pageid=214}
	\item Kid's Place: \vspace{-0.2cm}
		\begin{enumerate} \itemsep -2pt
		\item \url{http://www.aiaa.org/content.cfm?pageid=473}
		\item Enjoy games, puzzles, fun experiments, teen-recommended books and movies, and more.
		\end{enumerate}
	\item History of Flight Timeline: \url{http://www.aiaa.org/content.cfm?pageid=260}
	\item Ask Polaris: \vspace{-0.2cm}
		\begin{enumerate} \itemsep -2pt
		\item \url{http://www.askpolaris.org/}
		\item Resource for career exploration in aerospace engineering and related fields
		\end{enumerate}
	\end{enumerate}
\item Massachusetts Institute of Technology: \vspace{-0.3cm}
	\begin{enumerate} \itemsep -2pt
	\item MIT School of Engineering: \vspace{-0.2cm}
		\begin{enumerate} \itemsep -2pt
		\item Lemelson-MIT Program: \vspace{-0.1cm}
			\begin{enumerate} \itemsep -1pt
			\item \url{http://web.mit.edu/invent/}
			\item Inventor's Handbook: \url{http://web.mit.edu/invent/h-main.html}
			\item Games \& Trivia; \url{http://web.mit.edu/invent/g-main.html}
			\item Links \& Resources: \url{http://web.mit.edu/invent/r-main.html}
			\end{enumerate}
		\end{enumerate}
	\end{enumerate}
\item BT Group plc: \vspace{-0.3cm}
	\begin{enumerate} \itemsep -2pt
	\item British Telecommunications plc (BT): \vspace{-0.2cm}
		\begin{enumerate} \itemsep -2pt
		\item BT Young Scientist \& Technology Exhibition: \vspace{-0.1cm}
			\begin{enumerate} \itemsep -1pt
			\item \url{http://www.btyoungscientist.com/}
			\item \url{http://www.btyoungscientist.com/all-you-need-to-know/}
			\item Science and technology fair for high/secondary school students in Ireland
			\end{enumerate}
		\end{enumerate}
	\end{enumerate}
\item NHS Medical Careers: \vspace{-0.3cm}
	\begin{enumerate} \itemsep -2pt
	\item \url{http://www.medicalcareers.nhs.uk/Default.aspx}
	\item Provides information about careers in medicine for prospective medical students, medical students, medical school graduates (or young medical professionals), (medical speciality) trainers, and medical specialists.
	\end{enumerate}
\item British Science Association: \vspace{-0.3cm}
	\begin{enumerate} \itemsep -2pt
	\item British Science Festival: \vspace{-0.2cm}
		\begin{enumerate} \itemsep -2pt
		\item \url{http://www.britishscienceassociation.org/web/BritishScienceFestival/AboutFestival/index.htm}
		\item Festival Student Bursaries: \url{http://www.britishscienceassociation.org/web/BritishScienceFestival/StudentBursaries/index.htm}
		\end{enumerate}
	\item National Science \& Engineering Week: \url{http://www.britishscienceassociation.org/web/NSEW/index.htm}
	\item Clubs, CREST Awards and Fairs (programs and activities for children and youth, 5-19 years of age): \url{http://www.britishscienceassociation.org/web/ccaf/index.htm}
	\item National Science \& Engineering Competition: \url{http://www.britishscienceassociation.org/web/NSEC/index.htm} and \url{http://www.thebigbangfair.co.uk/nsec/}
	\end{enumerate}
\item Research Councils UK (RCUK): \vspace{-0.3cm}
	\begin{enumerate} \itemsep -2pt
	\item \url{http://www.rcuk.ac.uk/per/Pages/Schools.aspx}
	\item Schoolscience: \vspace{-0.2cm}
		\begin{enumerate} \itemsep -2pt
		\item \url{http://www.schoolscience.co.uk/}
		\item For students and educators in K-12 to enrich the learning experiences of science topics, and help students connect classroom material to the real world.
		\item Teacher Zone - professional resources for teachers: \url{http://www.schoolscience.co.uk/teacher_zone.cfm}
		\item Interactive Learning Resources: \url{http://www.schoolscience.co.uk/interactives.cfm}
		\item Free Resources: \url{http://www.schoolscience.co.uk/freebies.cfm}
		\item Competitions: \url{http://www.schoolscience.co.uk/competitions.cfm}
		\item Research focus: \url{http://www.schoolscience.co.uk/research_focus.cfm}
		\item Resources on the World Wide Web: \url{http://www.schoolscience.co.uk/sciencelink.cfm}
		\end{enumerate}
	\item Researchers in Residence (RinR): \vspace{-0.2cm}
		\begin{enumerate} \itemsep -2pt
		\item \url{http://www.researchersinresidence.ac.uk/cms/schools-colleges/}
		\item For students in middle and high schools to job shadow (observe first-hand) a Ph.D. student or postdoctoral researcher in her/his research activities for up to a week, so that students can learn what doing research in her/his research area is like. In addition, the researcher would explain in laypeople's terms what her/his research is about. It can be considered as an externship program.
		\end{enumerate}
	\item Nuffield Bursaries: \vspace{-0.2cm}
		\begin{enumerate} \itemsep -2pt
		\item \url{http://www.nuffieldfoundation.org/capacity-building}
		\item \url{http://www.nuffieldfoundation.org/science-bursaries-schools-and-colleges}
		\item For high school juniors/seniors to pursue a research internship in science and engineering.
		\end{enumerate}
	\item CREST (Creativity in Science and Technology): \vspace{-0.2cm}
		\begin{enumerate} \itemsep -2pt
		\item \url{http://www.britishscienceassociation.org/web/ccaf/CREST/index.htm}
		\item Program to help students get engaged in a science or engineering project, where they learn how to solve real problems in science or engineering.
		\end{enumerate}
	\end{enumerate}
\item Nuffield Foundation: \vspace{-0.3cm}
	\begin{enumerate} \itemsep -2pt
	\item Science bursaries for schools and colleges: \url{http://www.nuffieldfoundation.org/science-bursaries-schools-and-colleges}
	\item Students: \url{http://www.nuffieldfoundation.org/students}
	\item Twenty First Century Science: \vspace{-0.2cm}
		\begin{enumerate} \itemsep -2pt
		\item \url{http://www.21stcenturyscience.org/}
		\item ``Twenty First Century Science is a set of GCSE science courses giving all 14-16-year-olds a worthwhile and inspiring experience of science. The strength of the programme is that it meets the needs, through flexible options, of those who will go on to be professional scientists and of those who will not.''
		\item The Courses: \url{http://www.21stcenturyscience.org/the-courses/}
		\item Assessment overview: \url{http://www.21stcenturyscience.org/assess/}
		\item Teaching resources: \url{http://www.21stcenturyscience.org/resources/}
		\end{enumerate}
	\item Science in Society: \vspace{-0.2cm}
		\begin{enumerate} \itemsep -2pt
		\item \url{http://www.scienceinsocietyadvanced.org/}
		\item ``Science in Society is an interesting and topical GCE advanced level course. It aims to develop the knowledge and skills that are needed for students to understand how science works, analyse contemporary issues involving science and technology and communicate their scientific appreciation and understanding to others.''
		\end{enumerate}
	\item Parents: \url{http://www.nuffieldfoundation.org/parents}
	\item Education: \url{http://www.nuffieldfoundation.org/education}
	\item Teachers (has excellent resources for science and mathematics): \url{http://www.nuffieldfoundation.org/teachers}
	\item Capacity building: \url{http://www.nuffieldfoundation.org/capacity-building}
	\end{enumerate}
\item The Story of Stuff Project (by Annie Leonard): \vspace{-0.3cm}
	\begin{enumerate} \itemsep -2pt
	\item \url{http://www.storyofstuff.com/}
	\item ``The Story of Stuff Project was created by Annie Leonard to leverage and extend the film's impact. We amplify public discourse on a series of environmental, social and economic concerns and facilitate the growing Story of Stuff community's involvement in strategic efforts to build a more sustainable and just world.''
	\item Resources: \vspace{-0.2cm}
		\begin{enumerate} \itemsep -2pt
		\item \url{http://www.storyofstuff.com/resources.php}
		\item The Story of Stuff Project PDFs: \url{http://www.storyofstuff.com/dl-pdfs.php}
		\item Teaching Tools: \url{http://www.storyofstuff.com/teach.php}
		\item More About Stuff: \url{http://www.storyofstuff.com/aboutstuff.php}
		\item Recommended Reading \& Bibliography: \url{http://www.storyofstuff.com/reading.php}
		\item Get Involved: \url{http://www.storyofstuff.com/getinvolved.php}
		\item Curricula: \url{http://storyofstuff.org/curricula.php}
		\end{enumerate}
	\end{enumerate}
\item Facing the Future: \vspace{-0.3cm}
	\begin{enumerate} \itemsep -2pt
	\item \url{http://www.facingthefuture.org/}
	\item ``{\it Facing the Future} engages students in learning by making academics relevant to their lives. We empower students to think critically, develop a global perspective, and participate in positive solutions for a sustainable future.''
	\item Curriculum Alignment with Education Standards: \url{http://www.facingthefuture.org/Curriculum/AlignmentwithEducationStandards/tabid/116/Default.aspx}
	\item Global Sustainability Curriculum Finder: \url{http://www.facingthefuture.org/Curriculum/FindCurriculumthatisRightforYou/tabid/68/Default.aspx}
	\item Download FREE Global Issues and Sustainability Curriculum: \url{http://www.facingthefuture.org/Curriculum/DownloadFreeCurriculum/tabid/114/Default.aspx}
	\item Classroom Examples: How Engaging Curriculum Can Help Address Classroom Challenges, \url{http://www.facingthefuture.org/ForEducators/ClassroomExamples/tabid/213/Default.aspx}
	\item Our Impact on Student Achievement: \url{http://www.facingthefuture.org/ForEducators/OurImpactonStudentAchievement/tabid/73/Default.aspx}
	\item Action Project Database: \url{http://www.facingthefuture.org/ServiceLearning/ActionProjectDatabase/tabid/94/Default.aspx}
	\item Service Learning Examples: \url{http://www.facingthefuture.org/ServiceLearning/ExamplesofStudentsTakingAction/tabid/147/Default.aspx}
	\item Curriculum: \url{http://www.facingthefuture.org/Curriculum/CurriculumHome/tabid/113/Default.aspx}
	\end{enumerate}
\item U.S. Department of Energy: \vspace{-0.3cm}
	\begin{enumerate} \itemsep -2pt
	\item Office of Science: \vspace{-0.2cm}
		\begin{enumerate} \itemsep -2pt
		\item U.S. Department of Energy (DOE) National Science Bowl\textregistered: \vspace{-0.1cm}
			\begin{enumerate} \itemsep -1pt
			\item \url{http://www.scied.science.doe.gov/nsb/default.htm}
			\item ``The U.S. Department of Energy (DOE) National Science Bowl\textregistered\ is a nationwide academic competition that tests students' knowledge in all areas of science. High school and middle school students are quizzed in a fast paced question-and-answer format similar to Jeopardy. Competing teams from diverse backgrounds are comprised of four students, one alternate, and a teacher who serves as an advisor and coach.''
			\end{enumerate}
		\item Argonne National Laboratory: \vspace{-0.1cm}
			\begin{enumerate} \itemsep -1pt
			\item Division of Educational Programs: \vspace{-0.1cm}
				\begin{itemize} \itemsep -1pt
				\item Newton BBS Ask A Scientist: \url{http://www.newton.dep.anl.gov/aas.htm}
				\end{itemize}
			\end{enumerate}
		\end{enumerate}
	\item Office of Energy Efficiency and Renewable Energy (EERE): \vspace{-0.2cm}
		\begin{enumerate} \itemsep -2pt
		\item Kids Saving Energy: \vspace{-0.1cm}
			\begin{enumerate} \itemsep -1pt
			\item \url{http://www.eere.energy.gov/kids/index.html}
			\item K-12 Lesson Plans \& Activities: \url{http://www1.eere.energy.gov/education/lessonplans/}
			\item Energy Savers: \url{http://www.energysavers.gov/}
			\item Games and activities: \url{http://www.eere.energy.gov/kids/games.html}
			\item Smart home: \url{http://www.eere.energy.gov/kids/smart_home.html}
			\item About renewable energy: \url{http://www.eere.energy.gov/kids/renergy.html}
			\end{enumerate}
		\end{enumerate}
	\item Contest \& Competitions: \url{http://www.energy.gov/contests&competitions.htm}
	\end{enumerate}
\item United States Department of Defense (DoD): \vspace{-0.3cm}
	\begin{enumerate} \itemsep -2pt
	\item National Defense Education Program; Defense Advanced Research Projects Agency (DARPA): \vspace{-0.2cm}
		\begin{enumerate} \itemsep -2pt
		\item Resource for Students: \url{http://www.ndep.us/GetInvoStu.aspx}
		\item Resource for Educators: \url{http://www.ndep.us/GetInvoTea.aspx}
		\end{enumerate}
	\end{enumerate}
\item Project Lead The Way: \vspace{-0.3cm}
	\begin{enumerate} \itemsep -2pt
	\item \url{http://www.pltw.org/}
	\item Getting started: \url{http://www.pltw.org/getting-started/getting-started}
	\item Program support: \url{http://www.pltw.org/program-support/program-support}
	\item Grants available to schools and teachers: \url{http://www.pltw.org/pltw-in-the-news/grants-available-schools-teachers-and-classrooms}
	\item Students: \url{http://www.pltw.org/students/students}
	\item Educators and Administrators: \url{http://www.pltw.org/educators-administrators/educators-administrators-overview}
	\item Parents: \url{http://www.pltw.org/parents/parents}
	\end{enumerate}
\item National Science Teachers Association: \vspace{-0.3cm}
	\begin{enumerate} \itemsep -2pt
	\item \url{http://www.exploravision.org/}
	\item Science competition for K-12 students
	\end{enumerate}
\item American Mathematical Society: \vspace{-0.3cm}
	\begin{enumerate} \itemsep -2pt
	\item Some career resources for mathematics: \url{http://e-math.ams.org/samplings/samplings}
	\end{enumerate}
\item American Institute of Physics (AIP): \vspace{-0.3cm}
	\begin{enumerate} \itemsep -2pt
	\item Physics Success Stories: \url{http://www.aip.org/success/}
	\item Physics is for you; Career Services Division: \vspace{-0.2cm}
		\begin{enumerate} \itemsep -2pt
		\item \url{http://www.aip.org/careersvc/pify/}
		\item Physicists at work: \url{http://www.aip.org/careersvc/pify/yellow.html}
		\end{enumerate}
	\item Society of Physics Students (SPS): \vspace{-0.2cm}
		\begin{enumerate} \itemsep -2pt
		\item Careers Using Physics (CUP): \vspace{-0.1cm}
			\begin{enumerate} \itemsep -1pt
			\item \url{http://www.spsnational.org/cup/}
			\item Advice: \url{http://www.spsnational.org/cup/advice/index.html}
			\item Resources: \url{http://www.spsnational.org/cup/resources.html}
			\item Preparing to Teach: \url{http://www.spsnational.org/cup/teach/index.html}
			\end{enumerate}
		\end{enumerate}
	\item ComPADRE Digital Library: \vspace{-0.2cm}
		\begin{enumerate} \itemsep -2pt
		\item \url{http://www.compadre.org/}
		\item The Physics Career Resource: \url{http://www.compadre.org/careers/}
		\end{enumerate}
	\item Career guidance for high school and undergraduate students: \url{http://www.aip.org/statistics/trends/career.html}
	\item Gayle A. Buck, Jack G. Hehn, and Diandra L. Leslie-Pelecky (Editors), ``The Role of Physics Departments in Preparing K-12 Teachers,'' American Institute of Physics. Available online at: \url{http://www.aip.org/education/teacherprep/}; last accessed on January 9, 2010.
	\item American Geophysical Union: \vspace{-0.2cm}
		\begin{enumerate} \itemsep -2pt
		\item Students \& Teachers: \url{http://www.agu.org/education/students_teachers.shtml}
		\item Diversity Programs: \url{http://www.agu.org/education/diversity_programs/}
		\end{enumerate}
	\end{enumerate}
\item Institute for Operations Research and the Management Sciences (INFORMS): \vspace{-0.3cm}
	\begin{enumerate} \itemsep -2pt
	\item Career FAQ's: \url{http://www.informs.org/Build-Your-Career/INFORMS-Student-Union/Career-Center/Career-FAQ-s}
	\end{enumerate}
\item American Institute of Mathematics: \vspace{-0.3cm}
	\begin{enumerate} \itemsep -2pt
	\item Math Teachers' Circle Network: \vspace{-0.2cm}
		\begin{enumerate} \itemsep -2pt
		\item Classroom Materials: \url{http://www.mathteacherscircle.org/resources/classroommaterials.html}
		\item Helpful Resources: \url{http://www.mathteacherscircle.org/resources/general.html}
		\end{enumerate}
	\item Resources for the Math Community: \vspace{-0.2cm}
		\begin{enumerate} \itemsep -2pt
		\item \url{http://www.aimath.org/mathcommunity/}
		\item David W. Farmer, ``The AIM REU: individual projects with a common theme,'' in the {\it Proceedings of the Conference on Promoting Undergraduate Research in Mathematics}, American Mathematical Society, 2006. Available online at: \url{http://www.aimath.org/mathcommunity/farmerREU.pdf}; last accessed on January 9, 2010. [ ``AIM Research Experience for Undergraduates (REU)'' ]
		\item Sally Koutsoliotas and David W. Farmer, ``Preparing students to give talks,'' American Institute of Mathematics. Available online at: \url{http://www.aimath.org/mathcommunity/studenttalks.pdf}; last accessed on January 9, 2010. [ ``Preparing students to give talks'' ]
		\end{enumerate}
	\end{enumerate}
\item Invent Now: \vspace{-0.3cm}
	\begin{enumerate} \itemsep -2pt
	\item Camp Invention: \vspace{-0.2cm}
		\begin{enumerate} \itemsep -2pt
		\item ``Summer enrichment program for children entering grades one through six.''
		\item ``The Camp Invention program instills vital 21st century life skills such as problem-solving and teamwork through hands-on fun!''
		\item Parents: \url{http://www.invent.org/camp/parents.aspx}
		\item Teachers: \url{http://www.invent.org/camp/teachers.aspx}
		\end{enumerate}
	\end{enumerate}
\item Massachusetts Institute of Technology: \vspace{-0.3cm}
	\begin{enumerate} \itemsep -2pt
	\item MIT School of Engineering: \vspace{-0.2cm}
		\begin{enumerate} \itemsep -2pt
		\item Lemelson-MIT Program: \vspace{-0.1cm}
			\begin{enumerate} \itemsep -1pt
			\item \url{http://web.mit.edu/invent/}
			\item Invention Dimension (for children): \url{http://web.mit.edu/invent/invent-main.html}
			\end{enumerate}
		\end{enumerate}
	\end{enumerate}
\item The Lemelson Foundation: \vspace{-0.3cm}
	\begin{enumerate} \itemsep -2pt
	\item \url{http://web.mit.edu/invent/w-foundation.html}
	\item Programs \& Grants: \url{http://www.lemelson.org/programs-grants}
	\item Grantmaking: \url{http://www.lemelson.org/grantmaking}
	\end{enumerate}
\item Smithsonian Institution: \vspace{-0.3cm}
	\begin{enumerate} \itemsep -2pt
	\item Smithsonian Kids: \url{http://www.si.edu/Kids}
	\item National Museum of American History: \vspace{-0.2cm}
		\begin{enumerate} \itemsep -2pt
		\item Lemelson Center for the Study of Invention and Innovation: \vspace{-0.1cm}
			\begin{enumerate} \itemsep -1pt
			\item \url{http://inventionatplay.org/index.html}
			\item Resources: \url{http://inventionatplay.org/resources.html}
			\end{enumerate}
		\end{enumerate}
	\end{enumerate}
%%%%%%%%%%%%%%%%%%%%%%%%%%%%%%%%%%%%%%%%
%%%%%%%%%%%%%%%%%%%%%%%%%%%%%%%%%%%%%%%%
\item Scholarships: \vspace{-0.3cm}
	\begin{enumerate} \itemsep -2pt
	\item IEEE Presidents' Scholarship: \url{http://www.ieee.org/education_careers/education/preuniversity/scholarship.html}
	\item ACM/SIGDA {\it P. O. Pistilli scholarship}: \vspace{-0.1cm}
		\begin{enumerate} \itemsep -1pt
		\item Supported by the Design Automation Conference which ACM/SIGDA sponsors, the objective of the P. O. Pistilli Scholarship is to increase the pool of professionals in Electrical Engineering and Computer Science from underrepresented groups (Women, African American, Hispanic, American Indian, and Disabled).
		\item Scholarships of \$4000 per year, renewable for up to 5 years, are awarded annually to 2-7 high school seniors from the above mentioned under represented groups who have a 3.00 GPA or better (on a 4.00 scale), have demonstrated high achievement in math and science courses, have expressed a strong desire to pursue careers in electrical engineering, computer engineering, or computer science, and who have demonstrated substantial financial need.
		\item U.S. citizenship is not required, but applicants must be U.S. residents when they apply and must plan to attend an accredited US college or university.
		\item \url{http://www.sigda.org/pistilli.html}
		\end{enumerate}
	\item Engineering Education Service Center (EESC): \url{http://www.engineeringedu.com/scholars.html}
	\item ASME-ASME Auxiliary FIRST Clarke Scholarships: \url{http://www.asme.org/Education/College/FinancialAid/High_School_Seniors.cfm} and \url{http://www.asme.org/Education/College/FinancialAid/Auxiliary_FIRST_Clarke.cfm}
	\item International Petroleum Institute�s High School Scholarships (for individuals entering a college program in engineering): \url{http://www.asme-ipti.org/public/pagscholarshipprograms.aspx}
	\item American Institute of Chemical Engineers (AIChE): \vspace{-0.2cm}
		\begin{enumerate} \itemsep -2pt
		\item Fuels and Petrochemicals Division Scholarship (for high school students entering undergraduate programs in engineering or science that are related to fuels and petrochemicals): \url{http://www.aiche.org/Students/Awards/F_PDScholarship.aspx}
		\item Minority Scholarship Awards for Incoming College Freshmen (for underrepresented minorities entering an undergraduate chemical engineering program): \url{http://www.aiche.org/Students/Awards/MinorityScholarshipAwardsIncomingFreshmen.aspx}
		\end{enumerate}
	\item Sallie Mae Fund: \vspace{-0.3cm}
		\begin{enumerate} \itemsep -2pt
		\item \url{http://www.thesalliemaefund.org/smfnew/index.html}
		\item List of scholarship resources: \url{http://www.thesalliemaefund.org/smfnew/sections/search.html}
		\item Top 10 Tips for Planning and Paying for College: \url{http://www.thesalliemaefund.org/smfnew/fin_aid/index.html}
		\item Scholarships: \url{http://www.thesalliemaefund.org/smfnew/scholarship/index.html} and \url{http://www.thesalliemaefund.org/smfnew/sections/apply.html}
		\item Important information for parents about saving for college and getting financial aid: \vspace{-0.2cm}
			\begin{enumerate} \itemsep -2pt
			\item \url{http://www.thesalliemaefund.org/smfnew/sections/download.html}
			\item This information is also available in Spanish. Summaries are also available in other languages such as: \vspace{-0.1cm}
				\begin{itemize} \itemsep -1pt
				\item French
				\item German
				\item Italian
				\item Korean
				\item Russian
				\item Simplified and Traditional Chinese
				\item Tagalog
				\item Vietnamese
				\end{itemize}
			\item Top 10 Tips for Planning and Paying for College: \url{http://www.thesalliemaefund.org/smfnew/fin_aid/index.html}
			\end{enumerate}
		\item Kids2College program: \url{http://www.thesalliemaefund.org/smfnew/initiatives/kidscollege.html}
		\item For African-American individuals entering college: \vspace{-0.2cm}
			\begin{enumerate} \itemsep -2pt
			\item Black College Dollars: \url{http://www.thesalliemaefund.org/smfnew/scholarship_directory/index.html}
			\item \url{http://www.thesalliemaefund.org/smfnew/initiatives/aa.html}
			\end{enumerate}
		\item For Hispanic Americans, or Latinos/Latinas: \vspace{-0.2cm}
			\begin{enumerate} \itemsep -2pt
			\item \url{http://www.thesalliemaefund.org/smfnew/pdf/Scholarship_Directory.pdf}
			\item Latino College Dollars: \url{http://www.latinocollegedollars.org/}
			\end{enumerate}
		\end{enumerate}
	\item {\it American Chemical Society}: \vspace{-0.3cm}
		\begin{enumerate} \itemsep -2pt
		\item ACS Scholars Program (for underrepresented minorities in, or entering, an undergraduate program in chemistry, biochemistry, or chemical engineering): \url{http://portal.acs.org/portal/acs/corg/content?_nfpb=true&_pageLabel=PP_SUPERARTICLE&node_id=1650&use_sec=false&sec_url_var=region1&__uuid=b3b583cf-18ae-4fb0-9375-33f75a0ccf49}
		\item Project SEED Scholarships (for high school seniors who have worked at least one summer at a science institute under the Project SEED program): \url{http://portal.acs.org/portal/acs/corg/content?_nfpb=true&_pageLabel=PP_SUPERARTICLE&node_id=2031&use_sec=false&sec_url_var=region1&__uuid=99bc6a62-3e78-4b2a-be3f-50b28f7ff265}
		\end{enumerate}
	\item The Posse Foundation: \url{http://www.possefoundation.org/}
	\item Hispanic Scholarship Fund (HSF) scholarship programs for high school students: \url{http://www.hsf.net/innerContent.aspx?id=426}
	\item Asian \& Pacific Islander American Scholarship Fund (APIASF): scholarships for individuals entering college as freshmen; see \url{http://www.apiasf.org/scholarship_apiasf.html}
	\item Nationally Coveted College Scholarships, Graduate School Fellowships \& Postdoctoral Awards: \url{http://scholarships.fatomei.com/}
	\item {\it SPIE} Scholarship Program (for high school students entering college to study optics, photonics, imaging, optoelectronics, or related program): \url{http://spie.org//x1733.xml?WT.svl=mddm14}
	\item Susan G. Komen for the Cure\textregistered: The Komen College Scholarship Program, \url{http://ww5.komen.org/ResearchGrants/CollegeScholarshipAward.html}
	\item National Society of Professional Engineers's list of scholarships for high school students: \url{http://www.nspe.org/Students/Scholarships/index.html}
	\item AWM Essay Contest: Biographies of Contemporary Women in Mathematics; see \url{http://www.awm-math.org/biographies/contest.html}
	\item National Engineers Week Future City Competition (students from $6^{th}$--$8^{th}$ grades): \url{http://www.futurecity.org/}
	\item National Ocean Sciences Bowl: \vspace{-0.2cm}
		\begin{enumerate} \itemsep -2pt
		\item \url{http://www.nosb.org/ocean-careers/}
		\item National Ocean Scholar Program (for high school seniors who are current/past participants of the Bowl, and are seeking a career in the ocean sciences or a marine-related field): \url{http://www.nosb.org/ocean-careers/national-ocean-scholar-program/}
		\end{enumerate}
	\item National Center for Women \& Information Technology (NCWIT): \vspace{-0.2cm}
		\begin{enumerate} \itemsep -2pt
		\item NCWIT Award for Aspirations in Computing (for young women in high school): \url{http://www.ncwit.org/work.awards.aspiration.html}
		\end{enumerate}
	\end{enumerate}
%%%%%%%%%%%%%%%%%%%%%%%%%%%%%%%%%%%%%%%%
%%%%%%%%%%%%%%%%%%%%%%%%%%%%%%%%%%%%%%%%
\item Resources for teachers/educators: \vspace{-0.3cm}
	\begin{enumerate} \itemsep -2pt
	\item Google: \vspace{-0.2cm}
		\begin{enumerate} \itemsep -2pt
		\item Google Teacher Academy (for teachers to learn how to use Google technologies to facilitate teaching): \url{http://www.google.com/educators/gta.html}
		\item Classroom activities (suggestions): \url{http://www.google.com/educators/activities.html}
		\end{enumerate}
	\item IEEE Teacher In-Service Program (TISP): \vspace{-0.2cm}
		\begin{enumerate} \itemsep -2pt
		\item \url{http://www.ieee.org/education_careers/education/preuniversity/tispt/index.html}
		\item Lesson Plans for Pre-university Instructors: \url{http://www.ieee.org/education_careers/education/preuniversity/resources/index.html}
		\end{enumerate}
	\item Global Challenge Award: \url{http://www.globalchallengeaward.org/display/public/Home}
	\item Teachers' Domain (to teach students about science, engineering, and the arts): \url{http://www.teachersdomain.org/}
	\item {\it TeachEngineering} digital library: \vspace{-0.2cm}
		\begin{enumerate} \itemsep -2pt
		\item The {\it TeachEngineering} digital library provides teacher-tested, standards-based engineering content for K-12 teachers engineering content for K12 teachers to use in science and math classrooms. Engineering lessons connect real-world experiences with curricular content already taught in K-12 classrooms. Mapped to educational content standards, {\it TeachEngineering}'s comprehensive curricula are hands-on, free, and relevant to children's daily lives.
		\item \url{http://www.teachengineering.com/index.php}
		\end{enumerate}
	\item Engineering Pathway: \url{http://www.engineeringpathway.com/ep/index.jhtml}
	\item {\it American Society of Mechanical Engineers, ASME}: \url{http://www.asme.org/Education/PreCollege/TeacherResources/}
	\item {\it National Science Foundation} resources for the K-12 classroom: \url{http://nsf.gov/news/classroom/engineering.jsp}
	\item {\it NASA}: \url{http://www.nasa.gov/audience/foreducators/index.html}
	\item The Mathematical Association of America: \vspace{-0.2cm}
		\begin{enumerate} \itemsep -2pt
		\item Pre-College Programs: \url{http://www.maa.org/funding/pre_college.html}. Also, see \url{http://www.maa.org/funding/undergraduate.html}.
		\item Special Interest Group of the Mathematical Association of America on the use of the World-Wide Web in Undergraduate Mathematics Instruction (Web SIGMAA). Available at: \url{http://math.chapman.edu/websigmaa/index.php/Main_Page}; last accessed on September 2, 2010.
		\item SIGMAA TAHSM (Teaching Advanced High School Mathematics). Available at: \url{http://sigmaa.maa.org/tahsm/}; last accessed on September 2, 2010.
		\item Special Interest Group on Statistics Education: \url{http://sigmaa.maa.org/stat-ed/}
		\end{enumerate}
	\item Math for America: \vspace{-0.2cm}
		\begin{enumerate} \itemsep -2pt
		\item M$f$A Master Teacher Fellowship program: \vspace{-0.1cm}
			\begin{enumerate} \itemsep -1pt
			\item The Math for America Master Teacher Fellowship program rewards exceptional public secondary school math teachers with a four-year Fellowship.
			\item M$f$A Master Teacher Fellowships are currently available in Berkeley, Boston and New York City.
			\item \url{http://www.mathforamerica.org/web/guest/master-teachers}
			\end{enumerate}
		\item M$f$A Early Career Fellows: \vspace{-0.1cm}
			\begin{enumerate} \itemsep -1pt
			\item The Math for America Early Career Fellowship is awarded to public secondary school math teachers early in their careers.
			\item The M$f$A Early Career Fellowship requires a commitment of four years and is available in New York City. 
			\item \url{http://www.mathforamerica.org/early-career-fellows}
			\end{enumerate}
		\item M$f$A Fellows: \vspace{-0.1cm}
			\begin{enumerate} \itemsep -1pt
			\item \url{http://www.mathforamerica.org/web/guest/mfa-fellows}
			\end{enumerate}
		\item Teachers resources: \url{http://www.mathforamerica.org/web/guest/teacher-resources} and \url{http://www.mathforamerica.org/teacher-resources/classroom} (classroom resources)
		\item Resources for professional development (teachers): \url{http://www.mathforamerica.org/teacher-resources/professional}
		\item \url{http://www.mathforamerica.org/home}
		\end{enumerate}
	\item Association for Symbolic Logic (ASL): \vspace{-0.2cm}
		\begin{enumerate} \itemsep -2pt
		\item Guidelines on Logic Education: \url{http://www.ucalgary.ca/aslcle/guidelines}
		\item Educational Logic Software: \url{http://www.ucalgary.ca/aslcle/logic-courseware}
		\end{enumerate}
	\item Consortium for Ocean Leadership: \vspace{-0.2cm}
		\begin{enumerate} \itemsep -2pt
		\item Educational Resources: \url{http://www.oceanleadership.org/gulf-oil-spill/educational-resources/}
		\item The JOIDES Resolution (The JR) scientific research vessel [ Deep Earth Academy ]: \vspace{-0.1cm}
			\begin{enumerate} \itemsep -1pt
			\item Teacher Resources (to teach students about geology and physical geography): \url{http://joidesresolution.org/node/46}
			\item Teachers at Sea/On-board Education Officer (for teachers to go on scientific expeditions on board): \url{http://joidesresolution.org/node/453}
			\end{enumerate}
		\item Integrated Ocean Drilling Program (IODP) -- IODP United States Implementing Organization (IODP-USIO): \vspace{-0.1cm}
			\begin{enumerate} \itemsep -1pt
			\item Teaching Materials: \url{http://www.iodp-usio.org/Education/educ.html}
			\end{enumerate}
		\item Deep Earth Academy (includes suggested ``curriculum and classroom activities for kindergarten through college level''): \vspace{-0.1cm}
			\begin{enumerate} \itemsep -1pt
			\item \url{http://www.oceanleadership.org/education/deep-earth-academy/}
			\item For Educators: \url{http://www.oceanleadership.org/education/deep-earth-academy/educators/}
			\end{enumerate}
		\end{enumerate}
	\item Virginia Institute of Marine Science (College of William and Mary): \vspace{-0.2cm}
		\begin{enumerate} \itemsep -2pt
		\item Bridge Ocean Education Teacher Resource Center: \url{http://web.vims.edu/bridge/?svr=www#}
		\end{enumerate}
	\item American Geological Institute: \vspace{-0.2cm}
		\begin{enumerate} \itemsep -2pt
		\item Awards for teachers: \url{http://www.agiweb.org/education/awards/index.html}
		\item Edward C. Roy, Jr. Award For Excellence in K-8 Earth Science Teaching (for middle school teachers in the US who are teaching earth science): \url{http://www.agiweb.org/education/awards/ed-roy/}
		\item Presidential Awards for Excellence in Mathematics \& Science Teaching, PAEMST (for kindergarten and K-12 teachers in the US who are teaching students about STEM fields): \url{http://www.agiweb.org/education/awards/paemst.html}
		\item National Association of Geoscience Teachers (NAGT) Outstanding Earth Science Teacher Award: \url{http://www.agiweb.org/education/awards/nagt.html}
		\item American Association of Petroleum Geologists' (AAPG) National Earth Science Teacher of the Year Award: \url{http://www.agiweb.org/education/awards/aapg.html}
		\item Curriculum Materials and Activities: \url{http://www.agiweb.org/education/curriculum/index.html}
		\item K-12 Professional Development Programs: \url{http://www.agiweb.org/education/pd/index.html}
		\item Educational Resources: \url{http://www.agiweb.org/education/resource/index.html}
		\end{enumerate}
	\item Institute for Broadening Participation: \vspace{-0.2cm}
		\begin{enumerate} \itemsep -2pt
		\item PathwaysToScience.org: \vspace{-0.1cm}
			\begin{enumerate} \itemsep -1pt
			\item For K-12 teachers (resources to encourage students to be interested in STEM): \url{http://www.pathwaystoscience.org/Teachers.asp}
			\end{enumerate}
		\end{enumerate}
	\item National Science Foundation: \vspace{-0.2cm}
		\begin{enumerate} \itemsep -2pt
		\item The National Science Digital Library (NSDL): \vspace{-0.1cm}
			\begin{enumerate} \itemsep -1pt
			\item Resources for K-12 Teachers: \url{http://nsdl.org/resources_for/k12_teachers/}
			\end{enumerate}
		\end{enumerate}
	\item National Academy of Engineering, NAE: \vspace{-0.2cm}
		\begin{enumerate} \itemsep -2pt
		\item NAE Grand Challenges: \vspace{-0.1cm}
			\begin{enumerate} \itemsep -1pt
			\item Includes a list of NAE Grand Challenges, which indicate some of the challenges faced by people on a global scale that can be partially solved by engineers. This can be used to get children and youths to be excited about engineering. 
			\item NAE Grand Challenges: \vspace{-0.1cm}
				\begin{itemize} \itemsep -1pt
				\item Make solar energy economical
				\item Provide energy from fusion
				\item Develop carbon sequestration methods
				\item Manage the nitrogen cycle
				\item Provide access to clean water
				\item Restore and improve urban infrastructure
				\item Advance health informatics
				\item Engineer better medicines
				\item Reverse-engineer the brain
				\item Prevent nuclear terror
				\item Secure cyberspace
				\item Enhance virtual reality
				\item Advance personalized learning
				\item Engineer the tools of scientific discovery
				\end{itemize}
			\item \url{http://www.engineeringchallenges.org/}
			\end{enumerate}
		\item NAE Grand Challenge K12 Partners Program: \vspace{-0.1cm}
			\begin{enumerate} \itemsep -1pt
			\item Can be used by schools/teachers to raise awareness of global challenges among students and to encourage students to plan career paths to tackle these challenges
			\item 5-Part Make it Happen Plan (includes suggested activities for students in elementary school to learn about basic science and engineering concepts that are relevant to solve the NAE grand challenges): \url{http://www.grandchallengek12.org/5-part-plan}
			\item \url{http://www.grandchallengek12.org/about}
			\end{enumerate}
		\item {\it National Academy of Engineering}'s technological literacy program for people (students, parents, and educators) to learn more about technology: \url{http://www.nae.edu/nae/techlithome.nsf}
		\end{enumerate}
	\item Women in Technology (WIT): \vspace{-0.2cm}
		\begin{enumerate} \itemsep -2pt
		\item Girls In Technology (GIT): \vspace{-0.1cm}
			\begin{enumerate} \itemsep -1pt
			\item Get Involved: \vspace{-0.1cm}
				\begin{itemize} \itemsep -1pt
				\item \url{http://www.girlsintechnology.org/getinvolved.cfm}
				\item Teacher: teach girls about IT as an after-school activity or in a summer camp session
				\item Assistant Teacher: Assist instructors in GIT sessions, after-school activities, or summer camp sessions
				\item Develop Curriculum: Develop a curriculum for a supported GIT educational program
				\item Mentor: Mentor a girl in one of [GIT's] supported programs
				\item Job Shadow: ``Let a girl shadow you at work''
				\item Guest Speaker: ``Speak to a group of girls on a topic both you and they enjoy, such as computers, technology, education, how to take apart computers, how to build a web site, etc.''
				\end{itemize}
			\end{enumerate}
		\end{enumerate}
	\item Organization for Economic Co-operation and Development (OECD): \vspace{-0.2cm}
		\begin{enumerate} \itemsep -2pt
		\item Programme for International Student Assessment (PISA): \vspace{-0.1cm}
			\begin{enumerate} \itemsep -1pt
			\item {\it PISA 2009 Results}. Available online at: \url{http://www.oecd.org/document/61/0,3343,en_32252351_32235731_46567613_1_1_1_1,00.html}; last accessed on December 10, 2010. [ Includes suggestions to improve learning outcomes, as well as education policies and practices. ]
			\end{enumerate}
		\end{enumerate}
	\item American Institute of Aeronautics and Astronautics (AIAA): \vspace{-0.2cm}
		\begin{enumerate} \itemsep -2pt
		\item K-12 Educators: \url{http://www.aiaa.org/content.cfm?pageid=208}
		\end{enumerate}
	\item Research Councils UK (RCUK): \vspace{-0.2cm}
		\begin{enumerate} \itemsep -2pt
		\item Biotechnology and Biological Sciences Research Council (BBSRC): \vspace{-0.1cm}
			\begin{enumerate} \itemsep -1pt
			\item Resources for schools and young people: \url{http://www.bbsrc.ac.uk/society/schools/schools-index.aspx}
			\item Teaching resources: publications and web-based activities: \vspace{-0.1cm}
				\begin{itemize} \itemsep -1pt
				\item Primary (ages 5-12) resources: \url{http://www.bbsrc.ac.uk/society/schools/primary/primary-index.aspx}
				\item Secondary (ages 12-16) and post-16 resources: \url{http://www.bbsrc.ac.uk/society/schools/secondary/secondary-index.aspx}
				\end{itemize}
			\end{enumerate}
		\end{enumerate}
	\item Nuffield Foundation: \vspace{-0.2cm}
		\begin{enumerate} \itemsep -2pt
		\item Education: \url{http://www.nuffieldfoundation.org/education}
		\item Teachers: \vspace{-0.1cm}
			\begin{enumerate} \itemsep -1pt
			\item (Excellent) resources in science and mathematics: \url{http://www.nuffieldfoundation.org/teachers}
			\item \url{http://www.nuffieldfoundation.org/teachers-0}
			\end{enumerate}
		\end{enumerate}
	\item Wellcome Trust: \vspace{-0.2cm}
		\begin{enumerate} \itemsep -2pt
		\item Education resources: \url{http://www.wellcome.ac.uk/Education-resources/index.htm}
		\item {\it yourgenome.org}: \vspace{-0.1cm}
			\begin{enumerate} \itemsep -1pt
			\item \url{http://www.yourgenome.org/}
			\item Resources for teachers about genomics: \url{http://www.yourgenome.org/landing_teachers.shtml}
			\end{enumerate}
		\item Network of Science Learning Centers (Science Learning Centers): \vspace{-0.1cm}
			\begin{enumerate} \itemsep -1pt
			\item \url{https://www.sciencelearningcentres.org.uk/}
			\item Awards and Bursaries: \vspace{-0.1cm}
				\begin{itemize} \itemsep -1pt
				\item \url{https://www.sciencelearningcentres.org.uk/centres/national/awards-and-bursaries}
				\item \url{https://www.sciencelearningcentres.org.uk/about/impact-awards}
				\end{itemize}
			\item Resource collections: \url{https://www.sciencelearningcentres.org.uk/resources}
			\item Curriculum resources for primary, secondary, and tertiary education: \url{https://www.sciencelearningcentres.org.uk/curriculum}
			\end{enumerate}
		\end{enumerate}
	\end{enumerate}
%%%%%%%%%%%%%%%%%%%%%%%%%%%%%%%%%%%%%%%%
%%%%%%%%%%%%%%%%%%%%%%%%%%%%%%%%%%%%%%%%
\item Underrepresented minorities: \vspace{-0.3cm}
	\begin{enumerate} \itemsep -2pt
	\item University of Washington: \vspace{-0.2cm}
		\begin{enumerate} \itemsep -2pt
		\item Department of Computer Science and Engineering: \vspace{-0.1cm}
			\begin{enumerate} \itemsep -1pt
			\item {\it AccessComputing}: \vspace{-0.1cm}
				\begin{itemize} \itemsep -1pt
				\item \url{http://www.washington.edu/accesscomputing/}
				\item Has resources to help students with disabilities to pursue ``undergraduate and graduate degrees and careers in computing fields''.
				\item It runs the ``Summer Academy for Advancing Deaf \& Hard of Hearing in Computing'' for youths who are hearing impaired: \url{http://www.washington.edu/accesscomputing/dhh/academy/index.html}
				\end{itemize}
			\end{enumerate}
		\end{enumerate}
	%%%%%%%%%%%%%%%%%%%%%%%%%
	\item Engineer Girl: \vspace{-0.2cm}
		\begin{enumerate} \itemsep -2pt
		\item Resources for students, parents, and teachers to encourage girls to explore careers and educational opportunities in engineering
		\item Created by the National Academy of Sciences and The National Academy of Engineering
		\item Contests for K-12 students: \url{http://www.engineergirl.org/?id=4436}
		\item \url{http://www.engineergirl.org/}
		\end{enumerate}
	\item Engineering Your Life: \url{http://www.engineeryourlife.org/}
	\item GirlGeeks: \url{http://www.girlgeeks.org/}
	\item {\it Women in Science, Technology, Engineering, and Mathematics ON THE AIR!}: \vspace{-0.2cm}
		\begin{enumerate} \itemsep -2pt
		\item Audio resources that describe stories about women in science, technology, engineering, and mathematics (STEM) fields
		\item \url{http://www.womeninscience.org/}
		\end{enumerate}
	\item {\it Women Scientists in History}: \url{http://www.hypatiamaze.org/}
	\item Association for Women in Mathematics (AWM): \vspace{-0.2cm}
		\begin{enumerate} \itemsep -2pt
		\item \url{http://www.awm-math.org/}
		\item Education: \vspace{-0.1cm}
			\begin{enumerate} \itemsep -1pt
			\item \url{http://sites.google.com/site/awmmath/awm-resources/education}
			\item Includes information for students in middle school, high school, college and university (including graduate students). It also includes information for parents and teachers/educators.
			\end{enumerate}
		\item Women in Math, Science, and Society: \url{http://sites.google.com/site/awmmath/women-in-math-science-and-society}
		\item Essay contest on biographies of contemporary women in mathematics: \url{http://sites.google.com/site/awmmath/programs/essay-contest}
		\end{enumerate}
	\item Women in Technology (WIT): \vspace{-0.2cm}
		\begin{enumerate} \itemsep -2pt
		\item Girls in Technology: \vspace{-0.1cm}
			\begin{enumerate} \itemsep -1pt
			\item \url{http://www.girlsintechnology.org/}
			\item WIT Education Foundation: provides educational programs for girls in technology
			\item TeamBusiness Fundraiser: ``A combined fundraiser and program for girls in Grades 9-12 across the Metro DC area. Each year, up to forty girls participate with mentors and WIT volunteers in a full-day business simulation workshop conducted by TeamBusiness USA. The teams competed as companies, learning how to run a technology company in a fun and exciting simulation environment.''
			\item Hispanic Youth Foundation: ``In 2005, GIT established a partnership with the Hispanic Youth Foundation (HYF) and provided a grant to fund HYF�s innovative Laptops for Learning Dollars program, providing laptops and Internet connections for elementary and middle school students and their families in Arlington County and the City of Manassas.''
			\item Empower Girls -- CLCP Clubs: ``Empower Girls after-school programs were held at Hybla Valley Elementary School and Sacramento Community Center. GIT/WITEF provided funding to run these programs in conjunction with the Fairfax County Computer Learning Center Partnership (CLCP). The selected centers serve economically challenged communities in Fairfax County.''
			\end{enumerate}
		\end{enumerate}
	%%%%%%%%%%%%%%%%%%%%%%%%%
	\item National Society of Black Engineers (NSBE) competitions for high school/K-12 students: \url{http://www.nsbe.org/Programs/Competitions/NSBE-Jr-.aspx}
	\item The Society of Mexican American Engineers and Scientists (MAES): MAES PreCollege Outreach Programs, \url{http://www.maes-natl.org/index.php?module=ContentExpress&func=display&ceid=16&meid=236}
	\item {\it Center for the Advancement of Hispanics in Science and Engineering Education} (CAHSEE): \vspace{-0.2cm}
		\begin{enumerate} \itemsep -2pt
		\item STEM - The Science, Technology, Engineering \& Mathematics Institute (for students from grades 5 through 11): \url{http://www.cahsee.org/2programs/stem.asp.htm}
		\item YEP - Young Educators Program (fellows would learn how to train students in the aforementioned STEM Institute): \url{http://www.cahsee.org/2programs/yep.asp.htm}
		\item CAYSA - Central American Young Scholar Awards: \url{http://www.cahsee.org/2programs/caysa.asp.htm}. ``The CAYSA ceremonies honor more than 60 Washington, D.C. area high school seniors of Central American descent who have demonstrated remarkable success throughout all four years of high school. Students must be of Central American descent and have at least a 3.0 gpa.''
		\item Scholarships: \url{http://www.cahsee.org/6resources/scholarships.asp.htm}
		\item \url{http://www.cahsee.org/about/about.asp.htm}
		\end{enumerate}
	%%%%%%%%%%%%%%%%%%%%%%%%%
	\item International Computer Science Institute (UC Berkeley): \vspace{-0.2cm}
		\begin{enumerate} \itemsep -2pt
		\item Berkeley Foundation for Opportunities in Information Technology, BFOIT: \vspace{-0.1cm}
			\begin{enumerate} \itemsep -1pt
			\item BFOIT Programs for women and underrepresented minorities (African Americans and Chicanos/Latinos) in middle/high school who are interested in electrical/computer engineering and computer science careers: \url{http://www.bfoit.org/programs.html}
			\end{enumerate}
		\end{enumerate}
	\item Institute for Broadening Participation: \vspace{-0.2cm}
		\begin{enumerate} \itemsep -2pt
		\item PathwaysToScience.org: \vspace{-0.1cm}
			\begin{enumerate} \itemsep -1pt
			\item PathwaysToScience.org is a portal website supporting pathways to the STEM fields: science, technology, engineering, and mathematics.
			\item Particular emphasis is placed on connecting traditionally underrepresented groups with STEM programs and resources, including funding and mentoring opportunities. 
			\item For K-12 students: \url{http://www.pathwaystoscience.org/K12.asp}
			\item STEM Resources by Institution (colleges, universities, and US national research laboratories): \url{http://www.pathwaystoscience.org/Institution.asp}
			\item profiles of people and programs in STEM: \vspace{-0.3cm}
				\begin{itemize} \itemsep -2pt
				\item \url{http://www.pathwaystoscience.org/Profiles.asp}
				\item Find out about the career paths of underrepresented minorities in STEM
				\item Find out about programs that are offered by institutions for underrepresented minorities in STEM
				\end{itemize}
			\item Directory of partners (organizations that cooperate with or support the Institute for Broadening Participation): \url{http://www.pathwaystoscience.org/Partners.asp}
			\item Additional resources: \url{http://www.pathwaystoscience.org/Ideaexchange.asp}
			\end{enumerate}
		\item Maine Pathways to STEM (Science, Technology, Engineering \& Mathematics): \vspace{-0.1cm}
			\begin{enumerate} \itemsep -1pt
			\item \url{http://www.mainestem.org/}
			\item K-12 Teachers \& University Faculty: \url{http://www.mainestem.org/METeachersFaculty.asp}
			\item K-12 STEM Resources: \url{http://www.mainestem.org/MEK12.asp}
			\end{enumerate}
		\end{enumerate}
	\item Building Engineering and Science Talent, BEST: \vspace{-0.2cm}
		\begin{enumerate} \itemsep -2pt
		\item \url{http://www.bestworkforce.org/}
		\item Publications: \url{http://www.bestworkforce.org/publications.htm}
		\item List of programs to help underrepresented minority students in K-12 schools explore careers in STEM: \url{http://www.bestworkforce.org/links.htm}
		\end{enumerate}
	\item American Indian Science and Engineering Society (AISES): \vspace{-0.2cm}
		\begin{enumerate} \itemsep -2pt
		\item Pre-college programs: \vspace{-0.1cm}
			\begin{enumerate} \itemsep -1pt
			\item \url{http://www.aises.org/Programs}
			\item Resources: \url{http://www.aises.org/Programs/Resources}
			\end{enumerate}
		\end{enumerate}
	\end{enumerate}
\end{enumerate}







%%%%%%%%%%%%%%%%%%%%%%%%%%%%%%%%%%%%%%%%%%%
\subsection{Science \& Engineering Outreach for Undergraduates, Grad Students, \& Postdocs}
\label{stemoutreachcollegegradsch}


Science, mathematics, and engineering outreach to undergraduates, graduate students, and postdocs: \vspace{-0.3cm}
\begin{enumerate} \itemsep -4pt
\item Mac Hyman, ``Good Choices for Great Careers in the Mathematical Sciences,'' talk given at 2008 SIAM Annual Meeting. Available at: \url{http://client.blueskybroadcast.com/siam08/hyman/index.html}; last accessed on August 25, 2010. Also, see \url{http://meetings.siam.org/program.cfm?CONFCODE=AN08}, \url{http://www.siam.org/meetings/an08/program.php}, and \url{http://www.siam.org/meetings/an08/}.
\item {\it Accreditation.org}: \vspace{-0.3cm}
	\begin{enumerate} \itemsep -2pt
	\item Information about the accreditation of engineering degree programs around the world
	\item \url{http://www.accreditation.org/}
	\end{enumerate}
\item John Baez, ``How to Learn Math and Physics,'' Department of Mathematics, University of California, Riverside, December 24, 2007. Available at: \url{http://math.ucr.edu/home/baez/books.html}; last accessed on August 28, 2010.
\item {\it MentorNet}: \vspace{-0.3cm}
	\begin{enumerate} \itemsep -2pt
	\item \url{http://www.mentornet.net/}
	\item Enables people to network with scientists, engineers, and professors in Science, Technology, Engineering, and Mathematics (STEM)
	\item Is very supportive of minorities, so that more minorities (particularly underrepresented minorities) can be attracted to STEM careers
	\end{enumerate}
\item {\it The Indus Entrepreneurs (TiE)} for networking among high-tech entrepreneurs, start-up co-founders, venture capitalists, and angel investors: \url{http://www.tie.org/}
\item National Academy of Engineering, NAE: \vspace{-0.3cm}
	\begin{enumerate} \itemsep -2pt
	\item Includes a list of NAE Grand Challenges, which can provide some suggestions for research trajectories
	\item Summit Series on the Grand Challenges: Includes the National Grand Challenges Summits
	\item \url{http://www.engineeringchallenges.org/}
	\end{enumerate}
\item {\it National Society of Professional Engineers}: \vspace{-0.3cm}
	\begin{enumerate} \itemsep -2pt
	\item Student Resources: \vspace{-0.2cm}
		\begin{enumerate} \itemsep -2pt
		\item \url{http://www.nspe.org/Students/Resources/index.html}
		\item An Employment Guidelines Checklist for the Engineer Job Applicant: \url{http://www.nspe.org/Students/Resources/checklist.html}
		\end{enumerate}
	\item Career Center: \url{http://www.nspe.org/CareerCenter/index.html}
	\item A Sightseer's Guide to Engineering: \url{http://www.engineeringsights.org/}
	\end{enumerate}
\item {\it JustGarciaHill} ``Study Skills for Budding Scientists'': \url{http://www.justgarciahill.org/index.php/science-study-skills.html}
\item {\it NASA} resources for students: \vspace{-0.3cm}
	\begin{enumerate} \itemsep -2pt
	\item \url{http://www.nasa.gov/audience/forstudents/index.html}
	\item NASA University Student Launch Initiative, or USLI: \url{http://www.nasa.gov/offices/education/programs/descriptions/University_Student_Launch_Initiative.html}
	\end{enumerate}
\item {\it iTunes U}: \vspace{-0.3cm}
	\begin{enumerate} \itemsep -2pt
	\item {\it iTunes} is required to listen to or watch these lectures, talks, and presentations.
	\item Access {\it iTunes U} at: \url{http://www.apple.com/education/itunes-u/} or \url{http://deimos3.apple.com/indigo/main/main.html?v0=WWW-AMUS-ITUNESU070521-N48LX}
	\item {\it iTunes U} is a set of webcast and podcasts, where you can easily find audio and video clips for lectures, seminars, announcements, virtual tours, and so on. For example, some professors from schools like MIT or Berkeley will provide audio/video clips of their lectures on {\it iTunes U}.
	\item This can help in exploring different majors before a college student declares her/his major(s). If a student is not sure if she/he wants to double major in deaf studies and linguistics, this student can check out some linguistics lectures from her/his (preferred) college/university, if it uses {\it iTunes U}, or those from other universities.
	\end{enumerate}
\item Harvey Mudd College: \vspace{-0.3cm}
	\begin{enumerate} \itemsep -2pt
	\item Francis Edward Su, {\it Math Fun Facts!}, Department of Mathematics, Harvey Mudd College: \url{http://www.math.hmc.edu/funfacts/}
	\end{enumerate}
\item Engineering Pathway: \url{http://www.engineeringpathway.com/ep/index.jhtml}
\item Rochester Institute of Technology, ``Biology \& Biotechnology Paid Co-op/Internships for 2011,'' Department of Biological Sciences, Rochester Institute of Technology: \url{http://people.rit.edu/gtfsbi/Symp/summer.htm}
\item {\it Mathematical Association of America (MAA)} information on educational pathways and career opportunities: \vspace{-0.3cm}
	\begin{enumerate} \itemsep -2pt
	\item Undergraduate Students: \url{http://www.maa.org/students/undergrad/}
	\item Graduate Students: \url{http://www.maa.org/students/grad/}
	\item Underrepresented Groups: \url{http://www.maa.org/programs/underrep.html}
	\item Mathematical Association of America (MAA) MathFest (for students in mathematics): \url{http://www.maa.org/mathfest/}
	\item MAA Online Columns: \url{http://www.maa.org/news/columns.html}
	\end{enumerate}
\item New Zealand Institute of Mathematics and its Applications (NZIMA): \vspace{-0.3cm}
	\begin{enumerate} \itemsep -2pt
	\item {\it MathsReach}: Careers (information about careers based on a higher education in mathematics or related field): \url{http://www.mathsreach.org/Careers}
	\end{enumerate}
\item {\it Engineers Dedicated to a Better Tomorrow (a.k.a., DedicatedEngineers)}: \vspace{-0.3cm}
	\begin{enumerate} \itemsep -2pt
	\item [Resources for] College Students and Faculty/Staff Members: \url{http://www.dedicatedengineers.org/intro_for_college.htm}
	\item \url{http://www.dedicatedengineers.org/}
	\end{enumerate}
\item American Institute of Physics: \vspace{-0.3cm}
	\begin{enumerate} \itemsep -2pt
	\item GradschoolShopper.com: \vspace{-0.2cm}
		\begin{enumerate} \itemsep -2pt
		\item \url{http://www.gradschoolshopper.com/}
		\item ``Find information on graduate programs in physics, astronomy, and other physical sciences''
		\end{enumerate}
	\item Career guidance for high school and undergraduate students: \url{http://www.aip.org/statistics/trends/career.html}
	\item American Geophysical Union: \vspace{-0.2cm}
		\begin{enumerate} \itemsep -2pt
		\item Diversity Programs: \url{http://www.agu.org/education/diversity_programs/}
		\end{enumerate}
	\end{enumerate}
\item {\it icademic.org} resources for the life sciences and engineering: \url{http://www.icademic.org/}
\item The Oceanography Society: \vspace{-0.3cm}
	\begin{enumerate} \itemsep -2pt
	\item Hands-On Oceanography: peer-reviewed activities appropriate for undergraduate and/or graduate classes in oceanography, \url{http://www.tos.org/hands-on/index.html}
	\end{enumerate}
%%%%%%%%%%%%%%%%%%%%%%%%%%%%%%%%%%%%%%%
%%%%%%%%%%%%%%%%%%%%%%%%%%%%%%%%%%%%%%%
\item outreach activities (including mentoring) to students in K-12: \vspace{-0.3cm}
	\begin{enumerate} \itemsep -2pt
	\item Research Councils UK (RCUK): \vspace{-0.2cm}
		\begin{enumerate} \itemsep -2pt
		\item Researchers in Residence (RinR): \vspace{-0.1cm}
			\begin{enumerate} \itemsep -1pt
			\item \url{http://www.researchersinresidence.ac.uk/cms/}
			\item \url{http://www.researchersinresidence.ac.uk/cms/researchers/}
			\item Mentor middle and high school students who are job shadowing (observing you first-hand) in your research activities for up to a week, so that they can learn what doing research in your research area is like. You should explain in laypeople's terms what your research is about. That is, be a mentor for the externships of middle and high school students.
			\end{enumerate}
		\end{enumerate}
	\end{enumerate}
%%%%%%%%%%%%%%%%%%%%%%%%%%%%%%%%%%%%%%%
%%%%%%%%%%%%%%%%%%%%%%%%%%%%%%%%%%%%%%%
\item competitions: \vspace{-0.3cm}
	\begin{enumerate} \itemsep -2pt
	\item Invent Now, Inc.: \vspace{-0.2cm}
		\begin{enumerate} \itemsep -2pt
		\item Collegiate Inventors Competition: \url{http://www.invent.org/collegiate/} [ Resources for {\color{blue} Patent Search Strategy} are available. \colorbox{blue}{\bf This is the ultimate competition for US students in science and engineering.} ]
		\end{enumerate}
	\item INFORMS Doing Good with Good OR - Student Competition: \vspace{-0.2cm}
		\begin{enumerate} \itemsep -2pt
		\item Doing Good with Good OR-Student Competition is held each year to identify and honor outstanding projects in the field of operations research and the management sciences conducted by a student or student group that have a significant societal impact.
		\item \url{http://www.informs.org/Recognize-Excellence/INFORMS-Prizes-Awards/Doing-Good-with-Good-OR}
		\end{enumerate}
	\item AWM Essay Contest: Biographies of Contemporary Women in Mathematics; see \url{http://www.awm-math.org/biographies/contest.html}
	\item American Society of Mechanical Engineers (ASME): \vspace{-0.2cm}
		\begin{enumerate} \itemsep -2pt
		\item Student Design Competition: \url{http://www.asme.org/Events/Contests/DesignContest/Student_Design_Competition.cfm}
		\item ASME Student Mechanism and Robot Design Competition: \url{http://www.asme.org/Events/Contests/Student_Mechanism_Robot_2.cfm}
		\end{enumerate}
	\item American Institute of Chemical Engineers (AIChE) competitions: \url{http://www.aiche.org/Students/Awards/index.aspx}
	\item Association for Unmanned Vehicle Systems International (AUVSI): \vspace{-0.2cm}
		\begin{enumerate} \itemsep -2pt
		\item AUVSI Student Competitions: \vspace{-0.1cm}
			\begin{enumerate} \itemsep -1pt
			\item \url{http://www.auvsi.org/AUVSI/AUVSI/Home/Default.aspx}, or \url{http://www.auvsi.org/}
			\item Annual Intelligent Ground Vehicle Competition (IGVC): \url{http://www.igvc.org/}
			\item Annual Student Unmanned Air System (SUAS) Competition: \url{http://65.210.16.57/studentcomp2010/default.html}
			\item International Aerial Robotics Competition (IARC): \url{http://iarc.angel-strike.com/}
			\item AUVSI and ONR's International Autonomous Surface Vehicle (ASV) Competition [ASVC]
			\item AUVSI Foundation and ONR's (U.S. Office of Naval Research) 4th International RoboBoats Competition: \url{http://www.auvsifoundation.org/AUVSI/FOUNDATION/Competitions/ASVCompetition/Default.aspx?C=00000000-0000-0000-0000-000000000000}
			\item AUVSI Foundation and ONR's (U.S. Office of Naval Research) International RoboSub Competition (or AUVSI and ONR's International Autonomous Underwater Vehicle Competition): \url{http://www.auvsifoundation.org/AUVSI/FOUNDATION/Competitions/AUVCompetition/Default.aspx}
			\item ONR: U.S. Office of Naval Research
			\end{enumerate}
		\end{enumerate}
	\item American Institute of Aeronautics and Astronautics (AIAA): \vspace{-0.2cm}
		\begin{enumerate} \itemsep -2pt
		\item Design Competitions: \url{http://www.aiaa.org/content.cfm?pageid=210}
		\end{enumerate}
	\item National Aeronautics and Space Administration: \vspace{-0.2cm}
		\begin{enumerate} \itemsep -2pt
		\item NASA's Langley Research Center: \vspace{-0.1cm}
			\begin{enumerate} \itemsep -1pt
			\item SpaceTech Engineering Design Challenge: \url{http://spacetech.larc.nasa.gov}
			\end{enumerate}
		\end{enumerate}
	\item American Concrete Institute (ACI): \vspace{-0.2cm}
		\begin{enumerate} \itemsep -2pt
		\item Competitions: \url{http://www.concrete.org/STUDENTS/st_competitions.htm}
		\end{enumerate}
	\end{enumerate}
%%%%%%%%%%%%%%%%%%%%%%%%%%%%%%%%%%%%%%%
%%%%%%%%%%%%%%%%%%%%%%%%%%%%%%%%%%%%%%%
\item underrepresented minorities: \vspace{-0.3cm}
	\begin{enumerate} \itemsep -2pt
	\item The Society of Women Engineers: \url{http://societyofwomenengineers.swe.org/}
	\item Association for Women in Science (AWIS): \url{http://www.awis.org/} and \url{http://www.awis.affiniscape.com/displaycommon.cfm?an=1&subarticlenbr=19}
	\item Association for Women in Mathematics (AWM): \vspace{-0.2cm}
		\begin{enumerate} \itemsep -2pt
		\item \url{http://www.awm-math.org/}
		\item Education: \vspace{-0.1cm}
			\begin{enumerate} \itemsep -1pt
			\item \url{http://sites.google.com/site/awmmath/awm-resources/education}
			\item Includes information for students in middle school, high school, college and university (including graduate students). It also includes information for parents and teachers/educators.
			\end{enumerate}
		\item Career advice and opportunities: \url{http://sites.google.com/site/awmmath/awm-resources/career}
		\item Women in Math, Science, and Society: \url{http://sites.google.com/site/awmmath/women-in-math-science-and-society}
		\item Essay contest on biographies of contemporary women in mathematics: \url{http://sites.google.com/site/awmmath/programs/essay-contest}
		\end{enumerate}
	\item Sigma Delta Epsilon-Graduate Women in Science (GWIS): \url{http://www.gwis.org/}
	\item Society of Hispanic Professional Engineers (SHPE): \vspace{-0.2cm}
		\begin{enumerate} \itemsep -2pt
		\item Advancing Hispanic Excellence in Technology, Engineering, Math and Science (AHETEMS) Foundation: \url{http://www.ahetems.org/}
		\item AHETEMS Scholarship Program: \url{http://www.ahetems.org/scholarships/}
		\item Graduate \& Young Professional Fellowship Program (encourage young professionals to engage in {\bf public policy}): \url{http://www.ahetems.org/graduate/graduate-young-professional-fellowship-program/}
		\item SHPE/GEM Fellowship (for graduate students in STEM at GEM Member Universities): \url{http://www.ahetems.org/graduate/shpe-gem-graduate-award/}. See \url{http://www.gemfellowship.org/gem-universities/university-members} for a list of GEM member universities.
		\item Internship opportunities: \url{http://www.ahetems.org/scholar-internships/}
		\item \url{http://oneshpe.shpe.org/wps/portal/national}
		\end{enumerate}
	\item National Society of Black Engineers (NSBE): \vspace{-0.2cm}
		\begin{enumerate} \itemsep -2pt
		\item Scholarships: \url{http://www.nsbe.org/Programs/Scholarships.aspx}
		\item Competitions for undergraduates and graduate students: \url{http://www.nsbe.org/Programs/Competitions/Collegiate.aspx}
		\item \url{http://www.nsbe.org/}
		\end{enumerate}
	\item The Society of Mexican American Engineers and Scientists (MAES): \vspace{-0.2cm}
		\begin{enumerate} \itemsep -2pt
		\item MAES Undergraduate and Graduate Outreach Programs (including ``GRE/Graduate Application Fee Waivers''): \url{http://www.maes-natl.org/index.php?module=ContentExpress&func=display&ceid=90&meid=238}
		\item Scholarships \& Awards: \url{http://www.maes-natl.org/index.php?meid=328}
		\item MAES Scholarship Program: \url{http://www.maes-natl.org/index.php?module=ContentExpress&func=display&ceid=518&meid=241}
		\end{enumerate}
	\item SACNAS (Society for Advancement of Chicanos and Native Americans in Science): \vspace{-0.2cm}
		\begin{enumerate} \itemsep -2pt
		\item Scholarships: \url{http://www.sacnas.org/webadindex.cfm?webadcategory_id=7}
		\item Fellowships: \url{http://www.sacnas.org/webadIndex.cfm?webadcategory_id=5}
		\end{enumerate}
	\item {\it Center for the Advancement of Hispanics in Science and Engineering Education} (CAHSEE): \vspace{-0.2cm}
		\begin{enumerate} \itemsep -2pt
		\item YESP - Young Engineers \& Scientists Program: \url{http://www.cahsee.org/2programs/yesp.asp.htm}. ``This program places talented Hispanic college students in the research labs of government agencies.''
		\item Scholarships: \url{http://www.cahsee.org/6resources/scholarships.asp.htm}
		\end{enumerate}
	\item American Geophysical Union: \vspace{-0.2cm}
		\begin{enumerate} \itemsep -2pt
		\item Has a list of organizations for specific underrepresented ethnic-minority groups in the geosciences and physics: \vspace{-0.1cm}
			\begin{enumerate} \itemsep -1pt
			\item \url{http://www.agu.org/education/diversity_programs/}
			\item These organizations may have information about scholarships, fellowships, and educational material for K-12 and college students.
			\end{enumerate}
		\end{enumerate}
	\item Institute for Broadening Participation: \vspace{-0.2cm}
		\begin{enumerate} \itemsep -2pt
		\item Minorities Striving and Pursuing Higher Degrees of Success in Earth System Science (MS PHD'S\textregistered) initiative: \vspace{-0.1cm}
			\begin{enumerate} \itemsep -1pt
			\item \url{http://www.msphds.org/}
			\item Prospective Students/Mentees: \url{http://www.msphds.org/prospective.asp}
			\item For MS PHD'S Students: \url{http://www.msphds.org/students.asp}
			\end{enumerate}
		\item PathwaysToScience.org: \vspace{-0.1cm}
			\begin{enumerate} \itemsep -1pt
			\item Resources for undergraduate students: \url{http://www.pathwaystoscience.org/Undergrads.asp}
			\item Resources for graduate students: \url{http://www.pathwaystoscience.org/Grad.asp}
			\item Resources for postdocs: \url{http://www.pathwaystoscience.org/Postdocs_portal.asp}
			\item STEM Resources by Institution (colleges, universities, and US national research laboratories): \url{http://www.pathwaystoscience.org/Institution.asp}
			\item Additional resources: \url{http://www.pathwaystoscience.org/Ideaexchange.asp}
			\end{enumerate}
		\item National Alliance for Doctoral Studies in the Mathematical Sciences: \vspace{-0.1cm}
			\begin{enumerate} \itemsep -1pt
			\item \url{http://www.mathalliance.org/}
			\item Student/Alliance Scholars: \url{http://www.mathalliance.org/scholars.asp}
			\item Alliance Mentors / Alliance Undergraduate Mentors: \url{http://www.mathalliance.org/mentors.asp}
			\item Alliance Programs: \url{http://www.mathalliance.org/programs.asp}
			\end{enumerate}
		\item Alliances for Graduate Education and the Professoriate (AGEP): \vspace{-0.1cm}
			\begin{enumerate} \itemsep -1pt
			\item \url{http://www.agep.us/}
			\end{enumerate}
		\item Maine Pathways to STEM (Science, Technology, Engineering \& Mathematics): \vspace{-0.1cm}
			\begin{enumerate} \itemsep -1pt
			\item \url{http://www.mainestem.org/}
			\item K-12 Teachers \& University Faculty: \url{http://www.mainestem.org/METeachersFaculty.asp}
			\item Graduate \& Undergraduate Students: \url{http://www.mainestem.org/MEUndergradGrad.asp}
			\end{enumerate}
		\end{enumerate}
	\item ARTSI (Advancing Robotics Technology for Societal Impact) Alliance: \vspace{-0.2cm}
		\begin{enumerate} \itemsep -2pt
		\item \url{http://artsialliance.org/}
		\item ``The ARTSI (Advancing Robotics Technology for Societal Impact) Alliance is a collaborative education and research project centered around robotics for healthcare, the arts, and entrepreneurship.  Spelman College, a historically black college (HBCU) for women is leading the alliance in partnership with several other HBCUs and Research I (R1) institutions.''
		\item Summer REU (Research Experience for Undergraduates) program: \url{http://artsialliance.org/Summer-REU-Program}
		\end{enumerate}
	\item Women in Technology (WIT): \vspace{-0.2cm}
		\begin{enumerate} \itemsep -2pt
		\item \url{http://www.womenintechnology.org/index.asp}
		\item WIT Mentor-Prot{\'{e}}g{\'{e}} Program: \url{http://www.womenintechnology.org/content.asp?contentid=59}
		\item {\bf \color{blue} WIT Career Transition Resource Guide}: \url{http://www.womenintechnology.org/content.asp?contentid=146}
		\item Girls In Technology (GIT): \vspace{-0.1cm}
			\begin{enumerate} \itemsep -1pt
			\item Get Involved: \vspace{-0.1cm}
				\begin{itemize} \itemsep -1pt
				\item \url{http://www.girlsintechnology.org/getinvolved.cfm}
				\item Teacher: teach girls about IT as an after-school activity or in a summer camp session
				\item Assistant Teacher: Assist instructors in GIT sessions, after-school activities, or summer camp sessions
				\item Develop Curriculum: Develop a curriculum for a supported GIT educational program
				\item Mentor: Mentor a girl in one of [GIT's] supported programs
				\item Job Shadow: ``Let a girl shadow you at work''
				\item Guest Speaker: ``Speak to a group of girls on a topic both you and they enjoy, such as computers, technology, education, how to take apart computers, how to build a web site, etc.''
				\end{itemize}
			\end{enumerate}
		\end{enumerate}
	\item Arizona State University: \vspace{-0.2cm}
		\begin{enumerate} \itemsep -2pt
		\item {\it Career}WISE: \vspace{-0.1cm}
			\begin{enumerate} \itemsep -1pt
			\item \url{http://careerwise.asu.edu/}
			\item Helpful resources for female graduate/Ph.D. students in science and engineering.
			\end{enumerate}
		\end{enumerate}
	\item American Indian Science and Engineering Society (AISES): \vspace{-0.2cm}
		\begin{enumerate} \itemsep -2pt
		\item Programs for undergraduates and grad students (including scholarships and internships): \vspace{-0.1cm}
			\begin{enumerate} \itemsep -1pt
			\item \url{http://www.aises.org/Programs}
			\item Resources: \url{http://www.aises.org/Programs/Resources}
			\end{enumerate}
		\end{enumerate}
	\end{enumerate}
\end{enumerate}




%%%%%%%%%%%%%%%%%%%%%%%%%%%%%%%%%%%%%%%%%%%
\subsection{Other Science and Engineering Outreach}
\label{otherstemoutreach}

Other Science and Engineering Outreach: \vspace{-0.3cm}
\begin{enumerate} \itemsep -4pt
\item Frontiers of Engineering (networking event for mid-career engineers): \url{http://www.naefrontiers.org/}
\item Consortium for Ocean Leadership: \vspace{-0.3cm}
	\begin{enumerate} \itemsep -2pt
	\item Resources for scientists in the marine sciences to use in outreach activities: \url{http://www.oceanleadership.org/education/deep-earth-academy/scientists/}
	\end{enumerate}
\item The Oceanography Society: \vspace{-0.3cm}
	\begin{enumerate} \itemsep -2pt
	\item Education and Public Outreach (EPO): A Guide for Scientists [material that scientists and professors can use for outreach activities], \url{http://www.tos.org/epo_guide/index.html}
	\end{enumerate}
\item The Joy McCann Foundation: \vspace{-0.3cm}
	\begin{enumerate} \itemsep -2pt
	\item McCann Scholar (for professors in medicine, science, and nursing): \url{http://www.mccannfoundation.org/scholars.htm}
	\item The Joy McCann Professorship for Women in Medicine: \url{http://www.mccannfoundation.org/medicine.htm}
	\end{enumerate}
\item U.S. National Academies: \vspace{-0.3cm}
	\begin{enumerate} \itemsep -2pt
	\item International Activities of the U.S. National Academies -- Science, Engineering \& Medicine: Working toward a better world: \vspace{-0.2cm}
		\begin{enumerate} \itemsep -2pt
		\item \url{http://sites.nationalacademies.org/International/}
		\item Solving the grand challenges: \vspace{-0.1cm}
			\begin{enumerate} \itemsep -1pt
			\item Energy and the Environment
			\item Global Health
			\item Water Resources
			\item Agriculture and Food Security
			\item International Security
			\item Population
			\end{enumerate}
		\item Help other countries build/improve their capacities: \vspace{-0.1cm}
			\begin{enumerate} \itemsep -1pt
			\item Cooperative Program with Pakistan 
			\item African Science Academies 
			\item Visiting Math Lecturer Program in Cambodia 
			\item Humanitarian Relief Efforts
			\item Improved Road Safety
			\item Science-based Decision Making for Sustainability
			\item Science Academies' Input to G8 Summits
			\end{enumerate}
		\item Scientific Cooperation: \vspace{-0.1cm}
			\begin{enumerate} \itemsep -1pt
			\item Building Bridges in the Middle East
			\item Cooperation with Iran
			\item Human Rights
			\item Frontiers of Science and Engineering Symposia
			\item Travel Grants
			\item International Conference on Women's Issues in Transportation
			\end{enumerate}
		\item Advising the U.S. Government: \vspace{-0.1cm}
			\begin{enumerate} \itemsep -1pt
			\item Science \& Technology in Foreign Policy
			\item Health 
			\item Science and Security
			\end{enumerate}
		\end{enumerate}
	\end{enumerate}
\item National Academy of Engineering: \vspace{-0.3cm}
	\begin{enumerate} \itemsep -2pt
	\item The Charles Stark Draper Prize (``to recognize innovative engineering achievements and their reduction to practice in ways that have led to important benefits and significant improvement in the well being and freedom of humanity''): \url{http://www.draperprize.org/}
	\item NAE Grand Challenge Scholars Program: \url{http://www.grandchallengescholars.org/}
	\end{enumerate}
\item United States Department of Defense (DoD): \vspace{-0.3cm}
	\begin{enumerate} \itemsep -2pt
	\item National Defense Education Program; Defense Advanced Research Projects Agency (DARPA): \vspace{-0.2cm}
		\begin{enumerate} \itemsep -2pt
		\item Resource for scientists and engineers to mentor youths, so that they would look into pursuing careers in science and engineering: \url{http://www.ndep.us/GetInvoSci.aspx}
		\item STEM Learning Modules (SLM): \vspace{-0.1cm}
			\begin{enumerate} \itemsep -1pt
			\item \url{http://www.ndep.us/ProgSLM.aspx}
			\item Help educators develop programs in science and engineering in K-12 institutions, so that youths would be encouraged to explore careers in science and engineering
			\end{enumerate}
		\end{enumerate}
	\end{enumerate}
\item Hewlett-Packard Development Company: \vspace{-0.3cm}
	\begin{enumerate} \itemsep -2pt
	\item HP Catalyst Initiative (grants for STEM education in colleges and universities): \url{http://www.hp.com/hpinfo/socialinnovation/catalyst.html}
	\item HP EdTech Innovators Award (for higher educational institutions that integrate IT into the curricular): \url{http://www.hp.com/hpinfo/socialinnovation/edtech.html}
	\end{enumerate}
\item The William and Flora Hewlett Foundation (Hewlett Foundation): \vspace{-0.3cm}
	\begin{enumerate} \itemsep -2pt
	\item Funding Programs: \url{http://www.hewlett.org/programs}
	\item Grantseekers: \url{http://www.hewlett.org/grants/grantseekers}
	\end{enumerate}
\item The Sloan Consortium (Sloan-C): \vspace{-0.3cm}
	\begin{enumerate} \itemsep -2pt
	\item Sloan-C Awards (for recognizing outstanding work in the field of online education) and Sloan-C Fellows: \url{http://sloanconsortium.org/aboutus/awards}
	\item Mayadas Leadership Award in Online Education: \url{http://sloanconsortium.org/mayadas_award}
	\end{enumerate}
\item W.K. Kellogg Foundation: \vspace{-0.3cm}
	\begin{enumerate} \itemsep -2pt
	\item Grant database: \url{http://www.wkkf.org/grants/grants-database.aspx}
	\end{enumerate}
\item Hewlett-Packard Company: \vspace{-0.3cm}
	\begin{enumerate} \itemsep -2pt
	\item HP community investment for education, economic development, and the environment: \url{http://www.hp.com/hpinfo/socialinnovation/focus.html}
	\item Entrepreneurship education: \vspace{-0.2cm}
		\begin{enumerate} \itemsep -2pt
		\item \url{http://www.hp.com/hpinfo/globalcitizenship/society/social/entrepreneurship.html}
		\item HP Graduate Entrepreneurship Training through IT (GET-IT)
		\item HP Entrepreneurship Learning Program (HELP)
		\end{enumerate}
	\item HP Innovations in Education grants: \url{http://www.hp.com/hpinfo/globalcitizenship/society/social/innovations.html}
	\end{enumerate}
\item General Electric Company: \vspace{-0.3cm}
	\begin{enumerate} \itemsep -2pt
	\item GE Foundation: \vspace{-0.2cm}
		\begin{enumerate} \itemsep -2pt
		\item Developing Futures\texttrademark\ in Education program (which encompasses the GE College Bound Program): \url{http://www.ge.com/foundation/developing_futures_in_education/index.jsp}
		\item Environment, health and safety, and health industry training programs (outside the US): \url{http://www.ge.com/foundation/international_programs/training.jsp}
		\item Student, education and scholarship initiatives: \url{http://www.ge.com/foundation/international_programs/education_initiatives.jsp}
		\end{enumerate}
	\end{enumerate}
\item The GRAMMY Foundation: \vspace{-0.3cm}
	\begin{enumerate} \itemsep -2pt
	\item GRAMMY Foundation Grants: \vspace{-0.2cm}
		\begin{enumerate} \itemsep -2pt
		\item \url{http://www2.grammy.com/GRAMMY_Foundation/Grants/}
		\item It funds {\bf Scientific Research Projects} as well as {\it Archiving And Preservation Projects}.
		\item Concerning scientific research projects: ``The GRAMMY Foundation Grant Program awards grants to organizations and individuals to support research on the impact of music on the human condition. Examples might include the study of the effects of music on mood, cognition and healing, as well as the medical and occupational well-being of music professionals and the creative process underlying music.'' [ E.g., look at music therapy as a possible research topic/area. ]
		\end{enumerate}
	\end{enumerate}
\item The Dana Foundation: \vspace{-0.3cm}
	\begin{enumerate} \itemsep -2pt
	\item \url{http://www.dana.org/grants/}
	\item Has grants for: \vspace{-0.2cm}
		\begin{enumerate} \itemsep -2pt
		\item Brain and Immuno-Imaging
		\item Clinical Neuroscience
		\item Human Immunology
		\item Neuroimmunology of Brain Infections and Cancers
		\end{enumerate}
		\item Deadlines and Requests for Proposals (RFP): \url{http://www.dana.org/grants/deadlines.aspx}
	\end{enumerate}
%%%%%%%%%%%%%%%%%%%%%%%%%%%%%%%%%%%%%%%
% underrepresented minorities
\item Institute for Broadening Participation: \vspace{-0.3cm}
	\begin{enumerate} \itemsep -2pt
	\item PathwaysToScience.org: \vspace{-0.2cm}
		\begin{enumerate} \itemsep -2pt
		\item Resources for faculty and administrators (to facilitate STEM outreach activities as well as the recruitment of underrepresented minorities to the student body and faculty): \url{http://www.pathwaystoscience.org/Faculty.asp}
		\end{enumerate}
	\end{enumerate}
\item National Center for Women \& Information Technology (NCWIT): \vspace{-0.3cm}
	\begin{enumerate} \itemsep -2pt
	\item NCWIT Academic Alliance Seed Fund (for developing and implementing initiatives in colleges and universities to recruit and retain women in computing and information technology): \url{http://www.ncwit.org/work.awards.seed.html}
	\item NCWIT Symons Innovator Award (for outstanding women who have successfully built and funded an IT business): \url{http://www.ncwit.org/work.awards.innovator.html}
	\end{enumerate}
\item Women in Technology (WIT): \vspace{-0.3cm}
	\begin{enumerate} \itemsep -2pt
	\item Girls In Technology (GIT): \vspace{-0.2cm}
		\begin{enumerate} \itemsep -2pt
		\item Get Involved: \vspace{-0.1cm}
			\begin{itemize} \itemsep -1pt
			\item \url{http://www.girlsintechnology.org/getinvolved.cfm}
			\item Teacher: teach girls about IT as an after-school activity or in a summer camp session
			\item Assistant Teacher: Assist instructors in GIT sessions, after-school activities, or summer camp sessions
			\item Develop Curriculum: Develop a curriculum for a supported GIT educational program
			\item Mentor: Mentor a girl in one of [GIT's] supported programs
			\item Job Shadow: ``Let a girl shadow you at work''
			\item Guest Speaker: ``Speak to a group of girls on a topic both you and they enjoy, such as computers, technology, education, how to take apart computers, how to build a web site, etc.''
			\end{itemize}
		\end{enumerate}
	\end{enumerate}
\item European Platform of Women Scientists (EPWS): \vspace{-0.3cm}
	\begin{enumerate} \itemsep -2pt
	\item \url{http://www.epws.org/}
	\item Members: \url{http://www.epws.org/index.php?option=com_content&task=blogcategory&id=134&Itemid=4652}
	\end{enumerate}
\end{enumerate}





Commercializing academic research into products and services via start-ups: \vspace{-0.3cm}
\begin{enumerate} \itemsep -4pt
\item Ben Franklin Technology Partners (BFTP): \vspace{-0.3cm}
	\begin{enumerate} \itemsep -2pt
	\item Innovation Works (IW): \vspace{-0.2cm}
		\begin{enumerate} \itemsep -2pt
		\item For universities in the Pittsburgh metropolitan area
		\item University Innovation Grants (UIGs) / University Grants: \vspace{-0.1cm}
			\begin{enumerate} \itemsep -1pt
			\item For technology validation, market research, prototype development, and intellectual property evaluation
			\item Available online at: \url{http://www.innovationworks.org/OurPrograms/UniversityGrants/tabid/115/Default.aspx}; last accessed on November 14, 2010.
			\end{enumerate}
		\end{enumerate}
	\end{enumerate}
\end{enumerate}









%%%%%%%%%%%%%%%%%%%%%%%%%%%%%%%%%%%%%%%%%%%
\subsection{Electrical and Computer Engineering \& Computer Science Outreach}
\label{ececsoutreach}

Electrical and computer engineering, and computer science outreach: \vspace{-0.3cm}
\begin{enumerate} \itemsep -4pt
\item IEEE: \vspace{-0.3cm}
	\begin{enumerate} \itemsep -2pt
	\item {\it IEEE-USA Salary Service} provides a survey of jobs in electrical and computer engineering: \url{http://www.ieeeusa.org/careers/salary/}
	\item {\it IEEE Santa Clara Valley Section PACE}: Professional Activities Committee for Engineers (PACE); see \url{http://www.ewh.ieee.org/r6/scv/PACE/}
	\item {\it IEEE Santa Clara Valley Section}: \url{http://ewh.ieee.org/r6/scv/} and \url{http://www.ieee.org/scv}
	\item 
	\end{enumerate}
\item Association for Computing Machinery, ACM: \vspace{-0.3cm}
	\begin{enumerate} \itemsep -2pt
	\item Sanjeev Arora, Boaz Barak, and Luca Trevisan, ``Survey Papers and Essays,'' in {\it Theory Matters Wiki: Theoretical Computer Science (TCS) Advocacy Wiki}, SIGACT Committee for the Advancement of Theoretical Computer Science, ACM Special Interest Group on Algorithms and Computation Theory (SIGACT), Association for Computing Machinery, February 25, 2010. Available at: \url{http://theorymatters.org/pmwiki/pmwiki.php?n=Main.SurveyCollection}; last accessed on September 14, 2010.
	\item Online Resources for Graduating Students: \url{http://www.acm.org/membership/student/resources-for-grads}
	\end{enumerate}
\item VLSI design and verification: \vspace{-0.3cm}
	\begin{enumerate} \itemsep -2pt
	\item {\it DVClub} for individuals interested in VLSI verification: \url{http://www.dvclub.org/}
	\item {\it DeepChip.com}: \url{http://www.deepchip.com}
	\end{enumerate}
%%%%%%%%%%%%%%%%%%%%%%%%%%%%%%%
\item undergraduates: \vspace{-0.3cm}
	\begin{enumerate} \itemsep -2pt
	\item {\it Humanitarian FOSS Project}: \vspace{-0.2cm}
		\begin{enumerate} \itemsep -2pt
		\item Where FOSS refers to Free and Open Source Software
		\item For computer science and engineering students
		\item \url{http://www.hfoss.org/}
		\end{enumerate}
	\item {\it SIGDA Design Automation Summer School}: \vspace{-0.2cm}
		\begin{enumerate} \itemsep -2pt
		\item {\it NSF�SRC�SIGDA�DAC Design Automation Summer School}
		\item \url{http://www.sigda.org/dass.html}
		\item Travel grants are provided to defray travel and accommodation expenses
		\end{enumerate}
	\item {\it Young Student Support Program at DAC}: \vspace{-0.2cm}
		\begin{enumerate} \itemsep -2pt
		\item Also known as {\it DAC Young Student Support Program}
		\item \url{http://www.sigda.org/youngstudent.html}
		\item Travel grants are provided to defray travel and accommodation expenses
		\end{enumerate}
	\item {\it ACM Student Research Competition at Design Automation Conference}: \vspace{-0.2cm}
		\begin{enumerate} \itemsep -2pt
		\item Sponsored by {\it Microsoft Research}
		\item \url{http://www.sigda.org/studentcomp.html}
		\item Also, see {\it ACM Student Research Competition} @ \url{http://src.acm.org/}.
		\end{enumerate}
	\item Job database for positions in the Video Game, Animation, VFX, and Software/Technology industries: \url{http://www.creativeheads.net/}
	\end{enumerate}
%%%%%%%%%%%%%%%%%%%%%%%%%%%%%%
\item graduate students: \vspace{-0.3cm}
	\begin{enumerate} \itemsep -2pt
	\item {\it SIGDA Design Automation Summer School}: \vspace{-0.2cm}
		\begin{enumerate} \itemsep -2pt
		\item {\it NSF�SRC�SIGDA�DAC Design Automation Summer School}
		\item \url{http://www.sigda.org/dass.html}
		\item Travel grants are provided to defray travel and accommodation expenses
		\end{enumerate}
	\item {\it Young Student Support Program at DAC}: \vspace{-0.2cm}
		\begin{enumerate} \itemsep -2pt
		\item Also known as {\it DAC Young Student Support Program}
		\item \url{http://www.sigda.org/youngstudent.html}
		\item Travel grants are provided to defray travel and accommodation expenses
		\end{enumerate}
	\item {\it ACM Student Research Competition at Design Automation Conference}: \vspace{-0.2cm}
		\begin{enumerate} \itemsep -2pt
		\item Sponsored by {\it Microsoft Research}
		\item \url{http://www.sigda.org/studentcomp.html}
		\item Also, see {\it ACM Student Research Competition} @ \url{http://src.acm.org/}.
		\end{enumerate}
	\item {\it SIGDA University Booth at DAC}: \vspace{-0.2cm}
		\begin{enumerate} \itemsep -2pt
		\item Or, {\it SIGDA/DAC University Booth}
		\item \url{http://www.sigda.org/ubooth.html}
		\end{enumerate}
	\item {\it SIGDA Ph.D. Forum at DAC}: \vspace{-0.2cm}
		\begin{enumerate} \itemsep -2pt
		\item \url{http://www.sigda.org/phdforum.html}
		\item \url{http://www.sigda.org/daforum/}
		\end{enumerate}
	\item {\it DAC Graduate Scholarship}: \vspace{-0.2cm}
		\begin{enumerate} \itemsep -2pt
		\item {\it A. Richard Newton Graduate Scholarships} to Support Graduate Research and Study
		\item \url{http://www.sigda.org/gradscholarship.html}
		\end{enumerate}
	\end{enumerate}
%%%%%%%%%%%%%%%%%%%%%%%%%%%%%%
\item competitions, and programming contests and challenges: \vspace{-0.3cm}
	\begin{itemize} \itemsep -2pt
	\item {\it SIGDA CADathlon at ICCAD}: \vspace{-0.2cm}
		\begin{enumerate} \itemsep -2pt
		\item \url{http://www.sigda.org/programs/cadathlon/}
		\item \url{http://www.sigda.org/cadathlon.html}
		\item Travel grants are provided to defray travel and accommodation expenses
		\end{enumerate}
	\item ISPD Programming Contest: \url{http://www.ispd.cc/contests/}
	\item ACM International Workshop on Timing Issues in the Specification and Synthesis of Digital Systems (TAU Workshop): \vspace{-0.2cm}
		\begin{enumerate} \itemsep -2pt
		\item Power Grid Simulation Contest: \url{http://www.tauworkshop.com/PREVIOUS/contest_2011.html}
		\end{enumerate}
	\item IEEE Computer Society Simulator Design competition: \url{http://www.computer.org/portal/web/competition}
	\item {\it DAC/ISSCC Student Design Contest}: \vspace{-0.2cm}
		\begin{enumerate} \itemsep -2pt
		\item \url{http://www.dac.com}
		\end{enumerate}
	\item {\it ACM/IEEE International Conference on Formal Methods and Models for Codesign -- Design Contest}: \vspace{-0.2cm}
		\begin{enumerate} \itemsep -2pt
		\item MEMOCODE Hardware/Software Co-Design Contest (MEMOCODE HW/SW co-design contest)
		\item \url{http://www-memocode2010.imag.fr/}
		\item \url{http://memocode2010.csail.mit.edu/redmine/wiki/memocode2010/Results}
		\end{enumerate}
	\item {\it International Low Power Design Contest}: \vspace{-0.2cm}
		\begin{enumerate} \itemsep -2pt
		\item ACM/IEEE International Symposium on Low Power Electronics and Design (ISLPED) -- Design Contest
		\item The International Symposium on Low Power Electronics and Design is holding the International Low Power Design Contest to provide a forum for universities and research organizations to showcase original ``power-aware'' designs and to highlight the innovations and design choices targeted at low power.
		\item The goal is to encourage and highlight design-oriented approaches to power reduction.
		\item \url{http://www.islped.org/2010/index.html}
		\end{enumerate}
	\item {\it University LSI Design Contest @ ASP-DAC}: \vspace{-0.2cm}
		\begin{enumerate} \itemsep -2pt
		\item Application areas or types of circuits of the original LSI circuit designs include (but are not limited to): \vspace{-0.1cm}
			\begin{enumerate} \itemsep -1pt
			\item Analog, RF and Mixed-Signal Circuits
			\item Digital Signal Processing
			\item Microprocessors
			\item Custom ASIC
			\end{enumerate} 
		\item Methods or technology used for implementation include: \vspace{-0.1cm}
			\begin{enumerate} \itemsep -1pt
			\item Full Custom and Cell-Based LSIs
			\item Gate Arrays
			\item FPGA/PLDs.
			\end{enumerate}
		\item \url{http://www.aspdac.com/aspdac2011/cfd/}
		\end{enumerate}
	\item IEEE Programming Challenge at IWLS: \url{http://www.iwls.org/challenge/}
	\item IEEE Asian Solid-State Circuits Conference (A-SSCC) Student Design Contest: \url{http://a-sscc2010.a-sscc.org/contest.html}
	\item {\it VLSI Conference 2011 - Design Contest}: \vspace{-0.2cm}
		\begin{enumerate} \itemsep -2pt
		\item Design/project fields include (but not limited to): \vspace{-0.1cm}
			\begin{enumerate} \itemsep -1pt
			\item Digital Integrated Circuits
			\item Analog Integrated Circuits
			\item FPGA based designs
			\item Computer Architectures/ Processors
			\item Reconfigurable Computing Systems
			\item SoC / Platform-based designs
			\item Embedded Systems
			\item MEMS/Optics/Bio-Chips
			\item Innovative Design Methodologies and Verification Techniques.
			\end{enumerate}
		\item \url{http://vlsiconference.com/vlsi2011/submissions_design_contest.html}
		\end{enumerate}
	\item {\it Satisfiability Modulo Theories Competition} (SMT-COMP): \vspace{-0.2cm}
		\begin{enumerate} \itemsep -2pt
		\item Competition for SMT solvers
		\item \url{http://www.smtcomp.org/2010/}
		\end{enumerate}
	\item {\it SAT Competition 201X}, where $X > 0$ \& $X {\it mod} 2 = 1$: \vspace{-0.2cm}
		\begin{enumerate} \itemsep -2pt
		\item The purpose of the competition is to identify new challenging benchmarks and to promote new solvers for the propositional satisfiability problem (SAT) as well as to compare them with state-of-the-art solvers.
		\item \url{http://www.satcompetition.org/}
		\end{enumerate}
	\item {\it SAT-Race 201X}, where $X > 0$ \& $X {\it mod} 2 = 0$: \vspace{-0.2cm}
		\begin{enumerate} \itemsep -2pt
		\item SAT-Race 201X is a competitive event for solvers of the Boolean Satisfiability (SAT) problem. 
		\item In contrast to the SAT Competitions, the focus of SAT-Race is on application benchmarks only.
		\item \url{http://baldur.iti.uka.de/sat-race-2010/}
		\end{enumerate}
	\item Hardware Model Checking Competition (HWMCC): \url{http://fmv.jku.at/hwmcc10/}
	\item {\it CADE ATP System Competition} (CASC): \vspace{-0.2cm}
		\begin{enumerate} \itemsep -2pt
		\item It is a yearly competition of fully automated theorem provers for classical first order logic.
		\item \url{http://www.cs.miami.edu/~tptp/CASC/}
		\end{enumerate}
	\item Apple Design Awards: \url{http://developer.apple.com/wwdc/ada/index.html}
	\item {\it International Constraint Solver Competition}: \vspace{-0.2cm}
		\begin{enumerate} \itemsep -2pt
		\item Also known as: \vspace{-0.2cm}
			\begin{enumerate} \itemsep -2pt
			\item International Constraint Solver Competition (CSP, Max-CSP and Weighted-CSP competition)
			\item International CSP Solver Competition (CSP, Max-CSP and Weighted-CSP competition)
			\end{enumerate}
		\item The Fourth International Constraint Solver Competition (CSC'2009) is organized to improve our knowledge of what is behind the efficiency of constraint satisfaction algorithms, heuristics, solving strategies, and constraint systems.
		\item \url{http://cpai.ucc.ie/}
		\end{enumerate}
	\item International Conference on Field-Programmable Technology (FPT 201X): \vspace{-0.2cm}
		\begin{enumerate} \itemsep -2pt
		\item FPT Design Competition: \url{http://cas.ee.ic.ac.uk/people/as999/FPTDesignComp/}
		\end{enumerate}
	\item International Microwave Symposium: Student Design Competitions -- Jan (includes AMS circuit simulation, and AMS/RF EDA); \url{http://ims2011.org/Technical_Program/Student_Design_Competitions.html}
	\item {\it QBFEVAL'1X}: \vspace{-0.2cm}
		\begin{enumerate} \itemsep -2pt
		\item QBF Solver competition for solvers to determine Quantified Boolean Formula (QBF) satisfiability.
		\item QBFLIB is a collection of instances, solvers, and tools related to Quantified Boolean Formula (QBF) satisfiability. See \url{http://www.qbflib.org/}.
		\item \url{http://www.qbflib.org/index_eval.php}
		\end{enumerate}
	\item {\it Pseudo-Boolean Competition 201X}: \vspace{-0.2cm}
		\begin{enumerate} \itemsep -2pt
		\item Competition for pseudo-Boolean solvers.
		\item \url{http://www.cril.univ-artois.fr/PB10/}
		\end{enumerate}
	\item {\it Answer Set Programming System Competition}: \vspace{-0.2cm}
		\begin{enumerate} \itemsep -2pt
		\item \url{http://dtai.cs.kuleuven.be/events/ASP-competition/}
		\end{enumerate}
	\item {\it Max-SAT Evaluation, Max-SAT 201X}: \vspace{-0.2cm}
		\begin{enumerate} \itemsep -2pt
		\item Competition for Max-SAT solvers
		\item \url{http://www.maxsat.udl.cat/}
		\item \url{http://www.maxsat.udl.cat/09/}
		\end{enumerate}
	\item {\it IEEEXtreme 24 Hour Programming Challenge}: \vspace{-0.2cm}
		\begin{enumerate} \itemsep -2pt
		\item Programming contest for college students
		\item \url{http://portal.ieee.org/web/membership/students/scholarshipsawardscontests/ieeextreme.html}
		\end{enumerate}
	\item {\it ACM International Collegiate Programming Contest} (ACM-ICPC or ICPC): \vspace{-0.2cm}
		\begin{enumerate} \itemsep -2pt
		\item Programming contest for college students
		\item Official web page: \url{http://cm.baylor.edu/welcome.icpc}
		\item Other web resources: \vspace{-0.1cm}
			\begin{enumerate} \itemsep -1pt
			\item {\it Wikipedia}: \url{http://en.wikipedia.org/wiki/ACM_International_Collegiate_Programming_Contest}
			\item {\it }: \url{}
			\item {\it }: \url{}
			\item {\it Valladolid Online Judge Site}: \url{http://acm.uva.es/}
			\item {\it ACMSolver :: Art of Programming Contest, Tips and Tricks for C, C++, Java}: \url{http://www.acmsolver.org/}
			\end{enumerate}
		\item 
		\end{enumerate}
	\item {\it TopCoder} coding and design contests: \vspace{-0.2cm}
		\begin{enumerate} \itemsep -2pt
		\item The contests cover various fields, such as: \vspace{-0.1cm}
			\begin{enumerate} \itemsep -1pt
			\item Algorithm
			\item Conceptualization
			\item Specification
			\item Architecture
			\item Component Design
			\item Component Development
			\item Assembly
			\item Test Scenarios
			\item Test Suites
			\item UI Prototype
			\item Rich Internet Application (RIA) Build
			\item Bug Race
			\item Marathon Match
			\item High School (for high school students)
			\item Copilot Opportunities
			\end{enumerate}
		\item \url{http://www.topcoder.com/}
		\end{enumerate}
	\item IEEE Presidents' Change the World competition: \vspace{-0.2cm}
		\begin{enumerate} \itemsep -2pt
		\item The IEEE Presidents� Change the World Competition recognizes students who develop unique solutions to real-world problems using engineering, science, computing and leadership skills to benefit their community, the world at large, or both. 
		\item \url{http://www.ieeechangetheworld.org/}
		\end{enumerate}
	\item Google Code Jam (programming contest): \url{http://code.google.com/codejam/} and \url{http://en.wikipedia.org/wiki/Google_Code_Jam}
	\item {\it RoboCup}\texttrademark\ competitions: \vspace{-0.2cm}
		\begin{enumerate} \itemsep -2pt
		\item Has different categories, including soccer, rescue operations, and home applications.
		\item \url{http://www.robocup.org/}
		\end{enumerate}
	\item ICFP Programming Contest (ICFP refers to International Conference on Functional Programming): \url{http://icfpcontest.org/}
	\item Student Cluster Competition (SCC): \vspace{-0.2cm}
		\begin{enumerate} \itemsep -2pt
		\item SCC is held at each (annual) SC conference, which is the International Conference for High Performance Computing, Networking, Storage, and Analysis. IEEE Computer Society and the Association for Computing Machinery are the sponsors for this conference.
		\item During SC10, teams consisting of six students, undergraduate and/or high school, will showcase the amazing power of clusters and the ability to utilize open source software to solve interesting and important problems. They will compete in real-time on the exhibit floor to run a workload of real-world applications on clusters of their own design while never exceeding the dictated power limit.
		\item During SC10 in New Orleans, teams will assemble, test and tune their machines and run the HPCC benchmarks until the starting bell rings on Monday night at the Exhibit Opening Gala where they will be given the competition data sets. In full view of conference attendees, teams will execute the prescribed workload while showing progress and science visualization output on large high-resolution displays in their areas. Teams race to correctly complete the greatest number of application runs during the competition period until the close of the exhibit floor on Wednesday evening.
		\item \url{http://sc10.supercomputing.org/?pg=studentcluster.html}
		\end{enumerate}
	\item Cypress Semiconductor Corporation: \vspace{-0.2cm}
		\begin{enumerate} \itemsep -2pt
		\item ARM Cortex-M3 PSoC\textregistered\ 5 Design Challenge: \url{http://www.cypress.com/?id=3271}
		\end{enumerate}
	\item Mentor Graphics: \vspace{-0.2cm}
		\begin{enumerate} \itemsep -2pt
		\item PCB Technology Leadership Awards (PCB design contest): \url{http://www.mentor.com/products/pcb-system-design/tla/index.cfm?v=mentorgraphics&p=handout:tla&a=print_card&g=sdd&s=1x1&c=ocid_2203&cmpid=3911}, or \url{http://www.mentor.com/go/tla}
		\end{enumerate}
	\item INFORMS Data Mining Contest: \vspace{-0.2cm}
		\begin{enumerate} \itemsep -2pt
		\item \url{http://ifors.org/web/call-for-participation-informs-data-mining-contest-2010/}
		\item \url{http://kaggle.com/informs2010}
		\end{enumerate}
	\item INFORMS Doing Good with Good OR - Student Competition: \vspace{-0.2cm}
		\begin{enumerate} \itemsep -2pt
		\item Doing Good with Good OR-Student Competition is held each year to identify and honor outstanding projects in the field of operations research and the management sciences conducted by a student or student group that have a significant societal impact.
		\item \url{http://www.informs.org/Recognize-Excellence/INFORMS-Prizes-Awards/Doing-Good-with-Good-OR}
		\end{enumerate}
	\item HPC Challenge Award Competition: \url{http://www.hpcchallenge.org/}
	\item Sphere Online Judge, SPOJ (programming contest): \url{http://www.spoj.pl/}
	\item High Performance and Scientific Computing Contest (Argonne National Laboratory, U.S. Department of Energy, DOE): \url{https://wiki.alcf.anl.gov/index.php/HPSC_Contest_Information}
	\item Argonne National Laboratory, ANL; Mathematics and Computer Science Division: \vspace{-0.2cm}
		\begin{enumerate} \itemsep -2pt
		\item J. H. Wilkinson Prize for Numerical Software (for developers of numerical software): \url{http://www.mcs.anl.gov/research/opportunities/wilkinsonprize/index.php}
		\end{enumerate}
	\item Society for Industrial and Applied Mathematics, SIAM: \vspace{-0.2cm}
		\begin{enumerate} \itemsep -2pt
		\item SIAM/ACM Prize in Computational Science and Engineering: \url{http://www.siam.org/prizes/sponsored/cse.php}. [ For developers of mathematical and computational tools and methods for the solution of science and engineering. Or, for developers of computational science and engineering software. ]
		\end{enumerate}
	\end{itemize}
	\item Sun HPC Software Programming Challenge (Oracle Corporation): \url{http://wikis.sun.com/display/HPCContest/Home}
%%%%%%%%%%%%%%%%%%%%%%%%%%%%%%
\item News media: \vspace{-0.3cm}
	\begin{itemize} \itemsep -2pt
	\item --- --- --- --- --- --- --- --- --- --- --- --- --- --- --- --- --- --- --- --- --- --- --- --- --- --- --- --- --- --- ---
	\item \colorbox{blue}{\bf News media for Electronic Design Automation}
	% News media for Electronic Design Automation
	\item {\it EDACafe}: \url{http://www.edacafe.com/}
	\item {\it SIGDA E-Newsletter} (SIGDA Electronic Newsletter): \url{http://www.sigda.org/newsletter/}
	\item {\it DeepChip.com}: \url{http://www.deepchip.com}
	\item --- --- --- --- --- --- --- --- --- --- --- --- --- --- --- --- --- --- --- --- --- --- --- --- --- --- --- --- --- --- ---
	\item \colorbox{blue}{\bf News media for Electrical and Computer Engineering}
	% News media for Electrical and Computer Engineering
	\item {\it EE Times} (Electronic Engineering Times): \url{http://www.eetimes.com/}
	\item {\it EDN} (Electrical Design News): \url{http://www.edn.com/}
	\item {\it IEEE Spectrum}: \url{http://spectrum.ieee.org/}
	\item {\it The Institute} (from IEEE): \url{http://www.theinstitute.ieee.org}
	\item {\it IEEE-USA Today's Engineer}: \url{http://www.todaysengineer.org/}
	\item {\it DeepChip.com}: \url{http://www.deepchip.com}
	\item --- --- --- --- --- --- --- --- --- --- --- --- --- --- --- --- --- --- --- --- --- --- --- --- --- --- --- --- --- --- ---
	\item \colorbox{blue}{\bf News media for Computer Science and Engineering, Information Systems, and IT}
	% News media for Computer Science and Engineering, Information Systems, and IT
	\item {\it ACM TechNews}: \url{http://technews.acm.org/}
	\item {\it TechCareers}: \url{http://www.techcareers.com/}
	\item {\it }: \url{}
	\item {\it }: \url{}
	\item {\it }: \url{}
	\item {\it }: \url{}
	\item {\it }: \url{}
	\item {\it }: \url{}
	\item {\it }: \url{}
	\item --- --- --- --- --- --- --- --- --- --- --- --- --- --- --- --- --- --- --- --- --- --- --- --- --- --- --- --- --- --- ---
	\item \colorbox{blue}{\bf Other News Media}
	% Other News Media
	\item {\it iTunes U}
	\item {\it YouTube EDU}
	\end{itemize}
%%%%%%%%%%%%%%%%%%%%%%%%%%%%%%
\item underrepresented minorities: \vspace{-0.3cm}
	\begin{enumerate} \itemsep -2pt
	\item women: \vspace{-0.2cm}
		\begin{enumerate} \itemsep -2pt
		\item IEEE Women in Engineering (WIE): \url{http://www.ieee.org/membership_services/membership/women/index.html?WT.mc_id=WIE_nav1}
		\item ACM-W: \url{http://women.acm.org/}
		\item Computer Research Association's Committee on the Status of Women in Computing Research (CRA-W): \vspace{-0.1cm}
			\begin{enumerate} \itemsep -1pt
			\item \url{http://www.cra-w.org/}
			\item Computing Research Association's Committee on the Status of Women (CRA-W) and the Coalition to Diversify Computing (CDC), {\it CompArch Summer School on Parallel Programming and Architectures}. Available at: \url{http://www.princeton.edu/~archss/}; last accessed on September 3, 2010.
			\end{enumerate}
		\item National Center for Women \& Information Technology: \url{http://www.ncwit.org/}
		\item African-American Women in Technology organization (AAWIT): \url{http://www.aawit.net/09/index.cfm}
		\item Grace Hopper Celebration of Women in Computing (conference for female IT students, professors, and professionals): \url{http://gracehopper.org/} or \url{http://gracehopper.org/2010/}
		\item Anita Borg Institute for Women and Technology: \vspace{-0.1cm}
			\begin{enumerate} \itemsep -1pt
			\item Has many programs for female students and professionals: \url{http://anitaborg.org/}
			\end{enumerate}
		\end{enumerate}
	\end{enumerate}
\end{enumerate}







%%%%%%%%%%%%%%%%%%%%%%%%%%%%%%%%%%%%%%%%%%%
\section{Scholarships, Fellowships, Awards, and Financial Aid}
\label{scholarshipsfinaidawards}

Resources for scholarships, fellowships, and financial aid: \vspace{-0.3cm}
\begin{enumerate} \itemsep -4pt
\item --- --- --- --- --- --- --- --- --- --- --- --- --- --- --- --- --- --- --- --- --- --- --- --- --- --- --- --- --- --- ---
\item \colorbox{blue}{\bf Lists of Scholarships and Fellowships}
% Lists of Scholarships and Fellowships
\item List of scholarships: \vspace{-0.3cm}
	\begin{enumerate} \itemsep -2pt
	\item Engineering Education Service Center, EESC (Engineering): \url{http://www.engineeringedu.com/scholars.html}
	\item High Performance and Embedded Architecture and Compilation, HiPEAC (Computer Science and Engineering): \url{http://www.hipeac.net/all_jobs_op}
	\item Office of Doctoral Programs at USC Viterbi School of Engineering, {\bf University of Southern California}. External Fellowships and other support: \url{http://viterbi.usc.edu/students/phd/fellowships-and-other-support/external-fellowships.htm}. USC Fellowships: \url{http://viterbi.usc.edu/students/phd/fellowships-and-other-support/usc-fellowships.htm}
	\item Columbia College, {\bf Columbia University} in the City of New York: \url{http://www.college.columbia.edu/students/fellowships/catalog}
	\item {\bf New York University} School of Law: \url{http://www.law.nyu.edu/financialaid/supplementalaid/fellowships/index.htm}
	\item Swedish Institute: \vspace{-0.2cm}
		\begin{enumerate} \itemsep -2pt
		\item The Swedish Institute, a government agency, administers over 500 scholarships each year for students and researchers coming to Sweden to pursue their objectives at a Swedish university.
		\item Study in Sweden: scholarships, \url{http://www.studyinsweden.se/Scholarships/}
		\item Swedish Institute (SI): \url{http://www.si.se/English/Navigation/Scholarships-and-exchanges/} [ Has special programs for Pakistanis and Turkish citizens ]
		\end{enumerate}
	\item The Swedish Foundation for International Cooperation in Research and Higher Education (STINT): \vspace{-0.2cm}
		\begin{enumerate} \itemsep -2pt
		\item \url{http://www.stint.se/en}
		\item Scholarships and grants: \url{http://www.stint.se/en/scholarships_and_grants}
		\end{enumerate}
	\item Center for the Advancement of Hispanics in Science and Engineering Education (CAHSEE): \url{http://www.cahsee.org/6resources/scholarships.asp.htm}
	\item University of Wisconsin-Madison: \vspace{-0.2cm}
		\begin{enumerate} \itemsep -2pt
		\item Grants Information Collection: A Cooperating Collection of the Foundation Center Library Network, \url{http://grants.library.wisc.edu/}
		\end{enumerate}
	\item {\it Find A PhD}: \url{http://www.findaphd.com/}
	\item QS World Grad School Tour Scholarships (QS Quacquarelli Symonds Limited): \url{http://graduateschool.topuniversities.com/world-grad-school-tour/scholarships}
	\item GlobalGrant (requires paid access to the list of scholarships and fellowships): \url{http://www.globalgrant.com/en/stipendier.html} and \url{http://www.globalgrant.com/}
	\item Stockholm University: \vspace{-0.2cm}
		\begin{enumerate} \itemsep -2pt
		\item \url{http://www.su.se/pub/jsp/polopoly.jsp?d=777&a=1770}
		\item \url{http://www.su.se/pub/jsp/polopoly.jsp?d=797}
		\item \url{http://www.su.se/pub/jsp/polopoly.jsp?d=788}
		\item \url{http://www.su.se/pub/jsp/polopoly.jsp?d=777&a=1769}
		\end{enumerate}
	\item NordForsk (in Norwegian): \url{http://www.nordforsk.org/index.cfm}
	\item Wallenberg Scholars (in Swedish): \url{http://www.wallenberg.com/default.aspx} or \url{http://www.wallenberg.com/in-english.aspx}
	\item Royal Institute of Technology (in Swedish): \url{http://www.kth.se/aktuellt/stipendier/stipendier-och-anslag-1.2024}
	\item European Commission: \vspace{-0.2cm}
		\begin{enumerate} \itemsep -2pt
		\item Marie Curie Fellowships: \vspace{-0.1cm}
			\begin{enumerate} \itemsep -1pt
			\item \url{http://cordis.europa.eu/fp7/people/home_en.html}
			\item \url{http://ec.europa.eu/research/mariecurieactions/}
			\item \url{http://ec.europa.eu/research/fp6/mariecurie-actions/action/fellow_en.html}
			\item \url{http://www.mariecurie.org/}
			\end{enumerate}
		\item Euraxess: \url{http://ec.europa.eu/euraxess/}
		\item \url{http://ec.europa.eu/index_en.htm}
		\end{enumerate}
	\item Science Please (for research positions in life sciences in The Netherlands and Belgium, including Ph.D. and postdoc positions): \url{http://www.scienceplease.com/} or \url{http://www.scienceplease.com/about-us}
	\item University of Gothenburg: \vspace{-0.2cm}
		\begin{enumerate} \itemsep -2pt
		\item ResearchResearch: \url{http://www.researchresearch.com/} or \url{http://www.gu.se/english/research/scholarships/ResearchResearch/}
		\item Scholarship links: \url{http://www.gu.se/english/research/scholarships/scholarship_links/}
		\item Scholarships at University of Gothenburg: \url{http://www.gu.se/english/research/scholarships/gu/}
		\end{enumerate}
	\item Princeton University; The Graduate School: \url{http://gradschool.princeton.edu/financial/}
	\item National Association for Bilingual Education: \vspace{-0.2cm}
		\begin{enumerate} \itemsep -2pt
		\item List of Scholarships: \url{http://www.nabe.org/scholarship.html}
		\end{enumerate}
	\item {\bf Pennsylvania State University}: \vspace{-0.2cm}
		\begin{enumerate} \itemsep -2pt
		\item Office of Engineering Diversity; Penn State College of Engineering: \vspace{-0.1cm}
			\begin{enumerate} \itemsep -1pt
			\item Undergraduate Student Scholarships: \url{http://www.engr.psu.edu/oed/UnderScholarships.html}
			\item Graduate Student Scholarships: \url{http://www.engr.psu.edu/oed/GradScholarships.html}
			\item High School Student Scholarships: \url{http://www.engr.psu.edu/oed/HighSchoolScholarships.html}
			\item Disabled Student Scholarships: \url{http://www.engr.psu.edu/oed/DisabScholarships.html}
			\item Corporate Office of Engineering Diversity (OED) Scholarships: \url{http://www.engr.psu.edu/oed/OEDScholarships.html}
			\end{enumerate}
		\item University Fellowships Office: \vspace{-0.1cm}
			\begin{enumerate} \itemsep -1pt
			\item \url{http://sites.google.com/site/psuufo/}
			\item Prestigious Scholarships: \url{http://sites.google.com/site/psuufo/prestigious}
			\item Penn State Scholarships: \url{http://sites.google.com/site/psuufo/internal-scholarships}
			\item Other resources: \url{http://sites.google.com/site/psuufo/resources}
			\end{enumerate}
		\end{enumerate}
	\item {\bf Peterson's} college search: \vspace{-0.2cm}
		\begin{enumerate} \itemsep -2pt
		\item {\it College Scholarship Search}: \url{http://www.petersons.com/college-search/scholarship-search.aspx}
		\end{enumerate}
	\item Society for Industrial and Applied Mathematics (SIAM): \vspace{-0.2cm}
		\begin{enumerate} \itemsep -2pt
		\item Fellowship \& Research Opportunities: \url{http://www.siam.org/students/resources/fellowship.php}
		\end{enumerate}
	\item Institute of International Education (IIE): \vspace{-0.2cm}
		\begin{enumerate} \itemsep -2pt
		\item {\it Funding for US Study Online}: \vspace{-0.1cm}
			\begin{enumerate} \itemsep -1pt
			\item \url{http://www.fundingusstudy.org/}
			\end{enumerate}
		\end{enumerate}
	\end{enumerate}
\item --- --- --- --- --- --- --- --- --- --- --- --- --- --- --- --- --- --- --- --- --- --- --- --- --- --- --- --- --- --- ---
\item \colorbox{blue}{\bf Scholarships and Fellowships in Electrical and Computer Engineering}
% Scholarships and Fellowships in Electrical and Computer Engineering
\item IEEE: \vspace{-0.3cm}
	\begin{enumerate} \itemsep -2pt
	\item IEEE Awards, Competitions, and Scholarships: \url{http://www.ieee.org/membership_services/membership/students/awards/index.html}
	\item IEEE Circuits and Systems Society Pre-Doctoral Scholarships: Announced via email from IEEE Circuits and Systems Society
	\item IEEE Power \& Energy Society: \vspace{-0.2cm}
		\begin{enumerate} \itemsep -2pt
		\item G. Ray Ekenstam Memorial Scholarship: \vspace{-0.1cm}
			\begin{enumerate} \itemsep -1pt
			\item \url{http://www.ieee-pes.org/g-ray-ekenstam-memorial-scholarship}
			\item ``The Scholarship Fund awards, on an annual basis, a scholarship to a qualified undergraduate student who seeks an electrical engineering degree in the field of power or a related discipline, from an accredited US university or college.''
			\end{enumerate}
		\end{enumerate}
	\item IEEE Reliability Society: \vspace{-0.2cm}
		\begin{enumerate} \itemsep -2pt
		\item IEEE Reliability Society Scholarship: \url{http://www.ieee.org/portal/cms_docs_relsoc/relsoc/newsflipper/RS_Scholarship_Application.pdf} [ Look under the tab/option on ``Useful Information'' in the panel on the left. ]
		\end{enumerate}
	\end{enumerate}
\item The George Michael Memorial HPC Fellowship Program: \vspace{-0.3cm}
	\begin{enumerate} \itemsep -2pt
	\item The Association of Computing Machinery (ACM), IEEE Computer Society and SC Conference series have established the High Performance Computing (HPC) Ph.D. Fellowship Program. The SC conference is the International Conference for High Performance Computing, Networking, Storage, and Analysis. IEEE Computer Society and the Association for Computing Machinery are the sponsors for this conference.
	\item Every year, up to three fellowship recipients will each receive a stipend of at least \$5,000 (U.S.) for one academic year, plus travel support to attend the SC conference.
	\item See \url{http://sc10.supercomputing.org/?searchterm=fellowship&pg=GeorgeMichaelMemorial.html}
	\end{enumerate}
\item Intel: \vspace{-0.3cm}
	\begin{enumerate} \itemsep -2pt
	\item Intel Foundation Fellowship: \vspace{-0.2cm}
		\begin{enumerate} \itemsep -2pt
		\item Intel Foundation Ph.D. Fellowship % \url{http://intelscholarships.intel.com/}
		\item \url{http://www.intel.com/education/highered/studentprograms/fellowship.htm}
		\item This awards two-year fellowships to Ph.D. candidates pursuing leading-edge work in fields related to Intel's business and research interests.
		\item Fellowships are available at select U.S. universities, by invitation only, and focus on Ph.D. students who have completed at least one year of study.
		\item The fellowship includes a cash award (tuition/fees/stipend), an Intel mentor, and the opportunity to participate in an internship at Intel.
		\end{enumerate}
	\end{enumerate}
\item IBM: \vspace{-0.3cm}
	\begin{enumerate} \itemsep -2pt
	\item \url{http://www-304.ibm.com/jct01005c/university/scholars/phdfellowship}
	\item IBM Ph.D. Fellowship Award
	\item The IBM Ph.D. Fellowship Awards is an intensely competitive program which honors exceptional Ph.D. students in many academic disciplines and areas of study, for example: computer science and engineering, electrical and mechanical engineering , physical sciences (including chemistry, material sciences, and physics), mathematical sciences (including optimization), business sciences (including financial services, communication, and learning/knowledge), and service sciences, management, and engineering.
	\item IBM Herman Goldstine Postdoctoral Fellowship in Mathematical Sciences: \url{http://domino.research.ibm.com/comm/research_projects.nsf/pages/goldstine.index.html}
	\item Josef Raviv Memorial Postdoctoral Fellowship; see \url{http://domino.research.ibm.com/comm/research.nsf/pages/d.compsci.josef.raviv.general.info.html}, \url{http://domino.research.ibm.com/comm/research.nsf/pages/d.compsci.raviv.winner.html}, and \url{http://domino.research.ibm.com/comm/research.nsf/pages/d.compsci.raviv.winner2008.html}
	\end{enumerate}
\item AMD: Ph.D. fellowship, \url{http://developer.amd.com/programs/fellowship/Pages/default.aspx}
\item Qualcomm, {\it Qualcomm Innovation Fellowship} for Ph.D. students in Electrical Engineering and Computer Science at Stanford, UC Berkeley, UCLA, UCSD, and USC: \url{http://www.qualcomm.com/innovation/research/university_relations/innovation_fellowship/qinf10.html}
\item NVIDIA: \vspace{-0.3cm}
	\begin{enumerate} \itemsep -2pt
	\item NVIDIA Fellowship Program; see \url{http://www.nvidia.com/page/fellowship_programs.html}
	\end{enumerate}
\item Automatic RF Techniques Group (ARFTG): \vspace{-0.3cm}
	\begin{enumerate} \itemsep -2pt
	\item Microwave Measurement Student Fellowship (for ``graduate students who show promise and interest in pursuing research related to improvement of radio frequency and microwave measurement techniques''): \url{http://www.arftg.org/student_fellowship.html}
	\end{enumerate}
\item Gallium Arsenide Applications Symposium (GAAS) Association: \vspace{-0.3cm}
	\begin{enumerate} \itemsep -2pt
	\item GAAS PhD Student Fellowship (for Ph.D. students who have accepted papers at the European Microwave Integrated Circuits Conference, EuMIC): \url{http://www.gaas-symposium.org/english/awards_fellowship.htm} and \url{http://www.eumweek.com/2010/EuMIC.asp?id=c}
	\end{enumerate}
\item The Institution of Engineering and Technology, IET: \vspace{-0.3cm}
	\begin{enumerate} \itemsep -2pt
	\item Hudswell International Research Scholarship: \url{http://www.theiet.org/about/scholarships-awards/ambition/postgraduate1/hudswell-what.cfm}
	\item IET Postgraduate Scholarship: \url{http://www.theiet.org/about/scholarships-awards/ambition/postgraduate1/postgrad-what.cfm}
	\end{enumerate}
\item --- --- --- --- --- --- --- --- --- --- --- --- --- --- --- --- --- --- --- --- --- --- --- --- --- --- --- --- --- --- ---
\item \colorbox{blue}{\bf Scholarships and Fellowships in Computer Science}
% Scholarships and Fellowships in Computer Science
\item ACM Special Interest Group on Symbolic and Algebraic Manipulation (SIGSAM): List of Ph.D. positions in computer algebra and symbolic computation, as listed by SIGSAM; see \url{http://www.sigsam.org/opportunities.phtml?searchterm=fellowship}
\item Carnegie Mellon University: \vspace{-0.3cm}
	\begin{enumerate} \itemsep -2pt
	\item women@SCS School of Computer Science: \vspace{-0.2cm}
		\begin{enumerate} \itemsep -2pt
		\item Individuals, Corporations \& Organizations: \url{http://women.cs.cmu.edu/Resources/Funding/}
		\end{enumerate}
	\end{enumerate}
\item IBM: \vspace{-0.3cm}
	\begin{enumerate} \itemsep -2pt
	\item \url{http://www-304.ibm.com/jct01005c/university/scholars/phdfellowship}
	\item IBM Ph.D. Fellowship Award
	\item The IBM Ph.D. Fellowship Awards is an intensely competitive program which honors exceptional Ph.D. students in many academic disciplines and areas of study, for example: computer science and engineering, electrical and mechanical engineering , physical sciences (including chemistry, material sciences, and physics), mathematical sciences (including optimization), business sciences (including financial services, communication, and learning/knowledge), and service sciences, management, and engineering.
	\item IBM Herman Goldstine Postdoctoral Fellowship in Mathematical Sciences: \url{http://domino.research.ibm.com/comm/research_projects.nsf/pages/goldstine.index.html}
	\item Josef Raviv Memorial Postdoctoral Fellowship; see \url{http://domino.research.ibm.com/comm/research.nsf/pages/d.compsci.josef.raviv.general.info.html}, \url{http://domino.research.ibm.com/comm/research.nsf/pages/d.compsci.raviv.winner.html}, and \url{http://domino.research.ibm.com/comm/research.nsf/pages/d.compsci.raviv.winner2008.html}
	\end{enumerate}
\item Computing Innovation Fellows (CIFellows); post my profile on \url{http://cifellows.org/profiles/}; also see \url{http://www.cifellows.org/}
\item Microsoft: \vspace{-0.3cm}
	\begin{enumerate} \itemsep -2pt
	\item Microsoft Research Graduate Women's Scholarship: \url{http://research.microsoft.com/en-us/collaboration/awards/fellows-women.aspx}
	\item Microsoft Research PhD Fellowship: \url{http://research.microsoft.com/en-us/collaboration/awards/apply-us.aspx}
	\end{enumerate}
\item Google: \vspace{-0.3cm}
	\begin{enumerate} \itemsep -2pt
	\item Google Fellowship Program; see \url{http://googleblog.blogspot.com/2009/05/best-and-brightest.html}
	\end{enumerate}
\item NVIDIA: \vspace{-0.3cm}
	\begin{enumerate} \itemsep -2pt
	\item NVIDIA Fellowship Program; see \url{http://www.nvidia.com/page/fellowship_programs.html}
	\end{enumerate}
\item Facebook Ph.D. Fellowship: \url{http://www.facebook.com/careers/fellowship.php}
\item Yahoo! Labs: Yahoo! Key Scientific Challenges Program, \url{http://labs.yahoo.com/ksc}
\item Qualcomm, {\it Qualcomm Innovation Fellowship} for Ph.D. students in Electrical Engineering and Computer Science at Stanford, UC Berkeley, UCLA, UCSD, and USC: \url{http://www.qualcomm.com/innovation/research/university_relations/innovation_fellowship/qinf10.html} and \url{http://www.qualcomm.com/innovation/research/university_relations/innovation_fellowship/}
\item Computing Research Association (CRA): Outstanding Undergraduate Researchers, \url{http://www.cra.org/awards/undergrad-current/}
\item {\color{blue} European Research Consortium for Informatics and Mathematics (ERCIM)}: \vspace{-0.3cm}
	\begin{enumerate} \itemsep -2pt
	\item ERCIM Alain Bensoussan Fellowship Programme (for Ph.D. degree holders in selected research areas): \url{http://fellowship.ercim.eu/} and \url{http://www.ercim.eu/news/283-fellowship-programme}; research areas are listed at: \url{http://fellowship.ercim.eu/home/topic}. Deadlines are on April 30 and September 30 annually.
	\end{enumerate}
\item {\it Theory Matters Wiki}; Theoretical Computer Science (TCS) Advocacy Wiki: \vspace{-0.3cm}
	\begin{enumerate} \itemsep -2pt
	\item Funding Opportunities and Tips: \url{http://theorymatters.org/pmwiki/pmwiki.php?n=Main.FundingOpportunities}
	\end{enumerate}
\item Kurt G{\"{o}}del Research Prize Fellowship: \vspace{-0.3cm}
	\begin{enumerate} \itemsep -2pt
	\item 2 Ph.D. (pre-doctoral) fellowships
	\item 2 post-doctoral fellowships
	\item 1 unrestricted fellowship
	\item $[$Scope of the$]$ original fellowship proposals [includes] the areas of: \vspace{-0.2cm}
		\begin{enumerate} \itemsep -2pt
		\item set theory
		\item recursion theory
		\item proof theory/intuitionism
		\item model theory
		\item computer assisted reasoning
		\item philosophy of mathematics 
		\end{enumerate}
	\item All fellowship proposals, regardless of subject area, will be judged according to: \vspace{-0.2cm}
		\begin{enumerate} \itemsep -2pt
		\item the relevance and resemblance of the research (finished and proposed) to the great insights and originality of Kurt G{\"{o}}del
		\item its general interest and clarity of motivation
		\item its rigorous scientific quality and depth. 
		\end{enumerate}
	\item \url{http://fellowship.logic.at/}
	\end{enumerate}
\item Hewlett-Packard Company: \vspace{-0.3cm}
	\begin{enumerate} \itemsep -2pt
	\item Hewlett-Packard Labs India (Bengaluru / Bangalore): \vspace{-0.2cm}
		\begin{enumerate} \itemsep -2pt
		\item {\it BITS - HP Labs India Ph.D. Fellowship} for Research related to Information Technologies: \vspace{-0.1cm}
			\begin{enumerate} \itemsep -1pt
			\item \url{http://www.hpl.hp.com/india/bits-hplindia_phd/index.html} or \url{http://www.hpl.hp.com/india/bits-hplindia_phd/}
			\item \url{http://www.hpl.hp.com/india/bits-hplindia_phd/iiitbphd.html}
			\item BITS, Pilani and HP Labs India jointly offer a unique PhD fellowship for research in Information and Communication Technologies (ICT) relevant to fast-growing markets like India.
			\item HP Labs India currently has ongoing Ph.D. Fellowships with BITS Pilani and IIIT, Bangalore: \url{http://www.hpl.hp.com/india/bits-hplindia_phd/university.html}
			\end{enumerate}
		\item Open Innovation Office: \vspace{-0.1cm}
			\begin{enumerate} \itemsep -1pt
			\item \url{http://www.hpl.hp.com/open_innovation/}
			\item HP Labs Innovation Research Program (IRP): \url{http://www.hpl.hp.com/open_innovation/irp/index.html}
			\end{enumerate}
		\end{enumerate}
	\end{enumerate}
\item Code for America (CfA): \vspace{-0.3cm}
	\begin{enumerate} \itemsep -2pt
	\item CfA Fellowship (develop web applications for local governments in the US): \url{http://codeforamerica.org/fellows/}
	\end{enumerate}
\item University of Minnesota, Twin Cities: \vspace{-0.3cm}
	\begin{enumerate} \itemsep -2pt
	\item College of Science and Engineering: \vspace{-0.2cm}
		\begin{enumerate} \itemsep -2pt
		\item Charles Babbage Institute: \vspace{-0.1cm}
			\begin{enumerate} \itemsep -1pt
			\item Adelle and Erwin Tomash Graduate Fellowship (for Ph.D. candidates doing research in the history of IT/computing - all but dissertation Ph.D. students only): \url{http://www.cbi.umn.edu/research/tfellowship.html}
			\item Arthur L. Norberg Travel Fund (short-term grants-in-aid to help scholars with travel expenses to use archival collections at the Charles Babbage Institute): \url{http://www.cbi.umn.edu/research/ntravelfund.html}
			\end{enumerate}
		\end{enumerate}
	\end{enumerate}
\item --- --- --- --- --- --- --- --- --- --- --- --- --- --- --- --- --- --- --- --- --- --- --- --- --- --- --- --- --- --- ---
\item \colorbox{blue}{\bf Scholarships and Fellowships in Biomedical Engineering}
% Scholarships and Fellowships in Biomedical Engineering
\item Whitaker International Fellows and Scholars Program: \vspace{-0.3cm}
	\begin{enumerate} \itemsep -2pt
	\item For graduate/Ph.D. students and postdocs in biomedical engineering
	\item \url{http://www.whitaker.org/home}
	\end{enumerate}
\item --- --- --- --- --- --- --- --- --- --- --- --- --- --- --- --- --- --- --- --- --- --- --- --- --- --- --- --- --- --- ---
\item \colorbox{blue}{\bf Scholarships and Fellowships in Optical Engineering}
% Scholarships and Fellowships in Optical Engineering
\item {\it SPIE} -- The International Society for Optical Engineering: \vspace{-0.3cm}
	\begin{enumerate} \itemsep -2pt
	\item ``SPIE Scholarship Program'' for undergraduates or graduate students studying optics, photonics, imaging, or optoelectronics program or related discipline (e.g., physics, electrical engineering): \url{http://spie.org//x1733.xml?WT.svl=mddm14}
	\item Other scholarships (including scholarships for students doing research in nanolithography techniques and lasers): \url{http://spie.org/x1736.xml}
	\end{enumerate}
\item {\it Kidger Optics Associates} Michael Kidger Memorial Scholarship (to a college freshman, or sophomore of optical design): \url{http://www.kidger.com/mkms_requirements.html}
\item --- --- --- --- --- --- --- --- --- --- --- --- --- --- --- --- --- --- --- --- --- --- --- --- --- --- --- --- --- --- ---
\item \colorbox{blue}{\bf Scholarships and Fellowships in Mechanical Engineering}
% Scholarships and Fellowships in Mechanical Engineering
\item American Society of Mechanical Engineers (ASME): \vspace{-0.3cm}
	\begin{enumerate} \itemsep -2pt
	\item Graduate Teaching Fellowships (for Ph.D. students in mechanical engineering): \url{http://www.asme.org/Education/College/FinancialAid/Graduate_Teaching_Fellowships.cfm}
	\item ASME Scholarships: \vspace{-0.2cm}
		\begin{enumerate} \itemsep -2pt
		\item \url{http://www.asme.org/Education/College/FinancialAid/Scholarships.cfm}
		\item US Undergraduates: \url{http://www.asme.org/Education/College/FinancialAid/US_Undergraduates.cfm}
		\item Graduate Students: \url{http://www.asme.org/Education/College/FinancialAid/Graduate_Students.cfm}
		\item International Students: \url{http://www.asme.org/Education/College/FinancialAid/International_Undergraduates.cfm}
		\end{enumerate}
	\item Auxiliary Scholarships: \vspace{-0.2cm}
		\begin{enumerate} \itemsep -2pt
		\item \url{http://www.asme.org/Education/College/FinancialAid/Auxiliary_Scholarships.cfm}
		\item Undergraduate Scholarships: \url{http://www.asme.org/Education/College/FinancialAid/Undergraduate_Scholarships.cfm}
		\item Graduate Scholarships: \url{http://www.asme.org/Education/College/FinancialAid/Graduate_Scholarships.cfm}
		\item Rice-Cullimore Scholarship (for international graduate students in the US): \url{http://www.asme.org/Education/College/FinancialAid/RiceCullimore_Scholarship.cfm}
		\end{enumerate}
	\item International Petroleum Institute�s College Scholarships (for undergraduates): \url{http://www.asme-ipti.org/public/pagscholarshipprograms.aspx}
	\item International Petroleum Institute�s Graduate Fellowship (for individuals entering a graduate program in mechanical engineering, and has an interest in the petroleum industry): \url{http://www.asme-ipti.org/public/pagscholarshipprograms.aspx} and \url{http://www.asme.org/Communities/Students/Grad/Fellowships.cfm}
	\end{enumerate}
\item --- --- --- --- --- --- --- --- --- --- --- --- --- --- --- --- --- --- --- --- --- --- --- --- --- --- --- --- --- --- ---
\item \colorbox{blue}{\bf Scholarships and Fellowships in Civil Engineering}
% Scholarships and Fellowships in Civil Engineering
\item American Society of Civil Engineers (ASCE): \vspace{-0.3cm}
	\begin{enumerate} \itemsep -2pt
	\item Jack E. Leisch Memorial National Graduate Fellowship (for graduate students in transportation/traffic engineering): \url{http://www.asce.org/Content.aspx?id=25021}
	\item Scholarships \& Fellowships (for undergraduates and graduate students): \url{http://www.asce.org/Content.aspx?id=18337}
	\end{enumerate}
\item American Concrete Institute (ACI): \vspace{-0.3cm}
	\begin{enumerate} \itemsep -2pt
	\item ACI Foundation Fellowships \& Scholarships: \url{http://www.concrete.org/STUDENTS/ST_SCHOLARSHIPS.HTM}
	\end{enumerate}
\item Institute of Transportation Engineers: \vspace{-0.3cm}
	\begin{enumerate} \itemsep -2pt
	\item Burton W. Marsh Fellowship for Graduate Study in Traffic and Transportation Engineering: \url{http://www.ite.org/education/Burton_W_MarshFellowship.asp}
	\end{enumerate}
\item --- --- --- --- --- --- --- --- --- --- --- --- --- --- --- --- --- --- --- --- --- --- --- --- --- --- --- --- --- --- ---
\item \colorbox{blue}{\bf Scholarships and Fellowships in Chemical Engineering}
% Scholarships and Fellowships in Chemical Engineering
\item American Institute of Chemical Engineers (AIChE) scholarships (includes scholarships for underrepresented minorities): \url{http://www.aiche.org/Students/Scholarships/index.aspx}
\item --- --- --- --- --- --- --- --- --- --- --- --- --- --- --- --- --- --- --- --- --- --- --- --- --- --- --- --- --- --- ---
\item \colorbox{blue}{\bf Scholarships and Fellowships in Aerospace Engineering}
% Scholarships and Fellowships in Aerospace Engineering
\item American Institute of Aeronautics and Astronautics (AIAA): \vspace{-0.3cm}
	\begin{enumerate} \itemsep -2pt
	\item AIAA Foundation Scholarships: \vspace{-0.2cm}
		\begin{enumerate} \itemsep -2pt
		\item \url{http://www.aiaa.org/content.cfm?pageid=211}
		\item For undergraduates and graduate students
		\item Named scholarships for undergraduates are: \vspace{-0.1cm}
			\begin{enumerate} \itemsep -1pt
			\item \url{http://www.aiaa.org/content.cfm?pageid=226}
			\item A. Thomas Young Scholarship
			\item L. S. ``Skip'' Fletcher Scholarship 
			\item Sam F. Iacobellis Scholarship
			\item Robert L. Crippen Scholarship
			\item E. C. ``Pete'' Aldridge Scholarship
			\item Liquid Propulsion Technical Committee Scholarship
			\item Space Transportation Technical Committee Scholarship
			\item Digital Avionics Technical Committee Scholarship (4)
			\item Next Century of Flight Scholarship (2)
			\item Leatrice Gregory Pendray Scholarship
			\end{enumerate}
		\item Awards for graduate students: \vspace{-0.1cm}
			\begin{enumerate} \itemsep -1pt
			\item \url{http://www.aiaa.org/content.cfm?pageid=227}
			\item Martin Summerfield Propellants and Combustion Graduate Award
			\item Guidance, Navigation, And Control Graduate Award
			\item Gordon C. Oates Air Breathing Propulsion Graduate Award
			\item William T. Piper, Sr. General Aviation Systems Graduate Award
			\item Orville and Wilbur Wright Graduate Award
			\item John Leland Atwood Graduate Award
			\item Open Topic Graduate Award
			\end{enumerate}
		\end{enumerate}
	\item Student Design Competition Award: \url{http://www.aiaa.org/content.cfm?pageid=401}
	\end{enumerate}
\item --- --- --- --- --- --- --- --- --- --- --- --- --- --- --- --- --- --- --- --- --- --- --- --- --- --- --- --- --- --- ---
\item \colorbox{blue}{\bf Scholarships and Fellowships in Mathematics}
% Scholarships and Fellowships in Mathematics
\item Association for Women in Mathematics (AWM): \vspace{-0.3cm}
	\begin{enumerate} \itemsep -2pt
	\item Travel grants: \url{http://sites.google.com/site/awmmath/programs/travel-grants}
	\item Alice T. Schafer Mathematics Prize for excellence in mathematics by an undergraduate woman: \url{http://sites.google.com/site/awmmath/programs/schafer-prize}
	\item The {\it Ruth I. Michler Memorial Prize} of the AWM is awarded annually to a woman recently promoted to Associate Professor or an equivalent position in the mathematical sciences: \url{http://sites.google.com/site/awmmath/programs/michler-prize}
	\end{enumerate}
\item Seth Bonder Scholarship for Applied Operations Research in Health Services: \url{http://www.informs.org/Recognize-Excellence/INFORMS-Community-Prizes-and-Awards/Seth-Bonder-Scholarship-for-Applied-Operations-Research-in-Health-Services}
\item Oberwolfach Foundation: \vspace{-0.3cm}
	\begin{enumerate} \itemsep -2pt
	\item Oberwolfach Prize (for young European mathematicians): \url{http://www.mfo.de/programme/prize/}
	\item John Todd Fellowship (or John Todd Award) [for young excellent mathematicians working in numerical analysis]: \url{http://www.mfo.de/programme/todd/}
	\end{enumerate}
\item Clay Mathematics Institute: Clay Research Award, \url{http://www.claymath.org/research_award/}
\item --- --- --- --- --- --- --- --- --- --- --- --- --- --- --- --- --- --- --- --- --- --- --- --- --- --- --- --- --- --- ---
\item \colorbox{blue}{\bf Scholarships and Fellowships in Science}
% Scholarships and Fellowships in Science
\item {\it Science.gov} (USA.gov for Science): \vspace{-0.3cm}
	\begin{enumerate} \itemsep -2pt
	\item Internship and Fellowship Opportunities in Science for Undergraduate Students: \url{http://www.science.gov/internships/undergrad.html}
	\item Graduate Students/Postdoctoral Fellowships: \url{http://www.science.gov/internships/graduate.html}
	\end{enumerate}
\item Heinz Family Philanthropies: \vspace{-0.3cm}
	\begin{enumerate} \itemsep -2pt
	\item Teresa Heinz Scholars for Environmental Research program (for Ph.D./MS students working on their thesis in environmental science/engineering) at selected universities: \url{http://www.heinzfamily.org/programs/environmentalscholars.html}
	\item \url{http://www.heinzfamily.org/}
	\end{enumerate}
\item Mayo Clinic: \vspace{-0.3cm}
	\begin{enumerate} \itemsep -2pt
	\item Postbaccalaureate Research Education Program (PREP): \url{http://www.mayo.edu/mgs/postbac-program.html}
	\end{enumerate}
\item {\it American Chemical Society (ACS)}: \vspace{-0.3cm}
	\begin{enumerate} \itemsep -2pt
	\item ACS-Hach Land Grant Undergraduate Scholarship (for chemistry undergraduates at a partner institution of ACS, and who plan to become chemistry teachers in US high schools): \url{http://portal.acs.org/portal/acs/corg/content?_nfpb=true&_pageLabel=PP_SUPERARTICLE&node_id=2243&use_sec=false&sec_url_var=region1&__uuid=eb054647-53e0-4594-81e8-8ef49159f3f4}
	\item ACS-Hach Second Career Teacher Scholarship (for graduates in chemistry or related areas who are entering an education masters program or teacher certification program): \url{http://portal.acs.org/portal/acs/corg/content?_nfpb=true&_pageLabel=PP_SUPERARTICLE&node_id=2244&use_sec=false&sec_url_var=region1&__uuid=4c27333f-4aad-481e-aaa4-f1db045d4eb4}
	\item ACS Scholars Program (for undergraduate underrepresented minorities majoring in chemistry, biochemistry, or chemical engineering): \url{http://portal.acs.org/portal/acs/corg/content?_nfpb=true&_pageLabel=PP_SUPERARTICLE&node_id=1650&use_sec=false&sec_url_var=region1&__uuid=b3b583cf-18ae-4fb0-9375-33f75a0ccf49}
	\item Scholarships: \url{http://portal.acs.org/portal/acs/corg/content?_nfpb=true&_pageLabel=PP_TRANSITIONMAIN&node_id=630&use_sec=false&sec_url_var=region1&__uuid=98e85c05-be75-4283-a97c-7a63ab4c3178}
	\end{enumerate}
\item European Molecular Biology Organization: \vspace{-0.3cm}
	\begin{enumerate} \itemsep -2pt
	\item EMBO Short-Term Fellowships (for junior researchers, including Ph.D. students): \url{http://www.embo.org/programmes/fellowships/short-term.html}
	\item EMBO Long-Term Fellowships (for junior researchers/postdocs): \url{http://www.embo.org/programmes/fellowships/long-term.html}
	\end{enumerate}
\item L'OR{\'{E}}AL: \vspace{-0.3cm}
	\begin{enumerate} \itemsep -2pt
	\item ``For Women in Science'' program: \url{http://www.lorealusa.com/forwomeninscience} or \url{http://www.lorealusa.com/_en/_us/index.aspx?direct1=00008&direct2=00008/00001}
	\item Alternatively, go to \url{http://www.lorealusa.com/_en/_us/} and select the ``For Women in Science'' tab.
	\item Check out the ``L'Or{\'{e}}al USA Fellowships for Women in Science'' (US postdocs), ``UNESCO-L'Or{\'{e}}al Fellowships for Women in Science'' (for female Ph.D. students and postdocs in the life sciences), and the ``L'Or{\'{e}}al-UNESCO Awardss for Women in Science'' (for distinguished female scientists)
	\end{enumerate}
\item American Institute of Physics (AIP): \vspace{-0.3cm}
	\begin{enumerate} \itemsep -2pt
	\item AIP and Member Society Government Science Fellowships: \vspace{-0.2cm}
		\begin{enumerate} \itemsep -2pt
		\item \url{http://www.aip.org/gov/fellowships.html}
		\item American Institute of Physics State Department Science Fellowship: \url{http://www.aip.org/gov/fellowships/sdf.html}
		\item American Institute of Physics Congressional Science Fellowship: \url{http://www.aip.org/gov/fellowships/cf.html}
		\item American Physical Society Congressional Science Fellowship: \url{http://www.aps.org/policy/fellowships/congressional.cfm}
		\item American Geophysical Union Congressional Science Fellowship: \url{http://www.agu.org/sci_pol/cong_fellowship/}
		\item Optical Society of America Congressional Science Fellowships: \url{http://www.osa.org/about_osa/public_policy/congressional_fellowships/default.aspx}
		\item For US citizens with good track records in research
		\end{enumerate}
	\item American Geophysical Union: \vspace{-0.2cm}
		\begin{enumerate} \itemsep -2pt
		\item Research Grants and Awards: \url{http://www.agu.org/about/honors/research_grants/}
		\item Student Travel Grants: \url{http://www.agu.org/education/grants/travel.shtml}
		\item Research Grants \& Awards: \url{http://www.agu.org/education/grants/research.shtml}
		\item Mass Media Fellowship: \url{http://www.agu.org/news/mass_media_fellowship/}
		\end{enumerate}
	\item Society of Physics Students (SPS): \vspace{-0.2cm}
		\begin{enumerate} \itemsep -2pt
		\item SPS Scholarships: \url{http://www.spsnational.org/programs/scholarships/}
		\item SPS Awards: \url{http://www.spsnational.org/programs/awards/}
		\end{enumerate}
	\end{enumerate}
\item Consortium for Ocean Leadership: \vspace{-0.3cm}
	\begin{enumerate} \itemsep -2pt
	\item Employment, Internships, and Opportunities [ includes funding opportunities for researchers (professors, postdocs, and grad students) ]: \url{http://www.oceanleadership.org/about-ocean-leadership/ocean-of-opportunities/}
	\item HBCU Fellowship: Ocean Leadership/IODP-USIO for Students of Historically Black Colleges and Universities, \url{http://www.oceanleadership.org/education/diversity/hbcu-fellowship/}
	\item HBCU Educator at Sea: \url{http://www.oceanleadership.org/education/diversity/hbcu-educator/}
	\item MS PHD's Professional Development Program: The Minorities Striving and Pursuing Higher Degrees of Success in the Earth System Sciences (MS PHD'S) Professional Development Program, \url{http://www.oceanleadership.org/education/diversity/ms-phds-professional-development-program/}
	\item Schlanger Ocean Drilling Fellowship Program (merit-based awards for outstanding graduate students to conduct research related to the Integrated Ocean Drilling Program): \url{http://www.oceanleadership.org/programs-and-partnerships/usssp/schlanger-fellowship/}
	\end{enumerate}
\item American Geological Institute Foundation: \vspace{-0.3cm}
	\begin{enumerate} \itemsep -2pt
	\item William L. Fisher Congressional Geoscience Fellowship (for young geoscientists to get engaged in {\bf public policy}): \url{http://www.agifoundation.org/govtaffairs.html} and \url{http://www.agifoundation.org/endowments.html}
	\item AGI Minority Participation Program: Minority Participation Program Geoscience Student Scholarships for ``underrepresented ethnic-minority (undergraduate or graduate) students in the geosciences'', \url{http://www.agiweb.org/mpp/index.html}
	\end{enumerate}
\item Lady Davis Institute/Jewish General Hospital: \vspace{-0.3cm}
	\begin{enumerate} \itemsep -2pt
	\item Awards for ``graduate students (in biomedical science) and post-doctoral fellows/clinical fellows'': \url{http://www.ladydavis.ca/en/awards}
	\end{enumerate}
\item Adolph C. and Mary Sprague Miller Institute for Basic Research in Science: \vspace{-0.3cm}
	\begin{enumerate} \itemsep -2pt
	\item Miller Fellowships (for outstanding recent Ph.D.s / postdoctoral fellowship): \url{http://millerinstitute.berkeley.edu/topage.php?nav=11&to=1} or \url{http://millerinstitute.berkeley.edu/page.php?nav=11}
	\item Visiting Miller Research Professorships (for professors and research scientists): \url{http://millerinstitute.berkeley.edu/topage.php?nav=24&to=1} or \url{http://millerinstitute.berkeley.edu/page.php?nav=24}
	\item Miller Research Professorships (for professors in the UC system): \url{http://millerinstitute.berkeley.edu/topage.php?nav=15&to=1} or \url{http://millerinstitute.berkeley.edu/page.php?nav=15}
	\item Miller Senior Fellowships (Nominations are solicited by invitation only; Senior Fellow appointments are made to tenured UC Berkeley faculty for five years, possibly renewable for a subsequent five years, but no longer.): \url{http://millerinstitute.berkeley.edu/topage.php?nav=126&to=1}
	\end{enumerate}
\item Funda{\c{c}}{\~{a}}o para a Ci{\^{e}}ncia e a Tecnologia (FCT); Minist{\'{e}}rio da Ci{\^{e}}ncia, Technologia e Ensino Superior (MCTES): International Prize Fernando Gil in Philosophy of Science, \url{http://alfa.fct.mctes.pt/apoios/premios/fernando_gil/index.phtml.pt}
\item Wellcome Trust: \vspace{-0.3cm}
	\begin{enumerate} \itemsep -2pt
	\item Wellcome Trust Sanger Institute: \vspace{-0.2cm}
		\begin{enumerate} \itemsep -2pt
		\item \url{http://www.sanger.ac.uk/workstudy/}
		\item Postdoctoral fellows (for research in genomics): \url{http://www.sanger.ac.uk/workstudy/career/postdocs/}
		\item Graduate program (for research in genomics): \url{http://www.sanger.ac.uk/workstudy/phd/}
		\item Student placements and work experience (for research in genomics): \url{http://www.sanger.ac.uk/workstudy/placements/}
		\end{enumerate}
	\end{enumerate}
\item Paul B. Beeson Career Development Awards in Aging Research Program (formerly the Beeson Physician Faculty Scholars Program): \vspace{-0.3cm}
	\begin{enumerate} \itemsep -2pt
	\item \url{http://www.beeson.org/}
	\item ``Today, the Beeson program continues to foster the independent research careers of clinically trained investigators -- a growing cadre of talented physician-scientists -- whose research and leadership are enhancing the health and quality of life of Americans, particularly older people.''
	\item About the Program: \url{http://www.beeson.org/program_hx.cfm}
	\end{enumerate}
\item American Mathematical Society: \vspace{-0.3cm}
	\begin{enumerate} \itemsep -2pt
	\item AMS Fellowships and Scholarships: \vspace{-0.2cm}
		\begin{enumerate} \itemsep -2pt
		\item \url{http://e-math.ams.org/programs/ams-fellowships/ams-fellowships}
		\item AMS Centennial Research Fellowship Program: \url{http://e-math.ams.org/programs/ams-fellowships/centennial-fellow/emp-centflyer}
		\item Waldemar J. Trijitzinsky Memorial Awards: \url{http://e-math.ams.org/programs/ams-fellowships/trjitzinsky/trjitzinsky-award}
		\item Other Sources of Funding: \url{http://e-math.ams.org/programs/funding/funding}
		\end{enumerate}
	\end{enumerate}
\item --- --- --- --- --- --- --- --- --- --- --- --- --- --- --- --- --- --- --- --- --- --- --- --- --- --- --- --- --- --- ---
\item \colorbox{blue}{\bf Scholarships and Fellowships in Medicine}
% Scholarships and Fellowships in Medicine
\item Sarnoff Medical Student Research Fellowship Program (for US medical students interested in cardiovascular research): \url{http://www.sarnoffendowment.org/}
\item Mayo Clinic: \vspace{-0.3cm}
	\begin{enumerate} \itemsep -2pt
	\item Postbaccalaureate Research Education Program (PREP): \url{http://www.mayo.edu/mgs/postbac-program.html}
	\end{enumerate}
\item Paul B. Beeson Career Development Awards in Aging Research Program (formerly the Beeson Physician Faculty Scholars Program): \vspace{-0.3cm}
	\begin{enumerate} \itemsep -2pt
	\item \url{http://www.beeson.org/}
	\item ``Today, the Beeson program continues to foster the independent research careers of clinically trained investigators -- a growing cadre of talented physician-scientists -- whose research and leadership are enhancing the health and quality of life of Americans, particularly older people.''
	\item About the Program: \url{http://www.beeson.org/program_hx.cfm}
	\end{enumerate}
\item --- --- --- --- --- --- --- --- --- --- --- --- --- --- --- --- --- --- --- --- --- --- --- --- --- --- --- --- --- --- ---
\item \colorbox{blue}{\bf Scholarships and Fellowships in Science and Engineering}
% Scholarships and Fellowships in Science and Engineering
\item National Academies: \vspace{-0.3cm}
	\begin{enumerate} \itemsep -2pt
	\item Research Associateship Programs (graduate, postdoctoral, and senior level research opportunities): \url{http://sites.nationalacademies.org/pga/rap/}
	\item Ford Foundation Fellowship Programs (predoctoral, dissertation or postdoctoral fellowships for individuals seeking academic careers in science and engineering): \url{http://sites.nationalacademies.org/PGA/FordFellowships/index.htm}
	\item \url{http://nationalacademies.org/grantprograms.html}
	\item \url{http://sites.nationalacademies.org/pga/fellowships/}
	\item List of Fellowship, Scholarship, and Grant Databases: \url{http://sites.nationalacademies.org/PGA/Fellowships/PGA_046300}
	\item List of Outside Fellowships, Scholarships, and Grants Websites: \url{http://sites.nationalacademies.org/PGA/Fellowships/PGA_046301}
	\item Awards for junior and mid-career researchers: \url{http://www.nasonline.org/site/PageServer?pagename=AWARDS_main}
	\item National Academy of Engineering, NAE: \vspace{-0.2cm}
		\begin{enumerate} \itemsep -2pt
		\item NAE Grand Challenges Scholars Program: \url{http://www.grandchallengescholars.org/}
		\end{enumerate}
	\item National Science Foundation: \vspace{-0.2cm}
		\begin{enumerate} \itemsep -2pt
		\item International Research Fellowship Program (IRFP) for junior scientists and engineers: \url{http://www.nsf.gov/funding/pgm_summ.jsp?pims_id=5179}
		\item Integrative Graduate Education and Research Traineeship Program (IGERT) for undergraduates and graduate students in STEM: \url{http://www.nsf.gov/funding/pgm_summ.jsp?pims_id=12759}
		\item National Science Foundation's Graduate Research Fellowship Program (GRFP) for students seeking research degrees in STEM: \url{http://www.nsfgrfp.org/}
		\item NSF Alliances for Graduate Education and the Professoriate (AGEP) program (to help underrepresented minorities obtain graduate degrees in STEM and prepare them for faculty positions in academia): \url{http://www.nsfagep.org/}
		\item National Science Foundation's (NSF) East Asia and Pacific Summer Institutes (EAPSI) program: \vspace{-0.1cm}
			\begin{enumerate} \itemsep -1pt
			\item \url{http://www.nsf.gov/funding/pgm_summ.jsp?pims_id=5284}
			\item The East Asia and Pacific Summer Institutes (EAPSI) provide U.S. graduate students in science and engineering: \vspace{-0.1cm}
				\begin{itemize} \itemsep -1pt
				\item first-hand research experiences in Australia, China, Japan, Korea, New Zealand, Singapore or Taiwan
				\item an introduction to the science, science policy, and scientific infrastructure of the respective location
				\item an orientation to the society, culture and language.
				\end{itemize}
			\item ``The primary goals of EAPSI are to introduce students to East Asia and Pacific science and engineering in the context of a research setting, and to help students initiate scientific relationships that will better enable future collaboration with foreign counterparts.''
			\item ``All institutes, except Japan, last approximately eight weeks from June to August. Japan lasts approximately ten weeks from June to August (specific dates are available and updated at \url{http://www.nsfsi.org/}).''
			\item Example of Ph.D. student, Jakub Szefer, from Prof. Ruby Lee's lab at Princeton University, who interned with Prof. Cheng Chen-Mou from National Taiwan University: \url{http://www.nsf.gov/discoveries/disc_summ.jsp?cntn_id=118116&org=NSF}
			\end{enumerate}
		\end{enumerate}
	\end{enumerate}
\item United States Department of Defense (DoD): \vspace{-0.3cm}
	\begin{enumerate} \itemsep -2pt
	\item National Defense Education Program; Defense Advanced Research Projects Agency (DARPA): \vspace{-0.2cm}
		\begin{enumerate} \itemsep -2pt
		\item Science, Mathematics, and Research for Transformation (SMART) scholarship program: \vspace{-0.1cm}
			\begin{itemize} \itemsep -1pt
			\item \url{http://smart.asee.org/}
			\item Co-organized by the American Society for Engineering Education
			\end{itemize}
		\item National Security Science and Engineering Faculty Fellowships (NSSEFF): \url{http://www.ndep.us/ProgNSSEFF.aspx}
		\end{enumerate}
	\end{enumerate}
\item National Society of Professional Engineers: \vspace{-0.3cm}
	\begin{enumerate} \itemsep -2pt
	\item Scholarships for undergraduates and graduate students: \url{http://www.nspe.org/Students/Scholarships/index.html}
	\item NSPE-PEC George B. Hightower, P.E. Fellowship (for an outstanding engineering graduate student): \url{http://www.nspe.org/InterestGroups/PEC/Resources/Awards/hightower_fellowship.html}
	\item PEG Management Fellowship: \vspace{-0.2cm}
		\begin{enumerate} \itemsep -2pt
		\item \url{http://www.nspe.org/InterestGroups/PEG/Resources/AwardsAndScholarships/peg_fellowship.html}
		\item ``This scholarship is designed for graduate students who are pursuing an MBA, a master's degree in engineering management, or a master's degree in public administration.''
		\end{enumerate}
	\end{enumerate}
\item Technion -- Israel Institute of Technology: \vspace{-0.3cm}
	\begin{enumerate} \itemsep -2pt
	\item Department of Mathematics: Anna and Paul Erdos postdoctoral Fellowship, \url{http://www.math.technion.ac.il/Site/people/positions.html}
	\item Lady Davis Postdoctoral Fellowship
	\item Department of Electrical Engineering: \vspace{-0.2cm}
		\begin{enumerate} \itemsep -2pt
		\item The Andrew and Erna Finci Viterbi Fellowship Program (for graduate and post-doctoral fellows), \url{http://webee.technion.ac.il/Research/Fellowship-Programs}
		\item Lady Davis Fellowship Trust: Technion Fellowships (for visiting professors, post-doctoral researchers, as well as Masters and Ph.D. students), \url{http://ldft.huji.ac.il/upload/info/}
		\item \url{http://webee.technion.ac.il/Research/Fellowship-Programs}
		\end{enumerate}
	\end{enumerate}
\item Hebrew University: \vspace{-0.3cm}
	\begin{enumerate} \itemsep -2pt
	\item Lady Davis Fellowship Trust: Technion Fellowships (for visiting professors, post-doctoral researchers, as well as Masters and Ph.D. students), \url{http://ldft.huji.ac.il/upload/info/infoHUa.html}
	\end{enumerate}
\item Hertz Foundation: \vspace{-0.3cm}
	\begin{enumerate} \itemsep -2pt
	\item The Graduate Fellowship Award: \url{http://www.hertzfoundation.org/dx/Fellowships/award.aspx}
	\item Thesis Prize: \url{http://www.hertzfoundation.org/dx/Fellowships/thesis_winners.aspx}
	\end{enumerate}
\item Krell Institute, Inc.: \vspace{-0.3cm}
	\begin{enumerate} \itemsep -2pt
	\item DOE Computational Science Graduate Fellowship: \url{http://www.krellinst.org/csgf/index.shtml}
	\end{enumerate}
\item The Winston Churchill Foundation of the United States: \vspace{-0.3cm}
	\begin{enumerate} \itemsep -2pt
	\item The Churchill Scholarship: \url{http://winstonchurchillfoundation.org/index.php?hide=1&section=eligibility}
	\end{enumerate}
\item American Society for Engineering Education: \vspace{-0.3cm}
	\begin{enumerate} \itemsep -2pt
	\item \url{http://blogs.asee.org/fellowships/}
	\item Fellowship programs: \url{http://www.asee.org/fellowship-programs}
	\item Awards: \url{http://www.asee.org/member-resources/awards/full-list-of-awards}
	\item DuPont Minorities in Engineering Award: \vspace{-0.2cm}
		\begin{enumerate} \itemsep -2pt
		\item \url{http://www.asee.org/member-resources/awards/full-list-of-awards/national-awards/special#DuPont_Minorities_in_Engineering_Award}
		\item {\bf \color{blue} ``The DuPont Minorities in Engineering Award is conferred for outstanding achievements by an engineering or engineering technology educator in increasing student diversity within engineering and engineering technology programs.''}
		\end{enumerate}
	\end{enumerate}
\item Alexander von Humboldt-Stiftung/Foundation: \vspace{-0.3cm}
	\begin{enumerate} \itemsep -2pt
	\item Feodor Lynen Research Fellowship for Postdoctoral Researchers (junior postdocs): \url{http://www.humboldt-foundation.de/web/feodor-lynen-fellowship-postdoc.html}
	\item Friedrich Wilhelm Bessel Research Award (mid-career researchers): \url{http://www.humboldt-foundation.de/web/bessel-award.html}
	\item Georg Forster Research Fellowship for Postdoctoral Researchers (for non-German junior postdocs ``with above average qualifications''): \url{http://www.humboldt-foundation.de/web/georg-forster-fellowship-postdoc.html}
	\item Humboldt Research Fellowship for Postdoctoral Researchers (junior postdocs): \url{http://www.humboldt-foundation.de/web/771.html}
	\item Sofja Kovalevskaja Award (junior postdocs): \url{http://www.humboldt-foundation.de/web/kovalevskaja-award.html}
	\item Fraunhofer-Bessel Research Award: \url{http://www.humboldt-foundation.de/web/fraunhofer-bessel-award.html}
	\item \url{http://www.humboldt-foundation.de/web/home.html}
	\end{enumerate}
\item Santa Fe Institute: Omidyar Postdoctoral Fellowship; see \url{http://www.santafe.edu/education/fellowships/omidyar-postdoctoral/}
\item Applied Materials: Applied Materials Graduate Fellowship
\item American Society of Naval Engineers (ASNE): \vspace{-0.3cm}
	\begin{enumerate} \itemsep -2pt
	\item (Undergraduate and Graduate) Scholarships: \url{http://www.navalengineers.org/awards/scholarships/Pages/ASNELandingPage.aspx}
	\end{enumerate}
\item Lindau Meeting of Nobel Laureates and Students in Lindau (Oak Ridge Associated Universities, ORAU): \vspace{-0.3cm}
	\begin{enumerate} \itemsep -2pt
	\item Graduate Student Award program: \vspace{-0.2cm}
		\begin{enumerate} \itemsep -2pt
		\item \url{http://www.orau.org/lindau/}
		\item A student nominated to participate in this program must: \vspace{-0.1cm}
			\begin{enumerate} \itemsep -1pt
			\item Be a U.S. citizen
			\item Be currently enrolled as a full-time graduate student
			\item Be currently sponsored by, or working on, and supported by projects sponsored by, the agency to which the nomination is made, such as the U.S. Department of Energy Office of Science, the National Institutes of Health or other federal agency
			\item Have completed by June 2011 two years (but not more than four years) of study toward a doctoral degree in medicine or physiology, or in a related discipline, including the basic biomedical (or life) sciences
			\end{enumerate}
		\end{enumerate}
	\end{enumerate}
\item Research Councils UK (RCUK): \vspace{-0.3cm}
	\begin{enumerate} \itemsep -2pt
	\item RCUK Academic Fellowships: \vspace{-0.2cm}
		\begin{enumerate} \itemsep -2pt
		\item \url{http://www.rcuk.ac.uk/ResearchCareers/fellowships/Pages/home.aspx}
		\item \url{http://www.rcuk.ac.uk/ResearchCareers/fellowships/Pages/about.aspx}
		\item Dorothy Hodgkin Postgraduate Awards: \vspace{-0.1cm}
			\begin{enumerate} \itemsep -1pt
			\item \url{http://www.rcuk.ac.uk/ResearchCareers/dhpa/Pages/home.aspx}
			\item ``Dorothy Hodgkin Postgraduate Awards (DHPA) is a UK scheme to bring outstanding students from India, China, Hong Kong, South Africa, Brazil, Russia and the developing world to come and study for PhDs in top rated UK research facilities.''
			\end{enumerate}
		\end{enumerate}
	\item International Funding Opportunities: \vspace{-0.2cm}
		\begin{enumerate} \itemsep -2pt
		\item \url{http://www.rcuk.ac.uk/international/funding/FundingOpps/Pages/home.aspx}
		\item Early Career Researchers: \url{http://www.rcuk.ac.uk/international/funding/FundingOpps/Pages/EarlyCareer.aspx}
		\end{enumerate}
	\item Engineering and Physical Sciences Research Council: \vspace{-0.2cm}
		\begin{enumerate} \itemsep -2pt
		\item Programs: \vspace{-0.1cm}
			\begin{enumerate} \itemsep -1pt
			\item Physical sciences: \vspace{-0.1cm}
				\begin{itemize} \itemsep -1pt
				\item Organic synthetic chemistry studentships: \url{http://www.epsrc.ac.uk/about/progs/physsci/Pages/organicstudentships.aspx}
				\item Analytical science studentships: \url{http://www.epsrc.ac.uk/about/progs/physsci/Pages/analyticalstudentships.aspx}
				\end{itemize}
			\item Mathematical sciences: \vspace{-0.1cm}
				\begin{itemize} \itemsep -1pt
				\item Fellowships (for postdoctoral research): \url{http://www.epsrc.ac.uk/about/progs/maths/Pages/fellowships.aspx}
				\end{itemize}
			\end{enumerate}
		\item Funding: \vspace{-0.1cm}
			\begin{enumerate} \itemsep -1pt
			\item \url{http://www.epsrc.ac.uk/funding/Pages/default.aspx}
			\item Grants available [has funds for (new/junior) professors and to support international collaboration]: \url{http://www.epsrc.ac.uk/funding/grants/Pages/default.aspx}
			\item Calls for proposals (open/current funding calls for applications and future/proposed calls): \url{http://www.epsrc.ac.uk/funding/calls/Pages/default.aspx}
			\item Studentships (training grants for Ph.D. and Masters students, including international students): \url{http://www.epsrc.ac.uk/funding/students/Pages/default.aspx}
			\item Fellowships (from junior scientists and engineers engaged in postdoctoral research to senior researchers): \url{http://www.epsrc.ac.uk/funding/fellows/Pages/default.aspx}
			\end{enumerate}
		\end{enumerate}
	\item Biotechnology and Biological Sciences Research Council (BBSRC): \vspace{-0.2cm}
		\begin{enumerate} \itemsep -2pt
		\item ``The UK's leading funding agency for academic research and training in the non-clinical life sciences''
		\item Funding research: \vspace{-0.1cm}
			\begin{enumerate} \itemsep -1pt
			\item \url{http://www.bbsrc.ac.uk/funding/funding-index.aspx}
			\item Fellowships (for early career scientists, for supporting individuals seeking a change in research directions or scientists who are returning to research, and senior researchers): \url{http://www.bbsrc.ac.uk/funding/fellowships/fellowships-index.aspx}
			\item Studentships (Doctoral training grants, Masters training grants, postgraduate awards, and undergraduate research grants): \url{http://www.bbsrc.ac.uk/funding/studentships/studentships-index.aspx}
			\item Special opportunities (current calls for funding): \url{http://www.bbsrc.ac.uk/funding/opportunities/opportunities-index.aspx}
			\item Apply for funding (information about the process of applying for research funds): \url{http://www.bbsrc.ac.uk/funding/apply/apply-index.aspx}
			\end{enumerate}
		\end{enumerate}
	\item Science and Technology Facilities Council: \vspace{-0.2cm}
		\begin{enumerate} \itemsep -2pt
		\item STFC Grants and Awards: \vspace{-0.1cm}
			\begin{enumerate} \itemsep -1pt
			\item \url{http://www.stfc.ac.uk/Funding+and+Grants/501.aspx}
			\item ``The Science and Technology Facilities Council offers grants and support in Particle Physics, Astronomy, Nuclear Physics and Facility Development. It also provides support for research infrastructure, training, knowledge exchange and public engagement activities through a variety of funding schemes and activities.''
			\item STFC Funding Opportunities: \url{http://www.stfc.ac.uk/Funding%20and%20Grants/598.aspx}
			\item Postgraduate Studentships: \url{http://www.stfc.ac.uk/Funding+and+Grants/637.aspx} or \url{http://www.stfc.ac.uk/Funding%20and%20Grants/636.aspx}
			\end{enumerate}
		\item Fellowship opportunities: \vspace{-0.1cm}
			\begin{enumerate} \itemsep -1pt
			\item \url{http://www.stfc.ac.uk/Funding%20and%20Grants/508.aspx}
			\item ``Fellowship opportunities in Astronomy, Solar and Planetary Science, Particle Physics, Particle Astrophysics, Nuclear Physics, Development of STFC Neutron, Laser and Synchrotron Facilities within the UK.''
			\item There are postdoctoral and advanced research fellowships.
			\end{enumerate}
		\item Innovations Partnership Schemes (IPS and mini-IPS): \url{http://www.stfc.ac.uk/19213.aspx}
		\item IPS Fellowships: \vspace{-0.1cm}
			\begin{enumerate} \itemsep -1pt
			\item \url{http://www.stfc.ac.uk/19226.aspx}
			\item The IPS fellowship is a scheme designed to support a role to develop the commercial exploitation of technologies. This is not a research orientated fellowship.
			\end{enumerate}
		\item Follow-on-Funding: \vspace{-0.1cm}
			\begin{enumerate} \itemsep -1pt
			\item \url{http://www.stfc.ac.uk/19207.aspx}
			\item ``Follow on Funding is intended to provide financial support at the very early or pre-seed stage of turning research outputs into a commercial proposition. Unlike the other research councils, in STFC, industry partners are not allowed. If you have an industry partner, please use the mini-IPS or IPS scheme.''
			\item ``STFC staff, grant funded academics and researchers at CERN and ESO are eligible to apply for follow-on-funds (see the research grants handbook for CERN and ESO eligibility). STFC staff should first investigate whether they can be funded through proof of concept funding.''
			\end{enumerate}
		\end{enumerate}
	\item Natural Environment Research Council: \vspace{-0.2cm}
		\begin{enumerate} \itemsep -2pt
		\item Grants and studentships on the web: \vspace{-0.1cm}
			\begin{enumerate} \itemsep -1pt
			\item \url{http://www.nerc.ac.uk/research/gotw.asp}
			\item Grants on the web: \url{http://gotw.nerc.ac.uk/goti.asp?c=1}
			\end{enumerate}
		\item Funding: \vspace{-0.1cm}
			\begin{enumerate} \itemsep -1pt
			\item \url{http://www.nerc.ac.uk/funding/}
			\item Postgraduate training: \vspace{-0.1cm}
				\begin{itemize} \itemsep -1pt
				\item Postgraduate eligibility (requires UK/EU citizenship): \url{http://www.nerc.ac.uk/funding/available/postgrad/eligibility.asp}
				\end{itemize}
			\item Research Fellowship Scheme [for all nationalities]: \url{http://www.nerc.ac.uk/funding/available/fellowships/}
			\item Research Experience Placements (REP) scheme [for undergraduates]: \url{http://www.nerc.ac.uk/funding/available/rep.asp}
			\item Research Grants: \vspace{-0.1cm}
				\begin{itemize} \itemsep -1pt
				\item Eligibility: \url{http://www.nerc.ac.uk/funding/available/researchgrants/eligibility.asp}
				\end{itemize}
			\end{enumerate}
		\item {\bf Other potential sources of funding}: \vspace{-0.1cm}
			\begin{enumerate} \itemsep -1pt
			\item \url{http://www.nerc.ac.uk/funding/otherfunding.asp}
			\item Look at the ``Higher Education Funding Councils'' for each country (England, Wales, Northern Ireland, and Scotland)
			\end{enumerate}
		\end{enumerate}
	\end{enumerate}
\item Nuffield Foundation: \vspace{-0.3cm}
	\begin{enumerate} \itemsep -2pt
	\item Undergraduate research bursaries in science: \url{http://www.nuffieldfoundation.org/undergraduate-research-bursaries-0}
	\item Funding for social policy projects in the UK: \vspace{-0.2cm}
		\begin{enumerate} \itemsep -2pt
		\item \url{http://www.nuffieldfoundation.org/social-policy}
		\item \url{http://www.nuffieldfoundation.org/children-and-families-law-society-education-and-open-door}
		\end{enumerate}
	\item Apply for funding: \url{http://www.nuffieldfoundation.org/apply-for-funding}
	\item Africa program: \url{http://www.nuffieldfoundation.org/africa-programme-0}
	\item Nuffield Farming Scholarships Trust: \vspace{-0.2cm}
		\begin{enumerate} \itemsep -2pt
		\item Nuffield Farming Scholarships: \url{http://www.nuffieldscholar.org/}
		\end{enumerate}
	\item The Nuffield Trust (or, The Nuffield Trust for Research and Policy Studies in Health Services): \vspace{-0.2cm}
		\begin{enumerate} \itemsep -2pt
		\item Fellowships: \vspace{-0.1cm}
			\begin{enumerate} \itemsep -1pt
			\item \url{http://www.nuffieldtrust.org.uk/fellowships/index.aspx?id=43}
			\item Rock Carling fellowship (for senior researchers in public health): \url{http://www.nuffieldtrust.org.uk/fellowships/index.aspx?id=112}
			\item John Fry Fellowship (for senior researchers in public health): \url{http://www.nuffieldtrust.org.uk/fellowships/index.aspx?id=109}
			\item Harkness Fellowships in Health Care Policy: \vspace{-0.1cm}
				\begin{itemize} \itemsep -1pt
				\item ``Since September 2009 The Nuffield Trust have been the proud co-sponsors of the prestigious Harkness Fellowships programme with The Commonwealth Fund.''
				\item ``These offer an unparalleled opportunity for the health policy analysts of the future to conduct original research and learn about healthcare in North America.''
				\item ``Mid-career health policy researchers and practitioners (including doctors, health services managers, journalists and government officials) are supported to spend 9 to 12 months in the United States conducting a policy-oriented research project and working with leading U.S. health policy experts.''
				\end{itemize}
			\end{enumerate}
		\end{enumerate}
	\end{enumerate}
\item U.S. Department of Homeland Security (DHS): \vspace{-0.3cm}
	\begin{enumerate} \itemsep -2pt
	\item DHS Scholarship and Fellowship Program: \url{http://www.orau.gov/dhsed/}
	\end{enumerate}
\item ACT, Inc.: \vspace{-0.3cm}
	\begin{enumerate} \itemsep -2pt
	\item Barry M. Goldwater Scholarship and Excellence in Education Program (for US residents who will be college upperclassmen in STEM fields in the following academic year): \url{http://www.act.org/goldwater/}
	\end{enumerate}
\item Massachusetts Institute of Technology: \vspace{-0.3cm}
	\begin{enumerate} \itemsep -2pt
	\item MIT School of Engineering: \vspace{-0.2cm}
		\begin{enumerate} \itemsep -2pt
		\item Lemelson-MIT Program: \vspace{-0.1cm}
			\begin{enumerate} \itemsep -1pt
			\item \url{http://web.mit.edu/invent/}
			\item Lemelson-MIT Awards for Invention and Innovation: \url{http://web.mit.edu/invent/a-main.html}
			\end{enumerate}
		\end{enumerate}
	\end{enumerate}
\item --- --- --- --- --- --- --- --- --- --- --- --- --- --- --- --- --- --- --- --- --- --- --- --- --- --- --- --- --- --- ---
\item \colorbox{blue}{\bf Scholarships and Fellowships in Various Fields (Including Creative Arts, Teaching, and Sports)}
% Scholarships and Fellowships in Various Fields (Including Creative Arts, Teaching, and Sports)
\item U.S. Department of Education: \vspace{-0.3cm}
	\begin{enumerate} \itemsep -2pt
	\item Robert C. Byrd Honors Scholarship Program: \vspace{-0.2cm}
		\begin{enumerate} \itemsep -2pt
		\item High school graduates who have been accepted for enrollment at institutions of higher education (IHEs), have demonstrated outstanding academic achievement, and show promise of continued academic excellence may apply to states in which they are residents.
		\item \url{http://www2.ed.gov/programs/iduesbyrd/index.html}
		\end{enumerate}
	\item \colorbox{yellow}{\bf Jacob K. Javits Fellowships Program}: \vspace{-0.1cm}
		\begin{enumerate} \itemsep -1pt
		\item This program provides fellowships to students of superior academic ability -- selected on the basis of demonstrated achievement, financial need, and exceptional promise -- to undertake study at the doctoral and Master of Fine Arts level in selected fields of arts, humanities, and social sciences.
		\item \url{http://www2.ed.gov/programs/jacobjavits/index.html}
		\end{enumerate}
	\item Close Up Fellowship Program: \vspace{-0.2cm}
		\begin{enumerate} \itemsep -2pt
		\item This program provides financial aid to enable low-income students, their teachers, and recent immigrants to come to Washington, D.C., to study the operations of the three branches of the federal government.
		\item \url{http://www2.ed.gov/programs/closeup/index.html}
		\end{enumerate}
	\item {\bf \color{blue} B.J. Stupak Olympic Scholarships}: \vspace{-0.2cm}
		\begin{enumerate} \itemsep -2pt
		\item This program provides financial assistance to athletes who are training at the U.S. Olympic Education Center or one of the U.S. Olympic training centers and who are pursuing a postsecondary education at institutions of higher education (IHEs).
		\item \url{http://www2.ed.gov/programs/olympic/index.html}
		\end{enumerate}
	\item {\bf \color{blue} Teacher Education Assistance for College and Higher Education (TEACH) Grant Program}: \vspace{-0.2cm}
		\begin{enumerate} \itemsep -2pt
		\item Through the College Cost Reduction and Access Act of 2007, Congress created the Teacher Education Assistance for College and Higher Education (TEACH) Grant Program that provides grants of up to \$4,000 per year to students who intend to teach in a public or private elementary or secondary school that serves students from low-income families.
		\item \url{http://studentaid.ed.gov/PORTALSWebApp/students/english/TEACH.jsp}
		\end{enumerate}
	\item Scholarship search engine: \url{https://studentaid2.ed.gov/getmoney/scholarship/}
	\item Financial Aid: \vspace{-0.2cm}
		\begin{enumerate} \itemsep -2pt
		\item \url{http://www2.ed.gov/finaid/landing.jhtml?src=rt}
		\item \url{http://studentaid.ed.gov/PORTALSWebApp/students/english/funding.jsp}
		\item Paying for college: \url{http://www.college.gov}
		\item Student Aid (has information for students at all levels and parents): \url{http://studentaid.ed.gov/}
		\item Student Aid Eligibility: \url{http://studentaid.ed.gov/PORTALSWebApp/students/english/aideligibility.jsp?tab=funding}
		\item Federal Student Aid: \url{http://federalstudentaid.ed.gov/}
		\item Academic Competitiveness Grant: The Academic Competitiveness Grant provides up to \$750 for the first year of undergraduate study and up to \$1,300 for the second year of undergraduate study. See \url{http://studentaid.ed.gov/PORTALSWebApp/students/english/NewPrograms.jsp}.
		\end{enumerate}
	\item Free Application for Federal Student Aid (FAFSA): \vspace{-0.2cm}
		\begin{enumerate} \itemsep -2pt
		\item Financial Aid Estimator Tool (FAFSA4caster): \url{http://www.fafsa4caster.ed.gov/F4CApp/index/index.jsf}
		\item \url{http://www.fafsa.ed.gov/}
		\end{enumerate}
	\item Federal Pell Grant Program: \url{http://www2.ed.gov/programs/fpg/index.html}
	\end{enumerate}
\item European Commission: \vspace{-0.3cm}
	\begin{enumerate} \itemsep -2pt
	\item Erasmus Programme (for Europeans): \url{http://ec.europa.eu/education/lifelong-learning-programme/doc80_en.htm}
	\item Erasmus Mundus (for non-Europeans): \url{http://ec.europa.eu/education/external-relation-programmes/doc72_en.htm}
	\end{enumerate}
\item Woodrow Wilson Foundation: \vspace{-0.3cm}
	\begin{enumerate} \itemsep -2pt
	\item {\bf \color{blue} The Woodrow Wilson-Rockefeller Brothers Fund Fellowships for Aspiring Teachers of Color (for underrepresented minorities seeking a career as a K-12 public school teacher in the US)}: \url{http://www.woodrow.org/teaching-fellowships/wwrbf/index.php}
	\item {\bf \color{blue} Woodrow Wilson Teaching Fellowship (for a MS program in teacher education, who would teach at high-need urban and rural schools or $\ge$ 3 years)}: \url{http://www.wwteachingfellowship.org/}
	\item {\bf \color{blue} Leonore Annenberg Teaching Fellowship (for a MS program in teacher education, who would teach at high-need urban and rural schools or $\ge$ 3 years)}: \url{http://www.woodrow.org/teaching-fellowships/annenberg/index.php}
	\item MMUF Travel \& Research Grants (for graduate students who participated in the Mellon Mays Undergraduate Fellowship Program): \url{http://www.woodrow.org/higher-education-fellowships/opportunity/research/index.php}
	\item MMUF Dissertation Grants (for graduate students who participated in the Mellon Mays Undergraduate Fellowship Program): \url{http://www.woodrow.org/higher-education-fellowships/opportunity/dissertation/index.php}
	\item Charlotte W. Newcombe Doctoral Dissertation Fellowship (for Ph.D. students writing their theses on ethical or religious values in all fields of the humanities and social sciences): \url{http://www.woodrow.org/higher-education-fellowships/religion_ethics/index.php}
	\item {\bf \color{blue} Woodrow Wilson Dissertation Fellowship in Women�s Studies}: \url{http://www.woodrow.org/higher-education-fellowships/women_gender/index.php}
	\item Doris Duke Conservation Fellowship program (Masters students seeking careers as practicing conservationists): \url{http://www.woodrow.org/higher-education-fellowships/conservation/index.php}
	\item Thomas R. Pickering Graduate Foreign Affairs Fellowship: \vspace{-0.2cm}
		\begin{enumerate} \itemsep -2pt
		\item Prior to joining the United States Department of State Foreign Service, this fellowship supports students entering a Masters program in the following fields: \vspace{-0.1cm}
			\begin{enumerate} \itemsep -1pt
			\item {\bf public policy}
			\item international affairs
			\item public administration
			\item academic fields such as: \vspace{-0.1cm}
				\begin{itemize} \itemsep -1pt
				\item business
				\item economics
				\item political science
				\item sociology
				\item foreign languages
				\end{itemize}
			\end{enumerate}
		\item \url{http://www.woodrow.org/higher-education-fellowships/foreign_affairs/pickering_grad/index.php}
		\end{enumerate}
	\item Thomas R. Pickering Undergraduate Foreign Affairs Fellowship (for undergraduates seeking to join the United States Department of State Foreign Service): \url{http://www.woodrow.org/higher-education-fellowships/foreign_affairs/pickering_undergrad/index.php}
	\end{enumerate}
\item Burroughs Wellcome Fund: \vspace{-0.3cm}
	\begin{enumerate} \itemsep -2pt
	\item Career Awards for Medical Scientists (post-Ph.D.): \url{http://www.bwfund.org/pages/188/Career-Awards-for-Medical-Scientists/}
	\item {\bf \color{blue} Career Award for Science and Mathematics Teachers (science or mathematics K-12 teachers in North Carolina public schools)}: \url{http://www.bwfund.org/pages/379/Career-Awards-for-Science-and-Mathematics-Teachers/}
	\end{enumerate}
\item Susan G. Komen for the Cure\textregistered: The Komen College Scholarship Program, \url{http://ww5.komen.org/ResearchGrants/CollegeScholarshipAward.html}
\item University of Kansas Madison \& Lila Self Graduate Fellowship (Ph.D. fellowships for business, economics, and STEM): \url{http://www2.ku.edu/~selfpro/}
\item Nationally Coveted College Scholarships, Graduate School Fellowships \& Postdoctoral Awards: \url{http://scholarships.fatomei.com/}
\item The Andrew W. Mellon Foundation: \vspace{-0.3cm}
	\begin{enumerate} \itemsep -2pt
	\item Fellowships \& Scholarships for undergraduates: \url{http://www.mmuf.org/undergraduates/explore-your-opportunities/fellowships-scholorships}
	\end{enumerate}
\item Siebel Scholars Foundation: \vspace{-0.3cm}
	\begin{enumerate} \itemsep -2pt
	\item For students in selected business, bioengineering, and computer science graduate programs
	\item Only available for students at selected universities.
	\item \url{http://www.siebelscholars.com/scholars}
	\item \url{http://www.siebelscholars.com/}
	\end{enumerate}
\item Aspen Institute (for leaders, e.g. in business, education, community service, and politics): \vspace{-0.3cm}
	\begin{enumerate} \itemsep -2pt
	\item Catto Fellowship Program: \url{http://www.aspeninstitute.org/leadership-programs/catto-fellowship-program}
	\item Rodel Fellowship Program: \url{http://www.aspeninstitute.org/leadership-programs/aspen-institute-rodel-fellowships-public-le-/about-rodel-fellowship-program}
	\item Henry Crown Fellowship Program: \url{http://www.aspeninstitute.org/leadership-programs/henry-crown-fellowship-program}
	\end{enumerate}
\item Smithsonian Institution: \vspace{-0.3cm}
	\begin{enumerate} \itemsep -2pt
	\item Postdoctoral Fellowships, Predoctoral Fellowships, and Graduate Student Fellowships: \vspace{-0.2cm}
		\begin{enumerate} \itemsep -2pt
		\item \url{http://www.si.edu/ofg/infotoapply.htm}
		\item \url{http://www.si.edu/ofg/fell.htm}
		\item \url{http://www.si.edu/ofg/ofgapp.htm}
		\item fields of research and study: \vspace{-0.1cm}
			\begin{enumerate} \itemsep -1pt
			\item {\bf \color{blue} American History, American Material and Folk Culture, and the History of Music and Musical Instruments}
			\item History of Science and Technology
			\item {\bf \color{blue} History of Art, Design, Crafts, and the Decorative Arts}
			\item Anthropology, Archaeology, Linguistics, and Ethnic Studies
			\item Evolutionary, Systematic, Behavioral, Environmental, and Conservation Biology
			\item Earth, Mineral, and Planetary Science
			\item Materials Characterization and Conservation
			\end{enumerate}
		\end{enumerate}
	\item Internship opportunities: \url{http://www.si.edu/ofg/internopp.htm}
	\item Research centers: \url{http://www.si.edu/research/}. [ It also has lots of information for K-12 teachers. It has resources, funding, and internship opportunities for undergraduates and graduate students pursing research in various aspects of humanities, social science, and natural science. ]
	\item Freer Gallery of Art / Arthur M. Sackler Gallery: \vspace{-0.2cm}
		\begin{enumerate} \itemsep -2pt
		\item Fellowships: \url{http://www.asia.si.edu/research/fellowships.asp}
		\end{enumerate}
	\item National Museum of American History: \vspace{-0.2cm}
		\begin{enumerate} \itemsep -2pt
		\item Jerome and Dorothy Lemelson Center for the Study of Invention and Innovation: \vspace{-0.1cm}
			\begin{enumerate} \itemsep -1pt
			\item The Lemelson Center Fellows Program (for Ph.D. students and postdocs): \url{http://invention.smithsonian.org/resources/research_fellowships.aspx}
			\end{enumerate}
		\end{enumerate}
	\end{enumerate}
\item Intercollegiate Studies Institute (ISI): \vspace{-0.3cm}
	\begin{enumerate} \itemsep -2pt
	\item William E. Simon Fellowship for Noble Purpose (for American undergraduates who are planning to use the fellowship grant for serving humanity -- in their own ways): \url{http://www.isi.org/programs/fellowships/simon.html}
	\item {\bf \color{blue} Richard M. Weaver Fellowship (for Americans who are attending a graduate program and are intending to pursue a career in academia/teaching)}: \url{http://www.isi.org/programs/fellowships/richard_weaver.html}
	\item Western Civilization Fellowships (for Americans who are attending a graduate program about Western culture/civilization): \url{http://www.isi.org/programs/fellowships/western_civilization.html}
	\item Salvatori Fellowship (for Americans who are attending a graduate program about early American history): \url{http://www.isi.org/programs/fellowships/salvatori.html}
	\item Bache Renshaw Fellowship for Doctoral Study in Education (for Americans who plan to attend doctoral programs in education): \url{http://www.isi.org/programs/fellowships/bache_renshaw.html}
	\item \url{http://www.isi.org/programs/fellowships/fellowships.html}
	\end{enumerate}
\item Le Fonds qu{\'{e}}b{\'{e}}cois de la recherche sur la nature et les technologies (The Quebec Research Fund on nature and technology): \vspace{-0.3cm}
	\begin{enumerate} \itemsep -2pt
	\item Scholarships: \url{http://www.fqrnt.gouv.qc.ca/en/bourses/index.htm}
	\end{enumerate}
\item Horatio Alger Association of Distinguished Americans, Inc.: \vspace{-0.3cm}
	\begin{enumerate} \itemsep -2pt
	\item Scholarship Programs (for US high school seniors who have faced and overcome great obstacles in their young lives): \url{https://www.horatioalger.org/scholarships/sp.cfm}
	\item Awards: \vspace{-0.2cm}
		\begin{enumerate} \itemsep -2pt
		\item \url{http://www.horatioalger.org/aboutus.cfm}
		\item Horatio Alger Award: ``dedicated community leaders who demonstrate individual initiative and a commitment to excellence; as exemplified by remarkable achievements accomplished through honesty, hard work, self-reliance and perseverance over adversity''
		\item International Horatio Alger Award: ``recipients of this award must have overcome humble beginnings and/or adversity to achieve success. They serve as outstanding role models to the international community and are committed to the Association's mission of encouraging and educating today's young people.''
		\item Norman Vincent Peale Award: ``a Member who has made exceptional humanitarian contributions to society, who has been an active participant in the Association, and who continues to exhibit courage, tenacity and integrity in the face of great challenges. ''
		\end{enumerate}
	\end{enumerate}
\item The W. Garfield Weston Foundation: \vspace{-0.3cm}
	\begin{enumerate} \itemsep -2pt
	\item Entrance Awards \& Upper Year Garfield Weston Awards (for students pursuing college or CEGEP studies in Canada): \url{http://www.garfieldwestonawards.ca/en/about}
	\end{enumerate}
\item Canadian Merit Scholarship Foundation (\url{http://www.cmsf.ca/}): Loran Award (undergraduate funding for Canadian citizens and permanent residents), \url{http://www.loranaward.ca/}
\item StartingBloc: \vspace{-0.3cm}
	\begin{enumerate} \itemsep -2pt
	\item StartingBloc Fellowship: \vspace{-0.2cm}
		\begin{enumerate} \itemsep -2pt
		\item \url{http://www.startingbloc.org/fellowship}
		\item For people who believe that economic value creation and social value creation are complementary... For people who believe in making money and doing good, and creating social and economic impact... 
		\item The Institute for Social Innovation is a ``conference'' to learn about global issues, ``corporate social responsibility, social entrepreneurship, cross sector partnerships and sustainability. Sessions are led by top academics, corporate innovators, social entrepreneurs, activists and government officials.'' 
		\end{enumerate}
	\end{enumerate}
\item The John D. and Catherine T. MacArthur Foundation: \vspace{-0.3cm}
	\begin{enumerate} \itemsep -2pt
	\item Applying for Grants: \url{http://www.macfound.org/site/c.lkLXJ8MQKrH/b.913959/k.E1BE/Applying_for_Grants.htm}
	\item Financial \& Grant Information: \url{http://www.macfound.org/site/c.lkLXJ8MQKrH/b.938093/k.9E4C/Financial__Grant_Information.htm}
	\item MacArthur Fellows Program: \url{http://www.macfound.org/site/c.lkLXJ8MQKrH/b.959463/k.9D7D/Fellows_Program.htm}
	\end{enumerate}
\item Wenner-Gren Foundations (The Wenner-Gren Center Foundation for Scientific Research, The Axel Wenner-Gren Foundation for International Exchange of Scientists and The Foundation Wenner-Grenska Samfundet): Fellowships (for Swedish postdocs), \url{http://www.swgc.org/stipendier.aspx}
\item {\'{E}}gide: \vspace{-0.3cm}
	\begin{enumerate} \itemsep -2pt
	\item EGIDE Latitudes: \url{http://www.egidelatitudes.fr/jahia/Jahia/site/egidelatitudes}
	\item Call for applications to scholarship opportunities (including a scholarship for French citizens to study abroad): \url{http://www.egide.asso.fr/jahia/Jahia/accueil/appels}
	\item Eiffel excellence scholarship programme (organized by the French Ministry of Foreign and European Affairs): \vspace{-0.2cm}
		\begin{enumerate} \itemsep -2pt
		\item \url{http://www.egide.asso.fr/jahia/Jahia/appels/eiffel}
		\item For non-French citizens pursuing advanced degrees.
		\end{enumerate}
	\end{enumerate}
\item Gottlieb Daimler and Karl Benz Foundation: \vspace{-0.3cm}
	\begin{enumerate} \itemsep -2pt
	\item {\bf \color{blue} Ph.D. fellowship for international students to study in Germany}; see \url{http://www.daimler-benz-stiftung.de/home/fellowship/en/start.html}
	\end{enumerate}
\item The San Diego Foundation: \vspace{-0.3cm}
	\begin{enumerate} \itemsep -2pt
	\item San Diego Foundation Community Scholarship Program: \vspace{-0.2cm}
		\begin{enumerate} \itemsep -2pt
		\item \url{http://www.sdfoundation.org/GrantsScholarships/Scholarships.aspx}
		\item Available scholarships: \url{http://www.sdfoundation.org/GrantsScholarships/Scholarships/ForStudents/AvailableScholarships.aspx}. Also, see \url{http://www.sdfoundation.org/GrantsScholarships/Scholarships/ForStudents/AvailableScholarships/CommonApplicationScholarships.aspx#twomey}
		\item It has scholarships for: \vspace{-0.1cm}
			\begin{enumerate} \itemsep -1pt
			\item graduating high school seniors
			\item current undergraduates
			\item non-traditional college students: \vspace{-0.1cm}
				\begin{itemize} \itemsep -1pt
				\item mature-age students
				\item mature student
				\item adult learner
				\item adult student
				\item adults who are returning to college
				\end{itemize}
			\item people pursuing teaching certificates
			\item students attending grad school
			\item students attending trade/vocational school
			\item foster youth
			\item students in various ethnic groups
			\item students in different geographical locations
			\item {\bf \color{blue} students pursuing education in certain fields, such as engineering, nursing, music, and arts and humanities}
			\end{enumerate}
		\item Separate Scholarships: \url{http://www.sdfoundation.org/GrantsScholarships/Scholarships/ForStudents/AvailableScholarships/SeparateScholarships.aspx}
		\item Other Scholarships and Financial Aid Resources: \url{http://www.sdfoundation.org/GrantsScholarships/Scholarships/ForStudents/AvailableScholarships/OtherScholarshipsandFinancialAidResources.aspx}
		\item Financial Aid Information: \url{http://www.sdfoundation.org/GrantsScholarships/Scholarships/ForStudents/Resources/FinancialAidInformation.aspx}
		\end{enumerate}
	\item Grant Opportunities (for non-profit organizations): \url{http://www.sdfoundation.org/GrantsScholarships/ForNonprofits/GrantOpportunities.aspx}
	\end{enumerate}
\item Ewing Marion Kauffman Foundation: \vspace{-0.3cm}
	\begin{enumerate} \itemsep -2pt
	\item Kauffman Dissertation Fellowship Program (for ``Ph.D., D.B.A., or other doctoral students at accredited U.S. universities to support dissertations in the area of entrepreneurship''): \url{http://www.kauffman.org/research-and-policy/kauffman-dissertation-fellowship-program.aspx}
	\item Kauffman Junior Faculty Fellowship in Entrepreneurship Research: \vspace{-0.2cm}
		\begin{enumerate} \itemsep -2pt
		\item \url{http://www.kauffman.org/research-and-policy/kauffman-junior-faculty-fellowship-in-entrepreneurship.aspx}
		\item ``to recognize tenured or tenure-track junior faculty members at accredited U.S. universities who are beginning to establish a record of scholarship and exhibit the potential to make significant contributions to the body of research in the field of entrepreneurship''
		\end{enumerate}
	\item Ewing Marion Kauffman Prize Medal for Distinguished Research in Entrepreneurship (for promising young scholars in the field of entrepreneurship): \url{http://www.kauffman.org/research-and-policy/kauffman-prize-medal-for-entrepreneurship-research.aspx}
	\item Kauffman Legal Fellowship Program (for post-J.D. research fellowship): \url{http://www.kauffman.org/research-and-policy/kauffman-legal-fellowship-program.aspx}
	\item Kauffman Global Scholars Program (for non-American top young entrepreneurs): \url{http://www.kauffman.org/entrepreneurship/kauffman-global-scholars-program.aspx}
	\item Entrepreneur Fellows program (for M.D.s and Ph.D.s who want to become high-tech start-up entrepreneurs): \url{http://www.kauffman.org/entrepreneurship/entrepreneur-fellows-program.aspx}
	\item Entrepreneur Postdoctoral Fellows program (for postdocs who want to become high-tech start-up entrepreneurs): \url{http://www.kauffman.org/entrepreneurship/entrepreneur-postdoctoral-fellows-program.aspx}
	\item Kauffman Fellows Program (``to educate and train future venture capitalists and future leaders of high-growth companies''): \url{http://www.kauffman.org/entrepreneurship/kauffman-fellows.aspx}
	\item Kauffman Foundation Outstanding Postdoctoral Entrepreneur Award: \url{http://www.kauffman.org/entrepreneurship/outstanding-postdoctoral-entrepreneur-award.aspx}
	\end{enumerate}
\item Killam Fellowships Program: \vspace{-0.3cm}
	\begin{enumerate} \itemsep -2pt
	\item \url{http://www.killamfellowships.com/}
	\item The Killam Fellowships Program allows undergraduate students from Canada and the United States to participate in a program of binational residential exchange.
	\item Killam Fellows spend either one semester or a full academic year as an exchange student in the host country.
	\end{enumerate}
\item Canada Council for the Arts: \vspace{-0.3cm}
	\begin{enumerate} \itemsep -2pt
	\item Killam Research Fellowship: \vspace{-0.2cm}
		\begin{enumerate} \itemsep -2pt
		\item \url{http://killam.canadacouncil.ca/welcome.asp}
		\item For researchers in the following fields, and interdisciplinary fields between these fields: \vspace{-0.1cm}
			\begin{enumerate} \itemsep -1pt
			\item humanities
			\item social sciences
			\item natural sciences
			\item health sciences
			\item engineering
			\end{enumerate}
		\item For outstanding researchers who are Canadian citizens or permanent residents
		\end{enumerate}
	\item Killam Prizes (and Killam Research Fellowships): \url{http://www.canadacouncil.ca/prizes/killam}
	\end{enumerate}
\item Killam Trusts: \vspace{-0.3cm}
	\begin{enumerate} \itemsep -2pt
	\item Killam Scholarship and Prize Programs (multiple fields in selected Canadian universities): \url{http://www.killamtrusts.ca/index.asp}
	\item Killam Award winners: \url{http://www.killamtrusts.ca/awardwinners.asp}
	\item Killam Scholarship and Prize Programs at various institutions (including universities): \url{http://www.killamtrusts.ca/uofAlberta.asp}
	\end{enumerate}
\item U.S. Department of State: \vspace{-0.3cm}
	\begin{enumerate} \itemsep -2pt
	\item Bureau of Educational and Cultural Affairs: \vspace{-0.2cm}
		\begin{enumerate} \itemsep -2pt
		\item Institute of International Education (administrator of program): \vspace{-0.1cm}
			\begin{enumerate} \itemsep -1pt
			\item Council for International Exchange of Scholars: \vspace{-0.1cm}
				\begin{itemize} \itemsep -1pt
				\item Fulbright Programs (for U.S. and non-U.S. Scholars): \url{http://www.cies.org/Fulbright_programs.htm}; \url{http://www.cies.org/about_fulb.htm}; \url{http://us.fulbrightonline.org/about.html}; \url{http://foreign.fulbrightonline.org/}; \url{http://exchanges.state.gov/academicexchanges/index/fulbright-program.html}; and \url{http://fulbright.state.gov/}
				\item Hubert H. Humphrey Fellowship Program: \vspace{-0.1cm}
					\begin{itemize} \itemsep -1pt
					\item For mid-career professionals in the following fields: economic development/finance and banking, agricultural and rural development, natural resources, environmental policy, and climate change, human resource management, communications/journalism, teaching of English as a foreign language, educational administration, planning, and policy, substance abuse education, treatment, and prevention, HIV/AIDS policy and prevention, public health policy and management, {\bf public policy} analysis and public administration, law and human rights, urban and regional planning, trafficking in persons - policy and prevention, technology policy and management, and higher education administration
					\item \url{http://www.humphreyfellowship.org/}
					\item \url{http://exchanges.state.gov/globalexchanges/humphrey-fellowship.html}
					\end{itemize}
				\end{itemize}
			\item International programs for scholars (search under each continent): \url{http://www.iie.org/en/Our-Global-Reach}
			\end{enumerate}
		\item International Documentary Filmmakers Fellowship: \vspace{-0.1cm}
			\begin{enumerate} \itemsep -1pt
			\item \url{http://exchanges.state.gov/cultural/docfilmmakers.html}
			\item \url{http://smpa.gwu.edu/doccenter/fellowship.php}
			\item For ``emerging or mid-career documentary filmmakers''
			\item Intensive six-week program at the Documentary Center, The George Washington University
			\end{enumerate}
		\item Office of English Language Programs: \vspace{-0.1cm}
			\begin{enumerate} \itemsep -1pt
			\item English Language Fellow Program (for ``highly qualified U.S. educators in the field of Teaching English to Speakers of Other Languages, TESOL''): \url{http://exchanges.state.gov/englishteaching/el-fellow.html}
			\item English Language Specialist Program: \vspace{-0.1cm}
				\begin{itemize} \itemsep -1pt
				\item \url{http://exchanges.state.gov/englishteaching/el-specialist.html}
				\item U.S. academics in the fields of Teaching English as a Foreign Language (TEFL) / Teaching English as a Second Language (TESL) and Applied Linguistics
				\end{itemize}
			\item E-Teacher Scholarship Program (for English teaching professionals living outside of the United States): \url{http://exchanges.state.gov/englishteaching/eteacher.html}
			\item English Access Microscholarship Program (Access): \vspace{-0.1cm}
				\begin{itemize} \itemsep -1pt
				\item \url{http://exchanges.state.gov/englishteaching/eam.html}
				\item The English Access Microscholarship Program (Access) provides a foundation of English language skills to non-elite, 14 - 18 year old students through afterschool classes and intensive summer learning activities.
				\end{itemize}
			\item \url{http://exchanges.state.gov/englishteaching/index.html}
			\end{enumerate}
		\item Office of Global Educational Programs: \vspace{-0.1cm}
			\begin{enumerate} \itemsep -1pt
			\item Community College Initiative: \vspace{-0.1cm}
				\begin{itemize} \itemsep -1pt
				\item For ``individuals from Brazil, Egypt, Ghana, Indonesia, Pakistan, South Africa, Turkey, and selected countries in Central America to spend one year studying at community colleges in the United States and earn a vocational certificate.''
				\item ``The program provides academic instruction in selected fields including agriculture, applied engineering, business management and administration, health professions, information technology, media, and tourism and hospitality management, while also immersing participants in U.S. society and cultural life.''
				\item ``Participants are recruited from historically underserved populations and may not have had opportunities for formal job training or higher education. Most participants are in their early- to mid-twenties and many already have work experience.''
				\item \url{http://exchanges.state.gov/globalexchanges/community-colleges-initiative.html}
				\end{itemize}
			\item {\bf \color{blue} Benjamin A. Gilman International Scholarship Program}: \vspace{-0.1cm}
				\begin{itemize} \itemsep -1pt
				\item ``The Benjamin A. Gilman International Scholarship Program provides scholarships to U.S. undergraduates with financial need for study abroad, including students from diverse backgrounds and students going to non-traditional study abroad destinations.''
				\item ``The applicant must be receiving a Federal Pell Grant or provide proof that he/she will be receiving a Pell Grant at the time of application or during the term of his/her study abroad.''
				\item \url{http://exchanges.state.gov/globalexchanges/gilman-scholarship-program.html}
				\end{itemize}
			\item Global Undergraduate Exchange Program (Global UGRAD Program): \vspace{-0.1cm}
				\begin{itemize} \itemsep -1pt
				\item \url{http://exchanges.state.gov/academicexchanges/guep.html}
				\item The Global Undergraduate Exchange Program (also known as the Global UGRAD Program) provides one semester and academic year scholarships to outstanding undergraduate students from underrepresented sectors in East Asia, Eurasia and Central Asia, the Near East and South Asia and the Western Hemisphere for non-degree full-time study combined with community service, internships and cultural enrichment.
				\end{itemize}
			\item Professors and Research Scholars: \url{http://exchanges.state.gov/jexchanges/programs/professor.html}
			\item Short-Term Scholar: \url{http://exchanges.state.gov/jexchanges/programs/shortterm.html}
			\item Student, College/University: \vspace{-0.1cm}
				\begin{itemize} \itemsep -1pt
				\item \url{http://exchanges.state.gov/jexchanges/programs/ucstudent.html}
				\item The College/University Student Program gives foreign students the opportunity to study at an American degree-granting post-secondary accredited educational institution, including colleges and universities. Students may participate in degree and non-degree programs. They must pursue a full-time course of study and maintain satisfactory advancement toward the completion of their academic program.
				\end{itemize}
			\item Study of the United States Institutes for Scholars: \vspace{-0.1cm}
				\begin{itemize} \itemsep -1pt
				\item Study of the United States Institutes for Scholars  are designed to strengthen curricula and improve the quality of teaching about the United States in academic institutions overseas.
				\item Foreign university faculty, secondary educators and other scholars spend approximately four weeks at host universities where they take part in a series of lectures, seminar discussions and site visits related to each institute's theme.
				\item They learn about American educational philosophies, explore new teaching methods and pursue related research interests.
				\item Interests of these institutes: \vspace{-0.1cm}
					\begin{itemize} \itemsep -1pt
					\item American Politics and Political Thought
					\item Contemporary American Literature
					\item Journalism and Media
					\item Religious Pluralism in the United States
					\item Secondary School Educators
					\item U.S. Culture and Society
					\item U.S. Foreign Policy
					\item U.S. National Security
					\end{itemize}
				\item \url{http://exchanges.state.gov/academicexchanges/scholars.html}
				\end{itemize}
			\item Study of the United States Institutes for Student Leaders: \vspace{-0.1cm}
				\begin{itemize} \itemsep -1pt
				\item Study of the United States Institutes for Student Leaders are five-to-six-week academic programs for foreign undergraduate leaders.
				\item Hosted by U.S. academic institutions throughout the United States, the Student Leader Institutes include an intensive academic component, an educational tour of other regions of the country, local community service activities and a unique opportunity for participants to get to know their American peers.
				\item \url{http://exchanges.state.gov/academicexchanges/students.html}
				\item Interests of the institutes: \vspace{-0.1cm}
					\begin{itemize} \itemsep -1pt
					\item Comparative {\bf Public Policy} for Pakistani Student Leaders
					\item Energy and the Environment
					\item Global Environmental Issues
					\item New Media
					\item Religious Pluralism in the U.S.
					\item Social Entrepreneurship
					\item U.S. Foreign Policy for East Asian Student Leaders
					\item Western Hemisphere Student Leaders 
					\item Women's Leadership
					\end{itemize}
				\end{itemize}
			\item Edmund S. Muskie Graduate Fellowship: \vspace{-0.1cm}
				\begin{itemize} \itemsep -1pt
				\item \url{http://exchanges.state.gov/academicexchanges/muskie.html}
				\item The Edmund S. Muskie Graduate Fellowship Program (Muskie) confers fellowships for Master's degree-level study in the U.S. in the fields of business administration, economics, education, environmental policy and management, international affairs, journalism/mass communications, law, library and information science, public administration, public health and {\bf public policy} for students and professionals from Eurasia.
				\item Candidates are recruited through a merit-based competition administered by the International Research \& Exchanges Board (IREX).
				\item U.S. host campuses are also selected through a competition process and generally provide tuition waivers of fifty percent.
				\item Approximately 145 fellowships are awarded each academic year.
				\end{itemize}
			\item Critical Language Scholarship Program: \vspace{-0.1cm}
				\begin{itemize} \itemsep -1pt
				\item \url{http://exchanges.state.gov/academicexchanges/sli2.html}
				\item The Critical Language Scholarship (CLS) Program provides overseas foreign language instruction and cultural enrichment experiences in 13 critical need languages for U.S. students in higher education.
				\item The CLS Program is part of a U.S. government effort to expand dramatically the number of Americans studying and mastering critical need foreign languages.
				\item Undergraduate, master's and doctoral-level students of diverse disciplines and majors are encouraged to apply for the seven-to-10-week-long programs.
				\item Participants are expected to continue their language study beyond the scholarship period, and later apply their critical language skills in their future professional careers.
				\end{itemize}
			\item Critical Language Enhancement Award (CLEA): \vspace{-0.1cm}
				\begin{itemize} \itemsep -1pt
				\item \url{http://exchanges.state.gov/academicexchanges/clea2.html}
				\item The Critical Language Enhancement Award (CLEA) provides funding to eligible Fulbright U.S. Student Program Grantees who intend to use one of the following languages for their Fulbright project: \vspace{-0.1cm}
					\begin{itemize} \itemsep -1pt
					\item Arabic (all dialiects)
					\item Azeri
					\item Bangla/Bengali
					\item Bhasa Indonesia
					\item Chinese (Mandarin Only)
					\item Farsi
					\item Gujarati
					\item Hindi
					\item Korean
					\item Marathi
					\item Pashto
					\item Punjabi
					\item Russian
					\item Turkish
					\item Urdu
					\end{itemize}
				\end{itemize}
			\end{enumerate}
		\item Office of International Visitors: \vspace{-0.1cm}
			\begin{enumerate} \itemsep -1pt
			\item International Visitor Leadership Program (IVLP): \vspace{-0.1cm}
				\begin{itemize} \itemsep -1pt
				\item \url{http://exchanges.state.gov/ivlp/index.html}
				\item \url{http://exchanges.state.gov/ivlp/ivlp.html}
				\item The Office of International Visitors manages and funds the International Visitor Leadership Program (IVLP).
				\item Launched in 1940, the IVLP is a professional exchange program that seeks to build mutual understanding between the U.S. and other nations through carefully designed short-term visits to the U.S. for current and emerging foreign leaders.
				\item These visits reflect the International Visitors' professional interests and support the foreign policy goals of the United States.
				\end{itemize}
			\end{enumerate}
		\item Program Search (find international exchange programs sponsored by the Bureau of Educational and Cultural Affairs): \url{http://exchanges.state.gov/index/search.html}
		\end{enumerate}
	\end{enumerate}
\item Mexican American Legal Defense and Educational Fund (MALDEF): \vspace{-0.3cm}
	\begin{enumerate} \itemsep -2pt
	\item Scholarship Resources: \url{http://maldef.org/leadership/scholarships/}
	\item MALDEF Law School Scholarship Program: \vspace{-0.2cm}
		\begin{enumerate} \itemsep -2pt
		\item MALDEF's Law School Scholarship Program provides several scholarships in varying amounts to deserving law students with a commitment to advancing the civil rights of Latinos.
		\item MALDEF's Law School Scholarship Program is open to all law students who will be enrolled full-time in an American-accredited law school in 2010-2011.
		\item Scholarships are awarded to students based on their commitment to serve the Latino community through law; their past achievement and potential for achievement; and their financial need.
		\item \url{http://maldef.org/leadership/scholarships/law_school_scholarship_program/index.html}
		\end{enumerate}
	\item Undergraduate Scholarship Resource Guide: \url{http://maldef.org/leadership/scholarships/resources/index.html}
	\end{enumerate}
\item Ashoka: \vspace{-0.3cm}
	\begin{enumerate} \itemsep -2pt
	\item Ashoka Fellows (to promote and support social entrepreneurship): \url{http://www.ashoka.org/fellows}
	\end{enumerate}
\item Heinz Family Foundation: \vspace{-0.3cm}
	\begin{enumerate} \itemsep -2pt
	\item Heinz Award Criteria: \vspace{-0.2cm}
		\begin{enumerate} \itemsep -2pt
		\item \url{http://heinzawards.net/awards/criteria}
		\item The Heinz Endowments
		\item Attributes and qualities of awardees: \vspace{-0.1cm}
			\begin{enumerate} \itemsep -1pt
			\item an enormous capacity to love
			\item smile
			\item take risks
			\item question
			\item work hard
			\item believe in the power of the individual to improve the lives of others
			\end{enumerate}
		\item ``Candidates [should] possess a remarkable mix of vision, optimism, creativity and hard work which, when combined, produce tangible achievements of lasting good.''
		\item Nominees must exhibit the following personal characteristics: \vspace{-0.1cm}
			\begin{enumerate} \itemsep -1pt
			\item A passion for excellence that goes beyond intellectual curiosity;
			\item A concern for humanity rooted in a deep sensitivity for the well-being of others; 
			\item A knowledge of self which acknowledges weaknesses but relies on individual strengths;
			\item A gritty determination that will see a job through to completion despite the inevitable setbacks;
			\item A broad vision which extends far beyond the particular and embraces something universal.
			\end{enumerate}
		\item Work of the candidates for a Heinz Award must meet the following criteria: \vspace{-0.1cm}
			\begin{enumerate} \itemsep -1pt
			\item Be significant and not a ``quick fix.''
			\item Have an enduring and meaningful impact.
			\item Be creative and innovative, and
			\item Be sufficiently tangible to serve as a model for replication elsewhere.
			\end{enumerate}
		\item ``In addition, candidates should be actively working in the field in which they are nominated with the hope that, in receiving this award, their potential for future societal contribution will be enhanced.''
		\end{enumerate}
	\item Categories: \vspace{-0.2cm}
		\begin{enumerate} \itemsep -2pt
		\item Arts \& Humanities
		\item Environment
		\item Human Condition
		\item {\bf Public Policy}
		\item Technology, Economy, + Employment
		\end{enumerate}
	\end{enumerate}
\item Echoing Green: \vspace{-0.3cm}
	\begin{enumerate} \itemsep -2pt
	\item Echoing Green Fellowship: \vspace{-0.2cm}
		\begin{enumerate} \itemsep -2pt
		\item \url{http://www.echoinggreen.org/fellowship}
		\item Has information on eligibility, the benefits of the fellowship, and application cycle and dates.
		\end{enumerate}
	\item Echoing Green Fellows: \url{http://www.echoinggreen.org/fellows}
	\end{enumerate}
\item Ben Franklin Technology Partners (BFTP): \vspace{-0.3cm}
	\begin{enumerate} \itemsep -2pt
	\item Innovation Works (IW): \vspace{-0.2cm}
		\begin{enumerate} \itemsep -2pt
		\item AlphaLab: \vspace{-0.1cm}
			\begin{enumerate} \itemsep -1pt
			\item ``An immersive environment where entrepreneurs can tap IW's onsite experts for business and market advice and exchange ideas with other entrepreneurs launching in similar markets''
			\end{enumerate}
		\end{enumerate}
	\end{enumerate}
\item Carnegie Corporation of New York: \vspace{-0.3cm}
	\begin{enumerate} \itemsep -2pt
	\item Carnegie Scholars Program (not available in 2010): \url{http://carnegie.org/programs/carnegie-scholars/}
	\end{enumerate}
\item New York Women's Foundation: \vspace{-0.3cm}
	\begin{enumerate} \itemsep -2pt
	\item Finch Scholar Program (with the Finch College Alumnae Association): \vspace{-0.2cm}
		\begin{enumerate} \itemsep -2pt
		\item \url{http://www.nywf.org/internship.html} and \url{http://www.finchcollege.org/}
		\item ``Our partnership with the Finch Scholar Program allows us to provide practical community service experience to an outstanding local student enrolled in college. The internship affords the Finch Scholar opportunities to work in meaningful ways in a nonprofit organization with exposure to social change philanthropy, participatory grantmaking, advocacy and {\bf public policy}. Generally, we offer one scholarship per year with a stipend.''
		\item \url{http://www.finchcollege.org/newFinchScholarPrgm.html}
		\item \url{http://www.finchcollege.org/newscholarships.html}
		\end{enumerate}
	\end{enumerate}
\item The Rockefeller Foundation: \vspace{-0.3cm}
	\begin{enumerate} \itemsep -2pt
	\item The Bellagio Center: \vspace{-0.2cm}
		\begin{enumerate} \itemsep -2pt
		\item \url{http://www.rockefellerfoundation.org/bellagio-center}
		\item Residency Programs: \vspace{-0.1cm}
			\begin{enumerate} \itemsep -1pt
			\item \url{http://www.rockefellerfoundation.org/bellagio-center/residency-programs}
			\item ``The Bellagio Residency program offers scholars, artists, thought leaders, policymakers and practitioners a serene setting conducive to focused, goal-oriented work, and the unparalleled opportunity to establish new connections with fellow residents, across a stimulating array of disciplines and geographies.  The Bellagio Center community generates new knowledge to solve some of the most complex problems facing our world and creates art that inspires reflection, understanding, and imagination.''
			\item Scholarly Residencies: \vspace{-0.1cm}
				\begin{itemize} \itemsep -1pt
				\item ``Researchers in the humanities, natural sciences, social sciences and other academic disciplines''
				\item ``The Center typically offers one-month residencies for no more than 12 scholars and scientists at a time. Individuals in any discipline and from any part of the world are welcome to apply. The Center maintains a core focus on projects consistent with the Foundation's mission to expand opportunities for poor or vulnerable people and to help see that the benefits of globalization are shared more widely. It also seeks to include beyond that core a wide variety of projects from all academic disciplines.''
				\item \url{http://www.rockefellerfoundation.org/bellagio-center/residency-programs/scholarly-residencies}
				\end{itemize}
			\item Creative Artist Residencies: \vspace{-0.1cm}
				\begin{itemize} \itemsep -1pt
				\item ``Artists, composers, writers''
				\item ``Bellagio creative artist residencies for composers, novelists, playwrights, poets, video/filmmakers and visual artists provide time for disciplined work, individual reflection, and collegial engagement, uninterrupted by the usual professional and personal demands. The Center typically offers one-month stays for no more than three to five creative artists at a time. Artists of significant achievement from any country are welcome to apply.''
				\item \url{http://www.rockefellerfoundation.org/bellagio-center/residency-programs/creative-artist-residencies}
				\end{itemize}
			\item Practitioner Residencies: \vspace{-0.1cm}
				\begin{itemize} \itemsep -1pt
				\item ``Policymakers, nonprofit leaders, journalists and public advocates''
				\item ``The Center offers residencies to professionals in fields relevant to the Rockefeller Foundation's issue areas. The Center maintains a core focus on projects consistent with our mission, to expand opportunities for poor or vulnerable people and to help see that the benefits of globalization are shared more widely.   We seek practitioner applicants with demonstrated leadership qualities and the capacity to contribute to the intellectual life at the Center.''
				\item \url{http://www.rockefellerfoundation.org/bellagio-center/residency-programs/practitioner-residencies}
				\end{itemize}
			\end{enumerate}
		\item {\bf \color{blue} Creative Arts Fellowships}: \vspace{-0.1cm}
			\begin{enumerate} \itemsep -1pt
			\item ``This high-profile program hosts visual artists at the Bellagio Center for three-month residencies that inspire creativity and promote interaction between the arts and other fields. Creative Arts Fellows, like other participants in Bellagio residency programs, have the time and space to work independently during the day. They also enjoy and benefit from a lively community of scholars, writers, policymakers and other artists who gather in the evening for dinner and occasional presentations.  The combination of private work space, an extended stay, a generous stipend and a unique group of fellow residents makes a Creative Arts Fellowship at the Bellagio Center a remarkable opportunity.''
			\item \url{http://www.rockefellerfoundation.org/bellagio-center/creative-arts-fellowships}
			\end{enumerate}
		\end{enumerate}
	\end{enumerate}
\item Wellcome Trust: \vspace{-0.3cm}
	\begin{enumerate} \itemsep -2pt
	\item Wellcome Trust Book Prize: \vspace{-0.2cm}
		\begin{enumerate} \itemsep -2pt
		\item \url{http://www.wellcomebookprize.org/About-the-prize/index.htm}
		\item ``The Wellcome Trust Book Prize celebrates the best of medicine in literature by awarding 25 000 each year for the finest fiction or non-fiction book centered around medicine.''
		\end{enumerate}
	\end{enumerate}
\item The Kennedy Memorial Trust: \vspace{-0.3cm}
	\begin{enumerate} \itemsep -2pt
	\item \url{http://www.kennedytrust.org.uk/}
	\item Kennedy Scholarship: \url{http://www.kennedytrust.org.uk/display.aspx?Id=1165&pid=0}
	\item Frank Knox Fellowships: \url{http://www.kennedytrust.org.uk/display.aspx?Id=1175&pid=0}
	\end{enumerate}
\item Foreign \& Commonwealth Office / United Kingdom: \vspace{-0.3cm}
	\begin{enumerate} \itemsep -2pt
	\item Chevening scholarships: \vspace{-0.2cm}
		\begin{enumerate} \itemsep -2pt
		\item \url{http://www.fco.gov.uk/en/about-us/what-we-do/scholarships/}
		\item ``The Chevening programme, has, over 26 years, provided more than 30,000 Scholarships at Higher Education Institutions (HEIs) in the UK for postgraduate students or researchers from countries across the world.''
		\end{enumerate}
	\item {\bf Marshall Scholarships} finance young Americans of high ability to study for a graduate degree in the United Kingdom: \url{http://www.marshallscholarship.org/}
	\end{enumerate}
\item Ministry of Education, Culture, Sports, Science and Technology (MEXT) / Japan: \vspace{-0.3cm}
	\begin{enumerate} \itemsep -2pt
	\item \url{http://www.mext.go.jp/english/}
	\item Monbukagakusho Scholarship: \vspace{-0.2cm}
		\begin{enumerate} \itemsep -2pt
		\item \url{http://en.wikipedia.org/wiki/Monbukagakusho_Scholarship}
		\item \url{http://project.monbusho.org/old/} and \url{http://www.monbusho.org/}
		\end{enumerate}
	\end{enumerate}
\item Institute of International Education (IIE): \vspace{-0.3cm}
	\begin{enumerate} \itemsep -2pt
	\item GE Foundation Scholar-Leaders Program: \vspace{-0.2cm}
		\begin{enumerate} \itemsep -2pt
		\item \url{http://www.iie.org/en/Programs/GE-Foundation-Scholar-Leaders-Program}
		\item ``The GE Foundation Scholar-Leaders Program began in 1987 in Mexico and now supports outstanding students in higher education in fourteen countries around the world. The program initially provided traditional financial support for university education, but has developed into an exciting Leadership Development Program to complement the student's academic curriculum.''
		\item Eligibility: ``Students in their first year of study in engineering, technology, business, finance, management, or economics attending a participating university. GE Foundation Scholar-Leaders qualification requirements vary by region.''
		\end{enumerate}
	\end{enumerate}
\item British Council: \vspace{-0.3cm}
	\begin{enumerate} \itemsep -2pt
	\item Shine! 2011: International Student Awards: \vspace{-0.2cm}
		\begin{enumerate} \itemsep -2pt
		\item \url{http://www.educationuk.org/shine}
		\item For international students in the United Kingdom
		\end{enumerate}
	\item Funding your studies: \vspace{-0.2cm}
		\begin{enumerate} \itemsep -2pt
		\item \url{http://www.britishcouncil.org/learning-funding-your-studies.htm}
		\item Education UK: \url{http://www.educationuk.org/pls/hot_bc/page_pls_user_advice?x=&y=&a=0&d=4460}
		\item 9/11 Scholarship Fund: \vspace{-0.1cm}
			\begin{enumerate} \itemsep -1pt
			\item \url{http://www.britishcouncil.org/911scholarships.htm}
			\item ``The 9/11 Scholarship Fund supports international students who were directly affected by the 2001 terrorist events in the US. Find out more how each scholarship offers the opportunity to study at a UK college or university every year.''
			\end{enumerate}
		\end{enumerate}
	\item {\it Youth in Action} European program: \url{http://www.britishcouncil.org/youthinaction}
	\item British Council Arts Group: \vspace{-0.2cm}
		\begin{enumerate} \itemsep -2pt
		\item Support and funding overview: \url{http://www.britishcouncil.org/arts-support-and-funding-overview.htm}
		\item Visual arts support and funding: \url{http://www.britishcouncil.org/arts-visual-arts-funding.htm}
		\item Drama and dance support and funding: \url{http://www.britishcouncil.org/arts-performing-arts-funding.htm}
		\item Literature support and funding: \url{http://www.britishcouncil.org/arts-literature-support-and-funding.htm}
		\item Film support and funding: \url{http://www.britishcouncil.org/arts-film-funding.htm}
		\item Music support and funding: \url{http://www.britishcouncil.org/arts-music-funding.htm}
		\item Architecture, design, fashion support and funding: \url{http://www.britishcouncil.org/arts-adf-funding.htm}
		\item International Short Film Festival Support Scheme: \url{http://www.britishcouncil.org/arts-film-short-films-scheme.htm}
		\end{enumerate}
	\end{enumerate}
\item Alfred P. Sloan Foundation: \vspace{-0.3cm}
	\begin{enumerate} \itemsep -2pt
	\item Sloan Research Fellowships: \vspace{-0.2cm}
		\begin{enumerate} \itemsep -2pt
		\item \url{http://www.sloan.org/fellowships}
		\item Hold a Ph.D. (or equivalent) in chemistry, physics, mathematics, computer science, economics, neuroscience or computational and evolutionary molecular biology, or in a related interdisciplinary field;
		\item Be members of the regular faculty (i.e., tenure track) of a degree-granting college or university in the United States or Canada; and
		\item Normally, be no more than six years from completion of the most recent Ph.D. or equivalent as of the year of their nomination.
		\end{enumerate}
	\end{enumerate}
\item --- --- --- --- --- --- --- --- --- --- --- --- --- --- --- --- --- --- --- --- --- --- --- --- --- --- --- --- --- --- ---
\item \colorbox{blue}{\bf Scholarships and Fellowships in Business (including Finance, Entrepreneurship, and Accounting)}
% Scholarships and Fellowships in Business (including Finance, Entrepreneurship, and Accounting)
\item IREX: \vspace{-0.3cm}
	\begin{enumerate} \itemsep -2pt
	\item Opportunities ``for individuals, organizations, universities, and alumni'': \url{http://www.irex.org/apply}
	\item Edmund S. Muskie Graduate Fellowship Program: \vspace{-0.2cm}
		\begin{enumerate} \itemsep -2pt
		\item : \url{http://www.irex.org/application/edmund-s-muskie-graduate-fellowship-program-application}
		\item ``The Muskie Program is open to graduate students and professionals from Armenia, Azerbaijan, Belarus, Georgia, Kazakhstan, Kyrgyzstan, Moldova, Russia, Tajikistan, Turkmenistan, Ukraine and Uzbekistan for one-year non-degree, one-year degree, or two-year degree study in the United States.''
		\item ``Eligible fields of study for the Muskie Program are: business administration, economics, education, environmental management, international affairs, journalism and mass communication, law, library and information science, public administration, public health, and {\bf public policy}.''
		\end{enumerate}
	\end{enumerate}
\item Sponsors for Educational Opportunity (SEO): \vspace{-0.3cm}
	\begin{enumerate} \itemsep -2pt
	\item Alternative Investment Fellowship Program: \vspace{-0.2cm}
		\begin{enumerate} \itemsep -2pt
		\item \url{http://www.seo-usa.org/Fellowship}
		\item Eligibility: \vspace{-0.1cm}
			\begin{enumerate} \itemsep -1pt
			\item \url{http://www.seo-usa.org/FellowshipEligibility}
			\item The program is open to professionals traditionally underrepresented in alternative investments who are in the first year (or second year with a third-year offer) of an analyst program at an investment bank.
			\item Corporate finance, M\&A, leveraged finance and structured finance analysts are preferred.
			\item Management consultants will also be considered.
			\end{enumerate}
		\end{enumerate}
	\item The SEO Scholars Program: \vspace{-0.2cm}
		\begin{enumerate} \itemsep -2pt
		\item \url{http://www.seo-usa.org/Scholars}
		\item The SEO Scholars Program is a rigorous out-of-school academic enrichment program that prepares motivated New York City public high school students of color to gain admission to and succeed at competitive colleges and universities throughout the country.  Numerous studies confirm that rigorous academics are the single most important factor for low-income and minority students in gaining college admission and earning a degree.  However, U.S. Department of Education research shows that ``A'' work in low-income schools equals ``C'' work in affluent schools.
		\item Admissions: \url{http://www.seo-usa.org/ScholarsAdmissions}
		\item Roadmap To Success: \url{http://www.seo-usa.org/ScholarsRoadmapToSuccess}
		\item Enrichment Programs: \url{http://www.seo-usa.org/ScholarsEnrichmentPrograms}
		\item Volunteering: \url{http://www.seo-usa.org/ScholarsVolunteering}
		\item Andrew Golkin Fund: \vspace{-0.1cm}
			\begin{enumerate} \itemsep -1pt
			\item \url{http://www.seo-usa.org/ScholarsAndrewGolkinFund}
			\item \url{http://www.seo-usa.org/andrewgolkinfund/index.html}
			\end{enumerate}
		\item Franklin H. and Shirley B. Williams Scholarship Fund: \url{http://www.seo-usa.org/ScholarsFHSBW}
		\item The Advantages of Attending a Competitive College: \url{http://www.seo-usa.org/ScholarsAdvantages}
		\end{enumerate}
	\item Career program: \vspace{-0.2cm}
		\begin{enumerate} \itemsep -2pt
		\item \url{http://www.seo-usa.org/Career}
		\item The SEO Career Program places students of color interested in finance, philanthropy, business and corporate law in internships with competitive pay, rigorous training, support through mentors, and broad access to industry professionals. 
		\item Sponsors for Educational Opportunity (SEO) is the nation's premiere summer internship program for talented underrepresented students of color that can lead to full-time job offers.
		\item SEO offers internship opportunities in the following areas: \vspace{-0.1cm}
			\begin{enumerate} \itemsep -1pt
			\item Corporate Financial Leadership: \url{http://www.seo-usa.org/Career/Corporate_Financial_Leadership}
			\item Banking/Asset Management Areas: \vspace{-0.1cm}
				\begin{itemize} \itemsep -1pt
				\item Investment Banking: \url{http://www.seo-usa.org/Career/Investment_Banking}
				\item Sales \& Trading: \url{http://www.seo-usa.org/Career/Sales_&_Trading}
				\item Investment Research: \url{http://www.seo-usa.org/Career/Investment_Research}
				\item Transaction Services: \url{http://www.seo-usa.org/Career/Transaction_Services}
				\item Asset Management: \url{http://www.seo-usa.org/Career/Asset_Management}
				\item Accounting/Finance: \url{http://www.seo-usa.org/Career/Accounting/Finance}
				\item Information Technology: \url{http://www.seo-usa.org/Career/Information_Technology}
				\end{itemize}
			\item Corporate Law: \url{http://www.seo-usa.org/Career/Corporate_Law}
			\item Nonprofit: \url{http://www.seo-usa.org/Career/Nonprofit}
			\item SEO-U: Freshmen and Sophomore Training: \url{http://www.seo-usa.org/Career/SEO-U:Freshmen_&_Sophomore_Training}
			\end{enumerate}
		\item Application Deadlines: \url{http://www.seo-usa.org/CareerApplicationDeadlines}
		\item Eligibility Information: \url{http://www.seo-usa.org/CareerEligibilityInfo}
		\item Application Tips: \url{http://www.seo-usa.org/CareerApplicationTips}
		\item Interview Tips: \url{http://www.seo-usa.org/CareerInterviewTips}
		\end{enumerate}
	\end{enumerate}
\item --- --- --- --- --- --- --- --- --- --- --- --- --- --- --- --- --- --- --- --- --- --- --- --- --- --- --- --- --- --- ---
\item \colorbox{blue}{\bf Scholarships for Studying Abroad}
% Scholarships for Studying Abroad
\item U.S. Department of State: \vspace{-0.3cm}
	\begin{enumerate} \itemsep -2pt
	\item Bureau of Educational and Cultural Affairs: \vspace{-0.2cm}
		\begin{enumerate} \itemsep -2pt
		\item Benjamin A. Gilman International Scholarship: \vspace{-0.1cm}
			\begin{enumerate} \itemsep -1pt
			\item \url{http://exchanges.state.gov/globalexchanges/gilman-scholarship-program.html}
			\item ``The Benjamin A. Gilman International Scholarship Program provides scholarships to U.S. undergraduates with financial need for study abroad, including students from diverse backgrounds and students going to non-traditional study abroad destinations.  Established under the International Academic Opportunity Act of 2000, Gilman Scholarships provide up to \$5,000 for American students to pursue overseas study for college credit.''
			\item Critical Need Languages: Students studying critical need languages are eligible for up to \$3,000 in additional funding as part of the Gilman Critical Need Language Supplement program. Those critical need languages include: \vspace{-0.1cm}
				\begin{itemize} \itemsep -1pt
				\item Arabic
				\item Chinese
				\item Korean
				\item Russian
				\item Turkic (Azerbaijani, Kazakh, Kyrgyz, Turkish, Turkmen, Uzbek)
				\item Persian (Farsi, Dari, Kurdish, Pashto, Tajiki)
				\item Indic (Hindi, Urdu, Nepali, Sinhala, Bengali, Punjabi, Marathi, Gujurati, Sindhi)
				\end{itemize}
			\item \url{http://www.iie.org/en/Programs/Gilman-Scholarship-Program}
			\item \url{http://www.iie.org/en/Programs/Gilman-Scholarship-Program/About-the-Program}
			\end{enumerate}
		\end{enumerate}
	\end{enumerate}
\item Council on International Educational Exchange (CIEE): \vspace{-0.3cm}
	\begin{enumerate} \itemsep -2pt
	\item CIEE Scholarships: \url{http://www.ciee.org/study/scholarships/index.aspx}
	\end{enumerate}
\item IES Abroad (formerly Institute of European Studies / Institute for the International Education of Students): \vspace{-0.3cm}
	\begin{enumerate} \itemsep -2pt
	\item Scholarships and Financial Aid: \url{https://www.iesabroad.org/IES/Scholarships_and_Aid/financialAid.html}
	\item IES Abroad Need-Based Financial Aid: \url{https://www.iesabroad.org/IES/Scholarships_and_Aid/Need-Based/needBasedFinancialAid.html}
	\item IES Abroad Merit-Based Scholarships: \url{https://www.iesabroad.org/IES/Scholarships_and_Aid/Merit_Based/meritBasedFinancialAid.html}
	\item IES Abroad Public University Grants: \url{https://www.iesabroad.org/IES/Scholarships_and_Aid/publicScholarship.html}
	\end{enumerate}
\item American Institute For Foreign Study (AIFS): \vspace{-0.3cm}
	\begin{enumerate} \itemsep -2pt
	\item AIFS Study Abroad Programs: \vspace{-0.2cm}
		\begin{enumerate} \itemsep -2pt
		\item \url{http://www.aifsabroad.com/programs.asp}
		\item AIFS Study Abroad Scholarships: \url{http://www.aifsabroad.com/scholarships.asp}
		\end{enumerate}
	\end{enumerate}
\item --- --- --- --- --- --- --- --- --- --- --- --- --- --- --- --- --- --- --- --- --- --- --- --- --- --- --- --- --- --- ---
\item \colorbox{blue}{\bf Scholarships and Fellowships in Public Policy and Public Health}
% Scholarships and Fellowships in Public Policy and Public Health
\item The Commonwealth Fund: \vspace{-0.3cm}
	\begin{enumerate} \itemsep -2pt
	\item Commonwealth Fund fellowship programs: \vspace{-0.2cm}
		\begin{enumerate} \itemsep -2pt
		\item \url{http://www.commonwealthfund.org/Fellowships.aspx}
		\item ``Commonwealth Fund fellowship programs are designed to give promising young researchers the opportunity for in-depth study of various health care policy topics, working with investigators, policy analysts, government officials, and others in a number of U.S. and international settings.''
		\item The Commonwealth Fund/Harvard University Fellowship in Minority Health Policy: \url{http://www.commonwealthfund.org/Fellowships/Minority-Health-Policy-Fellowship.aspx}
		\item Harkness Fellowships in Health Care Policy and Practice: \url{http://www.commonwealthfund.org/Fellowships/Harkness-Fellowships.aspx}
		\item Australian-American Health Policy Fellowship: \url{http://www.commonwealthfund.org/Fellowships/Australian-American-Health-Policy-Fellowships.aspx}
		\item Ian Axford (New Zealand) Fellowships in Public Policy: \url{http://www.commonwealthfund.org/Fellowships/Ian-Axford-Fellowships.aspx}
		\end{enumerate}
	\end{enumerate}
\item American Institute of Aeronautics and Astronautics (AIAA): \vspace{-0.3cm}
	\begin{enumerate} \itemsep -2pt
	\item Federal Government Fellows Program: \vspace{-0.2cm}
		\begin{enumerate} \itemsep -2pt
		\item \url{http://www.aiaa.org/content.cfm?pageid=731}
		\item Shaping U.S. {\bf public policy} concerning aerospace research and the aerospace industry
		\end{enumerate}
	\end{enumerate}
\item IEEE-USA: \vspace{-0.3cm}
	\begin{enumerate} \itemsep -2pt
	\item Congressional Fellowship
	\item Engineering \& Diplomacy (State Department) Fellowship
	\item For IEEE-USA members to support the creation and modification of technology-related public policies
	\item \url{http://ieeeusa.org/policy/govfel/default.asp}
	\end{enumerate}
\item American Mathematical Society: \vspace{-0.3cm}
	\begin{enumerate} \itemsep -2pt
	\item Fellowships and Awards (Policy and Advocacy: Government Relations \& Programs): \vspace{-0.2cm}
		\begin{enumerate} \itemsep -2pt
		\item \url{http://e-math.ams.org/policy/government/fellow-awards/fellow-awards}
		\item Mass Media Fellowships: \url{http://e-math.ams.org/programs/ams-fellowships/media-fellow/massmediafellow}
		\item AMS-AAAS Congressional Fellowship: \url{http://e-math.ams.org/programs/ams-fellowships/ams-aaas/ams-aaas-congressional-fellowship}
		\end{enumerate}
	\end{enumerate}
\item American Association for the Advancement of Science: \vspace{-0.3cm}
	\begin{enumerate} \itemsep -2pt
	\item AAAS Science \& Technology Policy Fellowships: \url{http://fellowships.aaas.org/index.shtml}
	\end{enumerate}
\item --- --- --- --- --- --- --- --- --- --- --- --- --- --- --- --- --- --- --- --- --- --- --- --- --- --- --- --- --- --- ---
\item \colorbox{blue}{\bf Scholarships and Fellowships in Social Science and Humanities}
% Scholarships and Fellowships in Social Science and Humanities
\item United States Institute of Peace (USIP): \vspace{-0.3cm}
	\begin{enumerate} \itemsep -2pt
	\item Jennings Randolph Peace Scholarship Dissertation Program (for Ph.D. students working on topics related to peace, conflict, and international security): \url{http://www.usip.org/grants-fellowships/jennings-randolph-peace-scholarship-dissertation-program}
	\end{enumerate}
\item Library of Congress: \vspace{-0.3cm}
	\begin{enumerate} \itemsep -2pt
	\item Kluge Fellowships: \vspace{-0.2cm}
		\begin{enumerate} \itemsep -2pt
		\item Research in the humanities and social sciences, especially interdisciplinary, cross-cultural or multilingual
		\item Open to scholars worldwide with a Ph.D. or other terminal advanced degree conferred within seven years of the July 15 deadline
		\item \url{http://www.loc.gov/loc/kluge/fellowships/kluge.html}
		\end{enumerate}
	\item J. Franklin Jameson Fellowship Research in American History (junior postdocs): \url{http://www.loc.gov/loc/kluge/fellowships/jameson.html}
	\item Kislak Short Term Fellowship Opportunities in American Studies (students, postdocs, and faculty): \url{http://www.loc.gov/loc/kluge/fellowships/kislakshort.html}
	\item Kislak Fellowship in American Studies (Ph.D. requirement): \url{http://www.loc.gov/loc/kluge/fellowships/kislak.html}
	\end{enumerate}
\item American Historical Association (AHA): \vspace{-0.3cm}
	\begin{enumerate} \itemsep -2pt
	\item AHA Research Grants: \url{http://www.historians.org/prizes/Grants.htm}
	\item Fellowships: \url{http://www.historians.org/prizes/Fellowships.htm}
	\end{enumerate}
\item American Sociological Association: \vspace{-0.3cm}
	\begin{enumerate} \itemsep -2pt
	\item ASA Dissertation Award: \url{http://www.asanet.org/about/awards/dissertation.cfm}
	\end{enumerate}
\item American Psychological Association: \vspace{-0.3cm}
	\begin{enumerate} \itemsep -2pt
	\item Scholarships, Grants, and Awards: \url{http://www.apa.org/about/awards/index.aspx}
	\end{enumerate}
\item American Anthropological Association (AAA): \vspace{-0.3cm}
	\begin{enumerate} \itemsep -2pt
	\item AAA Minority Dissertation Fellowship Program (for minority Ph.D. candidates in anthropology): \url{http://www.aaanet.org/cmtes/minority/Minfellow.cfm}
	\item Margaret Mead Award (for young scholars in anthropology): \url{http://www.aaanet.org/about/Prizes-Awards/AAA-Margaret-Mead-Award.cfm}
	\item COSWA Award: \vspace{-0.2cm}
		\begin{enumerate} \itemsep -2pt
		\item The COSWA Award (formerly the Squeaky Wheel Award), sponsored by the Committee on the Status of Women in Anthropology (COSWA), recognizes individuals who have demonstrated the courage to bring to light and investigate practices in anthropology that are potentially discriminatory to women, or have acted to improve the status of women in anthropology through activities that raise awareness of women's contribution to anthropology or identify barriers to full participation by women in anthropology.
		\item \url{http://www.aaanet.org/about/Prizes-Awards/COSWA-Award.cfm}
		\end{enumerate}
	\item David M. Schneider Award (for Ph.D. students in anthropology): \url{http://www.aaanet.org/about/Prizes-Awards/David-Schneider-Award.cfm}
	\item Links to ``Section Prizes \& Awards'': \url{http://www.aaanet.org/about/Prizes-Awards/section_awards.cfm}
	\item List of national (US) and international ``Grants and Fellowships'': \url{http://www.aaanet.org/profdev/fellowships/}
	\item \url{http://www.aaanet.org/}
	\end{enumerate}
\item National Academy of Social Insurance: \vspace{-0.3cm}
	\begin{enumerate} \itemsep -2pt
	\item John Heinz Dissertation Award (Ph.D. students writing their thesis on the planning and implementation of social insurance): \url{http://www.nasi.org/studentopps/heinz}
	\end{enumerate}
\item National Endowment for the Humanities's Division of Research Programs, grants and fellowship opportunities: \url{http://www.neh.gov/grants/}
\item {\it The Henry Luce Foundation}'s Luce Scholars Program to help US graduates learn more about Asia and Asian culture(s): \url{http://www.hluce.org/lsprogram.aspx}
\item Institute for Humane Studies at George Mason University: \vspace{-0.3cm}
	\begin{enumerate} \itemsep -2pt
	\item Humane Studies Fellowships: \vspace{-0.2cm}
		\begin{enumerate} \itemsep -2pt
		\item \url{http://www.theihs.org/programs/humane-studies-fellowships}
		\item Humane Studies Fellowships are awarded to graduate students and outstanding undergraduates planning academic careers with liberty-advancing research interests.
		\item The fellowships are open to students in a range of fields, such as economics, philosophy, law, political science, anthropology, and literature.
		\end{enumerate}
	\end{enumerate}
\item The Gilder Lehrman Institute of American History: Gilder Lehrman History Scholars \& Gilder Lehrman One-Week Scholars (for sophomores or juniors majoring in American history or American Studies), \url{http://www.gilderlehrman.org/education/hs_program_details.php}
\item Myra Sadker Foundation: \vspace{-0.3cm}
	\begin{enumerate} \itemsep -2pt
	\item \url{http://www.sadker.org/awards.html}
	\item Teacher Award: Designed to promote and support teacher projects (K-12) that help students learn about and respect group differences, promote fairness, and in other ways build upon the values and contributions of Myra Sadker's work. Each project should have a gender dimension.
	\item Student Award: Designed to encourage student ideas, activities and projects (K-12) that promote respect for group differences, fairness, and in other ways build upon the values and contributions of Myra Sadker's work. Each project should have a gender dimension. 
	\item Doctoral Dissertation Award: Designed to promote and support graduate students engaged in educational equity research. Doctoral level dissertations that explore or promote educational equity and fairness based on gender, race, ethnicity, religion, class, sexual orientation, or other such variables will be considered for support. Each dissertation should have a gender dimension.
	\end{enumerate}
\item IREX: \vspace{-0.3cm}
	\begin{enumerate} \itemsep -2pt
	\item Opportunities ``for individuals, organizations, universities, and alumni'': \url{http://www.irex.org/apply}
	\item Edmund S. Muskie Graduate Fellowship Program: \vspace{-0.2cm}
		\begin{enumerate} \itemsep -2pt
		\item : \url{http://www.irex.org/application/edmund-s-muskie-graduate-fellowship-program-application}
		\item ``The Muskie Program is open to graduate students and professionals from Armenia, Azerbaijan, Belarus, Georgia, Kazakhstan, Kyrgyzstan, Moldova, Russia, Tajikistan, Turkmenistan, Ukraine and Uzbekistan for one-year non-degree, one-year degree, or two-year degree study in the United States.''
		\item ``Eligible fields of study for the Muskie Program are: business administration, economics, education, environmental management, international affairs, journalism and mass communication, law, library and information science, public administration, public health, and {\bf public policy}.''
		\end{enumerate}
	\item Legal Education and Development (LEAD) Fellowship: \vspace{-0.2cm}
		\begin{enumerate} \itemsep -2pt
		\item \url{http://www.irex.org/application/legal-education-and-development-lead-fellowship-application}
		\item Legal Education and Development Fellowship Program (LEAD) in Tajikistan
		\item Eligibility: \vspace{-0.1cm}
			\begin{enumerate} \itemsep -1pt
			\item Is a citizen, national, or permanent resident qualified to hold a valid passport issued by Tajikistan;
			\item Is the recipient of an undergraduate degree in law (four- or five-year study) by the time of the application;
			\item Is able to begin the academic exchange program in the United States in the summer of 2011;
			\item Is able to receive and maintain a United States J-1 visa.
			\end{enumerate}
		\end{enumerate}
	\item Community Solutions Program: \vspace{-0.2cm}
		\begin{enumerate} \itemsep -2pt
		\item \url{http://www.irex.org/application/community-solutions-information-applicants}
		\item ``a professional development program for the best and brightest global community leaders working in Transparency \& Accountability, Tolerance/Conflict Resolution, Environmental Issues, and Women's Issues''
		\item ``Competition for the Community Solutions Program is merit-based and open to community leaders, ages 25-38 at the time of application''
		\end{enumerate}
	\item Crimea Undergraduate Exchange Program (Crimea UGRAD) Application: \vspace{-0.2cm}
		\begin{enumerate} \itemsep -2pt
		\item \url{http://www.irex.org/application/crimea-undergraduate-exchange-program-crimea-ugrad-application}
		\item ``The Crimea UGRAD Program is open to undergraduate students from the Autonomous Republic of Crimea for one academic year of non-degree study in a US university or community college.''
		\end{enumerate}
	\end{enumerate}
\item {\it Demos}: \vspace{-0.3cm}
	\begin{enumerate} \itemsep -2pt
	\item The Ed Baker Fellowship in Democratic Values: \vspace{-0.2cm}
		\begin{enumerate} \itemsep -2pt
		\item \url{http://www.demos.org/edbakerfellowship.cfm}
		\item ``Based in our New York offices, Ed Baker Fellows will give voice to strong democratic values within a wide range of potential issues, including voting rights, citizen engagement, immigration policy and civic inclusion, campaign finance reform and money in politics, and media reform, among others.''
		\end{enumerate}
	\item Fellows Program: \vspace{-0.2cm}
		\begin{enumerate} \itemsep -2pt
		\item \url{http://www.demos.org/fellowsapp.cfm}
		\item \url{http://www.demos.org/program.cfm?currentprogramID=5A196E48-3FF4-6C82-50CBCA5825B661BA}
		\item ``The Fellows Program at Demos provides support and community for writers and thinkers with well-defined projects that aim to advance the values at the core of Demos' programs and mission: a robust and inclusive democracy; shared prosperity; strong \& effective public governance; and global interdependence.''
		\end{enumerate}
	\end{enumerate}
\item Research Councils UK (RCUK): \vspace{-0.3cm}
	\begin{enumerate} \itemsep -2pt
	\item Economic and Social Research Council (ESRC): \vspace{-0.2cm}
		\begin{enumerate} \itemsep -2pt
		\item Academic (funding opportunities for students, postdocs, and professors): \url{http://www.esrcsocietytoday.ac.uk/ESRCInfoCentre/index_academic.aspx}
		\item Professorial Fellowships (for leading senior social scientists): \url{http://www.esrcsocietytoday.ac.uk/ESRCInfoCentre/opportunities/professorial/}
		\item Funding opportunities: \vspace{-0.1cm}
			\begin{enumerate} \itemsep -1pt
			\item \url{http://www.esrcsocietytoday.ac.uk/ESRCInfoCentre/index_government.aspx}
			\item \url{http://www.esrcsocietytoday.ac.uk/ESRCInfoCentre/opportunities/}
			\item ESRC Research Funding Guide / ESRC's Funding Rules: \url{http://www.esrcsocietytoday.ac.uk/ESRCInfoCentre/opportunities/research_funding}
			\item Eligibility for Research Council Funding: \url{http://www.esrcsocietytoday.ac.uk/ESRCInfoCentre/opportunities/eligibility}
			\item Current Funding Opportunities: \url{http://www.esrcsocietytoday.ac.uk/ESRCInfoCentre/opportunities/current_funding_opportunities/}
			\item Forthcoming funding opportunities: \url{http://www.esrcsocietytoday.ac.uk/ESRCInfoCentre/opportunities/forthcoming_opportunities/}
			\item Placement Fellows Scheme: \url{http://www.esrcsocietytoday.ac.uk/ESRCInfoCentre/opportunities/placement/}
			\item Professorial Fellowships: \url{http://www.esrcsocietytoday.ac.uk/ESRCInfoCentre/opportunities/professorial/}
			\item Early Career Researchers (including Postdoctoral Fellowships, International Training, and Networking Opportunities): \url{http://www.esrcsocietytoday.ac.uk/ESRCInfoCentre/opportunities/earlycareer/}
			\item Postgraduate and Career Development Opportunities: \url{http://www.esrcsocietytoday.ac.uk/ESRCInfoCentre/opportunities/postgraduate/}
			\item International Funding Opportunities: \url{http://www.esrcsocietytoday.ac.uk/ESRCInfoCentre/opportunities/international/}
			\item Joint Funding Opportunities: \url{http://www.esrcsocietytoday.ac.uk/ESRCInfoCentre/opportunities/jointfunding/}
			\item Annual competitions: \url{http://www.esrcsocietytoday.ac.uk/ESRCInfoCentre/opportunities/annual/index.aspx#3}
			\end{enumerate}
		\end{enumerate}
	\item Arts and Humanities Research Council (AHRC): \vspace{-0.2cm}
		\begin{enumerate} \itemsep -2pt
		\item Funding Opportunities: \vspace{-0.1cm}
			\begin{enumerate} \itemsep -1pt
			\item \url{http://www.ahrc.ac.uk/FundingOpportunities/Pages/default.aspx}
			\item Fellowships: \url{http://www.ahrc.ac.uk/FundingOpportunities/Pages/Fellowships.aspx}
			\item Fellowships - route for early career researchers: \url{http://www.ahrc.ac.uk/FundingOpportunities/Pages/Fellowshipserc.aspx}
			\item Placement Fellowship based in the Department for Culture, Media and Sport (DCMS) - Climate Change: \url{http://www.ahrc.ac.uk/FundingOpportunities/Pages/PlacementFellowshipDCMS-Climatechange.aspx}
			\item Placement Fellowship based in the Department for Culture, Media and Sport (DCMS) - Health and Wellbeing: \url{http://www.ahrc.ac.uk/FundingOpportunities/Pages/PlacementFellowshipDCMShealthandwellbeing.aspx}
			\item Research Grants - route for early career researchers: \url{http://www.ahrc.ac.uk/FundingOpportunities/Pages/RG-EarlyCareers.aspx}
			\item Research Grants - Speculative Research: \url{http://www.ahrc.ac.uk/FundingOpportunities/Pages/RG-SpeculativeResearch.aspx}
			\item Research Grants - Standard Route: \url{http://www.ahrc.ac.uk/FundingOpportunities/Pages/RG-StandardRoute.aspx}
			\item Postgraduate Funding (for Masters and Ph.D. students): \url{http://www.ahrc.ac.uk/FundingOpportunities/Pages/summaryinformationforprospectivepostgraduatestudents.aspx}
			\item Browse Funding Opportunities: \url{http://www.ahrc.ac.uk/FundingOpportunities/Pages/BrowseOpportunities.aspx}
			\end{enumerate}
		\end{enumerate}
	\end{enumerate}
\item World Bank Institute (WBI): \vspace{-0.3cm}
	\begin{enumerate} \itemsep -2pt
	\item Or The World Bank Group
	\item Scholarships: \url{http://wbi.worldbank.org/wbi/scholarships} or \url{http://www.worldbank.org/wbi/scholarships/home.html}
	\end{enumerate}
\item --- --- --- --- --- --- --- --- --- --- --- --- --- --- --- --- --- --- --- --- --- --- --- --- --- --- --- --- --- --- ---
\item \colorbox{blue}{\bf Fellowships in Art and Music}
% Fellowships in Art and Music
\item The Kresge Foundation: \vspace{-0.3cm}
	\begin{enumerate} \itemsep -2pt
	\item \url{http://www.kresge.org/index.php/what/detroit_program/kresge_arts_in_detroit/}
	\item Kresge Artist Fellowships: \vspace{-0.2cm}
		\begin{enumerate} \itemsep -2pt
		\item ``Kresge Artist Fellowships seek to advance the art forms and professional careers of artists from the visual, performing and literary arts as well as elevate the profile of the artistic community and encourage creative expression in the region. Each year, Kresge will provide funding for 18 fellowships of \$25,000 each, which are awarded to artists living and working in metropolitan Detroit.''
		\item ``The fellowships recognize creative vision and commitment to excellence within a wide range of artistic disciplines, including artists who have been classically and academically trained, self taught artists and artists whose art forms have been passed down through cultural and traditional heritage.''
		\item ``Kresge Arts in Detroit is committed to supporting artists from diverse cultural backgrounds at all stages of their professional careers.''
		\item \url{http://kresge.collegeforcreativestudies.edu/}
		\item \url{http://kresge.collegeforcreativestudies.edu/kaf_guidelines.html}
		\item Information Sessions: \url{http://kresge.collegeforcreativestudies.edu/kaf_sessions.html}
		\end{enumerate}
	\item Kresge Eminent Artist Award: \vspace{-0.2cm}
		\begin{enumerate} \itemsep -2pt
		\item ``Kresge Eminent Artist Award recognizes an exceptional artist for his or her professional achievements and contributions to the cultural community, and encourages that individual's pursuit of a chosen art form as well as an ongoing commitment to metropolitan Detroit. Each year, one highly accomplished individual will be presented with the award which includes a \$50,000 prize.''
		\item \url{http://kresge.collegeforcreativestudies.edu/eminent-artist-award.html}
		\end{enumerate}
	\end{enumerate}
\item Guggenheim Fellowships from the {\it John Simon Guggenheim Memorial Foundation}: \url{http://www.gf.org/applicants}
\item The John F. Kennedy Center for the Performing Arts: \vspace{-0.3cm}
	\begin{enumerate} \itemsep -2pt
	\item DeVos Institute of Arts Management at the Kennedy Center: \vspace{-0.2cm}
		\begin{enumerate} \itemsep -2pt
		\item DeVos Institute Programs: \vspace{-0.1cm}
			\begin{enumerate} \itemsep -1pt
			\item Kennedy Center Fellowship Program: \vspace{-0.1cm}
				\begin{itemize} \itemsep -1pt
				\item \url{http://www.kennedy-center.org/education/artsmanagement/fellowships.cfm}
				\item \url{http://www.kennedy-center.org/education/artsmanagement/fellowships/home.html}
				\item ``The Kennedy Center Fellowship Program began in 2001, and provides comprehensive study to 10 arts managers at the Kennedy Center with coursework in strategic planning, marketing, and development; three practical work rotations in Center departments; and a series of professional development seminars. The paid fellowships are full-time and last nine months from September through May.''
				\end{itemize}
			\item DeVos Institute Summer International Fellowship Program at the Kennedy Center: \vspace{-0.1cm}
				\begin{itemize} \itemsep -1pt
				\item \url{http://www.kennedy-center.org/education/artsmanagement/fellowships.cfm}
				\item \url{http://www.kennedy-center.org/education/artsmanagement/international_faq.cfm}
				\item ``The Summer International Fellowship Program provides practical experience to 15 mid-to-high level arts leaders currently working in international nonprofit performing arts organizations. This full-time, four-week intensive program takes place at the Kennedy Center each July; Fellows attend each summer for three consecutive years. While at the Center, the fellows take classes and refine strategic plans for their home organizations.''
				\end{itemize}
			\item U.S. Department of State International Exchange Programs: \vspace{-0.1cm}
				\begin{itemize} \itemsep -1pt
				\item \url{http://www.kennedy-center.org/education/state/}
				\item ``The U.S. Department of State and The Kennedy Center have teamed to produce international exchange opportunities through the Performing Artists Cultural Visitors Program and International Cultural Fellows Mentoring Program.''
				\item Performing Artists Cultural Visitors Program: \url{http://www.kennedy-center.org/education/state/cultural/}
				\item International Cultural Fellows Mentoring Program: \url{http://www.kennedy-center.org/education/state/fellows/}
				\item ``Visitors, comprised of modern and hip-hop dancers, theater technicians/designers/actors, as well as classical and jazz musicians, engage with American colleagues in the creation and performance of their discipline in Washington, D.C. and in another American city.''
				\item ``The Fellows, comprised of arts managers and presenters from outside the United States, attend arts management seminars led by Kennedy Center staff, travel to another American city to study with a mentor organization, and visit New York City to meet with experts in their field.''
				\end{itemize}
			\end{enumerate}
		\end{enumerate}
	\item The National Symphony Orchestra (NSO): \vspace{-0.2cm}
		\begin{enumerate} \itemsep -2pt
		\item National Symphony Orchestra Youth Fellowship Program: \vspace{-0.1cm}
			\begin{itemize} \itemsep -1pt
			\item \url{http://www.kennedy-center.org/nso/nsoed/youthfellowship.cfm}
			\item \url{http://www.kennedy-center.org/explorer/artists/?entity_id=10811&source_type=B}
			\item ``Now in its 30th season, the National Symphony Orchestra Youth Fellowship Program is an orchestral training project for high school musicians.''
			\item ``From its inception in 1980-81 to the present, the program provides Washington metropolitan area high school students with scholarships to study privately with NSO members, as well as opportunities to observe NSO rehearsals; attend concerts; and to participate in seminars, discussions, and master classes with musicians, conductors, and NSO and Kennedy Center management.''
			\item ``There are 20 students in the NSO Youth Fellowship Program for 2009-10.''
			\item ``Participation by ethnic minorities is encouraged.''
			\item ``Priority is given to students entering 10th grade in order to provide as sustained a training as possible.''
			\end{itemize}
		\end{enumerate}
	\end{enumerate}
\item League of American Orchestras: \vspace{-0.3cm}
	\begin{enumerate} \itemsep -2pt
	\item Fellowships: \vspace{-0.2cm}
		\begin{enumerate} \itemsep -2pt
		\item \url{http://www.americanorchestras.org/learning_and_leadership/fellowships.html}
		\item Orchestra Management Fellowship Program: \vspace{-0.1cm}
			\begin{enumerate} \itemsep -1pt
			\item \url{http://www.americanorchestras.org/learning_and_leadership/omfp.html}
			\item ``This year-long, highly competitive program is designed to launch executive careers in orchestra management.''
			\item ``Along with an intense course of study, fellows undertake a series of residencies with orchestras of various sizes across the U.S. receiving invaluable work experience and the support of host orchestra staff, in particular that of the orchestra�s executive director.''
			\item ``Fellows also participate in other League leadership seminars throughout the year and receive a comprehensive overview of the classical music industry.''
			\end{enumerate}
		\item ``The League's Fellowship programs identify and prepare the future leaders of tomorrow, today.''
		\item ``Long-term curricula, developed for conductors, executive directors, and managers looking to advance, provide intensive education, hands-on learning, and valuable networking opportunities.''
		\end{enumerate}
	\end{enumerate}
\item Americans for the Arts: \vspace{-0.3cm}
	\begin{enumerate} \itemsep -2pt
	\item Event scholarships (scholarships to attend events): \url{http://www.artsusa.org/events/scholarships.asp}
	\item \url{http://www.artsusa.org/news/annual_awards/default.asp}
	\item Alene Valkanas State Arts Advocacy Award\url{http://www.artsusa.org/news/annual_awards/alene_valkanas/default.asp}
	\item Arts Education Award (awarded to institutions): \url{http://www.artsusa.org/news/annual_awards/arts_education/default.asp}
	\item Emerging Leader Award: \url{http://www.artsusa.org/news/annual_awards/emerging_leader/default.asp}
	\item Michael Newton Award for United Arts Funds Leadership (management and fundraising): \url{http://www.artsusa.org/news/annual_awards/michael_newton/default.asp}
	\item Selina Roberts Ottum Award (contributions to the field of the arts): \url{http://www.artsusa.org/news/annual_awards/selina_roberts_ottum/default.asp}
	\item United States Urban Arts Federation (USUAF): \vspace{-0.2cm}
		\begin{enumerate} \itemsep -2pt
		\item Ray Hanley Innovation Award: \url{http://www.artsusa.org/networks/usuaf/hanley.asp}
		\end{enumerate}
	\end{enumerate}
\item NEA National Heritage Fellowship (for master folk and traditional artists): \url{http://www.nea.gov/honors/heritage/index.html}
\item NEA Jazz Masters Fellowship (jazz artists): \url{http://www.arts.gov/honors/jazz/index.html}
\item Fellowships for Creative Writers [or NEA Literature Fellowships: Creative Writing]: \url{http://www.nea.gov/grants/apply/Lit/index.html} or \url{http://www.arts.gov/grants/apply/Lit/index.html}
\item Carnegie Investment Bank: Carnegie Art Award (for distinguished artists born or living in the Nordic countries), \url{http://www.carnegie.se/sv/ArtAward/About-Carnegie-Art-Award/}, \url{http://www.carnegie.se/artaward/}, and \url{http://www.carnegie.se/en/about/Operations/Carnegie-Art-Award/}
\item Robert McCann Foundation: \vspace{-0.3cm}
	\begin{enumerate} \itemsep -2pt
	\item Funding for artists and designers ``from all Scottish colleges and art schools'' to: \vspace{-0.2cm}
		\begin{enumerate} \itemsep -2pt
		\item extend their training in an area of specialization; OR
		\item finance a project ``in the craft industries associated with film and television''
		\end{enumerate}
	\item \url{http://robertmccannfoundation.com/how.html}
	\end{enumerate}
\item Alexander von Humboldt-Stiftung/Foundation: \vspace{-0.3cm}
	\begin{enumerate} \itemsep -2pt
	\item Hezekiah Wardwell Fellowship (for musicians or musicologists from Spain): \url{http://www.humboldt-foundation.de/web/wardwell-en.html}
	\end{enumerate}
\item Canada Council for the Arts: \vspace{-0.3cm}
	\begin{enumerate} \itemsep -2pt
	\item Endowments and Prizes: \vspace{-0.2cm}
		\begin{enumerate} \itemsep -2pt
		\item \url{http://www.canadacouncil.ca/prizes/}
		\item Prizes and fellowships for Canadian artists and scholars to recognize their contributions to the arts, humanities, and sciences
		\item Categories of prizes and fellowships: \vspace{-0.1cm}
			\begin{enumerate} \itemsep -1pt
			\item dance
			\item inter-arts
			\item media arts
			\item music
			\item theatre
			\item visual arts
			\item writing and publishing
			\end{enumerate}
		\end{enumerate}
	\item Grant Programs: \url{http://www.canadacouncil.ca/grants/}
	\end{enumerate}
\item Institute for Humane Studies at George Mason University: \vspace{-0.3cm}
	\begin{enumerate} \itemsep -2pt
	\item Film \& Fiction Scholarships: \vspace{-0.2cm}
		\begin{enumerate} \itemsep -2pt
		\item Students pursuing MFAs in a variety of areas are eligible: film directing, production, screenwriting, playwriting, fiction, and literary-nonfiction writing
		\item \url{http://www.theihs.org/node/448}
		\end{enumerate}
	\end{enumerate}
\item --- --- --- --- --- --- --- --- --- --- --- --- --- --- --- --- --- --- --- --- --- --- --- --- --- --- --- --- --- --- ---
\item \colorbox{blue}{\bf Scholarships and Fellowships for Underrepresented Minorities}
% Scholarships and Fellowships for Underrepresented Minorities
\item Lists of scholarships and fellowships for underrepresented minorities: \vspace{-0.3cm}
	\begin{enumerate} \itemsep -2pt
	\item Chris Enstrom, ``Cashing in on Diversity Grants and Scholarships,'' in Graduating Engineer \& Computer Careers. Available at: \url{http://www.graduatingengineer.com/higher-education/20061129/Cashing-in-on-Diversity-Grants-and-Scholarships-}; last accessed on August 25, 2010.
	\end{enumerate}
\item Gates Millennium Scholars (GMS) scholarship (for underrepresented minorities in the US): \url{http://www.gmsp.org/}
\item Society of Women Engineers (SWE): SWE Scholarships and other scholarships, \url{http://societyofwomenengineers.swe.org/index.php?option=com_content&task=view&id=222&Itemid=111}
\item Coalition to Diversify Computing: \url{http://www.cdc-computing.org/scholarships/}
\item IES Abroad (formerly Institute of European Studies / Institute for the International Education of Students): \vspace{-0.3cm}
	\begin{enumerate} \itemsep -2pt
	\item Diversity Abroad: \vspace{-0.2cm}
		\begin{enumerate} \itemsep -2pt
		\item \url{https://www.iesabroad.org/IES/Diversity/diversity.html}
		\item Programs to improve student diversity in study abroad programs
		\item IES Abroad Diversity Scholarships: \vspace{-0.1cm}
			\begin{enumerate} \itemsep -1pt
			\item IES Abroad Merit-Based Scholarship for Under-represented Students: \url{https://www.iesabroad.org/IES/Scholarships_and_Aid/Diversity_Scholarships/diversityScholarship.html}
			\item IES Abroad Merit-Based David Porter Diversity Scholarship (Up to \$5,000!): \url{https://www.iesabroad.org/IES/Scholarships_and_Aid/Merit_Based/davidPorterScholarship.html}
			\item HBCU Scholarships: \url{https://www.iesabroad.org/IES/Scholarships_and_Aid/Diversity_Scholarships/hbcuScholarship.html}
			\item HACU-IES Abroad Merit/Need-Based Scholarship: \url{https://www.iesabroad.org/IES/Scholarships_and_Aid/Diversity_Scholarships/HACUScholarship.html}
			\end{enumerate}
		\end{enumerate}
	\end{enumerate}
\item MassMutual Scholars Program: \vspace{-0.3cm}
	\begin{enumerate} \itemsep -2pt
	\item Applicants must be undergraduates of African American/Black, Asian/Pacific Islander or Hispanic decent in the US.
	\item Reside or plan to attend an institution in one of the following metropolitan areas: \vspace{-0.2cm}
		\begin{enumerate} \itemsep -2pt
		\item Atlanta, GA
		\item Chicago, IL
		\item Central New Jersey
		\item Denver, CO
		\item Houston, TX
		\item Miami, FL
		\item Los Angeles, CA
		\item San Antonio, TX
		\item San Francisco, CA
		\end{enumerate}
	\item Be majoring in business, economics, finance, financial planning, management, marketing or sales.
	\item \url{http://www.hsf.net/massmutual.aspx}
	\item \url{http://www.apiasf.org/scholarship_apiasf_massmutual.html}
	\end{enumerate}
\item {\it NASA}'s Minority University Research and Education Program (MUREP): \vspace{-0.3cm}
	\begin{enumerate} \itemsep -2pt
	\item \url{http://www.nasa.gov/offices/education/programs/national/murep/home/index.html}
	\item \url{http://www.nasa.gov/offices/education/about/murep_overview.html}
	\item Jenkins Pre-doctoral Fellowship Project, JPFP: \url{http://www.nasa.gov/offices/education/programs/descriptions/Jenkins_Predoctoral_Fellowship_Project.html}
	\end{enumerate}
\item UNCF: \vspace{-0.3cm}
	\begin{enumerate} \itemsep -2pt
	\item UNCF Special Programs Corporation: \vspace{-0.2cm}
		\begin{enumerate} \itemsep -2pt
		\item Harriett G. Jenkins Pre-doctoral Fellowship Program (JPFP) for underrepresented minorities pursuing graduate degrees in STEM: \url{http://www.uncfsp.org/spknowledge/default.aspx?page=program.view&areaid=1&contentid=177&typeid=jpfp}
		\item NASA Science and Technology Institute (NSTI) Summer Scholars Program (10-week summer research scholarship): \url{http://www.uncfsp.org/spknowledge/default.aspx?page=program.view&areaid=1&contentid=172&typeid=nstiinternship}
		\item Motivating Undergraduates in Science and Technology (MUST) Program for undergraduates in STEM: \url{http://www.uncfsp.org/spknowledge/default.aspx?page=program.view&areaid=1&contentid=346&typeid=must}
		\item Institute for International {\bf Public Policy} Fellows Program: \url{http://www.uncfsp.org/IIPP}
		\item \url{http://www.uncfsp.org/spknowledge/default.aspx?page=home.default}
		\end{enumerate}
	\item UNCF scholarship resources: \url{http://www.uncf.org/forstudents/scholarship.asp}
	\item UNCF $\cdot$ Merck Science Initiative: scholarships and fellowships: \url{http://umsi.uncf.org/ScholarshipsInternshipsFellowships/tabid/151/Default.aspx}
	\end{enumerate}
\item Hispanic College Fund: \vspace{-0.3cm}
	\begin{enumerate} \itemsep -2pt
	\item Scholarships: \url{http://www.hispanicfund.org/scholarships/} and \url{http://scholarships.hispanicfund.org/applications/}
	\item NASA MUST Scholarship Program: \url{http://www.hispanicfund.org/nasa-must/}
	\item Hispanic Youth Symposium (scholarships are awarded to winners of the art competition, talent competition, and speech competition): \url{http://www.hispanicyouth.org/about-the-program}
	\item \url{http://www.hispanicfund.org/}
	\end{enumerate}
\item Hispanic Heritage Foundation (HHF): \vspace{-0.3cm}
	\begin{enumerate} \itemsep -2pt
	\item Scholarships and Resources: \url{http://www.hispanicheritage.org/youth_int.php?sec=80}
	\item \url{http://www.hispanicheritage.org/}
	\end{enumerate}
\item Hispanic Scholarship Fund (HSF): \vspace{-0.3cm}
	\begin{enumerate} \itemsep -2pt
	\item Scholarship programs for: \vspace{-0.2cm}
		\begin{enumerate} \itemsep -2pt
		\item college students
		\item community college transfer students
		\item high school students
		\item Gates Millennium Scholars
		\item See \url{http://www.hsf.net/innercontent.aspx?id=34}
		\end{enumerate}
	\item \url{http://www.hsf.net/}
	\end{enumerate}
\item League of United Latin American Citizens (LULAC): \vspace{-0.3cm}
	\begin{enumerate} \itemsep -2pt
	\item LULAC National Educational Service Centers, Inc: \vspace{-0.2cm}
		\begin{enumerate} \itemsep -2pt
		\item \url{http://www.lnesc.org/}
		\item LULAC National Scholarship Fund (LNSF): \vspace{-0.1cm}
			\begin{enumerate} \itemsep -1pt
			\item \url{http://www.lulac.org/programs/education/scholarships/}
			\item \url{http://lnesc.org/index.asp?Type=B_BASIC&SEC={3AEDB506-F425-4E58-B9F6-44867E2FD943}}
%http://lnesc.org/index.asp?Type=B_BASIC&SEC={3AEDB506-F425-4E58-B9F6-44867E2FD943}
			\item Applicants must meet the following criteria to be considered for a scholarship: \vspace{-0.1cm}
				\begin{itemize} \itemsep -1pt
				\item Must be a U.S. citizen or legal resident
				\item Must have applied to or be enrolled in a   college, university, or graduate school, including 2-year colleges, or vocational schools that lead to an associate�s degree
				\item A student will not be eligible for a scholarship if he/she is related to a scholarship committee member, the Council President, or an individual contributor to the local funds of the Council
				\end{itemize}
			\item National Scholastic Achievement Awards (for high school seniors entering college, university, or vocational school)
			\item Honors Awards (for high school seniors entering college, university, or vocational school)
			\item General Awards (Need, community involvement, and leadership activities will also be considered)
			\item General Electric Foundation/ LULAC Scholarship program: for underrepresented minorities (US freshmen) entering their sophomore year as majors in Business or Engineering with a cumulative college G.P.A. $\leq$ 3.25/4.0; these students must be enrolled in a 4-year undergraduate program.
			\end{enumerate}
		\end{enumerate}
	\end{enumerate}
\item Hispanic Association of Colleges and Universities (HACU): \vspace{-0.3cm}
	\begin{enumerate} \itemsep -2pt
	\item HACU Student Programs Overview: \vspace{-0.2cm}
		\begin{enumerate} \itemsep -2pt
		\item \url{http://www.hacu.net/hacu/HACU_Student_Programs_EN.asp?SnID=1942709283}
		\item HACU Scholarship Programs: \vspace{-0.1cm}
			\begin{enumerate} \itemsep -1pt
			\item \url{http://www.hacu.net/hacu/Scholarships_EN.asp?SnID=1942709283}
			\item Includes scholarships for students in: \vspace{-0.1cm}
				\begin{itemize} \itemsep -1pt
				\item Accounting
				\item Behavioral Health
				\item Business
				\item Clinical Psychology
				\item Computer Engineering
				\item Computer Science
				\item Dental Technician
				\item Electrical Engineering
				\item Engineering
				\item Food Merchandising
				\item Information Technology
				\item International Business
				\item Management
				\item Marketing
				\item Mass Media
				\item Mental Health
				\item Merchandising
				\item Nursing
				\item Physician Assistant
				\item (Pre) Optometry
				\item (Pre) Dental
				\item (Pre) Medicine
				\item (Pre) Pharmacy
				\item Public Health
				\item Public Relations
				\item Retail Management
				\item Sports Marketing
				\item Technology
				\end{itemize}
			\end{enumerate}
		\item ``D{\'{a}}ndole Alas a Tu {\'{E}}xito/Giving Flight to Your Success'' travel award program (Southwest Airlines' Travel Award Program): \vspace{-0.1cm}
			\begin{enumerate} \itemsep -1pt
			\item For students with financial need who have to across the United States to participate in their undergraduate or graduate degree programs
			\item \url{http://www.hacu.net/hacu/Lanzate_EN.asp?SnID=1942709283}
			\item \url{http://www.hacu.net/hacu/Lanzate1_EN.asp?SnID=1808826658}
			\end{enumerate}
		\item HACU Study Abroad Scholarship Programs: \vspace{-0.1cm}
			\begin{enumerate} \itemsep -1pt
			\item \url{http://www.hacu.net/hacu/Study_Abroad_EN.asp?SnID=1808826658}
			\item HACU-Global Learning Semesters (GLS) Program: Hispanic Study Abroad Scholars: \url{http://www.studyabroadscholars.org/index.html}
			\item HACU-American Institute for Foreign Study (AIFS) Scholarship Program: \url{http://www.aifsabroad.com/scholarships.asp#hacu}
			\item HACU-Institute for the International Education of Students (IES) Scholarship Program: \url{https://www.iesabroad.org/IES/home.html}
			\item Hispanic Study Abroad Scholars program: \url{http://www.studyabroadscholars.org/index.html}
			\end{enumerate}
		\item Scholarship Resource List: \url{http://www.hacu.net/hacu/Scholarship_Resource_List_EN.asp?SnID=1109551622}
%		\item Scholarship Resource List: \url{http://www.hacu.net/hacu/Scholarship_Resource_List_EN.asp?SnID=1942709283}		-- Redundant
		\end{enumerate}
	\end{enumerate}
\item Congressional Hispanic Caucus Institute (CHCI): \vspace{-0.3cm}
	\begin{enumerate} \itemsep -2pt
	\item CHCI Scholarship: \vspace{-0.2cm}
		\begin{enumerate} \itemsep -2pt
		\item \url{http://www.chci.org/scholarships/}
		\item CHCI's scholarship opportunities are afforded to Latino students in the United States who have a history of performing public service-oriented activities in their communities and who demonstrate a desire to continue their civic engagement in the future. There is no GPA or academic major requirement. Students with excellent leadership potential are encouraged to apply.
		\item Scholarship awards are intended to provide assistance with tuition, room and board, textbooks, and other educational expenses associated with college enrollment.
		\item Students continue to receive annual disbursements as long as they maintain good academic standing.
		\item CHCI scholarships provide recipients with a one time scholarship of: \vspace{-0.1cm}
			\begin{enumerate} \itemsep -1pt
			\item \$1,000 community college or AA/AS granting institution
			\item \$2,500 4-year academic institution
			\item \$5,000 graduate-level institution
			\end{enumerate}
		\item Eligibility Criteria: \vspace{-0.1cm}
			\begin{enumerate} \itemsep -1pt
			\item Full-time enrollment in a United States Department of Education accredited community college, four-year university, or graduate/professional program during the period for which scholarship is requested
			\item Demonstrated financial need
			\item Consistent, active participation in public and/or community service activities
			\item Strong writing skills
			\item U.S. citizenship or legal permanent residency
			\end{enumerate}
		\end{enumerate}
	\item CHCI Fellowships: \vspace{-0.2cm}
		\begin{enumerate} \itemsep -2pt
		\item \url{http://www.chci.org/fellowships/}
		\item CHCI {\bf Public Policy} Fellowship: \vspace{-0.1cm}
			\begin{enumerate} \itemsep -1pt
			\item This is a paid Fellowship Program that offers talented Latinos, who have earned a bachelor's degree within two years of the program start date, the opportunity to gain hands-on experience at the national level in public policy.
			\item Fellows have the opportunity to work in congressional offices and federal agencies, depending on their area of interest.  Some past focus areas have included international affairs, economic development, health and education policy, housing, or local government.
			\item Program Dates: August to May (10-month internship)
			\item \url{http://www.chci.org/fellowships/page/chci-public-policy-fellowship}
			\end{enumerate}
		\item CHCI Graduate Fellowship Program: \vspace{-0.1cm}
			\begin{enumerate} \itemsep -1pt
			\item The CHCI Graduate Fellowship Program seeks to enhance participants' leadership abilities, strengthen professional skills and ultimately produce more competent and competitive Latino professionals in underserved {\bf public policy} issue areas.
			\item This paid Fellowship Program offers exceptional Latinos who have earned a graduate degree or higher related to a chosen policy issue area within three years of program start date unparalleled exposure to hands-on experience in public policy.
			\item This program focuses specifically on the areas of: \vspace{-0.1cm}
				\begin{itemize} \itemsep -1pt
				\item Higher Education: CHCI Graduate Higher Education Fellowship, \url{http://www.chci.org/fellowships/page/chci-graduate-higher-education-fellowship}
				\item Secondary Education: CHCI Graduate Secondary Education Fellowship, \url{http://www.chci.org/fellowships/page/chci-graduate-secondary-education-fellowship}
				\item Health: CHCI Graduate Health Fellowship, \url{http://www.chci.org/fellowships/page/chci-graduate-health-fellowship}
				\item Housing: CHCI Graduate Housing Fellowship, \url{http://www.chci.org/fellowships/page/chci-graduate-housing-fellowship}
				\item International Affairs (includes last three months abroad in Mexico): CHCI Graduate International Affairs Fellowship, \url{http://www.chci.org/fellowships/page/chci-graduate-international-affairs-fellowship}
				\item Law: CHCI Graduate Law Fellowship, \url{http://www.chci.org/fellowships/page/chci-graduate-law-fellowship}
				\item STEM (Science, Technology, Engineering and Math): CHCI Graduate STEM Fellowship, \url{http://www.chci.org/fellowships/page/chci-graduate-stem-fellowship}
				\end{itemize}
			\item Program Dates: August to May (10-month internship)
			\item \url{http://www.chci.org/fellowships/page/chci-graduate-fellowship-program}
			\end{enumerate}
		\end{enumerate}
	\end{enumerate}
\item American Indian Graduate Center (AIGC): \vspace{-0.3cm}
	\begin{enumerate} \itemsep -2pt
	\item AIGC scholarships and fellowships: \vspace{-0.2cm}
		\begin{enumerate} \itemsep -2pt
		\item for advanced degree students in art, music, environmental studies, journalism, communications, medicine, dentistry, public health, nursing, or other health-related fields
		\item for members of Wisconsin, New Mexico or Arizona tribes.
		\item \url{http://www.aigc.com/02scholarships/scholarships.htm}
		\item AIGC Fellowship (Graduate) for Native Americans and their descendants seeking advanced degrees: \url{http://www.aigc.com/02scholarships/aigc/fellowship.htm}
		\item Rainer Scholarship (for grad students): \url{http://www.aigc.com/02scholarships/rainer.htm}
		\end{enumerate}
	\item List of resources about scholarships and fellowships: \vspace{-0.2cm}
		\begin{enumerate} \itemsep -2pt
		\item \url{http://www.aigc.com/08otherscholarship/otherscholarships.html}
		\item Scholarships: \url{http://www.aigc.com/08otherscholarship/scholarships.htm}
		\item Fellowships: \url{http://www.aigc.com/08otherscholarship/fellowships.htm}
		\end{enumerate}
	\item Gates Millennium Scholar Program (for individuals seeking basic and advanced degrees): \url{http://www.aigc.com/03gms/gms.htm}
	\end{enumerate}
\item Asian \& Pacific Islander American Scholarship Fund (APIASF) scholarship resources: \url{http://www.apiasf.org/scholarships.html}
\item American Association of University Women: \vspace{-0.3cm}
	\begin{enumerate} \itemsep -2pt
	\item \url{http://www.aauw.org/learn/fellowships_grants/index.cfm}
	\end{enumerate}
\item Sigma Delta Epsilon-Graduate Women in Science (GWIS): \url{http://www.gwis.org/programs.html}
\item Society of Hispanic Professional Engineers (SHPE): \vspace{-0.3cm}
	\begin{enumerate} \itemsep -2pt
	\item Advancing Hispanic Excellence in Technology, Engineering, Math and Science (AHETEMS) Foundation: \url{http://www.ahetems.org/}
	\item AHETEMS Scholarship Program: \url{http://www.ahetems.org/scholarships/}
	\item Graduate \& Young Professional Fellowship Program (encourage young professionals to engage in {\bf public policy}): \url{http://www.ahetems.org/graduate/graduate-young-professional-fellowship-program/}
	\item SHPE/GEM Fellowship (for graduate students in STEM at GEM Member Universities): \url{http://www.ahetems.org/graduate/shpe-gem-graduate-award/}. See \url{http://www.gemfellowship.org/gem-universities/university-members} for a list of GEM member universities.
	\end{enumerate}
\item National Society of Black Engineers (NSBE): \vspace{-0.3cm}
	\begin{enumerate} \itemsep -2pt
	\item Scholarships: \url{http://www.nsbe.org/Programs/Scholarships.aspx}
	\end{enumerate}
\item The Society of Mexican American Engineers and Scientists (MAES): \vspace{-0.3cm}
	\begin{enumerate} \itemsep -2pt
	\item Scholarships \& Awards: \url{http://www.maes-natl.org/index.php?meid=328}
	\item MAES Scholarship Program: \url{http://www.maes-natl.org/index.php?module=ContentExpress&func=display&ceid=518&meid=241}
	\end{enumerate}
\item SACNAS (Society for Advancement of Chicanos and Native Americans in Science): \vspace{-0.3cm}
	\begin{enumerate} \itemsep -2pt
	\item Scholarships: \url{http://www.sacnas.org/webadindex.cfm?webadcategory_id=7}
	\item Fellowships: \url{http://www.sacnas.org/webadIndex.cfm?webadcategory_id=5}
	\end{enumerate}
\item {\it Center for the Advancement of Hispanics in Science and Engineering Education} (CAHSEE): \vspace{-0.3cm}
	\begin{enumerate} \itemsep -2pt
	\item Scholarships: \url{http://www.cahsee.org/6resources/scholarships.asp.htm}
	\end{enumerate}
\item National Consortium for Graduate Degrees for Minorities in Engineering and Science, Inc.: \vspace{-0.3cm}
	\begin{enumerate} \itemsep -2pt
	\item National GEM Consortium: GEM Fellowship, \url{http://www.gemfellowship.org/gem-fellowship/application-requirements}
	\end{enumerate}
\item National Physical Science Consortium (NPSC): \vspace{-0.3cm}
	\begin{enumerate} \itemsep -2pt
	\item NPSC Graduate Fellowship: \url{http://www.npsc.org/}
	\end{enumerate}
\item Finch College Alumnae Association: \vspace{-0.3cm}
	\begin{enumerate} \itemsep -2pt
	\item The Finch College Alumnae Foundation Education Grant: \vspace{-0.2cm}
		\begin{enumerate} \itemsep -2pt
		\item \url{http://www.finchcollege.org/newscholarships.html}
		\item \url{http://www.finchcollege.org/newFinchGrantQandA.html}
		\item ``THE FINCH GRANT, an annual program where four community college women entering a four year college are awarded a grant of \$1500 which can be used toward any needs to completing college.  The selection is determined by a panel of college professors.''
		\end{enumerate}
	\end{enumerate}
\item : \url{}
\item : \url{}
\item : \url{}
\item : \url{}
\item : \url{}
\item \S\ref{phdandpostdocfellowships} has more information concerning scholarships and fellowships in the following areas: \vspace{-0.3cm}
	\begin{enumerate} \itemsep -2pt
	\item electronic design automation (EDA), and related areas of design automation: \vspace{-0.2cm}
		\begin{enumerate} \itemsep -2pt
		\item bio design automation (BDA)
		\item Lab-on-chip design (LoC) automation
		\item MEMS/NEMS design automation
		\end{enumerate}
	\item digital VLSI design
	\item analog and mixed-signal (AMS) VLSI design
	\item computer architecture
	\item parallel computing
	\item concurrent programming
	\item data mining
	\item theoretical computer science
	\end{enumerate}
\item Ph.D. dissertation awards: \vspace{-0.3cm}
	\begin{enumerate} \itemsep -2pt
	\item --- --- --- --- --- --- --- --- --- --- --- --- --- --- --- --- --- --- --- --- --- --- --- --- --- --- --- --- --- --- ---
	\item \colorbox{blue}{\bf Ph.D. Dissertation Awards for Computer Science}
	% Ph.D. Dissertation Awards for Computer Science
	\item ACM Doctoral Dissertation Award: \url{http://awards.acm.org/doctoral_dissertation/}
	\item ACM Outstanding Ph.D. Dissertation Award in Electronic Design Automation: \url{http://www.sigda.org/opda.html}
	\item EDAA Outstanding Dissertation Award (European Design and Automation Association, EDAA): \url{http://www.edaa.com/dissertation_award.html} and \url{http://www.esat.kuleuven.be/micas/EDAA-Award/index.php}
	\item EuroSys Roger Needham PhD Award (in the systems area): \vspace{-0.2cm}
		\begin{enumerate} \itemsep -2pt
		\item Areas in systems include: \vspace{-0.1cm}
			\begin{enumerate} \itemsep -1pt
			\item operating systems
			\item distributed systems
			\item real-time systems
			\item systems aspects of databases
			\item language runtimes
			\item \colorbox{yellow}{\bf embedded systems}
			\item computer networks
			\end{enumerate}
		\item \url{http://www.eurosys.org/phdprize/index.php}
		\end{enumerate}
	\item ACM SIGPLAN Outstanding Doctoral Dissertation Award: \url{http://www.sigplan.org/award-dissertation.htm}
	\item ACM SIGKDD Doctoral Disseration Award (in data mining and knowledge discovery): \url{http://www.sigkdd.org/awards_dissertation.php}
	\item ACM SIGMOD Jim Gray Doctoral Dissertation Award (in the database field): \url{http://www.sigmod.org/sigmod-awards/doctoral-dissertation-award}
	\item Special Interest Group of the ACM on Management Information Systems (SIGMIS): \vspace{-0.2cm}
		\begin{enumerate} \itemsep -2pt
		\item ACM SIGMIS Doctoral Dissertation Award Competition (at the International Conference on Information Systems, ICIS): \url{http://ai.arizona.edu/icis2009/program/dissertation.html} and \url{http://icis2010.aisnet.org/dissertation_award.htm}
		\end{enumerate}
	\item Association for Symbolic Logic: \vspace{-0.2cm}
		\begin{enumerate} \itemsep -2pt
		\item ``The Sacks Prize is awarded for the most outstanding doctoral dissertation in mathematical logic''.
		\item \url{http://www.aslonline.org/Sacks_nominations.html} and \url{http://www.aslonline.org/info-prizes.html}
		\end{enumerate}
	\item European Association for Computer Science Logic (EACSL): \vspace{-0.2cm}
		\begin{enumerate} \itemsep -2pt
		\item Ackermann Award (for outstanding dissertations in Logic in Computer Science): \url{http://www.eacsl.org/} and \url{http://www.eacsl.org/award.html}
		\end{enumerate}
	\item European Coordinating Committee for Artificial Intelligence (ECCAI): \vspace{-0.2cm}
		\begin{enumerate} \itemsep -2pt
		\item 201X Artificial Intelligence Dissertation Award: \url{http://www.eccai.org/diss-award/current.shtml}
		\end{enumerate}
	\item European Conference on Wireless Sensor Networks (EWSN 201X, \url{http://www.nes.uni-due.de/ewsn2011}) and CONET, the Cooperating Objects Network of Excellence: Ph.D. Thesis Award Competition, \url{http://www.cooperating-objects.eu/}
	\item --- --- --- --- --- --- --- --- --- --- --- --- --- --- --- --- --- --- --- --- --- --- --- --- --- --- --- --- --- --- ---
	\item \colorbox{blue}{\bf Ph.D. Dissertation Awards for Mathematics}
	% Ph.D. Dissertation Awards for Mathematics
	\item International Center for Scientific Research (CIRS): \vspace{-0.2cm}
		\begin{enumerate} \itemsep -2pt
		\item E. W. Beth Dissertation Prize (for outstanding dissertations in the fields of Logic, Language and Information): \url{http://www.cirs.net/prix/awards.php?id=481}
		\end{enumerate}
	\item The Association for Operations Management, APICS (Advancing Productivity, Innovation, and Competitive Success): \vspace{-0.2cm}
		\begin{enumerate} \itemsep -2pt
		\item Plossl Doctoral Dissertation Competition: The APICS Educational and Research Foundation, will annually grant one award of \$2,500 for a doctoral dissertation dealing with any topic in operations management. Sample topics include operations strategy, operations planning and control systems, supply chain management, quality management, Six Sigma, facility location, forecasting, just-in-time/lean production systems, and project management. Entrants must be candidates for the doctorate in operations management. The dissertation must be approved by the primary thesis advisor and not more than 50\% completed at time of submission. See \url{http://www.apics.org/Education/ERFoundation/Competitions/plossl.htm}.
		\end{enumerate}
	\item SIAM Richard C. DiPrima Prize: \vspace{-0.2cm}
		\begin{enumerate} \itemsep -2pt
		\item The Richard C. DiPrima Prize is awarded every two years to a junior scientist, based on an outstanding doctoral dissertation in applied mathematics.
		\item \url{http://www.siam.org/prizes/nominations/nom_diprima.php}
		\item \url{http://www.siam.org/prizes/sponsored/diprima.php}
		\end{enumerate}
	\item MOS A.W. Tucker Prize: \vspace{-0.2cm}
		\begin{enumerate} \itemsep -2pt
		\item It is awarded for an outstanding doctoral thesis in any aspect of mathematical optimization.
		\item \url{http://www.mathprog.org/?nav=tucker}
		\end{enumerate}
	\item --- --- --- --- --- --- --- --- --- --- --- --- --- --- --- --- --- --- --- --- --- --- --- --- --- --- --- --- --- --- ---
	\item \colorbox{blue}{\bf Other Ph.D. Dissertation Awards}
	% Other Ph.D. Dissertation Awards
	\item Institute for Operations Research and the Management Sciences (INFORMS): \vspace{-0.2cm}
		\begin{enumerate} \itemsep -2pt
		\item INFORMS George B. Dantzig Dissertation Award: \url{http://www.informs.org/Recognize-Excellence/INFORMS-Prizes-Awards/George-B.-Dantzig-Dissertation-Award}
		\item Best Dissertation Award (Technology Management Section, for Ph.D. theses in technology management): \url{http://www.informs.org/Recognize-Excellence/INFORMS-Community-Prizes-and-Awards2/Technology-Management-Section/Best-Dissertation-Award}
		\item TSL Dissertation Prize (Transportation Science and Logistics Section, for doctoral dissertations in the transportation science and logistics area): \url{http://www.informs.org/Recognize-Excellence/INFORMS-Community-Prizes-and-Awards2/Transportation-Science-and-Logistics-Section/TSL-Dissertation-Prize}
		\item Best Dissertation Award (Telecommunications Section, for Ph.D. theses in telecommunications): \url{http://www.informs.org/Recognize-Excellence/INFORMS-Community-Prizes-and-Awards2/Telecommunications-Section/Best-Dissertation-Award}
		\item Frank M. Bass Dissertation Paper Award (Society for Marketing Science, for the best marketing paper derived from a Ph.D. thesis published in an INFORMS-sponsored journal): \url{http://www.informs.org/Recognize-Excellence/INFORMS-Community-Prizes-and-Awards2/Society-for-Marketing-Science/Frank-M.-Bass-Dissertation-Paper-Award}
		\item SOLA - Air Products Bi-Annual Dissertation Award (Section on Location Analysis, for Ph.D. theses on location related research): \url{http://www.informs.org/Recognize-Excellence/INFORMS-Community-Prizes-and-Awards2/Section-on-Location-Analysis/SOLA-Air-Products-Bi-Annual-Dissertation-Award}
		\end{enumerate}
	\item EURO Doctoral Dissertation Award (EDDA) (in operations research): \url{http://www.euro-online.org/display.php?page=edda1}
	\end{enumerate}
\item Other awards: \vspace{-0.3cm}
	\begin{itemize} \itemsep -2pt
	\item --- --- --- --- --- --- --- --- --- --- --- --- --- --- --- --- --- --- --- --- --- --- --- --- --- --- --- --- --- --- ---
	\item \colorbox{blue}{\bf Awards for Computer Science}
	% Awards for Computer Science
	\item ACM SIGMOD Undergraduate Award: \url{http://www.sigmod.org/sigmod-awards/sigmod-awards#undergraduate}
	\item European Association of Theoretical Computer Science (EATCS): Presburger Award (for young researchers in theoretical computer science), \url{http://www.eatcs.org/index.php/presburger}.
	\item Computer Research Association: \vspace{-0.2cm}
		\begin{enumerate} \itemsep -2pt
		\item Committee on the Status of Women in Computing Research (CRA-W): \vspace{-0.1cm}
			\begin{enumerate} \itemsep -1pt
			\item Borg Early Career Award (BECA): \url{http://www.cra-w.org/borg}
			\end{enumerate}
		\end{enumerate}
	\item European Conference on Wireless Sensor Networks (EWSN 201X, \url{http://www.nes.uni-due.de/ewsn2011}) and CONET, the Cooperating Objects Network of Excellence: Ph.D. Thesis Award Competition, \url{http://www.cooperating-objects.eu/}. ``Cooperating Objects combine the strong functional aspects of embedded systems, pervasive computing and wireless sensor networks. Cooperating objects entities federate themselves into dynamic and loose networks in order to reach a common goal. This common goal will typically be related to sensing or control.''
	\item --- --- --- --- --- --- --- --- --- --- --- --- --- --- --- --- --- --- --- --- --- --- --- --- --- --- --- --- --- --- ---
	\item \colorbox{blue}{\bf Awards for Biomedical Engineering}
	% Awards for Biomedical Engineering
	\item Biomedical Engineering Society (BMES): \vspace{-0.2cm}
		\begin{enumerate} \itemsep -2pt
		\item Rita Schaffer Young Investigator Award (for junior researchers in biomedical engineering): \url{http://www.bmes.org/aws/BMES/pt/sp/awards_investigator}
		\item Graduate and Undergraduate Student Awards: \url{http://www.bmes.org/aws/BMES/pt/sp/awards_student}
		\end{enumerate}
	\item --- --- --- --- --- --- --- --- --- --- --- --- --- --- --- --- --- --- --- --- --- --- --- --- --- --- --- --- --- --- ---
	\item \colorbox{blue}{\bf Awards for Mechanical Engineering}
	% Awards for Mechanical Engineering
	\item American Society of Mechanical Engineers (ASME): \vspace{-0.2cm}
		\begin{enumerate} \itemsep -2pt
		\item Henry Hess Award (authors of research papers who are below 31 years old): \url{http://www.asme.org/Governance/Honors/SocietyAwards/Henry_Hess_Award.cfm}
		\item Pi Tau Sigma Gold Medal (outstanding junior engineers): \url{http://www.asme.org/Governance/Honors/SocietyAwards/Pi_Tau_Sigma_Gold_Medal.cfm}
		\item Marshall B. Peterson Award (researchers in tribology who are below 30 years old): \url{http://www.asme.org/Governance/Honors/SocietyAwards/Marshall_B_Peterson_Award.cfm}
		\item Y.C. Fung Young Investigator Award (for young researchers in bioengineering): \url{http://www.asme.org/Governance/Honors/SocietyAwards/YC_Fung_Young_Investigator.cfm}
		\end{enumerate}
	\item --- --- --- --- --- --- --- --- --- --- --- --- --- --- --- --- --- --- --- --- --- --- --- --- --- --- --- --- --- --- ---
	\item \colorbox{blue}{\bf Awards for Civil Engineering}
	% Awards for Civil Engineering
	\item American Society of Civil Engineers (ASCE): \vspace{-0.3cm}
		\begin{enumerate} \itemsep -2pt
		\item Edmund Friedman Young Engineer Award for Professional Achievement (for junior engineers under the age of 36): \url{http://www.asce.org/AwardsContent.aspx?id=16776}
		\item Committee on Younger Members (CYM) Awards (for junior engineers): \url{http://www.asce.org/Content.aspx?id=11311}
		\item Collingwood Prize (for civil engineering researchers under the age of 35): \url{http://www.asce.org/AwardsContent.aspx?id=15352}
		\end{enumerate}
	\item --- --- --- --- --- --- --- --- --- --- --- --- --- --- --- --- --- --- --- --- --- --- --- --- --- --- --- --- --- --- ---
	\item \colorbox{blue}{\bf Awards for Chemical Engineering}
	% Awards for Chemical Engineering
	\item American Institute of Chemical Engineers (AIChE) awards: \url{http://www.aiche.org/Students/Awards/index.aspx}
	\item --- --- --- --- --- --- --- --- --- --- --- --- --- --- --- --- --- --- --- --- --- --- --- --- --- --- --- --- --- --- ---
	\item \colorbox{blue}{\bf Awards for Systems Engineering}
	% Awards for Systems Engineering
	\item International Council on Systems Engineering (INCOSE) Stevens Doctoral Award (for Promising Research in Systems Engineering and Integration; A.B.D.s / Ph.D. candidates): \url{http://www.incose.org/about/foundation/doctoralaward.aspx}
	\item --- --- --- --- --- --- --- --- --- --- --- --- --- --- --- --- --- --- --- --- --- --- --- --- --- --- --- --- --- --- ---
	\item \colorbox{blue}{\bf Awards for Mathematics, Operations Research, \& Management Sciences}
	% Awards for Mathematics, Operations Research, and Management Sciences
	\item Institute for Operations Research and the Management Sciences (INFORMS): \vspace{-0.2cm}
		\begin{enumerate} \itemsep -2pt
		\item INFORMS Undergraduate Operations Research Prize: \url{http://www.informs.org/Recognize-Excellence/INFORMS-Prizes-Awards/INFORMS-Undergraduate-Operations-Research-Prize}
		\item Optimization Prize for Young Researchers: \url{http://www.informs.org/Recognize-Excellence/INFORMS-Community-Prizes-and-Awards2/Optimization-Society/Optimization-Prize-for-Young-Researchers}
		\item Underrepresented Minorities and Women Honoraria: \url{http://www.informs.org/Recognize-Excellence/INFORMS-Community-Prizes-and-Awards2/Simulation-Society/Underrepresented-Minorities-and-Women-Honoraria}
		\item Best Dissertation Proposal Competition (College on Organization Science, for Ph.D. proposals in organizational science): \url{http://www.informs.org/Recognize-Excellence/INFORMS-Community-Prizes-and-Awards2/College-on-Organization-Science/Best-Dissertation-Proposal-Competition}
		\item ISMS Doctoral Dissertation Proposal Competition (Society for Marketing Science, for Ph.D. proposals in marketing): \url{http://www.informs.org/Recognize-Excellence/INFORMS-Community-Prizes-and-Awards2/Society-for-Marketing-Science/ISMS-Doctoral-Dissertation-Proposal-Competition}
		\end{enumerate}
	\item Alice T. Schafer Mathematics Prize For Excellence in Mathematics by an Undergraduate Woman: \url{http://www.awm-math.org/schaferprize.html}
	\item European Prize in Combinatorics: \vspace{-0.2cm}
		\begin{enumerate} \itemsep -2pt
		\item The prize is established to recognize excellent contributions in Combinatorics by young European researchers (eligibility of EU) not older than 35. 
		\item \url{http://www.math.tu-berlin.de/EuroComb05/prize.html}
		\end{enumerate}
	\item The AMS-MAA-SIAM Frank and Brennie Morgan Prize for Outstanding Research in Mathematics by an Undergraduate Student: \url{http://www.maa.org/awards/morgan.html}; \url{http://www.ams.org/profession/prizes-awards/ams-prizes/morgan-prize}; and \url{http://www.siam.org/prizes/sponsored/morgan.php}
	\item --- --- --- --- --- --- --- --- --- --- --- --- --- --- --- --- --- --- --- --- --- --- --- --- --- --- --- --- --- --- ---
	% Lists of awards
	\item \colorbox{blue}{\bf Lists of awards}: \vspace{-0.2cm}
		\begin{enumerate} \itemsep -2pt
		\item Association for Women in Science: \url{http://www.awis.org/displaycommon.cfm?an=1&subarticlenbr=69}
		\item International Center for Scientific Research (CIRS): \url{http://www.cirs.net/indexenglish.htm}
		\end{enumerate}
	\end{itemize}
\end{enumerate}














%%%%%%%%%%%%%%%%%%%%%%%%%%%%%%%%%%%%%%%%%%%
\section{Funding Nonprofit Organizations}
\label{fundingnonprofitorg}

Funding nonprofit organizations (including colleges and universities): \vspace{-0.3cm}
\begin{enumerate} \itemsep -4pt
\item Alfred P. Sloan Foundation: \vspace{-0.3cm}
	\begin{enumerate} \itemsep -2pt
	\item Major Program Areas: \url{http://www.sloan.org/program/1}
	\item Apply for Grants: \url{http://www.sloan.org/apply}
	\end{enumerate}
\item The Commonwealth Fund: \vspace{-0.3cm}
	\begin{enumerate} \itemsep -2pt
	\item Grants \& Programs: \vspace{-0.2cm}
		\begin{enumerate} \itemsep -2pt
		\item \url{http://www.commonwealthfund.org/Grants-and-Programs.aspx}
		\item ``The Fund supports independent research on health and social issues and makes grants to improve health care practice and policy. We are dedicated to helping people become more informed about their health care and improving care for vulnerable populations such as children, the elderly, low-income families, minorities, and the uninsured.''
		\end{enumerate}
	\end{enumerate}
\item The Heinz Endowments (Howard Heinz Endowment \& Vira I. Heinz Endowment): \vspace{-0.3cm}
	\begin{enumerate} \itemsep -2pt
	\item \url{http://www.heinz.org/grants.aspx}
	\item grant-making programs (for non-profit organizations): \vspace{-0.2cm}
		\begin{enumerate} \itemsep -2pt
		\item Arts \& Culture
		\item Children, Youth \& Families
		\item Education
		\item Environment
		\item Innovation Economy
		\end{enumerate}
	\end{enumerate}
\item Ford Foundation: \vspace{-0.3cm}
	\begin{enumerate} \itemsep -2pt
	\item Grants: \vspace{-0.2cm}
		\begin{enumerate} \itemsep -2pt
		\item \url{http://www.fordfoundation.org/grants/}
		\item Individuals Seeking Fellowships: \vspace{-0.1cm}
			\begin{enumerate} \itemsep -1pt
			\item \url{http://www.fordfoundation.org/grants/individuals-seeking-fellowships}
			\item Ford Foundation Fellowship Programs: \url{http://sites.nationalacademies.org/PGA/FordFellowships/index.htm}
			\item Ford Foundation International Fellowships Program: \url{http://www.fordifp.net/}
			\end{enumerate}
		\item Organizations Seeking Grants: \url{http://www.fordfoundation.org/grants/organizations-seeking-grants}
		\item Other Philanthropic Resources: \url{http://www.fordfoundation.org/grants/other-philanthropic-resources}
		\item Grant Search Results (list of grants): \url{http://www.fordfoundation.org/grants/search}
		\end{enumerate}
	\end{enumerate}
\item The Rockefeller Foundation: \vspace{-0.3cm}
	\begin{enumerate} \itemsep -2pt
	\item Grants \& Grantees: \vspace{-0.2cm}
		\begin{enumerate} \itemsep -2pt
		\item \url{http://www.rockefellerfoundation.org/grants}
		\item What We Fund: \url{http://www.rockefellerfoundation.org/grants/what-we-fund}
		\item Resources for Grantseekers: Links to other Philanthropic Sources, \url{http://www.rockefellerfoundation.org/grants/resources-grantseekers}
		\end{enumerate}
	\end{enumerate}
\item Carnegie Corporation of New York: \vspace{-0.3cm}
	\begin{enumerate} \itemsep -2pt
	\item Grantseekers: \vspace{-0.2cm}
		\begin{enumerate} \itemsep -2pt
		\item \url{http://carnegie.org/grants/grantseekers/}
		\item What we fund: \url{http://carnegie.org/grants/grantseekers/what-we-fund/}
		\item What we don't fund: \url{http://carnegie.org/grants/grantseekers/what-we-dont-fund/}
		\end{enumerate}
		\item Grants database: \url{http://carnegie.org/grants/grants-database/} and \url{http://carnegie.org/grants/}
		\item (Past) individual foundation grants: \url{http://carnegie.org/publications/carnegie-reporter/single/view/article/item/221/}
	\end{enumerate}
\item The Kresge Foundation: \vspace{-0.3cm}
	\begin{enumerate} \itemsep -2pt
	\item fields of interest: \vspace{-0.2cm}
		\begin{enumerate} \itemsep -2pt
		\item health,
		\item the environment,
		\item community development,
		\item arts and culture,
		\item education, and
		\item human services
		\end{enumerate}
	\item Values Criteria (for grantmaking): \url{http://www.kresge.org/index.php/who/our_values_criteria/}
	\item funding methods: \vspace{-0.2cm}
		\begin{enumerate} \itemsep -2pt
		\item \url{http://www.kresge.org/index.php/how/index/}
		\item \url{http://www.kresge.org/index.php/our_funding_methods/index/}
		\end{enumerate}
	\item Challenge Grant: \vspace{-0.2cm}
		\begin{enumerate} \itemsep -2pt
		\item \url{http://www.kresge.org/index.php/our_funding_methods/challenge_grant_program/}
		\item ``The Kresge Foundation awards facilities capital as a challenge grant to help nonprofit organizations build their base of private financial support as they conduct capital campaigns to build or renovate their facilities.''
		\item ``Facilities capital challenge grants are awarded to organizations that cater specifically to the needs of poor, disadvantaged and disenfranchised in six program areas: Health Program, the Environment Program, Arts and Culture Program, Education Program, Human Services Program, and Community Development / Detroit Program.''
		\item ``Most challenge grant awards are made to U.S.-based organizations. On rare occasions, we award challenge grants to international organizations undertaking exceptional projects that align with the strategic objectives of a given program and advance Kresge's values.''
		\end{enumerate}
	\item Detroit Program: \vspace{-0.2cm}
		\begin{enumerate} \itemsep -2pt
		\item Kresge Arts Support: \url{http://www.kresge.org/index.php/what/detroit_program/kresge_arts_support/}
		\item Kresge Arts in Detroit: \url{http://www.kresge.org/index.php/what/detroit_program/kresge_arts_in_detroit/}
		\end{enumerate}
	\item Our Grants: \vspace{-0.2cm}
		\begin{enumerate} \itemsep -2pt
		\item \url{http://www.kresge.org/index.php/our_grants/index/}
		\item grants database: \url{http://maps.foundationcenter.org/grantmakers/index.php?gmkey=KRES002}
		\item Arts and Community Building: \vspace{-0.1cm}
			\begin{enumerate} \itemsep -1pt
			\item \url{http://www.kresge.org/index.php/what/arts_and_culture/arts_and_community_building#Community Arts}
			\item ``Cultural institutions and artists animate our communities, bring disparate people together to share common experiences, and help us imagine a better future. As the demographics of our communities become more diverse, artists and cultural institutions help us bridge differences and build cross-cultural understanding. As our economy struggles, creative enterprises and creative sector leaders offer hope for community renewal and new job development.''
			\item two pilot initiatives: College/Arts initiative, and the Community Arts initiative
			\item ``The pilot cities [for the Community Arts initiative] include Baltimore, Maryland; Birmingham, Alabama; Detroit, Michigan; St. Louis, Missouri; and Tucson, Arizona.''
			\item ``Grants for Arts and Community Building are by invitation only.''
			\end{enumerate}
		\end{enumerate}
	\end{enumerate}
\item New York Women's Foundation: \vspace{-0.3cm}
	\begin{enumerate} \itemsep -2pt
	\item Grant Information and Application: \vspace{-0.2cm}
		\begin{enumerate} \itemsep -2pt
		\item \url{http://www.nywf.org/grant.html}
		\item focus areas: \vspace{-0.1cm}
			\begin{enumerate} \itemsep -1pt
			\item Anti-Violence and Safety
			\item Economic Security
			\item Health, Sexual Rights and Reproductive Justice
			\end{enumerate}
		\item ``Grants usually range from \$50,000 to a maximum of \$70,000 [that last for a year, and can be renewed up to 5 years].''
		\end{enumerate}
	\end{enumerate}
\item The Foundation Center: \vspace{-0.3cm}
	\begin{enumerate} \itemsep -2pt
	\item Grantseekers: \url{http://foundationcenter.org/getstarted/}
	\item Find funders: \url{http://foundationcenter.org/findfunders/}
	\item GrantSpace$^{\rm SM}$: \vspace{-0.2cm}
		\begin{enumerate} \itemsep -2pt
		\item \url{http://grantspace.org/}
		\item ``GrantSpace$^{\rm SM}$ will help you gain the knowledge and skills you need to get grants, manage your nonprofit, and improve your community.''
		\item ``Established in 1956 and today supported by close to 550 foundations, the Foundation Center is a national nonprofit service organization recognized as the nation�s leading authority on organized philanthropy, connecting nonprofits and the grantmakers supporting them to tools they can use and information they can trust. Its audiences include grantseekers, grantmakers, researchers, policymakers, the media, and the general public. The Center maintains the most comprehensive database on U.S. grantmakers and their grants; issues a wide variety of print, electronic, and online information resources; conducts and publishes research on trends in foundation growth, giving, and practice; and offers an array of free and affordable educational programs.''
		\item Resources for Non-U.S. Grantseekers: \url{http://grantspace.org/Tools/Knowledge-Base/Resources-for-Non-U.S.-Grantseekers}
		\item Resources for Individual Grantseekers: \vspace{-0.1cm}
			\begin{enumerate} \itemsep -1pt
			\item \url{http://grantspace.org/Tools/Knowledge-Base/Individual-Grantseekers}
			\item \url{http://gtionline.foundationcenter.org/}
			\item General: \url{http://grantspace.org/Tools/Knowledge-Base/Individual-Grantseekers/General}
			\item Artists: \url{http://grantspace.org/Tools/Knowledge-Base/Individual-Grantseekers/Artists}
			\item Students: \url{http://grantspace.org/Tools/Knowledge-Base/Individual-Grantseekers/Students}
			\item Fiscal Sponsorship: \url{http://grantspace.org/Tools/Knowledge-Base/Individual-Grantseekers/Fiscal-Sponsorship}
			\item For-Profit Enterprises: \url{http://grantspace.org/Tools/Knowledge-Base/Individual-Grantseekers/For-Profit-Enterprises}
			\end{enumerate}
		\end{enumerate}
	\end{enumerate}
\item The Lemelson Foundation: \vspace{-0.3cm}
	\begin{enumerate} \itemsep -2pt
	\item \url{http://web.mit.edu/invent/w-foundation.html}
	\item Programs \& Grants: \url{http://www.lemelson.org/programs-grants}
	\item Grantmaking: \url{http://www.lemelson.org/grantmaking}
	\end{enumerate}
\item Partnership for Higher Education in Africa (PHEA): \vspace{-0.3cm}
	\begin{enumerate} \itemsep -2pt
	\item \url{http://www.foundation-partnership.org/} and \url{http://www.foundation-partnership.org/index.php?id=1}
	\item Grants Database: \url{http://www.foundation-partnership.org/index.php?id=2}
	\item Partnership Publications: \url{http://www.foundation-partnership.org/index.php?id=3}
	\end{enumerate}
\item Smithsonian Institution: \vspace{-0.3cm}
	\begin{enumerate} \itemsep -2pt
	\item Smithsonian Institution Traveling Exhibition Service (SITES): \vspace{-0.2cm}
		\begin{enumerate} \itemsep -2pt
		\item Smithsonian Community Grant program (supported by MetLife Foundation): \vspace{-0.1cm}
			\begin{enumerate} \itemsep -1pt
			\item \url{http://www.sites.si.edu/funding/grant2.htm}
			\item ``This program seeks to deepen connections between SITES' host venues and their communities by encouraging exhibitors to engage their local audiences in new and exciting ways while creating broader access to our exhibitions.''
			\item ``Under this new program, eligible SITES exhibitors may apply for up to \$5,000 for expenses related to public, educational programming produced in conjunction with a SITES exhibit. Exhibitors may choose to enhance current program offerings or to create a new program especially suited to the topic of the exhibition.''
			\end{enumerate}
		\end{enumerate}
	\end{enumerate}
\end{enumerate}
















%%%%%%%%%%%%%%%%%%%%%%%%%%%%%%%%%%%%%%%%%%%
\section{Technology-Related Public Policy}
\label{techpublicpolicy}

Resources for engagement in creating technology-related public policy: \vspace{-0.3cm}
\begin{enumerate} \itemsep -4pt
\item Yale Journal of Law \& Technology (YJOLT): \vspace{-0.3cm}
	\begin{enumerate} \itemsep -2pt
	\item \url{http://www.yjolt.org/}
	\item \url{http://wingenroth.org/}
	\end{enumerate}
\item ACM Public Policy Office: \vspace{-0.3cm}
	\begin{enumerate} \itemsep -2pt
	\item It represents ACM and its US Public Policy Council (USACM) on information technology policy issues that impact the computing field.
	\item It seeks to educate policymakers and the public about policies that will that foster innovations in computing and related disciplines in ways that benefit society.
	\item It also informs ACM's members and the public about policy developments through its weblog, Washington Update newsletter and articles in ACM publications.
	\item ACM US Public Policy Council (USACM): \url{http://usacm.acm.org/}
	\item ACM Committee on Computers and Public Policy (CCPP): \url{http://www.acm.org/public-policy/acm-committee-on-computers-and-public-policy}
	\item \url{http://www.acm.org/public-policy}
	\end{enumerate}
\item IEEE: \vspace{-0.3cm}
	\begin{enumerate} \itemsep -2pt
	\item IEEE-USA: \url{http://www.ieeeusa.org/policy/default.asp}
	\item Smart Grids: \url{http://smartgrid.ieee.org/public-policy}
	\end{enumerate}
\item Computing Community Consortium (CCC): \url{http://www.cra.org/ccc/}
\item Computing Research Association (CRA): \vspace{-0.3cm}
	\begin{enumerate} \itemsep -2pt
	\item \url{http://www.cra.org/}
	\item CRA Government Affairs: \url{http://www.cra.org/govaffairs/index.php}
	\end{enumerate}
\item EngineeringPolicy.org: \url{http://www.engineeringpolicy.org/}
\item Congressional Bi-Partisan Robotics Caucus: \url{http://www.roboticscaucus.org/}
\item Advisory Committee for the Congressional Research and Development $[$R\&D$]$ Caucus: \url{http://www.researchcaucus.org/}
\item {\it National Academies Press} (NAP), from the (US) {\it National Academies}: \url{http://www.nap.edu/}
\item {\it Coalition to Diversify Computing}: \url{http://www.cdc-computing.org/}
\item American Institute of Aeronautics and Astronautics (AIAA): \vspace{-0.3cm}
	\begin{enumerate} \itemsep -2pt
	\item \url{http://www.aiaa.org/content.cfm?pageid=7}
	\end{enumerate}
\item : \url{}
\item : \url{}
\item : \url{}
\item : \url{}
\item : \url{}
\item : \url{}
\item : \url{}
\item : \url{}
\end{enumerate}






%%%%%%%%%%%%%%%%%%%%%%%%%%%%%%%%%%%%%%%%%%%
\section{Feminist Outreach}
\label{feministoutreach}

Feminist outreach: \vspace{-0.3cm}
\begin{enumerate} \itemsep -4pt
\item Myra Sadker Foundation: \vspace{-0.3cm}
	\begin{enumerate} \itemsep -2pt
	\item $100+$ Ideas to Promote Gender Equity in Schools and Beyond: \url{http://www.sadker.org/100ideas.html}
	\item Gender Equity Activities: \url{http://www.sadker.org/WhatYouCanDo.html}
	\item Gender Equity Activities for Concerned Citizens: \url{http://www.sadker.org/GenderEquity-citizens.html}
	\item Gender Equity Activities for Families: \url{http://www.sadker.org/GenderEquity-family.html}
	\item Gender Equity Activities for Teachers: \vspace{-0.2cm}
		\begin{enumerate} \itemsep -2pt
		\item Early Childhood: \url{http://www.sadker.org/GenderEquity-teacher1.html}
		\item Primary Grades: \url{http://www.sadker.org/GenderEquity-teacher2.html}
		\item Upper Elementary: \url{http://www.sadker.org/GenderEquity-teacher3.html}
		\item Middle and High School: \url{http://www.sadker.org/GenderEquity-teacher4.html}
		\end{enumerate}
	\item Resources for feminism and links to web pages of feminist organizations: \url{http://www.sadker.org/ReadsLinks.html}
	\end{enumerate}
\item Feminist student organizations at colleges and universities: \vspace{-0.3cm}
	\begin{enumerate} \itemsep -2pt
	\item For example, at the University of Southern California, the organizations associated with feminist causes are: \vspace{-0.2cm}
		\begin{enumerate} \itemsep -2pt
		\item {\it USC Center for Women \& Men}: \url{http://www.usc.edu/student-affairs/cwm/links.html}
		\item {\it USC Women's Student Assembly}: \url{http://www-scf.usc.edu/~wsausc/Welcome.html}
		\end{enumerate}
	\end{enumerate}
\item International Women's Day: \url{http://www.internationalwomensday.com/}
\item Gender Across Borders: \vspace{-0.3cm}
	\begin{enumerate} \itemsep -2pt
	\item Feminism Resources: \url{http://www.genderacrossborders.com/feminist-resources/}
	\end{enumerate}
\item {\it V-Day}: \vspace{-0.3cm}
	\begin{enumerate} \itemsep -2pt
	\item \url{http://www.vday.org/}
	\item Organization that helps women plan and organize events to bring awareness about sexual assault, and what we can do to reduce sexual assault.
	\end{enumerate}
\item {\it Take Back The Night}: \vspace{-0.3cm}
	\begin{enumerate} \itemsep -2pt
	\item \url{http://www.takebackthenight.org/}
	\item Organization that helps women plan and organize events to bring awareness about sexual assault, and what we can do to reduce sexual assault. It also encourages sexual assault survivors to speak out about their sexual assaults, so that they would shame their perpetrators and let other women (and men) know that they is nothing to be ashamed of as sexual assault survivors. This is because the faults lie 100\% with the perpetrators, and not with the survivors.
	\end{enumerate}
\item {\it United Nations Development Fund for Women} (UNIFEM): \vspace{-0.3cm}
	\begin{enumerate} \itemsep -2pt
	\item \url{http://www.unifem.org/}
	\item Organization that addresses many challenges faced by girls and women.
	\end{enumerate}
\item {\it National Organization for Women}: \vspace{-0.3cm}
	\begin{enumerate} \itemsep -2pt
	\item \url{http://www.now.org/}
	\item Feminist organization in the US.
	\end{enumerate}
\item {\it A Woman's Nation}: \vspace{-0.3cm}
	\begin{enumerate} \itemsep -2pt
	\item \url{http://www.shriverreport.com/awn/}
	\item \url{http://awomansnation.com} or \url{http://www.shriverreport.com/}
	\end{enumerate}
\item {\it Peace Over Violence} is a non-profit, feminist, multicultural, volunteer organization dedicated to a building healthy relationships, families and communities free from sexual, domestic and interpersonal violence: \url{http://peaceoverviolence.org/}
\item SoulSpeakOut: \url{http://www.soulspeakout.org/resources/}
\item {\it Haven Hills}: \url{http://havenhills.org/}
%\item MaleSurvivor: \url{http://www.malesurvivor.org/}
\end{enumerate}












%%%%%%%%%%%%%%%%%%%%%%%%%%%%%%%%%%%%%%%%%%%
\section{Outreach: Professional Organizations}
\label{outreachproorgs}

Professional organizations: \vspace{-0.3cm}
\begin{enumerate} \itemsep -4pt
\item --- --- --- --- --- --- --- --- --- --- --- --- --- --- --- --- --- --- --- --- --- --- --- --- --- --- --- --- --- --- ---
\item \colorbox{blue}{\bf Professional Organizations for the Performance, Literary, and Visual Arts}
% Professional Organizations for the Performance, Literary, and Visual Arts
\item Americans for the Arts: \vspace{-0.3cm}
	\begin{enumerate} \itemsep -2pt
	\item \url{http://www.americansforthearts.org/get_involved/membership/default.asp}
	\item \url{http://www.artsusa.org/get_involved/membership/default.asp}
	\item Provides membership for organizations and individuals
	\item Individual membership are available for: \vspace{-0.2cm}
		\begin{enumerate} \itemsep -2pt
		\item Students
		\item Entrepreneurs (e.g., people in art management)
		\item Innovators
		\item Colleagues (artists)
		\end{enumerate}
	\item Americans for the Arts {\bf Emerging Leader Program}: \vspace{-0.2cm}
		\begin{enumerate} \itemsep -2pt
		\item \url{http://www.artsusa.org/networks/emerging_leaders/resources/default.asp}
		\item Has various resources for professional development, including mentoring
		\end{enumerate}
	\item Advocacy ({\bf public policy}): \url{http://www.artsusa.org/get_involved/advocate.asp}
	\end{enumerate}
\item --- --- --- --- --- --- --- --- --- --- --- --- --- --- --- --- --- --- --- --- --- --- --- --- --- --- --- --- --- --- ---
\item \colorbox{blue}{\bf Professional Organizations for the Musical Artists}
% Professional Organizations for the Musical Artists
\item The Recording Academy: \url{http://www.grammy365.com/join/membership-types}
\end{enumerate}













%%%%%%%%%%%%%%%%%%%%%%%%%%%%%%%%%%%%%%%%%%%
\section{Other Outreach}
\label{otheroutreach}

Other outreach: \vspace{-0.3cm}
\begin{enumerate} \itemsep -4pt
\item The Joy McCann Foundation: \vspace{-0.3cm}
	\begin{enumerate} \itemsep -2pt
	\item The Joy McCann Professorships in Law: \url{http://www.mccannfoundation.org/law.htm}
	\end{enumerate}
\item National Academy of Sciences: \vspace{-0.3cm}
	\begin{enumerate} \itemsep -2pt
	\item {\it Science \& Entertainment Exchange} program: \vspace{-0.2cm}
		\begin{enumerate} \itemsep -2pt
		\item \url{http://www.scienceandentertainmentexchange.org/}
		\item Provide science and engineering knowledge to help professionals in the entertainment industry create engaging storylines involving science and technology.
		\end{enumerate}
	\end{enumerate}
\item U.S. Department of State: \vspace{-0.3cm}
	\begin{enumerate} \itemsep -2pt
	\item Bureau of Educational and Cultural Affairs: \vspace{-0.2cm}
		\begin{enumerate} \itemsep -2pt
		\item Programs: \url{http://exchanges.state.gov/jexchanges/programs.html}
		\item Fulbright Classroom Teacher Exchange Program: \vspace{-0.1cm}
			\begin{enumerate} \itemsep -1pt
			\item \url{http://exchanges.state.gov/globalexchanges/fulbright-teacher-exchange-program.html}
			\item ``The Fulbright Classroom Teacher Exchange provides opportunities for primary and secondary teachers to exchange positions with colleagues in other countries. The participants contribute to mutual understanding by bringing international knowledge and perspectives to the U.S. and foreign classrooms, schools and communities. Full-time U.S. teachers can take part in either a year-long or semester-long direct exchange with a counterpart in another country.''
			\end{enumerate}
		\item FORTUNE/U.S. State Department Global Women's Mentoring Partnership: \vspace{-0.1cm}
			\begin{enumerate} \itemsep -1pt
			\item \url{http://exchanges.state.gov/citizens/professionals/fortunepartnership.html}
			\item ``This public-private partnership places talented, emerging women leaders from all over the world in mentoring programs with FORTUNE's Most Powerful Women Leaders.''
			\end{enumerate}
		\item Edward R. Murrow Program for Journalists: \vspace{-0.1cm}
			\begin{enumerate} \itemsep -1pt
			\item \url{http://exchanges.state.gov/ivlp/murrow.html}
			\item ``The Edward R. Murrow Program for Journalists invites rising international journalists to travel to the United States and examine journalistic principles and practices.''
			\end{enumerate}
		\item International Visitor Leadership Program: \vspace{-0.1cm}
			\begin{enumerate} \itemsep -1pt
			\item \url{http://exchanges.state.gov/ivlp/ivlp.html}
			\item ``These visits reflect the International Visitors' professional interests and support the foreign policy goals of the United States.''
			\item ``International Visitors are current or emerging leaders in government, politics, the media, education, the arts, business and other key fields.''
			\item ``International Visitors travel to the U.S. for carefully designed programs that reflect their professional interests and U.S. foreign policy goals. They travel in a variety of thematic programs, either individually or in groups, for up to three weeks. While in the U.S., International Visitors typically visit Washington, DC and three additional towns or cities that highlight the tremendous diversity of the U.S. They attend professional appointments with their American counterparts, learn about the U.S. system of government at the national, state and local levels, visit American schools, and experience American culture and social life.''
			\item ``There is no application for this program. International Visitors are selected and nominated annually by American Foreign Service Officers at U.S. Embassies around the world.''
			\end{enumerate}
		\item Au Pair: \vspace{-0.1cm}
			\begin{enumerate} \itemsep -1pt
			\item \url{http://exchanges.state.gov/jexchanges/programs/aupair.html}
			\item ``Through the Au Pair program, foreign nationals between 18 and 26 years of age participate in the home life of a host family. Au pairs provide limited childcare services for up to 12 months. An extension of 6, 9, or 12 months may be granted in certain cases.''
			\end{enumerate}
		\item Summer Work Travel: \vspace{-0.1cm}
			\begin{enumerate} \itemsep -1pt
			\item \url{http://exchanges.state.gov/jexchanges/programs/swt.html}
			\item ``In the summer work travel program, post-secondary students may enter the United States to work and travel during their summer vacation. Participants can be admitted to the program more than once. The maximum length of the program is four months.''
			\end{enumerate}
		\item Internship: \vspace{-0.1cm}
			\begin{enumerate} \itemsep -1pt
			\item \url{http://exchanges.state.gov/jexchanges/programs/intern.html}
			\item ``Internship programs are designed to allow foreign professionals to come to the United States to gain exposure to U.S. culture and to receive training in U.S. business practices in their chosen occupational field.  The maximum duration of an internship in any occupational field is 12 months. Upon completion of their exchange programs, participants are expected to return to their home countries.''
			\end{enumerate}
		\item Professional Exchanges Division: \vspace{-0.1cm}
			\begin{enumerate} \itemsep -1pt
			\item \url{http://exchanges.state.gov/citizens/profs.html}
			\item ``The Professional Exchanges division provides grants to U.S. nonprofit organizations to carry out exchange programs that support the professional development of foreign participants. The purpose of each exchange program is to engage with foreign leaders in critical professions, to demonstrate respect for foreign cultures, and to promote mutual understanding between the people of the United States and other countries.''
			\item ``Professional exchanges typically last several years and include internships, study tours or workshops in the United States and in the host country. Participants come from a variety of professions including education administrators, public servants, journalists, labor union officials, entrepreneurs, environmental leaders, jurists, lawyers, and civic leaders.''
			\end{enumerate}
		\end{enumerate}
	\end{enumerate}
\item Teach For All: \url{http://teachforallnetwork.org/}
\item --- --- --- --- --- --- --- --- --- --- --- --- --- --- --- --- --- --- --- --- --- --- --- --- --- --- --- --- --- --- ---
\item \colorbox{blue}{\bf Resources for Artists and Musicians}
% Resources for Artists and Musicians
\item League of American Orchestras and the Association of Performing Arts Presenters: \vspace{-0.3cm}
	\begin{enumerate} \itemsep -2pt
	\item {\it ArtistsfromAbroad.org}: \vspace{-0.2cm}
		\begin{enumerate} \itemsep -2pt
		\item \url{http://www.artistsfromabroad.org/}
		\item ``{\it ArtistsfromAbroad.org} features complete and up-to-date guidance on the visa process and tax treatment for foreign guest artists.''
		\end{enumerate}
	\end{enumerate}
\item Young Concert Artists, Inc. \vspace{-0.3cm}
	\begin{enumerate} \itemsep -2pt
	\item Composer Program (for American composers between 20 and 26 years of age): \url{http://www.yca.org/auditions/}
	\end{enumerate}
\item The John F. Kennedy Center for the Performing Arts: \vspace{-0.3cm}
	\begin{enumerate} \itemsep -2pt
	\item Mary Lou Williams Women in Jazz Emerging Artist Workshop: \vspace{-0.2cm}
		\begin{enumerate} \itemsep -2pt
		\item \url{http://www.kennedy-center.org/programs/jazz/womeninjazz/competition.html}
		\item ``The workshop provides female jazz artists ages 18 to 35 with an opportunity to explore and develop their artistry under the guidance of leading jazz artists and instructors. Each year, the workshop will focus on a specific instrument.''
		\item ``The 2011 Mary Lou Williams Women in Jazz Emerging Artist Workshop is open to advanced female jazz pianists who plan to pursue jazz performance as a career. Eligibility is exclusive to pianists who will be 18-35 years old on May 18, 2011 and have never recorded or been contracted to record as a leader or co-leader on a major label at the time of application. All applicants must be proficient in English.''
		\end{enumerate}
	\end{enumerate}
\item Grantmakers in the Arts (GIA): \vspace{-0.3cm}
	\begin{enumerate} \itemsep -2pt
	\item ``The mission of Grantmakers in the Arts (GIA) is to provide leadership and service to advance the use of philanthropic resources on behalf of arts and culture.''
	\item Arts Funding Topics: \url{http://www.giarts.org/arts-funding-topics}
	\end{enumerate}
\item The Dana Foundation: \vspace{-0.3cm}
	\begin{enumerate} \itemsep -2pt
	\item Arts Education program: \vspace{-0.2cm}
		\begin{enumerate} \itemsep -2pt
		\item Arts Education Grants: \vspace{-0.1cm}
			\begin{enumerate} \itemsep -1pt
			\item \url{http://www.dana.org/grants/BrowseArtsGrants.aspx}
			\item ``In 2001, The Dana Foundation created the Arts Education program with a sole focus of providing grants to support professional development for teaching artists and in-school arts specialists. The first several years of grants were to  programs in New York City, Washington, DC, Los Angeles and to organizations with a 50 mile radius of the three.''
			\item ``The Rural Initiative launched in 2006 with 6 grants awarded to organizations providing professional development in rural areas of the United States.''
			\end{enumerate}
		\end{enumerate}
	\end{enumerate}
\item writing/poetry contests: \vspace{-0.3cm}
	\begin{enumerate} \itemsep -2pt
	\item International 3-Day Novel Contest: \url{http://www.3daynovel.com/about/?contest}
	\end{enumerate}
\end{enumerate}




%%%%%%%%%%%%%%%%%%%%%%%%%%%%%%%%%%%%%%%%%%%
\section{Christian Colleges and Universities}
\label{christianunis}

Christian colleges and universities: \vspace{-0.3cm}
\begin{enumerate} \itemsep -4pt
\item List of Christian colleges and universities: \vspace{-0.3cm}
	\begin{enumerate} \itemsep -2pt
	\item Council for Christian Colleges and Universities (CCCU): \vspace{-0.2cm}
		\begin{enumerate} \itemsep -2pt
		\item \url{http://en.wikipedia.org/wiki/Council_for_Christian_Colleges_and_Universities}
		\item \url{http://www.cccu.org/}
		\end{enumerate}
	\item Christian College Consortium: \vspace{-0.2cm}
		\begin{enumerate} \itemsep -2pt
		\item \url{http://en.wikipedia.org/wiki/Christian_College_Consortium}
		\item \url{http://www.ccconsortium.org/}
		\end{enumerate}
	\end{enumerate}
\item California Baptist University, Riverside
\item Messiah College (Grantham, PA): \vspace{-0.3cm}
	\begin{enumerate} \itemsep -2pt
	\item Department of Engineering: \vspace{-0.2cm}
		\begin{enumerate} \itemsep -2pt
		\item \url{http://www.messiah.edu/departments/engineering/}
		\item B.S. programs in: \vspace{-0.1cm}
			\begin{enumerate} \itemsep -1pt
			\item Biomedical Engineering
			\item Computer Engineering
			\item Electrical Engineering
			\end{enumerate}
		\end{enumerate}
	\item Department of Information and Mathematical Sciences: \vspace{-0.2cm}
		\begin{enumerate} \itemsep -2pt
		\item \url{http://www.messiah.edu/departments/mathsci/index.html}
		\item Offers a B.A. Computer Science program
		\end{enumerate}
	\end{enumerate}
\end{enumerate}
























%%%%%%%%%%%%%%%%%%%%%%%%%%%%%%%%%%%%%%%%%
% Thoughts and Resources for Specific Areas and Topics
%	% This is written by Zhiyang Ong for his management of information and tasks.
%
% It includes information on professional development, including membership of professional organizations and networking societies.





%%%%%%%%%%%%%%%%%%%%%%%%%%%%%%%%%%%%%%%%%%%
\section{Heuristic for Locating Outreach Resources}
\label{heuristiclocateoutreach}

\proc{Find}$(\varphi, \tau)$ is a heuristic for locating resources for outreach activities, which includes finding information about the following: \vspace{-0.3cm}
\begin{enumerate} \itemsep -4pt
\item awards
\item career resources (including material for career guidance)
\item competitions and contests
\item educational material (e.g., suggested activities and curricular) for specific areas, such as marine sciences and electrical/computer engineering
\item fellowships
\item internships
\item scholarships
\item summer camps
\item summer programs (or summer schools); here, summer schools refer to short educational programs that last from days (e.g., a weekend for the ACM SIGDA Design Automation Summer School) to about a month (e.g., Santa Fe Institute's Complex Systems Summer Schools)
\end{enumerate}
\ \\

Its input $\tau$ is the deadline by which this search process must terminate. For example, if I have to apply for internships by next week, I would use the date of a week from now as the deadline $\tau$. In line \ref{find-pt-professional-org}, an example of a professional organization is the Institute of Electrical and Electronics Engineers (IEEE). The term ``good'' that is used in line \ref{find-pt-gd-uni} is an arbitrary measure of quality determined by the reader/user. \\

A reading group (in line \ref{find-pt-reading-grp}) is a small group of (graduate) students, which may possibly include professors and postdocs, that meet regularly (e.g., once/twice a week) to discuss papers that they have read since the previous meeting/discussion. Each individual in the reading group can be assigned a paper to read and present at the next meeting. The aim of a reading group is to improve the coverage of papers in our research area that each member has read. This is important for interdisciplinary research, since grad students working in interdisciplinary research areas have so much ground to cover. \\

Line \ref{find-pt-athletics} uses the term ``athletics department'' to refer to an administrative department at an American college or university that is in charge of managing varsity/NCAA sports teams. An example of a profession-specific networking organization (line \ref{find-pt-netwk-org}) is DVClub. In line \ref{find-pt-domain-specific-www}, a domain-specific web page is {\it SAT Live!}. An example of a corporate research laboratory (line \ref{find-pt-corporate-research-labs}) is ``Cadence Research Laboratories'' (\url{http://www.cadence.com/cadence/cadence_labs/pages/default.aspx}), and an example of a research institute (line \ref{find-pt-research-institute}) is Santa Fe Institute.




\begin{codebox}
\Procname{$\proc{Find}(\varphi, \tau)$}
\zi	\Comment {\it Input $\varphi \gets $ Item to find out about}
\zi	\Comment {\it Input $\tau \gets $ Deadline for the search process}
\zi	\Comment {\it Output $\kappa \gets $ List of resources about $\varphi$}
\zi
\li \While ( [ resources about $\varphi$ are inadequate ] AND [ $\tau$ has not yet passed ] )
	\Do
\li	Find out the professional organizations for the field of $\varphi$	\label{find-pt-professional-org}
\li	\For each professional organization in the field
		\Do
\li		Check if it has information about $\varphi$ in its web pages, publications, or mailing list archive
\li		\If (it has information about $\varphi$)
			\Then
\li			Add that information to $\kappa$
			\End
		\End
\zi
\li	\For each good (college OR university)	\label{find-pt-gd-uni}
		\Do
\li		\If ($\varphi == $ summer programs )
			\Then
\li			Search for summer programs in the web pages of departments \& schools/colleges
\li		\ElseIf ($\varphi == $ summer camps )
			\Then
\li			Search for summer camps in the web pages of departments \& schools/colleges
\li			Search for summer camps in the web pages of administrative/athletics departments	\label{find-pt-athletics}
\li		\ElseNoIf
\li			Search for $\varphi$ in the web pages of the department(s), including its news section/archive
\li			Search for $\varphi$ in the web pages of professors, postdoctoral researchers, \& students
\li			Search for $\varphi$ in the web pages of reading groups		\label{find-pt-reading-grp}
\li			Search for $\varphi$ in the web pages of student organizations
\li			Search for $\varphi$ in the mailing list archive of classes \& the department
\li			Search for $\varphi$ in the mailing list archive of research groups/labs and projects
\li			Search for $\varphi$ in the mailing list archive of reading groups
\li			Search for $\varphi$ in the mailing list archive of student organizations
			\End
\zi
\li		\If (it has information about $\varphi$)
			\Then
\li			Add that information to $\kappa$
			\End
		\End
\zi
\li	Search for $\varphi$ in the mailing list archive of open-source projects
\li	Search for $\varphi$ in the mailing list archive of profession-specific networking organizations	\label{find-pt-netwk-org}
\li	Search for $\varphi$ in the web pages of domain-specific web pages	\label{find-pt-domain-specific-www}
\li	Search for $\varphi$ in the web pages of research scientists in corporate research labs	\label{find-pt-corporate-research-labs}
\li	Search for $\varphi$ in the web pages of research scientists in research institutes	\label{find-pt-research-institute}
\li	\If ( [ mailing list archive OR web page ] has information about $\varphi$)
		\Then
\li		Add that information to $\kappa$
		\End
	\End	
\li \Return $\kappa$
\end{codebox}




%%%%%%%%%%%%%%%%%%%%%%%%%%%%%%%%%%%%%%%%%%%
\section{General Outreach Resources}
\label{generaloutreachresources}

General outreach resources: \vspace{-0.3cm}
\begin{enumerate} \itemsep -4pt
\item volunteering opportunities: \vspace{-0.3cm}
	\begin{enumerate} \itemsep -2pt
	\item Engineers Without Borders: \url{http://www.ewb-international.org/}
	\item Australian Volunteers International: \url{http://www.australianvolunteers.com/}
	\item Youth Challenge Australia: \url{http://www.youthchallenge.com.au/}
	\item Go Volunteer: \url{http://www.govolunteer.com.au/}
	\item Volunteer Search: \url{http://www.volunteersearch.gov.au/}
	\item Conservation Volunteers: \url{http://www.conservationvolunteers.com.au/volunteer}
	\item Volunteering Australia: \url{http://www.volunteeringaustralia.org/html/s01_home/home.asp}
	\item Sponsors for Educational Opportunity (SEO): \vspace{-0.2cm}
		\begin{enumerate} \itemsep -2pt
		\item Philanthropy \& Volunteerism Resources, \url{http://www.seo-usa.org/AlumniResources}
		\item Volunteer Leadership Opportunities: \url{http://www.seo-usa.org/Alumni_Volunteer}
		\end{enumerate}
	\item : \url{}
	\end{enumerate}
\item public health and preventive medicine: \vspace{-0.3cm}
	\begin{enumerate} \itemsep -2pt
	\item U.S. Department of Health \& Human Services: \vspace{-0.2cm}
		\begin{enumerate} \itemsep -2pt
		\item Agency for Healthcare Research and Quality (AHRQ): \vspace{-0.1cm}
			\begin{enumerate} \itemsep -1pt
			\item Prevention \& Care Management: Resources and Materials, \url{http://www.ahrq.gov/clinic/ppipix.htm}
			\end{enumerate}
		\end{enumerate}
	\end{enumerate}
\item career resources: \vspace{-0.3cm}
	\begin{enumerate} \itemsep -2pt
	\item CRAC: The Career Development Organisation: \vspace{-0.2cm}
		\begin{enumerate} \itemsep -2pt
		\item {\it icould}: \vspace{-0.1cm}
			\begin{enumerate} \itemsep -1pt
			\item \url{http://icould.com/about/}
			\item Resource for students, people who are commencing their careers or are making changes in their careers, career counselors, parents, educators, human resource staff, and employers.
			\item icould, {\it Stories by Life Theme}, in icould: Watch Career Stories. Available online at: \url{http://icould.com/watch-career-stories/by-life-theme/}; last accessed on December 25, 2010. [ Has articles briefly describing how people pursued their career goals or their career paths as they went through different experiences in life. This includes people who ``blossomed after school,'' changed careers or became entrepreneurs, had no plans, took risks, encountered turning points, faced adversity, have disabilities, went through financial hardship, or got laid off. It also has stories of people who volunteered, took a gap year, or pursued internships. ]
			\item icould, {\it Stories by Job Type}, in icould: Watch Career Stories. Available online at: \url{http://icould.com/watch-career-stories/by-job-type/}; last accessed on December 25, 2010. [ Includes stories of people in automotive retail, customer services, engineering, education, and many other job types. ]
			\end{enumerate}
		\end{enumerate}
	\item Jobs for the Future: \vspace{-0.2cm}
		\begin{enumerate} \itemsep -2pt
		\item \url{http://www.jff.org/}
		\item Current Projects: \url{http://www.jff.org/projects/current}
		\item Publications: \url{http://www.jff.org/publications}
		\item Policy: \url{http://www.jff.org/policy}
		\item Funders (funding agencies/organizations): \url{http://www.jff.org/funders}
		\item Programs: \url{http://www.jff.org/index.php?select=work}
		\end{enumerate}
	\item SkillsUSA: \vspace{-0.2cm}
		\begin{enumerate} \itemsep -2pt
		\item ``SkillsUSA is a partnership of students, teachers and industry working together to ensure America has a skilled work force. SkillsUSA helps each student excel.''
		\item Educators: \vspace{-0.1cm}
			\begin{enumerate} \itemsep -1pt
			\item \url{http://www.skillsusa.org/educators/index.shtml}
			\item Programs and Curricula: \url{http://www.skillsusa.org/educators/programs.shtml}
			\end{enumerate}
		\item Students: \vspace{-0.1cm}
			\begin{enumerate} \itemsep -1pt
			\item \url{http://www.skillsusa.org/students/index.shtml}
			\item Scholarships \& Financial Aid--SkillsUSA-related Scholarships: \url{http://www.skillsusa.org/students/scholarships.shtml}
			\end{enumerate}
		\item SkillsUSA competitions: \url{http://www.skillsusa.org/compete/index.shtml}
		\end{enumerate}
	\item others: \vspace{-0.2cm}
		\begin{enumerate} \itemsep -2pt
		\item public speaking and leadership: \vspace{-0.1cm}
			\begin{enumerate} \itemsep -1pt
			\item {\it Toastmasters International} is a non-profit educational organization that teaches public speaking and leadership skills through a worldwide network of meeting locations. Available online at: \url{http://www.toastmasters.org/}; last accessed on January 7, 2010.
			\end{enumerate}
		\end{enumerate}
	\end{enumerate}
\end{enumerate}




%%%%%%%%%%%%%%%%%%%%%%%%%%%%%%%%%%%%%%%%%%%
\section{Youth Outreach}
\label{youthoutreach}

Resources for youth outreach: \vspace{-0.3cm}
\begin{enumerate} \itemsep -4pt
%%%%%%%%%%%%%%%%%%%%%%%
\item educational (computer) games: \vspace{-0.3cm}
	\begin{enumerate} \itemsep -2pt
	\item Chevron Corporation: \vspace{-0.2cm}
		\begin{enumerate} \itemsep -2pt
		\item Energyville (about issues concerning energy and the environment): \url{http://www.willyoujoinus.com/energyville/}
		\end{enumerate}
	\item {\it Lego Digital Designer (LDD)}: \vspace{-0.2cm}
		\begin{enumerate} \itemsep -2pt
		\item CAD software for building Lego toys on Windows and Mac OS X platforms
		\item Free software, as in free beer
		\item \url{http://designbyme.lego.com/en-us/Default.aspx} and \url{http://ldd.lego.com/}
		\end{enumerate}
	\item Robocode: \vspace{-0.2cm}
		\begin{enumerate} \itemsep -2pt
		\item \url{http://en.wikipedia.org/wiki/Robocode} and \url{http://robocode.sourceforge.net/}
		\item Learn how to develop computer programs that will control a robot
		\end{enumerate}
	\item {\it Skill-Life}: \vspace{-0.2cm}
		\begin{enumerate} \itemsep -2pt
		\item \url{http://skill-life.com/}
		\item Use online games to teach youth life skills concerning financial literacy, nutrition, and citizenship.
		\end{enumerate}
	\item PowerUp (IBM with TryScience/New York Hall of Science): \vspace{-0.2cm}
		\begin{enumerate} \itemsep -2pt
		\item \url{http://www.powerupthegame.org/}
		\item Computer game to teach youths about energy conservation, global warming, renewable energy, and sustainable engineering
		\end{enumerate}
	\item EnergyNet: \vspace{-0.2cm}
		\begin{enumerate} \itemsep -2pt
		\item \url{http://www.energynet.net/games/}
		\item Computer game to teach youths about energy efficiency, and other topics related to energy
		\end{enumerate}
	\end{enumerate}
%%%%%%%%%%%%%%%%%%%%%%%
\item summer camps: \vspace{-0.3cm}
	\begin{enumerate} \itemsep -2pt
	\item United States Naval Academy: \vspace{-0.2cm}
		\begin{enumerate} \itemsep -2pt
		\item Naval Academy Athletic Association: \vspace{-0.1cm}
			\begin{enumerate} \itemsep -1pt
			\item Sports camps: \url{http://www.navysports.com/camps/navy-camps.html}
			\end{enumerate}
		\end{enumerate}
	\end{enumerate}
%%%%%%%%%%%%%%%%%%%%%%%
\item competitions for youths: \vspace{-0.3cm}
	\begin{enumerate} \itemsep -2pt
	\item International Geography Olympiad (for high school students): \url{http://www.geoolympiad.org/}
	\item International Linguistic Olympiad (for high school students): \url{http://en.wikipedia.org/wiki/International_Linguistics_Olympiad}
	\item International Philosophy Olympiad (for high school students): \url{http://www.philosophy-olympiad.org/}
	\item JA Worldwide: Responsible People Business Competition (for students in North and South America, and Europe), \url{http://www.responsible-business.org/}
	\item The Choral Arts Society of Washington: \vspace{-0.2cm}
		\begin{enumerate} \itemsep -2pt
		\item \url{http://www.choralarts.org/MLK-Celebration-Community-Initiative/Writing-Competition.aspx}
		\item ``As part of our MLK Celebration Community Initiative and in celebration of Black History Month, The Choral Arts Society of Washington hosts an annual writing competition for students in grades K-12.''
		\item ``Each year, students are presented with a different writing prompt and are asked to respond in poetic form.''
		\item ``Students are encouraged to be creative in their writing and to use their knowledge of Martin Luther King, Jr.'s life, the Civil Rights Movement, and current events as inspiration for their writing.''
		\end{enumerate}
	\item Vocal Arts DC (or Vocal Arts Society): \vspace{-0.2cm}
		\begin{enumerate} \itemsep -2pt
		\item Young Artists Competition: \vspace{-0.1cm}
			\begin{enumerate} \itemsep -1pt
			\item \url{http://vocalartsdc.org/youngartists.shtml}
			\item ``Each year, Vocal Arts DC holds a vocal competition open to all singers who are residents of the greater DC area, including Baltimore and Annapolis.''
			\item ``Singers are asked to submit a CD for review along with a sample recital program that the singer is prepared to sing in recital. The CDs will be reviewed in a blind audition and finalist will be selected for live auditions.''
			\item ``Two winners are selected from the finalists and are presented in the Art Song Discovery Series in four different venues across the greater DC area.''
			\end{enumerate}
		\end{enumerate}
	\item The John F. Kennedy Center for the Performing Arts: \vspace{-0.2cm}
		\begin{enumerate} \itemsep -2pt
		\item The National Symphony Orchestra (NSO): \vspace{-0.1cm}
			\begin{enumerate} \itemsep -1pt
			\item Young Soloists' Competition (High School Division; Washington metropolitan area): \url{http://www.kennedy-center.org/nso/nsoed/youngsoloists.cfm#concerts}
			\end{enumerate}
		\end{enumerate}
	\item Center for Interactive Learning and Collaboration (CILC): \vspace{-0.2cm}
		\begin{enumerate} \itemsep -2pt
		\item Kids Creating Community Content KC$^{3}$ International Contest (for students in Middle and High School): \vspace{-0.1cm}
			\begin{enumerate} \itemsep -1pt
			\item \url{http://kc3.cilc.org/} and \url{http://kc3.cilc.org/guidelines.htm}
			\item Make a short film to educate others about the uniqueness of your community, geographical region, natural/agricultural resources, local/national treasures, culture/heritage, or country.
			\end{enumerate}
		\end{enumerate}
	\end{enumerate}
%%%%%%%%%%%%%%%%%%%%%%%
\item educational resources: \vspace{-0.3cm}
	\begin{itemize} \itemsep -2pt
	\item Xcel Energy Foundation: \vspace{-0.2cm}
		\begin{enumerate} \itemsep -2pt
		\item Focus Area Grants: \vspace{-0.1cm}
			\begin{enumerate} \itemsep -1pt
			\item \url{http://www.xcelenergy.com/Minnesota/Company/Community/Xcel%20Energy%20Foundation/Pages/Focus_Area_Grants.aspx}
			\item Scope of eligible funding, and details on the grant application process
			\end{enumerate}
		\item Education Initiatives: \vspace{-0.1cm}
			\begin{enumerate} \itemsep -1pt
			\item \url{http://www.xcelenergy.com/Minnesota/Company/Community/Education%20Initiatives/Pages/Education_Initiatives.aspx}
			\item Energy Safety Calendar Program, K-6: \vspace{-0.1cm}
				\begin{itemize} \itemsep -1pt
				\item \url{http://www.xcelenergy.com/New%20Mexico/Company/Community/Education%20Initiatives/Pages/Energy_Safety_Calendar_ProgramK-6.aspx}
				\item ``The Energy Safety Calendar Program offers K-6 students in our service territory a great opportunity to learn about electricity and natural gas safety.''
				\end{itemize}
			\end{enumerate}
		\item Safety World: \vspace{-0.1cm}
			\begin{enumerate} \itemsep -1pt
			\item \url{http://www.xcelenergy.com/New%20Mexico/Company/Community/Education%20Initiatives/Pages/Safety_World.aspx}
			\item e-SMART kid: \vspace{-0.1cm}
				\begin{itemize} \itemsep -1pt
				\item \url{http://www.e-smartonline.net/xcelenergy/}
				\item Help children and youth learn about ``electricity and natural gas and how to use them safely''
				\end{itemize}
			\end{enumerate}
		\item Energy Classroom: \vspace{-0.1cm}
			\begin{enumerate} \itemsep -1pt
			\item \url{http://www.energyclassroom.com/}
			\item \url{http://www.xcelenergy.com/Minnesota/Company/Community/Pages/Energy_Classroom.aspx}
			\item Educational material for students about energy sources, energy conservation, and environmental protection
			\item For Teachers (educational material and suggested class activities): \url{http://www.energyclassroom.com/index.php?id=34&page=For_Teachers}
			\end{enumerate}
		\item Power Plant Tour Information: \url{http://www.xcelenergy.com/New%20Mexico/Company/About_Energy_and_Rates/Power%20Generation/Pages/Power_Plant_Tour_Information.aspx}
		\end{enumerate}
	\item HowStuffWorks, Inc.: \url{http://www.howstuffworks.com/}
	\item Chevron Corporation: \vspace{-0.2cm}
		\begin{enumerate} \itemsep -2pt
		\item {\it Will you join us}: \vspace{-0.1cm}
			\begin{enumerate} \itemsep -1pt
			\item Energy issues: \url{http://www.willyoujoinus.com/energyissues/}
			\item Tools and resources: \vspace{-0.1cm}
				\begin{itemize} \itemsep -1pt
				\item \url{http://www.willyoujoinus.com/toolsresources/}
				\item Helpful links (includes K-12 educational material): \url{http://www.willyoujoinus.com/toolsresources/helpfullinks/}
				\end{itemize}
			\item MPG Optimizer: \url{http://www.willyoujoinus.com/usingenergywisely/mpgoptimizer/}
			\item Energy generator: \url{http://www.willyoujoinus.com/usingenergywisely/energygenerator/}
			\end{enumerate}
		\end{enumerate}
	\item National Energy Foundation: \vspace{-0.2cm}
		\begin{enumerate} \itemsep -2pt
		\item \url{http://www.nef.org.uk/} and \url{http://www.nef1.org/}
		\item Students: \url{http://www.nef1.org/students.html}
		\item Educators: \url{http://www.nef1.org/educators.html}
		\item Schools: \vspace{-0.1cm}
			\begin{enumerate} \itemsep -1pt
			\item \url{http://www.nef.org.uk/communities/schools/index.html}
			\item Helpful links: \url{http://www.nef.org.uk/communities/schools/energylinks.html}
			\item School Resources: \url{http://www.nef.org.uk/communities/schools/resources/index.html}
			\item {\it LogiCity} is a fun interactive computer game with a difference. It's a game set in a 3D virtual city with five main activities where you are set the task of reducing the carbon footprint of an average resident. See \url{http://www.nef.org.uk/communities/schools/logicity.html}.
			\end{enumerate}
		\item Resources: \url{http://www.nef.org.uk/actonCO2/index.asp}
		\item Igniting Creative Energy - A National Student Challenge: \vspace{-0.1cm}
			\begin{enumerate} \itemsep -1pt
			\item \url{http://www.ignitingcreativeenergy.org/}
			\item Students: \url{http://www.ignitingcreativeenergy.org/students.html}
			\end{enumerate}
		\end{enumerate}
	\item StartSpot Mediaworks: \vspace{-0.2cm}
		\begin{enumerate} \itemsep -2pt
		\item StartSpot Network: \vspace{-0.1cm}
			\begin{enumerate} \itemsep -1pt
			\item HomeworkSpot: \vspace{-0.1cm}
				\begin{itemize} \itemsep -1pt
				\item \url{http://www.homeworkspot.com/}
				\item Science Fair Project Center: \url{http://www.homeworkspot.com/sciencefair/}
				\end{itemize}
			\end{enumerate}
		\end{enumerate}
	\item Super Science Fair Projects: \url{http://www.super-science-fair-projects.com/}
	\item All Science Fair Projects: Science Fair Projects with Complete Instructions, \url{http://www.all-science-fair-projects.com/}
	\item The Science Club: \vspace{-0.2cm}
		\begin{enumerate} \itemsep -2pt
		\item \url{http://scienceclub.org/}
		\item Science Fair Idea Exchange: \url{http://scienceclub.org/scifair.html}
		\end{enumerate}
	\item Oracle Education Foundation: \vspace{-0.2cm}
		\begin{enumerate} \itemsep -2pt
		\item \url{http://www.oraclefoundation.org/}
		\item ThinkQuest: \vspace{-0.1cm}
			\begin{enumerate} \itemsep -1pt
			\item \url{http://www.thinkquest.org/en/}
			\item ThinkQuest International Competition: \url{http://www.thinkquest.org/competition/}
			\item Projects: \url{http://thinkquest.org/en/projects/index.html}
			\item Library: \url{http://thinkquest.org/pls/html/think.library}
			\item Example of a computer game developed by students: Crisis! - The Game, \url{http://library.thinkquest.org/20331/game/}
			\end{enumerate}
		\end{enumerate}
	\item University of Minnesota: \vspace{-0.2cm}
		\begin{enumerate} \itemsep -2pt
		\item Institute on Community Integration; College of Education and Human Development: \vspace{-0.1cm}
			\begin{enumerate} \itemsep -1pt
			\item National Center on Secondary Education and Transition (NCSET): \vspace{-0.1cm}
				\begin{itemize} \itemsep -1pt
				\item \url{http://www.ncset.org/}
				\item NCSET Topics: \url{http://www.ncset.org/topics/default.asp}
				\item Web Sites: \url{http://www.ncset.org/websites/default.asp}
				\item The Youthhood!: \url{http://www.youthhood.org/}
				\end{itemize}
			\end{enumerate}
		\end{enumerate}
	\item Jobs for America's Graduates: \vspace{-0.2cm}
		\begin{enumerate} \itemsep -2pt
		\item \url{http://www.jag.org/}
		\item JAG Model program applications: \vspace{-0.1cm}
			\begin{enumerate} \itemsep -1pt
			\item \url{http://www.jag.org/model.htm}
			\item Programs are available for students in middle school and high school, high school dropouts, high school seniors, students in alternative education programs, and college underclassmen
			\end{enumerate}
		\item JAG Career Corner: \url{http://www.jag.org/jag_career_corner.htm}
		\item Students: \url{http://www.jag.org/students.htm}
		\item Resource library: \url{http://www.jag.org/library.htm}
		\item Performance outcomes: \url{http://www.jag.org/outcomes.htm}
		\item Funding: \url{http://www.jag.org/funding.htm}
		\end{enumerate}
	\item Alliance to Save Energy: \vspace{-0.2cm}
		\begin{enumerate} \itemsep -2pt
		\item Energy Hog campaign: \vspace{-0.1cm}
			\begin{enumerate} \itemsep -1pt
			\item \url{http://www.energyhog.org/}
			\item Adults: \url{http://www.energyhog.org/adult/adults.htm}
			\item Children: \url{http://www.energyhog.org/childrens.htm}
			\end{enumerate}
		\end{enumerate}
	\item Learning First Alliance: \vspace{-0.2cm}
		\begin{enumerate} \itemsep -2pt
		\item \url{http://www.learningfirst.org/}
		\item Issues and publications: \url{http://www.learningfirst.org/issues}
		\item Resources: \url{http://www.learningfirst.org/resources}
		\end{enumerate}
	\item NaMaYa: \url{http://www.namaya.com/}
	\item NIXTY: \url{http://nixty.com/}
	\item K12 Open Ed: \url{http://www.k12opened.com/wiki/index.php/Main_Page}
	\item Learning Is For Everyone: \url{http://www.learningis4everyone.org/}
	\item The Smithsonian Commons Prototype: \url{http://www.si.edu/commons/prototype/}
	\item Futurelab: Resources for educators and parents, \url{http://www.futurelab.org.uk/resources}
	\item Innosight Institute: Resources for education, \url{http://www.innosightinstitute.org/practices/education/}
	\item WGBH Educational Foundation: \url{http://www.wgbh.org/}
	\item Discovery Education: \vspace{-0.2cm}
		\begin{enumerate} \itemsep -2pt
		\item Classroom resources: \url{http://school.discoveryeducation.com/}
		\item Home resources: \url{http://school.discoveryeducation.com/homeworkhelp/homework_help_home.html}
		\end{enumerate}
	\item The Gilder Lehrman Institute of American History: \vspace{-0.2cm}
		\begin{enumerate} \itemsep -2pt
		\item \url{http://www.gilderlehrman.org/}
		\item Resources for teachers and schools: \url{http://www.gilderlehrman.org/teachers/}
		\item Civil War Essay Contest (for students in selected K-12 schools): \url{http://www.gilderlehrman.org/affiliate/civil_war.php}
		\end{enumerate}
	\item The GRAMMY Museum: \vspace{-0.2cm}
		\begin{enumerate} \itemsep -2pt
		\item Teacher curriculum and resources. Available online at: \url{http://www.grammymuseum.org/interior.php?section=education&page=teachercurriculum}; last accessed on November 15, 2010.
		\end{enumerate}
	\item Purdue University: \vspace{-0.2cm}
		\begin{enumerate} \itemsep -2pt
		\item Department of Entomology: \vspace{-0.1cm}
			\begin{enumerate} \itemsep -1pt
			\item Genomics Analogy Model for Educators (G.A.M.E.): \url{http://www.entm.purdue.edu/extensiongenomics/GAME/default.html}
			\end{enumerate}
		\end{enumerate}
	\item Verizon Thinkfinity: \url{http://www.thinkfinity.org/about-us}
	\item Oregon Virtual School District (ORVSD): \vspace{-0.2cm}
		\begin{enumerate} \itemsep -2pt
		\item \url{http://orvsd.org/}
		\item ``Oregon Virtual School District (ORVSD) helps integrate technology into Oregon public school classrooms by giving teachers access to free tech tools and resources online.''
		\item ``The Oregon Virtual School District is a program led by the Oregon Department of Education that, in cooperation with a consortium of virtual learning providers throughout the state, seeks to increase access and availability of online learning and teaching resources free of charge to public school teachers of Oregon. Oregon State University is providing hosting and development resources through a partnership with the OSU Open Source Lab and the OSU Business Solutions Group.''
		\end{enumerate}
	\item The Association of Educational Publishers (AEP): \vspace{-0.2cm}
		\begin{enumerate} \itemsep -2pt
		\item The AEP Awards: \vspace{-0.1cm}
			\begin{enumerate} \itemsep -1pt
			\item \url{http://www.aepweb.org/awards/index.htm}
			\item Look at the winners of previous AEP awards to determine some of the good educational resources that are available
			\end{enumerate}
		\end{enumerate}
	\item Educational Dividends: \vspace{-0.2cm}
		\begin{enumerate} \itemsep -2pt
		\item \url{http://www.educationaldividends.com/}
		\item Teachers: \vspace{-0.1cm}
			\begin{enumerate} \itemsep -1pt
			\item \url{http://www.educationaldividends.com/index.asp?menu=Teachers}
			\item Teaching Tools: \url{http://www.educationaldividends.com/teachers/tools.asp}
			\item Reference Desk: \vspace{-0.1cm}
				\begin{itemize} \itemsep -1pt
				\item \url{http://www.educationaldividends.com/teachers/reference.asp}
				\item Standards Reference Desk (resources for education standards in the US at the national, state, and local levels): \url{http://www.educationaldividends.com/teachers/standards_desk.asp}
				\item How We Learn: Learning Styles, \url{http://www.educationaldividends.com/teachers/learning_styles.asp}
				\item How We Learn: Multiple Intelligences, \url{http://www.educationaldividends.com/teachers/multiple_intelligences.asp}
				\item Statistics Desk (statistical information about education in the US): \url{http://www.educationaldividends.com/teachers/statistics_desk.asp}
				\end{itemize}
			\item Information about the teaching profession: \vspace{-0.1cm}
				\begin{itemize} \itemsep -1pt
				\item \url{http://www.educationaldividends.com/teachers/welcome.asp}
				\item Office of Occupational Statistics and Employment Projections, ``Educational Services,'' in {\it Career Guide to Industries}, 2010-11 Edition, U.S. Bureau of Labor Statistics, U.S. Department of Labor, Washington, DC, December 17, 2009. Available online at: \url{http://stats.bls.gov/oco/cg/cgs034.htm}; last accessed on December 8, 2010. [ Suggested citation: Bureau of Labor Statistics, U.S. Department of Labor, {\it Career Guide to Industries, 2010-11 Edition}, Educational Services , on the Internet at \url{http://www.bls.gov/oco/cg/cgs034.htm} (visited December 07, 2010). ]
				\item Experience Teaching: \url{http://www.educationaldividends.com/teachers/experience.asp}
				\item Continuous Improvement: \url{http://www.educationaldividends.com/teachers/toolkit.asp}
				\end{itemize}
			\end{enumerate}
		\item Personality and Career Tests: \url{http://www.educationaldividends.com/teachers/tests.asp}
		\end{enumerate}
	\item Smithsonian Institution: \vspace{-0.2cm}
		\begin{enumerate} \itemsep -2pt
		\item Educators: \url{http://www.si.edu/Educators}
		\item Smithsonian Institution Traveling Exhibition Service (SITES): \vspace{-0.1cm}
			\begin{enumerate} \itemsep -1pt
			\item For Teachers: \url{http://www.sites.si.edu/education/teachers_res2.htm}
			\end{enumerate}
		\item Smithsonian Folkways Recordings (or simply, Smithsonian Folkways): \vspace{-0.1cm}
			\begin{enumerate} \itemsep -1pt
			\item Tools for Teaching: \url{http://www.folkways.si.edu/tools_for_teaching/introduction.aspx}
			\end{enumerate}
		\item Freer Gallery of Art / Arthur M. Sackler Gallery: \vspace{-0.1cm}
			\begin{enumerate} \itemsep -1pt
			\item Resources for Educators: \url{http://www.asia.si.edu/explore/teacherResources.asp}
			\item Explore + Learn: Browse Online Resources by Area: \vspace{-0.1cm}
				\begin{itemize} \itemsep -1pt
				\item \url{http://www.asia.si.edu/explore/default.asp}
				\item Has resources for art in: \vspace{-0.1cm}
					\begin{itemize} \itemsep -1pt
					\item The Americas
					\item Ancient Egypt
					\item Ancient Near East
					\item Islamic world
					\item China
					\item Japan
					\item Korea
					\item South Asia
					\item Himalayas
					\item Southeast Asian
					\item It also has biblical manuscripts and contemporary art
					\end{itemize}
				\end{itemize}
			\item Online Exhibition Features: \url{http://www.asia.si.edu/exhibitions/online.asp}
			\item Collections: \url{http://www.asia.si.edu/collections/default.asp}
			\end{enumerate}
		\item National Museum of American History: \vspace{-0.1cm}
			\begin{enumerate} \itemsep -1pt
			\item Jerome and Dorothy Lemelson Center for the Study of Invention and Innovation: \vspace{-0.1cm}
				\begin{itemize} \itemsep -1pt
				\item Resources: \vspace{-0.1cm}
					\begin{itemize} \itemsep -1pt
					\item \url{http://invention.smithsonian.org/resources/}
					\item \url{http://invention.smithsonian.org/resources/default_sites_weblinks.aspx}
					\item Invention stories - archives, articles, audio, and video: \url{http://invention.smithsonian.org/resources/default_index.aspx}
					\end{itemize}
				\item Educational Materials: \vspace{-0.1cm}
					\begin{itemize} \itemsep -1pt
					\item \url{http://invention.smithsonian.org/resources/menu_edu_materials.aspx}
					\item Experiments: \url{http://invention.smithsonian.org/resources/menu_edu_materials.aspx?MaterialTypeID=3&MaterialTypeDesc=Experiments}
					\item Educational Materials: \url{http://invention.smithsonian.org/resources/menu_edu_materials_f.aspx?MaterialTypeDesc=Features}
					\end{itemize}
				\item Centerpieces: \vspace{-0.1cm}
					\begin{itemize} \itemsep -1pt
					\item \url{http://invention.smithsonian.org/centerpieces/}
					\item \url{http://invention.smithsonian.org/centerpieces/iap-info.aspx}
					\item Electric guitar: \url{http://invention.smithsonian.org/centerpieces/electricguitar/index.htm}
					\item Innovative Lives: \url{http://invention.smithsonian.org/centerpieces/ilives/}
					\item ``Exploring the History of Women Inventors'' by J.E. Bedi (in {\it Innovative Lives}): \url{http://invention.smithsonian.org/centerpieces/ilives/womeninventors.html}
					\item Whole Cloth: \url{http://invention.smithsonian.org/centerpieces/whole_cloth/index.html}
					\item The Quartz Watch: \url{http://invention.smithsonian.org/centerpieces/quartz/index.html}
					\item Edison Invents!: All about Thomas Edison and his invention, \url{http://invention.smithsonian.org/centerpieces/edison/default.asp}
					\end{itemize}
				\item Modern Inventors Documentation Program (MIND): \url{http://invention.smithsonian.org/resources/mind_resources.aspx}
				\item Invention at Play: \vspace{-0.1cm}
					\begin{itemize} \itemsep -1pt
					\item \url{http://inventionatplay.org/}
					\item Resources: \url{http://inventionatplay.org/resources.html}
					\item Invention Playhouse: \url{http://inventionatplay.org/playhouse_main.html}
					\item Inventors' Stories: \url{http://inventionatplay.org/inventors_main.html}
					\item Does play matter? (using play to help children learn and think): \url{http://inventionatplay.org/matter_main.html}
					\end{itemize}
				\item Spark!Lab: \vspace{-0.1cm}
					\begin{itemize} \itemsep -1pt
					\item \url{http://sparklab.si.edu/}
					\item About Spark!Lab (introduce children to the process of innovation via play and fun activities): \url{http://sparklab.si.edu/spark-about.html}
					\item Activities \& Experiments: \url{http://sparklab.si.edu/spark-experiments.html}
					\item Inventor Profiles: \url{http://sparklab.si.edu/spark-inventors.html}
					\item Resources: \url{http://sparklab.si.edu/spark-resources.html}
					\end{itemize}
				\end{itemize}
			\end{enumerate}
		\end{enumerate}
	\item Economic and Social Research Council (ESRC): \vspace{-0.2cm}
		\begin{enumerate} \itemsep -2pt
		\item {\it Social Science for Schools}; Science in Society Team: \vspace{-0.1cm}
			\begin{enumerate} \itemsep -1pt
			\item \url{http://www.esrcsocietytoday.ac.uk/ESRCInfoCentre/ssfs/}
			\item Social science resources: \url{http://www.esrcsocietytoday.ac.uk/ESRCInfoCentre/ssfs/resources/}
			\item Career guides for different disciplines in social science and economics: \url{http://www.esrcsocietytoday.ac.uk/ESRCInfoCentre/ssfs/careers/}
			\item Related online resources: \url{http://www.esrcsocietytoday.ac.uk/ESRCInfoCentre/ssfs/links/}
			\end{enumerate}
		\end{enumerate}
	\end{itemize}
\item National Council for Accreditation of Teacher Education (NCATE): \vspace{-0.3cm}
	\begin{enumerate} \itemsep -2pt
	\item \url{http://www.ncate.org/}
	\item Has resources about degree programs in education and their accreditation, as well as how to become a teacher
	\item State-specific Recognized Programs by NCATE and Specialized Professional Associations (SPAs): \vspace{-0.2cm}
		\begin{enumerate} \itemsep -2pt
		\item \url{http://www.ncate.org/tabid/165/Default.aspx}
		\item Find out about educational programs in: \vspace{-0.1cm}
			\begin{enumerate} \itemsep -1pt
			\item special education
			\item early childhood education
			\item educational leadership
			\item educational technology specialist
			\item elementary education
			\item English
			\item health education
			\item foreign languages
			\item gifted education
			\item mathematics
			\item physical education
			\item science education
			\item school psychology
			\item secondary computer science education
			\item social studies
			\item Teachers of English to Speakers of Other Languages (TESOL)
			\item technology and engineering educators
			\end{enumerate}
		\end{enumerate}
	\item Financial Aid Resources for Teacher Education Students: \url{http://www.ncate.org/Public/CurrentFutureTeachers/FinancialAidResources/tabid/351/Default.aspx}
	\end{enumerate}
%%%%%%%%%%%%%%%%%%%%%%%
\item scholarships: \vspace{-0.3cm}
	\begin{enumerate} \itemsep -2pt
	\item U.S. Department of State: \vspace{-0.2cm}
		\begin{enumerate} \itemsep -2pt
		\item Bureau of Educational and Cultural Affairs: \vspace{-0.1cm}
			\begin{enumerate} \itemsep -1pt
			\item National Security Language Initiative for Youth (NSLI-Y): \vspace{-0.1cm}
				\begin{itemize} \itemsep -1pt
				\item \url{http://exchanges.state.gov/youth/programs/nsli.html}
				\item ``The State Department�s National Security Language Initiative for Youth (NSLI-Y) provides merit-based scholarships to U.S. high school students and recent graduates interested in learning less-commonly studied foreign languages.''
				\end{itemize}
			\end{enumerate}
		\end{enumerate}
	\end{enumerate}
%%%%%%%%%%%%%%%%%%%%%%%
\item underrepresented minorities: \vspace{-0.3cm}
	\begin{enumerate} \itemsep -2pt
	\item The University of North Carolina at Chapel Hill: \vspace{-0.2cm}
		\begin{enumerate} \itemsep -2pt
		\item Gary Bishop, {\it Research}, Department of Computer Science, The University of North Carolina at Chapel Hill. Available at: \url{http://wwwx.cs.unc.edu/~gb/wp/research/}; last accessed on September 3, 2010. [ Has plenty of information and resources (including learning aids and material) to help people who are visually or mobility impaired learn. ]
		\end{enumerate}
	\item Myra Sadker Foundation: \vspace{-0.2cm}
		\begin{enumerate} \itemsep -2pt
		\item $100+$ Ideas to Promote Gender Equity in Schools and Beyond: \url{http://www.sadker.org/100ideas.html}
		\item Gender Equity Activities: \url{http://www.sadker.org/WhatYouCanDo.html}
		\item Gender Equity Activities for Concerned Citizens: \url{http://www.sadker.org/GenderEquity-citizens.html}
		\item Gender Equity Activities for Families: \url{http://www.sadker.org/GenderEquity-family.html}
		\item Gender Equity Activities for Teachers: \vspace{-0.1cm}
			\begin{enumerate} \itemsep -1pt
			\item Early Childhood: \url{http://www.sadker.org/GenderEquity-teacher1.html}
			\item Primary Grades: \url{http://www.sadker.org/GenderEquity-teacher2.html}
			\item Upper Elementary: \url{http://www.sadker.org/GenderEquity-teacher3.html}
			\item Middle and High School: \url{http://www.sadker.org/GenderEquity-teacher4.html}
			\end{enumerate}
		\item Resources for feminism and links to web pages of feminist organizations: \url{http://www.sadker.org/ReadsLinks.html}
		\end{enumerate}
	\item League of United Latin American Citizens (LULAC): \vspace{-0.3cm}
		\begin{enumerate} \itemsep -2pt
		\item LULAC National Educational Service Centers, Inc: \vspace{-0.2cm}
			\begin{enumerate} \itemsep -2pt
			\item \url{http://www.lnesc.org/}
			\item Programs: \vspace{-0.1cm}
				\begin{itemize} \itemsep -1pt
				\item Improving literacy among Latino/Latina youth
				\item Encouraging Latino/Latina youth to pursue careers in science and engineering
				\item Helping Latino/Latina youth acquire leadership skills
				\item Improving college access for Latino/Latina youth by mentoring and summer programs (e.g., Gear-Up, Upward Bound, and the PALMS Initiative)
				\item Helping Latino/Latina families acquire financial success, so that Latino/Latina youth can pursue higher education
				\item Scholarships for Latino/Latina youth
				\item \url{http://lnesc.org/index.asp?Type=B_BASIC&SEC={808B6D04-913C-483F-8A05-5BD44B03ED62}}
				\end{itemize}
			\end{enumerate}
		\end{enumerate}
	\item ASPIRA: \vspace{-0.2cm}
		\begin{enumerate} \itemsep -2pt
		\item ASPIRA Programs for Latino/Latina youth: \url{http://aspira.org/manuals/aspira-programs}
		\end{enumerate}
	\end{enumerate}
%%%%%%%%%%%%%%%%%%%%%%%
\item places to visit: \vspace{-0.3cm}
	\begin{enumerate} \itemsep -2pt
	\item Exploratorium @ The Palace of Fine Arts (San Francisco, CA): \url{http://www.exploratorium.edu/}
	\item Educational Dividends: \vspace{-0.2cm}
		\begin{enumerate} \itemsep -2pt
		\item \url{http://www.educationaldividends.com/}
		\item Suggestions for organizing field trips to explore your interests: \url{http://www.educationaldividends.com/students/student_issues.asp}
		\item Career exploration: \url{http://www.educationaldividends.com/students/career_choices.asp}
		\item Computer skills: \url{http://www.educationaldividends.com/students/technology.asp}
		\item Quizzes to help you find out what is your preferred learning style and to discover more about your personality: \url{http://www.educationaldividends.com/students/learning_quiz.asp}
		\item Resources to help you learn about various topics in science, mathematics, social science, and humanities: \url{http://www.educationaldividends.com/students/resources.asp}
		\end{enumerate}
	\end{enumerate}
%%%%%%%%%%%%%%%%%%%%%%%
\item resources for at-risk youths: \vspace{-0.3cm}
	\begin{enumerate} \itemsep -2pt
	\item At-Risk Youth: \url{http://www.at-risk.org/}
	\item Peace First: \vspace{-0.2cm}
		\begin{enumerate} \itemsep -2pt
		\item \url{http://www.peacefirst.org/site/}
		\item To help youths become ``problem-solvers, rather than witnesses, or victims of their surrounding''
		\item To reduce youth homicide rates
		\item Teach children ``critical conflict resolution skills''
		\item Help teachers improve their ``conflict resolution and classroom management skills''
		\item To encourage youths to help each other, and get them to break up fights
		\item ``The Peace First curriculum is tailored to meet the developmental needs of students in Pre-K through eighth grade. Once a week, young adult volunteers and classroom teachers work together to teach students about friendship, communication, and conflict resolution through the use of experiential activities. First graders learn about communicating their feelings, third graders work on being peacemakers in their classroom, and fifth graders explore how to resolve and deescalate conflicts.''
		\item Has programs for students/youths, teachers, principals, and volunteers.
		\end{enumerate}
	\item Americans for the Arts: \vspace{-0.2cm}
		\begin{enumerate} \itemsep -2pt
		\item YouthARTS: \vspace{-0.1cm}
			\begin{enumerate} \itemsep -1pt
			\item \url{http://www.artsusa.org/youtharts/index.asp}
			\item ``The YouthARTS site is designed to give arts agencies, juvenile justice agencies, social service organizations, and other community-based organizations detailed information about how to plan, run, provide training, and evaluate arts programs for at-risk youth.''
			\end{enumerate}
		\end{enumerate}
	\end{enumerate}
%%%%%%%%%%%%%%%%%%%%%%%
\item general music and arts education: \vspace{-0.3cm}
	\begin{enumerate} \itemsep -2pt
	\item Americans for the Arts: \vspace{-0.2cm}
		\begin{enumerate} \itemsep -2pt
		\item Americans for the Arts, ``Ten Simple Ways Parents Can Get More Art in Their Kids' Lives.'' Available online at: \url{http://www.americansforthearts.org/public_awareness/get_involved/001.asp}; last accessed on November 30, 2010.
		\item YouthARTS: \vspace{-0.1cm}
			\begin{enumerate} \itemsep -1pt
			\item \url{http://www.artsusa.org/youtharts/index.asp}
			\item ``The YouthARTS site is designed to give arts agencies, juvenile justice agencies, social service organizations, and other community-based organizations detailed information about how to plan, run, provide training, and evaluate arts programs for at-risk youth.''
			\end{enumerate}
		\end{enumerate}
	\item The John F. Kennedy Center for the Performing Arts: \vspace{-0.2cm}
		\begin{enumerate} \itemsep -2pt
		\item Kennedy Center Institute for Arts Management: \url{http://artsmanagerfba.artsmanager.org/common/Pages/About.aspx}
		\item {\sc ArtsEdge}: \vspace{-0.1cm}
			\begin{enumerate} \itemsep -1pt
			\item The National Standards for Arts Education for Grades K-4, 5-8, and 9-12: \url{http://artsedge.kennedy-center.org/educators/standards.aspx}
			\item Tips and guides for educators: \url{http://artsedge.kennedy-center.org/educators/how-to.aspx}
			\item Lesson plans for educators: \url{http://artsedge.kennedy-center.org/educators/lessons.aspx}
			\item Information for parents, guardians, foster parents, baby-sitters, and grandparents: \url{http://artsedge.kennedy-center.org/families.aspx}
			\item Information for students: \url{http://artsedge.kennedy-center.org/students.aspx}
			\item Themes for artistic, cultural, academic, and intellectual exploration: \url{http://artsedge.kennedy-center.org/themes.aspx}
			\item Multimedia: \url{http://artsedge.kennedy-center.org/multimedia.aspx}
			\end{enumerate}
		\end{enumerate}
	\end{enumerate}
\item music education: \vspace{-0.3cm}
	\begin{enumerate} \itemsep -2pt
	\item Washington Performing Arts Society (WPAS): \vspace{-0.2cm}
		\begin{enumerate} \itemsep -2pt
		\item WPAS Education \& Community -- Connections through the Arts Education Programs for All Ages: \vspace{-0.1cm}
			\begin{enumerate} \itemsep -1pt
			\item The Capitol Jazz Project: \vspace{-0.1cm}
				\begin{itemize} \itemsep -1pt
				\item \url{http://www.wpas.org/educcomm/programsforyoungpeople/capitoljazzproject.aspx}
				\item ``Washington Performing Arts Society (WPAS) and the D.C. Public Schools, in collaboration with Jazz at Lincoln Center, has launched The Capitol Jazz Project, an important step in supporting music education for all students in the District of Colombia.''
				\item ``Through the Capitol Jazz Project, students hone their listening, performing, improvising, composing, arranging, music reading, and notation skills.''
				\item ``The Capitol Jazz Project is being implemented in 6 D.C. middle schools with a total enrollment of more than 500 music students.''
				\item ``A true collaboration, The Capitol Jazz Project brings the combined resources and expertise of WPAS, Jazz at Lincoln Center, and the D.C. Public Schools to create a model music education program.''
				\end{itemize}
			\item Joseph and Goldie Feder Memorial String Competition: \vspace{-0.1cm}
				\begin{itemize} \itemsep -1pt
				\item \url{http://www.wpas.org/educcomm/programsforyoungpeople/josephandgoldiefedermemorialstringcompetition.aspx}
				\item ``The Feder String Competition inspires and nurtures D.C. area youth in grades 6 through 12 who study violin, viola, cello, and double bass.''
				\item ``Each year, 80 students compete for 30 awards and scholarships.''
				\item ``Held each spring, WPAS awards cash prizes toward private lessons, scholarships for summer study programs, and tickets for top winners and their family members to attend a WPAS concert.''
				\item ``Winners of the competition are also given special performance opportunities such as on the Kennedy Center's Millennium Stage and The Shakespeare Theatre Company's Happenings at the Harman series.''
				\end{itemize}
			\item WPAS Summer Performing Arts Academy summer programs: \vspace{-0.1cm}
				\begin{itemize} \itemsep -1pt
				\item \url{http://www.wpas.org/educcomm/programsforyoungpeople/wpassummerperformingartsacademy.aspx}
				\end{itemize}
			\end{enumerate}
		\end{enumerate}
	\item Young Concert Artists, Inc. \vspace{-0.2cm}
		\begin{enumerate} \itemsep -2pt
		\item Annaliese Soros Educational Residency Program: \url{http://www.yca.org/auditions/}
		\end{enumerate}
	\item The Choral Arts Society of Washington: \vspace{-0.2cm}
		\begin{enumerate} \itemsep -2pt
		\item Classroom Resources: \url{http://www.choralarts.org/Education/Classroom-Resources.aspx}
		\end{enumerate}
	\item League of American Orchestras: \vspace{-0.2cm}
		\begin{enumerate} \itemsep -2pt
		\item Career planning: \vspace{-0.1cm}
			\begin{enumerate} \itemsep -1pt
			\item Resources for pre-college students, college students, and graduate students: \url{http://www.americanorchestras.org/career_center/career_planning.html}
			\item Arts Administration programs: \url{http://www.americanorchestras.org/career_center/arts_admin_programs.html}
			\item Non-profit management, {\bf public policy} and leadership programs: \url{http://www.americanorchestras.org/career_center/resources_non_prof_and.html}
			\end{enumerate}
		\end{enumerate}
	\item The John F. Kennedy Center for the Performing Arts: \vspace{-0.2cm}
		\begin{enumerate} \itemsep -2pt
		\item Betty Carter's Jazz Ahead: \vspace{-0.1cm}
			\begin{enumerate} \itemsep -1pt
			\item \url{http://www.kennedy-center.org/programs/jazz/jazzahead/}
			\item ``Music residency program for young people''
			\item ``The Jazz Ahead program identifies outstanding, emerging jazz artists in their mid-teens to age thirty, and brings them together under the tutelage of experienced artist-instructors who coach and counsel them, helping to polish their performance, composing and arranging skills.''
			\item ``The two week-long residency program includes daily workshops and rehearsals with established jazz artists, and culminate in three concerts on the Kennedy Center Millennium Stage, which will be broadcast live over the internet.''
			\end{enumerate}
		\item The National Symphony Orchestra (NSO): \vspace{-0.1cm}
			\begin{enumerate} \itemsep -1pt
			\item The National Symphony Orchestra's Summer Music Institute (SMI): \vspace{-0.1cm}
				\begin{itemize} \itemsep -1pt
				\item \url{http://www.kennedy-center.org/nso/nsoed/smi/home.cfm}
				\item ``Every summer, approximately 70 students (ages 15-20) from all over the nation meet in Washington, D.C., to attend the National Symphony Orchestra's Summer Music Institute (SMI).''
				\item ``The Institute offers four weeks of private lessons, rehearsals, coaching by National Symphony Orchestra members, classes, and lectures to prepare aspiring musicians for their futures in music.''
				\end{itemize}
			\item Young Associates' Program: \vspace{-0.1cm}
				\begin{itemize} \itemsep -1pt
				\item \url{http://www.kennedy-center.org/nso/nsoed/youngassociates.html}
				\item ``The National Symphony Orchestra (NSO) is sponsoring its Young Associates' Program for high school students in grades 11 and 12 in the Washington, DC, metropolitan area who are interested in pursuing a musical career.''
				\item ``Twenty outstanding instrumentalists (pianists are not included) will be selected to attend rehearsals of the National Symphony Orchestra and take part in seminars with conductors, artists, NSO musicians, and representatives of the arts management field.''
				\item ``Through this program, the Young Associates will acquire an appreciation of the wide range of skills, knowledge, and abilities--managerial as well as musical--that are required to put together a performance by a major symphony orchestra. Selection process is by application.''
				\item ``The core of the program involves attendance at rehearsals of the National Symphony Orchestra at the Kennedy Center and observation of various guest artists. In addition to attending NSO rehearsals, students participate in workshops to explore careers in management, music education, publicity, music library, and other professions that are essential to the life of every successful orchestra.''
				\item ``Students do not play their instruments as part of the program. Students learn through listening, observation, and asking questions of professionals.''
				\end{itemize}
			\end{enumerate}
		\end{enumerate}
	\end{enumerate}
\item dance education: \vspace{-0.3cm}
	\begin{enumerate} \itemsep -2pt
	\item The Washington Ballet: \vspace{-0.2cm}
		\begin{enumerate} \itemsep -2pt
		\item The Washington School of Ballet (TWSB): \vspace{-0.1cm}
			\begin{enumerate} \itemsep -1pt
			\item Summer Intensive program (requires an audition): \url{http://www.washingtonballet.org/the-school/summer-intensive/}
			\end{enumerate}
		\item TWB's EXCEL! scholarship program (for DanceDC students): \vspace{-0.1cm}
			\begin{enumerate} \itemsep -1pt
			\item \url{http://www.washingtonballet.org/community-engagement/default.htm}
			\item \url{http://www.washingtonballet.org/community-engagement/other-programs/}
			\item Also, has need-based scholarships
			\end{enumerate}
		\end{enumerate}
	\item The John F. Kennedy Center for the Performing Arts: \vspace{-0.2cm}
		\begin{enumerate} \itemsep -2pt
		\item Exploring Ballet With Suzanne Farrell: A Three-Week Summer Ballet Intensive for Young Dancers: \vspace{-0.1cm}
			\begin{enumerate} \itemsep -1pt
			\item \url{http://www.kennedy-center.org/education/farrell/}
			\item ``In July and August, students from across the United States and around the world will participate in the eighteenth annual session of the Kennedy Center's ballet training program Exploring Ballet with Suzanne Farrell. The three-week residency for dancers ages 14 to 18 with at least five years of ballet training will be held at the Kennedy Center from August 1 - August 20, 2011.''
			\item ``During the three-week period, students take two ballet technique classes a day, six days a week, with Ms. Farrell. Students also participate in a number of cultural activities to enhance their experience in Washington, D.C., including museum visits, trips to historical landmarks, and attending performances.''
			\end{enumerate}
		\item Dance Theatre of Harlem Residency program: \vspace{-0.1cm}
			\begin{enumerate} \itemsep -1pt
			\item \url{http://www.kennedy-center.org/education/community/programs.html#artistic}
			\item ``Since 1993, the Kennedy Center's Dance Theatre of Harlem Residency program has provided ballet training for male and female students age 8-18 with identified promise in ballet taught by Dance Theatre of Harlem (DTH) instructors or former principal dancers.''
			\item ``Students are selected by audition for a twenty-class series, culminating with a public demonstration and performance on a Kennedy Center main stage.''
			\item ``Classical ballet training is taught in four class levels, from novice to advance.''
			\item ``Students must have at least one year of ballet training to qualify for the program.''
			\end{enumerate}
		\end{enumerate}
	\end{enumerate}
%%%%%%%%%%%%%%%%%%%%%%%
\item JA Worldwide (Junior Achievement): \vspace{-0.3cm}
	\begin{enumerate} \itemsep -2pt
	\item \url{http://www.ja.org/}
	\item Resources for educators: \url{http://www.ja.org/involved/involved_educat.shtml}
	\item Resources for parents: \url{http://www.ja.org/involved/involved_parents.shtml}
	\item Resources for students: \url{http://www.ja.org/involved/involved_students.shtml}
	\end{enumerate}
\item U.S. Department of State: \vspace{-0.3cm}
	\begin{enumerate} \itemsep -2pt
	\item Programs for Americans and non-Americans.
	\item Summer Work Travel - In the summer work travel program: \url{http://exchanges.state.gov/}
	\item Cultural Programs Division: \url{http://exchanges.state.gov/cultural/index.html}
	\item Youth Programs Division: \url{http://exchanges.state.gov/youth/index.html}
	\item EducationUSA: \url{http://educationusa.state.gov/}
	\item International Visitor Leadership Program: \url{http://exchanges.state.gov/ivlp/ivlp.html}
	\item Programs for non-U.S. Citizens: \url{http://exchanges.state.gov/prog-non-us.html}
	\item Programs for U.S. Citizens: \url{http://exchanges.state.gov/prog-us.html}
	\item Resources for Students: \url{http://exchanges.state.gov/student.html}
	\item Bureau of Educational and Cultural Affairs: \vspace{-0.2cm}
		\begin{enumerate} \itemsep -2pt
		\item Future Leaders Exchange (FLEX) Program: \vspace{-0.1cm}
			\begin{enumerate} \itemsep -1pt
			\item \url{http://exchanges.state.gov/youth/programs/flex.html}
			\item ``The Future Leaders Exchange (FLEX) Program gives students (ages 15-17) the chance to live with a host family and attend a U.S. high school for a year.''
			\end{enumerate}
		\item Office of Citizen Exchanges: \vspace{-0.1cm}
			\begin{enumerate} \itemsep -1pt
			\item Youth Programs Division: \vspace{-0.1cm}
				\begin{itemize} \itemsep -1pt
				\item \url{http://exchanges.state.gov/youth/index.html}
				\item Has programs for youths in various parts of the world
				\item ``The Youth Programs Division is committed to empowering the next generation and establishing long-lasting ties between the United States and other countries through exchange programs and institutional partnerships. Programs focus primarily on secondary schools and promote mutual understanding, leadership development, educational transformation and democratic ideals.''
				\end{itemize}
			\item SportsUnited: \vspace{-0.1cm}
				\begin{itemize} \itemsep -1pt
				\item \url{http://exchanges.state.gov/sports/index.html}
				\item SportsUnited is an international sports programming initiative designed to help start a dialogue at the grassroots level with non-elite boys and girls ages 7-17.
				\item The programs aid youth in discovering how success in athletics can be translated into the development of life skills and achievement in the classroom.
				\item Foreign participants are given an opportunity to establish links with U.S. sports professionals and exposure to American life and culture.
				\item Americans learn about foreign cultures and the challenges young people from overseas face today.
				\item The U.S. Department of State has programmed initiatives in: baseball, basketball, football, track and field, soccer, volleyball, wrestling, archery, boxing, swimming, fencing, table tennis, ice skating, weightlifting, water polo and managing sports community centers.
				\item Countries covered by this program are listed on the web page.
				\item Sports Envoy Program: \vspace{-0.1cm}
					\begin{itemize} \itemsep -1pt
					\item \url{http://exchanges.state.gov/sports/envoy1.html}
					\item Working with the national sports leagues and the U.S. Olympic Committee, athletes and coaches in various sports are chosen to serve as envoys or ambassadors of sport in overseas programs that include conducting clinics, visiting schools and speaking to youth.
					\item The American athletes and coaches conduct drills and team building activities, as well as engage the youth in a dialogue on the importance of an education, positive health practices and respect for diversity.
					\end{itemize}
				\item Sports Grant Competition: \vspace{-0.1cm}
					\begin{itemize} \itemsep -1pt
					\item The Bureau of Educational and Cultural Affairs (ECA) has an annual open competition under its International Sports Programming Initiative.
					\item Public and private non-profit organizations, 501(c)(3), may submit proposals to discuss approaches designed to enhance and improve the infrastructure of youth sports programs.
					\item The focus of all programs must be reaching out to non-elite youth ages 7-17 and/or their coaches/administrators.
					\item There are four themes that a proposal can address; Youth Sports Management, Training Sports Coaches, Sport and Disability, and Sport and Health.
					\item The list of eligible countries changes each year.
					\item \url{http://exchanges.state.gov/sports/index/sports-grant-competition.html}
					\end{itemize}
				\item Sports Visitor Program: \vspace{-0.1cm}
					\begin{itemize} \itemsep -1pt
					\item Nominated by our U.S. embassies overseas, selected athletes, managers and coaches are brought to the U.S. for technical sports training, sports management, conflict resolution training and exposure to valuable U.S. sports contacts and then are encouraged to return to conduct in-country clinics for youth with their newly learned skills.
					\item \url{http://exchanges.state.gov/sports/visitors.html}
					\end{itemize}
				\end{itemize}
			\end{enumerate}
		\end{enumerate}
	\end{enumerate}
\item U.S. Department of Labor: \vspace{-0.3cm}
	\begin{enumerate} \itemsep -2pt
	\item Wage and Hour Division: \vspace{-0.2cm}
		\begin{enumerate} \itemsep -2pt
		\item YouthRules!: \vspace{-0.1cm}
			\begin{enumerate} \itemsep -1pt
			\item \url{http://youthrules.dol.gov/}
			\item Has information for youths, parents, educators, and employers on how to let youth work part-time safely
			\item Teens: \url{http://youthrules.dol.gov/teens/default.htm}
			\item Parents: \url{http://youthrules.dol.gov/parents/default.htm}
			\item Educators: \url{http://youthrules.dol.gov/educators/default.htm}
			\item Employers: \url{http://youthrules.dol.gov/employers/default.htm}
			\item Resources: \url{http://youthrules.dol.gov/resources.htm}
			\item Compliance Assistance: \url{http://youthrules.dol.gov/ca.htm}
			\end{enumerate}
		\end{enumerate}
	\end{enumerate}
\item ASCL Educational Services, Inc. (Marc McCulloch): \vspace{-0.3cm}
	\begin{enumerate} \itemsep -2pt
	\item Transitions: Life Skills for Personal Success!: \vspace{-0.2cm}
		\begin{enumerate} \itemsep -2pt
		\item Curriculum \& Materials: \url{http://transitions.ascl.info/infomaterials}
		\item Soft Skills: \url{http://transitions.ascl.info/infoskills}
		\end{enumerate}
	\end{enumerate}
\item Partnership for 21st Century Skills: \vspace{-0.3cm}
	\begin{enumerate} \itemsep -2pt
	\item \url{http://www.p21.org/}
	\item Framework for 21st Century Learning: \url{http://www.p21.org/index.php?option=com_content&task=view&id=254&Itemid=119}
	\item Tools and Resources: \url{http://www.p21.org/index.php?option=com_content&task=view&id=273&Itemid=139}
	\end{enumerate}
\item National Career and Technical Education Foundation (NCTEF): \vspace{-0.3cm}
	\begin{enumerate} \itemsep -2pt
	\item States' Career Clusters Initiative (SCCI): \vspace{-0.2cm}
		\begin{enumerate} \itemsep -2pt
		\item \url{http://www.careerclusters.org/}
		\item The 16 Career Clusters: \url{http://www.careerclusters.org/16clusters.cfm}
		\item Plans of Study: \url{http://www.careerclusters.org/resources/web/pos.cfm}
		\item Knowledge and Skills Charts: \url{http://www.careerclusters.org/resources/web/ks.php}
		\item Crosswalks: \url{http://www.careerclusters.org/crosswalks.cfm}
		\item Publications: \url{http://www.careerclusters.org/publications.php}
		\item Sixteen Career Clusters and Their Pathways: \url{http://www.careerclusters.org/list16clusters.php}
		\item Career Clusters Models: \url{http://www.careerclusters.org/resources/web/16ccall.php?action=models}
		\item Career Clusters Brochure Previews: \url{http://www.careerclusters.org/resources/web/16ccall.php?action=brochures}
		\item Career Clusters Interest Survey: \url{http://www.careerclusters.org/ccinterestsurvey.php}
		\item Related Websites: \url{http://www.careerclusters.org/related.php}
		\end{enumerate}
	\end{enumerate}
\item U. S. Department of Labor: \vspace{-0.3cm}
	\begin{enumerate} \itemsep -2pt
	\item Employment and Training Administration: \vspace{-0.2cm}
		\begin{enumerate} \itemsep -2pt
		\item CareerOneStop: \vspace{-0.1cm}
			\begin{enumerate} \itemsep -1pt
			\item \url{http://www.careeronestop.org/}
			\item Students, parents, and career advisors: \url{http://www.careeronestop.org/studentsandcareeradvisors/studentsandcareeradvisors.aspx}
			\end{enumerate}
		\end{enumerate}
	\end{enumerate}
\item U. S. Department of Defense: \vspace{-0.3cm}
	\begin{enumerate} \itemsep -2pt
	\item ASVAB Career Exploration Program: \vspace{-0.2cm}
		\begin{enumerate} \itemsep -2pt
		\item \url{http://www.asvabprogram.com/}
		\item Learn about yourself: \url{http://www.asvabprogram.com/index.cfm?fuseaction=learn.main}
		\item Explore careers: \url{http://www.asvabprogram.com/index.cfm?fuseaction=explore.main}
		\item Plan for your future: \url{http://www.asvabprogram.com/index.cfm?fuseaction=plan.main}
		\item Information for educators and career counselors: \url{http://www.asvabprogram.com/index.cfm?fuseaction=edu.main}
		\item Information for parents: \url{http://www.asvabprogram.com/index.cfm?fuseaction=parents.main}
		\end{enumerate}
	\end{enumerate}
\end{enumerate}



%%%%%%%%%%%%%%%%%%%%%%%%%%%%%%%%%%%%%%%%%%%
\section{Internship Opportunities}
\label{Internship Opportunities}

Internship opportunities: \vspace{-0.3cm}
\begin{enumerate} \itemsep -4pt
\item Canada: \vspace{-0.3cm}
	\begin{enumerate} \itemsep -2pt
	\item SWAP: \vspace{-0.2cm}
		\begin{enumerate} \itemsep -2pt
		\item \url{http://www.swap.ca/}
		\item For Canadians who want to work abroad: \url{http://www.swap.ca/out_eng/index.aspx}
		\item For citizens of selected countries who want to work in Canada: \url{http://www.swap.ca/in_eng/partner_organizations.aspx}
		\end{enumerate}
	\end{enumerate}
\item Singapore: \vspace{-0.3cm}
	\begin{enumerate} \itemsep -2pt
	\item Speedwing Training (Asia) Pte Ltd: \vspace{-0.2cm}
		\begin{enumerate} \itemsep -2pt
		\item \url{http://www.speedwing.org/}
		\item For Singaporeans who want to work in the United States, Canada, Germany, and New Zealand
		\item For citizens of selected countries who want to work in Singapore
		\end{enumerate}
	\end{enumerate}
\end{enumerate}

%%%%%%%%%%%%%%%%%%%%%%%%%%%%%%%%%%%%%%%%%%%
\subsection{Internship Opportunities in Australia}
\label{internshipaus}

Internship Opportunities in Australia: \vspace{-0.3cm}
\begin{enumerate} \itemsep -4pt
\item The Association of Professional Engineers, Scientists and Managers, Australia: \url{http://www.apesma.asn.au/index.asp} --- Ask for guide to internships in your region/major; free student membership
\item Engineers Australia: \url{http://www.engineersaustralia.org.au/} --- Ask for guide to internships in your region/major; free student membership
\item CPA Australia: \url{http://www.cpaaustralia.com.au/cps/rde/xchg/cpa/hs.xsl/index.html} and \url{http://www.cpaaustralia.com.au/cps/rde/xchg/careers/site/index_ENA_HTML.htm/cps/rde/xchg/SID-3F57FECB-EEFEF50E/careers/site/204_ENA_HTML.htm}
\item Institute of Chartered Accountants in Australia: \url{http://www.charteredaccountants.com.au/}
\item 
\end{enumerate}


%%%%%%%%%%%%%%%%%%%%%%%%%%%%%%%%%%%%%%%%%%%
\subsection{Internship Opportunities in Europe}
\label{internshipeu}

Internship Opportunities in Portugal: \vspace{-0.3cm}
\begin{enumerate} \itemsep -4pt
\item Portugal: \vspace{-0.3cm}
	\begin{enumerate} \itemsep -2pt
	\item IAESTE Portugal (The International Association for the Exchange of Students for Technical Experience): \url{http://www.iaeste.pt/en/foreign-trainees/why-portugal/}
	\end{enumerate}
\item United Kingdom: \vspace{-0.3cm}
	\begin{enumerate} \itemsep -2pt
	\item Graduate Talent Pool: \url{http://graduatetalentpool.direct.gov.uk/}
	\end{enumerate}
\end{enumerate}




%%%%%%%%%%%%%%%%%%%%%%%%%%%%%%%%%%%%%%%%%%%
\subsection{Internship Opportunities in the United States}
\label{internshipsus}

Internship Opportunities in the United States: \vspace{-0.3cm}
\begin{enumerate} \itemsep -4pt
\item Use the Procedure \proc{Find}$(\varphi, \tau)$ in \S\ref{heuristiclocateoutreach} to look up internship opportunities and lists of internship opportunities.

Look at government organizations (e.g., the White House), nonprofit organizations (e.g., Engineers Without Borders), professional organizations (e.g., IEEE and ACM), colleges and universities, and companies (e.g., Intel, Google, and start-ups).

You can start your search by looking at the organizations that provide resources for underrepresented minorities as well as resources for scholarships and fellowships. These information can be found in other sections of this document.

If you do not know where to start, speak to a professor or staff member at the career center of your college/university. Alternatively, you can ask your awesome resident advisors (RAs).

My personal advice is to start your search based on your interests and skill set. You can always narrow the search space based on factors, such as geographical location, later on.

Competitive internships, especially research internships in electrical and computer engineering or computer science, weed out many students from applying via demanding job requirements. For example, if you want to apply for research internships with electronic design automation (EDA) companies and corporate research labs, you would need to have significant experience designing integrated circuits and developing EDA software. The stringent job requirements also mean that students need to plan in advance (say, about a year) about the internships that they would like to seek, and plan to acquire the necessary skill set and experiences before the application deadlines (which can be several months before the start of your internship).

Taking as many challenging classes as you can possibly cope, especially in electrical and computer engineering or computer science, would provide you with a skill set that allows you to apply for competitive internships in many fields. Apart from taking challenging classes as well as engaging in research and/or open source projects, you can try to acquire additional skills and experience in your free time to boost the competitiveness of your internship application. Certain skills and experiences, such as compiler design, are hard to acquire in your free time, so it would be ``easier'' to take classes that would help you acquire those skills and experiences.

Note that you may want to look into creating your own entrepreneurial venture, say an EDA start-up or organization in social entrepreneurship, rather than to seek an internship. Also, seeking an internship abroad is always a good addition to your resume/CV.
\item National Science Foundation: \vspace{-0.3cm}
	\begin{enumerate} \itemsep -2pt
	\item Research Experiences for Undergraduates (REU): \vspace{-0.2cm}
		\begin{enumerate} \itemsep -2pt
		\item \url{http://www.nsf.gov/crssprgm/reu/reu_search.cfm}
		\item Academic fields: \vspace{-0.1cm}
			\begin{enumerate} \itemsep -1pt
			\item Astronomical Sciences
			\item Atmospheric and Geospace Sciences
			\item Biological Sciences
			\item Chemistry
			\item Computer and Information Science and Engineering
			\item Cyberinfrastructure
			\item Department of Defense (DoD)
			\item Earth Sciences
			\item Education and Human Resources
			\item Engineering
			\item Ethics and Values Studies
			\item International Science and Engineering
			\item Materials Research
			\item Mathematical Sciences
			\item Ocean Sciences
			\item Physics
			\item Polar Programs
			\item Social, Behavioral, and Economic Sciences
			\end{enumerate}
		\end{enumerate}
	\end{enumerate}
\item Society for Industrial and Applied Mathematics: \vspace{-0.3cm}
	\begin{enumerate} \itemsep -2pt
	\item Internship and Career Information in Industry, Research Institutions, and Government Labs: \url{http://www.siam.org/careers/internships.php}
	\end{enumerate}
\item American Institute of Physics (AIP): \vspace{-0.3cm}
	\begin{enumerate} \itemsep -2pt
	\item Society of Physics Students (SPS): \vspace{-0.2cm}
		\begin{enumerate} \itemsep -2pt
		\item SPS Internships: \url{http://www.spsnational.org/programs/internships/}
		\item Research Opportunities: \url{http://www.spsnational.org/programs/research/}
		\end{enumerate}
	\end{enumerate}
%%%%%%%%%%%%%%%%%%%%%%%%%%%%%%%%%%%%%%
%%%%%%%%%%%%%%%%%%%%%%%%%%%%%%%%%%%%%%
%%%%%%%%%%%%%%%%%%%%%%%%%%%%%%%%%%%%%%
\item United States Office of Personnel Management: \vspace{-0.3cm}
	\begin{enumerate} \itemsep -2pt
	\item USAJOBS: \vspace{-0.2cm}
		\begin{enumerate} \itemsep -2pt
		\item Student Jobs: \url{http://www.usajobs.gov/studentjobs/}
		\end{enumerate}
	\end{enumerate}
%%%%%%%%%%%%%%%%%%%%%%%%%%%%%%%%%%%%%%
%%%%%%%%%%%%%%%%%%%%%%%%%%%%%%%%%%%%%%
%%%%%%%%%%%%%%%%%%%%%%%%%%%%%%%%%%%%%%
\item Americans for the Arts: \vspace{-0.3cm}
	\begin{enumerate} \itemsep -2pt
	\item Internship Program: \url{http://www.americansforthearts.org/about_us/internships.asp}
	\end{enumerate}
\item New York Women's Foundation: \vspace{-0.3cm}
	\begin{enumerate} \itemsep -2pt
	\item Internship Opportunities: \url{http://www.nywf.org/internship.html}
	\item Volunteer Opportunities: \url{http://www.nywf.org/volunteer.html}
	\end{enumerate}
\item Council on International Educational Exchange (CIEE): \url{http://www.ciee.org/hire/index.aspx}
\item The John F. Kennedy Center for the Performing Arts: \vspace{-0.3cm}
	\begin{enumerate} \itemsep -2pt
	\item Kennedy Center Arts Management Internships: \url{http://www.kennedy-center.org/education/artsmanagement/internships/}
	\end{enumerate}
\item Washington Performing Arts Society (WPAS): \vspace{-0.3cm}
	\begin{enumerate} \itemsep -2pt
	\item Internships with WPAS: \vspace{-0.2cm}
		\begin{enumerate} \itemsep -2pt
		\item \url{http://www.wpas.org/aboutwpas/opportunities/intern.aspx}
		\item ``WPAS offers internships throughout the year. Applicants should be highly motivated, creative and hard-working individuals with an interest in all aspects of arts management. It is required that applicants have previous office experience.''
		\item In addition, applicants should possess: \vspace{-0.1cm}
			\begin{enumerate} \itemsep -1pt
			\item Interest/background in music, dance or performance art
			\item Strong organizational skills
			\item Effective writing and communication skills
			\item Ability to learn quickly, handle multiple tasks, take initiative, and work independently with little supervision
			\item High energy level and ability to work well in deadline and/or pressure situations
			\item Computer literacy
			\end{enumerate}
		\item ``WPAS interns leave our offices with a better understanding of arts management, knowledge of artists in a variety of fields (classical music, world music, dance and performance art), contacts in theaters throughout the D.C. metro area, practical experience and a portfolio of work. The internship is unpaid, however stipends are occasionally granted during the performance year (September - May). Interns are also invited to attend many WPAS performances on a complimentary basis.''
		\item Types of internships: \vspace{-0.1cm}
			\begin{enumerate} \itemsep -1pt
			\item Accounting Internship
			\item Development Internship
			\item Education Internship
			\item Marketing/Public Relations Internship
			\item Office Administration Internship
			\item Programming Internship
			\end{enumerate}
		\end{enumerate}
	\end{enumerate}
\item The Washington Ballet: Internships, \url{http://www.washingtonballet.org/about-twb/auditions-employment/#internships}
\item The Choral Arts Society of Washington: \vspace{-0.3cm}
	\begin{enumerate} \itemsep -2pt
	\item Internship and Apprenticeship Program: \url{http://www.choralarts.org/About-Us/Internships-and-Apprenticeships.aspx}
	\end{enumerate}
\item League of American Orchestras: Internships, \url{http://www.americanorchestras.org/career_center/internships.html}
%%%%%%%%%%%%%%%%%%%%%%%%%%%%%%%%%%%%%%
%%%%%%%%%%%%%%%%%%%%%%%%%%%%%%%%%%%%%%
%%%%%%%%%%%%%%%%%%%%%%%%%%%%%%%%%%%%%%
\item Congressional Hispanic Caucus Institute (CHCI): \vspace{-0.3cm}
	\begin{enumerate} \itemsep -2pt
	\item CHCI United Health Foundation Scholars: \vspace{-0.2cm}
		\begin{enumerate} \itemsep -2pt
		\item \url{http://www.chci.org/scholarships/page/chci-united-health-foundation-scholars-}
		\item In addition to providing scholarship opportunities for Latino youth, the United Health Foundation decided to partner with CHCI to create a six-month internship program for students interested in the medical field.
		\item Seventeen participants enrolled in either a full-time undergraduate or graduate course of study at an accredited two- or four-year college, university, vocational or technical school were selected.
		\end{enumerate}
	\item CHCI Congressional Internship: \vspace{-0.2cm}
		\begin{enumerate} \itemsep -2pt
		\item The purpose of the Congressional Internship Program (CIP) is to expose young Latinos to the legislative process and to strengthen their professional and leadership skills, ultimately promoting the presence of Latinos on Capitol Hill.
		\item The Congressional Internship Program provides college students with a paid Congressional work placement on Capitol Hill for a period of twelve weeks (Spring/Fall) or eight weeks (Summer). This unmatched experience allows students to learn first hand about our nation's legislative process.
		\end{enumerate}
	\end{enumerate}
\item Mexican American Legal Defense and Educational Fund (MALDEF): Law Clerk Summer Internship program, \url{http://maldef.org/about/jobs/index.html}
\item Hispanic Association of Colleges and Universities (HACU): \vspace{-0.3cm}
	\begin{enumerate} \itemsep -2pt
	\item HACU National Internship Program (HNIP): \url{http://www.hacu.net/hacu/HNIP_EN.asp}
	\end{enumerate}
%%%%%%%%%%%%%%%%%%%%%%%%%%%%
\item Smithsonian Institution: \vspace{-0.3cm}
	\begin{enumerate} \itemsep -2pt
	\item Smithsonian Institution Traveling Exhibition Service (SITES): \vspace{-0.2cm}
		\begin{enumerate} \itemsep -2pt
		\item Internship programs: \url{http://www.sites.si.edu/interns/internships.htm}
		\item ``The Smithsonian Institution Traveling Exhibition Service internship programs allows people with diverse interests, strengths, and goals to experience an educational environment where they can work and learn from professionals in the museum field.''
		\item ``SITES offers internship opportunities in a variety of different areas: public relations, development (fund raising), research, and project design.''
		\end{enumerate}
	\item Smithsonian Folkways Recordings (or simply, Smithsonian Folkways): \vspace{-0.2cm}
		\begin{enumerate} \itemsep -2pt
		\item Internships: \url{http://www.folkways.si.edu/about_us/jobs.aspx}
		\end{enumerate}
	\item Freer Gallery of Art / Arthur M. Sackler Gallery: \vspace{-0.2cm}
		\begin{enumerate} \itemsep -2pt
		\item Internships: \url{http://www.asia.si.edu/research/internships.asp}
		\end{enumerate}
	\item National Museum of American History: \vspace{-0.2cm}
		\begin{enumerate} \itemsep -2pt
		\item Jerome and Dorothy Lemelson Center for the Study of Invention and Innovation: \vspace{-0.1cm}
			\begin{enumerate} \itemsep -1pt
			\item Archival Internships: \url{http://invention.smithsonian.org/resources/research_interns.aspx}
			\end{enumerate}
		\end{enumerate}
	\end{enumerate}
%%%%%%%%%%%%%%%%%%%%%%%%%%%%
\item Council on International Educational Exchange (CIEE): \vspace{-0.3cm}
	\begin{enumerate} \itemsep -2pt
	\item CIEE's Trainee Program: \vspace{-0.2cm}
		\begin{enumerate} \itemsep -2pt
		\item part of the J-1 visa category of the US government�s Exchange Visitor Program
		\item \url{http://www.ciee.org/trainee/}
		\end{enumerate}
	\item CIEE Work \& Travel USA; and Internship USA: \vspace{-0.2cm}
		\begin{enumerate} \itemsep -2pt
		\item \url{http://www.ciee.org/hire/}
		\item \url{http://www.ciee.org/wat/}
		\end{enumerate}
	\end{enumerate}
\item American Institute For Foreign Study (AIFS): \vspace{-0.3cm}
	\begin{enumerate} \itemsep -2pt
	\item Camp America Counselors and Summer Staff: \url{http://www.aifs.com/work_travel.asp}
	\item Au Pair Placement: \url{http://www.aifs.com/au_pair.asp}
	\end{enumerate}
\item U.S. Department of State: \vspace{-0.3cm}
	\begin{enumerate} \itemsep -2pt
	\item Bureau of Educational and Cultural Affairs: \vspace{-0.2cm}
		\begin{enumerate} \itemsep -2pt
		\item International cultural programs: \url{http://exchanges.state.gov/cultural/related-cultural-programs.html}
		\item Office of Global Educational Programs: \vspace{-0.1cm}
			\begin{enumerate} \itemsep -1pt
			\item Camp Counselor: \vspace{-0.1cm}
				\begin{itemize} \itemsep -1pt
				\item \url{http://exchanges.state.gov/jexchanges/programs/camp.html}
				\item Camp counselors interact with groups of American youth by overseeing their camp activities during the U.S. summer.
				\item Through the Camp Counselor program, American campers have the chance to gain knowledge of foreign cultures, while foreign participants increase their knowledge of American culture.
				\item Participants must be at least 18 years of age and may work as counselors in U.S. summer camps for up to four months. Extensions are not allowed. They receive a combination a pay and benefits equal to Americans who work in the same position.
				\end{itemize}
			\end{enumerate}
		\item Private Sector Exchange office: \vspace{-0.1cm}
			\begin{enumerate} \itemsep -1pt
			\item \url{http://exchanges.state.gov/jexchanges/index.html}
			\item The Private Sector Exchange office designates, monitors and partners with U.S. organizations, including government agencies, academic institutions, educational and cultural organizations, and corporations, that administer the Exchange Visitor Program.
			\item Au Pair program: \vspace{-0.1cm}
				\begin{itemize} \itemsep -1pt
				\item Through the Au Pair program, foreign nationals between 18 and 26 years of age participate in the home life of a host family. Au pairs provide limited childcare services for up to 12 months. An extension of 6, 9, or 12 months may be granted in certain cases.
				\item \url{http://exchanges.state.gov/jexchanges/programs/aupair.html}
				\end{itemize}
			\item Internships: \vspace{-0.1cm}
				\begin{itemize} \itemsep -1pt
				\item \url{http://exchanges.state.gov/jexchanges/programs/intern.html}
				\item Internship programs are designed to allow foreign professionals to come to the United States to gain exposure to U.S. culture and to receive training in U.S. business practices in their chosen occupational field.
				\item The maximum duration of an internship in any occupational field is 12 months.
				\item Upon completion of their exchange programs, participants are expected to return to their home countries.
				\item The State Department allows internships in the following occupational categories: \vspace{-0.1cm}
					\begin{itemize} \itemsep -1pt
					\item Agriculture, Forestry, and Fishing
					\item Arts and Culture
					\item Construction and Building Trades
					\item Education, Social Sciences, Library Science, Counseling and Social Services
					\item Health Related Occupations
					\item Hospitality and Tourism
					\item Information Media and Communications
					\item Management, Business, Commerce and Finance
					\item Public Administration and Law
					\item The Sciences, Engineering, Architecture, Mathematics, and Industrial Occupations.
					\end{itemize}
				\item An Intern must be a foreign national: \vspace{-0.1cm}
					\begin{itemize} \itemsep -1pt
					\item Who is currently enrolled in and pursuing studies at a foreign degree- or certificate-granting post-secondary academic institution outside the United States, or
					\item Who has graduated from such an institution no more than 12 months prior to his or her exchange visitor program start date.
					\end{itemize}
				\item Interns cannot work in unskilled or casual labor positions, in positions that require or involve child care or elder care, or in any kind of position that involves medical patient care or contact. Nor can interns work in positions that require more than 20 per cent clerical or office support work.
				\end{itemize}
			\item The Summer Work Travel Program: \vspace{-0.1cm}
				\begin{itemize} \itemsep -1pt
				\item \url{http://exchanges.state.gov/jexchanges/programs/swt.html}
				\item In the summer work travel program, post-secondary students may enter the United States to work and travel during their summer vacation.
				\item Participants can be admitted to the program more than once.
				\item The maximum length of the program is four months.
				\item Most of the time, participants work in unskilled service positions at resorts, hotels, restaurants, and amusement parks. However, they may also work in other types of organizations.
				\item For example, they could work in architectural firms, scientific research organizations, graphic art/publishing and other media communication businesses, advertising agencies, computer software and electronics firms, legal offices, etc.
				\item The program may not exceed four-months and must be finished during the student's summer vacation.
				\item Participants receive pay and benefits equal to an American working in the same or similar position.
				\end{itemize}
			\item Training programs: \vspace{-0.1cm}
				\begin{itemize} \itemsep -1pt
				\item \url{http://exchanges.state.gov/jexchanges/programs/trainee.html}
				\item Training programs are designed to allow foreign professionals to come to the United States to gain exposure to U.S. culture and to receive training in U.S. business practices in their chosen occupational field.
				\item Foreign nationals have had the opportunity to train with some of the finest employers in the U.S., gaining real time experience in their chosen career fields.
				\item Upon completion of their exchange programs, participants are expected to return to their home countries to utilize their newly learned skills and knowledge to advance their careers and share their experiences with their communities.
				\item The State Department allows training programs in the following occupational categories: \vspace{-0.1cm}
					\begin{itemize} \itemsep -1pt
					\item Agriculture, Forestry, and Fishing
					\item Arts and Culture
					\item Construction and Building Trades
					\item Education, Social Sciences, Library Science, Counseling and Social Services
					\item Health Related Occupations
					\item Hospitality and Tourism
					\item Information Media and Communications
					\item Management, Business, Commerce and Finance
					\item Public Administration and Law
					\item The Sciences, Engineering, Architecture, Mathematics, and Industrial Occupations.
					\end{itemize}
				\item A trainee must be a foreign national who has: \vspace{-0.1cm}
					\begin{itemize} \itemsep -1pt
					\item A degree or professional certificate from a foreign post-secondary academic institution and at least one year of prior related work experience in his or her occupational field outside the United States, or
					\item Five years of work experience outside the United States in the occupational field in which they are seeking training.
					\end{itemize}
				\end{itemize}
			\item Specialists: \vspace{-0.1cm}
				\begin{itemize} \itemsep -1pt
				\item \url{http://exchanges.state.gov/jexchanges/programs/specialist.html}
				\item This category is for a participant who is an expert in a field of specialized knowledge or skill who will demonstrate such skills in the United States. Such exchanges are to provide opportunities to increase the exchange knowledge and ideas between American and foreign specialists. The maximum duration of this program is one year.
				\item This category is for foreign nationals who are experts in a field of specialized knowledge or skill, coming to the United States for observing, consulting, or demonstrating their special skills, except: Professors and Research Scholars, Short-Term Scholars, and Alien Physicians.
				\item Individuals participating in the specialist program are: \vspace{-0.1cm}
					\begin{itemize} \itemsep -1pt
					\item Experts in a field of specialized knowledge or skill;
					\item Seeks to travel to the United States for the purpose of observing, consulting, or demonstrating their special knowledge or skills;
					\item Does not fill a permanent or long-term position of employment while in the U.S.
					\end{itemize}
				\end{itemize}
			\item International Visitor: \vspace{-0.1cm}
				\begin{itemize} \itemsep -1pt
				\item \url{http://exchanges.state.gov/jexchanges/programs/intl_visitor.html}
				\item The international visitor category enables visitors to better understand American culture and enhanced American knowledge of foreign cultures.
				\item This category is for individuals who are recognized as potential leaders in their own country, selected by the Department of State to participate in observation tours, discussions, consultation, professional meetings, conferences, workshops and travel.
				\item The maximum duration of the program is one year.
				\end{itemize}
			\item Alien Physician: \vspace{-0.1cm}
				\begin{itemize} \itemsep -1pt
				\item \url{http://exchanges.state.gov/jexchanges/programs/physician.html}
				\item The Alien Physician program is for foreign national physicians seeking entry into U.S. graduate medical education programs or training at accredited U.S. schools of medicine or other U.S. institutions.
				\item There are generally two types of exchange programs in which a foreign national physician (also referred to as a foreign/international medical graduate) participates: \vspace{-0.1cm}
					\begin{itemize} \itemsep -1pt
					\item Clinical training in the �alien physician� category
					\item Non-Clinical training in the �research scholar� category
					\end{itemize}
				\end{itemize}
			\item FORTUNE/U.S. State Department Global Women's Mentoring Partnership: \vspace{-0.1cm}
				\begin{itemize} \itemsep -1pt
				\item \url{http://exchanges.state.gov/citizens/professionals/fortunepartnership.html}
				\item This public-private partnership places talented, emerging women leaders from all over the world in mentoring programs with FORTUNE's Most Powerful Women Leaders.
				\item For three weeks, American and international participants work together in mentoring relationships to share the skills and experiences necessary for strengthening women�s leadership.
				\end{itemize}
			\item American Council of Young Political Leaders (ACYPL): \vspace{-0.1cm}
				\begin{itemize} \itemsep -1pt
				\item \url{http://exchanges.state.gov/citizens/profs/acypl.html}
				\item \url{http://www.acypl.org/}
				\item For 44 years, the American Council of Young Political Leaders (ACYPL) has designed, organized and managed unique international educational exchanges for young political leaders (ages 25-40) worldwide.
				\item ACYPL programs are designed to promote mutual understanding, respect, and friendship and to cultivate long-lasting relationships among young people who are poised to become tomorrow's global leaders and policy makers.
				\item American participants are nominated by members of Congress, governors, political party leaders, and ACYPL alumni, while international delegates are selected from countries where ACYPL is currently conducting programs by international program partners with the U.S. Embassy input.
				\end{itemize}
			\item Edward R. Murrow Program for Journalists: \vspace{-0.1cm}
				\begin{itemize} \itemsep -1pt
				\item \url{http://exchanges.state.gov/ivlp/murrow.html}
				\item The Edward R. Murrow Program for Journalists invites rising international journalists to travel to the United States and examine journalistic principles and practices.
				\end{itemize}
			\end{enumerate}
		\item Office of Citizen Exchanges: \vspace{-0.1cm}
			\begin{enumerate} \itemsep -1pt
			\item Youth Programs Division: \vspace{-0.1cm}
				\begin{itemize} \itemsep -1pt
				\item \url{http://exchanges.state.gov/youth/index.html}
				\item The Youth Programs Division is committed to empowering the successor generation and establishing long-lasting ties between the United States and other countries through exchange programs and institutional partnerships.
				\item Programs focus primarily on secondary schools and promote mutual understanding, leadership development, educational transformation, and democratic ideals.
				\item Year-Long Programs, Short Term Programs, and Virtual Partnerships: \url{http://exchanges.state.gov/youth/programs-by-type.html}
				\item Programs for Young Americans, and Programs for International Students and Teachers: \url{http://exchanges.state.gov/youth/programs-by-participants.html}
				\item Opportunities for American Hosts: Families and Schools, \url{http://exchanges.state.gov/youth/opps-for-am-hosts.html}
				\item Programs for High School Students: \url{http://exchanges.state.gov/youth/programs.html}
				\end{itemize}
			\item Professional Exchanges Division: \vspace{-0.1cm}
				\begin{itemize} \itemsep -1pt
				\item \url{http://exchanges.state.gov/citizens/profs.html}
				\item The Professional Exchanges division provides grants to U.S. nonprofit organizations to carry out exchange programs that support the professional development of foreign participants. The purpose of each exchange program is to engage with foreign leaders in critical professions, to demonstrate respect for foreign cultures, and to promote mutual understanding between the people of the United States and other countries.
				\item Professional exchanges typically last several years and include internships, study tours or workshops in the United States and in the host country. Participants come from a variety of professions including education administrators, public servants, journalists, labor union officials, entrepreneurs, environmental leaders, jurists, lawyers, and civic leaders.
				\item ECA grant opportunities: \vspace{-0.1cm}
					\begin{itemize} \itemsep -1pt
					\item Open Funding Opportunities: Requests For Grant Proposals (RFGPs), \url{http://exchanges.state.gov/grants/open2.html}
					\item Grants.gov: \url{http://www.grants.gov/}
					\end{itemize}
				\item Grants by Region: \vspace{-0.1cm}
					\begin{itemize} \itemsep -1pt
					\item \url{http://exchanges.state.gov/citizens/professionals/grant-region.html}
					\item Africa 
					\item East Asia and the Pacific 
					\item Europe and Eurasia 
					\item North Africa and the Middle East 
					\item South and Central Asia 
					\item Western Hemisphere 
					\item Multi-regional
					\end{itemize}
				\end{itemize}
			\end{enumerate}
		\end{enumerate}
	\end{enumerate}
\end{enumerate}






%%%%%%%%%%%%%%%%%%%%%%%%%%%%%%%%%%%%%%%%%%%
\section{Resources on Studying Abroad}
\label{resourcesonstudyingabroad}

Resources on studying abroad: \vspace{-0.3cm}
\begin{enumerate} \itemsep -4pt
\item Council on International Educational Exchange (CIEE): \vspace{-0.3cm}
	\begin{enumerate} \itemsep -2pt
	\item Study abroad programs for high school students from the United States: \vspace{-0.2cm}
		\begin{enumerate} \itemsep -2pt
		\item \url{http://www.ciee.org/hsabroad/index.html}
		\item \url{http://www.ciee.org/hsabroad/high-school-study-abroad/index.html}
		\item These programs include:: \vspace{-0.1cm}
			\begin{enumerate} \itemsep -1pt
			\item High School Abroad programs (for U.S. high school students)
			\item Summer High School Abroad programs (for U.S. high school students)
			\item Gap Year Abroad programs (for recent U.S. high school graduates)
			\end{enumerate}
		\end{enumerate}
	\end{enumerate}
\item U.S. Department of State: \vspace{-0.3cm}
	\begin{enumerate} \itemsep -2pt
	\item Bureau of Educational and Cultural Affairs: \vspace{-0.2cm}
		\begin{enumerate} \itemsep -2pt
		\item Office of Global Educational Programs: \vspace{-0.1cm}
			\begin{enumerate} \itemsep -1pt
			\item EducationUSA: \vspace{-0.1cm}
				\begin{itemize} \itemsep -1pt
				\item EducationUSA is a network of more than 400 student advising centers, which offer accurate, comprehensive, objective and timely information about educational opportunities in the United States and guidance to qualified individuals on how best to access those opportunities. This includes information about application procedures, standardized test requirements, student visas, financial aid, and the full range of accredited U.S. higher education institutions.
				\item \url{http://exchanges.state.gov/globalexchanges/index/educationusa.html}
				\item \url{http://www.educationusa.state.gov/} and \url{http://www.educationusa.info/centers.php}
				\end{itemize}
			\item Open Doors: \vspace{-0.1cm}
				\begin{itemize} \itemsep -1pt
				\item The Educational Information and Resources Branch funds Open Doors, a census of foreign students and scholars in the U.S. and of U.S. students studying abroad published annually by the Institute for International Education.
				\item Open Doors data is used by U.S. embassies, the Departments of State, Commerce, and Education, and U.S. colleges and universities to inform policy decisions about educational exchanges, trade in educational services, and study abroad activity.
				\item \url{http://exchanges.state.gov/globalexchanges/index/open_doors.html}
				\item \url{http://www.opendoors.iienetwork.org/}
				\end{itemize}
			\end{enumerate}
		\item EducationUSA: \vspace{-0.1cm}
			\begin{enumerate} \itemsep -1pt
			\item \url{http://educationusa.state.gov/}
			\item For U.S. (college) students who want to study/work abroad: \url{http://www.educationusa.info/pages/students/forus.php}
			\end{enumerate}
		\end{enumerate}
	\end{enumerate}
\item IES Abroad (formerly Institute of European Studies / Institute for the International Education of Students): \vspace{-0.3cm}
	\begin{enumerate} \itemsep -2pt
	\item \url{https://www.iesabroad.org/} and \url{https://www.iesabroad.org/IES/home.html}
	\end{enumerate}
\item Global Learning Semesters, Inc.: \vspace{-0.3cm}
	\begin{enumerate} \itemsep -2pt
	\item Summer in the Mediterranean: \vspace{-0.2cm}
		\begin{enumerate} \itemsep -2pt
		\item \url{http://www.globalsemesters.com/Mediterranean.html}
		\item Has programs in the following areas: \vspace{-0.1cm}
			\begin{enumerate} \itemsep -1pt
			\item Art \& Photography
			\item Early Christianity
			\item Greek Heritage
			\item International Marketing
			\item Music
			\end{enumerate}
		\end{enumerate}
	\end{enumerate}
\item American Institute For Foreign Study (AIFS): \vspace{-0.3cm}
	\begin{enumerate} \itemsep -2pt
	\item \url{http://www.aifs.com/}
	\item College Study Abroad: \url{http://www.aifsabroad.com/}
	\item For high school students: \vspace{-0.2cm}
		\begin{enumerate} \itemsep -2pt
		\item Gifted Education: \url{http://www.aifs.com/gifted_education.asp}
		\item High School Study and Travel: \url{http://www.aifs.com/highschool_study_travel.asp}
		\item Academic Year in America (AYA): \url{http://www.academicyear.org/?source=AIFS}
		\end{enumerate}
	\end{enumerate}
\end{enumerate}





%%%%%%%%%%%%%%%%%%%%%%%%%%%%%%%%%%%%%%%%%%%
\section{College Preparation}
\label{collegepreparation}

College preparation: \vspace{-0.3cm}
\begin{enumerate} \itemsep -4pt
\item {\it Guide to Online Schools} [or {\it GuideToOnlineSchools.com}], {\it The Top 53 College Preparation Resources for Students}. Available at: \url{http://www.guidetoonlineschools.com/tips-and-tools/college-prep-resources}; last accessed on August 25, 2010.
\item U.S. Department of Education's resources for parents to help their children learn: \url{http://www2.ed.gov/parents/academic/help/hyc.html} and \url{http://www2.ed.gov/parents/academic/help/homework/index.html}
\item The College Board: \vspace{-0.3cm}
	\begin{enumerate} \itemsep -2pt
	\item Information about SATs, college preparation, and financial aid
	\item {\it Trends in Higher Education} series 201X: \url{http://trends.collegeboard.org/}
	\item \url{http://www.collegeboard.com/}
	\end{enumerate}
\item {\it Accreditation.org}: \vspace{-0.3cm}
	\begin{enumerate} \itemsep -2pt
	\item Information about the accreditation of engineering degree programs around the world
	\item \url{http://www.accreditation.org/}
	\end{enumerate}
\item {\it New York Times}: \vspace{-0.3cm}
	\begin{enumerate} \itemsep -2pt
	\item The Learning Network: \url{http://learning.blogs.nytimes.com/category/test-yourself/}
	\item New York Times Magazine: \vspace{-0.2cm}
		\begin{enumerate} \itemsep -2pt
		\item The Sep 20, 2010 issue has many articles covering how technology can be used to improve education in K-12 programs. Available online at: \url{http://www.nytimes.com/indexes/2010/09/19/magazine/index.html?ref=magazine}; last accessed on September 20, 2010.
		\item ``New York Times Magazine Features Technology in Education,'' in {\it CCC Blog}, Computing Community Consortium (CCC), Computing Research Association (CRA), Sep 20, 2010. Available online at: \url{http://www.cccblog.org/2010/09/20/new-york-times-magazine-features-technology-in-education/}; last accessed on September 20, 2010.
		\item Articles in this issue discuss: \vspace{-0.1cm}
			\begin{enumerate} \itemsep -1pt
			\item How journalists can make use of technology to automate certain tasks, and improve their productivity and effectiveness in covering news stories
			\item How children can create computer games that introduces them to careers in computing and helps them to develop skills in computational thinking
			\item How to learn things without a lot of rote learning, to have fun while learning, and to use technology to make learning more fun
			\end{enumerate}
		\end{enumerate}
	\end{enumerate}
\item University of Southern California, USC: \vspace{-0.3cm}
	\begin{enumerate} \itemsep -2pt
	\item USC Office of Continuing Education and Summer Programs: \vspace{-0.2cm}
		\begin{enumerate} \itemsep -2pt
		\item \url{http://cesp.usc.edu/}
		\item These programs allow students in K-12 to earn credit at USC, and exposes them to different majors/professions, like medicine, engineering, creative writing, or film making.
		\item Students can benefit from these programs, and learn about different academic disciplines before applying to college. This would help them in their college applications.
		\item Underrepresented minority students can get scholarships to attend these programs. So, if parents have financial difficulty paying for the programs, they can seek financial aid for this.
		\item Also, current undergraduates can also sign up for programs to learn about marketing, finance, and entrepreneurship. They can also do summer research with USC researchers.
		\end{enumerate}
	\item Summer sports programs for youths: \vspace{-0.2cm}
		\begin{enumerate} \itemsep -2pt
		\item SC Futbol Academy (USC Soccer Camps): \url{http://www.usctrojans.com/sports/w-soccer/spec-rel/021610aaa.html}
		\item Mick Haley's USC Girls Volleyball Camp: \url{http://www.usctrojans.com/sports/w-volley/spec-rel/volley-camp.html}
		\item Salo Swim Camp: \url{http://www.saloswimcamp.com/on-line/default.asp}
		\item USC NYSP Trojan KidSCamp: \url{http://sait.usc.edu/recsports/site_content/youth_sports/nysptk.html}
		\item After School Sports Connection, ASSC (operates in fall, spring, and summer): \url{http://sait.usc.edu/recsports/site_content/youth_sports/assc.html}
		\end{enumerate}
	\end{enumerate}
\item Telluride Association: \vspace{-0.3cm}
	\begin{enumerate} \itemsep -2pt
	\item Telluride Association Summer Program (TASP) [ for high school students ]: \url{http://www.tellurideassociation.org/programs/high_school_students/tasp/tasp_general_info.html}
	\item Telluride Association Sophomore Seminar (TASS) [ for high school students ]: \url{http://www.tellurideassociation.org/programs/high_school_students/tass/tass_general_info.html}
	\item Resources for high school educators to nominate summer program applicants: \url{http://www.tellurideassociation.org/programs/high_school_students/hs_resources/hs_resources_general_information.html}
	\end{enumerate}
\item MathNerds: \vspace{-0.3cm}
	\begin{enumerate} \itemsep -2pt
	\item \url{http://www.mathnerds.com/}
	\item ``Provides free, discovery-based, mathematical guidance via an international, volunteer network of mathematicians.''
	\item If you have a mathematical problem to solve, you can ask mathematicans at {\it MathNerds} for help.
	\item They would require you to discuss your attempted approaches/solutions.
	\item If you have not made attempts to solve the problem, they will not give you much guidance.
	\item In addition, they cannot solve problems for you.
	\item They provide guidance for mathematical problems from K-12 material through undergraduate mathematics and statistics classes.
	\item They also provide help for selected topics in advanced mathematics classes (for graduate students).
	\item Other resources: \url{http://www.mathnerds.com/links/links.aspx}
	\end{enumerate}
\item Hobsons: \vspace{-0.3cm}
	\begin{enumerate} \itemsep -2pt
	\item CollegeView (Hobsons' college recruiting services): \url{http://www.collegeview.com/index.jsp}
	\end{enumerate}
\item Sponsors for Educational Opportunity (SEO): \vspace{-0.3cm}
	\begin{enumerate} \itemsep -2pt
	\item Resources: \url{http://www.seo-usa.org/ScholarsResources}
	\end{enumerate}
\item U.S. Department of Education: \vspace{-0.3cm}
	\begin{enumerate} \itemsep -2pt
	\item Students.gov: \url{http://www.students.gov/STUGOVWebApp/index.jsp}
	\item college.gov: \url{http://www.college.gov/wps/portal}
	\end{enumerate}
\item U.S. Department of State: \vspace{-0.3cm}
	\begin{enumerate} \itemsep -2pt
	\item Bureau of Educational and Cultural Affairs: \vspace{-0.2cm}
		\begin{enumerate} \itemsep -2pt
		\item EducationUSA: \vspace{-0.1cm}
			\begin{enumerate} \itemsep -1pt
			\item Information for international students: \url{http://www.educationusa.info/students.php}
			\end{enumerate}
		\end{enumerate}
	\end{enumerate}
\item Congressional Hispanic Caucus Institute (CHCI): \vspace{-0.3cm}
	\begin{enumerate} \itemsep -2pt
	\item CHCI Education Center: \vspace{-0.2cm}
		\begin{enumerate} \itemsep -2pt
		\item \url{http://www.chci.org/education_center/}
		\item Has resources on college planning, financial aid, scholarships, college internships, and housing.
		\item For Parents: \url{http://www.chci.org/education_center/page/for-parents}
		\item For Students: \url{http://www.chci.org/education_center/page/for-students}
		\end{enumerate}
	\end{enumerate}
\item My College Options: \vspace{-0.3cm}
	\begin{enumerate} \itemsep -2pt
	\item \url{http://www.mycollegeoptions.org/}
	\item ``My College Options is a FREE college planning service, offering assistance to students, parents, high schools, counselors, and teachers nationwide.''
	\item ``It is designed to assist high school students in exploring a wide range of post-secondary opportunities, with special emphasis on the college search process.''
	\end{enumerate}
\end{enumerate}

Resources for financial aid: \vspace{-0.3cm}
\begin{enumerate} \itemsep -4pt
\item {\it Guide to Online Schools} [or {\it GuideToOnlineSchools.com}], {\it Financial Aid}. Available at: \url{http://www.guidetoonlineschools.com/financial-aid}; last accessed on August 25, 2010.
\item The Institute for College Access \& Success, {\it Links} [ Resources that provide information about student loans and student debt ]. Available at: \url{http://projectonstudentdebt.org/links.vp.html}; last accessed on September 4, 2010. [ Also, see \url{http://projectonstudentdebt.org/advice.vp.html} and \url{http://ticas.org/about.vp.html}. ]
\end{enumerate}


Information about colleges and universities: \vspace{-0.3cm}
\begin{enumerate} \itemsep -4pt
\item The Institute for College Access \& Success, {\it College InSight}. Available at: \url{http://college-insight.org/}; last accessed on September 4, 2010.
\item 
\end{enumerate}



%%%%%%%%%%%%%%%%%%%%%%%%%%%%%%%%%%%%%%%%%%%
\section{Outreach for Students in Colleges and Universities}
\label{outreachcollege}

Resources to reach out to students in colleges and universities: \vspace{-0.3cm}
\begin{enumerate} \itemsep -4pt
%%%%%%%%%%%%%%%%%%%%%%%%%%%%%
\item Film contests: \vspace{-0.3cm}
	\begin{enumerate} \itemsep -2pt
	\item Ed Wood Film Festival [@ USC]: \vspace{-0.2cm}
		\begin{enumerate} \itemsep -2pt
		\item Celebrating independent filmmaking at USC and named for the famous director, the Ed Wood Film Festival is put on by a committee of Residential Education staff members at New Residential College, chaired by the Cinema Floor RA's.
		\item Teams of students come together to obtain the year's secret theme in which to write, shoot, and edit their very own short film within 24 hours. A week later, the films are shown at USC's Norris Cinema and a panel of judges selects the Festival winners in a variety of categories.
		\item \url{http://sait.usc.edu/resed/Programs.aspx}
		\end{enumerate}
	\item Reel LA: Parkside International Film Festival [or USC Reel LA Film Festival at USC]; see \url{http://www-scf.usc.edu/~pirc/areagov/government.php}
	\end{enumerate}
%%%%%%%%%%%%%%%%%%%%%%%%%%%%%
\item residential education: \vspace{-0.3cm}
	\begin{enumerate} \itemsep -2pt
	\item Telluride Association: \vspace{-0.2cm}
		\begin{enumerate} \itemsep -2pt
		\item Information about how to reside at the Cornell Branch (also known as Telluride House or CBTA) and the Michigan Branch of Telluride Association, which are ``residential colleges'': \url{http://www.tellurideassociation.org/programs/university_students.html}
		\item Awards for residents at the Cornell or Michigan Branch: \url{http://www.tellurideassociation.org/programs/university_students/us_awards.html}
		\end{enumerate}
	\end{enumerate}
%%%%%%%%%%%%%%%%%%%%%%%%%%%%%
\item MathNerds: \vspace{-0.3cm}
	\begin{enumerate} \itemsep -2pt
	\item \url{http://www.mathnerds.com/}
	\item ``Provides free, discovery-based, mathematical guidance via an international, volunteer network of mathematicians.''
	\item If you have a mathematical problem to solve, you can ask mathematicans at {\it MathNerds} for help.
	\item They would require you to discuss your attempted approaches/solutions.
	\item If you have not made attempts to solve the problem, they will not give you much guidance.
	\item In addition, they cannot solve problems for you.
	\item They provide guidance for mathematical problems from K-12 material through undergraduate mathematics and statistics classes.
	\item They also provide help for selected topics in advanced mathematics classes (for graduate students).
	\end{enumerate}
%%%%%%%%%%%%%%%%%%%%%%%%%%%%%
\item Invent Now: \vspace{-0.3cm}
	\begin{enumerate} \itemsep -2pt
	\item 
	\end{enumerate}
\item Journal of Young Investigators (JYI): \vspace{-0.3cm}
	\begin{enumerate} \itemsep -2pt
	\item \url{http://www.jyi.org/}
	\item ``peer-reviewed journal for undergraduates''
	\item ``JYI's web journal (which is also called JYI) is dedicated to the presentation of undergraduate research in science, mathematics, and engineering. It publishes the best submissions from undergraduates, with an emphasis on both the quality of research and the manner in which it is communicated. The journal, JYI, also allows students to experience the other side of the scientific publication process: the review process. Students working with their faculty advisors review the work of their peers and determine whether that work is acceptable for publication in JYI.''
	\end{enumerate}
\item The Recording Academy: \vspace{-0.3cm}
	\begin{enumerate} \itemsep -2pt
	\item GRAMMY U: \vspace{-0.2cm}
		\begin{enumerate} \itemsep -2pt
		\item \url{http://www.grammy365.com/grammy-u}
		\item GRAMMY U is a unique and fast-growing community of full-time college students, primarily between the ages of 17 and 25,  who are pursuing a career in the recording industry.
		\item The Recording Academy created GRAMMY U to help prepare college students for their careers in the music industry through networking, educational programs and performance opportunities.
		\item GRAMMY U is designed to enhance students' current academic curriculum with access to recording industry professionals to give an ``out of classroom'' perspective on the recording industry.
		\end{enumerate}
	\end{enumerate}
%%%%%%%%%%%%%%%%%%%%%%%%%%%%%
\item --- --- --- --- --- --- --- --- --- --- --- --- --- --- --- --- --- --- --- --- --- --- --- --- --- --- --- --- --- --- ---
\item \colorbox{blue}{\bf Help for Underrepresented Minorities}
% Help for Underrepresented Minorities
\item INROADS, Inc.: \vspace{-0.3cm}
	\begin{enumerate} \itemsep -2pt
	\item Internships: \url{http://www.inroads.org/interns/internWhatItTakes.jsp}
	\end{enumerate}
\item The PhD Project: \vspace{-0.3cm}
	\begin{enumerate} \itemsep -2pt
	\item \url{http://www.phdproject.org/index.html}
	\item Program and informational network to encourage ``African-Americans, Hispanic-Americans and Native Americans'' to pursue Ph.D. programs in business and seek careers in academia.
	\item Annual PhD Project Conference: \vspace{-0.2cm}
		\begin{enumerate} \itemsep -2pt
		\item Conference: \vspace{-0.1cm}
			\begin{enumerate} \itemsep -1pt
			\item \url{http://www.phdproject.org/conference.html}
			\item \url{http://www.phdproject.org/conference_application.html}
			\item For prospective Ph.D. students in business to learn more about Ph.D. programs in business, the Ph.D. application process, and life in graduate school.
			\item Registration Policy: \vspace{-0.1cm}
				\begin{itemize} \itemsep -1pt
				\item If you are selected to attend the conference you will be required to pay a \$200 registration fee which can be processed via credit card during the registration process. All travel and conferences expenses will paid by The PhD Project (total conference expenses for hotel, meals, materials, and transportation are valued at approximately \$1,500 per invited attendee.) Your investment of the \$200 registration fee will be refunded if you enter a full-time, AACSB accredited business doctoral program within 3 years of attending the conference. 
				\item If you previously attended a PhD Project Conference, you may submit an application to be reviewed, however if you are selected to attend, The PhD Project will only cover hotel costs (shared room with another participant). You will be required to pay the registration and travel costs
				\end{itemize}
			\end{enumerate}
		\item Resources for Potential/Current Doctoral Students: \vspace{-0.1cm}
			\begin{enumerate} \itemsep -1pt
			\item \url{http://www.phdproject.org/resources.html}
			\item Information about good business schools that offer Ph.D. programs, preparation for the GMAT, and the life in graduate school as a Ph.D. student.
			\item Suggested Reading: \vspace{-0.1cm}
				\begin{itemize} \itemsep -1pt
				\item \url{http://www.phdproject.org/reading.html}
				\item Has information life in graduate school as a Ph.D. student, racial diversity/issues in higher education, job searching in academia, and work-life balance for female Ph.D. students.
				\end{itemize}
			\end{enumerate}
		\item The PhD Project Doctoral Student Association (DSA): \vspace{-0.1cm}
			\begin{enumerate} \itemsep -1pt
			\item The PhD Project network: \vspace{-0.1cm}
				\begin{itemize} \itemsep -1pt
				\item \url{http://www.myphdnetwork.org/}
				\item ``There are 5 discipline specific associations covering the major areas of business education: Accounting, Finance, Information Systems, Management, Marketing.''
				\end{itemize}
			\end{enumerate}
		\end{enumerate}
	\end{enumerate}
\item MS-to-Ph.D. program for underrepresented minorities at Fisk and Vanderbilt in certain areas of
science (including astronomy, material science, and physics)
\item Outreach programs for underrepresented minorities to help them get into medical (and/or graduate) schools. Search for ``PREP (Post-baccalaureate Research Education Programs),'' which have stipends. E.g., Georgetown University School of Medicine, and George Washington University's medical school
\item New York University: \vspace{-0.3cm}
	\begin{enumerate} \itemsep -2pt
	\item Leonard N. Stern School of Business: \vspace{-0.2cm}
		\begin{enumerate} \itemsep -2pt
		\item Stern Pre-Doctoral program: \url{http://www.stern.nyu.edu/AcademicPrograms/PhD/Pre-Doctoral/index.htm}
		\end{enumerate}
	\end{enumerate}
\end{enumerate}



%%%%%%%%%%%%%%%%%%%%%%%%%%%%%%%%%%%%%%%%%%%
\section{Science \& Engineering Outreach}
\label{stemoutreach}

%%%%%%%%%%%%%%%%%%%%%%%%%%%%%%%%%%%%%%%%%%%
\subsection{Precollege Science \& Engineering Outreach}
\label{stemoutreachk12}

Science and engineering outreach to high-school (and middle-school) students, and their parents, teachers, and career counselors: \vspace{-0.3cm}
\begin{enumerate} \itemsep -4pt
\item {\it MentorNet}: \vspace{-0.3cm}
	\begin{enumerate} \itemsep -2pt
	\item \url{http://www.mentornet.net/}
	\item Enables people to network with scientists, engineers, and professors in Science, Technology, Engineering, and Mathematics (STEM)
	\item Is very supportive of minorities, so that more minorities (particularly underrepresented minorities) can be attracted to STEM careers
	\end{enumerate}
\item International Science Olympiad (for high school students): \vspace{-0.3cm}
	\begin{enumerate} \itemsep -2pt
	\item International Olympiad in Informatics: \url{http://en.wikipedia.org/wiki/International_Olympiad_in_Informatics} and \url{http://www.ioinformatics.org/index.shtml}
	\item International Mathematical Olympiad: \url{http://www.imo-official.org/}
	\item International Physics Olympiad: \url{http://www.jyu.fi/tdk/kastdk/olympiads/}
	\item International Chemistry Olympiad: \url{http://www.icho.sk/}
	\item International Biology Olympiad: \url{http://www.ibo-info.org/}
	\item \url{http://scienceolympiads.org/}
	\end{enumerate}
\item International Astronomy Olympiad: \url{http://www.issp.ac.ru/iao/}
\item International Earth Science Olympiad: \url{http://en.wikipedia.org/wiki/International_Earth_Science_Olympiad}
\item International Junior Science Olympiad (for students younger than 15 years old): \url{http://www.ijso-official.org/home}
\item Teen Leadership Institute Science, Technology, Engineering, and Math (STEM) programs @ YWCA Greater Pittsburgh; see \url{http://www.ywcapgh.org/STEM_Programs.asp}
\item For Inspiration and Recognition of Science and Technology (FIRST): \url{http://www.usfirst.org/} (including resources and guides to mentoring); scholarships @ \url{http://www.usfirst.org/aboutus/content.aspx?id=508}; and robotics programs @ \url{http://www.usfirst.org/roboticsprograms/frc/default.aspx?=966}
\item Mac Hyman, ``Good Choices for Great Careers in the Mathematical Sciences,'' talk given at 2008 SIAM Annual Meeting. Available at: \url{http://client.blueskybroadcast.com/siam08/hyman/index.html}; last accessed on August 25, 2010. Also, see \url{http://meetings.siam.org/program.cfm?CONFCODE=AN08}, \url{http://www.siam.org/meetings/an08/program.php}, and \url{http://www.siam.org/meetings/an08/}.
\item {\it RoboCup}\texttrademark\ competitions: \vspace{-0.2cm}
	\begin{enumerate} \itemsep -2pt
	\item Junior category for K-12 students involves contests the these areas of challenges: \vspace{-0.1cm}
		\begin{enumerate} \itemsep -1pt
		\item soccer
		\item dance
		\item rescue operations
		\end{enumerate}
	\item \url{http://www.robocup.org/}
	\end{enumerate}
\item {\it Curriki}, which is an online educational resource for teachers, students, and parents in K-12: \url{http://www.curriki.org/xwiki/bin/view/Main/About}
%%%%%%%%%%%%%%%%%%%%%%%%%%%%%%%%%%%%%%%%
%%%%%%%%%%%%%%%%%%%%%%%%%%%%%%%%%%%%%%%%
\item Electrical and computer engineering and/or computer science: \vspace{-0.2cm}
	\begin{enumerate} \itemsep -2pt
	\item {\it TopCoder} coding and design contests: \vspace{-0.2cm}
		\begin{enumerate} \itemsep -2pt
		\item High School category
		\item \url{http://www.topcoder.com/}
		\end{enumerate}
	\item Student Cluster Competition (SCC): \vspace{-0.2cm}
		\begin{enumerate} \itemsep -2pt
		\item SCC is held at each (annual) SC conference, which is the International Conference for High Performance Computing, Networking, Storage, and Analysis. IEEE Computer Society and the Association for Computing Machinery are the sponsors for this conference.
		\item During SC10, teams consisting of six students, undergraduate and/or high school, will showcase the amazing power of clusters and the ability to utilize open source software to solve interesting and important problems. They will compete in real-time on the exhibit floor to run a workload of real-world applications on clusters of their own design while never exceeding the dictated power limit.
		\item During SC10 in New Orleans, teams will assemble, test and tune their machines and run the HPCC benchmarks until the starting bell rings on Monday night at the Exhibit Opening Gala where they will be given the competition data sets. In full view of conference attendees, teams will execute the prescribed workload while showing progress and science visualization output on large high-resolution displays in their areas. Teams race to correctly complete the greatest number of application runs during the competition period until the close of the exhibit floor on Wednesday evening.
		\item \url{http://sc10.supercomputing.org/?pg=studentcluster.html}
		\end{enumerate}
	\item Institute of Electrical and Electronics Engineers, IEEE: \vspace{-0.3cm}
		\begin{enumerate} \itemsep -2pt
		\item {\it IEEE Educational Activities} recommended resources: \url{http://www.ieee.org/education_careers/education/preuniversity/resources/index.html}
		\item Engineering Projects in Community Service (EPICS) in IEEE: \vspace{-0.2cm}
			\begin{enumerate} \itemsep -2pt
			\item High school students collaborate with college students in engineering projects to benefit the community
			\item \url{http://www.ieee.org/education_careers/education/preuniversity/epics_high.html}
			\end{enumerate}
		\item Talk given by John Cohn at the IEEE International Symposium on Circuits and Systems (ISCAS), May 18-21, 2008. The talk is titled, ``Kids these days. How we can inspire the next generation of Engineers and Scientists?'' See \url{http://ewh.ieee.org/soc/icss/IEEE-ISCAS-08-Tue-Keynote-JC/IEEE-ISCAS-08-Tue-Keynote-JC.HTML}. [ Alternatively, go to: IEEE Circuits and Systems Society, \url{http://www.ieee-cas.org/}: Select the ``Resources'' tab in the menu bar, and select the ``ISCAS Keynote Videos'' option. Click on the video link with the appropriate title. ]
		\end{enumerate}
	\item Association for Computing Machinery (ACM): \vspace{-0.2cm}
		\begin{enumerate} \itemsep -2pt
		\item Sanjeev Arora, Boaz Barak, and Luca Trevisan, ``Survey Papers and Essays,'' in {\it Theory Matters Wiki: Theoretical Computer Science (TCS) Advocacy Wiki}, SIGACT Committee for the Advancement of Theoretical Computer Science, ACM Special Interest Group on Algorithms and Computation Theory (SIGACT), Association for Computing Machinery, February 25, 2010. Available at: \url{http://theorymatters.org/pmwiki/pmwiki.php?n=Main.SurveyCollection}; last accessed on September 14, 2010.
		\end{enumerate}
	\item WGBH Educational Foundation: \vspace{-0.2cm}
		\begin{enumerate} \itemsep -2pt
		\item Dot Diva / New Image for Computing (NIC) initiative: \vspace{-0.1cm}
			\begin{enumerate} \itemsep -1pt
			\item \url{http://dotdiva.org/}
			\item Resource for parents and teachers: \url{http://dotdiva.org/parents.html}
			\end{enumerate}
		\end{enumerate}
	\item Silicon Valley StRUT: \vspace{-0.2cm}
		\begin{enumerate} \itemsep -2pt
		\item Students Recycling Used Technology, StRUT, Competition; StRUT Competition consists of: \vspace{-0.1cm}
			\begin{enumerate} \itemsep -1pt
			\item Disassemble and Reassemble A Computer 
			\item Create and Present a Powerpoint Presentation 
			\item Computer Parts Identification and Challenge Test  
			\item Team Quiz Bowl on Computer Technology and Related Subjects
			\item \url{http://www.svstrut.org/cms/content/section/1/5/}
			\item Teacher Resources: \url{http://www.svstrut.org/cms/component/option,com_weblinks/catid,11/Itemid,10/}
			\item [ Resources to Support ] Curriculum for Engineering and Computer Technology Education: \url{http://www.svstrut.org/cms/content/view/8/18/}
			\end{enumerate}
		\item \url{http://www.svstrut.org/cms/}
		\end{enumerate}
	\item Google Code Jam (programming contest): \url{http://code.google.com/codejam/} and \url{http://en.wikipedia.org/wiki/Google_Code_Jam}
	\item University of Illinois at Urbana-Champaign (UIUC): \vspace{-0.2cm}
		\begin{enumerate} \itemsep -2pt
		\item College of Engineering; Department of Computer Science: \vspace{-0.1cm}
			\begin{enumerate} \itemsep -1pt
			\item Outreach \& Diversity: \url{http://cs.illinois.edu/outreach}
			\item ChicTech: \url{http://cs.illinois.edu/outreach/chictech}
			\item Technical Ambassadors: \url{http://cs.illinois.edu/outreach/tac}
			\item Games4Girls: \url{http://cs.illinois.edu/outreach/games4girls}
			\item Workshops \& Camps: \url{http://cs.illinois.edu/outreach/k12}
			\item \url{http://cs.illinois.edu/outreach}
			\end{enumerate}
		\end{enumerate}
	\item Carnegie Mellon University: \vspace{-0.2cm}
		\begin{enumerate} \itemsep -2pt
		\item women@SCS School of Computer Science, Carnegie Mellon University: \vspace{-0.1cm}
			\begin{enumerate} \itemsep -1pt
			\item Papers: \url{http://women.cs.cmu.edu/Resources/Papers/}
			\item Alumnae Interviews / Profiles: \url{http://women.cs.cmu.edu/Who/Alumnae/alumInterviews.php}
			\item Job and Research Opportunities: \url{http://www.women.cs.cmu.edu/Resources/JobsResearch/}
			\item Career Advice: \url{http://women.cs.cmu.edu/Resources/JobsResearch/careeradvice.php}
			\item Other Sites: \url{http://www.women.cs.cmu.edu/Miscellaneous/Other/}
			\end{enumerate}
		\end{enumerate}
	\item {\it Quora}: \vspace{-0.2cm}
		\begin{enumerate} \itemsep -2pt
		\item ``If a 10-year-old wanted to start programming today, what language path would be the most valuable moving forward?'' Available online at: \url{http://www.quora.com/If-a-10-year-old-wanted-to-start-programming-today-what-language-path-would-be-the-most-valuable-moving-forward}; last accessed on November 23, 2010.
		\end{enumerate}
	\end{enumerate}
%%%%%%%%%%%%%%%%%%%%%%%%%%%%%%%%%%%%%%%%
%%%%%%%%%%%%%%%%%%%%%%%%%%%%%%%%%%%%%%%%
\item Engineering Education Service Center (EESC): \vspace{-0.3cm}
	\begin{enumerate} \itemsep -2pt
	\item Has lists of: \vspace{-0.2cm}
		\begin{enumerate} \itemsep -2pt
		\item Educational material: \vspace{-0.1cm}
			\begin{enumerate} \itemsep -1pt
			\item books
			\item DVDs
			\item resource kits for teachers
			\end{enumerate}
		\item engineering camps (for the summer in the United States): \url{http://www.engineeringedu.com/camps/}
		\item {\it Women in Engineering} programs at US engineering schools: \url{http://www.engineeringedu.com/wie.html}
		\item US engineering schools: \url{http://www.engineeringedu.com/engrschools.htm}
		\item competitions for youths, including high school students: \url{http://www.engineeringedu.com/competitions.html}
		\item online resources
		\item list of professional organizations in engineering (or engineering societies): \url{http://www.engineeringedu.com/soc1.html}
		\item scholarships: \url{http://www.engineeringedu.com/scholars.html}
		\end{enumerate}
	\item It has resources for K-12 students, and their teachers and parents. It also has information for girls who are seeking careers in engineering; in addition, it provides their parents and teachers with information to guide the girls.
	\item It runs a workshop (in the US) for mother-daughter pairs to encourage girls to pursue careers in engineering.
	\item \url{http://www.engineeringedu.com/}
	\end{enumerate}
\item TryNano.org: \vspace{-0.3cm}
	\begin{enumerate} \itemsep -2pt
	\item Information about educational opportunities and careers in nanotechnology and nanoscience
	\item \url{TryNano.org}
	\end{enumerate}
\item {\it Mathematical Association of America} (MAA): \vspace{-0.3cm}
	\begin{enumerate} \itemsep -2pt
	\item Middle/High School Students: \url{http://www.maa.org/students/middle_high/}
	\item Parents: \url{http://www.maa.org/students/Parents.html}
	\item MAA American Mathematics Competitions: \vspace{-0.2cm}
		\begin{enumerate} \itemsep -2pt
		\item {\it Students} [resources]. Available at: \url{http://amc.maa.org/a-activities/a4-for-students/s-index.shtml}; last accessed on September 2, 2010.
		\item It includes tips to help students do well in math contests and Olympiads, a reading list for students interested in mathematics, problems from past math contests and Olympiads, and other resources from the World Wide Web.
		\end{enumerate}
	\item {\it Fun Math Sites}. Available at: \url{http://www.maa.org/students/funsites.html}; last accessed on September 2, 2010.
	\item Special Interest Group on Mathematics and the Arts (SIGMAA-ARTS): Resources, see \url{http://myweb.cwpost.liu.edu/aburns/sigmaa-arts/resources.html}.
	\item Special Interest Group of the MAA on Quantitative Literacy (SIGMAA QL): \url{http://sigmaa.maa.org/ql/}
	\end{enumerate}
\item eGFI (Engineering, Go For It!): \vspace{-0.3cm}
	\begin{enumerate} \itemsep -2pt
	\item Provides information for students, parents, and teachers about educational pathways and careers in engineering
	\item \url{http://egfi-k12.org/}
	\end{enumerate}
\item {\it Sloan Career Cornerstone Center}: \vspace{-0.3cm}
	\begin{enumerate} \itemsep -2pt
	\item Career exploration resources in STEM (science, technology, engineering, mathematics, computing, and healthcare)
	\item \url{http://www.careercornerstone.org/}
	\end{enumerate}
\item {\it TryEngineering}: \vspace{-0.3cm}
	\begin{enumerate} \itemsep -2pt
	\item Career exploration resources for engineering
	\item \url{http://www.tryengineering.org/}
	\end{enumerate}
\item {\it Junior Engineering Technical Society, JETS}: \vspace{-0.3cm}
	\begin{enumerate} \itemsep -2pt
	\item Career exploration resources for engineering
	\item \url{http://www.jets.org/}
	\end{enumerate}
\item {\it American Society of Mechanical Engineers, ASME}: \vspace{-0.3cm}
	\begin{enumerate} \itemsep -2pt
	\item K-12 Student Resources: \url{http://www.asme.org/Communities/Students/K12/} and \url{http://www.asme.org/Education/PreCollege/EngineeringResources/}
	\item Engineering Camps: \url{http://www.asme.org/Communities/Students/K12/Camps.cfm}
	\end{enumerate}
\item BESTRobotics, Inc.: \vspace{-0.3cm}
	\begin{enumerate} \itemsep -2pt
	\item BEST (Boosting Engineering, Science, and Technology) competition: \vspace{-0.2cm}
		\begin{enumerate} \itemsep -2pt
		\item \url{http://best.eng.auburn.edu/}
		\item Hosted at Auburn University's Samuel Ginn College of Engineering
		\item BEST World Championship: \url{http://best.eng.auburn.edu/world-championship/}
		\end{enumerate}
	\end{enumerate}
\item {\it American Society of Civil Engineers, ASCE}: \vspace{-0.3cm}
	\begin{enumerate} \itemsep -2pt
	\item Outreach resource for K-12 students, and their parents and teachers
	\item \url{http://content.asce.org/asceville/index.html}
	\end{enumerate}
\item {\it Science.gov} (USA.gov for Science): Internship and Fellowship Opportunities in Science (for high school students); see \url{http://www.science.gov/internships/k-12.html}
\item {\it iTunes U}: \vspace{-0.3cm}
	\begin{enumerate} \itemsep -2pt
	\item {\it iTunes} is required to listen to or watch these lectures, talks, and presentations.
	\item Access {\it iTunes U} at: \url{http://deimos3.apple.com/indigo/main/main.html?v0=WWW-AMUS-ITUNESU070521-N48LX}
	\item WGBH's Teachers' Domain -- Boston's PBS Station: Video presentation on ``Engineering for the Red Planet''; see \url{http://deimos3.apple.com/WebObjects/Core.woa/Browse/wgbh.org.1416254059.01416254061.1416793683?i=1951581658}. Also, check out its video clip on ``Carbon Fiber Car of the Future''.
	\item {\it iTunes U} is a set of webcast and podcasts, where you can easily find audio and video clips for lectures, seminars, announcements, virtual tours, and so on. For example, some professors from schools like MIT or Berkeley will provide audio/video clips of their lectures on {\it iTunes U}.
	\item This can help in exploring different majors during the college application process, or before a college student declares her/his major(s). If a student is not sure if she/he wants to double major in deaf studies and linguistics, this student can check out some linguistics lectures from her/his (preferred) college/university, if it uses {\it iTunes U}, or those from other universities.
	\end{enumerate}
\item High School Ace's College Prep Guide: \url{http://highschoolace.com/ace/colleges.cfm}
\item {\it Dr. Sally Ride} (America�s first woman in space): \vspace{-0.3cm}
	\begin{enumerate} \itemsep -2pt
	\item {\it Sally Ride Science}'s resources for educators: \url{https://www.sallyridescience.com/for_educators}
	\item Sally Ride Science Educator Institutes (to educate K-12 teachers about science): \url{https://www.sallyridescience.com/for_educators/institutes}
	\item {\it Sally Ride Science Academy} helps teachers to increase their students' interest in science: \url{https://www.sallyridescience.com/academy}
	\item {\it Sally Ride Science}'s resources for teachers: \url{https://www.sallyridescience.com/resources}
	\item {\it Sally Ride Science Festivals} are events for girls from the $5^{th}$ grade to the $8^{th}$ grade: \url{https://www.sallyridescience.com/festivals}
	\item {\it Sally Ride Science Camps} are summer camps for girls from the $4^{th}$ grade to the $9^{th}$ grade: \url{http://www.sallyridecamps.com/}
	\item GRAIL MoonKAM: \vspace{-0.2cm}
		\begin{enumerate} \itemsep -2pt
		\item ``GRAIL MoonKAM (Moon Knowledge Acquired by Middle school students) is GRAIL's signature education and public outreach program.''
		\item ``GRAIL MoonKAM will engage middle schools across the country in the GRAIL mission and lunar exploration.''
		\item \url{https://www.grailmoonkam.com/}
		\end{enumerate}
	\item EarthKAM: \vspace{-0.2cm}
		\begin{enumerate} \itemsep -2pt
		\item EarthKAM (Earth Knowledge Acquired by Middle school students) is a NASA educational outreach program enabling students, teachers and the public to learn about Earth from the unique perspective of space.
		\item \url{https://earthkam.ucsd.edu/}
		\end{enumerate}
	\end{enumerate}
\item Andrew Rader Studios: \vspace{-0.3cm}
	\begin{enumerate} \itemsep -2pt
	\item Chem4Kids.com: \url{http://www.chem4kids.com/}
	\end{enumerate}
\item {\it American Association for the Advancement of Science, AAAS}: \vspace{-0.3cm}
	\begin{enumerate} \itemsep -2pt
	\item ENTRY POINT! for Students With Disabilities (in STEM): \url{http://www.aaas.org/careercenter/fellowships/} and \url{http://ehrweb.aaas.org/entrypoint/}
	\item AAAS Mass Media Science \& Engineering Fellows Program (for STEM grad students to intern in mass media companies): \url{http://www.aaas.org/programs/education/MassMedia/}
	\item Diversity Issues: \url{http://sciencecareers.sciencemag.org/career_magazine/diversity_issues/}
	\item Internships involving science and journalism, human rights, scientific freedom, responsibility, or law: \url{http://www.aaas.org/careercenter/} and \url{http://www.aaas.org/careercenter/internships/scienceminority.shtml} (AAAS Minority Science Writers Internship)
	\item Kinetic City: \url{http://www.kineticcity.com/}
	\end{enumerate}
\item {\it NASA} resources for students: \url{http://www.nasa.gov/audience/forstudents/index.html} and \url{http://www.nasa.gov/offices/education/programs/national/summer/education_resources/index.html} (NASA Summer of Innovation)
\item National Academy of Engineering, NAE: \vspace{-0.3cm}
	\begin{enumerate} \itemsep -2pt
	\item NAE Grand Challenges: \vspace{-0.2cm}
		\begin{enumerate} \itemsep -2pt
		\item Includes a list of NAE Grand Challenges, which indicate some of the challenges faced by people on a global scale that can be partially solved by engineers. This can be used to get children and youths to be excited about engineering.
		\item NAE Grand Challenges: \vspace{-0.1cm}
			\begin{enumerate} \itemsep -1pt
			\item Make solar energy economical
			\item Provide energy from fusion
			\item Develop carbon sequestration methods
			\item Manage the nitrogen cycle
			\item Provide access to clean water
			\item Restore and improve urban infrastructure
			\item Advance health informatics
			\item Engineer better medicines
			\item Reverse-engineer the brain
			\item Prevent nuclear terror
			\item Secure cyberspace
			\item Enhance virtual reality
			\item Advance personalized learning
			\item Engineer the tools of scientific discovery
			\end{enumerate}
		\item \url{http://www.engineeringchallenges.org/}
		\item NAE Grand Challenge K12 Partners Program: \vspace{-0.1cm}
			\begin{enumerate} \itemsep -1pt
			\item \url{http://www.grandchallengek12.org/about}
			\item 5-Part Make it Happen Plan: \url{http://www.grandchallengek12.org/5-part-plan}
			\end{enumerate}
		\end{enumerate}
	\item {\it National Academy of Engineering}'s technological literacy program for people (students, parents, and educators) to learn more about technology: \url{http://www.nae.edu/nae/techlithome.nsf}
	\item Greatest Engineering Achievements: \url{http://www.greatachievements.org/}
	\end{enumerate}
\item National Science Foundation: \vspace{-0.3cm}
	\begin{enumerate} \itemsep -2pt
	\item Broadening Participation in Computing (BPC): \vspace{-0.2cm}
		\begin{enumerate} \itemsep -2pt
		\item \url{http://www.bpcportal.org/}
		\item \url{http://www.bpcportal.org/bpc/shared/home.jhtml;jsessionid=0MIUYDR5U4ARXABAVRSSFEQ?_requestid=9445}
		\item \url{http://www.nsf.gov/funding/pgm_summ.jsp?pims_id=13510}
		\item \url{http://www.nsf.gov/funding/pgm_summ.jsp?pims_id=13510&org=NSF&sel_org=NSF&from=fund}
		\item ``Broadening Participation in Computing (BPC) is a NSF sponsored program with the goal of significantly increasing the number of underrepresented graduates in the computing disciplines, with an emphasis on women, persons with disabilities, and minorities (African Americans, Hispanics, American Indians, Alaska Natives, Native Hawaiians, and Pacific Islanders).''
		\item Broadening Participation in Computing Digital Library: \vspace{-0.1cm}
			\begin{enumerate} \itemsep -1pt
			\item \url{http://www.bpcportal.org/bpc/interdiscipline/dl_index.jhtml;jsessionid=ROYEHJV1UQYWNABAVRSSFEQ?comm=BPC}
			\item Includes resources for different target populations: \vspace{-0.1cm}
				\begin{itemize} \itemsep -1pt
				\item Women
				\item African Americans
				\item Hispanic Americans, or Latinas and Latinos
				\item People with disabilities
				\item Native Americans
				\end{itemize}
			\item It also includes resources for different topics, such as mentoring, recruitment, retention, and work-life balance.
			\end{enumerate}
		\item Alliances (other professional organizations): \url{http://www.bpcportal.org/bpc/comm/projects.jhtml}
		\end{enumerate}
	\item The National Science Digital Library (NSDL): \vspace{-0.2cm}
		\begin{enumerate} \itemsep -2pt
		\item \url{http://www.nsdl.org/} and \url{http://www.nsdl.org/browse/}
		\item ``The National Science Digital Library is a national network dedicated to advancing STEM teaching and learning for all learners, in both formal and informal settings, and the locus of activity for the National Science Foundation's National STEM Distributed Learning program.''
		\item Outreach materials: \vspace{-0.1cm}
			\begin{enumerate} \itemsep -1pt
			\item \url{http://www.nsdl.org/pd/?pager=materials}
			\item Has outreach materials for educators in K-12 and higher educational institutions.
			\end{enumerate}
		\item Resources for K-12 Teachers: \url{http://nsdl.org/resources_for/k12_teachers/}
		\item Resources for Librarians: \url{http://nsdl.org/resources_for/librarians/}
		\item Billingual Resources: \url{http://www.nsdlnetwork.org/collections/billingual-resources}
		\item NSDL on {\it iTunes U}: \url{http://www.nsdl.org/iTunesU/}
		\item Collections: \url{http://www.nsdl.org/browse/?subject=All}
		\item NSDL Pathways: \vspace{-0.1cm}
			\begin{enumerate} \itemsep -1pt
			\item \url{http://nsdl.org/about/?pager=pathways}
			\item ``Pathways are large projects that are aggregators and stewards of resources and services to broad categories of users---either discipline-focused (e.g. chemistry), or audience-focused (e.g. middle school educators), or resources of a specific type or format (e.g. multimedia content).''
			\item ``They are digital library portals developed and managed in partnership with organizations and institutions that have a history and expertise in serving their portal's target audiences.''
			\item ``They contribute metadata (descriptive information) about their resources to NSDL to make their resources searchable and discoverable via the NSDL.org portal, in addition to their own portals.''
			\end{enumerate}
		\item {\bf NSDL Science Literacy Maps}: \vspace{-0.1cm}
			\begin{enumerate} \itemsep -1pt
			\item \url{http://strandmaps.nsdl.org/}
			\item ``{\it NSDL Science Literacy Maps} are a tool for teachers and students to find resources that relate to specific science and math concepts. The maps illustrate connections between concepts as well as how concepts build upon one another across grade levels.''
			\end{enumerate}
		\item NSDL Professional Development: \url{http://www.nsdl.org/pd/}
		\item NSDL Technical Network Services: \vspace{-0.1cm}
			\begin{enumerate} \itemsep -1pt
			\item \url{http://www.nsdl.org/about/?pager=tns}
			\item \url{http://nsdlnetwork.org/}
			\item \url{http://nsdlnetwork.org/content/book/page/953/about-nsdl-technical-network-services}
			\end{enumerate}
		\item NSDL Resource Center: \url{http://nsdlnetwork.org/content/book/951/page/954/about-nsdl-resource-center}
		\end{enumerate}
	\end{enumerate}
\item {\it American Chemical Society} Science for Kids program (for students in K-12): \url{http://portal.acs.org/portal/acs/corg/content?_nfpb=true&_pageLabel=PP_TRANSITIONMAIN&node_id=878&use_sec=false&sec_url_var=region1&__uuid=984d4ee7-4214-4d35-9899-bc2f91dee58b}
\item {\it California Digital Educator Consortium}, ``Digital Educator,'' Digital Learning Center: \url{http://www.digitaleducator.com/}
\item Kenny Felder, ``Selected Other Educational Sites on the Web''. Available at: \url{http://www4.ncsu.edu/unity/lockers/users/f/felder/public/kenny/edulinks.html}; last accessed on August 28, 2010.
\item FHSST (Free High School Science Texts); free textbooks for grades 10-12 in Physics, Chemistry, and Mathematics. Available at: \url{http://www.fhsst.org/}; last accessed on August 28, 2010.
\item John Baez, {\it Usenet Physics FAQ}, Department of Mathematics, University of California, Riverside, September 2009. Available at: \url{http://math.ucr.edu/home/baez/physics/}; last accessed on August 28, 2010.
\item {\it American Society for Engineering Education}: \vspace{-0.3cm}
	\begin{enumerate} \itemsep -2pt
	\item Science and Engineering Apprenticeship Program (SEAP): \vspace{-0.2cm}
		\begin{enumerate} \itemsep -2pt
		\item ``The Science and Engineering Apprenticeship Program (SEAP) provides an opportunity for students to participate in research at a Department of Navy (DoN) laboratory during the summer.''
		\item ``The goals of SEAP are to encourage participating students to pursue science and engineering careers, to further their education via mentoring by laboratory personnel and their participation in research, and to make them aware of DoN Research and technology efforts, which can lead to employment within the DoN.''
		\item ``High school students who have completed at least Grade 9. A graduating senior is eligible to apply.''
		\item ``Must be 16 years of age for most laboratories. Some laboratories may accept a 15 year old applicant. Please check individual lab description for more details.''
		\item ``Applicants must be US citizens and participation by Permanent Resident Aliens is limited. Please check individual lab descriptions for participation of Permanent Resident Aliens.''
		\item \url{http://seap.asee.org/}
		\end{enumerate}
	\end{enumerate}
\item robots.net, {\it Robot Competitions} (list of robot competitions and contests) : \url{http://robots.net/rcfaq.html}
\item International Council on Systems Engineering (INCOSE): \vspace{-0.3cm}
	\begin{enumerate} \itemsep -2pt
	\item Careers in Systems Engineering: \url{http://www.incose.org/educationcareers/careersinsystemseng.aspx}
	\item Frequently Asked Questions for Students [about Systems Engineering]: \url{http://www.incose.org/educationcareers/faqsforstudents.aspx}
	\item What is Systems Engineering?: \url{http://www.incose.org/practice/whatissystemseng.aspx}
	\end{enumerate}
\item {\it National Society of Professional Engineers}: \vspace{-0.3cm}
	\begin{enumerate} \itemsep -2pt
	\item A Sightseer's Guide to Engineering: \url{http://www.engineeringsights.org/}
	\end{enumerate}
\item {\it Engineers Dedicated to a Better Tomorrow (a.k.a., DedicatedEngineers)}: \vspace{-0.3cm}
	\begin{enumerate} \itemsep -2pt
	\item The ``K-12 Crowd'' (Students, Teachers, Guidance Counselors and Parents): \url{http://www.dedicatedengineers.org/intro_for_K-12.htm}
	\item \url{http://www.dedicatedengineers.org/}
	\end{enumerate}
\item National Engineers Week Foundation: \vspace{-0.3cm}
	\begin{enumerate} \itemsep -2pt
	\item Discover Engineering: \url{http://www.discoverengineering.org/}
	\item Introduce A Girl to Engineering: \url{http://www.eweek.org/EngineersWeek/IntroduceAGirl.aspx}
	\item All About Engineering: \url{http://www.eweek.org/AboutEngineering/AboutEngineering.aspx}
	\end{enumerate}
\item University of California: \vspace{-0.3cm}
	\begin{enumerate} \itemsep -2pt
	\item The Coalition For Science After School: \vspace{-0.2cm}
		\begin{enumerate} \itemsep -2pt
		\item \url{http://afterschoolscience.org/}
		\item ``Promoting high-quality afterschool science'' ... ``The Coalition for Science After School envisions the day when young people from all backgrounds have access to high-quality science, technology, engineering and mathematics (STEM) learning beyond the classroom.''
		\item Tools for advocates--Championing afterschool science: \url{http://afterschoolscience.org/tools/}
		\item Program resources--Enhancing the quality of afterschool opportunities: \url{http://afterschoolscience.org/resources/}
		\item The National After School Science Directory: \vspace{-0.1cm}
			\begin{enumerate} \itemsep -1pt
			\item \url{http://afterschoolscience.org/directory/}
			\item ``The National After School Science Directory is a searchable database designed to increase access to high-quality science, technology, engineering and math (STEM) education beyond the classroom for youth and families across the nation. The Directory houses thousands of STEM opportunities, submitted by science centers, museums, schools and other youth-serving organizations. Search our Directory to view opportunities to connect the America's youth to high-quality STEM learning experiences.''
			\end{enumerate}
		\item Become an advocate: \url{http://afterschoolscience.org/tools/advocate.php}
		\item Funders (funding organizations/agencies): \url{http://afterschoolscience.org/tools/funders.php}
		\end{enumerate}
	\end{enumerate}
\item Harvey Mudd College: \vspace{-0.3cm}
	\begin{enumerate} \itemsep -2pt
	\item Francis Edward Su, {\it Math Fun Facts!}, Department of Mathematics, Harvey Mudd College: \url{http://www.math.hmc.edu/funfacts/}
	\end{enumerate}
\item Clay Mathematics Institute: \vspace{-0.3cm}
	\begin{enumerate} \itemsep -2pt
	\item Program in Mathematics for Young Scientists, PROMYS: \vspace{-0.2cm}
		\begin{enumerate} \itemsep -2pt
		\item \url{http://www.claymath.org/programs/outreach/PROMYS/}
		\item \url{http://math.bu.edu/people/promys/}
		\item \url{http://www.promys.org/}
		\end{enumerate}
	\item Ross Program (for pre-college students): \vspace{-0.2cm}
		\begin{enumerate} \itemsep -2pt
		\item \url{http://www.claymath.org/programs/outreach/ross/}
		\item \url{http://www.math.ohio-state.edu/ross/}
		\end{enumerate}
	\item CMI Summer Schools: \url{http://www.claymath.org/programs/summer_school/}
	\end{enumerate}
\item Consortium for Ocean Leadership: \vspace{-0.3cm}
	\begin{enumerate} \itemsep -2pt
	\item Oceans of Opportunity (for African American students in K-12, and colleges and universities -- includes undergraduates and grad students): \url{http://www.oceanleadership.org/education/diversity/oceans-of-opportunity/}
	\item The JOIDES Resolution (The JR) scientific research vessel [ Deep Earth Academy ]: \vspace{-0.2cm}
		\begin{enumerate} \itemsep -2pt
		\item Fun \& Games: \url{http://joidesresolution.org/node/53}
		\item Discovery Center: \url{http://joidesresolution.org/node/44}
		\item Just for Kids Blog: \url{http://joidesresolution.org/node/366}
		\end{enumerate}
	\item National Ocean Sciences Bowl (high school academic competition that provides a forum for talented students to test their knowledge of the marine sciences including biology, chemistry, physics, and geology): \vspace{-0.2cm}
		\begin{enumerate} \itemsep -2pt
		\item \url{http://www.nosb.org/}
		\item Career Resources: \url{http://www.nosb.org/ocean-careers/career-resources/}
		\end{enumerate}
	\item Integrated Ocean Drilling Program (IODP), IODP United States Implementing Organization (IODP-USIO): \vspace{-0.2cm}
		\begin{enumerate} \itemsep -2pt
		\item U.S.-sponsored Teacher at Sea Program (for US teachers to participate in seagoing research experiences aboard the JOIDES Resolution): \url{http://www.iodp-usio.org/Education/TAS.html}
		\end{enumerate}
	\item Careers: \url{http://www.oceanleadership.org/education/deep-earth-academy/students/careers/}
	\end{enumerate}
\item The Oceanography Society: \vspace{-0.3cm}
	\begin{enumerate} \itemsep -2pt
	\item Careers in Oceanography: Profiles, \url{http://www.tos.org/resources/career_profiles.html}
	\item Links [includes links to educational material for students in K-12]: \url{http://www.tos.org/resources/links.html}
	\end{enumerate}
\item American Geophysical Union: \vspace{-0.3cm}
	\begin{enumerate} \itemsep -2pt
	\item Bright Students Training as Research Scientists (Bright STaRS): \vspace{-0.2cm}
		\begin{enumerate} \itemsep -2pt
		\item \url{http://www.agu.org/education/diversity_programs/bstars.shtml}
		\item ``High school students participating in after-school and summer research experiences in the Earth and space sciences are invited to participate in the AGU Bright STaRS program. The Bright STaRS program provides a dedicated forum for $\sim$50 students to present their own research results to the scientific community and learn about exciting research, education, and career opportunities in the geosciences.''
		\end{enumerate}
	\end{enumerate}
\item American Geological Institute, AGI: \vspace{-0.3cm}
	\begin{enumerate} \itemsep -2pt
	\item AGI Education Department: \url{http://www.agiweb.org/geoeducation.html}
	\end{enumerate}
\item Society for Science \& the Public (SSP): \vspace{-0.3cm}
	\begin{enumerate} \itemsep -2pt
	\item Intel International Science \& Engineering Fair (Intel ISEF), which is a pre-college science competition: \url{http://www.societyforscience.org/isef/}
	\item Broadcom MASTERS\texttrademark\ competition (which stands for Broadcom Math, Applied Science, Technology and Engineering for Rising Stars): \vspace{-0.2cm}
		\begin{enumerate} \itemsep -2pt
		\item Is a U.S. ``national science, technology, engineering, and math competition for America's $6^{th}$, $7^{th}$, and $8^{th}$ graders.''
		\item \url{http://www.societyforscience.org/masters} or \url{http://www.broadcomfoundation.org/masters/}
		\end{enumerate} 
	\item Science resources: \url{http://www.societyforscience.org/resources}
	\item Science News: \url{http://www.sciencenews.org/}
	\item Science News for Kids (for ``children of ages 9-14, their teachers and their parents''): \url{http://www.societyforscience.org/sciencenewsforkids} and \url{http://www.sciencenewsforkids.org/}
	\end{enumerate}
\item Institute for Operations Research and the Management Sciences (INFORMS): \vspace{-0.3cm}
	\begin{enumerate} \itemsep -2pt
	\item Operations Research: The Science of Better, \url{http://www.scienceofbetter.org/}
	\end{enumerate}
\item Technion - Israel Institute of Technology: \vspace{-0.3cm}
	\begin{enumerate} \itemsep -2pt
	\item SciTech - the summer camp for talented students ($11^{th}$ and $12^{th}$ graders from all over the world): \url{http://www.scitech.technion.ac.il/}
	\end{enumerate}
\item USA Science \& Engineering Festival: \url{http://www.usasciencefestival.org/}
\item Girl Scouts: \vspace{-0.3cm}
	\begin{enumerate} \itemsep -2pt
	\item Girl Scouts of Western New York: \vspace{-0.2cm}
		\begin{enumerate} \itemsep -2pt
		\item STEM Resource Guide: \url{http://www.gswny.org/Data/Documents/STEM%2520Resource%2520Guide%25202010-Oct-11.pdf}
		\item Also, see \url{http://www.gswny.org/Programs/Awards/Gold/}; scroll to the bottom of the page and look under the subsection heading, ``Tell Us About Your Gold Award Project''
		\end{enumerate}
	\item Science, Technology, Engineering and Math (STEM): \url{http://www.girlscouts.org/program/program_opportunities/science/}
	\end{enumerate}
\item American Museum of Science and Energy (AMSE): \vspace{-0.3cm}
	\begin{enumerate} \itemsep -2pt
	\item \url{http://www.amse.org/}
	\item Owned by the US Department of Energy, and managed under Oak Ridge National Laboratory
	\item Educators: \url{http://www.amse.org/content.aspx?article=1140&parent=30}
	\item Educational Programs: \url{http://www.amse.org/content.aspx?article=1139&parent=30}
	\item Home school programs: \url{http://www.amse.org/content.aspx?article=1169&parent=30}
	\item Online resources: \url{http://www.amse.org/content.aspx?article=1170&parent=30}
	\end{enumerate}
\item Center for Energy Workforce Development (CEWD): \vspace{-0.3cm}
	\begin{enumerate} \itemsep -2pt
	\item Teachers and guidance counselors: \vspace{-0.2cm}
		\begin{enumerate} \itemsep -2pt
		\item \url{http://www.cewd.org/educators_index.asp}
		\item Lesson plans for teachers: \url{http://www.cewd.org/educators_lessonplans.asp}
		\end{enumerate}
	\item Parents: \url{http://www.cewd.org/parents_index.asp}
	\end{enumerate}
\item TryScience: \url{http://tryscience.net/tryscinetmain.nsf/Welcome?OpenPage}
\item The Dana Foundation: \vspace{-0.3cm}
	\begin{enumerate} \itemsep -2pt
	\item Brainy Kids: \vspace{-0.2cm}
		\begin{enumerate} \itemsep -2pt
		\item \url{http://www.dana.org/resources/brainykids/}
		\item Fun: \vspace{-0.1cm}
			\begin{enumerate} \itemsep -1pt
			\item \url{http://dana.org/resources/brainykids/detail.aspx?folder_id=104}
			\item Has interactive online games, activities, and fun quizzes on: \vspace{-0.1cm}
				\begin{itemize} \itemsep -1pt
				\item biology
				\item health
				\item neuroscience
				\item astronomy
				\item chemistry
				\item ecology
				\end{itemize}
			\end{enumerate}
		\item The Lab: \vspace{-0.1cm}
			\begin{enumerate} \itemsep -1pt
			\item \url{http://dana.org/resources/brainykids/detail.aspx?folder_id=106}
			\item Has maps of the brain, virtual dissections, resources for science fairs, and virtual microscopes
			\end{enumerate}
		\item Lesson Plans: \vspace{-0.1cm}
			\begin{enumerate} \itemsep -1pt
			\item \url{http://dana.org/resources/brainykids/detail.aspx?folder_id=108}
			\item Includes resources that cover the history of science and technology, lesson plans for K-12 science teachers, and science news for youths.
			\end{enumerate}
		\item The Mindboggling Workbook: \vspace{-0.1cm}
			\begin{enumerate} \itemsep -1pt
			\item \url{http://www.dana.org/uploadedFiles/The_Dana_Alliances/mindboggling_workbook.pdf}
			\item ``A fun-filled activity book about the brain for children in grades K-3 (ages 5-9). Provides an introduction to how the brain works, what the brain does, its importance, and how to take care of it.''
			\end{enumerate}
		\end{enumerate}
	\end{enumerate}
\item University of New Mexico: \vspace{-0.3cm}
	\begin{enumerate} \itemsep -2pt
	\item Department of Mathematics and Statistics: \vspace{-0.2cm}
		\begin{enumerate} \itemsep -2pt
		\item UNM - PNM Statewide Mathematics Contest (sponsored by the PNM Foundation): \url{http://mathcontest.unm.edu/}
		\end{enumerate}
	\end{enumerate}
\item Center for Energy Workforce (CEWD): \vspace{-0.3cm}
	\begin{enumerate} \itemsep -2pt
	\item Get Into Energy: \vspace{-0.2cm}
		\begin{enumerate} \itemsep -2pt
		\item \url{http://www.getintoenergy.com/index.asp} and \url{http://www.getintoenergy.com/careers.asp}
		\item Fun educational resources for students: \url{http://www.getintoenergy.com/students.asp}
		\item Career Quiz: \vspace{-0.1cm}
			\begin{enumerate} \itemsep -1pt
			\item \url{http://www.getintoenergy.com/search/careerquizj.asp}
			\item Help you find out more about career options in the energy field
			\end{enumerate}
		\item Career Resources: \vspace{-0.1cm}
			\begin{enumerate} \itemsep -1pt
			\item \url{http://www.getintoenergy.com/careerresources.asp}
			\item Has information on: \vspace{-0.1cm}
				\begin{itemize} \itemsep -1pt
				\item Training Programs (technical schools and colleges)
				\item Work-based Programs (apprenticeships and internships)
				\item Featured Employers
				\end{itemize}
			\end{enumerate}
		\item Skills Needed in the Energy Field: \vspace{-0.1cm}
			\begin{enumerate} \itemsep -1pt
			\item \url{http://www.getintoenergy.com/skills.asp}
			\item List skills for different kinds of jobs in the energy field
			\end{enumerate}
		\item Information for parents: \url{http://www.getintoenergy.com/Parents.asp}
		\item Information for teachers and guidance counselors: \url{http://www.getintoenergy.com/Educators.asp}
		\end{enumerate}
	\end{enumerate}
\item University of Utah: \vspace{-0.3cm}
	\begin{enumerate} \itemsep -2pt
	\item Department of Electrical and Computer Engineering: \vspace{-0.2cm}
		\begin{enumerate} \itemsep -2pt
		\item Prof. Cynthia Furse: \vspace{-0.1cm}
			\begin{enumerate} \itemsep -1pt
			\item Cynthia Furse, {\it K-12 Engineering Outreach}, August 2007. Available online at: \url{http://www.ece.utah.edu/~cfurse/K12.html}; last accessed on December 10, 2010.
			\item Cynthia Furse, {\it U Dream. U Design. U Create.}, Department of Electrical and Computer Engineering, University of Utah. Available online at: \url{http://www.ece.utah.edu/~cfurse/NSF/}; last accessed on December 10, 2010.
			\end{enumerate}
		\end{enumerate}
	\end{enumerate}
\item Society for Industrial and Applied Mathematics: \vspace{-0.3cm}
	\begin{enumerate} \itemsep -2pt
	\item Public Awareness: \vspace{-0.2cm}
		\begin{enumerate} \itemsep -2pt
		\item Math Competitions, \url{http://www.siam.org/publicawareness/competitions.php}
		\item Moody's Mega Math Challenge (M3 Challenge) is an applied mathematics competition for high school students. Available online at: \url{http://m3challenge.siam.org/}; last accessed on December 13, 2010.
		\item {\it Math Matters, Apply It!}: \url{http://www.siam.org/careers/matters.php}
		\item Nuggets: \url{http://www.siam.org/publicawareness/nuggets.php}
		\end{enumerate}
	\item Society for Industrial and Applied Mathematics, ``Unveiling Why Do Math,'' May 27, 2010. Available online at: \url{http://www.siam.org/about/news-siam.php?id=1741}; last accessed on December 13, 2010.
	\end{enumerate}
\item International Federation of Operational Research Societies (IFORS): \vspace{-0.3cm}
	\begin{enumerate} \itemsep -2pt
	\item Association of European Operational Research Societies (EURO): \vspace{-0.2cm}
		\begin{enumerate} \itemsep -2pt
		\item {\it What is Operational Research?}: \url{http://www.euro-online.org/display.php?pageid=197&}
		\item Applications of OR in music, literature, and aesthetics: \url{http://www.euro-online.org/display.php?pageid=211&}
		\item 24 Hours Operations Research: \url{http://www.24hor.org/}
		\item Branding OR: \url{http://www.euro-online.org/display.php?pageid=198&}
		\end{enumerate}
	\end{enumerate}
\item American Institute of Aeronautics and Astronautics (AIAA): \vspace{-0.3cm}
	\begin{enumerate} \itemsep -2pt
	\item Students \& Educators: \url{http://www.aiaa.org/content.cfm?pageid=5}
	\item Ask An Engineer: \url{http://www.aiaa.org/content.cfm?pageid=214}
	\item Kid's Place: \vspace{-0.2cm}
		\begin{enumerate} \itemsep -2pt
		\item \url{http://www.aiaa.org/content.cfm?pageid=473}
		\item Enjoy games, puzzles, fun experiments, teen-recommended books and movies, and more.
		\end{enumerate}
	\item History of Flight Timeline: \url{http://www.aiaa.org/content.cfm?pageid=260}
	\item Ask Polaris: \vspace{-0.2cm}
		\begin{enumerate} \itemsep -2pt
		\item \url{http://www.askpolaris.org/}
		\item Resource for career exploration in aerospace engineering and related fields
		\end{enumerate}
	\end{enumerate}
\item Massachusetts Institute of Technology: \vspace{-0.3cm}
	\begin{enumerate} \itemsep -2pt
	\item MIT School of Engineering: \vspace{-0.2cm}
		\begin{enumerate} \itemsep -2pt
		\item Lemelson-MIT Program: \vspace{-0.1cm}
			\begin{enumerate} \itemsep -1pt
			\item \url{http://web.mit.edu/invent/}
			\item Inventor's Handbook: \url{http://web.mit.edu/invent/h-main.html}
			\item Games \& Trivia; \url{http://web.mit.edu/invent/g-main.html}
			\item Links \& Resources: \url{http://web.mit.edu/invent/r-main.html}
			\end{enumerate}
		\end{enumerate}
	\end{enumerate}
\item BT Group plc: \vspace{-0.3cm}
	\begin{enumerate} \itemsep -2pt
	\item British Telecommunications plc (BT): \vspace{-0.2cm}
		\begin{enumerate} \itemsep -2pt
		\item BT Young Scientist \& Technology Exhibition: \vspace{-0.1cm}
			\begin{enumerate} \itemsep -1pt
			\item \url{http://www.btyoungscientist.com/}
			\item \url{http://www.btyoungscientist.com/all-you-need-to-know/}
			\item Science and technology fair for high/secondary school students in Ireland
			\end{enumerate}
		\end{enumerate}
	\end{enumerate}
\item NHS Medical Careers: \vspace{-0.3cm}
	\begin{enumerate} \itemsep -2pt
	\item \url{http://www.medicalcareers.nhs.uk/Default.aspx}
	\item Provides information about careers in medicine for prospective medical students, medical students, medical school graduates (or young medical professionals), (medical speciality) trainers, and medical specialists.
	\end{enumerate}
\item British Science Association: \vspace{-0.3cm}
	\begin{enumerate} \itemsep -2pt
	\item British Science Festival: \vspace{-0.2cm}
		\begin{enumerate} \itemsep -2pt
		\item \url{http://www.britishscienceassociation.org/web/BritishScienceFestival/AboutFestival/index.htm}
		\item Festival Student Bursaries: \url{http://www.britishscienceassociation.org/web/BritishScienceFestival/StudentBursaries/index.htm}
		\end{enumerate}
	\item National Science \& Engineering Week: \url{http://www.britishscienceassociation.org/web/NSEW/index.htm}
	\item Clubs, CREST Awards and Fairs (programs and activities for children and youth, 5-19 years of age): \url{http://www.britishscienceassociation.org/web/ccaf/index.htm}
	\item National Science \& Engineering Competition: \url{http://www.britishscienceassociation.org/web/NSEC/index.htm} and \url{http://www.thebigbangfair.co.uk/nsec/}
	\end{enumerate}
\item Research Councils UK (RCUK): \vspace{-0.3cm}
	\begin{enumerate} \itemsep -2pt
	\item \url{http://www.rcuk.ac.uk/per/Pages/Schools.aspx}
	\item Schoolscience: \vspace{-0.2cm}
		\begin{enumerate} \itemsep -2pt
		\item \url{http://www.schoolscience.co.uk/}
		\item For students and educators in K-12 to enrich the learning experiences of science topics, and help students connect classroom material to the real world.
		\item Teacher Zone - professional resources for teachers: \url{http://www.schoolscience.co.uk/teacher_zone.cfm}
		\item Interactive Learning Resources: \url{http://www.schoolscience.co.uk/interactives.cfm}
		\item Free Resources: \url{http://www.schoolscience.co.uk/freebies.cfm}
		\item Competitions: \url{http://www.schoolscience.co.uk/competitions.cfm}
		\item Research focus: \url{http://www.schoolscience.co.uk/research_focus.cfm}
		\item Resources on the World Wide Web: \url{http://www.schoolscience.co.uk/sciencelink.cfm}
		\end{enumerate}
	\item Researchers in Residence (RinR): \vspace{-0.2cm}
		\begin{enumerate} \itemsep -2pt
		\item \url{http://www.researchersinresidence.ac.uk/cms/schools-colleges/}
		\item For students in middle and high schools to job shadow (observe first-hand) a Ph.D. student or postdoctoral researcher in her/his research activities for up to a week, so that students can learn what doing research in her/his research area is like. In addition, the researcher would explain in laypeople's terms what her/his research is about. It can be considered as an externship program.
		\end{enumerate}
	\item Nuffield Bursaries: \vspace{-0.2cm}
		\begin{enumerate} \itemsep -2pt
		\item \url{http://www.nuffieldfoundation.org/capacity-building}
		\item \url{http://www.nuffieldfoundation.org/science-bursaries-schools-and-colleges}
		\item For high school juniors/seniors to pursue a research internship in science and engineering.
		\end{enumerate}
	\item CREST (Creativity in Science and Technology): \vspace{-0.2cm}
		\begin{enumerate} \itemsep -2pt
		\item \url{http://www.britishscienceassociation.org/web/ccaf/CREST/index.htm}
		\item Program to help students get engaged in a science or engineering project, where they learn how to solve real problems in science or engineering.
		\end{enumerate}
	\end{enumerate}
\item Nuffield Foundation: \vspace{-0.3cm}
	\begin{enumerate} \itemsep -2pt
	\item Science bursaries for schools and colleges: \url{http://www.nuffieldfoundation.org/science-bursaries-schools-and-colleges}
	\item Students: \url{http://www.nuffieldfoundation.org/students}
	\item Twenty First Century Science: \vspace{-0.2cm}
		\begin{enumerate} \itemsep -2pt
		\item \url{http://www.21stcenturyscience.org/}
		\item ``Twenty First Century Science is a set of GCSE science courses giving all 14-16-year-olds a worthwhile and inspiring experience of science. The strength of the programme is that it meets the needs, through flexible options, of those who will go on to be professional scientists and of those who will not.''
		\item The Courses: \url{http://www.21stcenturyscience.org/the-courses/}
		\item Assessment overview: \url{http://www.21stcenturyscience.org/assess/}
		\item Teaching resources: \url{http://www.21stcenturyscience.org/resources/}
		\end{enumerate}
	\item Science in Society: \vspace{-0.2cm}
		\begin{enumerate} \itemsep -2pt
		\item \url{http://www.scienceinsocietyadvanced.org/}
		\item ``Science in Society is an interesting and topical GCE advanced level course. It aims to develop the knowledge and skills that are needed for students to understand how science works, analyse contemporary issues involving science and technology and communicate their scientific appreciation and understanding to others.''
		\end{enumerate}
	\item Parents: \url{http://www.nuffieldfoundation.org/parents}
	\item Education: \url{http://www.nuffieldfoundation.org/education}
	\item Teachers (has excellent resources for science and mathematics): \url{http://www.nuffieldfoundation.org/teachers}
	\item Capacity building: \url{http://www.nuffieldfoundation.org/capacity-building}
	\end{enumerate}
\item The Story of Stuff Project (by Annie Leonard): \vspace{-0.3cm}
	\begin{enumerate} \itemsep -2pt
	\item \url{http://www.storyofstuff.com/}
	\item ``The Story of Stuff Project was created by Annie Leonard to leverage and extend the film's impact. We amplify public discourse on a series of environmental, social and economic concerns and facilitate the growing Story of Stuff community's involvement in strategic efforts to build a more sustainable and just world.''
	\item Resources: \vspace{-0.2cm}
		\begin{enumerate} \itemsep -2pt
		\item \url{http://www.storyofstuff.com/resources.php}
		\item The Story of Stuff Project PDFs: \url{http://www.storyofstuff.com/dl-pdfs.php}
		\item Teaching Tools: \url{http://www.storyofstuff.com/teach.php}
		\item More About Stuff: \url{http://www.storyofstuff.com/aboutstuff.php}
		\item Recommended Reading \& Bibliography: \url{http://www.storyofstuff.com/reading.php}
		\item Get Involved: \url{http://www.storyofstuff.com/getinvolved.php}
		\item Curricula: \url{http://storyofstuff.org/curricula.php}
		\end{enumerate}
	\end{enumerate}
\item Facing the Future: \vspace{-0.3cm}
	\begin{enumerate} \itemsep -2pt
	\item \url{http://www.facingthefuture.org/}
	\item ``{\it Facing the Future} engages students in learning by making academics relevant to their lives. We empower students to think critically, develop a global perspective, and participate in positive solutions for a sustainable future.''
	\item Curriculum Alignment with Education Standards: \url{http://www.facingthefuture.org/Curriculum/AlignmentwithEducationStandards/tabid/116/Default.aspx}
	\item Global Sustainability Curriculum Finder: \url{http://www.facingthefuture.org/Curriculum/FindCurriculumthatisRightforYou/tabid/68/Default.aspx}
	\item Download FREE Global Issues and Sustainability Curriculum: \url{http://www.facingthefuture.org/Curriculum/DownloadFreeCurriculum/tabid/114/Default.aspx}
	\item Classroom Examples: How Engaging Curriculum Can Help Address Classroom Challenges, \url{http://www.facingthefuture.org/ForEducators/ClassroomExamples/tabid/213/Default.aspx}
	\item Our Impact on Student Achievement: \url{http://www.facingthefuture.org/ForEducators/OurImpactonStudentAchievement/tabid/73/Default.aspx}
	\item Action Project Database: \url{http://www.facingthefuture.org/ServiceLearning/ActionProjectDatabase/tabid/94/Default.aspx}
	\item Service Learning Examples: \url{http://www.facingthefuture.org/ServiceLearning/ExamplesofStudentsTakingAction/tabid/147/Default.aspx}
	\item Curriculum: \url{http://www.facingthefuture.org/Curriculum/CurriculumHome/tabid/113/Default.aspx}
	\end{enumerate}
\item U.S. Department of Energy: \vspace{-0.3cm}
	\begin{enumerate} \itemsep -2pt
	\item Office of Science: \vspace{-0.2cm}
		\begin{enumerate} \itemsep -2pt
		\item U.S. Department of Energy (DOE) National Science Bowl\textregistered: \vspace{-0.1cm}
			\begin{enumerate} \itemsep -1pt
			\item \url{http://www.scied.science.doe.gov/nsb/default.htm}
			\item ``The U.S. Department of Energy (DOE) National Science Bowl\textregistered\ is a nationwide academic competition that tests students' knowledge in all areas of science. High school and middle school students are quizzed in a fast paced question-and-answer format similar to Jeopardy. Competing teams from diverse backgrounds are comprised of four students, one alternate, and a teacher who serves as an advisor and coach.''
			\end{enumerate}
		\item Argonne National Laboratory: \vspace{-0.1cm}
			\begin{enumerate} \itemsep -1pt
			\item Division of Educational Programs: \vspace{-0.1cm}
				\begin{itemize} \itemsep -1pt
				\item Newton BBS Ask A Scientist: \url{http://www.newton.dep.anl.gov/aas.htm}
				\end{itemize}
			\end{enumerate}
		\end{enumerate}
	\item Office of Energy Efficiency and Renewable Energy (EERE): \vspace{-0.2cm}
		\begin{enumerate} \itemsep -2pt
		\item Kids Saving Energy: \vspace{-0.1cm}
			\begin{enumerate} \itemsep -1pt
			\item \url{http://www.eere.energy.gov/kids/index.html}
			\item K-12 Lesson Plans \& Activities: \url{http://www1.eere.energy.gov/education/lessonplans/}
			\item Energy Savers: \url{http://www.energysavers.gov/}
			\item Games and activities: \url{http://www.eere.energy.gov/kids/games.html}
			\item Smart home: \url{http://www.eere.energy.gov/kids/smart_home.html}
			\item About renewable energy: \url{http://www.eere.energy.gov/kids/renergy.html}
			\end{enumerate}
		\end{enumerate}
	\item Contest \& Competitions: \url{http://www.energy.gov/contests&competitions.htm}
	\end{enumerate}
\item United States Department of Defense (DoD): \vspace{-0.3cm}
	\begin{enumerate} \itemsep -2pt
	\item National Defense Education Program; Defense Advanced Research Projects Agency (DARPA): \vspace{-0.2cm}
		\begin{enumerate} \itemsep -2pt
		\item Resource for Students: \url{http://www.ndep.us/GetInvoStu.aspx}
		\item Resource for Educators: \url{http://www.ndep.us/GetInvoTea.aspx}
		\end{enumerate}
	\end{enumerate}
\item Project Lead The Way: \vspace{-0.3cm}
	\begin{enumerate} \itemsep -2pt
	\item \url{http://www.pltw.org/}
	\item Getting started: \url{http://www.pltw.org/getting-started/getting-started}
	\item Program support: \url{http://www.pltw.org/program-support/program-support}
	\item Grants available to schools and teachers: \url{http://www.pltw.org/pltw-in-the-news/grants-available-schools-teachers-and-classrooms}
	\item Students: \url{http://www.pltw.org/students/students}
	\item Educators and Administrators: \url{http://www.pltw.org/educators-administrators/educators-administrators-overview}
	\item Parents: \url{http://www.pltw.org/parents/parents}
	\end{enumerate}
\item National Science Teachers Association: \vspace{-0.3cm}
	\begin{enumerate} \itemsep -2pt
	\item \url{http://www.exploravision.org/}
	\item Science competition for K-12 students
	\end{enumerate}
\item American Mathematical Society: \vspace{-0.3cm}
	\begin{enumerate} \itemsep -2pt
	\item Some career resources for mathematics: \url{http://e-math.ams.org/samplings/samplings}
	\end{enumerate}
\item American Institute of Physics (AIP): \vspace{-0.3cm}
	\begin{enumerate} \itemsep -2pt
	\item Physics Success Stories: \url{http://www.aip.org/success/}
	\item Physics is for you; Career Services Division: \vspace{-0.2cm}
		\begin{enumerate} \itemsep -2pt
		\item \url{http://www.aip.org/careersvc/pify/}
		\item Physicists at work: \url{http://www.aip.org/careersvc/pify/yellow.html}
		\end{enumerate}
	\item Society of Physics Students (SPS): \vspace{-0.2cm}
		\begin{enumerate} \itemsep -2pt
		\item Careers Using Physics (CUP): \vspace{-0.1cm}
			\begin{enumerate} \itemsep -1pt
			\item \url{http://www.spsnational.org/cup/}
			\item Advice: \url{http://www.spsnational.org/cup/advice/index.html}
			\item Resources: \url{http://www.spsnational.org/cup/resources.html}
			\item Preparing to Teach: \url{http://www.spsnational.org/cup/teach/index.html}
			\end{enumerate}
		\end{enumerate}
	\item ComPADRE Digital Library: \vspace{-0.2cm}
		\begin{enumerate} \itemsep -2pt
		\item \url{http://www.compadre.org/}
		\item The Physics Career Resource: \url{http://www.compadre.org/careers/}
		\end{enumerate}
	\item Career guidance for high school and undergraduate students: \url{http://www.aip.org/statistics/trends/career.html}
	\item Gayle A. Buck, Jack G. Hehn, and Diandra L. Leslie-Pelecky (Editors), ``The Role of Physics Departments in Preparing K-12 Teachers,'' American Institute of Physics. Available online at: \url{http://www.aip.org/education/teacherprep/}; last accessed on January 9, 2010.
	\item American Geophysical Union: \vspace{-0.2cm}
		\begin{enumerate} \itemsep -2pt
		\item Students \& Teachers: \url{http://www.agu.org/education/students_teachers.shtml}
		\item Diversity Programs: \url{http://www.agu.org/education/diversity_programs/}
		\end{enumerate}
	\end{enumerate}
\item Institute for Operations Research and the Management Sciences (INFORMS): \vspace{-0.3cm}
	\begin{enumerate} \itemsep -2pt
	\item Career FAQ's: \url{http://www.informs.org/Build-Your-Career/INFORMS-Student-Union/Career-Center/Career-FAQ-s}
	\end{enumerate}
\item American Institute of Mathematics: \vspace{-0.3cm}
	\begin{enumerate} \itemsep -2pt
	\item Math Teachers' Circle Network: \vspace{-0.2cm}
		\begin{enumerate} \itemsep -2pt
		\item Classroom Materials: \url{http://www.mathteacherscircle.org/resources/classroommaterials.html}
		\item Helpful Resources: \url{http://www.mathteacherscircle.org/resources/general.html}
		\end{enumerate}
	\item Resources for the Math Community: \vspace{-0.2cm}
		\begin{enumerate} \itemsep -2pt
		\item \url{http://www.aimath.org/mathcommunity/}
		\item David W. Farmer, ``The AIM REU: individual projects with a common theme,'' in the {\it Proceedings of the Conference on Promoting Undergraduate Research in Mathematics}, American Mathematical Society, 2006. Available online at: \url{http://www.aimath.org/mathcommunity/farmerREU.pdf}; last accessed on January 9, 2010. [ ``AIM Research Experience for Undergraduates (REU)'' ]
		\item Sally Koutsoliotas and David W. Farmer, ``Preparing students to give talks,'' American Institute of Mathematics. Available online at: \url{http://www.aimath.org/mathcommunity/studenttalks.pdf}; last accessed on January 9, 2010. [ ``Preparing students to give talks'' ]
		\end{enumerate}
	\end{enumerate}
\item Invent Now: \vspace{-0.3cm}
	\begin{enumerate} \itemsep -2pt
	\item Camp Invention: \vspace{-0.2cm}
		\begin{enumerate} \itemsep -2pt
		\item ``Summer enrichment program for children entering grades one through six.''
		\item ``The Camp Invention program instills vital 21st century life skills such as problem-solving and teamwork through hands-on fun!''
		\item Parents: \url{http://www.invent.org/camp/parents.aspx}
		\item Teachers: \url{http://www.invent.org/camp/teachers.aspx}
		\end{enumerate}
	\end{enumerate}
\item Massachusetts Institute of Technology: \vspace{-0.3cm}
	\begin{enumerate} \itemsep -2pt
	\item MIT School of Engineering: \vspace{-0.2cm}
		\begin{enumerate} \itemsep -2pt
		\item Lemelson-MIT Program: \vspace{-0.1cm}
			\begin{enumerate} \itemsep -1pt
			\item \url{http://web.mit.edu/invent/}
			\item Invention Dimension (for children): \url{http://web.mit.edu/invent/invent-main.html}
			\end{enumerate}
		\end{enumerate}
	\end{enumerate}
\item The Lemelson Foundation: \vspace{-0.3cm}
	\begin{enumerate} \itemsep -2pt
	\item \url{http://web.mit.edu/invent/w-foundation.html}
	\item Programs \& Grants: \url{http://www.lemelson.org/programs-grants}
	\item Grantmaking: \url{http://www.lemelson.org/grantmaking}
	\end{enumerate}
\item Smithsonian Institution: \vspace{-0.3cm}
	\begin{enumerate} \itemsep -2pt
	\item Smithsonian Kids: \url{http://www.si.edu/Kids}
	\item National Museum of American History: \vspace{-0.2cm}
		\begin{enumerate} \itemsep -2pt
		\item Lemelson Center for the Study of Invention and Innovation: \vspace{-0.1cm}
			\begin{enumerate} \itemsep -1pt
			\item \url{http://inventionatplay.org/index.html}
			\item Resources: \url{http://inventionatplay.org/resources.html}
			\end{enumerate}
		\end{enumerate}
	\end{enumerate}
%%%%%%%%%%%%%%%%%%%%%%%%%%%%%%%%%%%%%%%%
%%%%%%%%%%%%%%%%%%%%%%%%%%%%%%%%%%%%%%%%
\item Scholarships: \vspace{-0.3cm}
	\begin{enumerate} \itemsep -2pt
	\item IEEE Presidents' Scholarship: \url{http://www.ieee.org/education_careers/education/preuniversity/scholarship.html}
	\item ACM/SIGDA {\it P. O. Pistilli scholarship}: \vspace{-0.1cm}
		\begin{enumerate} \itemsep -1pt
		\item Supported by the Design Automation Conference which ACM/SIGDA sponsors, the objective of the P. O. Pistilli Scholarship is to increase the pool of professionals in Electrical Engineering and Computer Science from underrepresented groups (Women, African American, Hispanic, American Indian, and Disabled).
		\item Scholarships of \$4000 per year, renewable for up to 5 years, are awarded annually to 2-7 high school seniors from the above mentioned under represented groups who have a 3.00 GPA or better (on a 4.00 scale), have demonstrated high achievement in math and science courses, have expressed a strong desire to pursue careers in electrical engineering, computer engineering, or computer science, and who have demonstrated substantial financial need.
		\item U.S. citizenship is not required, but applicants must be U.S. residents when they apply and must plan to attend an accredited US college or university.
		\item \url{http://www.sigda.org/pistilli.html}
		\end{enumerate}
	\item Engineering Education Service Center (EESC): \url{http://www.engineeringedu.com/scholars.html}
	\item ASME-ASME Auxiliary FIRST Clarke Scholarships: \url{http://www.asme.org/Education/College/FinancialAid/High_School_Seniors.cfm} and \url{http://www.asme.org/Education/College/FinancialAid/Auxiliary_FIRST_Clarke.cfm}
	\item International Petroleum Institute�s High School Scholarships (for individuals entering a college program in engineering): \url{http://www.asme-ipti.org/public/pagscholarshipprograms.aspx}
	\item American Institute of Chemical Engineers (AIChE): \vspace{-0.2cm}
		\begin{enumerate} \itemsep -2pt
		\item Fuels and Petrochemicals Division Scholarship (for high school students entering undergraduate programs in engineering or science that are related to fuels and petrochemicals): \url{http://www.aiche.org/Students/Awards/F_PDScholarship.aspx}
		\item Minority Scholarship Awards for Incoming College Freshmen (for underrepresented minorities entering an undergraduate chemical engineering program): \url{http://www.aiche.org/Students/Awards/MinorityScholarshipAwardsIncomingFreshmen.aspx}
		\end{enumerate}
	\item Sallie Mae Fund: \vspace{-0.3cm}
		\begin{enumerate} \itemsep -2pt
		\item \url{http://www.thesalliemaefund.org/smfnew/index.html}
		\item List of scholarship resources: \url{http://www.thesalliemaefund.org/smfnew/sections/search.html}
		\item Top 10 Tips for Planning and Paying for College: \url{http://www.thesalliemaefund.org/smfnew/fin_aid/index.html}
		\item Scholarships: \url{http://www.thesalliemaefund.org/smfnew/scholarship/index.html} and \url{http://www.thesalliemaefund.org/smfnew/sections/apply.html}
		\item Important information for parents about saving for college and getting financial aid: \vspace{-0.2cm}
			\begin{enumerate} \itemsep -2pt
			\item \url{http://www.thesalliemaefund.org/smfnew/sections/download.html}
			\item This information is also available in Spanish. Summaries are also available in other languages such as: \vspace{-0.1cm}
				\begin{itemize} \itemsep -1pt
				\item French
				\item German
				\item Italian
				\item Korean
				\item Russian
				\item Simplified and Traditional Chinese
				\item Tagalog
				\item Vietnamese
				\end{itemize}
			\item Top 10 Tips for Planning and Paying for College: \url{http://www.thesalliemaefund.org/smfnew/fin_aid/index.html}
			\end{enumerate}
		\item Kids2College program: \url{http://www.thesalliemaefund.org/smfnew/initiatives/kidscollege.html}
		\item For African-American individuals entering college: \vspace{-0.2cm}
			\begin{enumerate} \itemsep -2pt
			\item Black College Dollars: \url{http://www.thesalliemaefund.org/smfnew/scholarship_directory/index.html}
			\item \url{http://www.thesalliemaefund.org/smfnew/initiatives/aa.html}
			\end{enumerate}
		\item For Hispanic Americans, or Latinos/Latinas: \vspace{-0.2cm}
			\begin{enumerate} \itemsep -2pt
			\item \url{http://www.thesalliemaefund.org/smfnew/pdf/Scholarship_Directory.pdf}
			\item Latino College Dollars: \url{http://www.latinocollegedollars.org/}
			\end{enumerate}
		\end{enumerate}
	\item {\it American Chemical Society}: \vspace{-0.3cm}
		\begin{enumerate} \itemsep -2pt
		\item ACS Scholars Program (for underrepresented minorities in, or entering, an undergraduate program in chemistry, biochemistry, or chemical engineering): \url{http://portal.acs.org/portal/acs/corg/content?_nfpb=true&_pageLabel=PP_SUPERARTICLE&node_id=1650&use_sec=false&sec_url_var=region1&__uuid=b3b583cf-18ae-4fb0-9375-33f75a0ccf49}
		\item Project SEED Scholarships (for high school seniors who have worked at least one summer at a science institute under the Project SEED program): \url{http://portal.acs.org/portal/acs/corg/content?_nfpb=true&_pageLabel=PP_SUPERARTICLE&node_id=2031&use_sec=false&sec_url_var=region1&__uuid=99bc6a62-3e78-4b2a-be3f-50b28f7ff265}
		\end{enumerate}
	\item The Posse Foundation: \url{http://www.possefoundation.org/}
	\item Hispanic Scholarship Fund (HSF) scholarship programs for high school students: \url{http://www.hsf.net/innerContent.aspx?id=426}
	\item Asian \& Pacific Islander American Scholarship Fund (APIASF): scholarships for individuals entering college as freshmen; see \url{http://www.apiasf.org/scholarship_apiasf.html}
	\item Nationally Coveted College Scholarships, Graduate School Fellowships \& Postdoctoral Awards: \url{http://scholarships.fatomei.com/}
	\item {\it SPIE} Scholarship Program (for high school students entering college to study optics, photonics, imaging, optoelectronics, or related program): \url{http://spie.org//x1733.xml?WT.svl=mddm14}
	\item Susan G. Komen for the Cure\textregistered: The Komen College Scholarship Program, \url{http://ww5.komen.org/ResearchGrants/CollegeScholarshipAward.html}
	\item National Society of Professional Engineers's list of scholarships for high school students: \url{http://www.nspe.org/Students/Scholarships/index.html}
	\item AWM Essay Contest: Biographies of Contemporary Women in Mathematics; see \url{http://www.awm-math.org/biographies/contest.html}
	\item National Engineers Week Future City Competition (students from $6^{th}$--$8^{th}$ grades): \url{http://www.futurecity.org/}
	\item National Ocean Sciences Bowl: \vspace{-0.2cm}
		\begin{enumerate} \itemsep -2pt
		\item \url{http://www.nosb.org/ocean-careers/}
		\item National Ocean Scholar Program (for high school seniors who are current/past participants of the Bowl, and are seeking a career in the ocean sciences or a marine-related field): \url{http://www.nosb.org/ocean-careers/national-ocean-scholar-program/}
		\end{enumerate}
	\item National Center for Women \& Information Technology (NCWIT): \vspace{-0.2cm}
		\begin{enumerate} \itemsep -2pt
		\item NCWIT Award for Aspirations in Computing (for young women in high school): \url{http://www.ncwit.org/work.awards.aspiration.html}
		\end{enumerate}
	\end{enumerate}
%%%%%%%%%%%%%%%%%%%%%%%%%%%%%%%%%%%%%%%%
%%%%%%%%%%%%%%%%%%%%%%%%%%%%%%%%%%%%%%%%
\item Resources for teachers/educators: \vspace{-0.3cm}
	\begin{enumerate} \itemsep -2pt
	\item Google: \vspace{-0.2cm}
		\begin{enumerate} \itemsep -2pt
		\item Google Teacher Academy (for teachers to learn how to use Google technologies to facilitate teaching): \url{http://www.google.com/educators/gta.html}
		\item Classroom activities (suggestions): \url{http://www.google.com/educators/activities.html}
		\end{enumerate}
	\item IEEE Teacher In-Service Program (TISP): \vspace{-0.2cm}
		\begin{enumerate} \itemsep -2pt
		\item \url{http://www.ieee.org/education_careers/education/preuniversity/tispt/index.html}
		\item Lesson Plans for Pre-university Instructors: \url{http://www.ieee.org/education_careers/education/preuniversity/resources/index.html}
		\end{enumerate}
	\item Global Challenge Award: \url{http://www.globalchallengeaward.org/display/public/Home}
	\item Teachers' Domain (to teach students about science, engineering, and the arts): \url{http://www.teachersdomain.org/}
	\item {\it TeachEngineering} digital library: \vspace{-0.2cm}
		\begin{enumerate} \itemsep -2pt
		\item The {\it TeachEngineering} digital library provides teacher-tested, standards-based engineering content for K-12 teachers engineering content for K12 teachers to use in science and math classrooms. Engineering lessons connect real-world experiences with curricular content already taught in K-12 classrooms. Mapped to educational content standards, {\it TeachEngineering}'s comprehensive curricula are hands-on, free, and relevant to children's daily lives.
		\item \url{http://www.teachengineering.com/index.php}
		\end{enumerate}
	\item Engineering Pathway: \url{http://www.engineeringpathway.com/ep/index.jhtml}
	\item {\it American Society of Mechanical Engineers, ASME}: \url{http://www.asme.org/Education/PreCollege/TeacherResources/}
	\item {\it National Science Foundation} resources for the K-12 classroom: \url{http://nsf.gov/news/classroom/engineering.jsp}
	\item {\it NASA}: \url{http://www.nasa.gov/audience/foreducators/index.html}
	\item The Mathematical Association of America: \vspace{-0.2cm}
		\begin{enumerate} \itemsep -2pt
		\item Pre-College Programs: \url{http://www.maa.org/funding/pre_college.html}. Also, see \url{http://www.maa.org/funding/undergraduate.html}.
		\item Special Interest Group of the Mathematical Association of America on the use of the World-Wide Web in Undergraduate Mathematics Instruction (Web SIGMAA). Available at: \url{http://math.chapman.edu/websigmaa/index.php/Main_Page}; last accessed on September 2, 2010.
		\item SIGMAA TAHSM (Teaching Advanced High School Mathematics). Available at: \url{http://sigmaa.maa.org/tahsm/}; last accessed on September 2, 2010.
		\item Special Interest Group on Statistics Education: \url{http://sigmaa.maa.org/stat-ed/}
		\end{enumerate}
	\item Math for America: \vspace{-0.2cm}
		\begin{enumerate} \itemsep -2pt
		\item M$f$A Master Teacher Fellowship program: \vspace{-0.1cm}
			\begin{enumerate} \itemsep -1pt
			\item The Math for America Master Teacher Fellowship program rewards exceptional public secondary school math teachers with a four-year Fellowship.
			\item M$f$A Master Teacher Fellowships are currently available in Berkeley, Boston and New York City.
			\item \url{http://www.mathforamerica.org/web/guest/master-teachers}
			\end{enumerate}
		\item M$f$A Early Career Fellows: \vspace{-0.1cm}
			\begin{enumerate} \itemsep -1pt
			\item The Math for America Early Career Fellowship is awarded to public secondary school math teachers early in their careers.
			\item The M$f$A Early Career Fellowship requires a commitment of four years and is available in New York City. 
			\item \url{http://www.mathforamerica.org/early-career-fellows}
			\end{enumerate}
		\item M$f$A Fellows: \vspace{-0.1cm}
			\begin{enumerate} \itemsep -1pt
			\item \url{http://www.mathforamerica.org/web/guest/mfa-fellows}
			\end{enumerate}
		\item Teachers resources: \url{http://www.mathforamerica.org/web/guest/teacher-resources} and \url{http://www.mathforamerica.org/teacher-resources/classroom} (classroom resources)
		\item Resources for professional development (teachers): \url{http://www.mathforamerica.org/teacher-resources/professional}
		\item \url{http://www.mathforamerica.org/home}
		\end{enumerate}
	\item Association for Symbolic Logic (ASL): \vspace{-0.2cm}
		\begin{enumerate} \itemsep -2pt
		\item Guidelines on Logic Education: \url{http://www.ucalgary.ca/aslcle/guidelines}
		\item Educational Logic Software: \url{http://www.ucalgary.ca/aslcle/logic-courseware}
		\end{enumerate}
	\item Consortium for Ocean Leadership: \vspace{-0.2cm}
		\begin{enumerate} \itemsep -2pt
		\item Educational Resources: \url{http://www.oceanleadership.org/gulf-oil-spill/educational-resources/}
		\item The JOIDES Resolution (The JR) scientific research vessel [ Deep Earth Academy ]: \vspace{-0.1cm}
			\begin{enumerate} \itemsep -1pt
			\item Teacher Resources (to teach students about geology and physical geography): \url{http://joidesresolution.org/node/46}
			\item Teachers at Sea/On-board Education Officer (for teachers to go on scientific expeditions on board): \url{http://joidesresolution.org/node/453}
			\end{enumerate}
		\item Integrated Ocean Drilling Program (IODP) -- IODP United States Implementing Organization (IODP-USIO): \vspace{-0.1cm}
			\begin{enumerate} \itemsep -1pt
			\item Teaching Materials: \url{http://www.iodp-usio.org/Education/educ.html}
			\end{enumerate}
		\item Deep Earth Academy (includes suggested ``curriculum and classroom activities for kindergarten through college level''): \vspace{-0.1cm}
			\begin{enumerate} \itemsep -1pt
			\item \url{http://www.oceanleadership.org/education/deep-earth-academy/}
			\item For Educators: \url{http://www.oceanleadership.org/education/deep-earth-academy/educators/}
			\end{enumerate}
		\end{enumerate}
	\item Virginia Institute of Marine Science (College of William and Mary): \vspace{-0.2cm}
		\begin{enumerate} \itemsep -2pt
		\item Bridge Ocean Education Teacher Resource Center: \url{http://web.vims.edu/bridge/?svr=www#}
		\end{enumerate}
	\item American Geological Institute: \vspace{-0.2cm}
		\begin{enumerate} \itemsep -2pt
		\item Awards for teachers: \url{http://www.agiweb.org/education/awards/index.html}
		\item Edward C. Roy, Jr. Award For Excellence in K-8 Earth Science Teaching (for middle school teachers in the US who are teaching earth science): \url{http://www.agiweb.org/education/awards/ed-roy/}
		\item Presidential Awards for Excellence in Mathematics \& Science Teaching, PAEMST (for kindergarten and K-12 teachers in the US who are teaching students about STEM fields): \url{http://www.agiweb.org/education/awards/paemst.html}
		\item National Association of Geoscience Teachers (NAGT) Outstanding Earth Science Teacher Award: \url{http://www.agiweb.org/education/awards/nagt.html}
		\item American Association of Petroleum Geologists' (AAPG) National Earth Science Teacher of the Year Award: \url{http://www.agiweb.org/education/awards/aapg.html}
		\item Curriculum Materials and Activities: \url{http://www.agiweb.org/education/curriculum/index.html}
		\item K-12 Professional Development Programs: \url{http://www.agiweb.org/education/pd/index.html}
		\item Educational Resources: \url{http://www.agiweb.org/education/resource/index.html}
		\end{enumerate}
	\item Institute for Broadening Participation: \vspace{-0.2cm}
		\begin{enumerate} \itemsep -2pt
		\item PathwaysToScience.org: \vspace{-0.1cm}
			\begin{enumerate} \itemsep -1pt
			\item For K-12 teachers (resources to encourage students to be interested in STEM): \url{http://www.pathwaystoscience.org/Teachers.asp}
			\end{enumerate}
		\end{enumerate}
	\item National Science Foundation: \vspace{-0.2cm}
		\begin{enumerate} \itemsep -2pt
		\item The National Science Digital Library (NSDL): \vspace{-0.1cm}
			\begin{enumerate} \itemsep -1pt
			\item Resources for K-12 Teachers: \url{http://nsdl.org/resources_for/k12_teachers/}
			\end{enumerate}
		\end{enumerate}
	\item National Academy of Engineering, NAE: \vspace{-0.2cm}
		\begin{enumerate} \itemsep -2pt
		\item NAE Grand Challenges: \vspace{-0.1cm}
			\begin{enumerate} \itemsep -1pt
			\item Includes a list of NAE Grand Challenges, which indicate some of the challenges faced by people on a global scale that can be partially solved by engineers. This can be used to get children and youths to be excited about engineering. 
			\item NAE Grand Challenges: \vspace{-0.1cm}
				\begin{itemize} \itemsep -1pt
				\item Make solar energy economical
				\item Provide energy from fusion
				\item Develop carbon sequestration methods
				\item Manage the nitrogen cycle
				\item Provide access to clean water
				\item Restore and improve urban infrastructure
				\item Advance health informatics
				\item Engineer better medicines
				\item Reverse-engineer the brain
				\item Prevent nuclear terror
				\item Secure cyberspace
				\item Enhance virtual reality
				\item Advance personalized learning
				\item Engineer the tools of scientific discovery
				\end{itemize}
			\item \url{http://www.engineeringchallenges.org/}
			\end{enumerate}
		\item NAE Grand Challenge K12 Partners Program: \vspace{-0.1cm}
			\begin{enumerate} \itemsep -1pt
			\item Can be used by schools/teachers to raise awareness of global challenges among students and to encourage students to plan career paths to tackle these challenges
			\item 5-Part Make it Happen Plan (includes suggested activities for students in elementary school to learn about basic science and engineering concepts that are relevant to solve the NAE grand challenges): \url{http://www.grandchallengek12.org/5-part-plan}
			\item \url{http://www.grandchallengek12.org/about}
			\end{enumerate}
		\item {\it National Academy of Engineering}'s technological literacy program for people (students, parents, and educators) to learn more about technology: \url{http://www.nae.edu/nae/techlithome.nsf}
		\end{enumerate}
	\item Women in Technology (WIT): \vspace{-0.2cm}
		\begin{enumerate} \itemsep -2pt
		\item Girls In Technology (GIT): \vspace{-0.1cm}
			\begin{enumerate} \itemsep -1pt
			\item Get Involved: \vspace{-0.1cm}
				\begin{itemize} \itemsep -1pt
				\item \url{http://www.girlsintechnology.org/getinvolved.cfm}
				\item Teacher: teach girls about IT as an after-school activity or in a summer camp session
				\item Assistant Teacher: Assist instructors in GIT sessions, after-school activities, or summer camp sessions
				\item Develop Curriculum: Develop a curriculum for a supported GIT educational program
				\item Mentor: Mentor a girl in one of [GIT's] supported programs
				\item Job Shadow: ``Let a girl shadow you at work''
				\item Guest Speaker: ``Speak to a group of girls on a topic both you and they enjoy, such as computers, technology, education, how to take apart computers, how to build a web site, etc.''
				\end{itemize}
			\end{enumerate}
		\end{enumerate}
	\item Organization for Economic Co-operation and Development (OECD): \vspace{-0.2cm}
		\begin{enumerate} \itemsep -2pt
		\item Programme for International Student Assessment (PISA): \vspace{-0.1cm}
			\begin{enumerate} \itemsep -1pt
			\item {\it PISA 2009 Results}. Available online at: \url{http://www.oecd.org/document/61/0,3343,en_32252351_32235731_46567613_1_1_1_1,00.html}; last accessed on December 10, 2010. [ Includes suggestions to improve learning outcomes, as well as education policies and practices. ]
			\end{enumerate}
		\end{enumerate}
	\item American Institute of Aeronautics and Astronautics (AIAA): \vspace{-0.2cm}
		\begin{enumerate} \itemsep -2pt
		\item K-12 Educators: \url{http://www.aiaa.org/content.cfm?pageid=208}
		\end{enumerate}
	\item Research Councils UK (RCUK): \vspace{-0.2cm}
		\begin{enumerate} \itemsep -2pt
		\item Biotechnology and Biological Sciences Research Council (BBSRC): \vspace{-0.1cm}
			\begin{enumerate} \itemsep -1pt
			\item Resources for schools and young people: \url{http://www.bbsrc.ac.uk/society/schools/schools-index.aspx}
			\item Teaching resources: publications and web-based activities: \vspace{-0.1cm}
				\begin{itemize} \itemsep -1pt
				\item Primary (ages 5-12) resources: \url{http://www.bbsrc.ac.uk/society/schools/primary/primary-index.aspx}
				\item Secondary (ages 12-16) and post-16 resources: \url{http://www.bbsrc.ac.uk/society/schools/secondary/secondary-index.aspx}
				\end{itemize}
			\end{enumerate}
		\end{enumerate}
	\item Nuffield Foundation: \vspace{-0.2cm}
		\begin{enumerate} \itemsep -2pt
		\item Education: \url{http://www.nuffieldfoundation.org/education}
		\item Teachers: \vspace{-0.1cm}
			\begin{enumerate} \itemsep -1pt
			\item (Excellent) resources in science and mathematics: \url{http://www.nuffieldfoundation.org/teachers}
			\item \url{http://www.nuffieldfoundation.org/teachers-0}
			\end{enumerate}
		\end{enumerate}
	\item Wellcome Trust: \vspace{-0.2cm}
		\begin{enumerate} \itemsep -2pt
		\item Education resources: \url{http://www.wellcome.ac.uk/Education-resources/index.htm}
		\item {\it yourgenome.org}: \vspace{-0.1cm}
			\begin{enumerate} \itemsep -1pt
			\item \url{http://www.yourgenome.org/}
			\item Resources for teachers about genomics: \url{http://www.yourgenome.org/landing_teachers.shtml}
			\end{enumerate}
		\item Network of Science Learning Centers (Science Learning Centers): \vspace{-0.1cm}
			\begin{enumerate} \itemsep -1pt
			\item \url{https://www.sciencelearningcentres.org.uk/}
			\item Awards and Bursaries: \vspace{-0.1cm}
				\begin{itemize} \itemsep -1pt
				\item \url{https://www.sciencelearningcentres.org.uk/centres/national/awards-and-bursaries}
				\item \url{https://www.sciencelearningcentres.org.uk/about/impact-awards}
				\end{itemize}
			\item Resource collections: \url{https://www.sciencelearningcentres.org.uk/resources}
			\item Curriculum resources for primary, secondary, and tertiary education: \url{https://www.sciencelearningcentres.org.uk/curriculum}
			\end{enumerate}
		\end{enumerate}
	\end{enumerate}
%%%%%%%%%%%%%%%%%%%%%%%%%%%%%%%%%%%%%%%%
%%%%%%%%%%%%%%%%%%%%%%%%%%%%%%%%%%%%%%%%
\item Underrepresented minorities: \vspace{-0.3cm}
	\begin{enumerate} \itemsep -2pt
	\item University of Washington: \vspace{-0.2cm}
		\begin{enumerate} \itemsep -2pt
		\item Department of Computer Science and Engineering: \vspace{-0.1cm}
			\begin{enumerate} \itemsep -1pt
			\item {\it AccessComputing}: \vspace{-0.1cm}
				\begin{itemize} \itemsep -1pt
				\item \url{http://www.washington.edu/accesscomputing/}
				\item Has resources to help students with disabilities to pursue ``undergraduate and graduate degrees and careers in computing fields''.
				\item It runs the ``Summer Academy for Advancing Deaf \& Hard of Hearing in Computing'' for youths who are hearing impaired: \url{http://www.washington.edu/accesscomputing/dhh/academy/index.html}
				\end{itemize}
			\end{enumerate}
		\end{enumerate}
	%%%%%%%%%%%%%%%%%%%%%%%%%
	\item Engineer Girl: \vspace{-0.2cm}
		\begin{enumerate} \itemsep -2pt
		\item Resources for students, parents, and teachers to encourage girls to explore careers and educational opportunities in engineering
		\item Created by the National Academy of Sciences and The National Academy of Engineering
		\item Contests for K-12 students: \url{http://www.engineergirl.org/?id=4436}
		\item \url{http://www.engineergirl.org/}
		\end{enumerate}
	\item Engineering Your Life: \url{http://www.engineeryourlife.org/}
	\item GirlGeeks: \url{http://www.girlgeeks.org/}
	\item {\it Women in Science, Technology, Engineering, and Mathematics ON THE AIR!}: \vspace{-0.2cm}
		\begin{enumerate} \itemsep -2pt
		\item Audio resources that describe stories about women in science, technology, engineering, and mathematics (STEM) fields
		\item \url{http://www.womeninscience.org/}
		\end{enumerate}
	\item {\it Women Scientists in History}: \url{http://www.hypatiamaze.org/}
	\item Association for Women in Mathematics (AWM): \vspace{-0.2cm}
		\begin{enumerate} \itemsep -2pt
		\item \url{http://www.awm-math.org/}
		\item Education: \vspace{-0.1cm}
			\begin{enumerate} \itemsep -1pt
			\item \url{http://sites.google.com/site/awmmath/awm-resources/education}
			\item Includes information for students in middle school, high school, college and university (including graduate students). It also includes information for parents and teachers/educators.
			\end{enumerate}
		\item Women in Math, Science, and Society: \url{http://sites.google.com/site/awmmath/women-in-math-science-and-society}
		\item Essay contest on biographies of contemporary women in mathematics: \url{http://sites.google.com/site/awmmath/programs/essay-contest}
		\end{enumerate}
	\item Women in Technology (WIT): \vspace{-0.2cm}
		\begin{enumerate} \itemsep -2pt
		\item Girls in Technology: \vspace{-0.1cm}
			\begin{enumerate} \itemsep -1pt
			\item \url{http://www.girlsintechnology.org/}
			\item WIT Education Foundation: provides educational programs for girls in technology
			\item TeamBusiness Fundraiser: ``A combined fundraiser and program for girls in Grades 9-12 across the Metro DC area. Each year, up to forty girls participate with mentors and WIT volunteers in a full-day business simulation workshop conducted by TeamBusiness USA. The teams competed as companies, learning how to run a technology company in a fun and exciting simulation environment.''
			\item Hispanic Youth Foundation: ``In 2005, GIT established a partnership with the Hispanic Youth Foundation (HYF) and provided a grant to fund HYF�s innovative Laptops for Learning Dollars program, providing laptops and Internet connections for elementary and middle school students and their families in Arlington County and the City of Manassas.''
			\item Empower Girls -- CLCP Clubs: ``Empower Girls after-school programs were held at Hybla Valley Elementary School and Sacramento Community Center. GIT/WITEF provided funding to run these programs in conjunction with the Fairfax County Computer Learning Center Partnership (CLCP). The selected centers serve economically challenged communities in Fairfax County.''
			\end{enumerate}
		\end{enumerate}
	%%%%%%%%%%%%%%%%%%%%%%%%%
	\item National Society of Black Engineers (NSBE) competitions for high school/K-12 students: \url{http://www.nsbe.org/Programs/Competitions/NSBE-Jr-.aspx}
	\item The Society of Mexican American Engineers and Scientists (MAES): MAES PreCollege Outreach Programs, \url{http://www.maes-natl.org/index.php?module=ContentExpress&func=display&ceid=16&meid=236}
	\item {\it Center for the Advancement of Hispanics in Science and Engineering Education} (CAHSEE): \vspace{-0.2cm}
		\begin{enumerate} \itemsep -2pt
		\item STEM - The Science, Technology, Engineering \& Mathematics Institute (for students from grades 5 through 11): \url{http://www.cahsee.org/2programs/stem.asp.htm}
		\item YEP - Young Educators Program (fellows would learn how to train students in the aforementioned STEM Institute): \url{http://www.cahsee.org/2programs/yep.asp.htm}
		\item CAYSA - Central American Young Scholar Awards: \url{http://www.cahsee.org/2programs/caysa.asp.htm}. ``The CAYSA ceremonies honor more than 60 Washington, D.C. area high school seniors of Central American descent who have demonstrated remarkable success throughout all four years of high school. Students must be of Central American descent and have at least a 3.0 gpa.''
		\item Scholarships: \url{http://www.cahsee.org/6resources/scholarships.asp.htm}
		\item \url{http://www.cahsee.org/about/about.asp.htm}
		\end{enumerate}
	%%%%%%%%%%%%%%%%%%%%%%%%%
	\item International Computer Science Institute (UC Berkeley): \vspace{-0.2cm}
		\begin{enumerate} \itemsep -2pt
		\item Berkeley Foundation for Opportunities in Information Technology, BFOIT: \vspace{-0.1cm}
			\begin{enumerate} \itemsep -1pt
			\item BFOIT Programs for women and underrepresented minorities (African Americans and Chicanos/Latinos) in middle/high school who are interested in electrical/computer engineering and computer science careers: \url{http://www.bfoit.org/programs.html}
			\end{enumerate}
		\end{enumerate}
	\item Institute for Broadening Participation: \vspace{-0.2cm}
		\begin{enumerate} \itemsep -2pt
		\item PathwaysToScience.org: \vspace{-0.1cm}
			\begin{enumerate} \itemsep -1pt
			\item PathwaysToScience.org is a portal website supporting pathways to the STEM fields: science, technology, engineering, and mathematics.
			\item Particular emphasis is placed on connecting traditionally underrepresented groups with STEM programs and resources, including funding and mentoring opportunities. 
			\item For K-12 students: \url{http://www.pathwaystoscience.org/K12.asp}
			\item STEM Resources by Institution (colleges, universities, and US national research laboratories): \url{http://www.pathwaystoscience.org/Institution.asp}
			\item profiles of people and programs in STEM: \vspace{-0.3cm}
				\begin{itemize} \itemsep -2pt
				\item \url{http://www.pathwaystoscience.org/Profiles.asp}
				\item Find out about the career paths of underrepresented minorities in STEM
				\item Find out about programs that are offered by institutions for underrepresented minorities in STEM
				\end{itemize}
			\item Directory of partners (organizations that cooperate with or support the Institute for Broadening Participation): \url{http://www.pathwaystoscience.org/Partners.asp}
			\item Additional resources: \url{http://www.pathwaystoscience.org/Ideaexchange.asp}
			\end{enumerate}
		\item Maine Pathways to STEM (Science, Technology, Engineering \& Mathematics): \vspace{-0.1cm}
			\begin{enumerate} \itemsep -1pt
			\item \url{http://www.mainestem.org/}
			\item K-12 Teachers \& University Faculty: \url{http://www.mainestem.org/METeachersFaculty.asp}
			\item K-12 STEM Resources: \url{http://www.mainestem.org/MEK12.asp}
			\end{enumerate}
		\end{enumerate}
	\item Building Engineering and Science Talent, BEST: \vspace{-0.2cm}
		\begin{enumerate} \itemsep -2pt
		\item \url{http://www.bestworkforce.org/}
		\item Publications: \url{http://www.bestworkforce.org/publications.htm}
		\item List of programs to help underrepresented minority students in K-12 schools explore careers in STEM: \url{http://www.bestworkforce.org/links.htm}
		\end{enumerate}
	\item American Indian Science and Engineering Society (AISES): \vspace{-0.2cm}
		\begin{enumerate} \itemsep -2pt
		\item Pre-college programs: \vspace{-0.1cm}
			\begin{enumerate} \itemsep -1pt
			\item \url{http://www.aises.org/Programs}
			\item Resources: \url{http://www.aises.org/Programs/Resources}
			\end{enumerate}
		\end{enumerate}
	\end{enumerate}
\end{enumerate}







%%%%%%%%%%%%%%%%%%%%%%%%%%%%%%%%%%%%%%%%%%%
\subsection{Science \& Engineering Outreach for Undergraduates, Grad Students, \& Postdocs}
\label{stemoutreachcollegegradsch}


Science, mathematics, and engineering outreach to undergraduates, graduate students, and postdocs: \vspace{-0.3cm}
\begin{enumerate} \itemsep -4pt
\item Mac Hyman, ``Good Choices for Great Careers in the Mathematical Sciences,'' talk given at 2008 SIAM Annual Meeting. Available at: \url{http://client.blueskybroadcast.com/siam08/hyman/index.html}; last accessed on August 25, 2010. Also, see \url{http://meetings.siam.org/program.cfm?CONFCODE=AN08}, \url{http://www.siam.org/meetings/an08/program.php}, and \url{http://www.siam.org/meetings/an08/}.
\item {\it Accreditation.org}: \vspace{-0.3cm}
	\begin{enumerate} \itemsep -2pt
	\item Information about the accreditation of engineering degree programs around the world
	\item \url{http://www.accreditation.org/}
	\end{enumerate}
\item John Baez, ``How to Learn Math and Physics,'' Department of Mathematics, University of California, Riverside, December 24, 2007. Available at: \url{http://math.ucr.edu/home/baez/books.html}; last accessed on August 28, 2010.
\item {\it MentorNet}: \vspace{-0.3cm}
	\begin{enumerate} \itemsep -2pt
	\item \url{http://www.mentornet.net/}
	\item Enables people to network with scientists, engineers, and professors in Science, Technology, Engineering, and Mathematics (STEM)
	\item Is very supportive of minorities, so that more minorities (particularly underrepresented minorities) can be attracted to STEM careers
	\end{enumerate}
\item {\it The Indus Entrepreneurs (TiE)} for networking among high-tech entrepreneurs, start-up co-founders, venture capitalists, and angel investors: \url{http://www.tie.org/}
\item National Academy of Engineering, NAE: \vspace{-0.3cm}
	\begin{enumerate} \itemsep -2pt
	\item Includes a list of NAE Grand Challenges, which can provide some suggestions for research trajectories
	\item Summit Series on the Grand Challenges: Includes the National Grand Challenges Summits
	\item \url{http://www.engineeringchallenges.org/}
	\end{enumerate}
\item {\it National Society of Professional Engineers}: \vspace{-0.3cm}
	\begin{enumerate} \itemsep -2pt
	\item Student Resources: \vspace{-0.2cm}
		\begin{enumerate} \itemsep -2pt
		\item \url{http://www.nspe.org/Students/Resources/index.html}
		\item An Employment Guidelines Checklist for the Engineer Job Applicant: \url{http://www.nspe.org/Students/Resources/checklist.html}
		\end{enumerate}
	\item Career Center: \url{http://www.nspe.org/CareerCenter/index.html}
	\item A Sightseer's Guide to Engineering: \url{http://www.engineeringsights.org/}
	\end{enumerate}
\item {\it JustGarciaHill} ``Study Skills for Budding Scientists'': \url{http://www.justgarciahill.org/index.php/science-study-skills.html}
\item {\it NASA} resources for students: \vspace{-0.3cm}
	\begin{enumerate} \itemsep -2pt
	\item \url{http://www.nasa.gov/audience/forstudents/index.html}
	\item NASA University Student Launch Initiative, or USLI: \url{http://www.nasa.gov/offices/education/programs/descriptions/University_Student_Launch_Initiative.html}
	\end{enumerate}
\item {\it iTunes U}: \vspace{-0.3cm}
	\begin{enumerate} \itemsep -2pt
	\item {\it iTunes} is required to listen to or watch these lectures, talks, and presentations.
	\item Access {\it iTunes U} at: \url{http://www.apple.com/education/itunes-u/} or \url{http://deimos3.apple.com/indigo/main/main.html?v0=WWW-AMUS-ITUNESU070521-N48LX}
	\item {\it iTunes U} is a set of webcast and podcasts, where you can easily find audio and video clips for lectures, seminars, announcements, virtual tours, and so on. For example, some professors from schools like MIT or Berkeley will provide audio/video clips of their lectures on {\it iTunes U}.
	\item This can help in exploring different majors before a college student declares her/his major(s). If a student is not sure if she/he wants to double major in deaf studies and linguistics, this student can check out some linguistics lectures from her/his (preferred) college/university, if it uses {\it iTunes U}, or those from other universities.
	\end{enumerate}
\item Harvey Mudd College: \vspace{-0.3cm}
	\begin{enumerate} \itemsep -2pt
	\item Francis Edward Su, {\it Math Fun Facts!}, Department of Mathematics, Harvey Mudd College: \url{http://www.math.hmc.edu/funfacts/}
	\end{enumerate}
\item Engineering Pathway: \url{http://www.engineeringpathway.com/ep/index.jhtml}
\item Rochester Institute of Technology, ``Biology \& Biotechnology Paid Co-op/Internships for 2011,'' Department of Biological Sciences, Rochester Institute of Technology: \url{http://people.rit.edu/gtfsbi/Symp/summer.htm}
\item {\it Mathematical Association of America (MAA)} information on educational pathways and career opportunities: \vspace{-0.3cm}
	\begin{enumerate} \itemsep -2pt
	\item Undergraduate Students: \url{http://www.maa.org/students/undergrad/}
	\item Graduate Students: \url{http://www.maa.org/students/grad/}
	\item Underrepresented Groups: \url{http://www.maa.org/programs/underrep.html}
	\item Mathematical Association of America (MAA) MathFest (for students in mathematics): \url{http://www.maa.org/mathfest/}
	\item MAA Online Columns: \url{http://www.maa.org/news/columns.html}
	\end{enumerate}
\item New Zealand Institute of Mathematics and its Applications (NZIMA): \vspace{-0.3cm}
	\begin{enumerate} \itemsep -2pt
	\item {\it MathsReach}: Careers (information about careers based on a higher education in mathematics or related field): \url{http://www.mathsreach.org/Careers}
	\end{enumerate}
\item {\it Engineers Dedicated to a Better Tomorrow (a.k.a., DedicatedEngineers)}: \vspace{-0.3cm}
	\begin{enumerate} \itemsep -2pt
	\item [Resources for] College Students and Faculty/Staff Members: \url{http://www.dedicatedengineers.org/intro_for_college.htm}
	\item \url{http://www.dedicatedengineers.org/}
	\end{enumerate}
\item American Institute of Physics: \vspace{-0.3cm}
	\begin{enumerate} \itemsep -2pt
	\item GradschoolShopper.com: \vspace{-0.2cm}
		\begin{enumerate} \itemsep -2pt
		\item \url{http://www.gradschoolshopper.com/}
		\item ``Find information on graduate programs in physics, astronomy, and other physical sciences''
		\end{enumerate}
	\item Career guidance for high school and undergraduate students: \url{http://www.aip.org/statistics/trends/career.html}
	\item American Geophysical Union: \vspace{-0.2cm}
		\begin{enumerate} \itemsep -2pt
		\item Diversity Programs: \url{http://www.agu.org/education/diversity_programs/}
		\end{enumerate}
	\end{enumerate}
\item {\it icademic.org} resources for the life sciences and engineering: \url{http://www.icademic.org/}
\item The Oceanography Society: \vspace{-0.3cm}
	\begin{enumerate} \itemsep -2pt
	\item Hands-On Oceanography: peer-reviewed activities appropriate for undergraduate and/or graduate classes in oceanography, \url{http://www.tos.org/hands-on/index.html}
	\end{enumerate}
%%%%%%%%%%%%%%%%%%%%%%%%%%%%%%%%%%%%%%%
%%%%%%%%%%%%%%%%%%%%%%%%%%%%%%%%%%%%%%%
\item outreach activities (including mentoring) to students in K-12: \vspace{-0.3cm}
	\begin{enumerate} \itemsep -2pt
	\item Research Councils UK (RCUK): \vspace{-0.2cm}
		\begin{enumerate} \itemsep -2pt
		\item Researchers in Residence (RinR): \vspace{-0.1cm}
			\begin{enumerate} \itemsep -1pt
			\item \url{http://www.researchersinresidence.ac.uk/cms/}
			\item \url{http://www.researchersinresidence.ac.uk/cms/researchers/}
			\item Mentor middle and high school students who are job shadowing (observing you first-hand) in your research activities for up to a week, so that they can learn what doing research in your research area is like. You should explain in laypeople's terms what your research is about. That is, be a mentor for the externships of middle and high school students.
			\end{enumerate}
		\end{enumerate}
	\end{enumerate}
%%%%%%%%%%%%%%%%%%%%%%%%%%%%%%%%%%%%%%%
%%%%%%%%%%%%%%%%%%%%%%%%%%%%%%%%%%%%%%%
\item competitions: \vspace{-0.3cm}
	\begin{enumerate} \itemsep -2pt
	\item Invent Now, Inc.: \vspace{-0.2cm}
		\begin{enumerate} \itemsep -2pt
		\item Collegiate Inventors Competition: \url{http://www.invent.org/collegiate/} [ Resources for {\color{blue} Patent Search Strategy} are available. \colorbox{blue}{\bf This is the ultimate competition for US students in science and engineering.} ]
		\end{enumerate}
	\item INFORMS Doing Good with Good OR - Student Competition: \vspace{-0.2cm}
		\begin{enumerate} \itemsep -2pt
		\item Doing Good with Good OR-Student Competition is held each year to identify and honor outstanding projects in the field of operations research and the management sciences conducted by a student or student group that have a significant societal impact.
		\item \url{http://www.informs.org/Recognize-Excellence/INFORMS-Prizes-Awards/Doing-Good-with-Good-OR}
		\end{enumerate}
	\item AWM Essay Contest: Biographies of Contemporary Women in Mathematics; see \url{http://www.awm-math.org/biographies/contest.html}
	\item American Society of Mechanical Engineers (ASME): \vspace{-0.2cm}
		\begin{enumerate} \itemsep -2pt
		\item Student Design Competition: \url{http://www.asme.org/Events/Contests/DesignContest/Student_Design_Competition.cfm}
		\item ASME Student Mechanism and Robot Design Competition: \url{http://www.asme.org/Events/Contests/Student_Mechanism_Robot_2.cfm}
		\end{enumerate}
	\item American Institute of Chemical Engineers (AIChE) competitions: \url{http://www.aiche.org/Students/Awards/index.aspx}
	\item Association for Unmanned Vehicle Systems International (AUVSI): \vspace{-0.2cm}
		\begin{enumerate} \itemsep -2pt
		\item AUVSI Student Competitions: \vspace{-0.1cm}
			\begin{enumerate} \itemsep -1pt
			\item \url{http://www.auvsi.org/AUVSI/AUVSI/Home/Default.aspx}, or \url{http://www.auvsi.org/}
			\item Annual Intelligent Ground Vehicle Competition (IGVC): \url{http://www.igvc.org/}
			\item Annual Student Unmanned Air System (SUAS) Competition: \url{http://65.210.16.57/studentcomp2010/default.html}
			\item International Aerial Robotics Competition (IARC): \url{http://iarc.angel-strike.com/}
			\item AUVSI and ONR's International Autonomous Surface Vehicle (ASV) Competition [ASVC]
			\item AUVSI Foundation and ONR's (U.S. Office of Naval Research) 4th International RoboBoats Competition: \url{http://www.auvsifoundation.org/AUVSI/FOUNDATION/Competitions/ASVCompetition/Default.aspx?C=00000000-0000-0000-0000-000000000000}
			\item AUVSI Foundation and ONR's (U.S. Office of Naval Research) International RoboSub Competition (or AUVSI and ONR's International Autonomous Underwater Vehicle Competition): \url{http://www.auvsifoundation.org/AUVSI/FOUNDATION/Competitions/AUVCompetition/Default.aspx}
			\item ONR: U.S. Office of Naval Research
			\end{enumerate}
		\end{enumerate}
	\item American Institute of Aeronautics and Astronautics (AIAA): \vspace{-0.2cm}
		\begin{enumerate} \itemsep -2pt
		\item Design Competitions: \url{http://www.aiaa.org/content.cfm?pageid=210}
		\end{enumerate}
	\item National Aeronautics and Space Administration: \vspace{-0.2cm}
		\begin{enumerate} \itemsep -2pt
		\item NASA's Langley Research Center: \vspace{-0.1cm}
			\begin{enumerate} \itemsep -1pt
			\item SpaceTech Engineering Design Challenge: \url{http://spacetech.larc.nasa.gov}
			\end{enumerate}
		\end{enumerate}
	\item American Concrete Institute (ACI): \vspace{-0.2cm}
		\begin{enumerate} \itemsep -2pt
		\item Competitions: \url{http://www.concrete.org/STUDENTS/st_competitions.htm}
		\end{enumerate}
	\end{enumerate}
%%%%%%%%%%%%%%%%%%%%%%%%%%%%%%%%%%%%%%%
%%%%%%%%%%%%%%%%%%%%%%%%%%%%%%%%%%%%%%%
\item underrepresented minorities: \vspace{-0.3cm}
	\begin{enumerate} \itemsep -2pt
	\item The Society of Women Engineers: \url{http://societyofwomenengineers.swe.org/}
	\item Association for Women in Science (AWIS): \url{http://www.awis.org/} and \url{http://www.awis.affiniscape.com/displaycommon.cfm?an=1&subarticlenbr=19}
	\item Association for Women in Mathematics (AWM): \vspace{-0.2cm}
		\begin{enumerate} \itemsep -2pt
		\item \url{http://www.awm-math.org/}
		\item Education: \vspace{-0.1cm}
			\begin{enumerate} \itemsep -1pt
			\item \url{http://sites.google.com/site/awmmath/awm-resources/education}
			\item Includes information for students in middle school, high school, college and university (including graduate students). It also includes information for parents and teachers/educators.
			\end{enumerate}
		\item Career advice and opportunities: \url{http://sites.google.com/site/awmmath/awm-resources/career}
		\item Women in Math, Science, and Society: \url{http://sites.google.com/site/awmmath/women-in-math-science-and-society}
		\item Essay contest on biographies of contemporary women in mathematics: \url{http://sites.google.com/site/awmmath/programs/essay-contest}
		\end{enumerate}
	\item Sigma Delta Epsilon-Graduate Women in Science (GWIS): \url{http://www.gwis.org/}
	\item Society of Hispanic Professional Engineers (SHPE): \vspace{-0.2cm}
		\begin{enumerate} \itemsep -2pt
		\item Advancing Hispanic Excellence in Technology, Engineering, Math and Science (AHETEMS) Foundation: \url{http://www.ahetems.org/}
		\item AHETEMS Scholarship Program: \url{http://www.ahetems.org/scholarships/}
		\item Graduate \& Young Professional Fellowship Program (encourage young professionals to engage in {\bf public policy}): \url{http://www.ahetems.org/graduate/graduate-young-professional-fellowship-program/}
		\item SHPE/GEM Fellowship (for graduate students in STEM at GEM Member Universities): \url{http://www.ahetems.org/graduate/shpe-gem-graduate-award/}. See \url{http://www.gemfellowship.org/gem-universities/university-members} for a list of GEM member universities.
		\item Internship opportunities: \url{http://www.ahetems.org/scholar-internships/}
		\item \url{http://oneshpe.shpe.org/wps/portal/national}
		\end{enumerate}
	\item National Society of Black Engineers (NSBE): \vspace{-0.2cm}
		\begin{enumerate} \itemsep -2pt
		\item Scholarships: \url{http://www.nsbe.org/Programs/Scholarships.aspx}
		\item Competitions for undergraduates and graduate students: \url{http://www.nsbe.org/Programs/Competitions/Collegiate.aspx}
		\item \url{http://www.nsbe.org/}
		\end{enumerate}
	\item The Society of Mexican American Engineers and Scientists (MAES): \vspace{-0.2cm}
		\begin{enumerate} \itemsep -2pt
		\item MAES Undergraduate and Graduate Outreach Programs (including ``GRE/Graduate Application Fee Waivers''): \url{http://www.maes-natl.org/index.php?module=ContentExpress&func=display&ceid=90&meid=238}
		\item Scholarships \& Awards: \url{http://www.maes-natl.org/index.php?meid=328}
		\item MAES Scholarship Program: \url{http://www.maes-natl.org/index.php?module=ContentExpress&func=display&ceid=518&meid=241}
		\end{enumerate}
	\item SACNAS (Society for Advancement of Chicanos and Native Americans in Science): \vspace{-0.2cm}
		\begin{enumerate} \itemsep -2pt
		\item Scholarships: \url{http://www.sacnas.org/webadindex.cfm?webadcategory_id=7}
		\item Fellowships: \url{http://www.sacnas.org/webadIndex.cfm?webadcategory_id=5}
		\end{enumerate}
	\item {\it Center for the Advancement of Hispanics in Science and Engineering Education} (CAHSEE): \vspace{-0.2cm}
		\begin{enumerate} \itemsep -2pt
		\item YESP - Young Engineers \& Scientists Program: \url{http://www.cahsee.org/2programs/yesp.asp.htm}. ``This program places talented Hispanic college students in the research labs of government agencies.''
		\item Scholarships: \url{http://www.cahsee.org/6resources/scholarships.asp.htm}
		\end{enumerate}
	\item American Geophysical Union: \vspace{-0.2cm}
		\begin{enumerate} \itemsep -2pt
		\item Has a list of organizations for specific underrepresented ethnic-minority groups in the geosciences and physics: \vspace{-0.1cm}
			\begin{enumerate} \itemsep -1pt
			\item \url{http://www.agu.org/education/diversity_programs/}
			\item These organizations may have information about scholarships, fellowships, and educational material for K-12 and college students.
			\end{enumerate}
		\end{enumerate}
	\item Institute for Broadening Participation: \vspace{-0.2cm}
		\begin{enumerate} \itemsep -2pt
		\item Minorities Striving and Pursuing Higher Degrees of Success in Earth System Science (MS PHD'S\textregistered) initiative: \vspace{-0.1cm}
			\begin{enumerate} \itemsep -1pt
			\item \url{http://www.msphds.org/}
			\item Prospective Students/Mentees: \url{http://www.msphds.org/prospective.asp}
			\item For MS PHD'S Students: \url{http://www.msphds.org/students.asp}
			\end{enumerate}
		\item PathwaysToScience.org: \vspace{-0.1cm}
			\begin{enumerate} \itemsep -1pt
			\item Resources for undergraduate students: \url{http://www.pathwaystoscience.org/Undergrads.asp}
			\item Resources for graduate students: \url{http://www.pathwaystoscience.org/Grad.asp}
			\item Resources for postdocs: \url{http://www.pathwaystoscience.org/Postdocs_portal.asp}
			\item STEM Resources by Institution (colleges, universities, and US national research laboratories): \url{http://www.pathwaystoscience.org/Institution.asp}
			\item Additional resources: \url{http://www.pathwaystoscience.org/Ideaexchange.asp}
			\end{enumerate}
		\item National Alliance for Doctoral Studies in the Mathematical Sciences: \vspace{-0.1cm}
			\begin{enumerate} \itemsep -1pt
			\item \url{http://www.mathalliance.org/}
			\item Student/Alliance Scholars: \url{http://www.mathalliance.org/scholars.asp}
			\item Alliance Mentors / Alliance Undergraduate Mentors: \url{http://www.mathalliance.org/mentors.asp}
			\item Alliance Programs: \url{http://www.mathalliance.org/programs.asp}
			\end{enumerate}
		\item Alliances for Graduate Education and the Professoriate (AGEP): \vspace{-0.1cm}
			\begin{enumerate} \itemsep -1pt
			\item \url{http://www.agep.us/}
			\end{enumerate}
		\item Maine Pathways to STEM (Science, Technology, Engineering \& Mathematics): \vspace{-0.1cm}
			\begin{enumerate} \itemsep -1pt
			\item \url{http://www.mainestem.org/}
			\item K-12 Teachers \& University Faculty: \url{http://www.mainestem.org/METeachersFaculty.asp}
			\item Graduate \& Undergraduate Students: \url{http://www.mainestem.org/MEUndergradGrad.asp}
			\end{enumerate}
		\end{enumerate}
	\item ARTSI (Advancing Robotics Technology for Societal Impact) Alliance: \vspace{-0.2cm}
		\begin{enumerate} \itemsep -2pt
		\item \url{http://artsialliance.org/}
		\item ``The ARTSI (Advancing Robotics Technology for Societal Impact) Alliance is a collaborative education and research project centered around robotics for healthcare, the arts, and entrepreneurship.  Spelman College, a historically black college (HBCU) for women is leading the alliance in partnership with several other HBCUs and Research I (R1) institutions.''
		\item Summer REU (Research Experience for Undergraduates) program: \url{http://artsialliance.org/Summer-REU-Program}
		\end{enumerate}
	\item Women in Technology (WIT): \vspace{-0.2cm}
		\begin{enumerate} \itemsep -2pt
		\item \url{http://www.womenintechnology.org/index.asp}
		\item WIT Mentor-Prot{\'{e}}g{\'{e}} Program: \url{http://www.womenintechnology.org/content.asp?contentid=59}
		\item {\bf \color{blue} WIT Career Transition Resource Guide}: \url{http://www.womenintechnology.org/content.asp?contentid=146}
		\item Girls In Technology (GIT): \vspace{-0.1cm}
			\begin{enumerate} \itemsep -1pt
			\item Get Involved: \vspace{-0.1cm}
				\begin{itemize} \itemsep -1pt
				\item \url{http://www.girlsintechnology.org/getinvolved.cfm}
				\item Teacher: teach girls about IT as an after-school activity or in a summer camp session
				\item Assistant Teacher: Assist instructors in GIT sessions, after-school activities, or summer camp sessions
				\item Develop Curriculum: Develop a curriculum for a supported GIT educational program
				\item Mentor: Mentor a girl in one of [GIT's] supported programs
				\item Job Shadow: ``Let a girl shadow you at work''
				\item Guest Speaker: ``Speak to a group of girls on a topic both you and they enjoy, such as computers, technology, education, how to take apart computers, how to build a web site, etc.''
				\end{itemize}
			\end{enumerate}
		\end{enumerate}
	\item Arizona State University: \vspace{-0.2cm}
		\begin{enumerate} \itemsep -2pt
		\item {\it Career}WISE: \vspace{-0.1cm}
			\begin{enumerate} \itemsep -1pt
			\item \url{http://careerwise.asu.edu/}
			\item Helpful resources for female graduate/Ph.D. students in science and engineering.
			\end{enumerate}
		\end{enumerate}
	\item American Indian Science and Engineering Society (AISES): \vspace{-0.2cm}
		\begin{enumerate} \itemsep -2pt
		\item Programs for undergraduates and grad students (including scholarships and internships): \vspace{-0.1cm}
			\begin{enumerate} \itemsep -1pt
			\item \url{http://www.aises.org/Programs}
			\item Resources: \url{http://www.aises.org/Programs/Resources}
			\end{enumerate}
		\end{enumerate}
	\end{enumerate}
\end{enumerate}




%%%%%%%%%%%%%%%%%%%%%%%%%%%%%%%%%%%%%%%%%%%
\subsection{Other Science and Engineering Outreach}
\label{otherstemoutreach}

Other Science and Engineering Outreach: \vspace{-0.3cm}
\begin{enumerate} \itemsep -4pt
\item Frontiers of Engineering (networking event for mid-career engineers): \url{http://www.naefrontiers.org/}
\item Consortium for Ocean Leadership: \vspace{-0.3cm}
	\begin{enumerate} \itemsep -2pt
	\item Resources for scientists in the marine sciences to use in outreach activities: \url{http://www.oceanleadership.org/education/deep-earth-academy/scientists/}
	\end{enumerate}
\item The Oceanography Society: \vspace{-0.3cm}
	\begin{enumerate} \itemsep -2pt
	\item Education and Public Outreach (EPO): A Guide for Scientists [material that scientists and professors can use for outreach activities], \url{http://www.tos.org/epo_guide/index.html}
	\end{enumerate}
\item The Joy McCann Foundation: \vspace{-0.3cm}
	\begin{enumerate} \itemsep -2pt
	\item McCann Scholar (for professors in medicine, science, and nursing): \url{http://www.mccannfoundation.org/scholars.htm}
	\item The Joy McCann Professorship for Women in Medicine: \url{http://www.mccannfoundation.org/medicine.htm}
	\end{enumerate}
\item U.S. National Academies: \vspace{-0.3cm}
	\begin{enumerate} \itemsep -2pt
	\item International Activities of the U.S. National Academies -- Science, Engineering \& Medicine: Working toward a better world: \vspace{-0.2cm}
		\begin{enumerate} \itemsep -2pt
		\item \url{http://sites.nationalacademies.org/International/}
		\item Solving the grand challenges: \vspace{-0.1cm}
			\begin{enumerate} \itemsep -1pt
			\item Energy and the Environment
			\item Global Health
			\item Water Resources
			\item Agriculture and Food Security
			\item International Security
			\item Population
			\end{enumerate}
		\item Help other countries build/improve their capacities: \vspace{-0.1cm}
			\begin{enumerate} \itemsep -1pt
			\item Cooperative Program with Pakistan 
			\item African Science Academies 
			\item Visiting Math Lecturer Program in Cambodia 
			\item Humanitarian Relief Efforts
			\item Improved Road Safety
			\item Science-based Decision Making for Sustainability
			\item Science Academies' Input to G8 Summits
			\end{enumerate}
		\item Scientific Cooperation: \vspace{-0.1cm}
			\begin{enumerate} \itemsep -1pt
			\item Building Bridges in the Middle East
			\item Cooperation with Iran
			\item Human Rights
			\item Frontiers of Science and Engineering Symposia
			\item Travel Grants
			\item International Conference on Women's Issues in Transportation
			\end{enumerate}
		\item Advising the U.S. Government: \vspace{-0.1cm}
			\begin{enumerate} \itemsep -1pt
			\item Science \& Technology in Foreign Policy
			\item Health 
			\item Science and Security
			\end{enumerate}
		\end{enumerate}
	\end{enumerate}
\item National Academy of Engineering: \vspace{-0.3cm}
	\begin{enumerate} \itemsep -2pt
	\item The Charles Stark Draper Prize (``to recognize innovative engineering achievements and their reduction to practice in ways that have led to important benefits and significant improvement in the well being and freedom of humanity''): \url{http://www.draperprize.org/}
	\item NAE Grand Challenge Scholars Program: \url{http://www.grandchallengescholars.org/}
	\end{enumerate}
\item United States Department of Defense (DoD): \vspace{-0.3cm}
	\begin{enumerate} \itemsep -2pt
	\item National Defense Education Program; Defense Advanced Research Projects Agency (DARPA): \vspace{-0.2cm}
		\begin{enumerate} \itemsep -2pt
		\item Resource for scientists and engineers to mentor youths, so that they would look into pursuing careers in science and engineering: \url{http://www.ndep.us/GetInvoSci.aspx}
		\item STEM Learning Modules (SLM): \vspace{-0.1cm}
			\begin{enumerate} \itemsep -1pt
			\item \url{http://www.ndep.us/ProgSLM.aspx}
			\item Help educators develop programs in science and engineering in K-12 institutions, so that youths would be encouraged to explore careers in science and engineering
			\end{enumerate}
		\end{enumerate}
	\end{enumerate}
\item Hewlett-Packard Development Company: \vspace{-0.3cm}
	\begin{enumerate} \itemsep -2pt
	\item HP Catalyst Initiative (grants for STEM education in colleges and universities): \url{http://www.hp.com/hpinfo/socialinnovation/catalyst.html}
	\item HP EdTech Innovators Award (for higher educational institutions that integrate IT into the curricular): \url{http://www.hp.com/hpinfo/socialinnovation/edtech.html}
	\end{enumerate}
\item The William and Flora Hewlett Foundation (Hewlett Foundation): \vspace{-0.3cm}
	\begin{enumerate} \itemsep -2pt
	\item Funding Programs: \url{http://www.hewlett.org/programs}
	\item Grantseekers: \url{http://www.hewlett.org/grants/grantseekers}
	\end{enumerate}
\item The Sloan Consortium (Sloan-C): \vspace{-0.3cm}
	\begin{enumerate} \itemsep -2pt
	\item Sloan-C Awards (for recognizing outstanding work in the field of online education) and Sloan-C Fellows: \url{http://sloanconsortium.org/aboutus/awards}
	\item Mayadas Leadership Award in Online Education: \url{http://sloanconsortium.org/mayadas_award}
	\end{enumerate}
\item W.K. Kellogg Foundation: \vspace{-0.3cm}
	\begin{enumerate} \itemsep -2pt
	\item Grant database: \url{http://www.wkkf.org/grants/grants-database.aspx}
	\end{enumerate}
\item Hewlett-Packard Company: \vspace{-0.3cm}
	\begin{enumerate} \itemsep -2pt
	\item HP community investment for education, economic development, and the environment: \url{http://www.hp.com/hpinfo/socialinnovation/focus.html}
	\item Entrepreneurship education: \vspace{-0.2cm}
		\begin{enumerate} \itemsep -2pt
		\item \url{http://www.hp.com/hpinfo/globalcitizenship/society/social/entrepreneurship.html}
		\item HP Graduate Entrepreneurship Training through IT (GET-IT)
		\item HP Entrepreneurship Learning Program (HELP)
		\end{enumerate}
	\item HP Innovations in Education grants: \url{http://www.hp.com/hpinfo/globalcitizenship/society/social/innovations.html}
	\end{enumerate}
\item General Electric Company: \vspace{-0.3cm}
	\begin{enumerate} \itemsep -2pt
	\item GE Foundation: \vspace{-0.2cm}
		\begin{enumerate} \itemsep -2pt
		\item Developing Futures\texttrademark\ in Education program (which encompasses the GE College Bound Program): \url{http://www.ge.com/foundation/developing_futures_in_education/index.jsp}
		\item Environment, health and safety, and health industry training programs (outside the US): \url{http://www.ge.com/foundation/international_programs/training.jsp}
		\item Student, education and scholarship initiatives: \url{http://www.ge.com/foundation/international_programs/education_initiatives.jsp}
		\end{enumerate}
	\end{enumerate}
\item The GRAMMY Foundation: \vspace{-0.3cm}
	\begin{enumerate} \itemsep -2pt
	\item GRAMMY Foundation Grants: \vspace{-0.2cm}
		\begin{enumerate} \itemsep -2pt
		\item \url{http://www2.grammy.com/GRAMMY_Foundation/Grants/}
		\item It funds {\bf Scientific Research Projects} as well as {\it Archiving And Preservation Projects}.
		\item Concerning scientific research projects: ``The GRAMMY Foundation Grant Program awards grants to organizations and individuals to support research on the impact of music on the human condition. Examples might include the study of the effects of music on mood, cognition and healing, as well as the medical and occupational well-being of music professionals and the creative process underlying music.'' [ E.g., look at music therapy as a possible research topic/area. ]
		\end{enumerate}
	\end{enumerate}
\item The Dana Foundation: \vspace{-0.3cm}
	\begin{enumerate} \itemsep -2pt
	\item \url{http://www.dana.org/grants/}
	\item Has grants for: \vspace{-0.2cm}
		\begin{enumerate} \itemsep -2pt
		\item Brain and Immuno-Imaging
		\item Clinical Neuroscience
		\item Human Immunology
		\item Neuroimmunology of Brain Infections and Cancers
		\end{enumerate}
		\item Deadlines and Requests for Proposals (RFP): \url{http://www.dana.org/grants/deadlines.aspx}
	\end{enumerate}
%%%%%%%%%%%%%%%%%%%%%%%%%%%%%%%%%%%%%%%
% underrepresented minorities
\item Institute for Broadening Participation: \vspace{-0.3cm}
	\begin{enumerate} \itemsep -2pt
	\item PathwaysToScience.org: \vspace{-0.2cm}
		\begin{enumerate} \itemsep -2pt
		\item Resources for faculty and administrators (to facilitate STEM outreach activities as well as the recruitment of underrepresented minorities to the student body and faculty): \url{http://www.pathwaystoscience.org/Faculty.asp}
		\end{enumerate}
	\end{enumerate}
\item National Center for Women \& Information Technology (NCWIT): \vspace{-0.3cm}
	\begin{enumerate} \itemsep -2pt
	\item NCWIT Academic Alliance Seed Fund (for developing and implementing initiatives in colleges and universities to recruit and retain women in computing and information technology): \url{http://www.ncwit.org/work.awards.seed.html}
	\item NCWIT Symons Innovator Award (for outstanding women who have successfully built and funded an IT business): \url{http://www.ncwit.org/work.awards.innovator.html}
	\end{enumerate}
\item Women in Technology (WIT): \vspace{-0.3cm}
	\begin{enumerate} \itemsep -2pt
	\item Girls In Technology (GIT): \vspace{-0.2cm}
		\begin{enumerate} \itemsep -2pt
		\item Get Involved: \vspace{-0.1cm}
			\begin{itemize} \itemsep -1pt
			\item \url{http://www.girlsintechnology.org/getinvolved.cfm}
			\item Teacher: teach girls about IT as an after-school activity or in a summer camp session
			\item Assistant Teacher: Assist instructors in GIT sessions, after-school activities, or summer camp sessions
			\item Develop Curriculum: Develop a curriculum for a supported GIT educational program
			\item Mentor: Mentor a girl in one of [GIT's] supported programs
			\item Job Shadow: ``Let a girl shadow you at work''
			\item Guest Speaker: ``Speak to a group of girls on a topic both you and they enjoy, such as computers, technology, education, how to take apart computers, how to build a web site, etc.''
			\end{itemize}
		\end{enumerate}
	\end{enumerate}
\item European Platform of Women Scientists (EPWS): \vspace{-0.3cm}
	\begin{enumerate} \itemsep -2pt
	\item \url{http://www.epws.org/}
	\item Members: \url{http://www.epws.org/index.php?option=com_content&task=blogcategory&id=134&Itemid=4652}
	\end{enumerate}
\end{enumerate}





Commercializing academic research into products and services via start-ups: \vspace{-0.3cm}
\begin{enumerate} \itemsep -4pt
\item Ben Franklin Technology Partners (BFTP): \vspace{-0.3cm}
	\begin{enumerate} \itemsep -2pt
	\item Innovation Works (IW): \vspace{-0.2cm}
		\begin{enumerate} \itemsep -2pt
		\item For universities in the Pittsburgh metropolitan area
		\item University Innovation Grants (UIGs) / University Grants: \vspace{-0.1cm}
			\begin{enumerate} \itemsep -1pt
			\item For technology validation, market research, prototype development, and intellectual property evaluation
			\item Available online at: \url{http://www.innovationworks.org/OurPrograms/UniversityGrants/tabid/115/Default.aspx}; last accessed on November 14, 2010.
			\end{enumerate}
		\end{enumerate}
	\end{enumerate}
\end{enumerate}









%%%%%%%%%%%%%%%%%%%%%%%%%%%%%%%%%%%%%%%%%%%
\subsection{Electrical and Computer Engineering \& Computer Science Outreach}
\label{ececsoutreach}

Electrical and computer engineering, and computer science outreach: \vspace{-0.3cm}
\begin{enumerate} \itemsep -4pt
\item IEEE: \vspace{-0.3cm}
	\begin{enumerate} \itemsep -2pt
	\item {\it IEEE-USA Salary Service} provides a survey of jobs in electrical and computer engineering: \url{http://www.ieeeusa.org/careers/salary/}
	\item {\it IEEE Santa Clara Valley Section PACE}: Professional Activities Committee for Engineers (PACE); see \url{http://www.ewh.ieee.org/r6/scv/PACE/}
	\item {\it IEEE Santa Clara Valley Section}: \url{http://ewh.ieee.org/r6/scv/} and \url{http://www.ieee.org/scv}
	\item 
	\end{enumerate}
\item Association for Computing Machinery, ACM: \vspace{-0.3cm}
	\begin{enumerate} \itemsep -2pt
	\item Sanjeev Arora, Boaz Barak, and Luca Trevisan, ``Survey Papers and Essays,'' in {\it Theory Matters Wiki: Theoretical Computer Science (TCS) Advocacy Wiki}, SIGACT Committee for the Advancement of Theoretical Computer Science, ACM Special Interest Group on Algorithms and Computation Theory (SIGACT), Association for Computing Machinery, February 25, 2010. Available at: \url{http://theorymatters.org/pmwiki/pmwiki.php?n=Main.SurveyCollection}; last accessed on September 14, 2010.
	\item Online Resources for Graduating Students: \url{http://www.acm.org/membership/student/resources-for-grads}
	\end{enumerate}
\item VLSI design and verification: \vspace{-0.3cm}
	\begin{enumerate} \itemsep -2pt
	\item {\it DVClub} for individuals interested in VLSI verification: \url{http://www.dvclub.org/}
	\item {\it DeepChip.com}: \url{http://www.deepchip.com}
	\end{enumerate}
%%%%%%%%%%%%%%%%%%%%%%%%%%%%%%%
\item undergraduates: \vspace{-0.3cm}
	\begin{enumerate} \itemsep -2pt
	\item {\it Humanitarian FOSS Project}: \vspace{-0.2cm}
		\begin{enumerate} \itemsep -2pt
		\item Where FOSS refers to Free and Open Source Software
		\item For computer science and engineering students
		\item \url{http://www.hfoss.org/}
		\end{enumerate}
	\item {\it SIGDA Design Automation Summer School}: \vspace{-0.2cm}
		\begin{enumerate} \itemsep -2pt
		\item {\it NSF�SRC�SIGDA�DAC Design Automation Summer School}
		\item \url{http://www.sigda.org/dass.html}
		\item Travel grants are provided to defray travel and accommodation expenses
		\end{enumerate}
	\item {\it Young Student Support Program at DAC}: \vspace{-0.2cm}
		\begin{enumerate} \itemsep -2pt
		\item Also known as {\it DAC Young Student Support Program}
		\item \url{http://www.sigda.org/youngstudent.html}
		\item Travel grants are provided to defray travel and accommodation expenses
		\end{enumerate}
	\item {\it ACM Student Research Competition at Design Automation Conference}: \vspace{-0.2cm}
		\begin{enumerate} \itemsep -2pt
		\item Sponsored by {\it Microsoft Research}
		\item \url{http://www.sigda.org/studentcomp.html}
		\item Also, see {\it ACM Student Research Competition} @ \url{http://src.acm.org/}.
		\end{enumerate}
	\item Job database for positions in the Video Game, Animation, VFX, and Software/Technology industries: \url{http://www.creativeheads.net/}
	\end{enumerate}
%%%%%%%%%%%%%%%%%%%%%%%%%%%%%%
\item graduate students: \vspace{-0.3cm}
	\begin{enumerate} \itemsep -2pt
	\item {\it SIGDA Design Automation Summer School}: \vspace{-0.2cm}
		\begin{enumerate} \itemsep -2pt
		\item {\it NSF�SRC�SIGDA�DAC Design Automation Summer School}
		\item \url{http://www.sigda.org/dass.html}
		\item Travel grants are provided to defray travel and accommodation expenses
		\end{enumerate}
	\item {\it Young Student Support Program at DAC}: \vspace{-0.2cm}
		\begin{enumerate} \itemsep -2pt
		\item Also known as {\it DAC Young Student Support Program}
		\item \url{http://www.sigda.org/youngstudent.html}
		\item Travel grants are provided to defray travel and accommodation expenses
		\end{enumerate}
	\item {\it ACM Student Research Competition at Design Automation Conference}: \vspace{-0.2cm}
		\begin{enumerate} \itemsep -2pt
		\item Sponsored by {\it Microsoft Research}
		\item \url{http://www.sigda.org/studentcomp.html}
		\item Also, see {\it ACM Student Research Competition} @ \url{http://src.acm.org/}.
		\end{enumerate}
	\item {\it SIGDA University Booth at DAC}: \vspace{-0.2cm}
		\begin{enumerate} \itemsep -2pt
		\item Or, {\it SIGDA/DAC University Booth}
		\item \url{http://www.sigda.org/ubooth.html}
		\end{enumerate}
	\item {\it SIGDA Ph.D. Forum at DAC}: \vspace{-0.2cm}
		\begin{enumerate} \itemsep -2pt
		\item \url{http://www.sigda.org/phdforum.html}
		\item \url{http://www.sigda.org/daforum/}
		\end{enumerate}
	\item {\it DAC Graduate Scholarship}: \vspace{-0.2cm}
		\begin{enumerate} \itemsep -2pt
		\item {\it A. Richard Newton Graduate Scholarships} to Support Graduate Research and Study
		\item \url{http://www.sigda.org/gradscholarship.html}
		\end{enumerate}
	\end{enumerate}
%%%%%%%%%%%%%%%%%%%%%%%%%%%%%%
\item competitions, and programming contests and challenges: \vspace{-0.3cm}
	\begin{itemize} \itemsep -2pt
	\item {\it SIGDA CADathlon at ICCAD}: \vspace{-0.2cm}
		\begin{enumerate} \itemsep -2pt
		\item \url{http://www.sigda.org/programs/cadathlon/}
		\item \url{http://www.sigda.org/cadathlon.html}
		\item Travel grants are provided to defray travel and accommodation expenses
		\end{enumerate}
	\item ISPD Programming Contest: \url{http://www.ispd.cc/contests/}
	\item ACM International Workshop on Timing Issues in the Specification and Synthesis of Digital Systems (TAU Workshop): \vspace{-0.2cm}
		\begin{enumerate} \itemsep -2pt
		\item Power Grid Simulation Contest: \url{http://www.tauworkshop.com/PREVIOUS/contest_2011.html}
		\end{enumerate}
	\item IEEE Computer Society Simulator Design competition: \url{http://www.computer.org/portal/web/competition}
	\item {\it DAC/ISSCC Student Design Contest}: \vspace{-0.2cm}
		\begin{enumerate} \itemsep -2pt
		\item \url{http://www.dac.com}
		\end{enumerate}
	\item {\it ACM/IEEE International Conference on Formal Methods and Models for Codesign -- Design Contest}: \vspace{-0.2cm}
		\begin{enumerate} \itemsep -2pt
		\item MEMOCODE Hardware/Software Co-Design Contest (MEMOCODE HW/SW co-design contest)
		\item \url{http://www-memocode2010.imag.fr/}
		\item \url{http://memocode2010.csail.mit.edu/redmine/wiki/memocode2010/Results}
		\end{enumerate}
	\item {\it International Low Power Design Contest}: \vspace{-0.2cm}
		\begin{enumerate} \itemsep -2pt
		\item ACM/IEEE International Symposium on Low Power Electronics and Design (ISLPED) -- Design Contest
		\item The International Symposium on Low Power Electronics and Design is holding the International Low Power Design Contest to provide a forum for universities and research organizations to showcase original ``power-aware'' designs and to highlight the innovations and design choices targeted at low power.
		\item The goal is to encourage and highlight design-oriented approaches to power reduction.
		\item \url{http://www.islped.org/2010/index.html}
		\end{enumerate}
	\item {\it University LSI Design Contest @ ASP-DAC}: \vspace{-0.2cm}
		\begin{enumerate} \itemsep -2pt
		\item Application areas or types of circuits of the original LSI circuit designs include (but are not limited to): \vspace{-0.1cm}
			\begin{enumerate} \itemsep -1pt
			\item Analog, RF and Mixed-Signal Circuits
			\item Digital Signal Processing
			\item Microprocessors
			\item Custom ASIC
			\end{enumerate} 
		\item Methods or technology used for implementation include: \vspace{-0.1cm}
			\begin{enumerate} \itemsep -1pt
			\item Full Custom and Cell-Based LSIs
			\item Gate Arrays
			\item FPGA/PLDs.
			\end{enumerate}
		\item \url{http://www.aspdac.com/aspdac2011/cfd/}
		\end{enumerate}
	\item IEEE Programming Challenge at IWLS: \url{http://www.iwls.org/challenge/}
	\item IEEE Asian Solid-State Circuits Conference (A-SSCC) Student Design Contest: \url{http://a-sscc2010.a-sscc.org/contest.html}
	\item {\it VLSI Conference 2011 - Design Contest}: \vspace{-0.2cm}
		\begin{enumerate} \itemsep -2pt
		\item Design/project fields include (but not limited to): \vspace{-0.1cm}
			\begin{enumerate} \itemsep -1pt
			\item Digital Integrated Circuits
			\item Analog Integrated Circuits
			\item FPGA based designs
			\item Computer Architectures/ Processors
			\item Reconfigurable Computing Systems
			\item SoC / Platform-based designs
			\item Embedded Systems
			\item MEMS/Optics/Bio-Chips
			\item Innovative Design Methodologies and Verification Techniques.
			\end{enumerate}
		\item \url{http://vlsiconference.com/vlsi2011/submissions_design_contest.html}
		\end{enumerate}
	\item {\it Satisfiability Modulo Theories Competition} (SMT-COMP): \vspace{-0.2cm}
		\begin{enumerate} \itemsep -2pt
		\item Competition for SMT solvers
		\item \url{http://www.smtcomp.org/2010/}
		\end{enumerate}
	\item {\it SAT Competition 201X}, where $X > 0$ \& $X {\it mod} 2 = 1$: \vspace{-0.2cm}
		\begin{enumerate} \itemsep -2pt
		\item The purpose of the competition is to identify new challenging benchmarks and to promote new solvers for the propositional satisfiability problem (SAT) as well as to compare them with state-of-the-art solvers.
		\item \url{http://www.satcompetition.org/}
		\end{enumerate}
	\item {\it SAT-Race 201X}, where $X > 0$ \& $X {\it mod} 2 = 0$: \vspace{-0.2cm}
		\begin{enumerate} \itemsep -2pt
		\item SAT-Race 201X is a competitive event for solvers of the Boolean Satisfiability (SAT) problem. 
		\item In contrast to the SAT Competitions, the focus of SAT-Race is on application benchmarks only.
		\item \url{http://baldur.iti.uka.de/sat-race-2010/}
		\end{enumerate}
	\item Hardware Model Checking Competition (HWMCC): \url{http://fmv.jku.at/hwmcc10/}
	\item {\it CADE ATP System Competition} (CASC): \vspace{-0.2cm}
		\begin{enumerate} \itemsep -2pt
		\item It is a yearly competition of fully automated theorem provers for classical first order logic.
		\item \url{http://www.cs.miami.edu/~tptp/CASC/}
		\end{enumerate}
	\item Apple Design Awards: \url{http://developer.apple.com/wwdc/ada/index.html}
	\item {\it International Constraint Solver Competition}: \vspace{-0.2cm}
		\begin{enumerate} \itemsep -2pt
		\item Also known as: \vspace{-0.2cm}
			\begin{enumerate} \itemsep -2pt
			\item International Constraint Solver Competition (CSP, Max-CSP and Weighted-CSP competition)
			\item International CSP Solver Competition (CSP, Max-CSP and Weighted-CSP competition)
			\end{enumerate}
		\item The Fourth International Constraint Solver Competition (CSC'2009) is organized to improve our knowledge of what is behind the efficiency of constraint satisfaction algorithms, heuristics, solving strategies, and constraint systems.
		\item \url{http://cpai.ucc.ie/}
		\end{enumerate}
	\item International Conference on Field-Programmable Technology (FPT 201X): \vspace{-0.2cm}
		\begin{enumerate} \itemsep -2pt
		\item FPT Design Competition: \url{http://cas.ee.ic.ac.uk/people/as999/FPTDesignComp/}
		\end{enumerate}
	\item International Microwave Symposium: Student Design Competitions -- Jan (includes AMS circuit simulation, and AMS/RF EDA); \url{http://ims2011.org/Technical_Program/Student_Design_Competitions.html}
	\item {\it QBFEVAL'1X}: \vspace{-0.2cm}
		\begin{enumerate} \itemsep -2pt
		\item QBF Solver competition for solvers to determine Quantified Boolean Formula (QBF) satisfiability.
		\item QBFLIB is a collection of instances, solvers, and tools related to Quantified Boolean Formula (QBF) satisfiability. See \url{http://www.qbflib.org/}.
		\item \url{http://www.qbflib.org/index_eval.php}
		\end{enumerate}
	\item {\it Pseudo-Boolean Competition 201X}: \vspace{-0.2cm}
		\begin{enumerate} \itemsep -2pt
		\item Competition for pseudo-Boolean solvers.
		\item \url{http://www.cril.univ-artois.fr/PB10/}
		\end{enumerate}
	\item {\it Answer Set Programming System Competition}: \vspace{-0.2cm}
		\begin{enumerate} \itemsep -2pt
		\item \url{http://dtai.cs.kuleuven.be/events/ASP-competition/}
		\end{enumerate}
	\item {\it Max-SAT Evaluation, Max-SAT 201X}: \vspace{-0.2cm}
		\begin{enumerate} \itemsep -2pt
		\item Competition for Max-SAT solvers
		\item \url{http://www.maxsat.udl.cat/}
		\item \url{http://www.maxsat.udl.cat/09/}
		\end{enumerate}
	\item {\it IEEEXtreme 24 Hour Programming Challenge}: \vspace{-0.2cm}
		\begin{enumerate} \itemsep -2pt
		\item Programming contest for college students
		\item \url{http://portal.ieee.org/web/membership/students/scholarshipsawardscontests/ieeextreme.html}
		\end{enumerate}
	\item {\it ACM International Collegiate Programming Contest} (ACM-ICPC or ICPC): \vspace{-0.2cm}
		\begin{enumerate} \itemsep -2pt
		\item Programming contest for college students
		\item Official web page: \url{http://cm.baylor.edu/welcome.icpc}
		\item Other web resources: \vspace{-0.1cm}
			\begin{enumerate} \itemsep -1pt
			\item {\it Wikipedia}: \url{http://en.wikipedia.org/wiki/ACM_International_Collegiate_Programming_Contest}
			\item {\it }: \url{}
			\item {\it }: \url{}
			\item {\it Valladolid Online Judge Site}: \url{http://acm.uva.es/}
			\item {\it ACMSolver :: Art of Programming Contest, Tips and Tricks for C, C++, Java}: \url{http://www.acmsolver.org/}
			\end{enumerate}
		\item 
		\end{enumerate}
	\item {\it TopCoder} coding and design contests: \vspace{-0.2cm}
		\begin{enumerate} \itemsep -2pt
		\item The contests cover various fields, such as: \vspace{-0.1cm}
			\begin{enumerate} \itemsep -1pt
			\item Algorithm
			\item Conceptualization
			\item Specification
			\item Architecture
			\item Component Design
			\item Component Development
			\item Assembly
			\item Test Scenarios
			\item Test Suites
			\item UI Prototype
			\item Rich Internet Application (RIA) Build
			\item Bug Race
			\item Marathon Match
			\item High School (for high school students)
			\item Copilot Opportunities
			\end{enumerate}
		\item \url{http://www.topcoder.com/}
		\end{enumerate}
	\item IEEE Presidents' Change the World competition: \vspace{-0.2cm}
		\begin{enumerate} \itemsep -2pt
		\item The IEEE Presidents� Change the World Competition recognizes students who develop unique solutions to real-world problems using engineering, science, computing and leadership skills to benefit their community, the world at large, or both. 
		\item \url{http://www.ieeechangetheworld.org/}
		\end{enumerate}
	\item Google Code Jam (programming contest): \url{http://code.google.com/codejam/} and \url{http://en.wikipedia.org/wiki/Google_Code_Jam}
	\item {\it RoboCup}\texttrademark\ competitions: \vspace{-0.2cm}
		\begin{enumerate} \itemsep -2pt
		\item Has different categories, including soccer, rescue operations, and home applications.
		\item \url{http://www.robocup.org/}
		\end{enumerate}
	\item ICFP Programming Contest (ICFP refers to International Conference on Functional Programming): \url{http://icfpcontest.org/}
	\item Student Cluster Competition (SCC): \vspace{-0.2cm}
		\begin{enumerate} \itemsep -2pt
		\item SCC is held at each (annual) SC conference, which is the International Conference for High Performance Computing, Networking, Storage, and Analysis. IEEE Computer Society and the Association for Computing Machinery are the sponsors for this conference.
		\item During SC10, teams consisting of six students, undergraduate and/or high school, will showcase the amazing power of clusters and the ability to utilize open source software to solve interesting and important problems. They will compete in real-time on the exhibit floor to run a workload of real-world applications on clusters of their own design while never exceeding the dictated power limit.
		\item During SC10 in New Orleans, teams will assemble, test and tune their machines and run the HPCC benchmarks until the starting bell rings on Monday night at the Exhibit Opening Gala where they will be given the competition data sets. In full view of conference attendees, teams will execute the prescribed workload while showing progress and science visualization output on large high-resolution displays in their areas. Teams race to correctly complete the greatest number of application runs during the competition period until the close of the exhibit floor on Wednesday evening.
		\item \url{http://sc10.supercomputing.org/?pg=studentcluster.html}
		\end{enumerate}
	\item Cypress Semiconductor Corporation: \vspace{-0.2cm}
		\begin{enumerate} \itemsep -2pt
		\item ARM Cortex-M3 PSoC\textregistered\ 5 Design Challenge: \url{http://www.cypress.com/?id=3271}
		\end{enumerate}
	\item Mentor Graphics: \vspace{-0.2cm}
		\begin{enumerate} \itemsep -2pt
		\item PCB Technology Leadership Awards (PCB design contest): \url{http://www.mentor.com/products/pcb-system-design/tla/index.cfm?v=mentorgraphics&p=handout:tla&a=print_card&g=sdd&s=1x1&c=ocid_2203&cmpid=3911}, or \url{http://www.mentor.com/go/tla}
		\end{enumerate}
	\item INFORMS Data Mining Contest: \vspace{-0.2cm}
		\begin{enumerate} \itemsep -2pt
		\item \url{http://ifors.org/web/call-for-participation-informs-data-mining-contest-2010/}
		\item \url{http://kaggle.com/informs2010}
		\end{enumerate}
	\item INFORMS Doing Good with Good OR - Student Competition: \vspace{-0.2cm}
		\begin{enumerate} \itemsep -2pt
		\item Doing Good with Good OR-Student Competition is held each year to identify and honor outstanding projects in the field of operations research and the management sciences conducted by a student or student group that have a significant societal impact.
		\item \url{http://www.informs.org/Recognize-Excellence/INFORMS-Prizes-Awards/Doing-Good-with-Good-OR}
		\end{enumerate}
	\item HPC Challenge Award Competition: \url{http://www.hpcchallenge.org/}
	\item Sphere Online Judge, SPOJ (programming contest): \url{http://www.spoj.pl/}
	\item High Performance and Scientific Computing Contest (Argonne National Laboratory, U.S. Department of Energy, DOE): \url{https://wiki.alcf.anl.gov/index.php/HPSC_Contest_Information}
	\item Argonne National Laboratory, ANL; Mathematics and Computer Science Division: \vspace{-0.2cm}
		\begin{enumerate} \itemsep -2pt
		\item J. H. Wilkinson Prize for Numerical Software (for developers of numerical software): \url{http://www.mcs.anl.gov/research/opportunities/wilkinsonprize/index.php}
		\end{enumerate}
	\item Society for Industrial and Applied Mathematics, SIAM: \vspace{-0.2cm}
		\begin{enumerate} \itemsep -2pt
		\item SIAM/ACM Prize in Computational Science and Engineering: \url{http://www.siam.org/prizes/sponsored/cse.php}. [ For developers of mathematical and computational tools and methods for the solution of science and engineering. Or, for developers of computational science and engineering software. ]
		\end{enumerate}
	\end{itemize}
	\item Sun HPC Software Programming Challenge (Oracle Corporation): \url{http://wikis.sun.com/display/HPCContest/Home}
%%%%%%%%%%%%%%%%%%%%%%%%%%%%%%
\item News media: \vspace{-0.3cm}
	\begin{itemize} \itemsep -2pt
	\item --- --- --- --- --- --- --- --- --- --- --- --- --- --- --- --- --- --- --- --- --- --- --- --- --- --- --- --- --- --- ---
	\item \colorbox{blue}{\bf News media for Electronic Design Automation}
	% News media for Electronic Design Automation
	\item {\it EDACafe}: \url{http://www.edacafe.com/}
	\item {\it SIGDA E-Newsletter} (SIGDA Electronic Newsletter): \url{http://www.sigda.org/newsletter/}
	\item {\it DeepChip.com}: \url{http://www.deepchip.com}
	\item --- --- --- --- --- --- --- --- --- --- --- --- --- --- --- --- --- --- --- --- --- --- --- --- --- --- --- --- --- --- ---
	\item \colorbox{blue}{\bf News media for Electrical and Computer Engineering}
	% News media for Electrical and Computer Engineering
	\item {\it EE Times} (Electronic Engineering Times): \url{http://www.eetimes.com/}
	\item {\it EDN} (Electrical Design News): \url{http://www.edn.com/}
	\item {\it IEEE Spectrum}: \url{http://spectrum.ieee.org/}
	\item {\it The Institute} (from IEEE): \url{http://www.theinstitute.ieee.org}
	\item {\it IEEE-USA Today's Engineer}: \url{http://www.todaysengineer.org/}
	\item {\it DeepChip.com}: \url{http://www.deepchip.com}
	\item --- --- --- --- --- --- --- --- --- --- --- --- --- --- --- --- --- --- --- --- --- --- --- --- --- --- --- --- --- --- ---
	\item \colorbox{blue}{\bf News media for Computer Science and Engineering, Information Systems, and IT}
	% News media for Computer Science and Engineering, Information Systems, and IT
	\item {\it ACM TechNews}: \url{http://technews.acm.org/}
	\item {\it TechCareers}: \url{http://www.techcareers.com/}
	\item {\it }: \url{}
	\item {\it }: \url{}
	\item {\it }: \url{}
	\item {\it }: \url{}
	\item {\it }: \url{}
	\item {\it }: \url{}
	\item {\it }: \url{}
	\item --- --- --- --- --- --- --- --- --- --- --- --- --- --- --- --- --- --- --- --- --- --- --- --- --- --- --- --- --- --- ---
	\item \colorbox{blue}{\bf Other News Media}
	% Other News Media
	\item {\it iTunes U}
	\item {\it YouTube EDU}
	\end{itemize}
%%%%%%%%%%%%%%%%%%%%%%%%%%%%%%
\item underrepresented minorities: \vspace{-0.3cm}
	\begin{enumerate} \itemsep -2pt
	\item women: \vspace{-0.2cm}
		\begin{enumerate} \itemsep -2pt
		\item IEEE Women in Engineering (WIE): \url{http://www.ieee.org/membership_services/membership/women/index.html?WT.mc_id=WIE_nav1}
		\item ACM-W: \url{http://women.acm.org/}
		\item Computer Research Association's Committee on the Status of Women in Computing Research (CRA-W): \vspace{-0.1cm}
			\begin{enumerate} \itemsep -1pt
			\item \url{http://www.cra-w.org/}
			\item Computing Research Association's Committee on the Status of Women (CRA-W) and the Coalition to Diversify Computing (CDC), {\it CompArch Summer School on Parallel Programming and Architectures}. Available at: \url{http://www.princeton.edu/~archss/}; last accessed on September 3, 2010.
			\end{enumerate}
		\item National Center for Women \& Information Technology: \url{http://www.ncwit.org/}
		\item African-American Women in Technology organization (AAWIT): \url{http://www.aawit.net/09/index.cfm}
		\item Grace Hopper Celebration of Women in Computing (conference for female IT students, professors, and professionals): \url{http://gracehopper.org/} or \url{http://gracehopper.org/2010/}
		\item Anita Borg Institute for Women and Technology: \vspace{-0.1cm}
			\begin{enumerate} \itemsep -1pt
			\item Has many programs for female students and professionals: \url{http://anitaborg.org/}
			\end{enumerate}
		\end{enumerate}
	\end{enumerate}
\end{enumerate}







%%%%%%%%%%%%%%%%%%%%%%%%%%%%%%%%%%%%%%%%%%%
\section{Scholarships, Fellowships, Awards, and Financial Aid}
\label{scholarshipsfinaidawards}

Resources for scholarships, fellowships, and financial aid: \vspace{-0.3cm}
\begin{enumerate} \itemsep -4pt
\item --- --- --- --- --- --- --- --- --- --- --- --- --- --- --- --- --- --- --- --- --- --- --- --- --- --- --- --- --- --- ---
\item \colorbox{blue}{\bf Lists of Scholarships and Fellowships}
% Lists of Scholarships and Fellowships
\item List of scholarships: \vspace{-0.3cm}
	\begin{enumerate} \itemsep -2pt
	\item Engineering Education Service Center, EESC (Engineering): \url{http://www.engineeringedu.com/scholars.html}
	\item High Performance and Embedded Architecture and Compilation, HiPEAC (Computer Science and Engineering): \url{http://www.hipeac.net/all_jobs_op}
	\item Office of Doctoral Programs at USC Viterbi School of Engineering, {\bf University of Southern California}. External Fellowships and other support: \url{http://viterbi.usc.edu/students/phd/fellowships-and-other-support/external-fellowships.htm}. USC Fellowships: \url{http://viterbi.usc.edu/students/phd/fellowships-and-other-support/usc-fellowships.htm}
	\item Columbia College, {\bf Columbia University} in the City of New York: \url{http://www.college.columbia.edu/students/fellowships/catalog}
	\item {\bf New York University} School of Law: \url{http://www.law.nyu.edu/financialaid/supplementalaid/fellowships/index.htm}
	\item Swedish Institute: \vspace{-0.2cm}
		\begin{enumerate} \itemsep -2pt
		\item The Swedish Institute, a government agency, administers over 500 scholarships each year for students and researchers coming to Sweden to pursue their objectives at a Swedish university.
		\item Study in Sweden: scholarships, \url{http://www.studyinsweden.se/Scholarships/}
		\item Swedish Institute (SI): \url{http://www.si.se/English/Navigation/Scholarships-and-exchanges/} [ Has special programs for Pakistanis and Turkish citizens ]
		\end{enumerate}
	\item The Swedish Foundation for International Cooperation in Research and Higher Education (STINT): \vspace{-0.2cm}
		\begin{enumerate} \itemsep -2pt
		\item \url{http://www.stint.se/en}
		\item Scholarships and grants: \url{http://www.stint.se/en/scholarships_and_grants}
		\end{enumerate}
	\item Center for the Advancement of Hispanics in Science and Engineering Education (CAHSEE): \url{http://www.cahsee.org/6resources/scholarships.asp.htm}
	\item University of Wisconsin-Madison: \vspace{-0.2cm}
		\begin{enumerate} \itemsep -2pt
		\item Grants Information Collection: A Cooperating Collection of the Foundation Center Library Network, \url{http://grants.library.wisc.edu/}
		\end{enumerate}
	\item {\it Find A PhD}: \url{http://www.findaphd.com/}
	\item QS World Grad School Tour Scholarships (QS Quacquarelli Symonds Limited): \url{http://graduateschool.topuniversities.com/world-grad-school-tour/scholarships}
	\item GlobalGrant (requires paid access to the list of scholarships and fellowships): \url{http://www.globalgrant.com/en/stipendier.html} and \url{http://www.globalgrant.com/}
	\item Stockholm University: \vspace{-0.2cm}
		\begin{enumerate} \itemsep -2pt
		\item \url{http://www.su.se/pub/jsp/polopoly.jsp?d=777&a=1770}
		\item \url{http://www.su.se/pub/jsp/polopoly.jsp?d=797}
		\item \url{http://www.su.se/pub/jsp/polopoly.jsp?d=788}
		\item \url{http://www.su.se/pub/jsp/polopoly.jsp?d=777&a=1769}
		\end{enumerate}
	\item NordForsk (in Norwegian): \url{http://www.nordforsk.org/index.cfm}
	\item Wallenberg Scholars (in Swedish): \url{http://www.wallenberg.com/default.aspx} or \url{http://www.wallenberg.com/in-english.aspx}
	\item Royal Institute of Technology (in Swedish): \url{http://www.kth.se/aktuellt/stipendier/stipendier-och-anslag-1.2024}
	\item European Commission: \vspace{-0.2cm}
		\begin{enumerate} \itemsep -2pt
		\item Marie Curie Fellowships: \vspace{-0.1cm}
			\begin{enumerate} \itemsep -1pt
			\item \url{http://cordis.europa.eu/fp7/people/home_en.html}
			\item \url{http://ec.europa.eu/research/mariecurieactions/}
			\item \url{http://ec.europa.eu/research/fp6/mariecurie-actions/action/fellow_en.html}
			\item \url{http://www.mariecurie.org/}
			\end{enumerate}
		\item Euraxess: \url{http://ec.europa.eu/euraxess/}
		\item \url{http://ec.europa.eu/index_en.htm}
		\end{enumerate}
	\item Science Please (for research positions in life sciences in The Netherlands and Belgium, including Ph.D. and postdoc positions): \url{http://www.scienceplease.com/} or \url{http://www.scienceplease.com/about-us}
	\item University of Gothenburg: \vspace{-0.2cm}
		\begin{enumerate} \itemsep -2pt
		\item ResearchResearch: \url{http://www.researchresearch.com/} or \url{http://www.gu.se/english/research/scholarships/ResearchResearch/}
		\item Scholarship links: \url{http://www.gu.se/english/research/scholarships/scholarship_links/}
		\item Scholarships at University of Gothenburg: \url{http://www.gu.se/english/research/scholarships/gu/}
		\end{enumerate}
	\item Princeton University; The Graduate School: \url{http://gradschool.princeton.edu/financial/}
	\item National Association for Bilingual Education: \vspace{-0.2cm}
		\begin{enumerate} \itemsep -2pt
		\item List of Scholarships: \url{http://www.nabe.org/scholarship.html}
		\end{enumerate}
	\item {\bf Pennsylvania State University}: \vspace{-0.2cm}
		\begin{enumerate} \itemsep -2pt
		\item Office of Engineering Diversity; Penn State College of Engineering: \vspace{-0.1cm}
			\begin{enumerate} \itemsep -1pt
			\item Undergraduate Student Scholarships: \url{http://www.engr.psu.edu/oed/UnderScholarships.html}
			\item Graduate Student Scholarships: \url{http://www.engr.psu.edu/oed/GradScholarships.html}
			\item High School Student Scholarships: \url{http://www.engr.psu.edu/oed/HighSchoolScholarships.html}
			\item Disabled Student Scholarships: \url{http://www.engr.psu.edu/oed/DisabScholarships.html}
			\item Corporate Office of Engineering Diversity (OED) Scholarships: \url{http://www.engr.psu.edu/oed/OEDScholarships.html}
			\end{enumerate}
		\item University Fellowships Office: \vspace{-0.1cm}
			\begin{enumerate} \itemsep -1pt
			\item \url{http://sites.google.com/site/psuufo/}
			\item Prestigious Scholarships: \url{http://sites.google.com/site/psuufo/prestigious}
			\item Penn State Scholarships: \url{http://sites.google.com/site/psuufo/internal-scholarships}
			\item Other resources: \url{http://sites.google.com/site/psuufo/resources}
			\end{enumerate}
		\end{enumerate}
	\item {\bf Peterson's} college search: \vspace{-0.2cm}
		\begin{enumerate} \itemsep -2pt
		\item {\it College Scholarship Search}: \url{http://www.petersons.com/college-search/scholarship-search.aspx}
		\end{enumerate}
	\item Society for Industrial and Applied Mathematics (SIAM): \vspace{-0.2cm}
		\begin{enumerate} \itemsep -2pt
		\item Fellowship \& Research Opportunities: \url{http://www.siam.org/students/resources/fellowship.php}
		\end{enumerate}
	\item Institute of International Education (IIE): \vspace{-0.2cm}
		\begin{enumerate} \itemsep -2pt
		\item {\it Funding for US Study Online}: \vspace{-0.1cm}
			\begin{enumerate} \itemsep -1pt
			\item \url{http://www.fundingusstudy.org/}
			\end{enumerate}
		\end{enumerate}
	\end{enumerate}
\item --- --- --- --- --- --- --- --- --- --- --- --- --- --- --- --- --- --- --- --- --- --- --- --- --- --- --- --- --- --- ---
\item \colorbox{blue}{\bf Scholarships and Fellowships in Electrical and Computer Engineering}
% Scholarships and Fellowships in Electrical and Computer Engineering
\item IEEE: \vspace{-0.3cm}
	\begin{enumerate} \itemsep -2pt
	\item IEEE Awards, Competitions, and Scholarships: \url{http://www.ieee.org/membership_services/membership/students/awards/index.html}
	\item IEEE Circuits and Systems Society Pre-Doctoral Scholarships: Announced via email from IEEE Circuits and Systems Society
	\item IEEE Power \& Energy Society: \vspace{-0.2cm}
		\begin{enumerate} \itemsep -2pt
		\item G. Ray Ekenstam Memorial Scholarship: \vspace{-0.1cm}
			\begin{enumerate} \itemsep -1pt
			\item \url{http://www.ieee-pes.org/g-ray-ekenstam-memorial-scholarship}
			\item ``The Scholarship Fund awards, on an annual basis, a scholarship to a qualified undergraduate student who seeks an electrical engineering degree in the field of power or a related discipline, from an accredited US university or college.''
			\end{enumerate}
		\end{enumerate}
	\item IEEE Reliability Society: \vspace{-0.2cm}
		\begin{enumerate} \itemsep -2pt
		\item IEEE Reliability Society Scholarship: \url{http://www.ieee.org/portal/cms_docs_relsoc/relsoc/newsflipper/RS_Scholarship_Application.pdf} [ Look under the tab/option on ``Useful Information'' in the panel on the left. ]
		\end{enumerate}
	\end{enumerate}
\item The George Michael Memorial HPC Fellowship Program: \vspace{-0.3cm}
	\begin{enumerate} \itemsep -2pt
	\item The Association of Computing Machinery (ACM), IEEE Computer Society and SC Conference series have established the High Performance Computing (HPC) Ph.D. Fellowship Program. The SC conference is the International Conference for High Performance Computing, Networking, Storage, and Analysis. IEEE Computer Society and the Association for Computing Machinery are the sponsors for this conference.
	\item Every year, up to three fellowship recipients will each receive a stipend of at least \$5,000 (U.S.) for one academic year, plus travel support to attend the SC conference.
	\item See \url{http://sc10.supercomputing.org/?searchterm=fellowship&pg=GeorgeMichaelMemorial.html}
	\end{enumerate}
\item Intel: \vspace{-0.3cm}
	\begin{enumerate} \itemsep -2pt
	\item Intel Foundation Fellowship: \vspace{-0.2cm}
		\begin{enumerate} \itemsep -2pt
		\item Intel Foundation Ph.D. Fellowship % \url{http://intelscholarships.intel.com/}
		\item \url{http://www.intel.com/education/highered/studentprograms/fellowship.htm}
		\item This awards two-year fellowships to Ph.D. candidates pursuing leading-edge work in fields related to Intel's business and research interests.
		\item Fellowships are available at select U.S. universities, by invitation only, and focus on Ph.D. students who have completed at least one year of study.
		\item The fellowship includes a cash award (tuition/fees/stipend), an Intel mentor, and the opportunity to participate in an internship at Intel.
		\end{enumerate}
	\end{enumerate}
\item IBM: \vspace{-0.3cm}
	\begin{enumerate} \itemsep -2pt
	\item \url{http://www-304.ibm.com/jct01005c/university/scholars/phdfellowship}
	\item IBM Ph.D. Fellowship Award
	\item The IBM Ph.D. Fellowship Awards is an intensely competitive program which honors exceptional Ph.D. students in many academic disciplines and areas of study, for example: computer science and engineering, electrical and mechanical engineering , physical sciences (including chemistry, material sciences, and physics), mathematical sciences (including optimization), business sciences (including financial services, communication, and learning/knowledge), and service sciences, management, and engineering.
	\item IBM Herman Goldstine Postdoctoral Fellowship in Mathematical Sciences: \url{http://domino.research.ibm.com/comm/research_projects.nsf/pages/goldstine.index.html}
	\item Josef Raviv Memorial Postdoctoral Fellowship; see \url{http://domino.research.ibm.com/comm/research.nsf/pages/d.compsci.josef.raviv.general.info.html}, \url{http://domino.research.ibm.com/comm/research.nsf/pages/d.compsci.raviv.winner.html}, and \url{http://domino.research.ibm.com/comm/research.nsf/pages/d.compsci.raviv.winner2008.html}
	\end{enumerate}
\item AMD: Ph.D. fellowship, \url{http://developer.amd.com/programs/fellowship/Pages/default.aspx}
\item Qualcomm, {\it Qualcomm Innovation Fellowship} for Ph.D. students in Electrical Engineering and Computer Science at Stanford, UC Berkeley, UCLA, UCSD, and USC: \url{http://www.qualcomm.com/innovation/research/university_relations/innovation_fellowship/qinf10.html}
\item NVIDIA: \vspace{-0.3cm}
	\begin{enumerate} \itemsep -2pt
	\item NVIDIA Fellowship Program; see \url{http://www.nvidia.com/page/fellowship_programs.html}
	\end{enumerate}
\item Automatic RF Techniques Group (ARFTG): \vspace{-0.3cm}
	\begin{enumerate} \itemsep -2pt
	\item Microwave Measurement Student Fellowship (for ``graduate students who show promise and interest in pursuing research related to improvement of radio frequency and microwave measurement techniques''): \url{http://www.arftg.org/student_fellowship.html}
	\end{enumerate}
\item Gallium Arsenide Applications Symposium (GAAS) Association: \vspace{-0.3cm}
	\begin{enumerate} \itemsep -2pt
	\item GAAS PhD Student Fellowship (for Ph.D. students who have accepted papers at the European Microwave Integrated Circuits Conference, EuMIC): \url{http://www.gaas-symposium.org/english/awards_fellowship.htm} and \url{http://www.eumweek.com/2010/EuMIC.asp?id=c}
	\end{enumerate}
\item The Institution of Engineering and Technology, IET: \vspace{-0.3cm}
	\begin{enumerate} \itemsep -2pt
	\item Hudswell International Research Scholarship: \url{http://www.theiet.org/about/scholarships-awards/ambition/postgraduate1/hudswell-what.cfm}
	\item IET Postgraduate Scholarship: \url{http://www.theiet.org/about/scholarships-awards/ambition/postgraduate1/postgrad-what.cfm}
	\end{enumerate}
\item --- --- --- --- --- --- --- --- --- --- --- --- --- --- --- --- --- --- --- --- --- --- --- --- --- --- --- --- --- --- ---
\item \colorbox{blue}{\bf Scholarships and Fellowships in Computer Science}
% Scholarships and Fellowships in Computer Science
\item ACM Special Interest Group on Symbolic and Algebraic Manipulation (SIGSAM): List of Ph.D. positions in computer algebra and symbolic computation, as listed by SIGSAM; see \url{http://www.sigsam.org/opportunities.phtml?searchterm=fellowship}
\item Carnegie Mellon University: \vspace{-0.3cm}
	\begin{enumerate} \itemsep -2pt
	\item women@SCS School of Computer Science: \vspace{-0.2cm}
		\begin{enumerate} \itemsep -2pt
		\item Individuals, Corporations \& Organizations: \url{http://women.cs.cmu.edu/Resources/Funding/}
		\end{enumerate}
	\end{enumerate}
\item IBM: \vspace{-0.3cm}
	\begin{enumerate} \itemsep -2pt
	\item \url{http://www-304.ibm.com/jct01005c/university/scholars/phdfellowship}
	\item IBM Ph.D. Fellowship Award
	\item The IBM Ph.D. Fellowship Awards is an intensely competitive program which honors exceptional Ph.D. students in many academic disciplines and areas of study, for example: computer science and engineering, electrical and mechanical engineering , physical sciences (including chemistry, material sciences, and physics), mathematical sciences (including optimization), business sciences (including financial services, communication, and learning/knowledge), and service sciences, management, and engineering.
	\item IBM Herman Goldstine Postdoctoral Fellowship in Mathematical Sciences: \url{http://domino.research.ibm.com/comm/research_projects.nsf/pages/goldstine.index.html}
	\item Josef Raviv Memorial Postdoctoral Fellowship; see \url{http://domino.research.ibm.com/comm/research.nsf/pages/d.compsci.josef.raviv.general.info.html}, \url{http://domino.research.ibm.com/comm/research.nsf/pages/d.compsci.raviv.winner.html}, and \url{http://domino.research.ibm.com/comm/research.nsf/pages/d.compsci.raviv.winner2008.html}
	\end{enumerate}
\item Computing Innovation Fellows (CIFellows); post my profile on \url{http://cifellows.org/profiles/}; also see \url{http://www.cifellows.org/}
\item Microsoft: \vspace{-0.3cm}
	\begin{enumerate} \itemsep -2pt
	\item Microsoft Research Graduate Women's Scholarship: \url{http://research.microsoft.com/en-us/collaboration/awards/fellows-women.aspx}
	\item Microsoft Research PhD Fellowship: \url{http://research.microsoft.com/en-us/collaboration/awards/apply-us.aspx}
	\end{enumerate}
\item Google: \vspace{-0.3cm}
	\begin{enumerate} \itemsep -2pt
	\item Google Fellowship Program; see \url{http://googleblog.blogspot.com/2009/05/best-and-brightest.html}
	\end{enumerate}
\item NVIDIA: \vspace{-0.3cm}
	\begin{enumerate} \itemsep -2pt
	\item NVIDIA Fellowship Program; see \url{http://www.nvidia.com/page/fellowship_programs.html}
	\end{enumerate}
\item Facebook Ph.D. Fellowship: \url{http://www.facebook.com/careers/fellowship.php}
\item Yahoo! Labs: Yahoo! Key Scientific Challenges Program, \url{http://labs.yahoo.com/ksc}
\item Qualcomm, {\it Qualcomm Innovation Fellowship} for Ph.D. students in Electrical Engineering and Computer Science at Stanford, UC Berkeley, UCLA, UCSD, and USC: \url{http://www.qualcomm.com/innovation/research/university_relations/innovation_fellowship/qinf10.html} and \url{http://www.qualcomm.com/innovation/research/university_relations/innovation_fellowship/}
\item Computing Research Association (CRA): Outstanding Undergraduate Researchers, \url{http://www.cra.org/awards/undergrad-current/}
\item {\color{blue} European Research Consortium for Informatics and Mathematics (ERCIM)}: \vspace{-0.3cm}
	\begin{enumerate} \itemsep -2pt
	\item ERCIM Alain Bensoussan Fellowship Programme (for Ph.D. degree holders in selected research areas): \url{http://fellowship.ercim.eu/} and \url{http://www.ercim.eu/news/283-fellowship-programme}; research areas are listed at: \url{http://fellowship.ercim.eu/home/topic}. Deadlines are on April 30 and September 30 annually.
	\end{enumerate}
\item {\it Theory Matters Wiki}; Theoretical Computer Science (TCS) Advocacy Wiki: \vspace{-0.3cm}
	\begin{enumerate} \itemsep -2pt
	\item Funding Opportunities and Tips: \url{http://theorymatters.org/pmwiki/pmwiki.php?n=Main.FundingOpportunities}
	\end{enumerate}
\item Kurt G{\"{o}}del Research Prize Fellowship: \vspace{-0.3cm}
	\begin{enumerate} \itemsep -2pt
	\item 2 Ph.D. (pre-doctoral) fellowships
	\item 2 post-doctoral fellowships
	\item 1 unrestricted fellowship
	\item $[$Scope of the$]$ original fellowship proposals [includes] the areas of: \vspace{-0.2cm}
		\begin{enumerate} \itemsep -2pt
		\item set theory
		\item recursion theory
		\item proof theory/intuitionism
		\item model theory
		\item computer assisted reasoning
		\item philosophy of mathematics 
		\end{enumerate}
	\item All fellowship proposals, regardless of subject area, will be judged according to: \vspace{-0.2cm}
		\begin{enumerate} \itemsep -2pt
		\item the relevance and resemblance of the research (finished and proposed) to the great insights and originality of Kurt G{\"{o}}del
		\item its general interest and clarity of motivation
		\item its rigorous scientific quality and depth. 
		\end{enumerate}
	\item \url{http://fellowship.logic.at/}
	\end{enumerate}
\item Hewlett-Packard Company: \vspace{-0.3cm}
	\begin{enumerate} \itemsep -2pt
	\item Hewlett-Packard Labs India (Bengaluru / Bangalore): \vspace{-0.2cm}
		\begin{enumerate} \itemsep -2pt
		\item {\it BITS - HP Labs India Ph.D. Fellowship} for Research related to Information Technologies: \vspace{-0.1cm}
			\begin{enumerate} \itemsep -1pt
			\item \url{http://www.hpl.hp.com/india/bits-hplindia_phd/index.html} or \url{http://www.hpl.hp.com/india/bits-hplindia_phd/}
			\item \url{http://www.hpl.hp.com/india/bits-hplindia_phd/iiitbphd.html}
			\item BITS, Pilani and HP Labs India jointly offer a unique PhD fellowship for research in Information and Communication Technologies (ICT) relevant to fast-growing markets like India.
			\item HP Labs India currently has ongoing Ph.D. Fellowships with BITS Pilani and IIIT, Bangalore: \url{http://www.hpl.hp.com/india/bits-hplindia_phd/university.html}
			\end{enumerate}
		\item Open Innovation Office: \vspace{-0.1cm}
			\begin{enumerate} \itemsep -1pt
			\item \url{http://www.hpl.hp.com/open_innovation/}
			\item HP Labs Innovation Research Program (IRP): \url{http://www.hpl.hp.com/open_innovation/irp/index.html}
			\end{enumerate}
		\end{enumerate}
	\end{enumerate}
\item Code for America (CfA): \vspace{-0.3cm}
	\begin{enumerate} \itemsep -2pt
	\item CfA Fellowship (develop web applications for local governments in the US): \url{http://codeforamerica.org/fellows/}
	\end{enumerate}
\item University of Minnesota, Twin Cities: \vspace{-0.3cm}
	\begin{enumerate} \itemsep -2pt
	\item College of Science and Engineering: \vspace{-0.2cm}
		\begin{enumerate} \itemsep -2pt
		\item Charles Babbage Institute: \vspace{-0.1cm}
			\begin{enumerate} \itemsep -1pt
			\item Adelle and Erwin Tomash Graduate Fellowship (for Ph.D. candidates doing research in the history of IT/computing - all but dissertation Ph.D. students only): \url{http://www.cbi.umn.edu/research/tfellowship.html}
			\item Arthur L. Norberg Travel Fund (short-term grants-in-aid to help scholars with travel expenses to use archival collections at the Charles Babbage Institute): \url{http://www.cbi.umn.edu/research/ntravelfund.html}
			\end{enumerate}
		\end{enumerate}
	\end{enumerate}
\item --- --- --- --- --- --- --- --- --- --- --- --- --- --- --- --- --- --- --- --- --- --- --- --- --- --- --- --- --- --- ---
\item \colorbox{blue}{\bf Scholarships and Fellowships in Biomedical Engineering}
% Scholarships and Fellowships in Biomedical Engineering
\item Whitaker International Fellows and Scholars Program: \vspace{-0.3cm}
	\begin{enumerate} \itemsep -2pt
	\item For graduate/Ph.D. students and postdocs in biomedical engineering
	\item \url{http://www.whitaker.org/home}
	\end{enumerate}
\item --- --- --- --- --- --- --- --- --- --- --- --- --- --- --- --- --- --- --- --- --- --- --- --- --- --- --- --- --- --- ---
\item \colorbox{blue}{\bf Scholarships and Fellowships in Optical Engineering}
% Scholarships and Fellowships in Optical Engineering
\item {\it SPIE} -- The International Society for Optical Engineering: \vspace{-0.3cm}
	\begin{enumerate} \itemsep -2pt
	\item ``SPIE Scholarship Program'' for undergraduates or graduate students studying optics, photonics, imaging, or optoelectronics program or related discipline (e.g., physics, electrical engineering): \url{http://spie.org//x1733.xml?WT.svl=mddm14}
	\item Other scholarships (including scholarships for students doing research in nanolithography techniques and lasers): \url{http://spie.org/x1736.xml}
	\end{enumerate}
\item {\it Kidger Optics Associates} Michael Kidger Memorial Scholarship (to a college freshman, or sophomore of optical design): \url{http://www.kidger.com/mkms_requirements.html}
\item --- --- --- --- --- --- --- --- --- --- --- --- --- --- --- --- --- --- --- --- --- --- --- --- --- --- --- --- --- --- ---
\item \colorbox{blue}{\bf Scholarships and Fellowships in Mechanical Engineering}
% Scholarships and Fellowships in Mechanical Engineering
\item American Society of Mechanical Engineers (ASME): \vspace{-0.3cm}
	\begin{enumerate} \itemsep -2pt
	\item Graduate Teaching Fellowships (for Ph.D. students in mechanical engineering): \url{http://www.asme.org/Education/College/FinancialAid/Graduate_Teaching_Fellowships.cfm}
	\item ASME Scholarships: \vspace{-0.2cm}
		\begin{enumerate} \itemsep -2pt
		\item \url{http://www.asme.org/Education/College/FinancialAid/Scholarships.cfm}
		\item US Undergraduates: \url{http://www.asme.org/Education/College/FinancialAid/US_Undergraduates.cfm}
		\item Graduate Students: \url{http://www.asme.org/Education/College/FinancialAid/Graduate_Students.cfm}
		\item International Students: \url{http://www.asme.org/Education/College/FinancialAid/International_Undergraduates.cfm}
		\end{enumerate}
	\item Auxiliary Scholarships: \vspace{-0.2cm}
		\begin{enumerate} \itemsep -2pt
		\item \url{http://www.asme.org/Education/College/FinancialAid/Auxiliary_Scholarships.cfm}
		\item Undergraduate Scholarships: \url{http://www.asme.org/Education/College/FinancialAid/Undergraduate_Scholarships.cfm}
		\item Graduate Scholarships: \url{http://www.asme.org/Education/College/FinancialAid/Graduate_Scholarships.cfm}
		\item Rice-Cullimore Scholarship (for international graduate students in the US): \url{http://www.asme.org/Education/College/FinancialAid/RiceCullimore_Scholarship.cfm}
		\end{enumerate}
	\item International Petroleum Institute�s College Scholarships (for undergraduates): \url{http://www.asme-ipti.org/public/pagscholarshipprograms.aspx}
	\item International Petroleum Institute�s Graduate Fellowship (for individuals entering a graduate program in mechanical engineering, and has an interest in the petroleum industry): \url{http://www.asme-ipti.org/public/pagscholarshipprograms.aspx} and \url{http://www.asme.org/Communities/Students/Grad/Fellowships.cfm}
	\end{enumerate}
\item --- --- --- --- --- --- --- --- --- --- --- --- --- --- --- --- --- --- --- --- --- --- --- --- --- --- --- --- --- --- ---
\item \colorbox{blue}{\bf Scholarships and Fellowships in Civil Engineering}
% Scholarships and Fellowships in Civil Engineering
\item American Society of Civil Engineers (ASCE): \vspace{-0.3cm}
	\begin{enumerate} \itemsep -2pt
	\item Jack E. Leisch Memorial National Graduate Fellowship (for graduate students in transportation/traffic engineering): \url{http://www.asce.org/Content.aspx?id=25021}
	\item Scholarships \& Fellowships (for undergraduates and graduate students): \url{http://www.asce.org/Content.aspx?id=18337}
	\end{enumerate}
\item American Concrete Institute (ACI): \vspace{-0.3cm}
	\begin{enumerate} \itemsep -2pt
	\item ACI Foundation Fellowships \& Scholarships: \url{http://www.concrete.org/STUDENTS/ST_SCHOLARSHIPS.HTM}
	\end{enumerate}
\item Institute of Transportation Engineers: \vspace{-0.3cm}
	\begin{enumerate} \itemsep -2pt
	\item Burton W. Marsh Fellowship for Graduate Study in Traffic and Transportation Engineering: \url{http://www.ite.org/education/Burton_W_MarshFellowship.asp}
	\end{enumerate}
\item --- --- --- --- --- --- --- --- --- --- --- --- --- --- --- --- --- --- --- --- --- --- --- --- --- --- --- --- --- --- ---
\item \colorbox{blue}{\bf Scholarships and Fellowships in Chemical Engineering}
% Scholarships and Fellowships in Chemical Engineering
\item American Institute of Chemical Engineers (AIChE) scholarships (includes scholarships for underrepresented minorities): \url{http://www.aiche.org/Students/Scholarships/index.aspx}
\item --- --- --- --- --- --- --- --- --- --- --- --- --- --- --- --- --- --- --- --- --- --- --- --- --- --- --- --- --- --- ---
\item \colorbox{blue}{\bf Scholarships and Fellowships in Aerospace Engineering}
% Scholarships and Fellowships in Aerospace Engineering
\item American Institute of Aeronautics and Astronautics (AIAA): \vspace{-0.3cm}
	\begin{enumerate} \itemsep -2pt
	\item AIAA Foundation Scholarships: \vspace{-0.2cm}
		\begin{enumerate} \itemsep -2pt
		\item \url{http://www.aiaa.org/content.cfm?pageid=211}
		\item For undergraduates and graduate students
		\item Named scholarships for undergraduates are: \vspace{-0.1cm}
			\begin{enumerate} \itemsep -1pt
			\item \url{http://www.aiaa.org/content.cfm?pageid=226}
			\item A. Thomas Young Scholarship
			\item L. S. ``Skip'' Fletcher Scholarship 
			\item Sam F. Iacobellis Scholarship
			\item Robert L. Crippen Scholarship
			\item E. C. ``Pete'' Aldridge Scholarship
			\item Liquid Propulsion Technical Committee Scholarship
			\item Space Transportation Technical Committee Scholarship
			\item Digital Avionics Technical Committee Scholarship (4)
			\item Next Century of Flight Scholarship (2)
			\item Leatrice Gregory Pendray Scholarship
			\end{enumerate}
		\item Awards for graduate students: \vspace{-0.1cm}
			\begin{enumerate} \itemsep -1pt
			\item \url{http://www.aiaa.org/content.cfm?pageid=227}
			\item Martin Summerfield Propellants and Combustion Graduate Award
			\item Guidance, Navigation, And Control Graduate Award
			\item Gordon C. Oates Air Breathing Propulsion Graduate Award
			\item William T. Piper, Sr. General Aviation Systems Graduate Award
			\item Orville and Wilbur Wright Graduate Award
			\item John Leland Atwood Graduate Award
			\item Open Topic Graduate Award
			\end{enumerate}
		\end{enumerate}
	\item Student Design Competition Award: \url{http://www.aiaa.org/content.cfm?pageid=401}
	\end{enumerate}
\item --- --- --- --- --- --- --- --- --- --- --- --- --- --- --- --- --- --- --- --- --- --- --- --- --- --- --- --- --- --- ---
\item \colorbox{blue}{\bf Scholarships and Fellowships in Mathematics}
% Scholarships and Fellowships in Mathematics
\item Association for Women in Mathematics (AWM): \vspace{-0.3cm}
	\begin{enumerate} \itemsep -2pt
	\item Travel grants: \url{http://sites.google.com/site/awmmath/programs/travel-grants}
	\item Alice T. Schafer Mathematics Prize for excellence in mathematics by an undergraduate woman: \url{http://sites.google.com/site/awmmath/programs/schafer-prize}
	\item The {\it Ruth I. Michler Memorial Prize} of the AWM is awarded annually to a woman recently promoted to Associate Professor or an equivalent position in the mathematical sciences: \url{http://sites.google.com/site/awmmath/programs/michler-prize}
	\end{enumerate}
\item Seth Bonder Scholarship for Applied Operations Research in Health Services: \url{http://www.informs.org/Recognize-Excellence/INFORMS-Community-Prizes-and-Awards/Seth-Bonder-Scholarship-for-Applied-Operations-Research-in-Health-Services}
\item Oberwolfach Foundation: \vspace{-0.3cm}
	\begin{enumerate} \itemsep -2pt
	\item Oberwolfach Prize (for young European mathematicians): \url{http://www.mfo.de/programme/prize/}
	\item John Todd Fellowship (or John Todd Award) [for young excellent mathematicians working in numerical analysis]: \url{http://www.mfo.de/programme/todd/}
	\end{enumerate}
\item Clay Mathematics Institute: Clay Research Award, \url{http://www.claymath.org/research_award/}
\item --- --- --- --- --- --- --- --- --- --- --- --- --- --- --- --- --- --- --- --- --- --- --- --- --- --- --- --- --- --- ---
\item \colorbox{blue}{\bf Scholarships and Fellowships in Science}
% Scholarships and Fellowships in Science
\item {\it Science.gov} (USA.gov for Science): \vspace{-0.3cm}
	\begin{enumerate} \itemsep -2pt
	\item Internship and Fellowship Opportunities in Science for Undergraduate Students: \url{http://www.science.gov/internships/undergrad.html}
	\item Graduate Students/Postdoctoral Fellowships: \url{http://www.science.gov/internships/graduate.html}
	\end{enumerate}
\item Heinz Family Philanthropies: \vspace{-0.3cm}
	\begin{enumerate} \itemsep -2pt
	\item Teresa Heinz Scholars for Environmental Research program (for Ph.D./MS students working on their thesis in environmental science/engineering) at selected universities: \url{http://www.heinzfamily.org/programs/environmentalscholars.html}
	\item \url{http://www.heinzfamily.org/}
	\end{enumerate}
\item Mayo Clinic: \vspace{-0.3cm}
	\begin{enumerate} \itemsep -2pt
	\item Postbaccalaureate Research Education Program (PREP): \url{http://www.mayo.edu/mgs/postbac-program.html}
	\end{enumerate}
\item {\it American Chemical Society (ACS)}: \vspace{-0.3cm}
	\begin{enumerate} \itemsep -2pt
	\item ACS-Hach Land Grant Undergraduate Scholarship (for chemistry undergraduates at a partner institution of ACS, and who plan to become chemistry teachers in US high schools): \url{http://portal.acs.org/portal/acs/corg/content?_nfpb=true&_pageLabel=PP_SUPERARTICLE&node_id=2243&use_sec=false&sec_url_var=region1&__uuid=eb054647-53e0-4594-81e8-8ef49159f3f4}
	\item ACS-Hach Second Career Teacher Scholarship (for graduates in chemistry or related areas who are entering an education masters program or teacher certification program): \url{http://portal.acs.org/portal/acs/corg/content?_nfpb=true&_pageLabel=PP_SUPERARTICLE&node_id=2244&use_sec=false&sec_url_var=region1&__uuid=4c27333f-4aad-481e-aaa4-f1db045d4eb4}
	\item ACS Scholars Program (for undergraduate underrepresented minorities majoring in chemistry, biochemistry, or chemical engineering): \url{http://portal.acs.org/portal/acs/corg/content?_nfpb=true&_pageLabel=PP_SUPERARTICLE&node_id=1650&use_sec=false&sec_url_var=region1&__uuid=b3b583cf-18ae-4fb0-9375-33f75a0ccf49}
	\item Scholarships: \url{http://portal.acs.org/portal/acs/corg/content?_nfpb=true&_pageLabel=PP_TRANSITIONMAIN&node_id=630&use_sec=false&sec_url_var=region1&__uuid=98e85c05-be75-4283-a97c-7a63ab4c3178}
	\end{enumerate}
\item European Molecular Biology Organization: \vspace{-0.3cm}
	\begin{enumerate} \itemsep -2pt
	\item EMBO Short-Term Fellowships (for junior researchers, including Ph.D. students): \url{http://www.embo.org/programmes/fellowships/short-term.html}
	\item EMBO Long-Term Fellowships (for junior researchers/postdocs): \url{http://www.embo.org/programmes/fellowships/long-term.html}
	\end{enumerate}
\item L'OR{\'{E}}AL: \vspace{-0.3cm}
	\begin{enumerate} \itemsep -2pt
	\item ``For Women in Science'' program: \url{http://www.lorealusa.com/forwomeninscience} or \url{http://www.lorealusa.com/_en/_us/index.aspx?direct1=00008&direct2=00008/00001}
	\item Alternatively, go to \url{http://www.lorealusa.com/_en/_us/} and select the ``For Women in Science'' tab.
	\item Check out the ``L'Or{\'{e}}al USA Fellowships for Women in Science'' (US postdocs), ``UNESCO-L'Or{\'{e}}al Fellowships for Women in Science'' (for female Ph.D. students and postdocs in the life sciences), and the ``L'Or{\'{e}}al-UNESCO Awardss for Women in Science'' (for distinguished female scientists)
	\end{enumerate}
\item American Institute of Physics (AIP): \vspace{-0.3cm}
	\begin{enumerate} \itemsep -2pt
	\item AIP and Member Society Government Science Fellowships: \vspace{-0.2cm}
		\begin{enumerate} \itemsep -2pt
		\item \url{http://www.aip.org/gov/fellowships.html}
		\item American Institute of Physics State Department Science Fellowship: \url{http://www.aip.org/gov/fellowships/sdf.html}
		\item American Institute of Physics Congressional Science Fellowship: \url{http://www.aip.org/gov/fellowships/cf.html}
		\item American Physical Society Congressional Science Fellowship: \url{http://www.aps.org/policy/fellowships/congressional.cfm}
		\item American Geophysical Union Congressional Science Fellowship: \url{http://www.agu.org/sci_pol/cong_fellowship/}
		\item Optical Society of America Congressional Science Fellowships: \url{http://www.osa.org/about_osa/public_policy/congressional_fellowships/default.aspx}
		\item For US citizens with good track records in research
		\end{enumerate}
	\item American Geophysical Union: \vspace{-0.2cm}
		\begin{enumerate} \itemsep -2pt
		\item Research Grants and Awards: \url{http://www.agu.org/about/honors/research_grants/}
		\item Student Travel Grants: \url{http://www.agu.org/education/grants/travel.shtml}
		\item Research Grants \& Awards: \url{http://www.agu.org/education/grants/research.shtml}
		\item Mass Media Fellowship: \url{http://www.agu.org/news/mass_media_fellowship/}
		\end{enumerate}
	\item Society of Physics Students (SPS): \vspace{-0.2cm}
		\begin{enumerate} \itemsep -2pt
		\item SPS Scholarships: \url{http://www.spsnational.org/programs/scholarships/}
		\item SPS Awards: \url{http://www.spsnational.org/programs/awards/}
		\end{enumerate}
	\end{enumerate}
\item Consortium for Ocean Leadership: \vspace{-0.3cm}
	\begin{enumerate} \itemsep -2pt
	\item Employment, Internships, and Opportunities [ includes funding opportunities for researchers (professors, postdocs, and grad students) ]: \url{http://www.oceanleadership.org/about-ocean-leadership/ocean-of-opportunities/}
	\item HBCU Fellowship: Ocean Leadership/IODP-USIO for Students of Historically Black Colleges and Universities, \url{http://www.oceanleadership.org/education/diversity/hbcu-fellowship/}
	\item HBCU Educator at Sea: \url{http://www.oceanleadership.org/education/diversity/hbcu-educator/}
	\item MS PHD's Professional Development Program: The Minorities Striving and Pursuing Higher Degrees of Success in the Earth System Sciences (MS PHD'S) Professional Development Program, \url{http://www.oceanleadership.org/education/diversity/ms-phds-professional-development-program/}
	\item Schlanger Ocean Drilling Fellowship Program (merit-based awards for outstanding graduate students to conduct research related to the Integrated Ocean Drilling Program): \url{http://www.oceanleadership.org/programs-and-partnerships/usssp/schlanger-fellowship/}
	\end{enumerate}
\item American Geological Institute Foundation: \vspace{-0.3cm}
	\begin{enumerate} \itemsep -2pt
	\item William L. Fisher Congressional Geoscience Fellowship (for young geoscientists to get engaged in {\bf public policy}): \url{http://www.agifoundation.org/govtaffairs.html} and \url{http://www.agifoundation.org/endowments.html}
	\item AGI Minority Participation Program: Minority Participation Program Geoscience Student Scholarships for ``underrepresented ethnic-minority (undergraduate or graduate) students in the geosciences'', \url{http://www.agiweb.org/mpp/index.html}
	\end{enumerate}
\item Lady Davis Institute/Jewish General Hospital: \vspace{-0.3cm}
	\begin{enumerate} \itemsep -2pt
	\item Awards for ``graduate students (in biomedical science) and post-doctoral fellows/clinical fellows'': \url{http://www.ladydavis.ca/en/awards}
	\end{enumerate}
\item Adolph C. and Mary Sprague Miller Institute for Basic Research in Science: \vspace{-0.3cm}
	\begin{enumerate} \itemsep -2pt
	\item Miller Fellowships (for outstanding recent Ph.D.s / postdoctoral fellowship): \url{http://millerinstitute.berkeley.edu/topage.php?nav=11&to=1} or \url{http://millerinstitute.berkeley.edu/page.php?nav=11}
	\item Visiting Miller Research Professorships (for professors and research scientists): \url{http://millerinstitute.berkeley.edu/topage.php?nav=24&to=1} or \url{http://millerinstitute.berkeley.edu/page.php?nav=24}
	\item Miller Research Professorships (for professors in the UC system): \url{http://millerinstitute.berkeley.edu/topage.php?nav=15&to=1} or \url{http://millerinstitute.berkeley.edu/page.php?nav=15}
	\item Miller Senior Fellowships (Nominations are solicited by invitation only; Senior Fellow appointments are made to tenured UC Berkeley faculty for five years, possibly renewable for a subsequent five years, but no longer.): \url{http://millerinstitute.berkeley.edu/topage.php?nav=126&to=1}
	\end{enumerate}
\item Funda{\c{c}}{\~{a}}o para a Ci{\^{e}}ncia e a Tecnologia (FCT); Minist{\'{e}}rio da Ci{\^{e}}ncia, Technologia e Ensino Superior (MCTES): International Prize Fernando Gil in Philosophy of Science, \url{http://alfa.fct.mctes.pt/apoios/premios/fernando_gil/index.phtml.pt}
\item Wellcome Trust: \vspace{-0.3cm}
	\begin{enumerate} \itemsep -2pt
	\item Wellcome Trust Sanger Institute: \vspace{-0.2cm}
		\begin{enumerate} \itemsep -2pt
		\item \url{http://www.sanger.ac.uk/workstudy/}
		\item Postdoctoral fellows (for research in genomics): \url{http://www.sanger.ac.uk/workstudy/career/postdocs/}
		\item Graduate program (for research in genomics): \url{http://www.sanger.ac.uk/workstudy/phd/}
		\item Student placements and work experience (for research in genomics): \url{http://www.sanger.ac.uk/workstudy/placements/}
		\end{enumerate}
	\end{enumerate}
\item Paul B. Beeson Career Development Awards in Aging Research Program (formerly the Beeson Physician Faculty Scholars Program): \vspace{-0.3cm}
	\begin{enumerate} \itemsep -2pt
	\item \url{http://www.beeson.org/}
	\item ``Today, the Beeson program continues to foster the independent research careers of clinically trained investigators -- a growing cadre of talented physician-scientists -- whose research and leadership are enhancing the health and quality of life of Americans, particularly older people.''
	\item About the Program: \url{http://www.beeson.org/program_hx.cfm}
	\end{enumerate}
\item American Mathematical Society: \vspace{-0.3cm}
	\begin{enumerate} \itemsep -2pt
	\item AMS Fellowships and Scholarships: \vspace{-0.2cm}
		\begin{enumerate} \itemsep -2pt
		\item \url{http://e-math.ams.org/programs/ams-fellowships/ams-fellowships}
		\item AMS Centennial Research Fellowship Program: \url{http://e-math.ams.org/programs/ams-fellowships/centennial-fellow/emp-centflyer}
		\item Waldemar J. Trijitzinsky Memorial Awards: \url{http://e-math.ams.org/programs/ams-fellowships/trjitzinsky/trjitzinsky-award}
		\item Other Sources of Funding: \url{http://e-math.ams.org/programs/funding/funding}
		\end{enumerate}
	\end{enumerate}
\item --- --- --- --- --- --- --- --- --- --- --- --- --- --- --- --- --- --- --- --- --- --- --- --- --- --- --- --- --- --- ---
\item \colorbox{blue}{\bf Scholarships and Fellowships in Medicine}
% Scholarships and Fellowships in Medicine
\item Sarnoff Medical Student Research Fellowship Program (for US medical students interested in cardiovascular research): \url{http://www.sarnoffendowment.org/}
\item Mayo Clinic: \vspace{-0.3cm}
	\begin{enumerate} \itemsep -2pt
	\item Postbaccalaureate Research Education Program (PREP): \url{http://www.mayo.edu/mgs/postbac-program.html}
	\end{enumerate}
\item Paul B. Beeson Career Development Awards in Aging Research Program (formerly the Beeson Physician Faculty Scholars Program): \vspace{-0.3cm}
	\begin{enumerate} \itemsep -2pt
	\item \url{http://www.beeson.org/}
	\item ``Today, the Beeson program continues to foster the independent research careers of clinically trained investigators -- a growing cadre of talented physician-scientists -- whose research and leadership are enhancing the health and quality of life of Americans, particularly older people.''
	\item About the Program: \url{http://www.beeson.org/program_hx.cfm}
	\end{enumerate}
\item --- --- --- --- --- --- --- --- --- --- --- --- --- --- --- --- --- --- --- --- --- --- --- --- --- --- --- --- --- --- ---
\item \colorbox{blue}{\bf Scholarships and Fellowships in Science and Engineering}
% Scholarships and Fellowships in Science and Engineering
\item National Academies: \vspace{-0.3cm}
	\begin{enumerate} \itemsep -2pt
	\item Research Associateship Programs (graduate, postdoctoral, and senior level research opportunities): \url{http://sites.nationalacademies.org/pga/rap/}
	\item Ford Foundation Fellowship Programs (predoctoral, dissertation or postdoctoral fellowships for individuals seeking academic careers in science and engineering): \url{http://sites.nationalacademies.org/PGA/FordFellowships/index.htm}
	\item \url{http://nationalacademies.org/grantprograms.html}
	\item \url{http://sites.nationalacademies.org/pga/fellowships/}
	\item List of Fellowship, Scholarship, and Grant Databases: \url{http://sites.nationalacademies.org/PGA/Fellowships/PGA_046300}
	\item List of Outside Fellowships, Scholarships, and Grants Websites: \url{http://sites.nationalacademies.org/PGA/Fellowships/PGA_046301}
	\item Awards for junior and mid-career researchers: \url{http://www.nasonline.org/site/PageServer?pagename=AWARDS_main}
	\item National Academy of Engineering, NAE: \vspace{-0.2cm}
		\begin{enumerate} \itemsep -2pt
		\item NAE Grand Challenges Scholars Program: \url{http://www.grandchallengescholars.org/}
		\end{enumerate}
	\item National Science Foundation: \vspace{-0.2cm}
		\begin{enumerate} \itemsep -2pt
		\item International Research Fellowship Program (IRFP) for junior scientists and engineers: \url{http://www.nsf.gov/funding/pgm_summ.jsp?pims_id=5179}
		\item Integrative Graduate Education and Research Traineeship Program (IGERT) for undergraduates and graduate students in STEM: \url{http://www.nsf.gov/funding/pgm_summ.jsp?pims_id=12759}
		\item National Science Foundation's Graduate Research Fellowship Program (GRFP) for students seeking research degrees in STEM: \url{http://www.nsfgrfp.org/}
		\item NSF Alliances for Graduate Education and the Professoriate (AGEP) program (to help underrepresented minorities obtain graduate degrees in STEM and prepare them for faculty positions in academia): \url{http://www.nsfagep.org/}
		\item National Science Foundation's (NSF) East Asia and Pacific Summer Institutes (EAPSI) program: \vspace{-0.1cm}
			\begin{enumerate} \itemsep -1pt
			\item \url{http://www.nsf.gov/funding/pgm_summ.jsp?pims_id=5284}
			\item The East Asia and Pacific Summer Institutes (EAPSI) provide U.S. graduate students in science and engineering: \vspace{-0.1cm}
				\begin{itemize} \itemsep -1pt
				\item first-hand research experiences in Australia, China, Japan, Korea, New Zealand, Singapore or Taiwan
				\item an introduction to the science, science policy, and scientific infrastructure of the respective location
				\item an orientation to the society, culture and language.
				\end{itemize}
			\item ``The primary goals of EAPSI are to introduce students to East Asia and Pacific science and engineering in the context of a research setting, and to help students initiate scientific relationships that will better enable future collaboration with foreign counterparts.''
			\item ``All institutes, except Japan, last approximately eight weeks from June to August. Japan lasts approximately ten weeks from June to August (specific dates are available and updated at \url{http://www.nsfsi.org/}).''
			\item Example of Ph.D. student, Jakub Szefer, from Prof. Ruby Lee's lab at Princeton University, who interned with Prof. Cheng Chen-Mou from National Taiwan University: \url{http://www.nsf.gov/discoveries/disc_summ.jsp?cntn_id=118116&org=NSF}
			\end{enumerate}
		\end{enumerate}
	\end{enumerate}
\item United States Department of Defense (DoD): \vspace{-0.3cm}
	\begin{enumerate} \itemsep -2pt
	\item National Defense Education Program; Defense Advanced Research Projects Agency (DARPA): \vspace{-0.2cm}
		\begin{enumerate} \itemsep -2pt
		\item Science, Mathematics, and Research for Transformation (SMART) scholarship program: \vspace{-0.1cm}
			\begin{itemize} \itemsep -1pt
			\item \url{http://smart.asee.org/}
			\item Co-organized by the American Society for Engineering Education
			\end{itemize}
		\item National Security Science and Engineering Faculty Fellowships (NSSEFF): \url{http://www.ndep.us/ProgNSSEFF.aspx}
		\end{enumerate}
	\end{enumerate}
\item National Society of Professional Engineers: \vspace{-0.3cm}
	\begin{enumerate} \itemsep -2pt
	\item Scholarships for undergraduates and graduate students: \url{http://www.nspe.org/Students/Scholarships/index.html}
	\item NSPE-PEC George B. Hightower, P.E. Fellowship (for an outstanding engineering graduate student): \url{http://www.nspe.org/InterestGroups/PEC/Resources/Awards/hightower_fellowship.html}
	\item PEG Management Fellowship: \vspace{-0.2cm}
		\begin{enumerate} \itemsep -2pt
		\item \url{http://www.nspe.org/InterestGroups/PEG/Resources/AwardsAndScholarships/peg_fellowship.html}
		\item ``This scholarship is designed for graduate students who are pursuing an MBA, a master's degree in engineering management, or a master's degree in public administration.''
		\end{enumerate}
	\end{enumerate}
\item Technion -- Israel Institute of Technology: \vspace{-0.3cm}
	\begin{enumerate} \itemsep -2pt
	\item Department of Mathematics: Anna and Paul Erdos postdoctoral Fellowship, \url{http://www.math.technion.ac.il/Site/people/positions.html}
	\item Lady Davis Postdoctoral Fellowship
	\item Department of Electrical Engineering: \vspace{-0.2cm}
		\begin{enumerate} \itemsep -2pt
		\item The Andrew and Erna Finci Viterbi Fellowship Program (for graduate and post-doctoral fellows), \url{http://webee.technion.ac.il/Research/Fellowship-Programs}
		\item Lady Davis Fellowship Trust: Technion Fellowships (for visiting professors, post-doctoral researchers, as well as Masters and Ph.D. students), \url{http://ldft.huji.ac.il/upload/info/}
		\item \url{http://webee.technion.ac.il/Research/Fellowship-Programs}
		\end{enumerate}
	\end{enumerate}
\item Hebrew University: \vspace{-0.3cm}
	\begin{enumerate} \itemsep -2pt
	\item Lady Davis Fellowship Trust: Technion Fellowships (for visiting professors, post-doctoral researchers, as well as Masters and Ph.D. students), \url{http://ldft.huji.ac.il/upload/info/infoHUa.html}
	\end{enumerate}
\item Hertz Foundation: \vspace{-0.3cm}
	\begin{enumerate} \itemsep -2pt
	\item The Graduate Fellowship Award: \url{http://www.hertzfoundation.org/dx/Fellowships/award.aspx}
	\item Thesis Prize: \url{http://www.hertzfoundation.org/dx/Fellowships/thesis_winners.aspx}
	\end{enumerate}
\item Krell Institute, Inc.: \vspace{-0.3cm}
	\begin{enumerate} \itemsep -2pt
	\item DOE Computational Science Graduate Fellowship: \url{http://www.krellinst.org/csgf/index.shtml}
	\end{enumerate}
\item The Winston Churchill Foundation of the United States: \vspace{-0.3cm}
	\begin{enumerate} \itemsep -2pt
	\item The Churchill Scholarship: \url{http://winstonchurchillfoundation.org/index.php?hide=1&section=eligibility}
	\end{enumerate}
\item American Society for Engineering Education: \vspace{-0.3cm}
	\begin{enumerate} \itemsep -2pt
	\item \url{http://blogs.asee.org/fellowships/}
	\item Fellowship programs: \url{http://www.asee.org/fellowship-programs}
	\item Awards: \url{http://www.asee.org/member-resources/awards/full-list-of-awards}
	\item DuPont Minorities in Engineering Award: \vspace{-0.2cm}
		\begin{enumerate} \itemsep -2pt
		\item \url{http://www.asee.org/member-resources/awards/full-list-of-awards/national-awards/special#DuPont_Minorities_in_Engineering_Award}
		\item {\bf \color{blue} ``The DuPont Minorities in Engineering Award is conferred for outstanding achievements by an engineering or engineering technology educator in increasing student diversity within engineering and engineering technology programs.''}
		\end{enumerate}
	\end{enumerate}
\item Alexander von Humboldt-Stiftung/Foundation: \vspace{-0.3cm}
	\begin{enumerate} \itemsep -2pt
	\item Feodor Lynen Research Fellowship for Postdoctoral Researchers (junior postdocs): \url{http://www.humboldt-foundation.de/web/feodor-lynen-fellowship-postdoc.html}
	\item Friedrich Wilhelm Bessel Research Award (mid-career researchers): \url{http://www.humboldt-foundation.de/web/bessel-award.html}
	\item Georg Forster Research Fellowship for Postdoctoral Researchers (for non-German junior postdocs ``with above average qualifications''): \url{http://www.humboldt-foundation.de/web/georg-forster-fellowship-postdoc.html}
	\item Humboldt Research Fellowship for Postdoctoral Researchers (junior postdocs): \url{http://www.humboldt-foundation.de/web/771.html}
	\item Sofja Kovalevskaja Award (junior postdocs): \url{http://www.humboldt-foundation.de/web/kovalevskaja-award.html}
	\item Fraunhofer-Bessel Research Award: \url{http://www.humboldt-foundation.de/web/fraunhofer-bessel-award.html}
	\item \url{http://www.humboldt-foundation.de/web/home.html}
	\end{enumerate}
\item Santa Fe Institute: Omidyar Postdoctoral Fellowship; see \url{http://www.santafe.edu/education/fellowships/omidyar-postdoctoral/}
\item Applied Materials: Applied Materials Graduate Fellowship
\item American Society of Naval Engineers (ASNE): \vspace{-0.3cm}
	\begin{enumerate} \itemsep -2pt
	\item (Undergraduate and Graduate) Scholarships: \url{http://www.navalengineers.org/awards/scholarships/Pages/ASNELandingPage.aspx}
	\end{enumerate}
\item Lindau Meeting of Nobel Laureates and Students in Lindau (Oak Ridge Associated Universities, ORAU): \vspace{-0.3cm}
	\begin{enumerate} \itemsep -2pt
	\item Graduate Student Award program: \vspace{-0.2cm}
		\begin{enumerate} \itemsep -2pt
		\item \url{http://www.orau.org/lindau/}
		\item A student nominated to participate in this program must: \vspace{-0.1cm}
			\begin{enumerate} \itemsep -1pt
			\item Be a U.S. citizen
			\item Be currently enrolled as a full-time graduate student
			\item Be currently sponsored by, or working on, and supported by projects sponsored by, the agency to which the nomination is made, such as the U.S. Department of Energy Office of Science, the National Institutes of Health or other federal agency
			\item Have completed by June 2011 two years (but not more than four years) of study toward a doctoral degree in medicine or physiology, or in a related discipline, including the basic biomedical (or life) sciences
			\end{enumerate}
		\end{enumerate}
	\end{enumerate}
\item Research Councils UK (RCUK): \vspace{-0.3cm}
	\begin{enumerate} \itemsep -2pt
	\item RCUK Academic Fellowships: \vspace{-0.2cm}
		\begin{enumerate} \itemsep -2pt
		\item \url{http://www.rcuk.ac.uk/ResearchCareers/fellowships/Pages/home.aspx}
		\item \url{http://www.rcuk.ac.uk/ResearchCareers/fellowships/Pages/about.aspx}
		\item Dorothy Hodgkin Postgraduate Awards: \vspace{-0.1cm}
			\begin{enumerate} \itemsep -1pt
			\item \url{http://www.rcuk.ac.uk/ResearchCareers/dhpa/Pages/home.aspx}
			\item ``Dorothy Hodgkin Postgraduate Awards (DHPA) is a UK scheme to bring outstanding students from India, China, Hong Kong, South Africa, Brazil, Russia and the developing world to come and study for PhDs in top rated UK research facilities.''
			\end{enumerate}
		\end{enumerate}
	\item International Funding Opportunities: \vspace{-0.2cm}
		\begin{enumerate} \itemsep -2pt
		\item \url{http://www.rcuk.ac.uk/international/funding/FundingOpps/Pages/home.aspx}
		\item Early Career Researchers: \url{http://www.rcuk.ac.uk/international/funding/FundingOpps/Pages/EarlyCareer.aspx}
		\end{enumerate}
	\item Engineering and Physical Sciences Research Council: \vspace{-0.2cm}
		\begin{enumerate} \itemsep -2pt
		\item Programs: \vspace{-0.1cm}
			\begin{enumerate} \itemsep -1pt
			\item Physical sciences: \vspace{-0.1cm}
				\begin{itemize} \itemsep -1pt
				\item Organic synthetic chemistry studentships: \url{http://www.epsrc.ac.uk/about/progs/physsci/Pages/organicstudentships.aspx}
				\item Analytical science studentships: \url{http://www.epsrc.ac.uk/about/progs/physsci/Pages/analyticalstudentships.aspx}
				\end{itemize}
			\item Mathematical sciences: \vspace{-0.1cm}
				\begin{itemize} \itemsep -1pt
				\item Fellowships (for postdoctoral research): \url{http://www.epsrc.ac.uk/about/progs/maths/Pages/fellowships.aspx}
				\end{itemize}
			\end{enumerate}
		\item Funding: \vspace{-0.1cm}
			\begin{enumerate} \itemsep -1pt
			\item \url{http://www.epsrc.ac.uk/funding/Pages/default.aspx}
			\item Grants available [has funds for (new/junior) professors and to support international collaboration]: \url{http://www.epsrc.ac.uk/funding/grants/Pages/default.aspx}
			\item Calls for proposals (open/current funding calls for applications and future/proposed calls): \url{http://www.epsrc.ac.uk/funding/calls/Pages/default.aspx}
			\item Studentships (training grants for Ph.D. and Masters students, including international students): \url{http://www.epsrc.ac.uk/funding/students/Pages/default.aspx}
			\item Fellowships (from junior scientists and engineers engaged in postdoctoral research to senior researchers): \url{http://www.epsrc.ac.uk/funding/fellows/Pages/default.aspx}
			\end{enumerate}
		\end{enumerate}
	\item Biotechnology and Biological Sciences Research Council (BBSRC): \vspace{-0.2cm}
		\begin{enumerate} \itemsep -2pt
		\item ``The UK's leading funding agency for academic research and training in the non-clinical life sciences''
		\item Funding research: \vspace{-0.1cm}
			\begin{enumerate} \itemsep -1pt
			\item \url{http://www.bbsrc.ac.uk/funding/funding-index.aspx}
			\item Fellowships (for early career scientists, for supporting individuals seeking a change in research directions or scientists who are returning to research, and senior researchers): \url{http://www.bbsrc.ac.uk/funding/fellowships/fellowships-index.aspx}
			\item Studentships (Doctoral training grants, Masters training grants, postgraduate awards, and undergraduate research grants): \url{http://www.bbsrc.ac.uk/funding/studentships/studentships-index.aspx}
			\item Special opportunities (current calls for funding): \url{http://www.bbsrc.ac.uk/funding/opportunities/opportunities-index.aspx}
			\item Apply for funding (information about the process of applying for research funds): \url{http://www.bbsrc.ac.uk/funding/apply/apply-index.aspx}
			\end{enumerate}
		\end{enumerate}
	\item Science and Technology Facilities Council: \vspace{-0.2cm}
		\begin{enumerate} \itemsep -2pt
		\item STFC Grants and Awards: \vspace{-0.1cm}
			\begin{enumerate} \itemsep -1pt
			\item \url{http://www.stfc.ac.uk/Funding+and+Grants/501.aspx}
			\item ``The Science and Technology Facilities Council offers grants and support in Particle Physics, Astronomy, Nuclear Physics and Facility Development. It also provides support for research infrastructure, training, knowledge exchange and public engagement activities through a variety of funding schemes and activities.''
			\item STFC Funding Opportunities: \url{http://www.stfc.ac.uk/Funding%20and%20Grants/598.aspx}
			\item Postgraduate Studentships: \url{http://www.stfc.ac.uk/Funding+and+Grants/637.aspx} or \url{http://www.stfc.ac.uk/Funding%20and%20Grants/636.aspx}
			\end{enumerate}
		\item Fellowship opportunities: \vspace{-0.1cm}
			\begin{enumerate} \itemsep -1pt
			\item \url{http://www.stfc.ac.uk/Funding%20and%20Grants/508.aspx}
			\item ``Fellowship opportunities in Astronomy, Solar and Planetary Science, Particle Physics, Particle Astrophysics, Nuclear Physics, Development of STFC Neutron, Laser and Synchrotron Facilities within the UK.''
			\item There are postdoctoral and advanced research fellowships.
			\end{enumerate}
		\item Innovations Partnership Schemes (IPS and mini-IPS): \url{http://www.stfc.ac.uk/19213.aspx}
		\item IPS Fellowships: \vspace{-0.1cm}
			\begin{enumerate} \itemsep -1pt
			\item \url{http://www.stfc.ac.uk/19226.aspx}
			\item The IPS fellowship is a scheme designed to support a role to develop the commercial exploitation of technologies. This is not a research orientated fellowship.
			\end{enumerate}
		\item Follow-on-Funding: \vspace{-0.1cm}
			\begin{enumerate} \itemsep -1pt
			\item \url{http://www.stfc.ac.uk/19207.aspx}
			\item ``Follow on Funding is intended to provide financial support at the very early or pre-seed stage of turning research outputs into a commercial proposition. Unlike the other research councils, in STFC, industry partners are not allowed. If you have an industry partner, please use the mini-IPS or IPS scheme.''
			\item ``STFC staff, grant funded academics and researchers at CERN and ESO are eligible to apply for follow-on-funds (see the research grants handbook for CERN and ESO eligibility). STFC staff should first investigate whether they can be funded through proof of concept funding.''
			\end{enumerate}
		\end{enumerate}
	\item Natural Environment Research Council: \vspace{-0.2cm}
		\begin{enumerate} \itemsep -2pt
		\item Grants and studentships on the web: \vspace{-0.1cm}
			\begin{enumerate} \itemsep -1pt
			\item \url{http://www.nerc.ac.uk/research/gotw.asp}
			\item Grants on the web: \url{http://gotw.nerc.ac.uk/goti.asp?c=1}
			\end{enumerate}
		\item Funding: \vspace{-0.1cm}
			\begin{enumerate} \itemsep -1pt
			\item \url{http://www.nerc.ac.uk/funding/}
			\item Postgraduate training: \vspace{-0.1cm}
				\begin{itemize} \itemsep -1pt
				\item Postgraduate eligibility (requires UK/EU citizenship): \url{http://www.nerc.ac.uk/funding/available/postgrad/eligibility.asp}
				\end{itemize}
			\item Research Fellowship Scheme [for all nationalities]: \url{http://www.nerc.ac.uk/funding/available/fellowships/}
			\item Research Experience Placements (REP) scheme [for undergraduates]: \url{http://www.nerc.ac.uk/funding/available/rep.asp}
			\item Research Grants: \vspace{-0.1cm}
				\begin{itemize} \itemsep -1pt
				\item Eligibility: \url{http://www.nerc.ac.uk/funding/available/researchgrants/eligibility.asp}
				\end{itemize}
			\end{enumerate}
		\item {\bf Other potential sources of funding}: \vspace{-0.1cm}
			\begin{enumerate} \itemsep -1pt
			\item \url{http://www.nerc.ac.uk/funding/otherfunding.asp}
			\item Look at the ``Higher Education Funding Councils'' for each country (England, Wales, Northern Ireland, and Scotland)
			\end{enumerate}
		\end{enumerate}
	\end{enumerate}
\item Nuffield Foundation: \vspace{-0.3cm}
	\begin{enumerate} \itemsep -2pt
	\item Undergraduate research bursaries in science: \url{http://www.nuffieldfoundation.org/undergraduate-research-bursaries-0}
	\item Funding for social policy projects in the UK: \vspace{-0.2cm}
		\begin{enumerate} \itemsep -2pt
		\item \url{http://www.nuffieldfoundation.org/social-policy}
		\item \url{http://www.nuffieldfoundation.org/children-and-families-law-society-education-and-open-door}
		\end{enumerate}
	\item Apply for funding: \url{http://www.nuffieldfoundation.org/apply-for-funding}
	\item Africa program: \url{http://www.nuffieldfoundation.org/africa-programme-0}
	\item Nuffield Farming Scholarships Trust: \vspace{-0.2cm}
		\begin{enumerate} \itemsep -2pt
		\item Nuffield Farming Scholarships: \url{http://www.nuffieldscholar.org/}
		\end{enumerate}
	\item The Nuffield Trust (or, The Nuffield Trust for Research and Policy Studies in Health Services): \vspace{-0.2cm}
		\begin{enumerate} \itemsep -2pt
		\item Fellowships: \vspace{-0.1cm}
			\begin{enumerate} \itemsep -1pt
			\item \url{http://www.nuffieldtrust.org.uk/fellowships/index.aspx?id=43}
			\item Rock Carling fellowship (for senior researchers in public health): \url{http://www.nuffieldtrust.org.uk/fellowships/index.aspx?id=112}
			\item John Fry Fellowship (for senior researchers in public health): \url{http://www.nuffieldtrust.org.uk/fellowships/index.aspx?id=109}
			\item Harkness Fellowships in Health Care Policy: \vspace{-0.1cm}
				\begin{itemize} \itemsep -1pt
				\item ``Since September 2009 The Nuffield Trust have been the proud co-sponsors of the prestigious Harkness Fellowships programme with The Commonwealth Fund.''
				\item ``These offer an unparalleled opportunity for the health policy analysts of the future to conduct original research and learn about healthcare in North America.''
				\item ``Mid-career health policy researchers and practitioners (including doctors, health services managers, journalists and government officials) are supported to spend 9 to 12 months in the United States conducting a policy-oriented research project and working with leading U.S. health policy experts.''
				\end{itemize}
			\end{enumerate}
		\end{enumerate}
	\end{enumerate}
\item U.S. Department of Homeland Security (DHS): \vspace{-0.3cm}
	\begin{enumerate} \itemsep -2pt
	\item DHS Scholarship and Fellowship Program: \url{http://www.orau.gov/dhsed/}
	\end{enumerate}
\item ACT, Inc.: \vspace{-0.3cm}
	\begin{enumerate} \itemsep -2pt
	\item Barry M. Goldwater Scholarship and Excellence in Education Program (for US residents who will be college upperclassmen in STEM fields in the following academic year): \url{http://www.act.org/goldwater/}
	\end{enumerate}
\item Massachusetts Institute of Technology: \vspace{-0.3cm}
	\begin{enumerate} \itemsep -2pt
	\item MIT School of Engineering: \vspace{-0.2cm}
		\begin{enumerate} \itemsep -2pt
		\item Lemelson-MIT Program: \vspace{-0.1cm}
			\begin{enumerate} \itemsep -1pt
			\item \url{http://web.mit.edu/invent/}
			\item Lemelson-MIT Awards for Invention and Innovation: \url{http://web.mit.edu/invent/a-main.html}
			\end{enumerate}
		\end{enumerate}
	\end{enumerate}
\item --- --- --- --- --- --- --- --- --- --- --- --- --- --- --- --- --- --- --- --- --- --- --- --- --- --- --- --- --- --- ---
\item \colorbox{blue}{\bf Scholarships and Fellowships in Various Fields (Including Creative Arts, Teaching, and Sports)}
% Scholarships and Fellowships in Various Fields (Including Creative Arts, Teaching, and Sports)
\item U.S. Department of Education: \vspace{-0.3cm}
	\begin{enumerate} \itemsep -2pt
	\item Robert C. Byrd Honors Scholarship Program: \vspace{-0.2cm}
		\begin{enumerate} \itemsep -2pt
		\item High school graduates who have been accepted for enrollment at institutions of higher education (IHEs), have demonstrated outstanding academic achievement, and show promise of continued academic excellence may apply to states in which they are residents.
		\item \url{http://www2.ed.gov/programs/iduesbyrd/index.html}
		\end{enumerate}
	\item \colorbox{yellow}{\bf Jacob K. Javits Fellowships Program}: \vspace{-0.1cm}
		\begin{enumerate} \itemsep -1pt
		\item This program provides fellowships to students of superior academic ability -- selected on the basis of demonstrated achievement, financial need, and exceptional promise -- to undertake study at the doctoral and Master of Fine Arts level in selected fields of arts, humanities, and social sciences.
		\item \url{http://www2.ed.gov/programs/jacobjavits/index.html}
		\end{enumerate}
	\item Close Up Fellowship Program: \vspace{-0.2cm}
		\begin{enumerate} \itemsep -2pt
		\item This program provides financial aid to enable low-income students, their teachers, and recent immigrants to come to Washington, D.C., to study the operations of the three branches of the federal government.
		\item \url{http://www2.ed.gov/programs/closeup/index.html}
		\end{enumerate}
	\item {\bf \color{blue} B.J. Stupak Olympic Scholarships}: \vspace{-0.2cm}
		\begin{enumerate} \itemsep -2pt
		\item This program provides financial assistance to athletes who are training at the U.S. Olympic Education Center or one of the U.S. Olympic training centers and who are pursuing a postsecondary education at institutions of higher education (IHEs).
		\item \url{http://www2.ed.gov/programs/olympic/index.html}
		\end{enumerate}
	\item {\bf \color{blue} Teacher Education Assistance for College and Higher Education (TEACH) Grant Program}: \vspace{-0.2cm}
		\begin{enumerate} \itemsep -2pt
		\item Through the College Cost Reduction and Access Act of 2007, Congress created the Teacher Education Assistance for College and Higher Education (TEACH) Grant Program that provides grants of up to \$4,000 per year to students who intend to teach in a public or private elementary or secondary school that serves students from low-income families.
		\item \url{http://studentaid.ed.gov/PORTALSWebApp/students/english/TEACH.jsp}
		\end{enumerate}
	\item Scholarship search engine: \url{https://studentaid2.ed.gov/getmoney/scholarship/}
	\item Financial Aid: \vspace{-0.2cm}
		\begin{enumerate} \itemsep -2pt
		\item \url{http://www2.ed.gov/finaid/landing.jhtml?src=rt}
		\item \url{http://studentaid.ed.gov/PORTALSWebApp/students/english/funding.jsp}
		\item Paying for college: \url{http://www.college.gov}
		\item Student Aid (has information for students at all levels and parents): \url{http://studentaid.ed.gov/}
		\item Student Aid Eligibility: \url{http://studentaid.ed.gov/PORTALSWebApp/students/english/aideligibility.jsp?tab=funding}
		\item Federal Student Aid: \url{http://federalstudentaid.ed.gov/}
		\item Academic Competitiveness Grant: The Academic Competitiveness Grant provides up to \$750 for the first year of undergraduate study and up to \$1,300 for the second year of undergraduate study. See \url{http://studentaid.ed.gov/PORTALSWebApp/students/english/NewPrograms.jsp}.
		\end{enumerate}
	\item Free Application for Federal Student Aid (FAFSA): \vspace{-0.2cm}
		\begin{enumerate} \itemsep -2pt
		\item Financial Aid Estimator Tool (FAFSA4caster): \url{http://www.fafsa4caster.ed.gov/F4CApp/index/index.jsf}
		\item \url{http://www.fafsa.ed.gov/}
		\end{enumerate}
	\item Federal Pell Grant Program: \url{http://www2.ed.gov/programs/fpg/index.html}
	\end{enumerate}
\item European Commission: \vspace{-0.3cm}
	\begin{enumerate} \itemsep -2pt
	\item Erasmus Programme (for Europeans): \url{http://ec.europa.eu/education/lifelong-learning-programme/doc80_en.htm}
	\item Erasmus Mundus (for non-Europeans): \url{http://ec.europa.eu/education/external-relation-programmes/doc72_en.htm}
	\end{enumerate}
\item Woodrow Wilson Foundation: \vspace{-0.3cm}
	\begin{enumerate} \itemsep -2pt
	\item {\bf \color{blue} The Woodrow Wilson-Rockefeller Brothers Fund Fellowships for Aspiring Teachers of Color (for underrepresented minorities seeking a career as a K-12 public school teacher in the US)}: \url{http://www.woodrow.org/teaching-fellowships/wwrbf/index.php}
	\item {\bf \color{blue} Woodrow Wilson Teaching Fellowship (for a MS program in teacher education, who would teach at high-need urban and rural schools or $\ge$ 3 years)}: \url{http://www.wwteachingfellowship.org/}
	\item {\bf \color{blue} Leonore Annenberg Teaching Fellowship (for a MS program in teacher education, who would teach at high-need urban and rural schools or $\ge$ 3 years)}: \url{http://www.woodrow.org/teaching-fellowships/annenberg/index.php}
	\item MMUF Travel \& Research Grants (for graduate students who participated in the Mellon Mays Undergraduate Fellowship Program): \url{http://www.woodrow.org/higher-education-fellowships/opportunity/research/index.php}
	\item MMUF Dissertation Grants (for graduate students who participated in the Mellon Mays Undergraduate Fellowship Program): \url{http://www.woodrow.org/higher-education-fellowships/opportunity/dissertation/index.php}
	\item Charlotte W. Newcombe Doctoral Dissertation Fellowship (for Ph.D. students writing their theses on ethical or religious values in all fields of the humanities and social sciences): \url{http://www.woodrow.org/higher-education-fellowships/religion_ethics/index.php}
	\item {\bf \color{blue} Woodrow Wilson Dissertation Fellowship in Women�s Studies}: \url{http://www.woodrow.org/higher-education-fellowships/women_gender/index.php}
	\item Doris Duke Conservation Fellowship program (Masters students seeking careers as practicing conservationists): \url{http://www.woodrow.org/higher-education-fellowships/conservation/index.php}
	\item Thomas R. Pickering Graduate Foreign Affairs Fellowship: \vspace{-0.2cm}
		\begin{enumerate} \itemsep -2pt
		\item Prior to joining the United States Department of State Foreign Service, this fellowship supports students entering a Masters program in the following fields: \vspace{-0.1cm}
			\begin{enumerate} \itemsep -1pt
			\item {\bf public policy}
			\item international affairs
			\item public administration
			\item academic fields such as: \vspace{-0.1cm}
				\begin{itemize} \itemsep -1pt
				\item business
				\item economics
				\item political science
				\item sociology
				\item foreign languages
				\end{itemize}
			\end{enumerate}
		\item \url{http://www.woodrow.org/higher-education-fellowships/foreign_affairs/pickering_grad/index.php}
		\end{enumerate}
	\item Thomas R. Pickering Undergraduate Foreign Affairs Fellowship (for undergraduates seeking to join the United States Department of State Foreign Service): \url{http://www.woodrow.org/higher-education-fellowships/foreign_affairs/pickering_undergrad/index.php}
	\end{enumerate}
\item Burroughs Wellcome Fund: \vspace{-0.3cm}
	\begin{enumerate} \itemsep -2pt
	\item Career Awards for Medical Scientists (post-Ph.D.): \url{http://www.bwfund.org/pages/188/Career-Awards-for-Medical-Scientists/}
	\item {\bf \color{blue} Career Award for Science and Mathematics Teachers (science or mathematics K-12 teachers in North Carolina public schools)}: \url{http://www.bwfund.org/pages/379/Career-Awards-for-Science-and-Mathematics-Teachers/}
	\end{enumerate}
\item Susan G. Komen for the Cure\textregistered: The Komen College Scholarship Program, \url{http://ww5.komen.org/ResearchGrants/CollegeScholarshipAward.html}
\item University of Kansas Madison \& Lila Self Graduate Fellowship (Ph.D. fellowships for business, economics, and STEM): \url{http://www2.ku.edu/~selfpro/}
\item Nationally Coveted College Scholarships, Graduate School Fellowships \& Postdoctoral Awards: \url{http://scholarships.fatomei.com/}
\item The Andrew W. Mellon Foundation: \vspace{-0.3cm}
	\begin{enumerate} \itemsep -2pt
	\item Fellowships \& Scholarships for undergraduates: \url{http://www.mmuf.org/undergraduates/explore-your-opportunities/fellowships-scholorships}
	\end{enumerate}
\item Siebel Scholars Foundation: \vspace{-0.3cm}
	\begin{enumerate} \itemsep -2pt
	\item For students in selected business, bioengineering, and computer science graduate programs
	\item Only available for students at selected universities.
	\item \url{http://www.siebelscholars.com/scholars}
	\item \url{http://www.siebelscholars.com/}
	\end{enumerate}
\item Aspen Institute (for leaders, e.g. in business, education, community service, and politics): \vspace{-0.3cm}
	\begin{enumerate} \itemsep -2pt
	\item Catto Fellowship Program: \url{http://www.aspeninstitute.org/leadership-programs/catto-fellowship-program}
	\item Rodel Fellowship Program: \url{http://www.aspeninstitute.org/leadership-programs/aspen-institute-rodel-fellowships-public-le-/about-rodel-fellowship-program}
	\item Henry Crown Fellowship Program: \url{http://www.aspeninstitute.org/leadership-programs/henry-crown-fellowship-program}
	\end{enumerate}
\item Smithsonian Institution: \vspace{-0.3cm}
	\begin{enumerate} \itemsep -2pt
	\item Postdoctoral Fellowships, Predoctoral Fellowships, and Graduate Student Fellowships: \vspace{-0.2cm}
		\begin{enumerate} \itemsep -2pt
		\item \url{http://www.si.edu/ofg/infotoapply.htm}
		\item \url{http://www.si.edu/ofg/fell.htm}
		\item \url{http://www.si.edu/ofg/ofgapp.htm}
		\item fields of research and study: \vspace{-0.1cm}
			\begin{enumerate} \itemsep -1pt
			\item {\bf \color{blue} American History, American Material and Folk Culture, and the History of Music and Musical Instruments}
			\item History of Science and Technology
			\item {\bf \color{blue} History of Art, Design, Crafts, and the Decorative Arts}
			\item Anthropology, Archaeology, Linguistics, and Ethnic Studies
			\item Evolutionary, Systematic, Behavioral, Environmental, and Conservation Biology
			\item Earth, Mineral, and Planetary Science
			\item Materials Characterization and Conservation
			\end{enumerate}
		\end{enumerate}
	\item Internship opportunities: \url{http://www.si.edu/ofg/internopp.htm}
	\item Research centers: \url{http://www.si.edu/research/}. [ It also has lots of information for K-12 teachers. It has resources, funding, and internship opportunities for undergraduates and graduate students pursing research in various aspects of humanities, social science, and natural science. ]
	\item Freer Gallery of Art / Arthur M. Sackler Gallery: \vspace{-0.2cm}
		\begin{enumerate} \itemsep -2pt
		\item Fellowships: \url{http://www.asia.si.edu/research/fellowships.asp}
		\end{enumerate}
	\item National Museum of American History: \vspace{-0.2cm}
		\begin{enumerate} \itemsep -2pt
		\item Jerome and Dorothy Lemelson Center for the Study of Invention and Innovation: \vspace{-0.1cm}
			\begin{enumerate} \itemsep -1pt
			\item The Lemelson Center Fellows Program (for Ph.D. students and postdocs): \url{http://invention.smithsonian.org/resources/research_fellowships.aspx}
			\end{enumerate}
		\end{enumerate}
	\end{enumerate}
\item Intercollegiate Studies Institute (ISI): \vspace{-0.3cm}
	\begin{enumerate} \itemsep -2pt
	\item William E. Simon Fellowship for Noble Purpose (for American undergraduates who are planning to use the fellowship grant for serving humanity -- in their own ways): \url{http://www.isi.org/programs/fellowships/simon.html}
	\item {\bf \color{blue} Richard M. Weaver Fellowship (for Americans who are attending a graduate program and are intending to pursue a career in academia/teaching)}: \url{http://www.isi.org/programs/fellowships/richard_weaver.html}
	\item Western Civilization Fellowships (for Americans who are attending a graduate program about Western culture/civilization): \url{http://www.isi.org/programs/fellowships/western_civilization.html}
	\item Salvatori Fellowship (for Americans who are attending a graduate program about early American history): \url{http://www.isi.org/programs/fellowships/salvatori.html}
	\item Bache Renshaw Fellowship for Doctoral Study in Education (for Americans who plan to attend doctoral programs in education): \url{http://www.isi.org/programs/fellowships/bache_renshaw.html}
	\item \url{http://www.isi.org/programs/fellowships/fellowships.html}
	\end{enumerate}
\item Le Fonds qu{\'{e}}b{\'{e}}cois de la recherche sur la nature et les technologies (The Quebec Research Fund on nature and technology): \vspace{-0.3cm}
	\begin{enumerate} \itemsep -2pt
	\item Scholarships: \url{http://www.fqrnt.gouv.qc.ca/en/bourses/index.htm}
	\end{enumerate}
\item Horatio Alger Association of Distinguished Americans, Inc.: \vspace{-0.3cm}
	\begin{enumerate} \itemsep -2pt
	\item Scholarship Programs (for US high school seniors who have faced and overcome great obstacles in their young lives): \url{https://www.horatioalger.org/scholarships/sp.cfm}
	\item Awards: \vspace{-0.2cm}
		\begin{enumerate} \itemsep -2pt
		\item \url{http://www.horatioalger.org/aboutus.cfm}
		\item Horatio Alger Award: ``dedicated community leaders who demonstrate individual initiative and a commitment to excellence; as exemplified by remarkable achievements accomplished through honesty, hard work, self-reliance and perseverance over adversity''
		\item International Horatio Alger Award: ``recipients of this award must have overcome humble beginnings and/or adversity to achieve success. They serve as outstanding role models to the international community and are committed to the Association's mission of encouraging and educating today's young people.''
		\item Norman Vincent Peale Award: ``a Member who has made exceptional humanitarian contributions to society, who has been an active participant in the Association, and who continues to exhibit courage, tenacity and integrity in the face of great challenges. ''
		\end{enumerate}
	\end{enumerate}
\item The W. Garfield Weston Foundation: \vspace{-0.3cm}
	\begin{enumerate} \itemsep -2pt
	\item Entrance Awards \& Upper Year Garfield Weston Awards (for students pursuing college or CEGEP studies in Canada): \url{http://www.garfieldwestonawards.ca/en/about}
	\end{enumerate}
\item Canadian Merit Scholarship Foundation (\url{http://www.cmsf.ca/}): Loran Award (undergraduate funding for Canadian citizens and permanent residents), \url{http://www.loranaward.ca/}
\item StartingBloc: \vspace{-0.3cm}
	\begin{enumerate} \itemsep -2pt
	\item StartingBloc Fellowship: \vspace{-0.2cm}
		\begin{enumerate} \itemsep -2pt
		\item \url{http://www.startingbloc.org/fellowship}
		\item For people who believe that economic value creation and social value creation are complementary... For people who believe in making money and doing good, and creating social and economic impact... 
		\item The Institute for Social Innovation is a ``conference'' to learn about global issues, ``corporate social responsibility, social entrepreneurship, cross sector partnerships and sustainability. Sessions are led by top academics, corporate innovators, social entrepreneurs, activists and government officials.'' 
		\end{enumerate}
	\end{enumerate}
\item The John D. and Catherine T. MacArthur Foundation: \vspace{-0.3cm}
	\begin{enumerate} \itemsep -2pt
	\item Applying for Grants: \url{http://www.macfound.org/site/c.lkLXJ8MQKrH/b.913959/k.E1BE/Applying_for_Grants.htm}
	\item Financial \& Grant Information: \url{http://www.macfound.org/site/c.lkLXJ8MQKrH/b.938093/k.9E4C/Financial__Grant_Information.htm}
	\item MacArthur Fellows Program: \url{http://www.macfound.org/site/c.lkLXJ8MQKrH/b.959463/k.9D7D/Fellows_Program.htm}
	\end{enumerate}
\item Wenner-Gren Foundations (The Wenner-Gren Center Foundation for Scientific Research, The Axel Wenner-Gren Foundation for International Exchange of Scientists and The Foundation Wenner-Grenska Samfundet): Fellowships (for Swedish postdocs), \url{http://www.swgc.org/stipendier.aspx}
\item {\'{E}}gide: \vspace{-0.3cm}
	\begin{enumerate} \itemsep -2pt
	\item EGIDE Latitudes: \url{http://www.egidelatitudes.fr/jahia/Jahia/site/egidelatitudes}
	\item Call for applications to scholarship opportunities (including a scholarship for French citizens to study abroad): \url{http://www.egide.asso.fr/jahia/Jahia/accueil/appels}
	\item Eiffel excellence scholarship programme (organized by the French Ministry of Foreign and European Affairs): \vspace{-0.2cm}
		\begin{enumerate} \itemsep -2pt
		\item \url{http://www.egide.asso.fr/jahia/Jahia/appels/eiffel}
		\item For non-French citizens pursuing advanced degrees.
		\end{enumerate}
	\end{enumerate}
\item Gottlieb Daimler and Karl Benz Foundation: \vspace{-0.3cm}
	\begin{enumerate} \itemsep -2pt
	\item {\bf \color{blue} Ph.D. fellowship for international students to study in Germany}; see \url{http://www.daimler-benz-stiftung.de/home/fellowship/en/start.html}
	\end{enumerate}
\item The San Diego Foundation: \vspace{-0.3cm}
	\begin{enumerate} \itemsep -2pt
	\item San Diego Foundation Community Scholarship Program: \vspace{-0.2cm}
		\begin{enumerate} \itemsep -2pt
		\item \url{http://www.sdfoundation.org/GrantsScholarships/Scholarships.aspx}
		\item Available scholarships: \url{http://www.sdfoundation.org/GrantsScholarships/Scholarships/ForStudents/AvailableScholarships.aspx}. Also, see \url{http://www.sdfoundation.org/GrantsScholarships/Scholarships/ForStudents/AvailableScholarships/CommonApplicationScholarships.aspx#twomey}
		\item It has scholarships for: \vspace{-0.1cm}
			\begin{enumerate} \itemsep -1pt
			\item graduating high school seniors
			\item current undergraduates
			\item non-traditional college students: \vspace{-0.1cm}
				\begin{itemize} \itemsep -1pt
				\item mature-age students
				\item mature student
				\item adult learner
				\item adult student
				\item adults who are returning to college
				\end{itemize}
			\item people pursuing teaching certificates
			\item students attending grad school
			\item students attending trade/vocational school
			\item foster youth
			\item students in various ethnic groups
			\item students in different geographical locations
			\item {\bf \color{blue} students pursuing education in certain fields, such as engineering, nursing, music, and arts and humanities}
			\end{enumerate}
		\item Separate Scholarships: \url{http://www.sdfoundation.org/GrantsScholarships/Scholarships/ForStudents/AvailableScholarships/SeparateScholarships.aspx}
		\item Other Scholarships and Financial Aid Resources: \url{http://www.sdfoundation.org/GrantsScholarships/Scholarships/ForStudents/AvailableScholarships/OtherScholarshipsandFinancialAidResources.aspx}
		\item Financial Aid Information: \url{http://www.sdfoundation.org/GrantsScholarships/Scholarships/ForStudents/Resources/FinancialAidInformation.aspx}
		\end{enumerate}
	\item Grant Opportunities (for non-profit organizations): \url{http://www.sdfoundation.org/GrantsScholarships/ForNonprofits/GrantOpportunities.aspx}
	\end{enumerate}
\item Ewing Marion Kauffman Foundation: \vspace{-0.3cm}
	\begin{enumerate} \itemsep -2pt
	\item Kauffman Dissertation Fellowship Program (for ``Ph.D., D.B.A., or other doctoral students at accredited U.S. universities to support dissertations in the area of entrepreneurship''): \url{http://www.kauffman.org/research-and-policy/kauffman-dissertation-fellowship-program.aspx}
	\item Kauffman Junior Faculty Fellowship in Entrepreneurship Research: \vspace{-0.2cm}
		\begin{enumerate} \itemsep -2pt
		\item \url{http://www.kauffman.org/research-and-policy/kauffman-junior-faculty-fellowship-in-entrepreneurship.aspx}
		\item ``to recognize tenured or tenure-track junior faculty members at accredited U.S. universities who are beginning to establish a record of scholarship and exhibit the potential to make significant contributions to the body of research in the field of entrepreneurship''
		\end{enumerate}
	\item Ewing Marion Kauffman Prize Medal for Distinguished Research in Entrepreneurship (for promising young scholars in the field of entrepreneurship): \url{http://www.kauffman.org/research-and-policy/kauffman-prize-medal-for-entrepreneurship-research.aspx}
	\item Kauffman Legal Fellowship Program (for post-J.D. research fellowship): \url{http://www.kauffman.org/research-and-policy/kauffman-legal-fellowship-program.aspx}
	\item Kauffman Global Scholars Program (for non-American top young entrepreneurs): \url{http://www.kauffman.org/entrepreneurship/kauffman-global-scholars-program.aspx}
	\item Entrepreneur Fellows program (for M.D.s and Ph.D.s who want to become high-tech start-up entrepreneurs): \url{http://www.kauffman.org/entrepreneurship/entrepreneur-fellows-program.aspx}
	\item Entrepreneur Postdoctoral Fellows program (for postdocs who want to become high-tech start-up entrepreneurs): \url{http://www.kauffman.org/entrepreneurship/entrepreneur-postdoctoral-fellows-program.aspx}
	\item Kauffman Fellows Program (``to educate and train future venture capitalists and future leaders of high-growth companies''): \url{http://www.kauffman.org/entrepreneurship/kauffman-fellows.aspx}
	\item Kauffman Foundation Outstanding Postdoctoral Entrepreneur Award: \url{http://www.kauffman.org/entrepreneurship/outstanding-postdoctoral-entrepreneur-award.aspx}
	\end{enumerate}
\item Killam Fellowships Program: \vspace{-0.3cm}
	\begin{enumerate} \itemsep -2pt
	\item \url{http://www.killamfellowships.com/}
	\item The Killam Fellowships Program allows undergraduate students from Canada and the United States to participate in a program of binational residential exchange.
	\item Killam Fellows spend either one semester or a full academic year as an exchange student in the host country.
	\end{enumerate}
\item Canada Council for the Arts: \vspace{-0.3cm}
	\begin{enumerate} \itemsep -2pt
	\item Killam Research Fellowship: \vspace{-0.2cm}
		\begin{enumerate} \itemsep -2pt
		\item \url{http://killam.canadacouncil.ca/welcome.asp}
		\item For researchers in the following fields, and interdisciplinary fields between these fields: \vspace{-0.1cm}
			\begin{enumerate} \itemsep -1pt
			\item humanities
			\item social sciences
			\item natural sciences
			\item health sciences
			\item engineering
			\end{enumerate}
		\item For outstanding researchers who are Canadian citizens or permanent residents
		\end{enumerate}
	\item Killam Prizes (and Killam Research Fellowships): \url{http://www.canadacouncil.ca/prizes/killam}
	\end{enumerate}
\item Killam Trusts: \vspace{-0.3cm}
	\begin{enumerate} \itemsep -2pt
	\item Killam Scholarship and Prize Programs (multiple fields in selected Canadian universities): \url{http://www.killamtrusts.ca/index.asp}
	\item Killam Award winners: \url{http://www.killamtrusts.ca/awardwinners.asp}
	\item Killam Scholarship and Prize Programs at various institutions (including universities): \url{http://www.killamtrusts.ca/uofAlberta.asp}
	\end{enumerate}
\item U.S. Department of State: \vspace{-0.3cm}
	\begin{enumerate} \itemsep -2pt
	\item Bureau of Educational and Cultural Affairs: \vspace{-0.2cm}
		\begin{enumerate} \itemsep -2pt
		\item Institute of International Education (administrator of program): \vspace{-0.1cm}
			\begin{enumerate} \itemsep -1pt
			\item Council for International Exchange of Scholars: \vspace{-0.1cm}
				\begin{itemize} \itemsep -1pt
				\item Fulbright Programs (for U.S. and non-U.S. Scholars): \url{http://www.cies.org/Fulbright_programs.htm}; \url{http://www.cies.org/about_fulb.htm}; \url{http://us.fulbrightonline.org/about.html}; \url{http://foreign.fulbrightonline.org/}; \url{http://exchanges.state.gov/academicexchanges/index/fulbright-program.html}; and \url{http://fulbright.state.gov/}
				\item Hubert H. Humphrey Fellowship Program: \vspace{-0.1cm}
					\begin{itemize} \itemsep -1pt
					\item For mid-career professionals in the following fields: economic development/finance and banking, agricultural and rural development, natural resources, environmental policy, and climate change, human resource management, communications/journalism, teaching of English as a foreign language, educational administration, planning, and policy, substance abuse education, treatment, and prevention, HIV/AIDS policy and prevention, public health policy and management, {\bf public policy} analysis and public administration, law and human rights, urban and regional planning, trafficking in persons - policy and prevention, technology policy and management, and higher education administration
					\item \url{http://www.humphreyfellowship.org/}
					\item \url{http://exchanges.state.gov/globalexchanges/humphrey-fellowship.html}
					\end{itemize}
				\end{itemize}
			\item International programs for scholars (search under each continent): \url{http://www.iie.org/en/Our-Global-Reach}
			\end{enumerate}
		\item International Documentary Filmmakers Fellowship: \vspace{-0.1cm}
			\begin{enumerate} \itemsep -1pt
			\item \url{http://exchanges.state.gov/cultural/docfilmmakers.html}
			\item \url{http://smpa.gwu.edu/doccenter/fellowship.php}
			\item For ``emerging or mid-career documentary filmmakers''
			\item Intensive six-week program at the Documentary Center, The George Washington University
			\end{enumerate}
		\item Office of English Language Programs: \vspace{-0.1cm}
			\begin{enumerate} \itemsep -1pt
			\item English Language Fellow Program (for ``highly qualified U.S. educators in the field of Teaching English to Speakers of Other Languages, TESOL''): \url{http://exchanges.state.gov/englishteaching/el-fellow.html}
			\item English Language Specialist Program: \vspace{-0.1cm}
				\begin{itemize} \itemsep -1pt
				\item \url{http://exchanges.state.gov/englishteaching/el-specialist.html}
				\item U.S. academics in the fields of Teaching English as a Foreign Language (TEFL) / Teaching English as a Second Language (TESL) and Applied Linguistics
				\end{itemize}
			\item E-Teacher Scholarship Program (for English teaching professionals living outside of the United States): \url{http://exchanges.state.gov/englishteaching/eteacher.html}
			\item English Access Microscholarship Program (Access): \vspace{-0.1cm}
				\begin{itemize} \itemsep -1pt
				\item \url{http://exchanges.state.gov/englishteaching/eam.html}
				\item The English Access Microscholarship Program (Access) provides a foundation of English language skills to non-elite, 14 - 18 year old students through afterschool classes and intensive summer learning activities.
				\end{itemize}
			\item \url{http://exchanges.state.gov/englishteaching/index.html}
			\end{enumerate}
		\item Office of Global Educational Programs: \vspace{-0.1cm}
			\begin{enumerate} \itemsep -1pt
			\item Community College Initiative: \vspace{-0.1cm}
				\begin{itemize} \itemsep -1pt
				\item For ``individuals from Brazil, Egypt, Ghana, Indonesia, Pakistan, South Africa, Turkey, and selected countries in Central America to spend one year studying at community colleges in the United States and earn a vocational certificate.''
				\item ``The program provides academic instruction in selected fields including agriculture, applied engineering, business management and administration, health professions, information technology, media, and tourism and hospitality management, while also immersing participants in U.S. society and cultural life.''
				\item ``Participants are recruited from historically underserved populations and may not have had opportunities for formal job training or higher education. Most participants are in their early- to mid-twenties and many already have work experience.''
				\item \url{http://exchanges.state.gov/globalexchanges/community-colleges-initiative.html}
				\end{itemize}
			\item {\bf \color{blue} Benjamin A. Gilman International Scholarship Program}: \vspace{-0.1cm}
				\begin{itemize} \itemsep -1pt
				\item ``The Benjamin A. Gilman International Scholarship Program provides scholarships to U.S. undergraduates with financial need for study abroad, including students from diverse backgrounds and students going to non-traditional study abroad destinations.''
				\item ``The applicant must be receiving a Federal Pell Grant or provide proof that he/she will be receiving a Pell Grant at the time of application or during the term of his/her study abroad.''
				\item \url{http://exchanges.state.gov/globalexchanges/gilman-scholarship-program.html}
				\end{itemize}
			\item Global Undergraduate Exchange Program (Global UGRAD Program): \vspace{-0.1cm}
				\begin{itemize} \itemsep -1pt
				\item \url{http://exchanges.state.gov/academicexchanges/guep.html}
				\item The Global Undergraduate Exchange Program (also known as the Global UGRAD Program) provides one semester and academic year scholarships to outstanding undergraduate students from underrepresented sectors in East Asia, Eurasia and Central Asia, the Near East and South Asia and the Western Hemisphere for non-degree full-time study combined with community service, internships and cultural enrichment.
				\end{itemize}
			\item Professors and Research Scholars: \url{http://exchanges.state.gov/jexchanges/programs/professor.html}
			\item Short-Term Scholar: \url{http://exchanges.state.gov/jexchanges/programs/shortterm.html}
			\item Student, College/University: \vspace{-0.1cm}
				\begin{itemize} \itemsep -1pt
				\item \url{http://exchanges.state.gov/jexchanges/programs/ucstudent.html}
				\item The College/University Student Program gives foreign students the opportunity to study at an American degree-granting post-secondary accredited educational institution, including colleges and universities. Students may participate in degree and non-degree programs. They must pursue a full-time course of study and maintain satisfactory advancement toward the completion of their academic program.
				\end{itemize}
			\item Study of the United States Institutes for Scholars: \vspace{-0.1cm}
				\begin{itemize} \itemsep -1pt
				\item Study of the United States Institutes for Scholars  are designed to strengthen curricula and improve the quality of teaching about the United States in academic institutions overseas.
				\item Foreign university faculty, secondary educators and other scholars spend approximately four weeks at host universities where they take part in a series of lectures, seminar discussions and site visits related to each institute's theme.
				\item They learn about American educational philosophies, explore new teaching methods and pursue related research interests.
				\item Interests of these institutes: \vspace{-0.1cm}
					\begin{itemize} \itemsep -1pt
					\item American Politics and Political Thought
					\item Contemporary American Literature
					\item Journalism and Media
					\item Religious Pluralism in the United States
					\item Secondary School Educators
					\item U.S. Culture and Society
					\item U.S. Foreign Policy
					\item U.S. National Security
					\end{itemize}
				\item \url{http://exchanges.state.gov/academicexchanges/scholars.html}
				\end{itemize}
			\item Study of the United States Institutes for Student Leaders: \vspace{-0.1cm}
				\begin{itemize} \itemsep -1pt
				\item Study of the United States Institutes for Student Leaders are five-to-six-week academic programs for foreign undergraduate leaders.
				\item Hosted by U.S. academic institutions throughout the United States, the Student Leader Institutes include an intensive academic component, an educational tour of other regions of the country, local community service activities and a unique opportunity for participants to get to know their American peers.
				\item \url{http://exchanges.state.gov/academicexchanges/students.html}
				\item Interests of the institutes: \vspace{-0.1cm}
					\begin{itemize} \itemsep -1pt
					\item Comparative {\bf Public Policy} for Pakistani Student Leaders
					\item Energy and the Environment
					\item Global Environmental Issues
					\item New Media
					\item Religious Pluralism in the U.S.
					\item Social Entrepreneurship
					\item U.S. Foreign Policy for East Asian Student Leaders
					\item Western Hemisphere Student Leaders 
					\item Women's Leadership
					\end{itemize}
				\end{itemize}
			\item Edmund S. Muskie Graduate Fellowship: \vspace{-0.1cm}
				\begin{itemize} \itemsep -1pt
				\item \url{http://exchanges.state.gov/academicexchanges/muskie.html}
				\item The Edmund S. Muskie Graduate Fellowship Program (Muskie) confers fellowships for Master's degree-level study in the U.S. in the fields of business administration, economics, education, environmental policy and management, international affairs, journalism/mass communications, law, library and information science, public administration, public health and {\bf public policy} for students and professionals from Eurasia.
				\item Candidates are recruited through a merit-based competition administered by the International Research \& Exchanges Board (IREX).
				\item U.S. host campuses are also selected through a competition process and generally provide tuition waivers of fifty percent.
				\item Approximately 145 fellowships are awarded each academic year.
				\end{itemize}
			\item Critical Language Scholarship Program: \vspace{-0.1cm}
				\begin{itemize} \itemsep -1pt
				\item \url{http://exchanges.state.gov/academicexchanges/sli2.html}
				\item The Critical Language Scholarship (CLS) Program provides overseas foreign language instruction and cultural enrichment experiences in 13 critical need languages for U.S. students in higher education.
				\item The CLS Program is part of a U.S. government effort to expand dramatically the number of Americans studying and mastering critical need foreign languages.
				\item Undergraduate, master's and doctoral-level students of diverse disciplines and majors are encouraged to apply for the seven-to-10-week-long programs.
				\item Participants are expected to continue their language study beyond the scholarship period, and later apply their critical language skills in their future professional careers.
				\end{itemize}
			\item Critical Language Enhancement Award (CLEA): \vspace{-0.1cm}
				\begin{itemize} \itemsep -1pt
				\item \url{http://exchanges.state.gov/academicexchanges/clea2.html}
				\item The Critical Language Enhancement Award (CLEA) provides funding to eligible Fulbright U.S. Student Program Grantees who intend to use one of the following languages for their Fulbright project: \vspace{-0.1cm}
					\begin{itemize} \itemsep -1pt
					\item Arabic (all dialiects)
					\item Azeri
					\item Bangla/Bengali
					\item Bhasa Indonesia
					\item Chinese (Mandarin Only)
					\item Farsi
					\item Gujarati
					\item Hindi
					\item Korean
					\item Marathi
					\item Pashto
					\item Punjabi
					\item Russian
					\item Turkish
					\item Urdu
					\end{itemize}
				\end{itemize}
			\end{enumerate}
		\item Office of International Visitors: \vspace{-0.1cm}
			\begin{enumerate} \itemsep -1pt
			\item International Visitor Leadership Program (IVLP): \vspace{-0.1cm}
				\begin{itemize} \itemsep -1pt
				\item \url{http://exchanges.state.gov/ivlp/index.html}
				\item \url{http://exchanges.state.gov/ivlp/ivlp.html}
				\item The Office of International Visitors manages and funds the International Visitor Leadership Program (IVLP).
				\item Launched in 1940, the IVLP is a professional exchange program that seeks to build mutual understanding between the U.S. and other nations through carefully designed short-term visits to the U.S. for current and emerging foreign leaders.
				\item These visits reflect the International Visitors' professional interests and support the foreign policy goals of the United States.
				\end{itemize}
			\end{enumerate}
		\item Program Search (find international exchange programs sponsored by the Bureau of Educational and Cultural Affairs): \url{http://exchanges.state.gov/index/search.html}
		\end{enumerate}
	\end{enumerate}
\item Mexican American Legal Defense and Educational Fund (MALDEF): \vspace{-0.3cm}
	\begin{enumerate} \itemsep -2pt
	\item Scholarship Resources: \url{http://maldef.org/leadership/scholarships/}
	\item MALDEF Law School Scholarship Program: \vspace{-0.2cm}
		\begin{enumerate} \itemsep -2pt
		\item MALDEF's Law School Scholarship Program provides several scholarships in varying amounts to deserving law students with a commitment to advancing the civil rights of Latinos.
		\item MALDEF's Law School Scholarship Program is open to all law students who will be enrolled full-time in an American-accredited law school in 2010-2011.
		\item Scholarships are awarded to students based on their commitment to serve the Latino community through law; their past achievement and potential for achievement; and their financial need.
		\item \url{http://maldef.org/leadership/scholarships/law_school_scholarship_program/index.html}
		\end{enumerate}
	\item Undergraduate Scholarship Resource Guide: \url{http://maldef.org/leadership/scholarships/resources/index.html}
	\end{enumerate}
\item Ashoka: \vspace{-0.3cm}
	\begin{enumerate} \itemsep -2pt
	\item Ashoka Fellows (to promote and support social entrepreneurship): \url{http://www.ashoka.org/fellows}
	\end{enumerate}
\item Heinz Family Foundation: \vspace{-0.3cm}
	\begin{enumerate} \itemsep -2pt
	\item Heinz Award Criteria: \vspace{-0.2cm}
		\begin{enumerate} \itemsep -2pt
		\item \url{http://heinzawards.net/awards/criteria}
		\item The Heinz Endowments
		\item Attributes and qualities of awardees: \vspace{-0.1cm}
			\begin{enumerate} \itemsep -1pt
			\item an enormous capacity to love
			\item smile
			\item take risks
			\item question
			\item work hard
			\item believe in the power of the individual to improve the lives of others
			\end{enumerate}
		\item ``Candidates [should] possess a remarkable mix of vision, optimism, creativity and hard work which, when combined, produce tangible achievements of lasting good.''
		\item Nominees must exhibit the following personal characteristics: \vspace{-0.1cm}
			\begin{enumerate} \itemsep -1pt
			\item A passion for excellence that goes beyond intellectual curiosity;
			\item A concern for humanity rooted in a deep sensitivity for the well-being of others; 
			\item A knowledge of self which acknowledges weaknesses but relies on individual strengths;
			\item A gritty determination that will see a job through to completion despite the inevitable setbacks;
			\item A broad vision which extends far beyond the particular and embraces something universal.
			\end{enumerate}
		\item Work of the candidates for a Heinz Award must meet the following criteria: \vspace{-0.1cm}
			\begin{enumerate} \itemsep -1pt
			\item Be significant and not a ``quick fix.''
			\item Have an enduring and meaningful impact.
			\item Be creative and innovative, and
			\item Be sufficiently tangible to serve as a model for replication elsewhere.
			\end{enumerate}
		\item ``In addition, candidates should be actively working in the field in which they are nominated with the hope that, in receiving this award, their potential for future societal contribution will be enhanced.''
		\end{enumerate}
	\item Categories: \vspace{-0.2cm}
		\begin{enumerate} \itemsep -2pt
		\item Arts \& Humanities
		\item Environment
		\item Human Condition
		\item {\bf Public Policy}
		\item Technology, Economy, + Employment
		\end{enumerate}
	\end{enumerate}
\item Echoing Green: \vspace{-0.3cm}
	\begin{enumerate} \itemsep -2pt
	\item Echoing Green Fellowship: \vspace{-0.2cm}
		\begin{enumerate} \itemsep -2pt
		\item \url{http://www.echoinggreen.org/fellowship}
		\item Has information on eligibility, the benefits of the fellowship, and application cycle and dates.
		\end{enumerate}
	\item Echoing Green Fellows: \url{http://www.echoinggreen.org/fellows}
	\end{enumerate}
\item Ben Franklin Technology Partners (BFTP): \vspace{-0.3cm}
	\begin{enumerate} \itemsep -2pt
	\item Innovation Works (IW): \vspace{-0.2cm}
		\begin{enumerate} \itemsep -2pt
		\item AlphaLab: \vspace{-0.1cm}
			\begin{enumerate} \itemsep -1pt
			\item ``An immersive environment where entrepreneurs can tap IW's onsite experts for business and market advice and exchange ideas with other entrepreneurs launching in similar markets''
			\end{enumerate}
		\end{enumerate}
	\end{enumerate}
\item Carnegie Corporation of New York: \vspace{-0.3cm}
	\begin{enumerate} \itemsep -2pt
	\item Carnegie Scholars Program (not available in 2010): \url{http://carnegie.org/programs/carnegie-scholars/}
	\end{enumerate}
\item New York Women's Foundation: \vspace{-0.3cm}
	\begin{enumerate} \itemsep -2pt
	\item Finch Scholar Program (with the Finch College Alumnae Association): \vspace{-0.2cm}
		\begin{enumerate} \itemsep -2pt
		\item \url{http://www.nywf.org/internship.html} and \url{http://www.finchcollege.org/}
		\item ``Our partnership with the Finch Scholar Program allows us to provide practical community service experience to an outstanding local student enrolled in college. The internship affords the Finch Scholar opportunities to work in meaningful ways in a nonprofit organization with exposure to social change philanthropy, participatory grantmaking, advocacy and {\bf public policy}. Generally, we offer one scholarship per year with a stipend.''
		\item \url{http://www.finchcollege.org/newFinchScholarPrgm.html}
		\item \url{http://www.finchcollege.org/newscholarships.html}
		\end{enumerate}
	\end{enumerate}
\item The Rockefeller Foundation: \vspace{-0.3cm}
	\begin{enumerate} \itemsep -2pt
	\item The Bellagio Center: \vspace{-0.2cm}
		\begin{enumerate} \itemsep -2pt
		\item \url{http://www.rockefellerfoundation.org/bellagio-center}
		\item Residency Programs: \vspace{-0.1cm}
			\begin{enumerate} \itemsep -1pt
			\item \url{http://www.rockefellerfoundation.org/bellagio-center/residency-programs}
			\item ``The Bellagio Residency program offers scholars, artists, thought leaders, policymakers and practitioners a serene setting conducive to focused, goal-oriented work, and the unparalleled opportunity to establish new connections with fellow residents, across a stimulating array of disciplines and geographies.  The Bellagio Center community generates new knowledge to solve some of the most complex problems facing our world and creates art that inspires reflection, understanding, and imagination.''
			\item Scholarly Residencies: \vspace{-0.1cm}
				\begin{itemize} \itemsep -1pt
				\item ``Researchers in the humanities, natural sciences, social sciences and other academic disciplines''
				\item ``The Center typically offers one-month residencies for no more than 12 scholars and scientists at a time. Individuals in any discipline and from any part of the world are welcome to apply. The Center maintains a core focus on projects consistent with the Foundation's mission to expand opportunities for poor or vulnerable people and to help see that the benefits of globalization are shared more widely. It also seeks to include beyond that core a wide variety of projects from all academic disciplines.''
				\item \url{http://www.rockefellerfoundation.org/bellagio-center/residency-programs/scholarly-residencies}
				\end{itemize}
			\item Creative Artist Residencies: \vspace{-0.1cm}
				\begin{itemize} \itemsep -1pt
				\item ``Artists, composers, writers''
				\item ``Bellagio creative artist residencies for composers, novelists, playwrights, poets, video/filmmakers and visual artists provide time for disciplined work, individual reflection, and collegial engagement, uninterrupted by the usual professional and personal demands. The Center typically offers one-month stays for no more than three to five creative artists at a time. Artists of significant achievement from any country are welcome to apply.''
				\item \url{http://www.rockefellerfoundation.org/bellagio-center/residency-programs/creative-artist-residencies}
				\end{itemize}
			\item Practitioner Residencies: \vspace{-0.1cm}
				\begin{itemize} \itemsep -1pt
				\item ``Policymakers, nonprofit leaders, journalists and public advocates''
				\item ``The Center offers residencies to professionals in fields relevant to the Rockefeller Foundation's issue areas. The Center maintains a core focus on projects consistent with our mission, to expand opportunities for poor or vulnerable people and to help see that the benefits of globalization are shared more widely.   We seek practitioner applicants with demonstrated leadership qualities and the capacity to contribute to the intellectual life at the Center.''
				\item \url{http://www.rockefellerfoundation.org/bellagio-center/residency-programs/practitioner-residencies}
				\end{itemize}
			\end{enumerate}
		\item {\bf \color{blue} Creative Arts Fellowships}: \vspace{-0.1cm}
			\begin{enumerate} \itemsep -1pt
			\item ``This high-profile program hosts visual artists at the Bellagio Center for three-month residencies that inspire creativity and promote interaction between the arts and other fields. Creative Arts Fellows, like other participants in Bellagio residency programs, have the time and space to work independently during the day. They also enjoy and benefit from a lively community of scholars, writers, policymakers and other artists who gather in the evening for dinner and occasional presentations.  The combination of private work space, an extended stay, a generous stipend and a unique group of fellow residents makes a Creative Arts Fellowship at the Bellagio Center a remarkable opportunity.''
			\item \url{http://www.rockefellerfoundation.org/bellagio-center/creative-arts-fellowships}
			\end{enumerate}
		\end{enumerate}
	\end{enumerate}
\item Wellcome Trust: \vspace{-0.3cm}
	\begin{enumerate} \itemsep -2pt
	\item Wellcome Trust Book Prize: \vspace{-0.2cm}
		\begin{enumerate} \itemsep -2pt
		\item \url{http://www.wellcomebookprize.org/About-the-prize/index.htm}
		\item ``The Wellcome Trust Book Prize celebrates the best of medicine in literature by awarding 25 000 each year for the finest fiction or non-fiction book centered around medicine.''
		\end{enumerate}
	\end{enumerate}
\item The Kennedy Memorial Trust: \vspace{-0.3cm}
	\begin{enumerate} \itemsep -2pt
	\item \url{http://www.kennedytrust.org.uk/}
	\item Kennedy Scholarship: \url{http://www.kennedytrust.org.uk/display.aspx?Id=1165&pid=0}
	\item Frank Knox Fellowships: \url{http://www.kennedytrust.org.uk/display.aspx?Id=1175&pid=0}
	\end{enumerate}
\item Foreign \& Commonwealth Office / United Kingdom: \vspace{-0.3cm}
	\begin{enumerate} \itemsep -2pt
	\item Chevening scholarships: \vspace{-0.2cm}
		\begin{enumerate} \itemsep -2pt
		\item \url{http://www.fco.gov.uk/en/about-us/what-we-do/scholarships/}
		\item ``The Chevening programme, has, over 26 years, provided more than 30,000 Scholarships at Higher Education Institutions (HEIs) in the UK for postgraduate students or researchers from countries across the world.''
		\end{enumerate}
	\item {\bf Marshall Scholarships} finance young Americans of high ability to study for a graduate degree in the United Kingdom: \url{http://www.marshallscholarship.org/}
	\end{enumerate}
\item Ministry of Education, Culture, Sports, Science and Technology (MEXT) / Japan: \vspace{-0.3cm}
	\begin{enumerate} \itemsep -2pt
	\item \url{http://www.mext.go.jp/english/}
	\item Monbukagakusho Scholarship: \vspace{-0.2cm}
		\begin{enumerate} \itemsep -2pt
		\item \url{http://en.wikipedia.org/wiki/Monbukagakusho_Scholarship}
		\item \url{http://project.monbusho.org/old/} and \url{http://www.monbusho.org/}
		\end{enumerate}
	\end{enumerate}
\item Institute of International Education (IIE): \vspace{-0.3cm}
	\begin{enumerate} \itemsep -2pt
	\item GE Foundation Scholar-Leaders Program: \vspace{-0.2cm}
		\begin{enumerate} \itemsep -2pt
		\item \url{http://www.iie.org/en/Programs/GE-Foundation-Scholar-Leaders-Program}
		\item ``The GE Foundation Scholar-Leaders Program began in 1987 in Mexico and now supports outstanding students in higher education in fourteen countries around the world. The program initially provided traditional financial support for university education, but has developed into an exciting Leadership Development Program to complement the student's academic curriculum.''
		\item Eligibility: ``Students in their first year of study in engineering, technology, business, finance, management, or economics attending a participating university. GE Foundation Scholar-Leaders qualification requirements vary by region.''
		\end{enumerate}
	\end{enumerate}
\item British Council: \vspace{-0.3cm}
	\begin{enumerate} \itemsep -2pt
	\item Shine! 2011: International Student Awards: \vspace{-0.2cm}
		\begin{enumerate} \itemsep -2pt
		\item \url{http://www.educationuk.org/shine}
		\item For international students in the United Kingdom
		\end{enumerate}
	\item Funding your studies: \vspace{-0.2cm}
		\begin{enumerate} \itemsep -2pt
		\item \url{http://www.britishcouncil.org/learning-funding-your-studies.htm}
		\item Education UK: \url{http://www.educationuk.org/pls/hot_bc/page_pls_user_advice?x=&y=&a=0&d=4460}
		\item 9/11 Scholarship Fund: \vspace{-0.1cm}
			\begin{enumerate} \itemsep -1pt
			\item \url{http://www.britishcouncil.org/911scholarships.htm}
			\item ``The 9/11 Scholarship Fund supports international students who were directly affected by the 2001 terrorist events in the US. Find out more how each scholarship offers the opportunity to study at a UK college or university every year.''
			\end{enumerate}
		\end{enumerate}
	\item {\it Youth in Action} European program: \url{http://www.britishcouncil.org/youthinaction}
	\item British Council Arts Group: \vspace{-0.2cm}
		\begin{enumerate} \itemsep -2pt
		\item Support and funding overview: \url{http://www.britishcouncil.org/arts-support-and-funding-overview.htm}
		\item Visual arts support and funding: \url{http://www.britishcouncil.org/arts-visual-arts-funding.htm}
		\item Drama and dance support and funding: \url{http://www.britishcouncil.org/arts-performing-arts-funding.htm}
		\item Literature support and funding: \url{http://www.britishcouncil.org/arts-literature-support-and-funding.htm}
		\item Film support and funding: \url{http://www.britishcouncil.org/arts-film-funding.htm}
		\item Music support and funding: \url{http://www.britishcouncil.org/arts-music-funding.htm}
		\item Architecture, design, fashion support and funding: \url{http://www.britishcouncil.org/arts-adf-funding.htm}
		\item International Short Film Festival Support Scheme: \url{http://www.britishcouncil.org/arts-film-short-films-scheme.htm}
		\end{enumerate}
	\end{enumerate}
\item Alfred P. Sloan Foundation: \vspace{-0.3cm}
	\begin{enumerate} \itemsep -2pt
	\item Sloan Research Fellowships: \vspace{-0.2cm}
		\begin{enumerate} \itemsep -2pt
		\item \url{http://www.sloan.org/fellowships}
		\item Hold a Ph.D. (or equivalent) in chemistry, physics, mathematics, computer science, economics, neuroscience or computational and evolutionary molecular biology, or in a related interdisciplinary field;
		\item Be members of the regular faculty (i.e., tenure track) of a degree-granting college or university in the United States or Canada; and
		\item Normally, be no more than six years from completion of the most recent Ph.D. or equivalent as of the year of their nomination.
		\end{enumerate}
	\end{enumerate}
\item --- --- --- --- --- --- --- --- --- --- --- --- --- --- --- --- --- --- --- --- --- --- --- --- --- --- --- --- --- --- ---
\item \colorbox{blue}{\bf Scholarships and Fellowships in Business (including Finance, Entrepreneurship, and Accounting)}
% Scholarships and Fellowships in Business (including Finance, Entrepreneurship, and Accounting)
\item IREX: \vspace{-0.3cm}
	\begin{enumerate} \itemsep -2pt
	\item Opportunities ``for individuals, organizations, universities, and alumni'': \url{http://www.irex.org/apply}
	\item Edmund S. Muskie Graduate Fellowship Program: \vspace{-0.2cm}
		\begin{enumerate} \itemsep -2pt
		\item : \url{http://www.irex.org/application/edmund-s-muskie-graduate-fellowship-program-application}
		\item ``The Muskie Program is open to graduate students and professionals from Armenia, Azerbaijan, Belarus, Georgia, Kazakhstan, Kyrgyzstan, Moldova, Russia, Tajikistan, Turkmenistan, Ukraine and Uzbekistan for one-year non-degree, one-year degree, or two-year degree study in the United States.''
		\item ``Eligible fields of study for the Muskie Program are: business administration, economics, education, environmental management, international affairs, journalism and mass communication, law, library and information science, public administration, public health, and {\bf public policy}.''
		\end{enumerate}
	\end{enumerate}
\item Sponsors for Educational Opportunity (SEO): \vspace{-0.3cm}
	\begin{enumerate} \itemsep -2pt
	\item Alternative Investment Fellowship Program: \vspace{-0.2cm}
		\begin{enumerate} \itemsep -2pt
		\item \url{http://www.seo-usa.org/Fellowship}
		\item Eligibility: \vspace{-0.1cm}
			\begin{enumerate} \itemsep -1pt
			\item \url{http://www.seo-usa.org/FellowshipEligibility}
			\item The program is open to professionals traditionally underrepresented in alternative investments who are in the first year (or second year with a third-year offer) of an analyst program at an investment bank.
			\item Corporate finance, M\&A, leveraged finance and structured finance analysts are preferred.
			\item Management consultants will also be considered.
			\end{enumerate}
		\end{enumerate}
	\item The SEO Scholars Program: \vspace{-0.2cm}
		\begin{enumerate} \itemsep -2pt
		\item \url{http://www.seo-usa.org/Scholars}
		\item The SEO Scholars Program is a rigorous out-of-school academic enrichment program that prepares motivated New York City public high school students of color to gain admission to and succeed at competitive colleges and universities throughout the country.  Numerous studies confirm that rigorous academics are the single most important factor for low-income and minority students in gaining college admission and earning a degree.  However, U.S. Department of Education research shows that ``A'' work in low-income schools equals ``C'' work in affluent schools.
		\item Admissions: \url{http://www.seo-usa.org/ScholarsAdmissions}
		\item Roadmap To Success: \url{http://www.seo-usa.org/ScholarsRoadmapToSuccess}
		\item Enrichment Programs: \url{http://www.seo-usa.org/ScholarsEnrichmentPrograms}
		\item Volunteering: \url{http://www.seo-usa.org/ScholarsVolunteering}
		\item Andrew Golkin Fund: \vspace{-0.1cm}
			\begin{enumerate} \itemsep -1pt
			\item \url{http://www.seo-usa.org/ScholarsAndrewGolkinFund}
			\item \url{http://www.seo-usa.org/andrewgolkinfund/index.html}
			\end{enumerate}
		\item Franklin H. and Shirley B. Williams Scholarship Fund: \url{http://www.seo-usa.org/ScholarsFHSBW}
		\item The Advantages of Attending a Competitive College: \url{http://www.seo-usa.org/ScholarsAdvantages}
		\end{enumerate}
	\item Career program: \vspace{-0.2cm}
		\begin{enumerate} \itemsep -2pt
		\item \url{http://www.seo-usa.org/Career}
		\item The SEO Career Program places students of color interested in finance, philanthropy, business and corporate law in internships with competitive pay, rigorous training, support through mentors, and broad access to industry professionals. 
		\item Sponsors for Educational Opportunity (SEO) is the nation's premiere summer internship program for talented underrepresented students of color that can lead to full-time job offers.
		\item SEO offers internship opportunities in the following areas: \vspace{-0.1cm}
			\begin{enumerate} \itemsep -1pt
			\item Corporate Financial Leadership: \url{http://www.seo-usa.org/Career/Corporate_Financial_Leadership}
			\item Banking/Asset Management Areas: \vspace{-0.1cm}
				\begin{itemize} \itemsep -1pt
				\item Investment Banking: \url{http://www.seo-usa.org/Career/Investment_Banking}
				\item Sales \& Trading: \url{http://www.seo-usa.org/Career/Sales_&_Trading}
				\item Investment Research: \url{http://www.seo-usa.org/Career/Investment_Research}
				\item Transaction Services: \url{http://www.seo-usa.org/Career/Transaction_Services}
				\item Asset Management: \url{http://www.seo-usa.org/Career/Asset_Management}
				\item Accounting/Finance: \url{http://www.seo-usa.org/Career/Accounting/Finance}
				\item Information Technology: \url{http://www.seo-usa.org/Career/Information_Technology}
				\end{itemize}
			\item Corporate Law: \url{http://www.seo-usa.org/Career/Corporate_Law}
			\item Nonprofit: \url{http://www.seo-usa.org/Career/Nonprofit}
			\item SEO-U: Freshmen and Sophomore Training: \url{http://www.seo-usa.org/Career/SEO-U:Freshmen_&_Sophomore_Training}
			\end{enumerate}
		\item Application Deadlines: \url{http://www.seo-usa.org/CareerApplicationDeadlines}
		\item Eligibility Information: \url{http://www.seo-usa.org/CareerEligibilityInfo}
		\item Application Tips: \url{http://www.seo-usa.org/CareerApplicationTips}
		\item Interview Tips: \url{http://www.seo-usa.org/CareerInterviewTips}
		\end{enumerate}
	\end{enumerate}
\item --- --- --- --- --- --- --- --- --- --- --- --- --- --- --- --- --- --- --- --- --- --- --- --- --- --- --- --- --- --- ---
\item \colorbox{blue}{\bf Scholarships for Studying Abroad}
% Scholarships for Studying Abroad
\item U.S. Department of State: \vspace{-0.3cm}
	\begin{enumerate} \itemsep -2pt
	\item Bureau of Educational and Cultural Affairs: \vspace{-0.2cm}
		\begin{enumerate} \itemsep -2pt
		\item Benjamin A. Gilman International Scholarship: \vspace{-0.1cm}
			\begin{enumerate} \itemsep -1pt
			\item \url{http://exchanges.state.gov/globalexchanges/gilman-scholarship-program.html}
			\item ``The Benjamin A. Gilman International Scholarship Program provides scholarships to U.S. undergraduates with financial need for study abroad, including students from diverse backgrounds and students going to non-traditional study abroad destinations.  Established under the International Academic Opportunity Act of 2000, Gilman Scholarships provide up to \$5,000 for American students to pursue overseas study for college credit.''
			\item Critical Need Languages: Students studying critical need languages are eligible for up to \$3,000 in additional funding as part of the Gilman Critical Need Language Supplement program. Those critical need languages include: \vspace{-0.1cm}
				\begin{itemize} \itemsep -1pt
				\item Arabic
				\item Chinese
				\item Korean
				\item Russian
				\item Turkic (Azerbaijani, Kazakh, Kyrgyz, Turkish, Turkmen, Uzbek)
				\item Persian (Farsi, Dari, Kurdish, Pashto, Tajiki)
				\item Indic (Hindi, Urdu, Nepali, Sinhala, Bengali, Punjabi, Marathi, Gujurati, Sindhi)
				\end{itemize}
			\item \url{http://www.iie.org/en/Programs/Gilman-Scholarship-Program}
			\item \url{http://www.iie.org/en/Programs/Gilman-Scholarship-Program/About-the-Program}
			\end{enumerate}
		\end{enumerate}
	\end{enumerate}
\item Council on International Educational Exchange (CIEE): \vspace{-0.3cm}
	\begin{enumerate} \itemsep -2pt
	\item CIEE Scholarships: \url{http://www.ciee.org/study/scholarships/index.aspx}
	\end{enumerate}
\item IES Abroad (formerly Institute of European Studies / Institute for the International Education of Students): \vspace{-0.3cm}
	\begin{enumerate} \itemsep -2pt
	\item Scholarships and Financial Aid: \url{https://www.iesabroad.org/IES/Scholarships_and_Aid/financialAid.html}
	\item IES Abroad Need-Based Financial Aid: \url{https://www.iesabroad.org/IES/Scholarships_and_Aid/Need-Based/needBasedFinancialAid.html}
	\item IES Abroad Merit-Based Scholarships: \url{https://www.iesabroad.org/IES/Scholarships_and_Aid/Merit_Based/meritBasedFinancialAid.html}
	\item IES Abroad Public University Grants: \url{https://www.iesabroad.org/IES/Scholarships_and_Aid/publicScholarship.html}
	\end{enumerate}
\item American Institute For Foreign Study (AIFS): \vspace{-0.3cm}
	\begin{enumerate} \itemsep -2pt
	\item AIFS Study Abroad Programs: \vspace{-0.2cm}
		\begin{enumerate} \itemsep -2pt
		\item \url{http://www.aifsabroad.com/programs.asp}
		\item AIFS Study Abroad Scholarships: \url{http://www.aifsabroad.com/scholarships.asp}
		\end{enumerate}
	\end{enumerate}
\item --- --- --- --- --- --- --- --- --- --- --- --- --- --- --- --- --- --- --- --- --- --- --- --- --- --- --- --- --- --- ---
\item \colorbox{blue}{\bf Scholarships and Fellowships in Public Policy and Public Health}
% Scholarships and Fellowships in Public Policy and Public Health
\item The Commonwealth Fund: \vspace{-0.3cm}
	\begin{enumerate} \itemsep -2pt
	\item Commonwealth Fund fellowship programs: \vspace{-0.2cm}
		\begin{enumerate} \itemsep -2pt
		\item \url{http://www.commonwealthfund.org/Fellowships.aspx}
		\item ``Commonwealth Fund fellowship programs are designed to give promising young researchers the opportunity for in-depth study of various health care policy topics, working with investigators, policy analysts, government officials, and others in a number of U.S. and international settings.''
		\item The Commonwealth Fund/Harvard University Fellowship in Minority Health Policy: \url{http://www.commonwealthfund.org/Fellowships/Minority-Health-Policy-Fellowship.aspx}
		\item Harkness Fellowships in Health Care Policy and Practice: \url{http://www.commonwealthfund.org/Fellowships/Harkness-Fellowships.aspx}
		\item Australian-American Health Policy Fellowship: \url{http://www.commonwealthfund.org/Fellowships/Australian-American-Health-Policy-Fellowships.aspx}
		\item Ian Axford (New Zealand) Fellowships in Public Policy: \url{http://www.commonwealthfund.org/Fellowships/Ian-Axford-Fellowships.aspx}
		\end{enumerate}
	\end{enumerate}
\item American Institute of Aeronautics and Astronautics (AIAA): \vspace{-0.3cm}
	\begin{enumerate} \itemsep -2pt
	\item Federal Government Fellows Program: \vspace{-0.2cm}
		\begin{enumerate} \itemsep -2pt
		\item \url{http://www.aiaa.org/content.cfm?pageid=731}
		\item Shaping U.S. {\bf public policy} concerning aerospace research and the aerospace industry
		\end{enumerate}
	\end{enumerate}
\item IEEE-USA: \vspace{-0.3cm}
	\begin{enumerate} \itemsep -2pt
	\item Congressional Fellowship
	\item Engineering \& Diplomacy (State Department) Fellowship
	\item For IEEE-USA members to support the creation and modification of technology-related public policies
	\item \url{http://ieeeusa.org/policy/govfel/default.asp}
	\end{enumerate}
\item American Mathematical Society: \vspace{-0.3cm}
	\begin{enumerate} \itemsep -2pt
	\item Fellowships and Awards (Policy and Advocacy: Government Relations \& Programs): \vspace{-0.2cm}
		\begin{enumerate} \itemsep -2pt
		\item \url{http://e-math.ams.org/policy/government/fellow-awards/fellow-awards}
		\item Mass Media Fellowships: \url{http://e-math.ams.org/programs/ams-fellowships/media-fellow/massmediafellow}
		\item AMS-AAAS Congressional Fellowship: \url{http://e-math.ams.org/programs/ams-fellowships/ams-aaas/ams-aaas-congressional-fellowship}
		\end{enumerate}
	\end{enumerate}
\item American Association for the Advancement of Science: \vspace{-0.3cm}
	\begin{enumerate} \itemsep -2pt
	\item AAAS Science \& Technology Policy Fellowships: \url{http://fellowships.aaas.org/index.shtml}
	\end{enumerate}
\item --- --- --- --- --- --- --- --- --- --- --- --- --- --- --- --- --- --- --- --- --- --- --- --- --- --- --- --- --- --- ---
\item \colorbox{blue}{\bf Scholarships and Fellowships in Social Science and Humanities}
% Scholarships and Fellowships in Social Science and Humanities
\item United States Institute of Peace (USIP): \vspace{-0.3cm}
	\begin{enumerate} \itemsep -2pt
	\item Jennings Randolph Peace Scholarship Dissertation Program (for Ph.D. students working on topics related to peace, conflict, and international security): \url{http://www.usip.org/grants-fellowships/jennings-randolph-peace-scholarship-dissertation-program}
	\end{enumerate}
\item Library of Congress: \vspace{-0.3cm}
	\begin{enumerate} \itemsep -2pt
	\item Kluge Fellowships: \vspace{-0.2cm}
		\begin{enumerate} \itemsep -2pt
		\item Research in the humanities and social sciences, especially interdisciplinary, cross-cultural or multilingual
		\item Open to scholars worldwide with a Ph.D. or other terminal advanced degree conferred within seven years of the July 15 deadline
		\item \url{http://www.loc.gov/loc/kluge/fellowships/kluge.html}
		\end{enumerate}
	\item J. Franklin Jameson Fellowship Research in American History (junior postdocs): \url{http://www.loc.gov/loc/kluge/fellowships/jameson.html}
	\item Kislak Short Term Fellowship Opportunities in American Studies (students, postdocs, and faculty): \url{http://www.loc.gov/loc/kluge/fellowships/kislakshort.html}
	\item Kislak Fellowship in American Studies (Ph.D. requirement): \url{http://www.loc.gov/loc/kluge/fellowships/kislak.html}
	\end{enumerate}
\item American Historical Association (AHA): \vspace{-0.3cm}
	\begin{enumerate} \itemsep -2pt
	\item AHA Research Grants: \url{http://www.historians.org/prizes/Grants.htm}
	\item Fellowships: \url{http://www.historians.org/prizes/Fellowships.htm}
	\end{enumerate}
\item American Sociological Association: \vspace{-0.3cm}
	\begin{enumerate} \itemsep -2pt
	\item ASA Dissertation Award: \url{http://www.asanet.org/about/awards/dissertation.cfm}
	\end{enumerate}
\item American Psychological Association: \vspace{-0.3cm}
	\begin{enumerate} \itemsep -2pt
	\item Scholarships, Grants, and Awards: \url{http://www.apa.org/about/awards/index.aspx}
	\end{enumerate}
\item American Anthropological Association (AAA): \vspace{-0.3cm}
	\begin{enumerate} \itemsep -2pt
	\item AAA Minority Dissertation Fellowship Program (for minority Ph.D. candidates in anthropology): \url{http://www.aaanet.org/cmtes/minority/Minfellow.cfm}
	\item Margaret Mead Award (for young scholars in anthropology): \url{http://www.aaanet.org/about/Prizes-Awards/AAA-Margaret-Mead-Award.cfm}
	\item COSWA Award: \vspace{-0.2cm}
		\begin{enumerate} \itemsep -2pt
		\item The COSWA Award (formerly the Squeaky Wheel Award), sponsored by the Committee on the Status of Women in Anthropology (COSWA), recognizes individuals who have demonstrated the courage to bring to light and investigate practices in anthropology that are potentially discriminatory to women, or have acted to improve the status of women in anthropology through activities that raise awareness of women's contribution to anthropology or identify barriers to full participation by women in anthropology.
		\item \url{http://www.aaanet.org/about/Prizes-Awards/COSWA-Award.cfm}
		\end{enumerate}
	\item David M. Schneider Award (for Ph.D. students in anthropology): \url{http://www.aaanet.org/about/Prizes-Awards/David-Schneider-Award.cfm}
	\item Links to ``Section Prizes \& Awards'': \url{http://www.aaanet.org/about/Prizes-Awards/section_awards.cfm}
	\item List of national (US) and international ``Grants and Fellowships'': \url{http://www.aaanet.org/profdev/fellowships/}
	\item \url{http://www.aaanet.org/}
	\end{enumerate}
\item National Academy of Social Insurance: \vspace{-0.3cm}
	\begin{enumerate} \itemsep -2pt
	\item John Heinz Dissertation Award (Ph.D. students writing their thesis on the planning and implementation of social insurance): \url{http://www.nasi.org/studentopps/heinz}
	\end{enumerate}
\item National Endowment for the Humanities's Division of Research Programs, grants and fellowship opportunities: \url{http://www.neh.gov/grants/}
\item {\it The Henry Luce Foundation}'s Luce Scholars Program to help US graduates learn more about Asia and Asian culture(s): \url{http://www.hluce.org/lsprogram.aspx}
\item Institute for Humane Studies at George Mason University: \vspace{-0.3cm}
	\begin{enumerate} \itemsep -2pt
	\item Humane Studies Fellowships: \vspace{-0.2cm}
		\begin{enumerate} \itemsep -2pt
		\item \url{http://www.theihs.org/programs/humane-studies-fellowships}
		\item Humane Studies Fellowships are awarded to graduate students and outstanding undergraduates planning academic careers with liberty-advancing research interests.
		\item The fellowships are open to students in a range of fields, such as economics, philosophy, law, political science, anthropology, and literature.
		\end{enumerate}
	\end{enumerate}
\item The Gilder Lehrman Institute of American History: Gilder Lehrman History Scholars \& Gilder Lehrman One-Week Scholars (for sophomores or juniors majoring in American history or American Studies), \url{http://www.gilderlehrman.org/education/hs_program_details.php}
\item Myra Sadker Foundation: \vspace{-0.3cm}
	\begin{enumerate} \itemsep -2pt
	\item \url{http://www.sadker.org/awards.html}
	\item Teacher Award: Designed to promote and support teacher projects (K-12) that help students learn about and respect group differences, promote fairness, and in other ways build upon the values and contributions of Myra Sadker's work. Each project should have a gender dimension.
	\item Student Award: Designed to encourage student ideas, activities and projects (K-12) that promote respect for group differences, fairness, and in other ways build upon the values and contributions of Myra Sadker's work. Each project should have a gender dimension. 
	\item Doctoral Dissertation Award: Designed to promote and support graduate students engaged in educational equity research. Doctoral level dissertations that explore or promote educational equity and fairness based on gender, race, ethnicity, religion, class, sexual orientation, or other such variables will be considered for support. Each dissertation should have a gender dimension.
	\end{enumerate}
\item IREX: \vspace{-0.3cm}
	\begin{enumerate} \itemsep -2pt
	\item Opportunities ``for individuals, organizations, universities, and alumni'': \url{http://www.irex.org/apply}
	\item Edmund S. Muskie Graduate Fellowship Program: \vspace{-0.2cm}
		\begin{enumerate} \itemsep -2pt
		\item : \url{http://www.irex.org/application/edmund-s-muskie-graduate-fellowship-program-application}
		\item ``The Muskie Program is open to graduate students and professionals from Armenia, Azerbaijan, Belarus, Georgia, Kazakhstan, Kyrgyzstan, Moldova, Russia, Tajikistan, Turkmenistan, Ukraine and Uzbekistan for one-year non-degree, one-year degree, or two-year degree study in the United States.''
		\item ``Eligible fields of study for the Muskie Program are: business administration, economics, education, environmental management, international affairs, journalism and mass communication, law, library and information science, public administration, public health, and {\bf public policy}.''
		\end{enumerate}
	\item Legal Education and Development (LEAD) Fellowship: \vspace{-0.2cm}
		\begin{enumerate} \itemsep -2pt
		\item \url{http://www.irex.org/application/legal-education-and-development-lead-fellowship-application}
		\item Legal Education and Development Fellowship Program (LEAD) in Tajikistan
		\item Eligibility: \vspace{-0.1cm}
			\begin{enumerate} \itemsep -1pt
			\item Is a citizen, national, or permanent resident qualified to hold a valid passport issued by Tajikistan;
			\item Is the recipient of an undergraduate degree in law (four- or five-year study) by the time of the application;
			\item Is able to begin the academic exchange program in the United States in the summer of 2011;
			\item Is able to receive and maintain a United States J-1 visa.
			\end{enumerate}
		\end{enumerate}
	\item Community Solutions Program: \vspace{-0.2cm}
		\begin{enumerate} \itemsep -2pt
		\item \url{http://www.irex.org/application/community-solutions-information-applicants}
		\item ``a professional development program for the best and brightest global community leaders working in Transparency \& Accountability, Tolerance/Conflict Resolution, Environmental Issues, and Women's Issues''
		\item ``Competition for the Community Solutions Program is merit-based and open to community leaders, ages 25-38 at the time of application''
		\end{enumerate}
	\item Crimea Undergraduate Exchange Program (Crimea UGRAD) Application: \vspace{-0.2cm}
		\begin{enumerate} \itemsep -2pt
		\item \url{http://www.irex.org/application/crimea-undergraduate-exchange-program-crimea-ugrad-application}
		\item ``The Crimea UGRAD Program is open to undergraduate students from the Autonomous Republic of Crimea for one academic year of non-degree study in a US university or community college.''
		\end{enumerate}
	\end{enumerate}
\item {\it Demos}: \vspace{-0.3cm}
	\begin{enumerate} \itemsep -2pt
	\item The Ed Baker Fellowship in Democratic Values: \vspace{-0.2cm}
		\begin{enumerate} \itemsep -2pt
		\item \url{http://www.demos.org/edbakerfellowship.cfm}
		\item ``Based in our New York offices, Ed Baker Fellows will give voice to strong democratic values within a wide range of potential issues, including voting rights, citizen engagement, immigration policy and civic inclusion, campaign finance reform and money in politics, and media reform, among others.''
		\end{enumerate}
	\item Fellows Program: \vspace{-0.2cm}
		\begin{enumerate} \itemsep -2pt
		\item \url{http://www.demos.org/fellowsapp.cfm}
		\item \url{http://www.demos.org/program.cfm?currentprogramID=5A196E48-3FF4-6C82-50CBCA5825B661BA}
		\item ``The Fellows Program at Demos provides support and community for writers and thinkers with well-defined projects that aim to advance the values at the core of Demos' programs and mission: a robust and inclusive democracy; shared prosperity; strong \& effective public governance; and global interdependence.''
		\end{enumerate}
	\end{enumerate}
\item Research Councils UK (RCUK): \vspace{-0.3cm}
	\begin{enumerate} \itemsep -2pt
	\item Economic and Social Research Council (ESRC): \vspace{-0.2cm}
		\begin{enumerate} \itemsep -2pt
		\item Academic (funding opportunities for students, postdocs, and professors): \url{http://www.esrcsocietytoday.ac.uk/ESRCInfoCentre/index_academic.aspx}
		\item Professorial Fellowships (for leading senior social scientists): \url{http://www.esrcsocietytoday.ac.uk/ESRCInfoCentre/opportunities/professorial/}
		\item Funding opportunities: \vspace{-0.1cm}
			\begin{enumerate} \itemsep -1pt
			\item \url{http://www.esrcsocietytoday.ac.uk/ESRCInfoCentre/index_government.aspx}
			\item \url{http://www.esrcsocietytoday.ac.uk/ESRCInfoCentre/opportunities/}
			\item ESRC Research Funding Guide / ESRC's Funding Rules: \url{http://www.esrcsocietytoday.ac.uk/ESRCInfoCentre/opportunities/research_funding}
			\item Eligibility for Research Council Funding: \url{http://www.esrcsocietytoday.ac.uk/ESRCInfoCentre/opportunities/eligibility}
			\item Current Funding Opportunities: \url{http://www.esrcsocietytoday.ac.uk/ESRCInfoCentre/opportunities/current_funding_opportunities/}
			\item Forthcoming funding opportunities: \url{http://www.esrcsocietytoday.ac.uk/ESRCInfoCentre/opportunities/forthcoming_opportunities/}
			\item Placement Fellows Scheme: \url{http://www.esrcsocietytoday.ac.uk/ESRCInfoCentre/opportunities/placement/}
			\item Professorial Fellowships: \url{http://www.esrcsocietytoday.ac.uk/ESRCInfoCentre/opportunities/professorial/}
			\item Early Career Researchers (including Postdoctoral Fellowships, International Training, and Networking Opportunities): \url{http://www.esrcsocietytoday.ac.uk/ESRCInfoCentre/opportunities/earlycareer/}
			\item Postgraduate and Career Development Opportunities: \url{http://www.esrcsocietytoday.ac.uk/ESRCInfoCentre/opportunities/postgraduate/}
			\item International Funding Opportunities: \url{http://www.esrcsocietytoday.ac.uk/ESRCInfoCentre/opportunities/international/}
			\item Joint Funding Opportunities: \url{http://www.esrcsocietytoday.ac.uk/ESRCInfoCentre/opportunities/jointfunding/}
			\item Annual competitions: \url{http://www.esrcsocietytoday.ac.uk/ESRCInfoCentre/opportunities/annual/index.aspx#3}
			\end{enumerate}
		\end{enumerate}
	\item Arts and Humanities Research Council (AHRC): \vspace{-0.2cm}
		\begin{enumerate} \itemsep -2pt
		\item Funding Opportunities: \vspace{-0.1cm}
			\begin{enumerate} \itemsep -1pt
			\item \url{http://www.ahrc.ac.uk/FundingOpportunities/Pages/default.aspx}
			\item Fellowships: \url{http://www.ahrc.ac.uk/FundingOpportunities/Pages/Fellowships.aspx}
			\item Fellowships - route for early career researchers: \url{http://www.ahrc.ac.uk/FundingOpportunities/Pages/Fellowshipserc.aspx}
			\item Placement Fellowship based in the Department for Culture, Media and Sport (DCMS) - Climate Change: \url{http://www.ahrc.ac.uk/FundingOpportunities/Pages/PlacementFellowshipDCMS-Climatechange.aspx}
			\item Placement Fellowship based in the Department for Culture, Media and Sport (DCMS) - Health and Wellbeing: \url{http://www.ahrc.ac.uk/FundingOpportunities/Pages/PlacementFellowshipDCMShealthandwellbeing.aspx}
			\item Research Grants - route for early career researchers: \url{http://www.ahrc.ac.uk/FundingOpportunities/Pages/RG-EarlyCareers.aspx}
			\item Research Grants - Speculative Research: \url{http://www.ahrc.ac.uk/FundingOpportunities/Pages/RG-SpeculativeResearch.aspx}
			\item Research Grants - Standard Route: \url{http://www.ahrc.ac.uk/FundingOpportunities/Pages/RG-StandardRoute.aspx}
			\item Postgraduate Funding (for Masters and Ph.D. students): \url{http://www.ahrc.ac.uk/FundingOpportunities/Pages/summaryinformationforprospectivepostgraduatestudents.aspx}
			\item Browse Funding Opportunities: \url{http://www.ahrc.ac.uk/FundingOpportunities/Pages/BrowseOpportunities.aspx}
			\end{enumerate}
		\end{enumerate}
	\end{enumerate}
\item World Bank Institute (WBI): \vspace{-0.3cm}
	\begin{enumerate} \itemsep -2pt
	\item Or The World Bank Group
	\item Scholarships: \url{http://wbi.worldbank.org/wbi/scholarships} or \url{http://www.worldbank.org/wbi/scholarships/home.html}
	\end{enumerate}
\item --- --- --- --- --- --- --- --- --- --- --- --- --- --- --- --- --- --- --- --- --- --- --- --- --- --- --- --- --- --- ---
\item \colorbox{blue}{\bf Fellowships in Art and Music}
% Fellowships in Art and Music
\item The Kresge Foundation: \vspace{-0.3cm}
	\begin{enumerate} \itemsep -2pt
	\item \url{http://www.kresge.org/index.php/what/detroit_program/kresge_arts_in_detroit/}
	\item Kresge Artist Fellowships: \vspace{-0.2cm}
		\begin{enumerate} \itemsep -2pt
		\item ``Kresge Artist Fellowships seek to advance the art forms and professional careers of artists from the visual, performing and literary arts as well as elevate the profile of the artistic community and encourage creative expression in the region. Each year, Kresge will provide funding for 18 fellowships of \$25,000 each, which are awarded to artists living and working in metropolitan Detroit.''
		\item ``The fellowships recognize creative vision and commitment to excellence within a wide range of artistic disciplines, including artists who have been classically and academically trained, self taught artists and artists whose art forms have been passed down through cultural and traditional heritage.''
		\item ``Kresge Arts in Detroit is committed to supporting artists from diverse cultural backgrounds at all stages of their professional careers.''
		\item \url{http://kresge.collegeforcreativestudies.edu/}
		\item \url{http://kresge.collegeforcreativestudies.edu/kaf_guidelines.html}
		\item Information Sessions: \url{http://kresge.collegeforcreativestudies.edu/kaf_sessions.html}
		\end{enumerate}
	\item Kresge Eminent Artist Award: \vspace{-0.2cm}
		\begin{enumerate} \itemsep -2pt
		\item ``Kresge Eminent Artist Award recognizes an exceptional artist for his or her professional achievements and contributions to the cultural community, and encourages that individual's pursuit of a chosen art form as well as an ongoing commitment to metropolitan Detroit. Each year, one highly accomplished individual will be presented with the award which includes a \$50,000 prize.''
		\item \url{http://kresge.collegeforcreativestudies.edu/eminent-artist-award.html}
		\end{enumerate}
	\end{enumerate}
\item Guggenheim Fellowships from the {\it John Simon Guggenheim Memorial Foundation}: \url{http://www.gf.org/applicants}
\item The John F. Kennedy Center for the Performing Arts: \vspace{-0.3cm}
	\begin{enumerate} \itemsep -2pt
	\item DeVos Institute of Arts Management at the Kennedy Center: \vspace{-0.2cm}
		\begin{enumerate} \itemsep -2pt
		\item DeVos Institute Programs: \vspace{-0.1cm}
			\begin{enumerate} \itemsep -1pt
			\item Kennedy Center Fellowship Program: \vspace{-0.1cm}
				\begin{itemize} \itemsep -1pt
				\item \url{http://www.kennedy-center.org/education/artsmanagement/fellowships.cfm}
				\item \url{http://www.kennedy-center.org/education/artsmanagement/fellowships/home.html}
				\item ``The Kennedy Center Fellowship Program began in 2001, and provides comprehensive study to 10 arts managers at the Kennedy Center with coursework in strategic planning, marketing, and development; three practical work rotations in Center departments; and a series of professional development seminars. The paid fellowships are full-time and last nine months from September through May.''
				\end{itemize}
			\item DeVos Institute Summer International Fellowship Program at the Kennedy Center: \vspace{-0.1cm}
				\begin{itemize} \itemsep -1pt
				\item \url{http://www.kennedy-center.org/education/artsmanagement/fellowships.cfm}
				\item \url{http://www.kennedy-center.org/education/artsmanagement/international_faq.cfm}
				\item ``The Summer International Fellowship Program provides practical experience to 15 mid-to-high level arts leaders currently working in international nonprofit performing arts organizations. This full-time, four-week intensive program takes place at the Kennedy Center each July; Fellows attend each summer for three consecutive years. While at the Center, the fellows take classes and refine strategic plans for their home organizations.''
				\end{itemize}
			\item U.S. Department of State International Exchange Programs: \vspace{-0.1cm}
				\begin{itemize} \itemsep -1pt
				\item \url{http://www.kennedy-center.org/education/state/}
				\item ``The U.S. Department of State and The Kennedy Center have teamed to produce international exchange opportunities through the Performing Artists Cultural Visitors Program and International Cultural Fellows Mentoring Program.''
				\item Performing Artists Cultural Visitors Program: \url{http://www.kennedy-center.org/education/state/cultural/}
				\item International Cultural Fellows Mentoring Program: \url{http://www.kennedy-center.org/education/state/fellows/}
				\item ``Visitors, comprised of modern and hip-hop dancers, theater technicians/designers/actors, as well as classical and jazz musicians, engage with American colleagues in the creation and performance of their discipline in Washington, D.C. and in another American city.''
				\item ``The Fellows, comprised of arts managers and presenters from outside the United States, attend arts management seminars led by Kennedy Center staff, travel to another American city to study with a mentor organization, and visit New York City to meet with experts in their field.''
				\end{itemize}
			\end{enumerate}
		\end{enumerate}
	\item The National Symphony Orchestra (NSO): \vspace{-0.2cm}
		\begin{enumerate} \itemsep -2pt
		\item National Symphony Orchestra Youth Fellowship Program: \vspace{-0.1cm}
			\begin{itemize} \itemsep -1pt
			\item \url{http://www.kennedy-center.org/nso/nsoed/youthfellowship.cfm}
			\item \url{http://www.kennedy-center.org/explorer/artists/?entity_id=10811&source_type=B}
			\item ``Now in its 30th season, the National Symphony Orchestra Youth Fellowship Program is an orchestral training project for high school musicians.''
			\item ``From its inception in 1980-81 to the present, the program provides Washington metropolitan area high school students with scholarships to study privately with NSO members, as well as opportunities to observe NSO rehearsals; attend concerts; and to participate in seminars, discussions, and master classes with musicians, conductors, and NSO and Kennedy Center management.''
			\item ``There are 20 students in the NSO Youth Fellowship Program for 2009-10.''
			\item ``Participation by ethnic minorities is encouraged.''
			\item ``Priority is given to students entering 10th grade in order to provide as sustained a training as possible.''
			\end{itemize}
		\end{enumerate}
	\end{enumerate}
\item League of American Orchestras: \vspace{-0.3cm}
	\begin{enumerate} \itemsep -2pt
	\item Fellowships: \vspace{-0.2cm}
		\begin{enumerate} \itemsep -2pt
		\item \url{http://www.americanorchestras.org/learning_and_leadership/fellowships.html}
		\item Orchestra Management Fellowship Program: \vspace{-0.1cm}
			\begin{enumerate} \itemsep -1pt
			\item \url{http://www.americanorchestras.org/learning_and_leadership/omfp.html}
			\item ``This year-long, highly competitive program is designed to launch executive careers in orchestra management.''
			\item ``Along with an intense course of study, fellows undertake a series of residencies with orchestras of various sizes across the U.S. receiving invaluable work experience and the support of host orchestra staff, in particular that of the orchestra�s executive director.''
			\item ``Fellows also participate in other League leadership seminars throughout the year and receive a comprehensive overview of the classical music industry.''
			\end{enumerate}
		\item ``The League's Fellowship programs identify and prepare the future leaders of tomorrow, today.''
		\item ``Long-term curricula, developed for conductors, executive directors, and managers looking to advance, provide intensive education, hands-on learning, and valuable networking opportunities.''
		\end{enumerate}
	\end{enumerate}
\item Americans for the Arts: \vspace{-0.3cm}
	\begin{enumerate} \itemsep -2pt
	\item Event scholarships (scholarships to attend events): \url{http://www.artsusa.org/events/scholarships.asp}
	\item \url{http://www.artsusa.org/news/annual_awards/default.asp}
	\item Alene Valkanas State Arts Advocacy Award\url{http://www.artsusa.org/news/annual_awards/alene_valkanas/default.asp}
	\item Arts Education Award (awarded to institutions): \url{http://www.artsusa.org/news/annual_awards/arts_education/default.asp}
	\item Emerging Leader Award: \url{http://www.artsusa.org/news/annual_awards/emerging_leader/default.asp}
	\item Michael Newton Award for United Arts Funds Leadership (management and fundraising): \url{http://www.artsusa.org/news/annual_awards/michael_newton/default.asp}
	\item Selina Roberts Ottum Award (contributions to the field of the arts): \url{http://www.artsusa.org/news/annual_awards/selina_roberts_ottum/default.asp}
	\item United States Urban Arts Federation (USUAF): \vspace{-0.2cm}
		\begin{enumerate} \itemsep -2pt
		\item Ray Hanley Innovation Award: \url{http://www.artsusa.org/networks/usuaf/hanley.asp}
		\end{enumerate}
	\end{enumerate}
\item NEA National Heritage Fellowship (for master folk and traditional artists): \url{http://www.nea.gov/honors/heritage/index.html}
\item NEA Jazz Masters Fellowship (jazz artists): \url{http://www.arts.gov/honors/jazz/index.html}
\item Fellowships for Creative Writers [or NEA Literature Fellowships: Creative Writing]: \url{http://www.nea.gov/grants/apply/Lit/index.html} or \url{http://www.arts.gov/grants/apply/Lit/index.html}
\item Carnegie Investment Bank: Carnegie Art Award (for distinguished artists born or living in the Nordic countries), \url{http://www.carnegie.se/sv/ArtAward/About-Carnegie-Art-Award/}, \url{http://www.carnegie.se/artaward/}, and \url{http://www.carnegie.se/en/about/Operations/Carnegie-Art-Award/}
\item Robert McCann Foundation: \vspace{-0.3cm}
	\begin{enumerate} \itemsep -2pt
	\item Funding for artists and designers ``from all Scottish colleges and art schools'' to: \vspace{-0.2cm}
		\begin{enumerate} \itemsep -2pt
		\item extend their training in an area of specialization; OR
		\item finance a project ``in the craft industries associated with film and television''
		\end{enumerate}
	\item \url{http://robertmccannfoundation.com/how.html}
	\end{enumerate}
\item Alexander von Humboldt-Stiftung/Foundation: \vspace{-0.3cm}
	\begin{enumerate} \itemsep -2pt
	\item Hezekiah Wardwell Fellowship (for musicians or musicologists from Spain): \url{http://www.humboldt-foundation.de/web/wardwell-en.html}
	\end{enumerate}
\item Canada Council for the Arts: \vspace{-0.3cm}
	\begin{enumerate} \itemsep -2pt
	\item Endowments and Prizes: \vspace{-0.2cm}
		\begin{enumerate} \itemsep -2pt
		\item \url{http://www.canadacouncil.ca/prizes/}
		\item Prizes and fellowships for Canadian artists and scholars to recognize their contributions to the arts, humanities, and sciences
		\item Categories of prizes and fellowships: \vspace{-0.1cm}
			\begin{enumerate} \itemsep -1pt
			\item dance
			\item inter-arts
			\item media arts
			\item music
			\item theatre
			\item visual arts
			\item writing and publishing
			\end{enumerate}
		\end{enumerate}
	\item Grant Programs: \url{http://www.canadacouncil.ca/grants/}
	\end{enumerate}
\item Institute for Humane Studies at George Mason University: \vspace{-0.3cm}
	\begin{enumerate} \itemsep -2pt
	\item Film \& Fiction Scholarships: \vspace{-0.2cm}
		\begin{enumerate} \itemsep -2pt
		\item Students pursuing MFAs in a variety of areas are eligible: film directing, production, screenwriting, playwriting, fiction, and literary-nonfiction writing
		\item \url{http://www.theihs.org/node/448}
		\end{enumerate}
	\end{enumerate}
\item --- --- --- --- --- --- --- --- --- --- --- --- --- --- --- --- --- --- --- --- --- --- --- --- --- --- --- --- --- --- ---
\item \colorbox{blue}{\bf Scholarships and Fellowships for Underrepresented Minorities}
% Scholarships and Fellowships for Underrepresented Minorities
\item Lists of scholarships and fellowships for underrepresented minorities: \vspace{-0.3cm}
	\begin{enumerate} \itemsep -2pt
	\item Chris Enstrom, ``Cashing in on Diversity Grants and Scholarships,'' in Graduating Engineer \& Computer Careers. Available at: \url{http://www.graduatingengineer.com/higher-education/20061129/Cashing-in-on-Diversity-Grants-and-Scholarships-}; last accessed on August 25, 2010.
	\end{enumerate}
\item Gates Millennium Scholars (GMS) scholarship (for underrepresented minorities in the US): \url{http://www.gmsp.org/}
\item Society of Women Engineers (SWE): SWE Scholarships and other scholarships, \url{http://societyofwomenengineers.swe.org/index.php?option=com_content&task=view&id=222&Itemid=111}
\item Coalition to Diversify Computing: \url{http://www.cdc-computing.org/scholarships/}
\item IES Abroad (formerly Institute of European Studies / Institute for the International Education of Students): \vspace{-0.3cm}
	\begin{enumerate} \itemsep -2pt
	\item Diversity Abroad: \vspace{-0.2cm}
		\begin{enumerate} \itemsep -2pt
		\item \url{https://www.iesabroad.org/IES/Diversity/diversity.html}
		\item Programs to improve student diversity in study abroad programs
		\item IES Abroad Diversity Scholarships: \vspace{-0.1cm}
			\begin{enumerate} \itemsep -1pt
			\item IES Abroad Merit-Based Scholarship for Under-represented Students: \url{https://www.iesabroad.org/IES/Scholarships_and_Aid/Diversity_Scholarships/diversityScholarship.html}
			\item IES Abroad Merit-Based David Porter Diversity Scholarship (Up to \$5,000!): \url{https://www.iesabroad.org/IES/Scholarships_and_Aid/Merit_Based/davidPorterScholarship.html}
			\item HBCU Scholarships: \url{https://www.iesabroad.org/IES/Scholarships_and_Aid/Diversity_Scholarships/hbcuScholarship.html}
			\item HACU-IES Abroad Merit/Need-Based Scholarship: \url{https://www.iesabroad.org/IES/Scholarships_and_Aid/Diversity_Scholarships/HACUScholarship.html}
			\end{enumerate}
		\end{enumerate}
	\end{enumerate}
\item MassMutual Scholars Program: \vspace{-0.3cm}
	\begin{enumerate} \itemsep -2pt
	\item Applicants must be undergraduates of African American/Black, Asian/Pacific Islander or Hispanic decent in the US.
	\item Reside or plan to attend an institution in one of the following metropolitan areas: \vspace{-0.2cm}
		\begin{enumerate} \itemsep -2pt
		\item Atlanta, GA
		\item Chicago, IL
		\item Central New Jersey
		\item Denver, CO
		\item Houston, TX
		\item Miami, FL
		\item Los Angeles, CA
		\item San Antonio, TX
		\item San Francisco, CA
		\end{enumerate}
	\item Be majoring in business, economics, finance, financial planning, management, marketing or sales.
	\item \url{http://www.hsf.net/massmutual.aspx}
	\item \url{http://www.apiasf.org/scholarship_apiasf_massmutual.html}
	\end{enumerate}
\item {\it NASA}'s Minority University Research and Education Program (MUREP): \vspace{-0.3cm}
	\begin{enumerate} \itemsep -2pt
	\item \url{http://www.nasa.gov/offices/education/programs/national/murep/home/index.html}
	\item \url{http://www.nasa.gov/offices/education/about/murep_overview.html}
	\item Jenkins Pre-doctoral Fellowship Project, JPFP: \url{http://www.nasa.gov/offices/education/programs/descriptions/Jenkins_Predoctoral_Fellowship_Project.html}
	\end{enumerate}
\item UNCF: \vspace{-0.3cm}
	\begin{enumerate} \itemsep -2pt
	\item UNCF Special Programs Corporation: \vspace{-0.2cm}
		\begin{enumerate} \itemsep -2pt
		\item Harriett G. Jenkins Pre-doctoral Fellowship Program (JPFP) for underrepresented minorities pursuing graduate degrees in STEM: \url{http://www.uncfsp.org/spknowledge/default.aspx?page=program.view&areaid=1&contentid=177&typeid=jpfp}
		\item NASA Science and Technology Institute (NSTI) Summer Scholars Program (10-week summer research scholarship): \url{http://www.uncfsp.org/spknowledge/default.aspx?page=program.view&areaid=1&contentid=172&typeid=nstiinternship}
		\item Motivating Undergraduates in Science and Technology (MUST) Program for undergraduates in STEM: \url{http://www.uncfsp.org/spknowledge/default.aspx?page=program.view&areaid=1&contentid=346&typeid=must}
		\item Institute for International {\bf Public Policy} Fellows Program: \url{http://www.uncfsp.org/IIPP}
		\item \url{http://www.uncfsp.org/spknowledge/default.aspx?page=home.default}
		\end{enumerate}
	\item UNCF scholarship resources: \url{http://www.uncf.org/forstudents/scholarship.asp}
	\item UNCF $\cdot$ Merck Science Initiative: scholarships and fellowships: \url{http://umsi.uncf.org/ScholarshipsInternshipsFellowships/tabid/151/Default.aspx}
	\end{enumerate}
\item Hispanic College Fund: \vspace{-0.3cm}
	\begin{enumerate} \itemsep -2pt
	\item Scholarships: \url{http://www.hispanicfund.org/scholarships/} and \url{http://scholarships.hispanicfund.org/applications/}
	\item NASA MUST Scholarship Program: \url{http://www.hispanicfund.org/nasa-must/}
	\item Hispanic Youth Symposium (scholarships are awarded to winners of the art competition, talent competition, and speech competition): \url{http://www.hispanicyouth.org/about-the-program}
	\item \url{http://www.hispanicfund.org/}
	\end{enumerate}
\item Hispanic Heritage Foundation (HHF): \vspace{-0.3cm}
	\begin{enumerate} \itemsep -2pt
	\item Scholarships and Resources: \url{http://www.hispanicheritage.org/youth_int.php?sec=80}
	\item \url{http://www.hispanicheritage.org/}
	\end{enumerate}
\item Hispanic Scholarship Fund (HSF): \vspace{-0.3cm}
	\begin{enumerate} \itemsep -2pt
	\item Scholarship programs for: \vspace{-0.2cm}
		\begin{enumerate} \itemsep -2pt
		\item college students
		\item community college transfer students
		\item high school students
		\item Gates Millennium Scholars
		\item See \url{http://www.hsf.net/innercontent.aspx?id=34}
		\end{enumerate}
	\item \url{http://www.hsf.net/}
	\end{enumerate}
\item League of United Latin American Citizens (LULAC): \vspace{-0.3cm}
	\begin{enumerate} \itemsep -2pt
	\item LULAC National Educational Service Centers, Inc: \vspace{-0.2cm}
		\begin{enumerate} \itemsep -2pt
		\item \url{http://www.lnesc.org/}
		\item LULAC National Scholarship Fund (LNSF): \vspace{-0.1cm}
			\begin{enumerate} \itemsep -1pt
			\item \url{http://www.lulac.org/programs/education/scholarships/}
			\item \url{http://lnesc.org/index.asp?Type=B_BASIC&SEC={3AEDB506-F425-4E58-B9F6-44867E2FD943}}
%http://lnesc.org/index.asp?Type=B_BASIC&SEC={3AEDB506-F425-4E58-B9F6-44867E2FD943}
			\item Applicants must meet the following criteria to be considered for a scholarship: \vspace{-0.1cm}
				\begin{itemize} \itemsep -1pt
				\item Must be a U.S. citizen or legal resident
				\item Must have applied to or be enrolled in a   college, university, or graduate school, including 2-year colleges, or vocational schools that lead to an associate�s degree
				\item A student will not be eligible for a scholarship if he/she is related to a scholarship committee member, the Council President, or an individual contributor to the local funds of the Council
				\end{itemize}
			\item National Scholastic Achievement Awards (for high school seniors entering college, university, or vocational school)
			\item Honors Awards (for high school seniors entering college, university, or vocational school)
			\item General Awards (Need, community involvement, and leadership activities will also be considered)
			\item General Electric Foundation/ LULAC Scholarship program: for underrepresented minorities (US freshmen) entering their sophomore year as majors in Business or Engineering with a cumulative college G.P.A. $\leq$ 3.25/4.0; these students must be enrolled in a 4-year undergraduate program.
			\end{enumerate}
		\end{enumerate}
	\end{enumerate}
\item Hispanic Association of Colleges and Universities (HACU): \vspace{-0.3cm}
	\begin{enumerate} \itemsep -2pt
	\item HACU Student Programs Overview: \vspace{-0.2cm}
		\begin{enumerate} \itemsep -2pt
		\item \url{http://www.hacu.net/hacu/HACU_Student_Programs_EN.asp?SnID=1942709283}
		\item HACU Scholarship Programs: \vspace{-0.1cm}
			\begin{enumerate} \itemsep -1pt
			\item \url{http://www.hacu.net/hacu/Scholarships_EN.asp?SnID=1942709283}
			\item Includes scholarships for students in: \vspace{-0.1cm}
				\begin{itemize} \itemsep -1pt
				\item Accounting
				\item Behavioral Health
				\item Business
				\item Clinical Psychology
				\item Computer Engineering
				\item Computer Science
				\item Dental Technician
				\item Electrical Engineering
				\item Engineering
				\item Food Merchandising
				\item Information Technology
				\item International Business
				\item Management
				\item Marketing
				\item Mass Media
				\item Mental Health
				\item Merchandising
				\item Nursing
				\item Physician Assistant
				\item (Pre) Optometry
				\item (Pre) Dental
				\item (Pre) Medicine
				\item (Pre) Pharmacy
				\item Public Health
				\item Public Relations
				\item Retail Management
				\item Sports Marketing
				\item Technology
				\end{itemize}
			\end{enumerate}
		\item ``D{\'{a}}ndole Alas a Tu {\'{E}}xito/Giving Flight to Your Success'' travel award program (Southwest Airlines' Travel Award Program): \vspace{-0.1cm}
			\begin{enumerate} \itemsep -1pt
			\item For students with financial need who have to across the United States to participate in their undergraduate or graduate degree programs
			\item \url{http://www.hacu.net/hacu/Lanzate_EN.asp?SnID=1942709283}
			\item \url{http://www.hacu.net/hacu/Lanzate1_EN.asp?SnID=1808826658}
			\end{enumerate}
		\item HACU Study Abroad Scholarship Programs: \vspace{-0.1cm}
			\begin{enumerate} \itemsep -1pt
			\item \url{http://www.hacu.net/hacu/Study_Abroad_EN.asp?SnID=1808826658}
			\item HACU-Global Learning Semesters (GLS) Program: Hispanic Study Abroad Scholars: \url{http://www.studyabroadscholars.org/index.html}
			\item HACU-American Institute for Foreign Study (AIFS) Scholarship Program: \url{http://www.aifsabroad.com/scholarships.asp#hacu}
			\item HACU-Institute for the International Education of Students (IES) Scholarship Program: \url{https://www.iesabroad.org/IES/home.html}
			\item Hispanic Study Abroad Scholars program: \url{http://www.studyabroadscholars.org/index.html}
			\end{enumerate}
		\item Scholarship Resource List: \url{http://www.hacu.net/hacu/Scholarship_Resource_List_EN.asp?SnID=1109551622}
%		\item Scholarship Resource List: \url{http://www.hacu.net/hacu/Scholarship_Resource_List_EN.asp?SnID=1942709283}		-- Redundant
		\end{enumerate}
	\end{enumerate}
\item Congressional Hispanic Caucus Institute (CHCI): \vspace{-0.3cm}
	\begin{enumerate} \itemsep -2pt
	\item CHCI Scholarship: \vspace{-0.2cm}
		\begin{enumerate} \itemsep -2pt
		\item \url{http://www.chci.org/scholarships/}
		\item CHCI's scholarship opportunities are afforded to Latino students in the United States who have a history of performing public service-oriented activities in their communities and who demonstrate a desire to continue their civic engagement in the future. There is no GPA or academic major requirement. Students with excellent leadership potential are encouraged to apply.
		\item Scholarship awards are intended to provide assistance with tuition, room and board, textbooks, and other educational expenses associated with college enrollment.
		\item Students continue to receive annual disbursements as long as they maintain good academic standing.
		\item CHCI scholarships provide recipients with a one time scholarship of: \vspace{-0.1cm}
			\begin{enumerate} \itemsep -1pt
			\item \$1,000 community college or AA/AS granting institution
			\item \$2,500 4-year academic institution
			\item \$5,000 graduate-level institution
			\end{enumerate}
		\item Eligibility Criteria: \vspace{-0.1cm}
			\begin{enumerate} \itemsep -1pt
			\item Full-time enrollment in a United States Department of Education accredited community college, four-year university, or graduate/professional program during the period for which scholarship is requested
			\item Demonstrated financial need
			\item Consistent, active participation in public and/or community service activities
			\item Strong writing skills
			\item U.S. citizenship or legal permanent residency
			\end{enumerate}
		\end{enumerate}
	\item CHCI Fellowships: \vspace{-0.2cm}
		\begin{enumerate} \itemsep -2pt
		\item \url{http://www.chci.org/fellowships/}
		\item CHCI {\bf Public Policy} Fellowship: \vspace{-0.1cm}
			\begin{enumerate} \itemsep -1pt
			\item This is a paid Fellowship Program that offers talented Latinos, who have earned a bachelor's degree within two years of the program start date, the opportunity to gain hands-on experience at the national level in public policy.
			\item Fellows have the opportunity to work in congressional offices and federal agencies, depending on their area of interest.  Some past focus areas have included international affairs, economic development, health and education policy, housing, or local government.
			\item Program Dates: August to May (10-month internship)
			\item \url{http://www.chci.org/fellowships/page/chci-public-policy-fellowship}
			\end{enumerate}
		\item CHCI Graduate Fellowship Program: \vspace{-0.1cm}
			\begin{enumerate} \itemsep -1pt
			\item The CHCI Graduate Fellowship Program seeks to enhance participants' leadership abilities, strengthen professional skills and ultimately produce more competent and competitive Latino professionals in underserved {\bf public policy} issue areas.
			\item This paid Fellowship Program offers exceptional Latinos who have earned a graduate degree or higher related to a chosen policy issue area within three years of program start date unparalleled exposure to hands-on experience in public policy.
			\item This program focuses specifically on the areas of: \vspace{-0.1cm}
				\begin{itemize} \itemsep -1pt
				\item Higher Education: CHCI Graduate Higher Education Fellowship, \url{http://www.chci.org/fellowships/page/chci-graduate-higher-education-fellowship}
				\item Secondary Education: CHCI Graduate Secondary Education Fellowship, \url{http://www.chci.org/fellowships/page/chci-graduate-secondary-education-fellowship}
				\item Health: CHCI Graduate Health Fellowship, \url{http://www.chci.org/fellowships/page/chci-graduate-health-fellowship}
				\item Housing: CHCI Graduate Housing Fellowship, \url{http://www.chci.org/fellowships/page/chci-graduate-housing-fellowship}
				\item International Affairs (includes last three months abroad in Mexico): CHCI Graduate International Affairs Fellowship, \url{http://www.chci.org/fellowships/page/chci-graduate-international-affairs-fellowship}
				\item Law: CHCI Graduate Law Fellowship, \url{http://www.chci.org/fellowships/page/chci-graduate-law-fellowship}
				\item STEM (Science, Technology, Engineering and Math): CHCI Graduate STEM Fellowship, \url{http://www.chci.org/fellowships/page/chci-graduate-stem-fellowship}
				\end{itemize}
			\item Program Dates: August to May (10-month internship)
			\item \url{http://www.chci.org/fellowships/page/chci-graduate-fellowship-program}
			\end{enumerate}
		\end{enumerate}
	\end{enumerate}
\item American Indian Graduate Center (AIGC): \vspace{-0.3cm}
	\begin{enumerate} \itemsep -2pt
	\item AIGC scholarships and fellowships: \vspace{-0.2cm}
		\begin{enumerate} \itemsep -2pt
		\item for advanced degree students in art, music, environmental studies, journalism, communications, medicine, dentistry, public health, nursing, or other health-related fields
		\item for members of Wisconsin, New Mexico or Arizona tribes.
		\item \url{http://www.aigc.com/02scholarships/scholarships.htm}
		\item AIGC Fellowship (Graduate) for Native Americans and their descendants seeking advanced degrees: \url{http://www.aigc.com/02scholarships/aigc/fellowship.htm}
		\item Rainer Scholarship (for grad students): \url{http://www.aigc.com/02scholarships/rainer.htm}
		\end{enumerate}
	\item List of resources about scholarships and fellowships: \vspace{-0.2cm}
		\begin{enumerate} \itemsep -2pt
		\item \url{http://www.aigc.com/08otherscholarship/otherscholarships.html}
		\item Scholarships: \url{http://www.aigc.com/08otherscholarship/scholarships.htm}
		\item Fellowships: \url{http://www.aigc.com/08otherscholarship/fellowships.htm}
		\end{enumerate}
	\item Gates Millennium Scholar Program (for individuals seeking basic and advanced degrees): \url{http://www.aigc.com/03gms/gms.htm}
	\end{enumerate}
\item Asian \& Pacific Islander American Scholarship Fund (APIASF) scholarship resources: \url{http://www.apiasf.org/scholarships.html}
\item American Association of University Women: \vspace{-0.3cm}
	\begin{enumerate} \itemsep -2pt
	\item \url{http://www.aauw.org/learn/fellowships_grants/index.cfm}
	\end{enumerate}
\item Sigma Delta Epsilon-Graduate Women in Science (GWIS): \url{http://www.gwis.org/programs.html}
\item Society of Hispanic Professional Engineers (SHPE): \vspace{-0.3cm}
	\begin{enumerate} \itemsep -2pt
	\item Advancing Hispanic Excellence in Technology, Engineering, Math and Science (AHETEMS) Foundation: \url{http://www.ahetems.org/}
	\item AHETEMS Scholarship Program: \url{http://www.ahetems.org/scholarships/}
	\item Graduate \& Young Professional Fellowship Program (encourage young professionals to engage in {\bf public policy}): \url{http://www.ahetems.org/graduate/graduate-young-professional-fellowship-program/}
	\item SHPE/GEM Fellowship (for graduate students in STEM at GEM Member Universities): \url{http://www.ahetems.org/graduate/shpe-gem-graduate-award/}. See \url{http://www.gemfellowship.org/gem-universities/university-members} for a list of GEM member universities.
	\end{enumerate}
\item National Society of Black Engineers (NSBE): \vspace{-0.3cm}
	\begin{enumerate} \itemsep -2pt
	\item Scholarships: \url{http://www.nsbe.org/Programs/Scholarships.aspx}
	\end{enumerate}
\item The Society of Mexican American Engineers and Scientists (MAES): \vspace{-0.3cm}
	\begin{enumerate} \itemsep -2pt
	\item Scholarships \& Awards: \url{http://www.maes-natl.org/index.php?meid=328}
	\item MAES Scholarship Program: \url{http://www.maes-natl.org/index.php?module=ContentExpress&func=display&ceid=518&meid=241}
	\end{enumerate}
\item SACNAS (Society for Advancement of Chicanos and Native Americans in Science): \vspace{-0.3cm}
	\begin{enumerate} \itemsep -2pt
	\item Scholarships: \url{http://www.sacnas.org/webadindex.cfm?webadcategory_id=7}
	\item Fellowships: \url{http://www.sacnas.org/webadIndex.cfm?webadcategory_id=5}
	\end{enumerate}
\item {\it Center for the Advancement of Hispanics in Science and Engineering Education} (CAHSEE): \vspace{-0.3cm}
	\begin{enumerate} \itemsep -2pt
	\item Scholarships: \url{http://www.cahsee.org/6resources/scholarships.asp.htm}
	\end{enumerate}
\item National Consortium for Graduate Degrees for Minorities in Engineering and Science, Inc.: \vspace{-0.3cm}
	\begin{enumerate} \itemsep -2pt
	\item National GEM Consortium: GEM Fellowship, \url{http://www.gemfellowship.org/gem-fellowship/application-requirements}
	\end{enumerate}
\item National Physical Science Consortium (NPSC): \vspace{-0.3cm}
	\begin{enumerate} \itemsep -2pt
	\item NPSC Graduate Fellowship: \url{http://www.npsc.org/}
	\end{enumerate}
\item Finch College Alumnae Association: \vspace{-0.3cm}
	\begin{enumerate} \itemsep -2pt
	\item The Finch College Alumnae Foundation Education Grant: \vspace{-0.2cm}
		\begin{enumerate} \itemsep -2pt
		\item \url{http://www.finchcollege.org/newscholarships.html}
		\item \url{http://www.finchcollege.org/newFinchGrantQandA.html}
		\item ``THE FINCH GRANT, an annual program where four community college women entering a four year college are awarded a grant of \$1500 which can be used toward any needs to completing college.  The selection is determined by a panel of college professors.''
		\end{enumerate}
	\end{enumerate}
\item : \url{}
\item : \url{}
\item : \url{}
\item : \url{}
\item : \url{}
\item \S\ref{phdandpostdocfellowships} has more information concerning scholarships and fellowships in the following areas: \vspace{-0.3cm}
	\begin{enumerate} \itemsep -2pt
	\item electronic design automation (EDA), and related areas of design automation: \vspace{-0.2cm}
		\begin{enumerate} \itemsep -2pt
		\item bio design automation (BDA)
		\item Lab-on-chip design (LoC) automation
		\item MEMS/NEMS design automation
		\end{enumerate}
	\item digital VLSI design
	\item analog and mixed-signal (AMS) VLSI design
	\item computer architecture
	\item parallel computing
	\item concurrent programming
	\item data mining
	\item theoretical computer science
	\end{enumerate}
\item Ph.D. dissertation awards: \vspace{-0.3cm}
	\begin{enumerate} \itemsep -2pt
	\item --- --- --- --- --- --- --- --- --- --- --- --- --- --- --- --- --- --- --- --- --- --- --- --- --- --- --- --- --- --- ---
	\item \colorbox{blue}{\bf Ph.D. Dissertation Awards for Computer Science}
	% Ph.D. Dissertation Awards for Computer Science
	\item ACM Doctoral Dissertation Award: \url{http://awards.acm.org/doctoral_dissertation/}
	\item ACM Outstanding Ph.D. Dissertation Award in Electronic Design Automation: \url{http://www.sigda.org/opda.html}
	\item EDAA Outstanding Dissertation Award (European Design and Automation Association, EDAA): \url{http://www.edaa.com/dissertation_award.html} and \url{http://www.esat.kuleuven.be/micas/EDAA-Award/index.php}
	\item EuroSys Roger Needham PhD Award (in the systems area): \vspace{-0.2cm}
		\begin{enumerate} \itemsep -2pt
		\item Areas in systems include: \vspace{-0.1cm}
			\begin{enumerate} \itemsep -1pt
			\item operating systems
			\item distributed systems
			\item real-time systems
			\item systems aspects of databases
			\item language runtimes
			\item \colorbox{yellow}{\bf embedded systems}
			\item computer networks
			\end{enumerate}
		\item \url{http://www.eurosys.org/phdprize/index.php}
		\end{enumerate}
	\item ACM SIGPLAN Outstanding Doctoral Dissertation Award: \url{http://www.sigplan.org/award-dissertation.htm}
	\item ACM SIGKDD Doctoral Disseration Award (in data mining and knowledge discovery): \url{http://www.sigkdd.org/awards_dissertation.php}
	\item ACM SIGMOD Jim Gray Doctoral Dissertation Award (in the database field): \url{http://www.sigmod.org/sigmod-awards/doctoral-dissertation-award}
	\item Special Interest Group of the ACM on Management Information Systems (SIGMIS): \vspace{-0.2cm}
		\begin{enumerate} \itemsep -2pt
		\item ACM SIGMIS Doctoral Dissertation Award Competition (at the International Conference on Information Systems, ICIS): \url{http://ai.arizona.edu/icis2009/program/dissertation.html} and \url{http://icis2010.aisnet.org/dissertation_award.htm}
		\end{enumerate}
	\item Association for Symbolic Logic: \vspace{-0.2cm}
		\begin{enumerate} \itemsep -2pt
		\item ``The Sacks Prize is awarded for the most outstanding doctoral dissertation in mathematical logic''.
		\item \url{http://www.aslonline.org/Sacks_nominations.html} and \url{http://www.aslonline.org/info-prizes.html}
		\end{enumerate}
	\item European Association for Computer Science Logic (EACSL): \vspace{-0.2cm}
		\begin{enumerate} \itemsep -2pt
		\item Ackermann Award (for outstanding dissertations in Logic in Computer Science): \url{http://www.eacsl.org/} and \url{http://www.eacsl.org/award.html}
		\end{enumerate}
	\item European Coordinating Committee for Artificial Intelligence (ECCAI): \vspace{-0.2cm}
		\begin{enumerate} \itemsep -2pt
		\item 201X Artificial Intelligence Dissertation Award: \url{http://www.eccai.org/diss-award/current.shtml}
		\end{enumerate}
	\item European Conference on Wireless Sensor Networks (EWSN 201X, \url{http://www.nes.uni-due.de/ewsn2011}) and CONET, the Cooperating Objects Network of Excellence: Ph.D. Thesis Award Competition, \url{http://www.cooperating-objects.eu/}
	\item --- --- --- --- --- --- --- --- --- --- --- --- --- --- --- --- --- --- --- --- --- --- --- --- --- --- --- --- --- --- ---
	\item \colorbox{blue}{\bf Ph.D. Dissertation Awards for Mathematics}
	% Ph.D. Dissertation Awards for Mathematics
	\item International Center for Scientific Research (CIRS): \vspace{-0.2cm}
		\begin{enumerate} \itemsep -2pt
		\item E. W. Beth Dissertation Prize (for outstanding dissertations in the fields of Logic, Language and Information): \url{http://www.cirs.net/prix/awards.php?id=481}
		\end{enumerate}
	\item The Association for Operations Management, APICS (Advancing Productivity, Innovation, and Competitive Success): \vspace{-0.2cm}
		\begin{enumerate} \itemsep -2pt
		\item Plossl Doctoral Dissertation Competition: The APICS Educational and Research Foundation, will annually grant one award of \$2,500 for a doctoral dissertation dealing with any topic in operations management. Sample topics include operations strategy, operations planning and control systems, supply chain management, quality management, Six Sigma, facility location, forecasting, just-in-time/lean production systems, and project management. Entrants must be candidates for the doctorate in operations management. The dissertation must be approved by the primary thesis advisor and not more than 50\% completed at time of submission. See \url{http://www.apics.org/Education/ERFoundation/Competitions/plossl.htm}.
		\end{enumerate}
	\item SIAM Richard C. DiPrima Prize: \vspace{-0.2cm}
		\begin{enumerate} \itemsep -2pt
		\item The Richard C. DiPrima Prize is awarded every two years to a junior scientist, based on an outstanding doctoral dissertation in applied mathematics.
		\item \url{http://www.siam.org/prizes/nominations/nom_diprima.php}
		\item \url{http://www.siam.org/prizes/sponsored/diprima.php}
		\end{enumerate}
	\item MOS A.W. Tucker Prize: \vspace{-0.2cm}
		\begin{enumerate} \itemsep -2pt
		\item It is awarded for an outstanding doctoral thesis in any aspect of mathematical optimization.
		\item \url{http://www.mathprog.org/?nav=tucker}
		\end{enumerate}
	\item --- --- --- --- --- --- --- --- --- --- --- --- --- --- --- --- --- --- --- --- --- --- --- --- --- --- --- --- --- --- ---
	\item \colorbox{blue}{\bf Other Ph.D. Dissertation Awards}
	% Other Ph.D. Dissertation Awards
	\item Institute for Operations Research and the Management Sciences (INFORMS): \vspace{-0.2cm}
		\begin{enumerate} \itemsep -2pt
		\item INFORMS George B. Dantzig Dissertation Award: \url{http://www.informs.org/Recognize-Excellence/INFORMS-Prizes-Awards/George-B.-Dantzig-Dissertation-Award}
		\item Best Dissertation Award (Technology Management Section, for Ph.D. theses in technology management): \url{http://www.informs.org/Recognize-Excellence/INFORMS-Community-Prizes-and-Awards2/Technology-Management-Section/Best-Dissertation-Award}
		\item TSL Dissertation Prize (Transportation Science and Logistics Section, for doctoral dissertations in the transportation science and logistics area): \url{http://www.informs.org/Recognize-Excellence/INFORMS-Community-Prizes-and-Awards2/Transportation-Science-and-Logistics-Section/TSL-Dissertation-Prize}
		\item Best Dissertation Award (Telecommunications Section, for Ph.D. theses in telecommunications): \url{http://www.informs.org/Recognize-Excellence/INFORMS-Community-Prizes-and-Awards2/Telecommunications-Section/Best-Dissertation-Award}
		\item Frank M. Bass Dissertation Paper Award (Society for Marketing Science, for the best marketing paper derived from a Ph.D. thesis published in an INFORMS-sponsored journal): \url{http://www.informs.org/Recognize-Excellence/INFORMS-Community-Prizes-and-Awards2/Society-for-Marketing-Science/Frank-M.-Bass-Dissertation-Paper-Award}
		\item SOLA - Air Products Bi-Annual Dissertation Award (Section on Location Analysis, for Ph.D. theses on location related research): \url{http://www.informs.org/Recognize-Excellence/INFORMS-Community-Prizes-and-Awards2/Section-on-Location-Analysis/SOLA-Air-Products-Bi-Annual-Dissertation-Award}
		\end{enumerate}
	\item EURO Doctoral Dissertation Award (EDDA) (in operations research): \url{http://www.euro-online.org/display.php?page=edda1}
	\end{enumerate}
\item Other awards: \vspace{-0.3cm}
	\begin{itemize} \itemsep -2pt
	\item --- --- --- --- --- --- --- --- --- --- --- --- --- --- --- --- --- --- --- --- --- --- --- --- --- --- --- --- --- --- ---
	\item \colorbox{blue}{\bf Awards for Computer Science}
	% Awards for Computer Science
	\item ACM SIGMOD Undergraduate Award: \url{http://www.sigmod.org/sigmod-awards/sigmod-awards#undergraduate}
	\item European Association of Theoretical Computer Science (EATCS): Presburger Award (for young researchers in theoretical computer science), \url{http://www.eatcs.org/index.php/presburger}.
	\item Computer Research Association: \vspace{-0.2cm}
		\begin{enumerate} \itemsep -2pt
		\item Committee on the Status of Women in Computing Research (CRA-W): \vspace{-0.1cm}
			\begin{enumerate} \itemsep -1pt
			\item Borg Early Career Award (BECA): \url{http://www.cra-w.org/borg}
			\end{enumerate}
		\end{enumerate}
	\item European Conference on Wireless Sensor Networks (EWSN 201X, \url{http://www.nes.uni-due.de/ewsn2011}) and CONET, the Cooperating Objects Network of Excellence: Ph.D. Thesis Award Competition, \url{http://www.cooperating-objects.eu/}. ``Cooperating Objects combine the strong functional aspects of embedded systems, pervasive computing and wireless sensor networks. Cooperating objects entities federate themselves into dynamic and loose networks in order to reach a common goal. This common goal will typically be related to sensing or control.''
	\item --- --- --- --- --- --- --- --- --- --- --- --- --- --- --- --- --- --- --- --- --- --- --- --- --- --- --- --- --- --- ---
	\item \colorbox{blue}{\bf Awards for Biomedical Engineering}
	% Awards for Biomedical Engineering
	\item Biomedical Engineering Society (BMES): \vspace{-0.2cm}
		\begin{enumerate} \itemsep -2pt
		\item Rita Schaffer Young Investigator Award (for junior researchers in biomedical engineering): \url{http://www.bmes.org/aws/BMES/pt/sp/awards_investigator}
		\item Graduate and Undergraduate Student Awards: \url{http://www.bmes.org/aws/BMES/pt/sp/awards_student}
		\end{enumerate}
	\item --- --- --- --- --- --- --- --- --- --- --- --- --- --- --- --- --- --- --- --- --- --- --- --- --- --- --- --- --- --- ---
	\item \colorbox{blue}{\bf Awards for Mechanical Engineering}
	% Awards for Mechanical Engineering
	\item American Society of Mechanical Engineers (ASME): \vspace{-0.2cm}
		\begin{enumerate} \itemsep -2pt
		\item Henry Hess Award (authors of research papers who are below 31 years old): \url{http://www.asme.org/Governance/Honors/SocietyAwards/Henry_Hess_Award.cfm}
		\item Pi Tau Sigma Gold Medal (outstanding junior engineers): \url{http://www.asme.org/Governance/Honors/SocietyAwards/Pi_Tau_Sigma_Gold_Medal.cfm}
		\item Marshall B. Peterson Award (researchers in tribology who are below 30 years old): \url{http://www.asme.org/Governance/Honors/SocietyAwards/Marshall_B_Peterson_Award.cfm}
		\item Y.C. Fung Young Investigator Award (for young researchers in bioengineering): \url{http://www.asme.org/Governance/Honors/SocietyAwards/YC_Fung_Young_Investigator.cfm}
		\end{enumerate}
	\item --- --- --- --- --- --- --- --- --- --- --- --- --- --- --- --- --- --- --- --- --- --- --- --- --- --- --- --- --- --- ---
	\item \colorbox{blue}{\bf Awards for Civil Engineering}
	% Awards for Civil Engineering
	\item American Society of Civil Engineers (ASCE): \vspace{-0.3cm}
		\begin{enumerate} \itemsep -2pt
		\item Edmund Friedman Young Engineer Award for Professional Achievement (for junior engineers under the age of 36): \url{http://www.asce.org/AwardsContent.aspx?id=16776}
		\item Committee on Younger Members (CYM) Awards (for junior engineers): \url{http://www.asce.org/Content.aspx?id=11311}
		\item Collingwood Prize (for civil engineering researchers under the age of 35): \url{http://www.asce.org/AwardsContent.aspx?id=15352}
		\end{enumerate}
	\item --- --- --- --- --- --- --- --- --- --- --- --- --- --- --- --- --- --- --- --- --- --- --- --- --- --- --- --- --- --- ---
	\item \colorbox{blue}{\bf Awards for Chemical Engineering}
	% Awards for Chemical Engineering
	\item American Institute of Chemical Engineers (AIChE) awards: \url{http://www.aiche.org/Students/Awards/index.aspx}
	\item --- --- --- --- --- --- --- --- --- --- --- --- --- --- --- --- --- --- --- --- --- --- --- --- --- --- --- --- --- --- ---
	\item \colorbox{blue}{\bf Awards for Systems Engineering}
	% Awards for Systems Engineering
	\item International Council on Systems Engineering (INCOSE) Stevens Doctoral Award (for Promising Research in Systems Engineering and Integration; A.B.D.s / Ph.D. candidates): \url{http://www.incose.org/about/foundation/doctoralaward.aspx}
	\item --- --- --- --- --- --- --- --- --- --- --- --- --- --- --- --- --- --- --- --- --- --- --- --- --- --- --- --- --- --- ---
	\item \colorbox{blue}{\bf Awards for Mathematics, Operations Research, \& Management Sciences}
	% Awards for Mathematics, Operations Research, and Management Sciences
	\item Institute for Operations Research and the Management Sciences (INFORMS): \vspace{-0.2cm}
		\begin{enumerate} \itemsep -2pt
		\item INFORMS Undergraduate Operations Research Prize: \url{http://www.informs.org/Recognize-Excellence/INFORMS-Prizes-Awards/INFORMS-Undergraduate-Operations-Research-Prize}
		\item Optimization Prize for Young Researchers: \url{http://www.informs.org/Recognize-Excellence/INFORMS-Community-Prizes-and-Awards2/Optimization-Society/Optimization-Prize-for-Young-Researchers}
		\item Underrepresented Minorities and Women Honoraria: \url{http://www.informs.org/Recognize-Excellence/INFORMS-Community-Prizes-and-Awards2/Simulation-Society/Underrepresented-Minorities-and-Women-Honoraria}
		\item Best Dissertation Proposal Competition (College on Organization Science, for Ph.D. proposals in organizational science): \url{http://www.informs.org/Recognize-Excellence/INFORMS-Community-Prizes-and-Awards2/College-on-Organization-Science/Best-Dissertation-Proposal-Competition}
		\item ISMS Doctoral Dissertation Proposal Competition (Society for Marketing Science, for Ph.D. proposals in marketing): \url{http://www.informs.org/Recognize-Excellence/INFORMS-Community-Prizes-and-Awards2/Society-for-Marketing-Science/ISMS-Doctoral-Dissertation-Proposal-Competition}
		\end{enumerate}
	\item Alice T. Schafer Mathematics Prize For Excellence in Mathematics by an Undergraduate Woman: \url{http://www.awm-math.org/schaferprize.html}
	\item European Prize in Combinatorics: \vspace{-0.2cm}
		\begin{enumerate} \itemsep -2pt
		\item The prize is established to recognize excellent contributions in Combinatorics by young European researchers (eligibility of EU) not older than 35. 
		\item \url{http://www.math.tu-berlin.de/EuroComb05/prize.html}
		\end{enumerate}
	\item The AMS-MAA-SIAM Frank and Brennie Morgan Prize for Outstanding Research in Mathematics by an Undergraduate Student: \url{http://www.maa.org/awards/morgan.html}; \url{http://www.ams.org/profession/prizes-awards/ams-prizes/morgan-prize}; and \url{http://www.siam.org/prizes/sponsored/morgan.php}
	\item --- --- --- --- --- --- --- --- --- --- --- --- --- --- --- --- --- --- --- --- --- --- --- --- --- --- --- --- --- --- ---
	% Lists of awards
	\item \colorbox{blue}{\bf Lists of awards}: \vspace{-0.2cm}
		\begin{enumerate} \itemsep -2pt
		\item Association for Women in Science: \url{http://www.awis.org/displaycommon.cfm?an=1&subarticlenbr=69}
		\item International Center for Scientific Research (CIRS): \url{http://www.cirs.net/indexenglish.htm}
		\end{enumerate}
	\end{itemize}
\end{enumerate}














%%%%%%%%%%%%%%%%%%%%%%%%%%%%%%%%%%%%%%%%%%%
\section{Funding Nonprofit Organizations}
\label{fundingnonprofitorg}

Funding nonprofit organizations (including colleges and universities): \vspace{-0.3cm}
\begin{enumerate} \itemsep -4pt
\item Alfred P. Sloan Foundation: \vspace{-0.3cm}
	\begin{enumerate} \itemsep -2pt
	\item Major Program Areas: \url{http://www.sloan.org/program/1}
	\item Apply for Grants: \url{http://www.sloan.org/apply}
	\end{enumerate}
\item The Commonwealth Fund: \vspace{-0.3cm}
	\begin{enumerate} \itemsep -2pt
	\item Grants \& Programs: \vspace{-0.2cm}
		\begin{enumerate} \itemsep -2pt
		\item \url{http://www.commonwealthfund.org/Grants-and-Programs.aspx}
		\item ``The Fund supports independent research on health and social issues and makes grants to improve health care practice and policy. We are dedicated to helping people become more informed about their health care and improving care for vulnerable populations such as children, the elderly, low-income families, minorities, and the uninsured.''
		\end{enumerate}
	\end{enumerate}
\item The Heinz Endowments (Howard Heinz Endowment \& Vira I. Heinz Endowment): \vspace{-0.3cm}
	\begin{enumerate} \itemsep -2pt
	\item \url{http://www.heinz.org/grants.aspx}
	\item grant-making programs (for non-profit organizations): \vspace{-0.2cm}
		\begin{enumerate} \itemsep -2pt
		\item Arts \& Culture
		\item Children, Youth \& Families
		\item Education
		\item Environment
		\item Innovation Economy
		\end{enumerate}
	\end{enumerate}
\item Ford Foundation: \vspace{-0.3cm}
	\begin{enumerate} \itemsep -2pt
	\item Grants: \vspace{-0.2cm}
		\begin{enumerate} \itemsep -2pt
		\item \url{http://www.fordfoundation.org/grants/}
		\item Individuals Seeking Fellowships: \vspace{-0.1cm}
			\begin{enumerate} \itemsep -1pt
			\item \url{http://www.fordfoundation.org/grants/individuals-seeking-fellowships}
			\item Ford Foundation Fellowship Programs: \url{http://sites.nationalacademies.org/PGA/FordFellowships/index.htm}
			\item Ford Foundation International Fellowships Program: \url{http://www.fordifp.net/}
			\end{enumerate}
		\item Organizations Seeking Grants: \url{http://www.fordfoundation.org/grants/organizations-seeking-grants}
		\item Other Philanthropic Resources: \url{http://www.fordfoundation.org/grants/other-philanthropic-resources}
		\item Grant Search Results (list of grants): \url{http://www.fordfoundation.org/grants/search}
		\end{enumerate}
	\end{enumerate}
\item The Rockefeller Foundation: \vspace{-0.3cm}
	\begin{enumerate} \itemsep -2pt
	\item Grants \& Grantees: \vspace{-0.2cm}
		\begin{enumerate} \itemsep -2pt
		\item \url{http://www.rockefellerfoundation.org/grants}
		\item What We Fund: \url{http://www.rockefellerfoundation.org/grants/what-we-fund}
		\item Resources for Grantseekers: Links to other Philanthropic Sources, \url{http://www.rockefellerfoundation.org/grants/resources-grantseekers}
		\end{enumerate}
	\end{enumerate}
\item Carnegie Corporation of New York: \vspace{-0.3cm}
	\begin{enumerate} \itemsep -2pt
	\item Grantseekers: \vspace{-0.2cm}
		\begin{enumerate} \itemsep -2pt
		\item \url{http://carnegie.org/grants/grantseekers/}
		\item What we fund: \url{http://carnegie.org/grants/grantseekers/what-we-fund/}
		\item What we don't fund: \url{http://carnegie.org/grants/grantseekers/what-we-dont-fund/}
		\end{enumerate}
		\item Grants database: \url{http://carnegie.org/grants/grants-database/} and \url{http://carnegie.org/grants/}
		\item (Past) individual foundation grants: \url{http://carnegie.org/publications/carnegie-reporter/single/view/article/item/221/}
	\end{enumerate}
\item The Kresge Foundation: \vspace{-0.3cm}
	\begin{enumerate} \itemsep -2pt
	\item fields of interest: \vspace{-0.2cm}
		\begin{enumerate} \itemsep -2pt
		\item health,
		\item the environment,
		\item community development,
		\item arts and culture,
		\item education, and
		\item human services
		\end{enumerate}
	\item Values Criteria (for grantmaking): \url{http://www.kresge.org/index.php/who/our_values_criteria/}
	\item funding methods: \vspace{-0.2cm}
		\begin{enumerate} \itemsep -2pt
		\item \url{http://www.kresge.org/index.php/how/index/}
		\item \url{http://www.kresge.org/index.php/our_funding_methods/index/}
		\end{enumerate}
	\item Challenge Grant: \vspace{-0.2cm}
		\begin{enumerate} \itemsep -2pt
		\item \url{http://www.kresge.org/index.php/our_funding_methods/challenge_grant_program/}
		\item ``The Kresge Foundation awards facilities capital as a challenge grant to help nonprofit organizations build their base of private financial support as they conduct capital campaigns to build or renovate their facilities.''
		\item ``Facilities capital challenge grants are awarded to organizations that cater specifically to the needs of poor, disadvantaged and disenfranchised in six program areas: Health Program, the Environment Program, Arts and Culture Program, Education Program, Human Services Program, and Community Development / Detroit Program.''
		\item ``Most challenge grant awards are made to U.S.-based organizations. On rare occasions, we award challenge grants to international organizations undertaking exceptional projects that align with the strategic objectives of a given program and advance Kresge's values.''
		\end{enumerate}
	\item Detroit Program: \vspace{-0.2cm}
		\begin{enumerate} \itemsep -2pt
		\item Kresge Arts Support: \url{http://www.kresge.org/index.php/what/detroit_program/kresge_arts_support/}
		\item Kresge Arts in Detroit: \url{http://www.kresge.org/index.php/what/detroit_program/kresge_arts_in_detroit/}
		\end{enumerate}
	\item Our Grants: \vspace{-0.2cm}
		\begin{enumerate} \itemsep -2pt
		\item \url{http://www.kresge.org/index.php/our_grants/index/}
		\item grants database: \url{http://maps.foundationcenter.org/grantmakers/index.php?gmkey=KRES002}
		\item Arts and Community Building: \vspace{-0.1cm}
			\begin{enumerate} \itemsep -1pt
			\item \url{http://www.kresge.org/index.php/what/arts_and_culture/arts_and_community_building#Community Arts}
			\item ``Cultural institutions and artists animate our communities, bring disparate people together to share common experiences, and help us imagine a better future. As the demographics of our communities become more diverse, artists and cultural institutions help us bridge differences and build cross-cultural understanding. As our economy struggles, creative enterprises and creative sector leaders offer hope for community renewal and new job development.''
			\item two pilot initiatives: College/Arts initiative, and the Community Arts initiative
			\item ``The pilot cities [for the Community Arts initiative] include Baltimore, Maryland; Birmingham, Alabama; Detroit, Michigan; St. Louis, Missouri; and Tucson, Arizona.''
			\item ``Grants for Arts and Community Building are by invitation only.''
			\end{enumerate}
		\end{enumerate}
	\end{enumerate}
\item New York Women's Foundation: \vspace{-0.3cm}
	\begin{enumerate} \itemsep -2pt
	\item Grant Information and Application: \vspace{-0.2cm}
		\begin{enumerate} \itemsep -2pt
		\item \url{http://www.nywf.org/grant.html}
		\item focus areas: \vspace{-0.1cm}
			\begin{enumerate} \itemsep -1pt
			\item Anti-Violence and Safety
			\item Economic Security
			\item Health, Sexual Rights and Reproductive Justice
			\end{enumerate}
		\item ``Grants usually range from \$50,000 to a maximum of \$70,000 [that last for a year, and can be renewed up to 5 years].''
		\end{enumerate}
	\end{enumerate}
\item The Foundation Center: \vspace{-0.3cm}
	\begin{enumerate} \itemsep -2pt
	\item Grantseekers: \url{http://foundationcenter.org/getstarted/}
	\item Find funders: \url{http://foundationcenter.org/findfunders/}
	\item GrantSpace$^{\rm SM}$: \vspace{-0.2cm}
		\begin{enumerate} \itemsep -2pt
		\item \url{http://grantspace.org/}
		\item ``GrantSpace$^{\rm SM}$ will help you gain the knowledge and skills you need to get grants, manage your nonprofit, and improve your community.''
		\item ``Established in 1956 and today supported by close to 550 foundations, the Foundation Center is a national nonprofit service organization recognized as the nation�s leading authority on organized philanthropy, connecting nonprofits and the grantmakers supporting them to tools they can use and information they can trust. Its audiences include grantseekers, grantmakers, researchers, policymakers, the media, and the general public. The Center maintains the most comprehensive database on U.S. grantmakers and their grants; issues a wide variety of print, electronic, and online information resources; conducts and publishes research on trends in foundation growth, giving, and practice; and offers an array of free and affordable educational programs.''
		\item Resources for Non-U.S. Grantseekers: \url{http://grantspace.org/Tools/Knowledge-Base/Resources-for-Non-U.S.-Grantseekers}
		\item Resources for Individual Grantseekers: \vspace{-0.1cm}
			\begin{enumerate} \itemsep -1pt
			\item \url{http://grantspace.org/Tools/Knowledge-Base/Individual-Grantseekers}
			\item \url{http://gtionline.foundationcenter.org/}
			\item General: \url{http://grantspace.org/Tools/Knowledge-Base/Individual-Grantseekers/General}
			\item Artists: \url{http://grantspace.org/Tools/Knowledge-Base/Individual-Grantseekers/Artists}
			\item Students: \url{http://grantspace.org/Tools/Knowledge-Base/Individual-Grantseekers/Students}
			\item Fiscal Sponsorship: \url{http://grantspace.org/Tools/Knowledge-Base/Individual-Grantseekers/Fiscal-Sponsorship}
			\item For-Profit Enterprises: \url{http://grantspace.org/Tools/Knowledge-Base/Individual-Grantseekers/For-Profit-Enterprises}
			\end{enumerate}
		\end{enumerate}
	\end{enumerate}
\item The Lemelson Foundation: \vspace{-0.3cm}
	\begin{enumerate} \itemsep -2pt
	\item \url{http://web.mit.edu/invent/w-foundation.html}
	\item Programs \& Grants: \url{http://www.lemelson.org/programs-grants}
	\item Grantmaking: \url{http://www.lemelson.org/grantmaking}
	\end{enumerate}
\item Partnership for Higher Education in Africa (PHEA): \vspace{-0.3cm}
	\begin{enumerate} \itemsep -2pt
	\item \url{http://www.foundation-partnership.org/} and \url{http://www.foundation-partnership.org/index.php?id=1}
	\item Grants Database: \url{http://www.foundation-partnership.org/index.php?id=2}
	\item Partnership Publications: \url{http://www.foundation-partnership.org/index.php?id=3}
	\end{enumerate}
\item Smithsonian Institution: \vspace{-0.3cm}
	\begin{enumerate} \itemsep -2pt
	\item Smithsonian Institution Traveling Exhibition Service (SITES): \vspace{-0.2cm}
		\begin{enumerate} \itemsep -2pt
		\item Smithsonian Community Grant program (supported by MetLife Foundation): \vspace{-0.1cm}
			\begin{enumerate} \itemsep -1pt
			\item \url{http://www.sites.si.edu/funding/grant2.htm}
			\item ``This program seeks to deepen connections between SITES' host venues and their communities by encouraging exhibitors to engage their local audiences in new and exciting ways while creating broader access to our exhibitions.''
			\item ``Under this new program, eligible SITES exhibitors may apply for up to \$5,000 for expenses related to public, educational programming produced in conjunction with a SITES exhibit. Exhibitors may choose to enhance current program offerings or to create a new program especially suited to the topic of the exhibition.''
			\end{enumerate}
		\end{enumerate}
	\end{enumerate}
\end{enumerate}
















%%%%%%%%%%%%%%%%%%%%%%%%%%%%%%%%%%%%%%%%%%%
\section{Technology-Related Public Policy}
\label{techpublicpolicy}

Resources for engagement in creating technology-related public policy: \vspace{-0.3cm}
\begin{enumerate} \itemsep -4pt
\item Yale Journal of Law \& Technology (YJOLT): \vspace{-0.3cm}
	\begin{enumerate} \itemsep -2pt
	\item \url{http://www.yjolt.org/}
	\item \url{http://wingenroth.org/}
	\end{enumerate}
\item ACM Public Policy Office: \vspace{-0.3cm}
	\begin{enumerate} \itemsep -2pt
	\item It represents ACM and its US Public Policy Council (USACM) on information technology policy issues that impact the computing field.
	\item It seeks to educate policymakers and the public about policies that will that foster innovations in computing and related disciplines in ways that benefit society.
	\item It also informs ACM's members and the public about policy developments through its weblog, Washington Update newsletter and articles in ACM publications.
	\item ACM US Public Policy Council (USACM): \url{http://usacm.acm.org/}
	\item ACM Committee on Computers and Public Policy (CCPP): \url{http://www.acm.org/public-policy/acm-committee-on-computers-and-public-policy}
	\item \url{http://www.acm.org/public-policy}
	\end{enumerate}
\item IEEE: \vspace{-0.3cm}
	\begin{enumerate} \itemsep -2pt
	\item IEEE-USA: \url{http://www.ieeeusa.org/policy/default.asp}
	\item Smart Grids: \url{http://smartgrid.ieee.org/public-policy}
	\end{enumerate}
\item Computing Community Consortium (CCC): \url{http://www.cra.org/ccc/}
\item Computing Research Association (CRA): \vspace{-0.3cm}
	\begin{enumerate} \itemsep -2pt
	\item \url{http://www.cra.org/}
	\item CRA Government Affairs: \url{http://www.cra.org/govaffairs/index.php}
	\end{enumerate}
\item EngineeringPolicy.org: \url{http://www.engineeringpolicy.org/}
\item Congressional Bi-Partisan Robotics Caucus: \url{http://www.roboticscaucus.org/}
\item Advisory Committee for the Congressional Research and Development $[$R\&D$]$ Caucus: \url{http://www.researchcaucus.org/}
\item {\it National Academies Press} (NAP), from the (US) {\it National Academies}: \url{http://www.nap.edu/}
\item {\it Coalition to Diversify Computing}: \url{http://www.cdc-computing.org/}
\item American Institute of Aeronautics and Astronautics (AIAA): \vspace{-0.3cm}
	\begin{enumerate} \itemsep -2pt
	\item \url{http://www.aiaa.org/content.cfm?pageid=7}
	\end{enumerate}
\item : \url{}
\item : \url{}
\item : \url{}
\item : \url{}
\item : \url{}
\item : \url{}
\item : \url{}
\item : \url{}
\end{enumerate}






%%%%%%%%%%%%%%%%%%%%%%%%%%%%%%%%%%%%%%%%%%%
\section{Feminist Outreach}
\label{feministoutreach}

Feminist outreach: \vspace{-0.3cm}
\begin{enumerate} \itemsep -4pt
\item Myra Sadker Foundation: \vspace{-0.3cm}
	\begin{enumerate} \itemsep -2pt
	\item $100+$ Ideas to Promote Gender Equity in Schools and Beyond: \url{http://www.sadker.org/100ideas.html}
	\item Gender Equity Activities: \url{http://www.sadker.org/WhatYouCanDo.html}
	\item Gender Equity Activities for Concerned Citizens: \url{http://www.sadker.org/GenderEquity-citizens.html}
	\item Gender Equity Activities for Families: \url{http://www.sadker.org/GenderEquity-family.html}
	\item Gender Equity Activities for Teachers: \vspace{-0.2cm}
		\begin{enumerate} \itemsep -2pt
		\item Early Childhood: \url{http://www.sadker.org/GenderEquity-teacher1.html}
		\item Primary Grades: \url{http://www.sadker.org/GenderEquity-teacher2.html}
		\item Upper Elementary: \url{http://www.sadker.org/GenderEquity-teacher3.html}
		\item Middle and High School: \url{http://www.sadker.org/GenderEquity-teacher4.html}
		\end{enumerate}
	\item Resources for feminism and links to web pages of feminist organizations: \url{http://www.sadker.org/ReadsLinks.html}
	\end{enumerate}
\item Feminist student organizations at colleges and universities: \vspace{-0.3cm}
	\begin{enumerate} \itemsep -2pt
	\item For example, at the University of Southern California, the organizations associated with feminist causes are: \vspace{-0.2cm}
		\begin{enumerate} \itemsep -2pt
		\item {\it USC Center for Women \& Men}: \url{http://www.usc.edu/student-affairs/cwm/links.html}
		\item {\it USC Women's Student Assembly}: \url{http://www-scf.usc.edu/~wsausc/Welcome.html}
		\end{enumerate}
	\end{enumerate}
\item International Women's Day: \url{http://www.internationalwomensday.com/}
\item Gender Across Borders: \vspace{-0.3cm}
	\begin{enumerate} \itemsep -2pt
	\item Feminism Resources: \url{http://www.genderacrossborders.com/feminist-resources/}
	\end{enumerate}
\item {\it V-Day}: \vspace{-0.3cm}
	\begin{enumerate} \itemsep -2pt
	\item \url{http://www.vday.org/}
	\item Organization that helps women plan and organize events to bring awareness about sexual assault, and what we can do to reduce sexual assault.
	\end{enumerate}
\item {\it Take Back The Night}: \vspace{-0.3cm}
	\begin{enumerate} \itemsep -2pt
	\item \url{http://www.takebackthenight.org/}
	\item Organization that helps women plan and organize events to bring awareness about sexual assault, and what we can do to reduce sexual assault. It also encourages sexual assault survivors to speak out about their sexual assaults, so that they would shame their perpetrators and let other women (and men) know that they is nothing to be ashamed of as sexual assault survivors. This is because the faults lie 100\% with the perpetrators, and not with the survivors.
	\end{enumerate}
\item {\it United Nations Development Fund for Women} (UNIFEM): \vspace{-0.3cm}
	\begin{enumerate} \itemsep -2pt
	\item \url{http://www.unifem.org/}
	\item Organization that addresses many challenges faced by girls and women.
	\end{enumerate}
\item {\it National Organization for Women}: \vspace{-0.3cm}
	\begin{enumerate} \itemsep -2pt
	\item \url{http://www.now.org/}
	\item Feminist organization in the US.
	\end{enumerate}
\item {\it A Woman's Nation}: \vspace{-0.3cm}
	\begin{enumerate} \itemsep -2pt
	\item \url{http://www.shriverreport.com/awn/}
	\item \url{http://awomansnation.com} or \url{http://www.shriverreport.com/}
	\end{enumerate}
\item {\it Peace Over Violence} is a non-profit, feminist, multicultural, volunteer organization dedicated to a building healthy relationships, families and communities free from sexual, domestic and interpersonal violence: \url{http://peaceoverviolence.org/}
\item SoulSpeakOut: \url{http://www.soulspeakout.org/resources/}
\item {\it Haven Hills}: \url{http://havenhills.org/}
%\item MaleSurvivor: \url{http://www.malesurvivor.org/}
\end{enumerate}












%%%%%%%%%%%%%%%%%%%%%%%%%%%%%%%%%%%%%%%%%%%
\section{Outreach: Professional Organizations}
\label{outreachproorgs}

Professional organizations: \vspace{-0.3cm}
\begin{enumerate} \itemsep -4pt
\item --- --- --- --- --- --- --- --- --- --- --- --- --- --- --- --- --- --- --- --- --- --- --- --- --- --- --- --- --- --- ---
\item \colorbox{blue}{\bf Professional Organizations for the Performance, Literary, and Visual Arts}
% Professional Organizations for the Performance, Literary, and Visual Arts
\item Americans for the Arts: \vspace{-0.3cm}
	\begin{enumerate} \itemsep -2pt
	\item \url{http://www.americansforthearts.org/get_involved/membership/default.asp}
	\item \url{http://www.artsusa.org/get_involved/membership/default.asp}
	\item Provides membership for organizations and individuals
	\item Individual membership are available for: \vspace{-0.2cm}
		\begin{enumerate} \itemsep -2pt
		\item Students
		\item Entrepreneurs (e.g., people in art management)
		\item Innovators
		\item Colleagues (artists)
		\end{enumerate}
	\item Americans for the Arts {\bf Emerging Leader Program}: \vspace{-0.2cm}
		\begin{enumerate} \itemsep -2pt
		\item \url{http://www.artsusa.org/networks/emerging_leaders/resources/default.asp}
		\item Has various resources for professional development, including mentoring
		\end{enumerate}
	\item Advocacy ({\bf public policy}): \url{http://www.artsusa.org/get_involved/advocate.asp}
	\end{enumerate}
\item --- --- --- --- --- --- --- --- --- --- --- --- --- --- --- --- --- --- --- --- --- --- --- --- --- --- --- --- --- --- ---
\item \colorbox{blue}{\bf Professional Organizations for the Musical Artists}
% Professional Organizations for the Musical Artists
\item The Recording Academy: \url{http://www.grammy365.com/join/membership-types}
\end{enumerate}













%%%%%%%%%%%%%%%%%%%%%%%%%%%%%%%%%%%%%%%%%%%
\section{Other Outreach}
\label{otheroutreach}

Other outreach: \vspace{-0.3cm}
\begin{enumerate} \itemsep -4pt
\item The Joy McCann Foundation: \vspace{-0.3cm}
	\begin{enumerate} \itemsep -2pt
	\item The Joy McCann Professorships in Law: \url{http://www.mccannfoundation.org/law.htm}
	\end{enumerate}
\item National Academy of Sciences: \vspace{-0.3cm}
	\begin{enumerate} \itemsep -2pt
	\item {\it Science \& Entertainment Exchange} program: \vspace{-0.2cm}
		\begin{enumerate} \itemsep -2pt
		\item \url{http://www.scienceandentertainmentexchange.org/}
		\item Provide science and engineering knowledge to help professionals in the entertainment industry create engaging storylines involving science and technology.
		\end{enumerate}
	\end{enumerate}
\item U.S. Department of State: \vspace{-0.3cm}
	\begin{enumerate} \itemsep -2pt
	\item Bureau of Educational and Cultural Affairs: \vspace{-0.2cm}
		\begin{enumerate} \itemsep -2pt
		\item Programs: \url{http://exchanges.state.gov/jexchanges/programs.html}
		\item Fulbright Classroom Teacher Exchange Program: \vspace{-0.1cm}
			\begin{enumerate} \itemsep -1pt
			\item \url{http://exchanges.state.gov/globalexchanges/fulbright-teacher-exchange-program.html}
			\item ``The Fulbright Classroom Teacher Exchange provides opportunities for primary and secondary teachers to exchange positions with colleagues in other countries. The participants contribute to mutual understanding by bringing international knowledge and perspectives to the U.S. and foreign classrooms, schools and communities. Full-time U.S. teachers can take part in either a year-long or semester-long direct exchange with a counterpart in another country.''
			\end{enumerate}
		\item FORTUNE/U.S. State Department Global Women's Mentoring Partnership: \vspace{-0.1cm}
			\begin{enumerate} \itemsep -1pt
			\item \url{http://exchanges.state.gov/citizens/professionals/fortunepartnership.html}
			\item ``This public-private partnership places talented, emerging women leaders from all over the world in mentoring programs with FORTUNE's Most Powerful Women Leaders.''
			\end{enumerate}
		\item Edward R. Murrow Program for Journalists: \vspace{-0.1cm}
			\begin{enumerate} \itemsep -1pt
			\item \url{http://exchanges.state.gov/ivlp/murrow.html}
			\item ``The Edward R. Murrow Program for Journalists invites rising international journalists to travel to the United States and examine journalistic principles and practices.''
			\end{enumerate}
		\item International Visitor Leadership Program: \vspace{-0.1cm}
			\begin{enumerate} \itemsep -1pt
			\item \url{http://exchanges.state.gov/ivlp/ivlp.html}
			\item ``These visits reflect the International Visitors' professional interests and support the foreign policy goals of the United States.''
			\item ``International Visitors are current or emerging leaders in government, politics, the media, education, the arts, business and other key fields.''
			\item ``International Visitors travel to the U.S. for carefully designed programs that reflect their professional interests and U.S. foreign policy goals. They travel in a variety of thematic programs, either individually or in groups, for up to three weeks. While in the U.S., International Visitors typically visit Washington, DC and three additional towns or cities that highlight the tremendous diversity of the U.S. They attend professional appointments with their American counterparts, learn about the U.S. system of government at the national, state and local levels, visit American schools, and experience American culture and social life.''
			\item ``There is no application for this program. International Visitors are selected and nominated annually by American Foreign Service Officers at U.S. Embassies around the world.''
			\end{enumerate}
		\item Au Pair: \vspace{-0.1cm}
			\begin{enumerate} \itemsep -1pt
			\item \url{http://exchanges.state.gov/jexchanges/programs/aupair.html}
			\item ``Through the Au Pair program, foreign nationals between 18 and 26 years of age participate in the home life of a host family. Au pairs provide limited childcare services for up to 12 months. An extension of 6, 9, or 12 months may be granted in certain cases.''
			\end{enumerate}
		\item Summer Work Travel: \vspace{-0.1cm}
			\begin{enumerate} \itemsep -1pt
			\item \url{http://exchanges.state.gov/jexchanges/programs/swt.html}
			\item ``In the summer work travel program, post-secondary students may enter the United States to work and travel during their summer vacation. Participants can be admitted to the program more than once. The maximum length of the program is four months.''
			\end{enumerate}
		\item Internship: \vspace{-0.1cm}
			\begin{enumerate} \itemsep -1pt
			\item \url{http://exchanges.state.gov/jexchanges/programs/intern.html}
			\item ``Internship programs are designed to allow foreign professionals to come to the United States to gain exposure to U.S. culture and to receive training in U.S. business practices in their chosen occupational field.  The maximum duration of an internship in any occupational field is 12 months. Upon completion of their exchange programs, participants are expected to return to their home countries.''
			\end{enumerate}
		\item Professional Exchanges Division: \vspace{-0.1cm}
			\begin{enumerate} \itemsep -1pt
			\item \url{http://exchanges.state.gov/citizens/profs.html}
			\item ``The Professional Exchanges division provides grants to U.S. nonprofit organizations to carry out exchange programs that support the professional development of foreign participants. The purpose of each exchange program is to engage with foreign leaders in critical professions, to demonstrate respect for foreign cultures, and to promote mutual understanding between the people of the United States and other countries.''
			\item ``Professional exchanges typically last several years and include internships, study tours or workshops in the United States and in the host country. Participants come from a variety of professions including education administrators, public servants, journalists, labor union officials, entrepreneurs, environmental leaders, jurists, lawyers, and civic leaders.''
			\end{enumerate}
		\end{enumerate}
	\end{enumerate}
\item Teach For All: \url{http://teachforallnetwork.org/}
\item --- --- --- --- --- --- --- --- --- --- --- --- --- --- --- --- --- --- --- --- --- --- --- --- --- --- --- --- --- --- ---
\item \colorbox{blue}{\bf Resources for Artists and Musicians}
% Resources for Artists and Musicians
\item League of American Orchestras and the Association of Performing Arts Presenters: \vspace{-0.3cm}
	\begin{enumerate} \itemsep -2pt
	\item {\it ArtistsfromAbroad.org}: \vspace{-0.2cm}
		\begin{enumerate} \itemsep -2pt
		\item \url{http://www.artistsfromabroad.org/}
		\item ``{\it ArtistsfromAbroad.org} features complete and up-to-date guidance on the visa process and tax treatment for foreign guest artists.''
		\end{enumerate}
	\end{enumerate}
\item Young Concert Artists, Inc. \vspace{-0.3cm}
	\begin{enumerate} \itemsep -2pt
	\item Composer Program (for American composers between 20 and 26 years of age): \url{http://www.yca.org/auditions/}
	\end{enumerate}
\item The John F. Kennedy Center for the Performing Arts: \vspace{-0.3cm}
	\begin{enumerate} \itemsep -2pt
	\item Mary Lou Williams Women in Jazz Emerging Artist Workshop: \vspace{-0.2cm}
		\begin{enumerate} \itemsep -2pt
		\item \url{http://www.kennedy-center.org/programs/jazz/womeninjazz/competition.html}
		\item ``The workshop provides female jazz artists ages 18 to 35 with an opportunity to explore and develop their artistry under the guidance of leading jazz artists and instructors. Each year, the workshop will focus on a specific instrument.''
		\item ``The 2011 Mary Lou Williams Women in Jazz Emerging Artist Workshop is open to advanced female jazz pianists who plan to pursue jazz performance as a career. Eligibility is exclusive to pianists who will be 18-35 years old on May 18, 2011 and have never recorded or been contracted to record as a leader or co-leader on a major label at the time of application. All applicants must be proficient in English.''
		\end{enumerate}
	\end{enumerate}
\item Grantmakers in the Arts (GIA): \vspace{-0.3cm}
	\begin{enumerate} \itemsep -2pt
	\item ``The mission of Grantmakers in the Arts (GIA) is to provide leadership and service to advance the use of philanthropic resources on behalf of arts and culture.''
	\item Arts Funding Topics: \url{http://www.giarts.org/arts-funding-topics}
	\end{enumerate}
\item The Dana Foundation: \vspace{-0.3cm}
	\begin{enumerate} \itemsep -2pt
	\item Arts Education program: \vspace{-0.2cm}
		\begin{enumerate} \itemsep -2pt
		\item Arts Education Grants: \vspace{-0.1cm}
			\begin{enumerate} \itemsep -1pt
			\item \url{http://www.dana.org/grants/BrowseArtsGrants.aspx}
			\item ``In 2001, The Dana Foundation created the Arts Education program with a sole focus of providing grants to support professional development for teaching artists and in-school arts specialists. The first several years of grants were to  programs in New York City, Washington, DC, Los Angeles and to organizations with a 50 mile radius of the three.''
			\item ``The Rural Initiative launched in 2006 with 6 grants awarded to organizations providing professional development in rural areas of the United States.''
			\end{enumerate}
		\end{enumerate}
	\end{enumerate}
\item writing/poetry contests: \vspace{-0.3cm}
	\begin{enumerate} \itemsep -2pt
	\item International 3-Day Novel Contest: \url{http://www.3daynovel.com/about/?contest}
	\end{enumerate}
\end{enumerate}




%%%%%%%%%%%%%%%%%%%%%%%%%%%%%%%%%%%%%%%%%%%
\section{Christian Colleges and Universities}
\label{christianunis}

Christian colleges and universities: \vspace{-0.3cm}
\begin{enumerate} \itemsep -4pt
\item List of Christian colleges and universities: \vspace{-0.3cm}
	\begin{enumerate} \itemsep -2pt
	\item Council for Christian Colleges and Universities (CCCU): \vspace{-0.2cm}
		\begin{enumerate} \itemsep -2pt
		\item \url{http://en.wikipedia.org/wiki/Council_for_Christian_Colleges_and_Universities}
		\item \url{http://www.cccu.org/}
		\end{enumerate}
	\item Christian College Consortium: \vspace{-0.2cm}
		\begin{enumerate} \itemsep -2pt
		\item \url{http://en.wikipedia.org/wiki/Christian_College_Consortium}
		\item \url{http://www.ccconsortium.org/}
		\end{enumerate}
	\end{enumerate}
\item California Baptist University, Riverside
\item Messiah College (Grantham, PA): \vspace{-0.3cm}
	\begin{enumerate} \itemsep -2pt
	\item Department of Engineering: \vspace{-0.2cm}
		\begin{enumerate} \itemsep -2pt
		\item \url{http://www.messiah.edu/departments/engineering/}
		\item B.S. programs in: \vspace{-0.1cm}
			\begin{enumerate} \itemsep -1pt
			\item Biomedical Engineering
			\item Computer Engineering
			\item Electrical Engineering
			\end{enumerate}
		\end{enumerate}
	\item Department of Information and Mathematical Sciences: \vspace{-0.2cm}
		\begin{enumerate} \itemsep -2pt
		\item \url{http://www.messiah.edu/departments/mathsci/index.html}
		\item Offers a B.A. Computer Science program
		\end{enumerate}
	\end{enumerate}
\end{enumerate}
























%%%%%%%%%%%%%%%%%%%%%%%%%%%%%%%%%%%%%%%%%
% Thoughts and Resources for Specific Areas and Topics
%	\input{./have2innovate/outreach}

%	http://sage.math.washington.edu/home/mvngu/network.html
%	http://www1.ccny.cuny.edu/advancement/news/Location-Determines-Social-Network-Influence.cfm
%	http://www.computerworld.com/s/article/9182138/MIT_builds_swimming_oil_eating_robots
%	http://creativecommons.org/licenses/by-sa/3.0/us/
%	http://www.nsf.gov/news/special_reports/science_nation/biologicalclocks.jsp
%	http://ascnetworksnetwork.org/
%	http://webscience.org/WSTNet.html
%	http://www.researcherid.com/Home.action
%	dmi unict.it
%	universit� degli studi di Palermo
%	universit� degli studi di Pavia
%	universit� degli studi di Mediterranea di Reggio Calabria
%	universit� degli studi di Napoli "Parthenope"
%	universit� degli studi di Parma
%	universit� degli studi di Pisa
%	universit� degli studi di Salerno
%	universit� degli studi di Sannio
%	universit� degli studi di Siena
%	universit� degli studi di Salento
%	universit� degli studi di Torino
%	universit� degli studi di Trieste
%	universit� degli studi di Tuscia
%	http://people.bu.edu/dougd/studentInfo.html
%	http://www.ee.ucla.edu/Prospective-home.htm	
%	http://www.youtube.com/user/messengerofChrist
%	http://160.97.10.132/comson/about/contact/comson-members/ciuprina
%	http://www.wordhacker.com/en/article/Barron_gre_list_a.htm
%	Compilers and Operating Systems for Low Power	
%	TU budapest
%	algorithmic-level synthesis
%	clustered voltage scaling / extended clustered voltage scaling
%	http://www.acm.org/globalizationreport/
%	Agnolotti del plin 

%	http://sage.math.washington.edu/home/mvngu/network.html
%	http://www1.ccny.cuny.edu/advancement/news/Location-Determines-Social-Network-Influence.cfm
%	http://www.computerworld.com/s/article/9182138/MIT_builds_swimming_oil_eating_robots
%	http://creativecommons.org/licenses/by-sa/3.0/us/
%	http://www.nsf.gov/news/special_reports/science_nation/biologicalclocks.jsp
%	http://ascnetworksnetwork.org/
%	http://webscience.org/WSTNet.html
%	http://www.researcherid.com/Home.action
%	dmi unict.it
%	universit� degli studi di Palermo
%	universit� degli studi di Pavia
%	universit� degli studi di Mediterranea di Reggio Calabria
%	universit� degli studi di Napoli "Parthenope"
%	universit� degli studi di Parma
%	universit� degli studi di Pisa
%	universit� degli studi di Salerno
%	universit� degli studi di Sannio
%	universit� degli studi di Siena
%	universit� degli studi di Salento
%	universit� degli studi di Torino
%	universit� degli studi di Trieste
%	universit� degli studi di Tuscia
%	http://people.bu.edu/dougd/studentInfo.html
%	http://www.ee.ucla.edu/Prospective-home.htm	
%	http://www.youtube.com/user/messengerofChrist
%	http://160.97.10.132/comson/about/contact/comson-members/ciuprina
%	http://www.wordhacker.com/en/article/Barron_gre_list_a.htm
%	Compilers and Operating Systems for Low Power	
%	TU budapest
%	algorithmic-level synthesis
%	clustered voltage scaling / extended clustered voltage scaling
%	http://www.acm.org/globalizationreport/
%	Agnolotti del plin 

%	http://sage.math.washington.edu/home/mvngu/network.html
%	http://www1.ccny.cuny.edu/advancement/news/Location-Determines-Social-Network-Influence.cfm
%	http://www.computerworld.com/s/article/9182138/MIT_builds_swimming_oil_eating_robots
%	http://creativecommons.org/licenses/by-sa/3.0/us/
%	http://www.nsf.gov/news/special_reports/science_nation/biologicalclocks.jsp
%	http://ascnetworksnetwork.org/
%	http://webscience.org/WSTNet.html
%	http://www.researcherid.com/Home.action
%	dmi unict.it
%	universit� degli studi di Palermo
%	universit� degli studi di Pavia
%	universit� degli studi di Mediterranea di Reggio Calabria
%	universit� degli studi di Napoli "Parthenope"
%	universit� degli studi di Parma
%	universit� degli studi di Pisa
%	universit� degli studi di Salerno
%	universit� degli studi di Sannio
%	universit� degli studi di Siena
%	universit� degli studi di Salento
%	universit� degli studi di Torino
%	universit� degli studi di Trieste
%	universit� degli studi di Tuscia
%	http://people.bu.edu/dougd/studentInfo.html
%	http://www.ee.ucla.edu/Prospective-home.htm	
%	http://www.youtube.com/user/messengerofChrist
%	http://160.97.10.132/comson/about/contact/comson-members/ciuprina
%	http://www.wordhacker.com/en/article/Barron_gre_list_a.htm
%	Compilers and Operating Systems for Low Power	
%	TU budapest
%	algorithmic-level synthesis
%	clustered voltage scaling / extended clustered voltage scaling
%	http://www.acm.org/globalizationreport/
%	Agnolotti del plin 


%%%%%%%%%%%%%%%%%%%%%%%%%%%%%%%%%%%%%%%%%%%
\section{Research Heuristics}
\label{researchheuristics}


The heuristic \proc{Find-Seminal-Papers} helps me find the seminal papers of an academic field, while the heuristic \proc{Keeping-Abreast-with-Researchers} helps me keep up with research trends (include determining emerging research trends and nascent research topics). \\

The references that I used for these heuristics are given as follows.\\
{\it Quora} [Online], ``How do I find the seminal papers of an academic field?''. Available online at: \url{http://www.quora.com/How-do-I-find-the-seminal-papers-of-an-academic-field}; last accessed on September 27, 2010. \\
{\it Quora} [Online], ``As a non-student, how do you keep up with interesting work in academia?''. Available online at: \url{http://www.quora.com/As-a-non-student-how-do-you-keep-up-with-interesting-work-in-academia}; last accessed on September 27, 2010. \\

Concerning \proc{Find-Seminal-Papers}, the sources of literature review include: \vspace{-0.3cm}
\begin{enumerate} \itemsep -4pt
\item journals
\item conference proceedings
\item Ph.D. theses
\item books
\item book chapters
\item technical reports
\item reading list of a reading group (or a seminar class in good graduate programs)
\item publication lists of researchers (preferably leading experts working in the field $\varphi$)
\item key publications of researchers who win achievement awards in the field $\varphi$: \vspace{-0.3cm}
	\begin{enumerate} \itemsep -2pt
	\item An example of this is the first conference paper on {\it Chaff} in {\it DAC 2001}.
	\end{enumerate}
\item repositories and digital libraries for good publications (e.g., {\it IEEE Xplore}, {\it ACM Digital Library}, {\it ScienceDirect}, {\it ProQuest Digital Dissertations}, {\it JSTOR} (short for Journal Storage), and {\it PubMed}), and domain-specific search engines (e.g., {\it Web of Science}, {\it Compendex}, {\it Scopus}, {\it Scirus}, {\it EMBASE}, and {\it Engineering Village})
\item generic scholarly search engines (e.g., Google Scholar)
\item academic social networking sites (e.g., Academia.edu)
\item reference lists of patents -- {\it Beware of patent infringement}
\end{enumerate}

Note that in some fields (e.g., social science), researchers publish foundational work in books, rather than in research/academic journals. As for the experts mentioned in line \ref{find-seminal-experts}, network and collaborate with these leading experts. The verification step in line \ref{find-seminal-verify} requires me to network with researchers in $\varphi$. While carrying out this verification process, ask these researchers for suggestions. For example, I can ask, ``What are the seminal publications/papers in $\varphi$?'' I note that carrying out this literature review is doing my homework before I seek help. If I approach others without doing my homework, it would be embarrassing if I don't understand what they are talking about; recall the brief conversation with Prof. Ofer Strichman (Technion) at {\it DAC 2009} about software verification and compiler validation. I am also aware that a high number of citations does not imply that a research paper is necessarily good. A significant, or considerable, proportion of the citations may be self-citations. Citations may refer to more recent publications, since not many people would read old publications. \\

Researchers who I can validate my list of seminal papers include: \vspace{-0.3cm}
\begin{enumerate} \itemsep -4pt
\item the list of researchers who I believe are the leading experts in $\varphi$
\item authors of my references
\item advanced Ph.D. students and postdocs: \vspace{-0.3cm}
	\begin{enumerate} \itemsep -2pt
	\item Good Ph.D. students and postdocs would have done a lot of literature research, and should know their research topic at least as well as their advisors.
	\end{enumerate}
\item editors of prestigious journals in $\varphi$
\item chairs of conferences and workshops (especially new/young workshops that are organized for nascent or fairly new research fields/topics)
\item keynote speakers at prestigious conferences
\item members of panel discussions at prestigious conferences
\item authors of famous textbooks in $\varphi$
\end{enumerate}



\begin{codebox}
\Procname{$\proc{Find-Seminal-Papers}(\varphi)$}
\zi	\Comment {\it Input $\varphi \gets $ research area that I am interested in}
\zi	\Comment {\it Output : seminal papers/publications in $\varphi$}
\li	\While I am still interested in $\varphi$
\li		\Do
		Perform a literature review on the research field $\varphi$.
\li		\While performing my literature review on $\varphi$
			\Do
\li			Determine which research publications are cited more frequently.
\li			Remember the authors who (co-)author the frequently cited research publications.
\li			Find out who the leading experts in $\varphi$ are, and look at their publications.	\label{find-seminal-experts}
\li			Verify my list of leading experts in $\varphi$ with researchers in $\varphi$.	\label{find-seminal-verify}
\zi
\zi			\Comment {\it Terminating Condition: When I have read more than enough papers/publications}
\zi			\Comment {\it (e.g., 50-200 journal/conference papers).}
\li			\End
\li		Take advantage of folksonomy, and use the tags of {\it arXiv} and {\it Mendeley}
\zi		(and perhaps, {\it CiteULike}) to locate new/recent submissions in $\varphi$.
\li		Use {\it Google Scholar} to help me keep track of the number of citations
\zi		for a given research publication.
\zi		\End
\end{codebox}

{\color{blue} A general rule for using \proc{Keeping-Abreast-with-Researchers} is that the selected conference or journal papers shall be well-written.} If the paper is difficult to read (as in poorly written), it is hard to follow the rest of their work. Also, just because a paper is technically complicated, it does not mean that it is a good paper. Concerning line \ref{keeping-abreast-pub-list} of \proc{Keeping-Abreast-with-Researchers}, I can take advantage of resources such as the web pages of researchers and research groups/labs, {\it DBLP}, {\it IEEE Xplore}, and {\it ACM Digital Library}. To find out about recently published books (see line \ref{keeping-abreast-books}), look at the collection of books in bookstores that sell academic books (course textbooks and scholarly books). In particular, check out the student bookstores of good colleges and universities, as well as the pamphlets of academic publishers, such as {\it John Wiley {\rm \&} Sons} (or {\it Wiley}), {\it Morgan Kaufmann Publishers}, {\it Springer Science+Business Media} (or {\it Springer}), {\it Elsevier}, and {\it IOS Press}. In addition, look at the collections of university presses, such as {\it MIT Press}, {\it Cambridge University Press}, and {\it Oxford University Press}. Elaborating on line \ref{keeping-abreast-follow-conf}, make a shortlist of 5-10 conference papers from the advanced program of conferences (which includes a list of accepted papers), and email the authors for copies of their papers. After each major conference, ask friends who have attended that conference to discuss their top 3-5 presentations. Subsequently, read the corresponding conference papers for these presentations.

\begin{codebox}
\Procname{$\proc{Keeping-Abreast-with-Researchers}(\beta)$}
\li	\Comment {\it Input: $\beta \gets $ research area that I am interested in}
\li	\Comment {\it Output: conference papers, journal papers, {\rm \&} patents, }
\li	\While I am interested in $\beta$
		\Do
\li		Keep track of the publication record of researchers.		\label{keeping-abreast-pub-list}
\li		Take advantage of folksonomy, and use the tags of {\it arXiv} and {\it Mendeley}.
\zi		(and perhaps, {\it CiteULike}) to locate new/recent submissions in $\beta$.
\li		For each publication, determine its key references and look them up.
\zi		\Comment {\it That is, form the directed graph of research publications in $\beta$.}
\zi		\Comment {\it This will lead me to other interesting papers and new researchers to keep track of.}
\li		Find recent books that interest me, and look up their primary references/citations on
\zi		the best techniques/methodologies (or recent techniques/methodologies).	\label{keeping-abreast-books}
\li		Keep track of the specialists in $\beta$ and the relevant topics on {\it Quora} and {\it Facebook}.
\li		Ask open-ended questions on {\it Quora} and {\it Facebook}, which specialists may answer.
\li		Information about nascent or fairly new research fields/topics may emerge on {\it Facebook}.
\zi		\Comment {\it Information about new/recent tools and computer languages}
\zi		\Comment {\it may also emerge on {\it Facebook} and blogs (and perhaps {\it Wikipedia}).}
\li		Follow the major conferences in $\beta$ casually.		\label{keeping-abreast-follow-conf}
\li		Attend academic talks whenever possible, especially at good research universities.
\zi		\Comment {\it High-tech companies with research labs have their own academic talks,}
\zi		\Comment {\it which are closed to public.}
\li		Spend at least 1 day per month digging deep into the archive, \& recursively follow the citations.
\li		Be familiar with the rough work of superstars in $\beta$.
\li		Follow the work of good research labs in $\beta$.
\li		Pay attention to keynote speeches and panel discussions at major conferences.
\li		Look at the roadmaps for $\beta$ (if any), which present challenges to be tackled
\zi		under long-term research (8-15 years).
\li		Check out workshops for nascent or fairly new research fields/topics.
\li		Look at handbooks, which provide a wealth of information on an academic field or research topic.
\li		Look at the {\it Future Work} sections of Ph.D. theses in $\beta$.
\zi		\Comment {\it Especially if they win some sort of dissertation award (e.g., that from ACM).}
\li		Check out ``generic'' journal publications that have good survey papers.
\zi		\Comment {\it E.g., ``Proceedings of the IEEE'' or ``Journal of the ACM''.}
\li		Look at the web pages of government agencies and professional organizations,
\zi		which may include roadmaps.
\li		Look at the publication lists (and job advertisements) of start-ups (if any).
\li		Ask researchers in $\beta$ what are they reading.
\li		Look for highly-rated books in $\beta$, and go through the sorted list (based on year of publication).
\li		Keep track of good blogs written by good researchers (i.e., academic blogs)
\zi		and industry analysts via RSS feeds.
\zi		\End
\end{codebox}




%%%%%%%%%%%%%%%%%%%%%%%%%%%%%%%%%%%%%%%%%%%
\section{Research Resources}
\label{researchresources}

Resources for research, and educational and learning material: \vspace{-0.3cm}
\begin{enumerate} \itemsep -4pt
\item International Technology Roadmap for Semiconductors (ITRS): \url{http://public.itrs.net/}
\item CATRENE (Cluster for Application and Technology Research in Europe on NanoElectronics): \url{http://www.catrene.org/index.php}
\item Social networking sites: \vspace{-0.3cm}
	\begin{enumerate} \itemsep -2pt
	\item For academics, see \url{http://www.academia.edu/}
	\item ResearchGATE Corporation: \url{http://www.researchgate.net/}
	\item MyNetResearch: \vspace{-0.2cm}
		\begin{enumerate} \itemsep -2pt
		\item Features: \url{http://www.mynetresearch.com/Features.aspx}
		\item Resources: \url{http://www.mynetresearch.com/ResearcherLinks.aspx}
		\item Can facilitate project management, and provides public and private file sharing
		\item \url{http://www.mynetresearch.com/}
		\end{enumerate}
	\item {\it weSRCH.com}: \vspace{-0.2cm}
		\begin{enumerate} \itemsep -2pt
		\item ``For professionals who engage in the fields of High Tech, Green Tech, and Medicine''
		\item It has an online forum, and is a resource for news in the high tech, green tech, and bio tech industries.
		\item \url{http://www.wesrch.com/}
		\end{enumerate}
	\end{enumerate}
\item Educational resources and ``open-source'' textbooks: \vspace{-0.3cm}
	\begin{itemize} \itemsep -2pt
	\item OpenCourseWare Consortium: \vspace{-0.2cm}
		\begin{enumerate} \itemsep -2pt
		\item \url{http://www.ocwconsortium.org/}
		\item OCW Finder: \url{http://www.ocwfinder.org/}
		\item MIT OpenCourseWare: \url{http://ocw.mit.edu/index.htm}
		\end{enumerate}
	\item Connexions: \url{http://cnx.org/}
	\item Nature Publishing Group (a division of Macmillan Publishers Limited): \vspace{-0.2cm}
		\begin{enumerate} \itemsep -2pt
		\item {\it Scitable} is a free science library and personal learning tool brought to you by Nature Publishing Group, the world's leading publisher of science. Available online at: \url{http://www.nature.com/scitable}; last accessed on January 1, 2010.: \vspace{-0.1cm}
			\begin{enumerate} \itemsep -1pt
			\item {\it eBooks}. Nature Education e-books are intuitive introductions to a range of topics relevant to science students, young scientists, and science enthusiasts of all ages. Available online at: \url{http://www.nature.com/scitable/topics}; last accessed on January 1, 2010.
			\end{enumerate}
		\end{enumerate}
	\item {\it Quora}, which is a collection of questions and answers: \url{http://www.quora.com/}
	\item {\it SelfSolved} is a personal and social information repository for people who solve problems: \url{http://selfsolved.com/}
	\item Flat World Knowledge: \url{http://www.flatworldknowledge.com/}
	\item Liquid Publications (LiquidPub): \url{http://project.liquidpub.org/}
	\item OER Commons: \vspace{-0.2cm}
		\begin{enumerate} \itemsep -2pt
		\item A project of the Institute for the Study of Knowledge Management in Education, ISKME
		\item \url{http://www.oercommons.org/}
		\end{enumerate}
	\item NIXTY: \url{http://nixty.com/}
	\item Learning Is For Everyone: \url{http://www.learningis4everyone.org/}
	\item BCcampus OER Portal: \url{http://freelearning.ca/about/}
	\item iBerry: \vspace{-0.2cm}
		\begin{enumerate} \itemsep -2pt
		\item \url{http://iberry.com/}
		\item \url{http://iberry.com/cms/} and \url{http://iberry.com/cms/OCW.htm}
		\end{enumerate}
	\item InTech: \vspace{-0.2cm}
		\begin{enumerate} \itemsep -2pt
		\item \url{http://www.intechweb.org/}
		\item \url{http://www.intechopen.com/}
		\end{enumerate}
	\item Open Educational Resources (OER): \url{http://wiki.creativecommons.org/OER}
	\item Open Learning Initiative: \url{http://oli.web.cmu.edu/openlearning/}
	\item Peer 2 Peer University (P2PU): \url{http://www.p2pu.org/}
	\item Curtis J. Bonk, Book Resources [ for ] ``The World Is Open: How Web Technology Is Revolutionizing Education,'' Jossey-Bass (a Wiley imprint), San Francisco, CA. Available at: \url{http://worldisopen.com/resources.php}; last accessed on September 4, 2010.
	\item Wikiversity (from the Wikimedia Foundation): \url{http://en.wikiversity.org/wiki/Wikiversity:Main_Page}
	\item Einztein Knowledge Network: \url{http://www.einztein.com/}
	\item Saylor Foundation: \url{http://www.saylor.org/}
	\item Knewton: \url{http://www.knewton.com/}
	\item wePapers: \url{http://www.wepapers.com/}
	\item CourseSmart: \url{http://www.coursesmart.com/}
	\item WikiEducator: \url{http://wikieducator.org/Main_Page}
	\item Community College Consortium for Open Educational Resources: \url{http://oerconsortium.org/}
	\item Online encyclopedia: \url{http://openresearch.org/wiki/Main_Page}
	\item Google: \vspace{-0.2cm}
		\begin{enumerate} \itemsep -2pt
		\item Google ebookstore (Google eBooks): \url{http://books.google.com/ebooks}
		\item Google Books Ngram Viewer: \vspace{-0.1cm}
			\begin{enumerate} \itemsep -1pt
			\item \url{http://ngrams.googlelabs.com/}
			\item Datasets: \url{http://ngrams.googlelabs.com/datasets}
			\end{enumerate}
		\end{enumerate}
	\item {\bf \color{blue} The Assayer} (has textbooks in many topics): \url{http://theassayer.org/}
	\item {\bf \color{blue} Culturomics} Resources: \url{http://www.culturomics.org/Resources/links}
	\item The Dana Foundation: \vspace{-0.2cm}
		\begin{enumerate} \itemsep -2pt
		\item \url{http://www.dana.org/news/publications/}
		\item Literature review on brain research that is updated annually
		\end{enumerate}
	\item Cramster: \vspace{-0.2cm}
		\begin{enumerate} \itemsep -2pt
		\item \url{http://www.cramster.com/}
		\item Offers help in electrical engineering and computer science, among other subjects (e.g., writing, mathematics, and physics)
		\end{enumerate}
	\item University of Southern California: \vspace{-0.2cm}
		\begin{enumerate} \itemsep -2pt
		\item Material for classes (mostly in engineering): \url{http://www-classes.usc.edu/}
		\item USC Andrew and Erna Viterbi School of Engineering: \vspace{-0.1cm}
			\begin{enumerate} \itemsep -1pt
			\item \url{http://www-classes.usc.edu/engr/}
			\item Ming Hsieh Department of Electrical Engineering: \url{http://www-classes.usc.edu/engr/ee-s/}
			\end{enumerate}
		\end{enumerate}
	\item University of Washington: \vspace{-0.2cm}
		\begin{enumerate} \itemsep -2pt
		\item Department of Mathematics: \vspace{-0.1cm}
			\begin{enumerate} \itemsep -1pt
			\item List of open source projects in symbolic and numerical computing, and {\bf online computer science and mathematics books} (by Minh Van Nguyen): \url{http://sage.math.washington.edu/home/mvngu/misc.html}
			\end{enumerate}
		\end{enumerate}
	\item Princeton University: \vspace{-0.2cm}
		\begin{enumerate} \itemsep -2pt
		\item UChannel (University Channel): Contains ``videos of academic lectures and events'', \url{http://uc.princeton.edu/main/}; \url{http://uc.princeton.edu/main/index.php/about-us-mainmenu-2}; or \url{http://uc.princeton.edu/main/index.php/home-mainmenu-1}
		\item Office of Information Technology: \vspace{-0.1cm}
			\begin{enumerate} \itemsep -1pt
			\item Event Streaming WebMedia: Source of academic lectures, \url{http://hulk03.princeton.edu:8080/WebMedia/lectures/}
			\end{enumerate}
		\item Educational Technologies Center: Almagest, \url{http://etcweb.princeton.edu/almagest3/}
		\end{enumerate}
	\item Virginia Tech (Virginia Polytechnic Institute and State University): \vspace{-0.2cm}
		\begin{enumerate} \itemsep -2pt
		\item Department of Computer Science: \vspace{-0.1cm}
			\begin{enumerate} \itemsep -1pt
			\item Graduate classes: \url{http://www.cs.vt.edu/graduate/courses}
			\item Undergraduate classes: \url{http://www.cs.vt.edu/undergraduate/courses}
			\item List of all classes: \url{http://ei.cs.vt.edu/courses.html}
			\end{enumerate}
		\end{enumerate}
	\item University of Adelaide: \vspace{-0.2cm}
		\begin{enumerate} \itemsep -2pt
		\item eBooks@Adelaide: \vspace{-0.1cm}
			\begin{enumerate} \itemsep -1pt
			\item Classic Works of Literature, Philosophy, Science, History, and Exploration and Travel: \url{http://ebooks.adelaide.edu.au/}
			\end{enumerate}
		\end{enumerate}
	\item Colorado School of Mines: \vspace{-0.2cm}
		\begin{enumerate} \itemsep -2pt
		\item Department of Physics: \vspace{-0.1cm}
			\begin{enumerate} \itemsep -1pt
			\item Samizdat Press, \url{http://samizdat.mines.edu/}, contains books in: \vspace{-0.1cm}
				\begin{itemize} \itemsep -1pt
				\item Linear Algebra and Multidimensional Geometry 
				\item Differential Geometry
				\item Tensor Analysis
				\item Classical Electrodynamics and the Theory of Relativity
				\item Continuum Mechanics 
				\item Genetic Algorithms
				\item Geomechanics
				\item Greek Seismology
				\item Geophysical Inverse Theory 
				\item Geophysics
				\item finite element analysis
				\item Inverse Problem Theory 
				\end{itemize}
			\end{enumerate}
		\end{enumerate}
	\item IEEE: \vspace{-0.2cm}
		\begin{enumerate} \itemsep -2pt
		\item IEEE eLearning Library: \url{http://ieee-elearning.org/}
		\item IEEE Career Resources: \url{http://ieee.org/education_careers/careers/index.html}
		\item IEEE Online Professional Development: \url{http://ieee.org/education_careers/education/prodev/index.html}
		\item IEEE Computer Society: \vspace{-0.1cm}
			\begin{enumerate} \itemsep -1pt
			\item e-learning campus: \url{http://www.computer.org/portal/web/e-learning/home}
			\end{enumerate}
		\end{enumerate}
	\item ACM: \vspace{-0.2cm}
		\begin{enumerate} \itemsep -2pt
		\item ACM Tech Pack: \url{http://techpack.acm.org/}
		\item ACM Learning Center (has online books): \url{http://learning.acm.org/}
		\end{enumerate}
	\item IBM: \vspace{-0.2cm}
		\begin{enumerate} \itemsep -2pt
		\item Program Analysis Group; IBM T.J. Watson Research Center: \vspace{-0.1cm}
			\begin{enumerate} \itemsep -1pt
			\item Eran Yahav, {\it misc}, IBM. Available online at: \url{http://www.research.ibm.com/people/e/eyahav/misc.html}; last accessed on September 28, 2010. [ Also has resources about learning German and how to play the guitar. ]
			\end{enumerate}
		\end{enumerate}
	\item U.S. National Academies: \vspace{-0.2cm}
		\begin{enumerate} \itemsep -2pt
		\item National Academies Press: \url{http://www.nap.edu/}
		\item Proceedings of the National Academy of Sciences: \url{http://www.pnas.org/}
		\item Publications: \url{http://www.nationalacademies.org/publications/}
		\end{enumerate}
	\item The National Science Foundation: \vspace{-0.2cm}
		\begin{enumerate} \itemsep -2pt
		\item {\it Science and Engineering Statistics: Publications, data, and analyses about the nation's science and engineering resources}, Division of Science Resources Statistics (SRS). Available online at: \url{http://www.nsf.gov/statistics/}; last accessed on September 25, 2010.
		\item {\it Science and Engineering Indicators: 201X}, Division of Science Resources Statistics, National Science Foundation, Arlington, VA, January 2010. Available online at: \url{http://www.nsf.gov/statistics/indicators/}; last accessed on September 25, 2010. [ The 2010 issue is available at: \url{http://www.nsf.gov/statistics/seind10/}. ]
		\end{enumerate}
	\item Ewing Marion Kauffman Foundation: \vspace{-0.2cm}
		\begin{enumerate} \itemsep -2pt
		\item Links: \url{http://sciencecommons.org/resources/links/}
		\end{enumerate}
	\item Open Culture: \vspace{-0.2cm}
		\begin{enumerate} \itemsep -2pt
		\item \url{http://www.openculture.com/}
		\item ``free cultural \& educational media on the web''
		\end{enumerate}
	\item R\&D Magazine (from Advantage Business Media): \vspace{-0.2cm}
		\begin{enumerate} \itemsep -2pt
		\item \url{http://www.rdmag.com/}
		\item R\&D 100 Awards 201X: \url{http://www.rdmag.com/Awards/RD-100-Awards/R-D-100-Awards/}
		\end{enumerate}
	\item The Academy of American Poets: \vspace{-0.2cm}
		\begin{enumerate} \itemsep -2pt
		\item \url{http://www.poets.org/index.php}
		\item ``The Road Not Taken,'' by Robert Frost. Available online at: \url{http://www.poets.org/viewmedia.php/prmMID/15717}; last accessed on September 28, 2010.
		\end{enumerate}
	\item William Stallings: \vspace{-0.2cm}
		\begin{enumerate} \itemsep -2pt
		\item Xentrik: \vspace{-0.1cm}
			\begin{enumerate} \itemsep -1pt
			\item Tutorial for creating and maintaining web pages: \url{http://www.xentrik.net/}
			\end{enumerate}
		\item Resources for CS students: \vspace{-0.1cm}
			\begin{enumerate} \itemsep -1pt
			\item {\it Computer Science Student Resource Site}, \url{http://www.computersciencestudent.com/}
			\item {\it Computer Science Student Resource Site: How-To}, \url{http://www.computersciencestudent.com/SS/SS-howto.html}
			\item {\it Computer Science Student Resource Site: Computer Science Careers}, \url{http://www.computersciencestudent.com/SS/SS-career.html}
			\end{enumerate}
		\end{enumerate}
	\end{itemize}
\item Videos of lectures, research talks/presentations, and seminars: \vspace{-0.3cm}
	\begin{enumerate} \itemsep -2pt
	\item {\it iTunes U}: \url{http://www.apple.com/education/itunes-u/}
	\item {\it YouTube}: \vspace{-0.2cm}
		\begin{enumerate} \itemsep -2pt
		\item \url{http://www.youtube.com/}
		\item YouTube EDU: \url{http://www.youtube.com/education?b=400}
		\end{enumerate}
	\item \url{http://videolectures.net/}
	\item Academic Earth: \url{http://www.academicearth.org/}
	\item MyNetResearch Videos: \url{http://videos.mynetresearch.com/Default.aspx}
	\item Computer Science: \vspace{-0.2cm}
		\begin{enumerate} \itemsep -2pt
		\item University of Illinois at Urbana-Champaign (UIUC): \vspace{-0.1cm}
			\begin{enumerate} \itemsep -1pt
			\item College of Engineering; Department of Computer Science: \url{http://cs.illinois.edu/lectures}
			\end{enumerate}
		\end{enumerate}
	\item Mathematics: \vspace{-0.2cm}
		\begin{enumerate} \itemsep -2pt
		\item Institute for Mathematics and its Applications (IMA) at the University of Minnesota, Twin Cities: \url{http://www.ima.umn.edu/videos/}
		\end{enumerate}
	\end{enumerate}
\item Xuropa: \vspace{-0.3cm}
	\begin{enumerate} \itemsep -2pt
	\item the electronic design online community; the {\tt Facebook} for people working in EDA, IP design/verification/services, (rest of the) semiconductor industry (including design, CAD, verification, test, and device engineers), software industry, and systems people (including those in embedded systems, automobile industry, and medical systems)
	\item see \url{http://xuropa.com/}
	\end{enumerate}
\end{enumerate}


%%%%%%%%%%%%%%%%%%%%%%%%%%%%%%%%%%%%%%%%%%%
\section{Resources for Market Research as well as Social, Economic, and Political Issues}
\label{resourcesmktresearch}

Resources for market research as well as social, economic, and political issues: \vspace{-0.3cm}
\begin{enumerate} \itemsep -4pt
\item Gary Smith EDA (GSEDA): \url{http://www.garysmitheda.com/}
\item VLSI Research Inc: \url{https://www.vlsiresearch.com/}
\item Future Horizons: \url{http://www.futurehorizons.com/}
\item Gartner [or Gartner Dataquest]: \url{http://www.gartner.com/technology/home.jsp}
\item iSuppli Corporation: \url{http://www.isuppli.com/Pages/Home.aspx}
\item Linley Group: \vspace{-0.3cm}
	\begin{enumerate} \itemsep -2pt
	\item {\it Processor Watch}, ``a free electronic newsletter on high-performance microprocessors'': \url{http://www.mdronline.com/processor_watch/}
	\item \url{http://www.linleygroup.com/}
	\end{enumerate}
\item Semico Research Corp: \url{http://www.semico.com/}
\item Other organizations/companies that provide forecasts for the semiconductor market: \vspace{-0.3cm}
	\begin{enumerate} \itemsep -2pt
	\item Semiconductor Intelligence: \url{http://www.semiconductorintelligence.com/}
	\item Semiconductor Industry Capacity Statistics (SICAS): \url{http://www.sicas.info/}
	\item Semiconductor Equipment and Materials International (SEMI\textregistered): \url{http://www.semi.org/en/index.htm}
	\item WSTS (World Semiconductor Trade Statistics): \url{http://www.wsts.org/}
	\item Semiconductor Industry Association (SIA): \url{http://www.sia-online.org/}
	\item International Data Corporation (IDC): \url{http://www.idc.com/}
	\item IC Insights: \vspace{-0.2cm}
		\begin{enumerate} \itemsep -2pt
		\item {\it The McClean Report: A Complete Analysis and Forecast of the Integrated Circuit Industry}: \url{http://www.icinsights.com/prodsrvs/mcclean/mcclean.html}
		\item {\it IC Market Drivers: A Study of Emerging and Major End-Use Applications Fueling Demand for Integrated Circuits} helps identify opportunities for growth and evaluates the potential for new applications that are expected to fuel the market for integrated circuits through 2013. See \url{http://www.icinsights.com/prodsrvs/marketdrivers/marketdrivers.html}.
		\item {\it O$\cdot$S$\cdot$D Report: A Market Analysis and Forecast for Optoelectronics, Sensors, and Discretes} provides information about the end-use application, regional market analysis, leading supplier rankings, device history, and technology trends for: \vspace{-0.2cm}
			\begin{itemize} \itemsep -2pt
			\item optoelectronics: \vspace{-0.1cm}
				\begin{itemize} \itemsep -1pt
				\item CCD and CMOS Image Sensors
				\item Laser Transmitters and Pick-Ups (for fiber-optic networks)
				\item Solid-State Lamps and LEDs
				\item Infrared Devices
				\item Couplers, Isolators
				\item Optical Switches
				\item Digital Character Displays
				\end{itemize}
			\item sensors/actuators (including MEMS-based): \vspace{-0.1cm}
				\begin{itemize} \itemsep -1pt
				\item Pressure Sensors
				\item Acceleration and Yaw Sensors
				\item Magnetic-Field Sensors
				\item Temperature Sensors
				\item Fingerprint ID Chips
				\item Actuators
				\end{itemize}
			\item discretes: \vspace{-0.1cm}
				\begin{itemize} \itemsep -1pt
				\item Power Transistors and Modules (IGBTs, power IGBTs, and power FETs)
				\item Small-Signal Transistors
				\item Switching Transistors
				\item Diodes, Rectifiers, and Thyristors
				\item RF/Microwave Transistors and Modules
				\end{itemize}
			\end{itemize}
		\item \url{http://www.icinsights.com/prodsrvs/osdreport/osdreport.html}
		\item {\it MEMS 2010: A Realistic Look Beyond the Hype} (Special Study). See \url{http://www.icinsights.com/prodsrvs/specialstudies/mems/mems.html}.
		\item {\it Global Wafer Capacity Analysis and Forecast} (Special Study). See \url{http://www.icinsights.com/prodsrvs/specialstudies/globalcapacity/globalcapacity.html}.
		\item {\it Strategic Reviews Online} offers quick access to thorough examinations of companies involved in the design and manufacture of integrated circuits. Whether a supplier owns a fab or is fabless, has sales of several million or several billion dollars,  {\it Strategic Reviews Online} provides detailed reviews of the operations and activities of more than 200 of the world's established and emerging IC companies (totaling about 770 printed pages worth of valuable information). Access {\it Strategic Reviews Online: Extensive Profiles of the World's IC Manufacturers and Fabless Suppliers} @ \url{http://www.icinsights.com/prodsrvs/reviews/reviews.html}.
		\item \url{http://www.icinsights.com/}
		\end{enumerate}
	\item ABI Research: \url{http://www.abiresearch.com/home.jsp}
	\item HTE Research, Inc.: \url{http://www.hteresearch.com/}; also see {\it InsideChips} @ \url{http://www.insidechips.com/}
	\item In-Stat, LLC: \url{http://www.instat.com/}
	\item Chipworks: \url{http://www.ice-corp.com/}
	\end{enumerate}
\item Yole D{\'{e}}veloppement: \url{http://www.yole.fr/}
\item Lux Research (solar, nanomaterials, alternative power, water, biosciences): \url{http://www.luxresearchinc.com/}
\item Pike Research (global clean technology markets: smart energy, clean transportation, clean industry, and building efficiency): \url{http://www.pikeresearch.com/}
\item Deloitte Consulting: \vspace{-0.3cm}
	\begin{enumerate} \itemsep -2pt
	\item see market survey of industries @ \url{http://www.deloitte.com/view/en_US/us/industries/index.htm}
	\item \url{http://www.deloitte.com/view/en_US/us/Insights/centers/index.htm}
	\item see Deloitte Review @ \url{http://www.deloitte.com/view/en_US/us/Insights/Browse-by-Content-Type/deloitte-review/index.htm}
	\item see Deloitte Research @ \url{http://www.deloitte.com/view/en_US/us/Insights/Browse-by-Content-Type/research/index.htm}
	\item see Case Studies @ \url{http://www.deloitte.com/view/en_US/us/Insights/Browse-by-Content-Type/Case-Studies/index.htm}
	\item see Deloitte Technology Services Consulting @ \url{http://www.deloitte.com/view/en_US/us/Services/consulting/technology-consulting/index.htm}
	\item see Deloitte Debates @ \url{http://www.deloitte.com/view/en_US/us/Insights/Browse-by-Content-Type/deloitte-debates/index.htm}
	\end{enumerate}
\item Morgan Stanley: \vspace{-0.3cm}
	\begin{enumerate} \itemsep -2pt
	\item Technology Research: \url{http://www.morganstanley.com/institutional/techresearch/}.
	\item Journal of Applied Corporate Finance: \url{http://www.morganstanley.com/views/jacf/index.html}.
	\item Perspectives: \url{http://www.morganstanley.com/views/perspectives/index.html}.
	\item Global Strategy Roundup: \url{http://www.morganstanley.com/views/gsr/index.html}.
	\item Global Economic Forum: \url{http://www.morganstanley.com/views/gef/index.html}.
	\end{enumerate}
\item McKinsey \& Company: See McKinsey Quarterly @ \url{http://www.mckinseyquarterly.com/home.aspx?srid=6} and McKinsey Global Institute (MGI) @ \url{http://www.mckinsey.com/mgi/}
\item Accenture Research \& Insights: \url{http://accenture.ie/global/research_and_insights/research_and_insights_int}; also, see \url{http://accenture.ie/Global/Services/By_Industry/Electronics_and_High_Tech/Services/ServicesSemiconductorInd.htm} for Accenture's Semiconductor Business
\item Ernst \& Young: \url{http://www.ey.com/SG/en/Industries}
\item Boston Consulting Group (BCG): \vspace{-0.3cm}
	\begin{enumerate} \itemsep -2pt
	\item \url{http://www.bcg.com/expertise_impact/publications/default.aspx}
	\item Industries that BCG provides services for and analysis of: \url{http://www.bcg.com/expertise_impact/industries/default.aspx}
	\item BCG Strategy Institute: \url{http://www.bcg.com/about_bcg/strategyinstitute/default.aspx} and \url{http://www.bcg.com/about_bcg/strategyinstitute/research/default.aspx}.
	\end{enumerate}
\item PricewaterhouseCoopers: \vspace{-0.3cm}
	\begin{enumerate} \itemsep -2pt
	\item Industry sectors: \url{http://www.pwc.com/gx/en/industry-sectors/index.jhtml}
	\item Research \& insights: \url{http://www.pwc.com/gx/en/research-insights/index.jhtml}
	\end{enumerate}
\item Pew Research Center, \url{http://pewresearch.org/}: \vspace{-0.3cm}
	\begin{enumerate} \itemsep -2pt
	\item Pew Global Attitudes Project: \url{http://pewglobal.org/}
	\item Pew Internet and American Life Project: \url{http://www.pewinternet.org/}
	\item Pew Social and Demographic Trends Project: \url{http://pewsocialtrends.org/}
	\item Pew Forum on Religion and Public Life: \url{http://pewforum.org/}
	\item Pew Research Center for the People and the Press: \url{http://people-press.org/}
	\item Project for Excellence in Journalism: \url{http://pewsocialtrends.org/}
	\item Pew Hispanic Center: \url{http://pewhispanic.org/}
	\end{enumerate}
\item Brookings Institution: \url{http://www.brookings.edu/}
\item Knowledge@Wharton from University of Pennsylvania's Wharton School [of business]: \url{http://knowledge.wharton.upenn.edu/}
\item Goldman Sachs: See Global Markets Institute @ \url{http://www2.goldmansachs.com/ideas/global-markets-institute/index.html}
\item Capgemini: \vspace{-0.3cm}
	\begin{enumerate} \itemsep -2pt
	\item Publishes a ``World Retail Banking Report 20XY''; see \url{http://www.capgemini.com/insights-and-resources/by-publication/world_retail_banking_report_2009/}
	\item Insights \& Resources: Publications; see \url{http://www.capgemini.com/insights-and-resources/by-publication/}
	\end{enumerate}
\item Carnegie Corporation of New York: \vspace{-0.3cm}
	\begin{enumerate} \itemsep -2pt
	\item Publications (including magazines, reports, and books): \url{http://carnegie.org/publications/}
	\end{enumerate}
\item {\it Demos}: Publications, \url{http://www.demos.org/publication_list.cfm}
\item Manhattan Institute for Policy Research: \url{http://www.manhattan-institute.org/tools/bytopic.php}
\item Ford Foundation: Library, \url{http://www.fordfoundation.org/library}
\item The Foundation Center: \url{http://foundationcenter.org/gainknowledge/}
\item Working Group on Extreme Inequality: Resources, \url{http://extremeinequality.org/?page_id=4}
\item The Rockefeller Foundation: \url{http://www.rockefellerfoundation.org/news/publications}
\item Emergent Research: \vspace{-0.3cm}
	\begin{enumerate} \itemsep -2pt
	\item \url{http://www.emergentresearch.com/}
	\item Has research reports about entrepreneurship, and the types of entrepreneurship.
	\end{enumerate}
\end{enumerate}


%%%%%%%%%%%%%%%%%%%%%%%%%%%%%%%%%%%%%%%%%%%
\section{Resources for Research Publications}
\label{researchpubsresources}


Resources for research publications: \vspace{-0.3cm}
\begin{enumerate} \itemsep -4pt
\item {\it Google Scholar}: \url{http://scholar.google.com/}
\item {\it CiteSeer$^{\rm x}$}: \vspace{-0.3cm}
	\begin{enumerate} \itemsep -2pt
	\item \url{http://citeseerx.ist.psu.edu/}
	\item Scientific Literature Digital Library and Search Engine
	\end{enumerate}
\item {\it arXiv}: \vspace{-0.3cm}
	\begin{enumerate} \itemsep -2pt
	\item Open access to 624,659 e-prints in Physics, Mathematics, Computer Science, Quantitative Biology, Quantitative Finance and Statistics
	\item \url{http://arxiv.org/}
	\end{enumerate}
\item Microsoft Academic Search: \url{http://academic.research.microsoft.com/}
\item Scitopia: \url{http://www.scitopia.org/scitopia/}
\item Scirus: \url{http://www.scirus.com/}
\item Scopus: \url{http://www.scopus.com/home.url}
\item SciVerse Scopus: \url{}
\item {\it eScholarship}, California Digital Library and The Berkeley Electronic Press: \vspace{-0.3cm}
	\begin{enumerate} \itemsep -2pt
	\item \url{http://www.escholarship.org/}
	\item eScholarship provides a suite of open access, scholarly publishing services and research tools that enable departments, research units, publishing programs, and individual scholars associated with the University of California to have direct control over the creation and dissemination of the full range of their scholarship.
	\item With eScholarship, you can publish the following original scholarly works on a dynamic research platform available to scholars worldwide: \vspace{-0.2cm}
		\begin{enumerate} \itemsep -2pt
		\item Journals
		\item Books
		\item Working Papers
		\item Conference Proceedings
		\item Seminar/Paper Series
		\end{enumerate}
	\end{enumerate}
\item MyNetResearch's Global Directory of Doctoral Dissertations: \url{http://www.mynetresearch.com/Wiki/Default.aspx}
\item Social Science Research Network (SSRN): \url{http://ssrn.com/}
\item Universitat Polit{\`{e}}cnica de Catalunya: \vspace{-0.3cm}
	\begin{enumerate} \itemsep -2pt
	\item Department of Computer Languages and Systems (Departament de Llenguatges i Sistemes Inform{\`{a}}tics, LSI): \vspace{-0.2cm}
		\begin{enumerate} \itemsep -2pt
		\item LSI Tech Reports archive: \url{http://www.lsi.upc.edu/dept/techreps/techreps.html}
		\end{enumerate}
	\end{enumerate}
\end{enumerate}





Selected research publications: \vspace{-0.3cm}
\begin{enumerate} \itemsep -4pt
\item {\it Ubiquity}: \vspace{-0.3cm}
	\begin{enumerate} \itemsep -2pt
	\item \url{http://ubiquity.acm.org/}
	\item ``A peer-reviewed, online publication of ACM dedicated to the future of computing and the people who are creating it.''
	\end{enumerate}
\end{enumerate}








%%%%%%%%%%%%%%%%%%%%%%%%%%%%%%%%%%%%%%%%%%%
\section{Resources on Technical Writing}
\label{technicalwritingresources}

Resources for academic/technical writing: \vspace{-0.3cm}
\begin{itemize} \itemsep -4pt
\item Individuals: \vspace{-0.3cm}
	\begin{enumerate} \itemsep -2pt
	\item Kenneth M. Hanson (Los Alamos National Laboratory): \url{http://kmh-lanl.hansonhub.com/techwriting.html} or \url{http://public.lanl.gov/kmh/techwriting.html}
	\item William Stallings, ``Writing Guide''. Available at: \url{http://www.williamstallings.com/Extras/Writing_Guide.html}; last accessed on August 25, 2010.
	\item Lorraine Lica, ``The Distinction Between WHICH and THAT With Diagrams: Especially for Scientists''. Available at: \url{http://home.earthlink.net/~llica/wichthat.htm}; last accessed on September 3, 2010.
	\item Michael Nielsen, {\it Six Rules for Rewriting}, posted on his blog on August 19, 2008. Available online at: \url{http://michaelnielsen.org/blog/six-rules-for-rewriting/}; last accessed on December 26, 2010.
	\end{enumerate}
\item ACM: \vspace{-0.3cm}
	\begin{enumerate} \itemsep -2pt
	\item ACM SIG Proceedings: \vspace{-0.2cm}
		\begin{enumerate} \itemsep -2pt
		\item Gerry Murray, ``Conference Proceedings \LaTeXe\ Submission FAQ,'' Association for Computing Machinery, March 10, 2010. Available at: \url{http://www.acm.org/sigs/publications/sigfaq}; last accessed on September 14, 2010.
		\end{enumerate}
	\end{enumerate}
\item IEEE: \vspace{-0.3cm}
	\begin{enumerate} \itemsep -2pt
	\item {\it A Plagiarism FAQ}, IEEE Intellectual Property Rights, Piscataway, NJ. Available online at: \url{http://portal.ieee.org/web/publications/rights/plagiarism_FAQ.html}; last accessed on September 25, 2010.
	\item {\it IEEE Intellectual Property Rights}, IEEE Intellectual Property Rights, Piscataway, NJ. Available online at: \url{http://portal.ieee.org/web/publications/rights/index.html}; last accessed on September 25, 2010.
	\item IEEE Solid-State Circuits Society: \vspace{-0.2cm}
		\begin{enumerate} \itemsep -2pt
		\item Jan Van der Spiegel and Kenneth C. Smith, ``Tools: ISSCC Paper Submissions - Increasing the Likelihood of Success (Tips on Increasing Your Chance of ISSCC Acceptance).'' Available online at: \url{http://www.ieee.org/portal/site/sscs/menuitem.f07ee9e3b2a01d06bb9305765bac26c8/index.jsp?&pName=sscs_level1_article&TheCat=2010&path=sscs/07Summer&file=Tools.xml}; last accessed on September 18, 2010 \cite{VanderSpiegel2007}. [ ``How to Write a Paper for ISSCC'' is available as a flash presentation, prepared for students at A-SSCC in November, 2006, at \url{http://sscs.org/Chapters/07ChptLnch/07FEBCafe.htm}. Alternative reference for this set of presentation slides: Jan Van der Spiegel and Kenneth C. Smith, ``Writing a good ISSCC paper: Tips on how to increase the chances of paper acceptance,'' presentation at the IEEE Asian Solid-State Circuits Conference 2006, November, 2006. Available online at: \url{http://isscc.org/doc/2008/WritingISSCCpaperJVdS_Nov06b.ppt}; last accessed on December 28, 2010. ]. \vspace{-0.1cm}
			\begin{enumerate} \itemsep -1pt
			\item Before I commence writing the paper, ask myself the following questions: \vspace{-0.1cm}
				\begin{itemize} \itemsep -1pt
				\item What results do I want to communicate?
				\item How does my work improve on previously published work?
				\item Who are the key players in this area?
				\item What are the latest references?
				\end{itemize}
			\item Know the latest key references related to your work
			\item Do not use old references, except to emphasize the time scale of the problem.
			\item The use of good references tells the reviewers that you are aware of the latest developments in the field.
			\item Refer to all references in the text of the paper, and comment briefly on each.
			\item Do not refer to only your own work.
			\item Start writing the paper with the Conclusions section. This forces you to think about what you want to say.
			\item Be quantitative in the Conclusions: Summarize the important measured results, give numerical data; and relate them to earlier work.
			\item Once the conclusions are written, backfill the paper.
			\item Be explicit and concrete: Quantify the results.
			\item Put your results in context: Compare them to results of others (refer explicitly to the references).
			\item Introduction: \vspace{-0.1cm}
				\begin{itemize} \itemsep -1pt
				\item This should make it clear to the expert reviewer that you know your area and what others have done
				\item Discuss the state-of-the-art in terms of what others have done recently. Make use of references.
				\item What is the problem you want to solve?
				\item Capture the different approaches to solving the problem and show which of these approaches you have picked and why.  
				\item Continue with explaining your approach
				\end{itemize}
			\item Plan [the main body of the paper] in advance (like a system architecture): \vspace{-0.1cm}
				\begin{itemize} \itemsep -1pt
				\item List 2-to-3 innovative aspects.
				\item Explain the importance of these aspects in terms of new design, performance achievements, how it advances the state-of-the-art.
				\end{itemize}
			\item Body of the paper: \vspace{-0.1cm}
				\begin{itemize} \itemsep -1pt
				\item This should focus on the key ideas and build up the paper incrementally.
				\item Use a figure or diagram to show your approach.
				\item Preferably, show circuit schematics and explain how the circuit works and what is new about it.  
				\item Show measurement results: \vspace{-0.1cm}
					\begin{itemize} \itemsep -1pt
					\item If needed, summarize results in a table format.
					\item If appropriate, provide a Figure-of-Merit to prove that your work advances the state-of-the-art.
					\end{itemize}
				\end{itemize}
			\item Compare your results with those of others: \vspace{-0.1cm}
				\begin{itemize} \itemsep -1pt
				\item Be straightforward in the comparison.
				\item Do not ignore bad results; discuss and explain any shortcomings, rather than ignoring them.
				\item Compare your results with a paper that uses a similar test technique, and which deals with a similar system.
				\end{itemize}
			\item Conclusion: \vspace{-0.1cm}
				\begin{itemize} \itemsep -1pt
				\item Highlight the results.
				\item The final or pre-final paragraph should list all important measured results, give the reviewers a complete picture of your system and convince them of the technical accuracy of your results.
				\item Mention how your results advance the state-of-the-art.
				\end{itemize}
			\item Title of the paper: \vspace{-0.1cm}
				\begin{itemize} \itemsep -1pt
				\item It should give a good idea of the paper's contents and highlights. Do NOT make the title too broad or general, since it may appear to be a marketing paper.
				\item Eg, when your paper talks about Cache and how the Cache is built, do NOT use a title like ``High-speed processors,'' but use a title like ``A fast Cache for a High-Speed Processor''
				\item Or, use a title like ``An 800mW 10Gb$/$s Ethernet Transceiver in 0.13$\mu$m CMOS,'' and NOT: ``A novel, high-speed transceiver''
				\end{itemize}
			\item Body of the paper: \vspace{-0.1cm}
				\begin{itemize} \itemsep -1pt
				\item Don't repeat too much of the abstract. 
				\item Don't present much theory; Refer to other sources of such material in a reference. 
				\item Don't give too many equations; This is not a Ph.D. thesis, and hence only relevant equations should be stated, if any. If an equation is used, you must explain the equation. Don't make assumptions. Everyone has a different way of interpreting such information.
				\item Don't write a tutorial-type paper. ISSCC papers must be very concise and innovative.
				\end{itemize}
			\item Technical content -- innovation is key: \vspace{-0.1cm}
				\begin{itemize} \itemsep -1pt
				\item Highlight the INNOVATION in your paper, early on. Innovation can include one or more of the following: \vspace{-0.1cm}
					\begin{itemize} \itemsep -1pt
					\item Scalability
					\item circuit or architecture innovation
					\item implementation of a new system approach
					\item use of a new technology
					\item best-performance reported
					\end{itemize}
				\item Address the innovation aspect clearly: \vspace{-0.1cm}
					\begin{itemize} \itemsep -1pt
					\item What is new
					\item Accuracy of the proposed approach, circuit, or system
					\item Solutions to the problem
					\item Feasibility of implementation
					\item Comparison with previously proposed techniques
					\end{itemize}
				\item Show at least one important circuit diagram.
				\item When showing a circuit or diagram: \vspace{-0.1cm}
					\begin{itemize} \itemsep -1pt
					\item Explain what is new about it (give an explanation beyond that of a data sheet)
					\item Explain its operation. Do not expect the reviewer to dissect it. Help the reviewer to understand its operation. But, be concise and brief.
					\item What are the advantages, what are the shortcomings? 
					\end{itemize}
				\item Replace words like ``Fastest,'' ``Smallest,'' ``Lowest power consumption,'' etc, by quantitative and accurate comparisons with earlier work. 
				\item Make sure you mention each reference. Include also pending publications at conferences or in journals that appear before ISSCC (see also pre-publication policy)
				\end{itemize}
			\item Stating and describing experimental results in the paper: \vspace{-0.1cm}
				\begin{itemize} \itemsep -1pt
				\item Include a die photo, and give the chip size and technology used.
				\item Include measurements of the fabricated chip, I-V curves, power, etc. Be precise and quantitative.
				\item Compare measured results against stated requirements, and to prior art.
				\item Include a summary table of the design that highlights the specification and performance metrics. 
				\end{itemize}
			\item Do not submit: \vspace{-0.1cm}
				\begin{itemize} \itemsep -1pt
				\item A paper that gives only simulations and has no silicon implementation and test results.
				\item A paper with only modeling and/or equations: submit these to ISCAS, ICCAD or DAC.
				\item A paper that is outside the scope of ISSCC topics. 
				\item Work that has been published somewhere else.
				\end{itemize}
			\item Common reasons for paper rejection: \vspace{-0.1cm}
				\begin{itemize} \itemsep -1pt
				\item A lack of clear evidence of what is novel in the work, and the extent to which it advances the state-of-the-art. 
				\item Successful submissions contain specific new results with sufficient detail and data to be understood, with schematics and measured results for key circuits, when appropriate.
				\item Wrong conference, or pre-publication. 
				\end{itemize}
			\item Pre-publication: \vspace{-0.1cm}
				\begin{itemize} \itemsep -1pt
				\item If a substantial part of a paper has been published before the upcoming ISSCC, the paper will not be accepted. This is the case when: \vspace{-0.1cm}
					\begin{itemize} \itemsep -1pt
					\item Disclosure of the innovative circuitry, architectures, algorithms, etc, occurs in articles, data sheets, trade journals, or other conferences.
					\item Any detailed disclosure of innovative technical ideas on the World-Wide Web before the paper presentation at the Conference will be considered pre-publication.
					\end{itemize}
				\end{itemize}
			\item Pre-publication policy: \vspace{-0.1cm}
				\begin{itemize} \itemsep -1pt
				\item However, a paper may be acceptable in cases where: \vspace{-0.1cm}
					\begin{itemize} \itemsep -1pt
					\item The chip has been sampled, entered production, and/or appeared in a publication that addressed only the marketing or applications aspects of the product.
					\item Disclosure consisting only of abbreviated data sheets that provide only specifications, a feature list, and a coarse block diagram. 
					\item The work has been presented at a workshop or niche conference with limited attendance and no published proceedings or press coverage.
					\end{itemize}
				\item After your paper has been accepted, DO NOT publish any details or summaries on the web, press releases or any other articles before the conference!
				\end{itemize}
			\item Pre-publication material: \vspace{-0.1cm}
				\begin{itemize} \itemsep -1pt
				\item If any material related to your ISSCC submission will have been published prior to the Conference, copies of these prior publications should be submitted.
				\item Such material includes data sheets, press releases, papers or abstracts submitted or accepted at another conference or in a journal appearing before the Conference, and any other forms of publication such as Web presentations.
				\end{itemize}
			\item In summary, it's all about: \vspace{-0.1cm}
				\begin{itemize} \itemsep -1pt
				\item innovation
				\item advancing state-of-the-art
				\item technical quality of the results
				\item results clearly explained
				\end{itemize}
			\end{enumerate}
		\item Editors of the {\it IEEE Journal of Solid-State Circuits}, ``JSSC Submission: Information for JSSC Authors''. Available online at: \url{http://ewh.ieee.org/soc/sscs/index.php?option=com_content&task=view&id=72&Itemid=1}; last accessed on September 18, 2010.
		\item Bram Nauta, ``How to write a good Journal of Solid State Circuits paper,'' presentation slides for the IEEE Asian Solid-State Circuits Conference, Fukuoka, Japan, November 2008. Available online at: \url{http://sscs.ieee.org/images/files/education/how%2520to%2520write%2520jssc%2520paper-v3.pdf}; last accessed on December 28, 2010. \vspace{-0.1cm}
			\begin{enumerate} \itemsep -1pt
			\item Conference papers can get accepted or rejected if they are incomplete or lack key references.
			\item Special issues on IEEE Solid-State Circuits Society conferences: \vspace{-0.1cm}
				\begin{itemize} \itemsep -1pt
				\item December -- ISSCC-analog, RF (issue)
				\item January -- ISSCC-dig+rest (issue)
				\item April -- VLSI (issue)
				\item May -- RFIC (section)
				\item July -- ESSCIRC (issue)
				\item August -- CICC (issue)
				\item September -- BCTM (section)
				\item October -- CSIC (section)
				\item November -- A-SSC (section)
				\end{itemize}
			\item The title of the paper: \vspace{-0.1cm}
				\begin{itemize} \itemsep -1pt
				\item Must describe the paper
				\item Not too vague: \vspace{-0.1cm}
					\begin{itemize} \itemsep -1pt
					\item ``A novel receiver'' -- Do not use ``novel'' anyway
					\item ``5-GHz RF Frontends for Ultra-Low-Voltage and Ultra-Low-Power Operations'' -- How much is is Ultra?
					\end{itemize}
				\item But exactly what is really new -- ``Noise canceling technique for wideband receivers''
				\item Or exactly what is achieved -- ``A 1.5GHz 1.3dB NF, 10mW down converter in 65nm CMOS for GPS applications''
				\item Or both!
				\end{itemize}
			\item Introduction: \vspace{-0.1cm}
				\begin{itemize} \itemsep -1pt
				\item Describe the problem you solve: open the subject; zoom in step by step; describe your assumptions; \& each step is one paragraph
				\item Describe the state-of-the-art: use plenty of references
				\item Tell your basic idea: \vspace{-0.1cm}
					\begin{itemize} \itemsep -1pt
					\item This motivates the reader to continue reading.
					\item Cite your prepublications and tell the difference
					\end{itemize}
				\item Give outline
				\end{itemize}
			\item The body: \vspace{-0.1cm}
				\begin{itemize} \itemsep -1pt
				\item Explain your key idea
				\item Build up step by step: \vspace{-0.1cm}
					\begin{itemize} \itemsep -1pt
					\item one thinking step at a time
					\item each step is one paragraph
					\end{itemize}
				\item Proof that it makes sense: \vspace{-0.1cm}
					\begin{itemize} \itemsep -1pt
					\item Use mathematics
					\item Give exactly your boundary conditions
					\item Give experimental results in a comprehensive way
					\end{itemize}
				\item Be self-critical and realistic: Does it really make sense?
				\item E.g., for a linearity improvement technique: \vspace{-0.1cm}
					\begin{itemize} \itemsep -1pt
					\item If power dissipation is larger
					\item And noise is also larger
					\item And you know that {\it P} $\sim$ {\it SNR}: Does this make sense?
					\end{itemize}
				\item Is it just the technology or your smartness? E.g., speed $\sim f_{T}$ or $f_{\rm max}$
				\item Are practical boundary conditions met? VCO at high frequency but $P_{\rm out} = -30 {\rm dBm}$.
				\end{itemize}
			\item Experimental results: \vspace{-0.1cm}
				\begin{itemize} \itemsep -1pt
				\item Describe exactly what has been measured and how: \vspace{-0.1cm}
					\begin{itemize} \itemsep -1pt
					\item describe setup
					\item ``Bio biased''? (manual tweaking and tuning)
					\item probe or PCB?
					\item What equipment?
					\item How many samples?
					\item PVT?
					\item Batch to batch spread?
					\end{itemize}
				\item Experiment must be repeatable and of practical use (e.g., for industry)
				\item compare with theory/simulations
				\item Does it prove your idea and theory?
				\item Always indicate if a result is measured, simulated, or calculated
				\item ``Figure X shows the noise figure versus frequency'' Is this measured? Simulated? Calculated? Estimated?
				\end{itemize}
			\item IC realization: \vspace{-0.1cm}
				\begin{itemize} \itemsep -1pt
				\item Give chip photograph: \vspace{-0.1cm}
					\begin{itemize} \itemsep -1pt
					\item dimensions
					\item What is what?
					\end{itemize}
				\item Give technology + options: e.g., state the process technology used to manufacture the chip
				\end{itemize}
			\item Discuss results: \vspace{-0.1cm}
				\begin{itemize} \itemsep -1pt
				\item compare to state-of-the-art in a fair way: \vspace{-0.1cm}
					\begin{itemize} \itemsep -1pt
					\item show all relevant data and papers
					\item A table can help, although measurements are hard to compare
					\end{itemize}
				\item Use common figure-of-merit (FoM) definitions: e.g., determine the common FoM for ADCs, VCOs, and filters
				\item Be careful to define your own FoM: \vspace{-0.1cm}
					\begin{itemize} \itemsep -1pt
					\item do not misuse FoM for showing off
					\item Power $\sim$ SNR. BW makes sense
					\item Power/bondpad is NOT a good FoM!!
					\end{itemize}
				\item Help the reader to interpret the results
				\item Absolute accuracy needed? Show many samples, and proof batch to batch robustness
				\item Matching needed? Show many samples
				\item Calibrated circuits? Describe what input signal is used/required. When does it go wrong? How realistic is it?
				\end{itemize}
			\item Conclusions: \vspace{-0.1cm}
				\begin{itemize} \itemsep -1pt
				\item Start writing with this
				\item First, make a bullet list for yourself: \vspace{-0.1cm}
					\begin{itemize} \itemsep -1pt
					\item A handful of bullets
					\item So you know where to write towards
					\item This gives your paper focus
					\end{itemize}
				\item The Conclusion should be readable without reading the whole paper
				\item Convince the reader: What did we learn?
				\end{itemize}
			\item References: \vspace{-0.1cm}
				\begin{itemize} \itemsep -1pt
				\item Include latest state of the art: for benchmark
				\item But, also refer to the original papers: \vspace{-0.1cm}
					\begin{itemize} \itemsep -1pt
					\item Go back in time!
					\item Most references are younger than 5 years
					\item While most ideas are much older
					\end{itemize}
				\item Textbooks are useful too
				\end{itemize}
			\item General Writing Tips: \vspace{-0.1cm}
				\begin{itemize} \itemsep -1pt
				\item A well-written paper gives the impression of a good idea
				\item If a paper is too complex: \vspace{-0.1cm}
					\begin{itemize} \itemsep -1pt
					\item Reviewers don't understand it
					\item Reviewers don't believe it
					\item Reviewers will not like it
					\end{itemize}
				\item If a paper is too simplistic: \vspace{-0.1cm}
					\begin{itemize} \itemsep -1pt
					\item Reviewers think its nothing special
					\item Even if the results are good
					\end{itemize}
				\item Make your problem relevant
				\item Start with the ``big picture''
				\item Take the reader by the hand: step by step explanation
				\item Highlight innovation
				\item Do not give too much theoretical details
				\item Do not try to make a tutorial
				\item Do not use ``very'' but give the numbers
				\item Avoid using the word ``novel''; everything that you don't cite should be novel
				\item Use short sentences
				\item Use simple words
				\item One point per paragraph: \vspace{-0.1cm}
					\begin{itemize} \itemsep -1pt
					\item First of last sentence is most important
					\item The rest is explanation
					\end{itemize}
				\item If you are stuck: \vspace{-0.1cm}
					\begin{itemize} \itemsep -1pt
					\item Tell a friend what you did
					\item Use the words and slides like on your conference paper
					\item Polish the text later
					\end{itemize}
				\item Let a fellow student read and comment
				\item Ask native speaker to correct language
				\item Polish, polish, polish: \vspace{-0.1cm}
					\begin{itemize} \itemsep -1pt
					\item reviewers hate mistakes
					\item ``It iz raely anojing to raed tekst width misstakes''
					\end{itemize}
				\end{itemize}
			\item Figures: \vspace{-0.1cm}
				\begin{itemize} \itemsep -1pt
				\item Make the figures look like a cartoon: \vspace{-0.1cm}
					\begin{itemize} \itemsep -1pt
					\item Reader can understand idea by looking at figures and caption only
					\end{itemize}
				\item Spend a lot of time making good figures: \vspace{-0.1cm}
					\begin{itemize} \itemsep -1pt
					\item Papers with bad figures almost always get rejected
					\end{itemize}
				\item Must be readable in single column
				\item Avoid placing figures in double columns	
				\end{itemize}
			\item Do not: \vspace{-0.1cm}
				\begin{itemize} \itemsep -1pt
				\item Publish the same material elsewhere. Reviewers and readers always see this. This is unethical.
				\item Change your paper after acceptance and before publication: \vspace{-0.1cm}
					\begin{itemize} \itemsep -1pt
					\item E.g., remove reference to competitor
					\item Reviewers always see this
					\end{itemize}
				\item Use someone else's ideas: ``Someone else'' is reading too
				\item Hide ``unpleasant'' measurements
				\item Fabricate or falsify results: \vspace{-0.1cm}
					\begin{itemize} \itemsep -1pt
					\item Do not tune bias for each measuring point
					\item Do not make a few chips and measure different parameters on different chips
					\item Or even completely falsify results
					\end{itemize}
				\end{itemize}
			\item In summary: \vspace{-0.1cm}
				\begin{itemize} \itemsep -1pt
				\item Needs an innovative idea: \vspace{-0.1cm}
					\begin{itemize} \itemsep -1pt
					\item Working silicon is not enough
					\item Must improve state-of-the-art
					\end{itemize}
				\item Needs new material after prepublication
				\item Reviewers are demanding
				\item Your writing technique can help improve the paper's chances of getting accepted for publication
				\end{itemize}
			\end{enumerate}
		\end{enumerate}
	\end{enumerate}
\item University of California, Berkeley: \vspace{-0.3cm}
	\begin{enumerate} \itemsep -2pt
	\item Department of City and Regional Planning; College of Environmental Design: \vspace{-0.2cm}
		\begin{enumerate} \itemsep -2pt
		\item Alvaro Huerta, {\it Resources: Writing \& Research Links / Attachments}, Department of City and Regional Planning, College of Environmental Design, University of California, Berkeley, 2010. Available online at: \url{http://sites.google.com/site/alvarohuertasite/links-academic-more}; last accessed on January 5, 2010.
		\end{enumerate}
	\end{enumerate}
\item Stanford University: \vspace{-0.3cm}
	\begin{enumerate} \itemsep -2pt
	\item Stanford University InfoLab: \vspace{-0.2cm}
		\begin{enumerate} \itemsep -2pt
		\item Prof. Jennifer Widom: \url{http://infolab.stanford.edu/~widom/paper-writing.html}
		\end{enumerate}
	\end{enumerate}
\item Carnegie Mellon University: \vspace{-0.3cm}
	\begin{enumerate} \itemsep -2pt
	\item Philip Koopman, ``How to Write an Abstract,'' Department of Electrical and Computer Engineering, Carnegie Mellon University, October 1997. Available at: \url{http://www.ece.cmu.edu/~koopman/essays/abstract.html}; last accessed on August 26, 2010. Also available at: \url{http://www.computersciencestudent.com/Extras/Abstract.html}.
	\end{enumerate}
\item University of Washington: \vspace{-0.3cm}
	\begin{enumerate} \itemsep -2pt
	\item Department of Mathematics: \vspace{-0.2cm}
		\begin{enumerate} \itemsep -2pt
		\item Minh Van Nguyen, {\it Academia}, Department of Mathematics, University of Washington, 2010. Available online at: \url{http://sage.math.washington.edu/home/mvngu/academia.html}; last accessed on October 25, 2010.
		\end{enumerate}
	\end{enumerate}
\item University of Pennsylvania: \vspace{-0.3cm}
	\begin{enumerate} \itemsep -2pt
	\item Steve Zdancewic, ``Writing Tips,'' Department of Computer and Information Science, School of Engineering and Applied Science, University of Pennsylvania, July 29, 2003. Available at: \url{http://www.cis.upenn.edu/~stevez/writing-tips.html}; last accessed on September 5, 2010.
\item University of California, Los Angeles: \vspace{-0.3cm}
	\end{enumerate}
	\begin{enumerate} \itemsep -2pt
	\item Terence Tao, {\it On writing}, Department of Mathematics, University of California, Los Angeles. Available at: \url{http://terrytao.wordpress.com/advice-on-writing-papers/}; last accessed on September 1, 2010.
	\end{enumerate}
\item Purdue University: \vspace{-0.3cm}
	\begin{enumerate} \itemsep -2pt
	\item Purdue Online Writing Lab (OWL): \url{http://owl.english.purdue.edu/owl/}
	\end{enumerate}
\item Royal Institute of Technology (KTH): \vspace{-0.3cm}
	\begin{enumerate} \itemsep -2pt
	\item Department of Microelectronics and Information Technology; Laboratory of Electronics and Computer Systems: \vspace{-0.2cm}
		\begin{enumerate} \itemsep -2pt
		\item Elena Dubrova, ``The Art of Doctoral Research,'' (FIL3001, PhD course, 7.5 pt), Sept. 2008 - May 2009. Available at: \url{http://web.it.kth.se/~dubrova/coursePhD.html}; last accessed on September 14, 2010.
		\end{enumerate}
	\end{enumerate}
\item University of California, Davis: \vspace{-0.3cm}
	\begin{enumerate} \itemsep -2pt
	\item Office of Student Judicial Affairs: \vspace{-0.2cm}
		\begin{enumerate} \itemsep -2pt
		\item Publications: \url{http://sja.ucdavis.edu/publications.html}
		\end{enumerate}
	\end{enumerate}
\item Pennsylvania State University: \vspace{-0.3cm}
	\begin{enumerate} \itemsep -2pt
	\item Penn State College of Engineering: \url{http://www.writing.engr.psu.edu/}
	\end{enumerate}
\item University of Toronto: \vspace{-0.3cm}
	\begin{enumerate} \itemsep -2pt
	\item University College (UC) Writing Centre: \url{http://www.utoronto.ca/ucwriting/handouts.html}
	\end{enumerate}
\item Link{\"{o}}ping University: \vspace{-0.3cm}
	\begin{enumerate} \itemsep -2pt
	\item Department for Computer and Information Science: \vspace{-0.2cm}
		\begin{enumerate} \itemsep -2pt
		\item Christoph Kessler, ``Stylistic advice to my exjobb and PhD students for writing a thesis,'' Department for Computer and Information Science, Link{\"{o}}ping University. Available at: \url{http://www.ida.liu.se/~chrke/exjobb/writing.html}; last accessed on September 1, 2010.
		\item Christoph Kessler, ``Exjobb project plan guidelines,'' Department for Computer and Information Science, Link{\"{o}}ping University. Available at: \url{http://www.ida.liu.se/~chrke/exjobb/exj_plan.shtml}; last accessed on September 1, 2010. [ This provides information on writing research/thesis proposals. ]
		\end{enumerate}
	\end{enumerate}
\item University of Crete: \vspace{-0.3cm}
	\begin{enumerate} \itemsep -2pt
	\item Department of Computer Science: \vspace{-0.2cm}
		\begin{enumerate} \itemsep -2pt
		\item Panagiota Fatourou: \vspace{-0.1cm}
			\begin{enumerate} \itemsep -1pt
			\item \url{http://www.ics.forth.gr/~faturu/}
			\item American Mathematical Society, ``A Manual for Authors of Mathematical Papers.'' Available online at: \url{http://www.ams.org/journals/bull/1943-49-03/S0002-9904-1943-07884-6/S0002-9904-1943-07884-6.pdf}; last accessed on December 22, 2010.
			\end{enumerate}
		\end{enumerate}
	\end{enumerate}
\item University of Cambridge: \vspace{-0.3cm}
	\begin{enumerate} \itemsep -2pt
	\item University Offices -- Unified Administrative Services / The Old Schools: \vspace{-0.2cm}
		\begin{enumerate} \itemsep -2pt
		\item University Offices, ``Information for students,'' University of Cambridge, November 30, 2010. Available online at: \url{http://www.admin.cam.ac.uk/univ/plagiarism/students/}; last accessed on January 8, 2010.
		\end{enumerate}
	\end{enumerate}
\item University College London: \vspace{-0.3cm}
	\begin{enumerate} \itemsep -2pt
	\item Department of Medical Physics \& Bioengineering: \vspace{-0.2cm}
		\begin{enumerate} \itemsep -2pt
		\item Adam Gibson, {\it Teaching links: Tips on how to write scientific articles}, Department of Medical Physics \& Bioengineering, University College London, October 8, 2008. Available online at: \url{http://www.medphys.ucl.ac.uk/~agibson/work/history.html#writing}; last accessed on October 21, 2010. [ Also available at: \url{http://www.ucl.ac.uk/medphys/staff/people/agibson/www/teaching/#writing}]
		\end{enumerate}
	\end{enumerate}
\item Swarthmore College: \vspace{-0.3cm}
	\begin{enumerate} \itemsep -2pt
	\item Department of History: \vspace{-0.2cm}
		\begin{enumerate} \itemsep -2pt
		\item Timothy Burke, ``Beyond the Five-Paragraph Essay,'' in his blog {\it Easily Distracted: Culture, Politics, Academia and Other Shiny Objects}, Department of History, Swarthmore College. Available at: \url{http://weblogs.swarthmore.edu/burke/permanent-features-advice-on-academia/beyond-the-five-paragraph-essay/}; last accessed on September 14, 2010.
		\item Timothy Burke, ``How to Read in College -- Staying Afloat: Some Scattered Suggestions on Reading in College,'' in his blog {\it Easily Distracted: Culture, Politics, Academia and Other Shiny Objects}, Department of History, Swarthmore College. Available at: \url{http://weblogs.swarthmore.edu/burke/permanent-features-advice-on-academia/how-to-read-in-college/}; last accessed on September 14, 2010.
		\end{enumerate}
	\end{enumerate}
\item Indiana University: \vspace{-0.3cm}
	\begin{enumerate} \itemsep -2pt
	\item Writing Tutorial Services: \vspace{-0.2cm}
		\begin{enumerate} \itemsep -2pt
		\item \url{http://www.indiana.edu/~wts/}
		\item WTS Pamphlets: \url{http://www.indiana.edu/~wts/pamphlets.shtml}
		\end{enumerate}
	\end{enumerate}
\item University of Alberta: \vspace{-0.3cm}
	\begin{enumerate} \itemsep -2pt
	\item Faculty of Arts: \vspace{-0.2cm}
		\begin{enumerate} \itemsep -2pt
		\item Centre for Writers: \vspace{-0.1cm}
			\begin{enumerate} \itemsep -1pt
			\item Centre for Writers Resources: \url{http://www.c4w.arts.ualberta.ca/Resources/Resource.aspx}
			\item Other Writing Centres: \url{http://www.c4w.arts.ualberta.ca/OtherWritingCentres/OtherWritingCentres.aspx}
			\end{enumerate}
		\end{enumerate}
	\end{enumerate}
\item Universit{\'{e}} du Qu{\'{e}}bec {\`{a}} Montr{\'{e}}al (UQAM, University of Quebec): \vspace{-0.3cm}
	\begin{enumerate} \itemsep -2pt
	\item Pavillon Saint-Urbain, LICEF Research Center: \vspace{-0.2cm}
		\begin{enumerate} \itemsep -2pt
		\item Daniel Lemire, ``Write Good Papers,'' in {\it Daniel Lemire's blog}, LICEF Research Center, Pavillon Saint-Urbain, Universit{\'{e}} du Qu{\'{e}}bec {\`{a}} Montr{\'{e}}al. Available at: \url{http://www.daniel-lemire.com/blog/rules-to-write-a-good-research-paper/}; last accessed on September 14, 2010.
		\end{enumerate}
	\end{enumerate}
\item State University of New York at Buffalo: \vspace{-0.3cm}
	\begin{enumerate} \itemsep -2pt
	\item William J. Rapaport, ``How to Write (How to Prepare Technical Reports),'' Department of Computer Science and Engineering, State University of New York at Buffalo, Buffalo, NY. Available at: \url{http://www.cse.buffalo.edu/~rapaport/howtowrite.html}; last accessed on August 25, 2010.
	\end{enumerate}
\item Tufts University: \vspace{-0.3cm}
	\begin{enumerate} \itemsep -2pt
	\item Norman Ramsey, {\it Norman Ramsey's Resources for Writers}, Department of Computer Science, Tufts University. Available at: \url{http://www.cs.tufts.edu/~nr/students/writing.html}; last accessed on September 2, 2010.
	\end{enumerate}
\item University of Maryland, Baltimore County: \vspace{-0.3cm}
	\begin{enumerate} \itemsep -2pt
	\item Department of Computer Science and Electrical Engineering: \vspace{-0.2cm}
		\begin{enumerate} \itemsep -2pt
		\item Alan T. Sherman (Alan Theodore Sherman), ``Some Advice on Writing a Technical Report,'' Department of Computer Science and Electrical Engineering, University of Maryland, Baltimore County, April 27, 1996. Available at: \url{http://www.csee.umbc.edu/~sherman/Courses/documents/TR_how_to.html}; last accessed on August 28, 2010.
		\end{enumerate}
	\end{enumerate}
\item University of Maryland, College Park: \vspace{-0.3cm}
	\begin{enumerate} \itemsep -2pt
	\item Department of Computer Science: \vspace{-0.2cm}
		\begin{enumerate} \itemsep -2pt
		\item Neil Spring, ``style-check.rb,'' Department of Computer Science, University of Maryland, College Park, December 1, 2007. Available at: \url{http://www.cs.umd.edu/~nspring/software/style-check-readme.html}; last accessed on September 14, 2010.
		\item Neil Spring, ``Advising Notes,'' Department of Computer Science, University of Maryland, College Park. Available at: \url{http://www.cs.umd.edu/~nspring/advising.html}; last accessed on September 14, 2010.
		\end{enumerate}
	\end{enumerate}
\item Missouri University of Science and Technology (formerly University of Missouri-Rolla): \vspace{-0.3cm}
	\begin{enumerate} \itemsep -2pt
	\item Writing Across the Curriculum Program: \vspace{-0.2cm}
		\begin{enumerate} \itemsep -2pt
		\item Includes the Moeller Writing Center, and the Center for Writing Technologies
		\item \url{http://writingcenter.mst.edu/}
		\item {\it Writing Center Documentation Style Guides} and {\it Writing Center Helpful Handouts}: \vspace{-0.1cm}
			\begin{enumerate} \itemsep -1pt
			\item \url{http://writingcenter.mst.edu/writinghandouts.html}
			\item It includes the following documentation style guides: \vspace{-0.1cm}
				\begin{itemize} \itemsep -1pt
				\item American Society of Civil Engineers (ASCE) Documentation Style
				\item American Society of Mechanical Engineers (ASME) Publication Information
				\item American Psychological Association (APA) Documentation Style
				\item Chicago Manual of Style Documentation I - author/date
				\item Chicago Manual of Style Documentation II - notes/bibliography
				\item Institute of Electrical and Electronics Engineers (IEEE) Documentation Style
				\item Modern Language Association (MLA) Documentation Style
				\end{itemize}
			\end{enumerate}
		\end{enumerate}
	\end{enumerate}
\item Colorado State University: \vspace{-0.3cm}
	\begin{enumerate} \itemsep -2pt
	\item Department of Electrical \& Computer Engineering; College of Engineering: \vspace{-0.2cm}
		\begin{enumerate} \itemsep -2pt
		\item Edwin K. P. Chong, ``Edwin Chong's links of interest,'' Department of Electrical \& Computer Engineering, College of Engineering, Colorado State University, October 18, 2010. Available online at: \url{http://www.engr.colostate.edu/~echong/links.shtml}; last accessed on November 3, 2010.
		\end{enumerate}
	\end{enumerate}
\item Wright State University: \vspace{-0.3cm}
	\begin{enumerate} \itemsep -2pt
	\item University College; University Writing Center: \vspace{-0.2cm}
		\begin{enumerate} \itemsep -2pt
		\item Writing Center Style Guide: \url{http://www.wright.edu/uc/success/services/writingctr/styleguidefaq.html}
		\item Academic Writing: \url{http://www.wright.edu/uc/success/services/writingctr/academicwriting.html}
		\item Plagiarism: \url{http://www.wright.edu/uc/success/services/writingctr/plagiarism.html}
		\item Thesis Statements: \url{http://www.wright.edu/academics/writingctr/resources/thesisstatements.html}
		\item Writer Resources: \url{http://www.wright.edu/academics/writingctr/resources/}
		\item University Writing Center, University College, Wright State University
		\end{enumerate}
	\item Writing Across the Curriculum (WAC) program: \url{http://www.wright.edu/academics/wac/}
	\item Department of English Language and Literature, College of Liberal Arts: \vspace{-0.2cm}
		\begin{enumerate} \itemsep -2pt
		\item Department of English Language and Literature, {\it WSU Writing Web: Citation Resources}, Department of English Language and Literature, College of Liberal Arts, Wright State University. Available at: \url{http://www.wright.edu/cola/Dept/eng/wsuwweb/pageindxs/citation.htm}; last accessed on September 14, 2010.
		\end{enumerate}
	\end{enumerate}
\item Capital Community College: \vspace{-0.3cm}
	\begin{enumerate} \itemsep -2pt
	\item Guide to Grammar and Writing: \url{http://grammar.ccc.commnet.edu/grammar/}
	\item Online Resources for Writers: \url{http://webster.commnet.edu/writing/writing.htm}
	\end{enumerate}
\item TAFE NSW (Technical and Further Education, New South Wales): \vspace{-0.3cm}
	\begin{enumerate} \itemsep -2pt
	\item The Information Centre, {\it Study Skills: Writing Skills}, The Information Centre, The Learning Centre, North Coast Institute, TAFE NSW (Technical and Further Education, New South Wales): \url{http://www.ncistudent.net/StudySkills/WritingSkills/Introduction.htm}
	\end{enumerate}
\item IBM: \vspace{-0.3cm}
	\begin{enumerate} \itemsep -2pt
	\item Program Analysis Group; IBM T.J. Watson Research Center: \vspace{-0.2cm}
		\begin{enumerate} \itemsep -2pt
		\item Eran Yahav, {\it Resources: Writing Resources}, IBM. Available online at: \url{http://www.research.ibm.com/people/e/eyahav/resources.html}; last accessed on September 28, 2010. [ Also, see \url{http://www.research.ibm.com/people/e/eyahav/misc.html} under {\it misc: Resources} ]
		\end{enumerate}
	\end{enumerate}
\item The Chronicle of Higher Education: \vspace{-0.3cm}
	\begin{enumerate} \itemsep -2pt
	\item Michael C. Munger, ``10 Tips on How to Write Less Badly,'' in {\it Advice: Do Your Job Better}, The Chronicle of Higher Education, September 6, 2010. Available at: \url{http://chronicle.com/article/10-Tips-on-How-to-Write-Less/124268/}; last accessed on September 13, 2010.
	\end{enumerate}
\item {\it Guide to Online Schools}, or {\it GuideToOnlineSchools.com}: \vspace{-0.3cm}
	\begin{enumerate} \itemsep -2pt
	\item {\it The Ultimate Style Guide Resources for MLA, APA, Chicago, and CSE}. Available at: \url{http://www.guidetoonlineschools.com/tips-and-tools/mla-apa-chicago-cse}; last accessed on August 25, 2010.
	\item {\it The Complete Plagiarism Resource}. Available at: \url{http://www.guidetoonlineschools.com/tips-and-tools/plagiarism}; last accessed on August 25, 2010.
	\item {\it The Top 50 Academic Writing Resources Online}. Available at: \url{http://www.guidetoonlineschools.com/tips-and-tools/best-writing-resources}; last accessed on August 25, 2010.
	\end{enumerate}
\end{itemize}



Resources for improving my English language skills: \vspace{-0.3cm}
\begin{enumerate} \itemsep -4pt
\item RubyWorks (resource on English grammar): \url{http://rubyworks.github.com/english/grammar/gramdex.html}
\item U.S. Department of State: \vspace{-0.3cm}
	\begin{enumerate} \itemsep -2pt
	\item Bureau of Educational and Cultural Affairs: \vspace{-0.2cm}
		\begin{enumerate} \itemsep -2pt
		\item Materials for Teaching and Learning English: \url{http://exchanges.state.gov/englishteaching/resources-et.html}
		\end{enumerate}
	\end{enumerate}
\end{enumerate}




Resources for detecting plagiarism: \vspace{-0.3cm}
\begin{enumerate} \itemsep -4pt
\item PlagiarismDetect.com: \url{http://www.plagiarismdetect.com/}
\item Plagiarism Checker (by Darren Hom): \url{http://www.plagiarismchecker.com/}
\item ArticleChecker.com: \url{http://www.articlechecker.com/}
\item CheckForPlagiarism.net (by Plagiarism-Checkers, Inc.): \url{http://www.checkforplagiarism.net/}
\item Turnitin (by iParadigms, LLC): \url{http://www.turnitin.com}
\end{enumerate}














%%%%%%%%%%%%%%%%%%%%%%%%%%%%%%%%%%%%%%%%%%%
\section{Resources for Postdoc Positions}
\label{resourcesforpostdocpositions}

Resources for postdoc positions: \vspace{-0.3cm}
\begin{enumerate} \itemsep -4pt
\item The Chronicle of Higher Education: \vspace{-0.3cm}
	\begin{enumerate} \itemsep -2pt
	\item Zoe Smith and Ariana Sutton-Grier, ``Making the Most of Your Postdoc,'' The Chronicle of Higher Education: Advice: Do Your Job Better, July 15, 2010. Available at: \url{http://chronicle.com/article/Making-the-Most-of-Your/66265/}; last accessed on September 6, 2010.
	\end{enumerate}
\item Inside Higher Ed: \vspace{-0.3cm}
	\begin{enumerate} \itemsep -2pt
	\item \url{http://www.insidehighered.com/}
	\item Has information concerning the application for faculty positions, and advice for career paths in higher education institutions. 
	\item Kerry Ann Rockquemore, ``Winning Tenure Without Losing Your Soul: Stop Talking, Start Walking,'' Inside Higher Ed: Career Advice, Inside Higher Ed, January 25, 2010. Available at: \url{http://www.insidehighered.com/advice/winning/winning2}; last accessed on September 7, 2010.
	\end{enumerate}
\item Nature Publishing Group (a division of Macmillan Publishers Limited): \vspace{-0.3cm}
	\begin{enumerate} \itemsep -2pt
	\item Kendall Powell, ``Careers and Recruitment: A foot in the door,'' in {\it Nature}, Vol. 463, No. 7281, pp. 696-697, February 4, 2010. Available online at: \url{http://dx.doi.org/10.1038/nj7281-696a}; last accessed on December 31, 2010. \vspace{-0.2cm}
		\begin{enumerate} \itemsep -2pt
		\item I should commence looking for a postdoctoral research position early in my penultimate year.
		\item When looking for a postdoctoral research position, I shall take note of the name of research labs which journal/conference papers that I find are interesting. That is, each time I read an interesting journal/conference paper, take note of the name of the research labs and its affiliation (name of department, and organization/university).
		\item Compile a list of 20-30 research labs that I find are interesting. Pare down this list to about 5-6 labs, which I will apply to. Use the same heuristics for applying to graduate school. These 5-6 research labs shall include: \vspace{-0.1cm}
			\begin{enumerate} \itemsep -1pt
			\item A reach lab, which I want to be part of but may have difficulty securing a postdoctoral research position in
			\item 3-4 match/target labs, which postdocs and Ph.D. students are comparable to me in terms of research excellence, and which I shall have a good chance of securing a postdoctoral research position in
			\item 1-2 reliable/safety labs, which I should be able to secure a postdoctoral research position in
			\end{enumerate}
		\item Network with students at those labs in conferences, workshops, symposiums, and other events (such as ``summer schools''). Network with the principal/primary investigators (PIs) too!
		\item Make notes/summaries of poster presentations and talks from researchers in these labs.
		\item Make notes/summaries of their conference and journal papers. Try to get access to Ph.D. theses from their labs, and read them too. Pay special attention to the ``Future Work'' sections of the Ph.D. theses and conference papers.
		\item To secure a good postdoctoral research position, I must carry out ``diligent background research aimed to answer the question, `{\bf What do I want to get out of my postdoc?}'.''
		\item The question, ``What do I want to get out of my postdoc?,'' is analogous to the question for prospective Ph.D. students, ``What do I want to get out of my Ph.D. program?''
		\item ``Although they are short-term assignments, postdoc positions should be viewed as stepping stones to a longer-term independent career -- whether in academia, industry or another science-related post.''
		\item ``For that reason, it is hard to overstate the importance of the postdoc application. It is the fledgling scientist's bid to get noticed -- to gain a phone or in-person interview with labs. Background research, a carefully crafted curriculum vitae (CV) and cover letter, and personalization of each application will open doors. Form letters and typos will get applicants nowhere.''
		\item ``I wanted to apply to proven, top-notch labs where I was going to have the success and track record of the people coming out of these labs'' -- Toby Franks
		\item {\bf WARNING: Note that some postdoc positions require applications 12-18 months in advance.}
		\item Treat the process of applying for postdoctoral research positions like applying for R\&D jobs in the EDA/semiconductor industry. Apply to several positions simultaneously, so that I can interview in person with potential employers and lab colleagues prior to making my decision (about which offer of a postdoctoral research position to accept).
		\item ``An application typically consists of a cover letter introducing the applicant and his or her reasons for joining this particular lab; a CV outlining education, publication record, honors and accomplishments; and three referees who will provide supportive letters of recommendation on request. Some students also include a research summary of their graduate work; others incorporate this into their cover letter.''
		\item ``A little preparation goes a long way at this stage. Consider taking a workshop on writing cover letters and CVs, have senior colleagues review them, and proofread them carefully.''
		\item The cover letter and CV shall be free of mistakes.
		\item Things that PIs look for when hiring postdocs: \vspace{-0.1cm}
			\begin{enumerate} \itemsep -1pt
			\item First-author conference and journal publications, which prove that I ``can complete a project from start to finish'' as a junior scientist.
			\item Given the size of my home institution and resources available to my Ph.D. research lab, what have I accomplished? PIs want to see if Ph.D. students can accomplish a significant amount of things, despite the constraints in resources.
			\item List experiences that illustrate non-research responsibilities, including sitting on grad school or department committees, or hosting seminar speakers.
			\end{enumerate}
		\item ``Applicants should highlight what they hope to accomplish in general in a postdoc position. Specific details of projects should be left for the interview.''
		\item Quoted: Agneta Nordenskj{\"{o}}ld, a genetics researcher at the Karolinska Institute in Stockholm, advises spelling out your contributions to a graduate research project. ``Write it in a way that says, `I did this' or `My part of the project was', especially if you did something outstanding,'' she says.
		\item ``Those applying after taking a break from science must work harder to convince a lab head. Kristofor Langlais had been teaching high-school science at a ski academy in Vermont when he applied for postdoc positions in the Washington DC area.''
		\item ``After extensive research into each lab's publications, websites and even annual reports, he wrote his cover letters from the angle of someone already in the lab. He mentioned specific results he found interesting and the next natural steps the lab might take. ``I tried to make it sound like I could walk in that day and be self-sufficient immediately.'' He spent 20 hours or more on each application and his strategy paid off -- he had four phone interviews, and ended up in a molecular-genetics fellowship at the US National Institute of Child Health and Human Development in Bethesda, Maryland.''
		\item Quoted: Likewise, when Xiaoli Du was finishing up her doctorate at Peking Union Medical College in Beijing, she knew she would need to send applications to 30-40 labs if she wanted to obtain a postdoc in the United States. But she avoided the form-letter strategy. `` `Dear Professor' does not show respect or that you are really interested in their lab,'' she says. Instead, she personalized each application and stated how her training and experience would distinguish her from other applicants. Her hard work led to a postdoc at the US National Cancer Institute in Bethesda, Maryland. Du suggests attending international meetings to make first contact with potential advisers.
		\item Quoted: Few things, though, confer more of an advantage than secured funding. ``If a postdoc has their own fellowship, they can write their application to me in crayon and I'll take them,'' says Phil Baran, an organic chemist at the Scripps Research Institute in La Jolla, California. Unfunded applicants should assure the lab head that they have checked on specific fellowship possibilities and outline a plan to apply for them.
		\item Quoted: There are some definite `wrong ways' to apply. Goldstein, whose e-mail inbox is so overloaded that his system sends an automated response to direct queries to assistants and lab managers, says there is no room for red flags in the competitive arena. Avoid telling personal-life woes, bad-mouthing previous labs or advisers or expressing a desire to work at night so that you can surf during the day. Explain gaps in a CV or publication record.
		\item Quoted: ``Anything that signals the person is a prima donna, no matter how great they are, I don't go for,'' says Ken Yamada, laboratory chief at the National Institute of Dental and Craniofacial Research in Bethesda, Maryland. ``Research requires teamwork.''
		\item Quoted: Lab heads want a clear indication that applicants have carefully thought through their career goals and chosen this lab as the appropriate stepping stone. ``Does a genuine passion, drive, and hunger for research come through in their letters or on the phone?'' asks Yamada. ``Would they be doing the same thing if they were suddenly independently wealthy?''
		\item Postdoc application to-do list: \vspace{-0.1cm}
			\begin{enumerate} \itemsep -1pt
			\item Send your application by e-mail or overnight delivery. Consider a paper packet if you have unpublished manuscripts you want to include.
			\item Make it easy for lab heads to contact you by e-mail or phone.
			\item Follow-up by e-mail in 1-2 weeks to make sure they received your application. Don't phone.
			\item Choose referees who really know you, such as collaborators, unofficial advisers or others beyond the standard committee members.
			\item Meet with your referees to explain your career goals to them.
			\item Encourage referees to send their letters promptly (Salk Institute cell biologist Martin Hetzer says that the speed with which a letter lands in his inbox is usually much more telling than the letter's content).
			\item Prepare for the possibility of phone interviews, which may be scheduled or spontaneous. Make sure the conversation is two-way and ask your own questions, too. Have a list of bullet points handy in case you get nervous.
			\end{enumerate}
		\end{enumerate}
	\item Rania Sanford, ``How to navigate the road ahead,'' in {\it Nature}, Vol. 467, No. 7315, pp. 624, September 30, 2010. Available online at: \url{http://dx.doi.org/10.1038/nj7315-624a}; last accessed on December 31, 2010. \vspace{-0.2cm}
		\begin{enumerate} \itemsep -2pt
		\item Check out how much mentoring is being provided to postdoctoral researchers (postdocs) by principal investigators (PIs).
		\item Check if the postdoc position allow me time to learn and seek guidance.
		\item Choose a research lab to do my postdoc, so that I can have a ``meaningful experience,'' obtain ``strong [research] results,'' and would be able to get ``a desirable job'' based on my postdoc research. Avoid research labs that don't allow me to satisfy these preferences.
		\item ``Good mentoring is an acquired skill.''
		\item Find out if the PI: \vspace{-0.1cm}
			\begin{enumerate} \itemsep -1pt
			\item Is aware of ``the stages in developing a mentoring relationship''
			\item Coaches students and postdocs, rather than supervises them
			\item ``Build[s] trust with prot{\'{e}}g{\'{e}}s and postdocs' expectations''
			\item Replicates the style of mentoring to all her/his mentees. Mentoring should be individualized/customized, since each person is a unique individual and differences exist between cultures, genders, and generations.
			\end{enumerate}
		\item Find out if the institution/university penalizes postdocs who have their own funding or fellowships (including competitive awards). The institution/university may provide medical insurance and other benefits to postdocs who are funded by research grants; hence, they are treated as staff members. If postdocs have their own funding or fellowships, they may not be eligible for medical insurance and other benefits that are provided by the university.
		\item Find out if support is given to non-US residents with regards to applying for non-immigrant J-1 visas or equivalent, relocation to the university's community, and assimilation into the university's community.
		\end{enumerate}
	\item Katharine Sanderson, ``Training: The career doctor,'' in {\it Nature}, Vol. 467, No. 7315, pp. 623, September 30, 2010. Available online at: \url{http://dx.doi.org/10.1038/nj7315-623a}; last accessed on December 31, 2010.
	\item Karen Kaplan, ``Careers and Recruitment: Industrial endeavours,'' in {\it Nature}, Vol. 461, No. 7263, pp. 554--555 September 24, 2009. Available online at: \url{http://dx.doi.org/10.1038/nj7263-554a}; last accessed on December 31, 2010. [ Has information on industrial postdoctoral research positions in the biotech industry. ]
	\item {\it naturejobs.com}: \vspace{-0.2cm}
		\begin{enumerate} \itemsep -2pt
		\item Career toolkit: \vspace{-0.1cm}
			\begin{enumerate} \itemsep -1pt
			\item Podcasts ({\it Naturejobs} podcasts): \vspace{-0.1cm}
				\begin{itemize} \itemsep -1pt
				\item Available online at: \url{http://www.nature.com/naturejobs/career-toolkit/podcasts/index.html}; last accessed on December 31, 2010.
				\end{itemize}
			\end{enumerate}
		\end{enumerate}
	\end{enumerate}
\item National Postdoctoral Association: \vspace{-0.3cm}
	\begin{enumerate} \itemsep -2pt
	\item Resources on Becoming a Postdoc: \url{http://www.nationalpostdoc.org/graduate-students}
	\item Faculty \& Administrators: \url{http://www.nationalpostdoc.org/faculty-administrators}
	\item Diversity Programs \& Resources: \url{http://www.nationalpostdoc.org/diversity-issues}
	\item International Postdocs: \url{http://www.nationalpostdoc.org/international-issues}
	\item The NPA Postdoctoral Core Competencies Toolkit (NPA Core Competencies): \url{http://www.nationalpostdoc.org/competencies}
	\end{enumerate}
\item Research Foundation -- Flanders (FWO), or ``Fonds voor Wetenschappelijk Onderzoek -- Vlaanderen'' (FWO): \vspace{-0.2cm}
	\begin{enumerate} \itemsep -2pt
	\item See \url{http://www.fwo.be/en/index.aspx}
	\item This is sponsored by the National Fund for Scientific Research (Belgium) 
	\item Research Foundation - Flanders (FWO): five-yearly prizes: \vspace{-0.2cm}
		\begin{enumerate} \itemsep -2pt
		\item Two Dr. A. De Leeuw-Damry-Bourlart prizes (100,000 Euro): For exact and applied sciences.
		\item Two Dr. Joseph Maisin prizes: For fundamental biomedical sciences and clinical biomedical sciences.
		\item The Ernest John Solvay prize: For humanities and social sciences.
		\end{enumerate}
	\item The Research Foundation � Flanders (FWO) pays a researcher�s income for the following categories: \vspace{-0.2cm}
		\begin{enumerate} \itemsep -2pt
		\item Ph.D. student
		\item Postdoctoral researcher
		\item Clinical fellowship
		\end{enumerate}
	\end{enumerate}
\item {\it PhDjobs.com}: \url{http://www.phdjobs.com/}
\item {\it innovation report}: \vspace{-0.3cm}
	\begin{enumerate} \itemsep -2pt
	\item Jobs (in industry and academia): \url{http://www.innovations-report.com/jobs/jobs.php}
	\end{enumerate}
\item Princeton University: \vspace{-0.3cm}
	\begin{enumerate} \itemsep -2pt
	\item Department of Computer Science, School of Engineering and Applied Science (SEAS): \vspace{-0.2cm}
		\begin{enumerate} \itemsep -2pt
		\item Jeff Erickson and Boaz Barak, {\it TCS opportunities: Postdocs and other positions in theoretical computer science}, Center for Computational Intractability, Department of Computer Science, School of Engineering and Applied Science (SEAS), Princeton University. Available at: \url{http://intractability.princeton.edu/jobs/}; last accessed on September 15, 2010.
		\end{enumerate}
	\end{enumerate}
\item Simons Foundation: Simons Postdoctoral Fellows Program, \url{https://simonsfoundation.org/funding-guidelines/simons-postdoctoral-fellows-program}
\item IBM Postdoctoral Fellowships: \vspace{-0.3cm}
	\begin{enumerate} \itemsep -2pt
	\item IBM Herman Goldstine Postdoctoral Fellowship in Mathematical and Computer Sciences; see \url{http://domino.research.ibm.com/comm/research_projects.nsf/pages/goldstine.index.html}
	\item Josef Raviv Memorial Postdoctoral Fellowship; see \url{http://domino.research.ibm.com/comm/research.nsf/pages/d.compsci.josef.raviv.general.info.html}, \url{http://domino.research.ibm.com/comm/research.nsf/pages/d.compsci.raviv.winner.html}, and \url{http://domino.research.ibm.com/comm/research.nsf/pages/d.compsci.raviv.winner2008.html}
	\end{enumerate}
\item Computing Innovation Fellows (CIFellows); post my profile on \url{http://cifellows.org/profiles/}; also see \url{http://www.cifellows.org/}
\item Society for Industrial and Applied Mathematics: \vspace{-0.3cm}
	\begin{enumerate} \itemsep -2pt
	\item Job Search Resources for Students: \url{http://www.siam.org/careers/resources.php}
	\item SIAM Job Board: \url{http://jobs.siam.org/}
	\item Careers in the Math Sciences: \vspace{-0.2cm}
		\begin{enumerate} \itemsep -2pt
		\item \url{http://www.siam.org/careers/sinews.php}
		\item Has advice about: \vspace{-0.1cm}
			\begin{enumerate} \itemsep -1pt
			\item planning for my career (life after grad school)
			\item research careers in research institutes (and/or corporate research labs)
			\item seeking a junior academic position in the US
			\item obtaining/procuring reference letters
			\end{enumerate}
		\end{enumerate}
	\end{enumerate}
\item The European Mathematical Society: \vspace{-0.3cm}
	\begin{enumerate} \itemsep -2pt
	\item Open Positions: \url{http://www.euro-math-soc.eu/jobs.html}
	\end{enumerate}
\item Mathematical Optimization Society (MOS; formerly known as Mathematical Programming Society, MPS): \vspace{-0.3cm}
	\begin{enumerate} \itemsep -2pt
	\item Job Postings: \url{http://www.mathprog.org/?nav=links}
	\end{enumerate}
\item NSF Mathematical Sciences Institutes: \url{http://mathinstitutes.org/}
\end{enumerate}






%%%%%%%%%%%%%%%%%%%%%%%%%%%%%%%%%%%%%%%%%%%
\section{Resources Concerning the Application for Junior Faculty Positions}
\label{resourcesforjnrprof}

Resources concerning the application for junior faculty positions: \vspace{-0.3cm}
\begin{enumerate} \itemsep -4pt
\item Computing Research Association (CRA): \vspace{-0.3cm}
	\begin{enumerate} \itemsep -2pt
	\item \url{http://www.cra.org/for-students/}
	\item CRA's Job Announcements: \url{http://www.cra.org/ads/}
	\item Computing Postdoc Job Opportunities: \url{http://cifellows.org/opportunities}
	\item Computing Postdoc Profiles: \url{http://cifellows.org/profiles}
	\item Mailing lists of the Computer Research Association's Committee on the Status of Women in Computing Research (CRA-W): \vspace{-0.2cm}
		\begin{enumerate} \itemsep -2pt
		\item Subscribe to these and receive announcements concerning openings for junior faculty positions.
		\item \url{http://www.cra-w.org/mailinglists}
		\item \url{http://www.cra-w.org/PhdjobhuntHers}
		\end{enumerate}
	\end{enumerate}
\item American Association of University Professors: \vspace{-0.3cm}
	\begin{enumerate} \itemsep -2pt
	\item Career Center: \url{http://careercenter.aaup.org/search.cfm}
	\item Issues in Higher Education: \url{http://www.aaup.org/AAUP/issues/}
	\end{enumerate}
\item The Chronicle of Higher Education: \vspace{-0.3cm}
	\begin{enumerate} \itemsep -2pt
	\item Global Jobs: \url{http://chronicle.com/section/Global-Jobs/434/}
	\item Great Colleges to Work For: \vspace{-0.2cm}
		\begin{enumerate} \itemsep -2pt
		\item \url{http://chronicle.com/section/Great-Colleges-to-Work-For/156/}
		\item \url{http://chronicle.com/section/The-Academic-Workplace/156}
		\item \url{http://chronicle.com/article/Great-Colleges-to-Work-For/65724/}
		\end{enumerate}
	\item CV Doctor: \url{http://chronicle.com/article/The-CV-Doctor-Is-Back/49086/}
	\end{enumerate}
\item Nature Publishing Group (a division of Macmillan Publishers Limited): \vspace{-0.3cm}
	\begin{enumerate} \itemsep -2pt
	\item {\it naturejobs.com}: \vspace{-0.2cm}
		\begin{enumerate} \itemsep -2pt
		\item Jobs and careers: \url{http://www.nature.com/naturejobs/index.html}
		\item Career toolkit: \vspace{-0.1cm}
			\begin{enumerate} \itemsep -1pt
			\item \url{http://www.nature.com/naturejobs/career-toolkit/index.html}
			\item Salaries: \vspace{-0.1cm}
				\begin{itemize} \itemsep -1pt
				\item \url{http://www.nature.com/naturejobs/career-toolkit/salaries/index.html}
				\item Nature Publishing Group, ``Naturejobs international salary survey, 2010,'' in {\it naturejobs.com}: Career toolkit: Salaries: Salary survey, 2010. Available online at: \url{http://www.nature.com/naturejobs/salary/survey/2010/index.html}; last accessed on December 31, 2010.: \vspace{-0.1cm}
					\begin{itemize} \itemsep -1pt
					\item There are significant differences in the incomes of researchers between academia and industry in the U.S. and in Europe.
					\item Academics in Northern and Central Europe tend to make more than those in Southern Europe.
					\item Academics in the U.S. still dominate in terms of wages and benefits.
					\item There are less differences between the salaries of men and women in Southern Europe than the rest of Europe and the U.S..
					\end{itemize}
				\end{itemize}
			\item Podcasts: \vspace{-0.1cm}
				\begin{itemize} \itemsep -1pt
				\item Nature Publishing Group, ``Lecturer jobs,'' {\it naturejobs.com}: Career toolkit: Podcasts. Available online at: \url{http://media.nature.com/download/nature/podcast/naturejobs/naturejobs-2010-06-25.mp3}; last accessed on December 31, 2010. \vspace{-0.1cm}
					\begin{itemize} \itemsep -1pt
					\item Being a tenured/tenure-track professor involves spending about 20\% of my time on teaching, 10\% of my time on administrative activities, and the rest of my time (70\%) on research.
					\item Being a tenured/tenure-track or teaching professor involves: curricular design; and dealing with students from a broad spectrum of cultures and abilities (including students with disabilities and behavioral problems) -- undergraduates tend to start college in their late teens, and many of them are not sufficiently matured.
					\item The time spend on research would involve: managing a research lab; hiring students (undergraduates and grad students), postdocs, and technicians; and applying for funding.
					\item Two stages of getting tenure: 1) find a research topic/problem to work on, plan my research projects, broaden my social network of researchers so that I can collaborate with them or allow some members of the review committee for funding applications to get to know me, and be a co-instructor for graduate classes (\& possibly, lead the discussions/tutorials) and teach undergraduate classes; 2) write and submit research grant applications, \& start teaching classes as the primary instructor.
					\item Try to finish writing research papers based on my postdoctoral research projects prior to becoming a professor, so that I can start writing research grants based on my publications.
					\item Take note that my start-up fund as a junior professor may be low (say tens of thousands of dollars/Euro). This may prevent me from doing much research, unless only a small capital is needed to get my research started.
					\item When writing my grant proposal, specify and explain what would make my proposed technique stand out from the others.
					\item Give talks, including those in department seminars.
					\end{itemize}
				\end{itemize}
			\end{enumerate}
		\end{enumerate}
	\item Quirin Schiermeier, ``Graphic Detail: The real value of a scientist's wage,'' in {\it Nature}, Vol. 450, No. 7170, pp. 597, November 29, 2007. Available online at: \url{http://dx.doi.org/10.1038/450597a}; last accessed on December 31, 2010. [ ``Graphic Detail: The real value of a scientist's wage -- Researchers' spending power is not what it seems.'' ] \vspace{-0.2cm}
		\begin{enumerate} \itemsep -2pt
		\item Salaries of academics in Switzerland are the highest, but Switzerland has the highest cost of living.
		\item Salaries in Northern and Central Europe are better than those in Southern Europe.
		\item Also, salaries in the US and Japan are comparable to those in Northern and Central Europe.
		\item Salaries in Israel are comparable to those in Southern Europe. E.g., they are comparable to those in France and Italy.
		\end{enumerate}
	\end{enumerate}
\item Times Higher Education (THE): \vspace{-0.3cm}
	\begin{enumerate} \itemsep -2pt
	\item Advanced Job Search: \url{http://www.timeshighereducation.co.uk/jobs_home.asp?navCode=84}
	\item Career: \url{http://www.timeshighereducation.co.uk/section.asp?navcode=96}
	\end{enumerate}
\item {\it Mathematical Association of America} (MAA): \vspace{-0.3cm}
	\begin{enumerate} \itemsep -2pt
	\item Information for new Ph.D.s seeking academic careers
	\item \url{http://www.maa.org/careers/}
	\item MAA Math Classifieds: \url{http://www.mathclassifieds.org/home/index.cfm?site_id=1925}
	\end{enumerate}
\item American Mathematical Society: \vspace{-0.3cm}
	\begin{enumerate} \itemsep -2pt
	\item Information on applying for positions in academia: \url{http://www.ams.org/programs/students/programs/students/gradinfo/gradinfo}
	\item \url{http://www.ams.org/profession/career-info/new-phds/new-phds}
	\end{enumerate}
\item Association for Women in Science (AWIS) career library: \url{http://www.awis.affiniscape.com/displaycommon.cfm?an=1&subarticlenbr=249}
\item American Psychological Association's resources for grad students and postdocs: \url{http://www.apa.org/education/grad/index.aspx}
\item {\it HigherEdJobs.com}: \url{http://www.higheredjobs.com/}
\item IEEE Real World Engineering Projects (RWEP): \vspace{-0.3cm}
	\begin{enumerate} \itemsep -2pt
	\item resources to help plan and run projects in EECS and related fields that will benefit humanity
	\item \url{http://www.realworldengineering.org/}
	\end{enumerate}
\item iBerry (collection of job lists): \url{http://iberry.com/cms/jobs.htm}
\item Common CV Network (for research and academic positions in Canada): \url{http://www.cvcommun.net/index_e.html}
\item University of Pennsylvania: \vspace{-0.3cm}
	\begin{enumerate} \itemsep -2pt
	\item Stephanie Weirich, {\it Computer Science Faculty Job Search Resources}, Department of Computer and Information Science, School of Engineering and Applied Science, University of Pennsylvania. Available at: \url{http://www.seas.upenn.edu/~sweirich/resources.htm}; last accessed on September 5, 2010.
	\end{enumerate}
\item {\tt Blue Lab Coats}, {\it Academic Job Applications}. Available at: \url{http://bluelabcoats.wordpress.com/application-pkgs/}; last accessed on September 10, 2010.
\item Inside Higher Ed: \url{http://www.insidehighered.com/career/seekers}
\item Advice on giving the job talk: \vspace{-0.3cm}
	\begin{enumerate} \itemsep -2pt
	\item John Farrell, ``What to Say in a Good Research Talk,'' Department of Computer Science, James Cook University, May 1994. Available at: \url{http://www.computersciencestudent.com/SS/HowTo/ResearchTalkJohnFarrell.html}; last accessed on August 25, 2010.
	\end{enumerate}
%%%%%%%%%%%%%%%%%%%%%%%%%%%%
\item --- --- --- --- --- --- --- --- --- --- --- --- --- --- --- --- --- --- --- --- --- --- --- --- --- --- --- --- --- --- ---
% Residential Education
\item \colorbox{blue}{\bf Residential Education}: \vspace{-0.3cm}
	\begin{enumerate} \itemsep -2pt
	\item Telluride Association: \vspace{-0.2cm}
		\begin{enumerate} \itemsep -2pt
		\item Information about how to become a Faculty Fellow at the Cornell Branch and the Michigan Branch of Telluride Association, which are ``residential colleges'': \url{http://www.tellurideassociation.org/programs/college_faculty.html}
		\end{enumerate}
	\end{enumerate}
%%%%%%%%%%%%%%%%%%%%%%%%%%%%
\item --- --- --- --- --- --- --- --- --- --- --- --- --- --- --- --- --- --- --- --- --- --- --- --- --- --- --- --- --- --- ---
% Awards, grants, and fellowships to look out for
\item \colorbox{blue}{\bf Awards, grants, and fellowships to look out for}: \vspace{-0.3cm}
	\begin{enumerate} \itemsep -2pt
	\item NSF Career Grant / NSF Career Award
	\item Heinz Family Philanthropies, The Heinz Awards: \url{http://www.heinzawards.net/awards}
	\item Sloan Research Fellowships; Alfred P. Sloan Foundation: \vspace{-0.2cm}
		\begin{enumerate} \itemsep -2pt
		\item Applicants must ``hold a Ph.D. (or equivalent) in chemistry, physics, mathematics, computer science, economics, neuroscience or computational and evolutionary molecular biology, or in a related interdisciplinary field''
		\item Applicants must be tenure-track junior faculty in a North American college or university
		\item \url{http://www.sloan.org/fellowships}
		\end{enumerate}
	\item Santa Fe Institute's Miller Distinguished Scholarship (for excellent interdisciplinary researchers): \url{http://www.santafe.edu/research/miller-scholars/}
	\item Lemelson-MIT Prize (for mid-career innovators): \url{http://web.mit.edu/invent/a-prize.html}
	\item Research Corporation for Science Advancement: \vspace{-0.2cm}
		\begin{enumerate} \itemsep -2pt
		\item Cottrell College Science Awards (for science researchers at predominantly undergraduate US colleges and universities): \url{http://www.rescorp.org/cottrell-college-science-awards}
		\item Cottrell Scholar Awards (for science researchers at US colleges and universities): \url{http://www.rescorp.org/cottrell-scholar-awards/}
		\end{enumerate}
	\item {\it Semiconductor Industry Association} University Researcher Awards: \url{http://www.sia-online.org/cs/papers_publications/press_release_detail?pressrelease.id=1722}
	\item Packard Fellowships for Science and Engineering (for professors at selected US universities): \url{http://www.packard.org/genericDetails.aspx?RootCatID=3&CategoryID=152}
	\item {\it American Society for Engineering Education} awards: \url{http://www.asee.org/activities/awards/division.cfm}
	\item Hellman Fellowship in Science and Technology Policy (for early-career professionals with training in science or engineering who are interested in transitioning to a career in public policy and administration): \url{http://www.amacad.org/hellman.aspx}
	\item Simons Foundation's {\it Collaboration Grants for Mathematicians}: \url{https://simonsfoundation.org/funding-guidelines/current-funding-opportunities/collaboration-grants-for-mathematicians}
	\item The National Science Foundation: \vspace{-0.2cm}
		\begin{enumerate} \itemsep -2pt
		\item Alan T. Waterman Award (for junior faculty in the US): \url{http://www.nsf.gov/od/waterman/waterman.jsp} 
		\end{enumerate}
	\item Computing Research Association (CRA): A. Nico Habermann Award, \url{http://www.cra.org/awards/habermann-current/}
	\item Microsoft New Faculty Fellowship Award: \url{http://research.microsoft.com/en-us/collaboration/awards/msrff_all.aspx#2005}
	\item Global Semiconductor Alliance (GSA): Dr. Morris Chang Exemplary Leadership Award
	\end{enumerate}
%%%%%%%%%%%%%%%%%%%%%%%%%%%%
\item --- --- --- --- --- --- --- --- --- --- --- --- --- --- --- --- --- --- --- --- --- --- --- --- --- --- --- --- --- --- ---
% Underrepresented minority outreach
\item \colorbox{blue}{\bf Underrepresented minority outreach}: \vspace{-0.3cm}
	\begin{enumerate} \itemsep -2pt
	\item The Mathematical Association of America: \vspace{-0.2cm}
		\begin{enumerate} \itemsep -2pt
		\item Pre-College Programs: \url{http://www.maa.org/funding/pre_college.html}
		\item Programs for Undergraduate Students: \url{http://www.maa.org/funding/undergraduate.html}
		\item Special Interest Group of the MAA on Research in Undergraduate Mathematics Education (SIGMAA RUME): \url{http://sigmaa.maa.org/rume/}. Also, see \url{http://sigmaa.maa.org/rume/highlights.html}.
		\end{enumerate}
	\end{enumerate}
%%%%%%%%%%%%%%%%%%%%%%%%%%%%
\item --- --- --- --- --- --- --- --- --- --- --- --- --- --- --- --- --- --- --- --- --- --- --- --- --- --- --- --- --- --- ---
% Resources for preparing the teaching statement, and developing a teaching philosophy and methodology
\item \colorbox{blue}{\bf Resources for preparing the teaching statement, and developing a teaching philosophy and methodology}: \vspace{-0.3cm}
	\begin{enumerate} \itemsep -2pt
	\item Missouri University of Science and Technology (formerly University of Missouri-Rolla): \vspace{-0.2cm}
		\begin{enumerate} \itemsep -2pt
		\item Department of Electrical and Computer Engineering: \vspace{-0.1cm}
			\begin{enumerate} \itemsep -1pt
			\item Cp Eng 382 - Teaching Engineering (Summer 20XY): \vspace{-0.1cm}
				\begin{itemize} \itemsep -1pt
				\item \url{http://ece.mst.edu/documents/CpE_382.doc}
				\item Learn about: \vspace{-0.1cm}
					\begin{itemize} \itemsep -1pt
					\item teaching objectives and techniques
					\item knowledge taxonomies
					\item cognitive development
					\end{itemize}
				\item Learn how to: \vspace{-0.1cm}
					\begin{itemize} \itemsep -1pt
					\item use course objectives to design a course
					\item communicate using traditional and cutting-edge media, which includes making effective presentations via various communication modes, including traditional lecture, transparencies, and PowerPoint slides
					\item select textbook(s) and recommended books
					\item assess student learning
					\item grade/evaluate assignments, presentations, and projects in a way that addresses the psychological aspects of learning
					\item recognize and adapt to different student learning styles, 
					\item facilitate cooperative/active learning
					\item improve student discipline
					\end{itemize}
				\item Develop course content and student evaluation techniques to meet the elements of ABET criteria
				\item Critique evaluation methods including homework, exams, quizzes, reports/papers, and portfolios
				\item Learn to speak and/or write competently on educational hot topics, including collaborative learning, distance education, self-paced learning, web tools/educational software, and learning communities
				\item Develop course objectives
				\item Describe the parts of a syllabus, and prepare a syllabus
				\item Describe methods to prevent cheating
				\item Describe the elements of effective student advising at both the undergraduate and graduate levels
				\item List what makes one a good listener
				\item Prepare an academic application and identify the issues involved in the academic hire
				\item Describe appropriate responses to student situations such as death, depression, cheating, pregnancy, hugs�
				\item Name sources of additional teaching training
				\end{itemize}
			\end{enumerate}
		\end{enumerate}
	\item Sergio Toral Marin, Mar{\'{i}}a del Roc{\'{i}}o Mart{\'{i}}nez-Torres, Federico Barrero, ``Reforming ICT Graduate Programs to Meet Professional Needs,'' {\it Computer}, vol. 43, no. 10, IEEE Computer Society Press, pp. 22-29, June 2010, DOI:10.1109/MC.2010.186.: \vspace{-0.2cm}
		\begin{enumerate} \itemsep -2pt
		\item To access this article, use its DOI Bookmark: \url{http://doi.ieeecomputersociety.org/10.1109/MC.2010.186} or \url{http://dx.doi.org/10.1109/MC.2010.186}.
		\item It discusses how ICT Masters degree programs in Europe can meet the needs and demands of the ICT industry and society. That is, it discusses how well European universities empower Masters students for jobs and careers in the ICT industry.
		\item Note that the definition and usage of ICT is non-standard among researchers of computer science and engineering higher education in the United States. Here, I am referring to researchers (mostly professors and Ph.D. students) who do research about:: \vspace{-0.1cm}
			\begin{enumerate} \itemsep -1pt
			\item the quality of computer science and engineering education;
			\item how well do these education programs empower computer science and engineering students to pursue the careers that they desire; and
			\item issues concerning the diversity of the student body and faculty in computer science and engineering.
			\end{enumerate}
		\item ICT is reasonably well-defined and used in Europe, and poorly defined and used elsewhere (especially in Asia).
		\item In American, programs related to the computing professions are: computer science; computer engineering; electrical engineering; and information systems. There may be more esoteric ``majors'' (as as ``gaming and animation''), but these majors don't usually have recommended frameworks and standards for their curriculum.
		\end{enumerate}
	\item ACM and IEEE Computer Society are professional organizations for the IT industry (and related industries, such as the semiconductor and electronic industries) that are leaders in helping educators plan and design their curriculum in computer science and engineering.: \vspace{-0.2cm}
		\begin{enumerate} \itemsep -2pt
		\item A gist of what computer science and computer engineering degree programs should empower students to have are listed in the short article on, ``Skills You'll Learn if You Study Computing''. Available online at: \url{http://computingcareers.acm.org/?page_id=15}; last accessed on November 22, 2010.
		\item Recommended computer science and computer engineering curricular can be found online at: \url{http://www.acm.org/education/curricula-recommendations} or \url{http://www.computer.org/portal/web/education/Curricula}; last accessed on November 22, 2010: \vspace{-0.1cm}
			\begin{enumerate} \itemsep -1pt
			\item As a Ph.D. student in computer science (or computer engineering), you should have most of the skills and knowledge specified in the recommended curricular for students in computer science (or computer engineering).
			\item You mostly probably need these skills and knowledge to pass your Ph.D. preliminary exam in a decent US CS Ph.D. program.
			\end{enumerate}
		\item ACM Special Interest Group on Computer Science Education (SIGCSE): \url{http://www.sigcse.org/}
		\item Association for Computing Machinery's Special Interest Group for Information Technology Education (SIGITE): \url{http://www.sigite.org/}
		\item ``Future of Computing Education Summit'': \url{http://www.acm.org/education/future-of-computing-education-summit}
		\end{enumerate}
	\item ABET, Inc. (formerly the Accreditation Board for Engineering and Technology): \vspace{-0.2cm}
		\begin{enumerate} \itemsep -2pt
		\item ABET, {\it Download Accreditation Criteria and Forms}, 2010: \vspace{-0.1cm}
			\begin{enumerate} \itemsep -1pt
			\item \url{http://www.abet.org/forms.shtml}
			\item Look at the sections on ``student outcomes,'' and ``program criteria (including curriculum).''
			\item ABET Engineering Accreditation Commission, ``Criteria for Accrediting Engineering Programs: Effective for Evaluations During the 2011-2012 Accreditation Cycle,'' ABET, Baltimore, MD, October 30, 2010. Available online at: \url{http://www.abet.org/Linked%2520Documents-UPDATE/Program%2520Docs/abet-eac-criteria-2011-2012.pdf}; last accessed on November 22, 2010.
			\item ABET Engineering Accreditation Commission, ``Criteria for Accrediting Computing Programs: Effective for Evaluations During the 2011-2012 Accreditation Cycle,'' ABET, Baltimore, MD, October 30, 2010. Available online at: \url{http://www.abet.org/Linked%2520Documents-UPDATE/Program%2520Docs/abet-cac-criteria-2011-2012.pdf}; last accessed on November 22, 2010.
			\end{enumerate}
		\end{enumerate}
	\item American Institute of Mathematics: \vspace{-0.2cm}
		\begin{enumerate} \itemsep -2pt
		\item Resources for the Math Community: \vspace{-0.1cm}
			\begin{enumerate} \itemsep -1pt
			\item \url{http://www.aimath.org/mathcommunity/}
			\item David W. Farmer, ``The AIM REU: individual projects with a common theme,'' in the {\it Proceedings of the Conference on Promoting Undergraduate Research in Mathematics}, American Mathematical Society, 2006. Available online at: \url{http://www.aimath.org/mathcommunity/farmerREU.pdf}; last accessed on January 9, 2010. [ ``AIM Research Experience for Undergraduates (REU)'' ]
			\item Sally Koutsoliotas and David W. Farmer, ``Preparing students to give talks,'' American Institute of Mathematics. Available online at: \url{http://www.aimath.org/mathcommunity/studenttalks.pdf}; last accessed on January 9, 2010. [ ``Preparing students to give talks'' ]
			\end{enumerate}
		\end{enumerate}
	\end{enumerate}
\end{enumerate}











%%%%%%%%%%%%%%%%%%%%%%%%%%%%%%%%%%%%%%%%%%%
\section{Grad School Information}
\label{gradschinfo}

Things for me to look into when considering graduate and professional programs: \vspace{-0.3cm}
\begin{enumerate} \itemsep -4pt
\item Is the graduate or professional program accredited?
\item Would attending the graduate/professional program help me obtain my career goals?
\item Would attending the graduate/professional program help me obtain my career goals?
\item See \url{http://gradschool.princeton.edu/facts/time_to_degree/naturalsciences/} for statistics of graduates from various programs at Princeton University.
\item Some statistics that you may wanna look into include, average number of Ph.D. students per Ph.D. advisor (tells you about advisor-to-advisee ratios), number of Ph.D. graduates for the Ph.D. program per year (tells you about the size of the department), graduate outcomes (what did alumni do after their Ph.D. programs), drop-out rate, failure rate for Ph.D. preliminary and qualifying examinations, and reasons for dropping out of the Ph.D. program.
\item Some research labs do list the names of alumni members (Ph.D. and Masters students, undergraduates, and postdocs). They may also include their first vocation after their Ph.D. program, and what are they currently doing. You can use this to help you gauge the strengths/weaknesses of the lab/advisor.
\item Accommodation options. Is it hard to find housing in the surrounding area? Is housing in the surrounding area cheap? What are some of the statistics of the local population? LA Times has some statistics of local populations in LA. You can determine the average income, highest level of education, political views, and so on... Try to determine if there is an equivalent of that for the local area.
\item 
\end{enumerate}


Grad school info: \vspace{-0.3cm}
\begin{itemize} \itemsep -4pt
%%%%%%%%%%%%%%%%%%%%%%%%%%%%
\item general information and advice about grad school: \vspace{-0.3cm}
	\begin{enumerate} \itemsep -2pt
	\item IEEE: \vspace{-0.2cm}
		\begin{enumerate} \itemsep -2pt
		\item Susan Karlin, ``How to Choose A Grad School: Figure out what you want and who can give it to you,'' {\it IEEE Spectrum}, September 2005. Available at: \url{http://spectrum.ieee.org/at-work/education/how-to-choose-a-grad-school}; last accessed on August 28, 2010.
		\item IEEE Potentials, Volume 21, Issue 3, Aug/Sep 2002.
		\item IEEE Potentials, Volume 24, Issue 3, Aug/Sep 2005.
		\end{enumerate}
	\item ACM: \vspace{-0.2cm}
		\begin{enumerate} \itemsep -2pt
		\item ACM, {\it ACM Crossroads Student Resources}, ACM, New York, NY, Aug 17, 2005. Available at: \url{http://oldwww.acm.org/crossroads/resources/}; last accessed on August 29, 2010.
		\item ACM, {\it Graduate Educational Resources from ACM Crossroads}, ACM, New York, NY, Aug 7, 2005. Available at: \url{http://oldwww.acm.org/crossroads/resources/graduate.html}; last accessed on August 29, 2010.
% http://oldwww.acm.org/crossroads/resources/graduate.html
		\item {\it ACM Crossroads} articles on grad school application and life in grad school: \cite{Agre1997,desJardins1994,desJardins1995,desJardins2001,desJardins2008,Pottinger1999,Wendler2010}.	
		\end{enumerate}
	\item European Commission: \vspace{-0.2cm}
		\begin{enumerate} \itemsep -2pt
		\item Marie Curie Actions: \url{http://ec.europa.eu/research/mariecurieactions/index.htm}... {\bf \color{blue} SCHOLARSHIPS!!!}
		\item {\it EURAXESS} Research Job Vacancies: \url{http://ec.europa.eu/euraxess/index.cfm/jobs/jvSearch} or \url{http://ec.europa.eu/euraxess/index_en.cfm?l1=13&l2=3&initSearch=1#}... {\bf \color{blue} SCHOLARSHIPS!!!}
		\item European Commission, ``Third European Report on Science \& Technology Indicators 2003,'' Directorate-General for Research, European Commission. Available at: \url{http://cordis.europa.eu/indicators/contacts.htm}; last accessed on September 1, 2010. Report on European universities and research initiatives.
		\item Publications by the Directorate-General for Research, European Commission, about research initiatives and output in Europe: \url{http://cordis.europa.eu/indicators/publications.htm}
		\end{enumerate}
	\item American Psychological Association: \vspace{-0.2cm}
		\begin{enumerate} \itemsep -2pt
		\item American Psychological Association, {\it gradPSYCH} [ magazine ], American Psychological Association, Washington, DC. Available at: \url{http://www.apa.org/gradpsych/}; last accessed on September 1, 2010. [ Read issues from May 2003 till January 2005; {\color{blue} \bf This is an excellence source of information about doing well in graduate school and internships, and seeking research careers in academia and the industry.} ]
		\end{enumerate}
	\item Computing Research Association (CRA): \vspace{-0.2cm}
		\begin{enumerate} \itemsep -2pt
		\item Information for Undergraduate and Graduate Students: \url{http://www.cra.org/for-students/}
		\item Computer Research Association's Committee on the Status of Women in Computing Research (CRA-W), ``Graduate Student Information Guide''. Available at: \url{http://www.cra-w.org/sites/default/files/grad-guide.pdf}; CRA-W $\Rightarrow$ Resources $\Rightarrow$ Publications $\Rightarrow$ link and description to the guide, ``Graduate Student Information Guide''; last accessed on September 3, 2010.
		\end{enumerate}
	\item American Mathematical Society: \vspace{-0.2cm}
		\begin{enumerate} \itemsep -2pt
		\item {\it Applying to Graduate School}. Available at: \url{http://www.ams.org/profession/career-info/grad-school/grad-school}; last accessed on September 2, 2010.
		\end{enumerate}
	\item The Mathematical Association of America: \vspace{-0.2cm}
		\begin{enumerate} \itemsep -2pt
		\item {\it MAA Students}. Available at: \url{http://www.maa.org/students/}; last accessed on September 2, 2010.
		\end{enumerate}
	\item American Institute of Mathematics: \vspace{-0.2cm}
		\begin{enumerate} \itemsep -2pt
		\item Resources for the Math Community: \vspace{-0.1cm}
			\begin{enumerate} \itemsep -1pt
			\item \url{http://www.aimath.org/mathcommunity/}
			\item David W. Farmer, ``The AIM REU: individual projects with a common theme,'' in the {\it Proceedings of the Conference on Promoting Undergraduate Research in Mathematics}, American Mathematical Society, 2006. Available online at: \url{http://www.aimath.org/mathcommunity/farmerREU.pdf}; last accessed on January 9, 2010. [ ``AIM Research Experience for Undergraduates (REU)'' ]
			\item Sally Koutsoliotas and David W. Farmer, ``Preparing students to give talks,'' American Institute of Mathematics. Available online at: \url{http://www.aimath.org/mathcommunity/studenttalks.pdf}; last accessed on January 9, 2010. [ ``Preparing students to give talks'' ]
			\end{enumerate}
		\end{enumerate}
	\item University of California, Berkeley: \vspace{-0.2cm}
		\begin{enumerate} \itemsep -2pt
		\item Matthew Moskewicz, Parallel Computing Laboratory (Par Lab), Department of Electrical Engineering and Computer Sciences: \vspace{-0.1cm}
			\begin{enumerate} \itemsep -1pt
			\item Developed the {\it Chaff} SAT solver with Conor Madigan as undergrads that is 10-100X faster than then existing SAT solvers.
			\item He is named a co-winner of the 2009 CAV Award, along with his co-developers of {\it Chaff} and the developers of the {\it GRASP} SAT solver, for his fundamental contribution to the field of Computer Aided Verification.
			\item \url{http://www.princeton.edu/engineering/eqnews/spring01/feature5.html}
			\item \url{http://parlab.eecs.berkeley.edu/people/matthew-moskewicz}
			\end{enumerate}
		\item Mark Borgschulte, ``Economics Grad School Application Advice,'' Department of Economics, University of California, Berkeley. Available online at: \url{http://sites.google.com/site/markborgschulte/economicsgradschoolapplicationadvice}; last accessed on January 9, 2010.
		\item Mark Borgschulte, ``Info for Berkeley Admits,'' Department of Economics, University of California, Berkeley. Available online at: \url{http://sites.google.com/site/markborgschulte/infoforberkeleyadmits}; last accessed on January 9, 2010.
		\item Mark Borgschulte, ``Reading List,'' Department of Economics, University of California, Berkeley. Available online at: \url{http://sites.google.com/site/markborgschulte/readinglist}; last accessed on January 9, 2010.
		\item {\it Secret Blogging Seminar} is a blog written by recent Ph.D. graduates from Berkeley's Department of Mathematics. Noah Snyder, ``Thoughts on graduate school,'' May 13, 2009. Available at: \url{http://sbseminar.wordpress.com/2009/05/13/thoughts-on-graduate-school/}; last accessed on September 1, 2010.
		\item {\it UC Berkeley Career Center}, ``Graduate School - Letters of Recommendation,'' UC Berkeley. Available at: \url{https://career.berkeley.edu/grad/gradletter.stm}; last accessed on September 5, 2010.
		\end{enumerate}
	\item Carnegie Mellon University, Computer Science Department: \vspace{-0.2cm}
		\begin{enumerate} \itemsep -2pt
		\item Carnegie Mellon University, {\it Ph.D. in Computer Science}, Computer Science Department, Carnegie Mellon University. Available at: \url{http://www.csd.cs.cmu.edu/education/phd/index.html}; last accessed on August 28, 2010.
		\item Mark Leone, {\it Advice on Research and Writing}, Computer Science Department, Carnegie Mellon University. Available at: \url{http://www-2.cs.cmu.edu/afs/cs.cmu.edu/user/mleone/web/how-to.html}; last accessed on August 28, 2010. Also, see \url{http://www.cs.cmu.edu/~mleone/how-to.html} for another copy. [Mark Leone has graduated with a MS CS from CMU.]
		\item Jason I. Hong, {\it Grad School Advice}, Human Computer Interaction Institute, School of Computer Science, Carnegie Mellon University, Sept 20, 2006. Available online at: \url{http://www.cs.cmu.edu/~jasonh/advice.html}; last accessed on December 17, 2010.
		\item women@SCS School of Computer Science: \vspace{-0.1cm}
			\begin{enumerate} \itemsep -1pt
			\item Career Advice: \url{http://women.cs.cmu.edu/Resources/JobsResearch/careeradvice.php}
			\end{enumerate}
		\end{enumerate}
	\item University of California, San Diego: \vspace{-0.2cm}
		\begin{enumerate} \itemsep -2pt
		\item UCSD VLSI CAD Laboratory, {\it Useful tips on how to succeed in graduate school and your subsequent research career}, Department of Computer Science and Engineering \& Department of Electrical and Computer Engineering, University of California, San Diego. Available at: \url{http://vlsicad.ucsd.edu/Research/Advice/index.html}; last accessed on August 28, 2010. \colorbox{yellow}{\bf EXCELLENT!!!}
		\item Mihir Bellare, ``The Ph.D Experience,'' Department of Computer Science \& Engineering, University of California at San Diego. Available at: \url{http://cseweb.ucsd.edu/~mihir/phd.html}; last accessed on September 13, 2010. [ See {\it Educational} material about graduate school, research, and technical writing at: \url{http://cseweb.ucsd.edu/users/mihir/education.html}. ]
		\item Fan Chung Graham, ``A few words on research for graduate students (especially for those potential combinatorialists),'' in {\it Teaching}, Department of Mathematics, University of California, San Diego. Available online at: \url{http://math.ucsd.edu/~fan/teach/gradpol.html}; last accessed on January 9, 2010.
		\end{enumerate}
	\item University of Michigan, Ann Arbor: \vspace{-0.2cm}
		\begin{enumerate} \itemsep -2pt
		\item  Igor Markov, Department of Electrical Engineering and Computer Science: \url{http://www.eecs.umich.edu/~imarkov/i_students.html}. In particular, see his ``Advice for graduate students''.
		\end{enumerate}
	\item University of California, Los Angeles: \vspace{-0.2cm}
		\begin{enumerate} \itemsep -2pt
		\item Philip E. Agre, ``Advice for Undergraduates Considering Graduate School,'' UCLA Department of Information Studies, University of California, Los Angeles, October 1996 (Modified: May 2001). Available at: \url{http://polaris.gseis.ucla.edu/pagre/grad-school.html}; last accessed on August 28, 2010. See \url{http://polaris.gseis.ucla.edu/pagre/grad-school.pdf} for a PDF copy of this article. \colorbox{yellow}{\bf CLASSIC!!!}. See \url{http://polaris.gseis.ucla.edu/pagre/index.html} for more articles.
		\item Terence Tao, {\it Career advice}, Department of Mathematics, University of California, Los Angeles. Available at: \url{http://terrytao.wordpress.com/career-advice/}; last accessed on September 1, 2010. Additional information can be found at: \url{http://www.math.ucla.edu/~tao/}.
		\item Yu Hu: \vspace{-0.1cm}
			\begin{enumerate} \itemsep -1pt
			\item Yu Hu, ``Links: Programming Tools and Tips; and Documentation and Presentation,'' Electrical Engineering Department, University of California, Los Angeles, Aug 27, 2007. Available online at: \url{http://www.ee.ucla.edu/~hu/links.htm}; last accessed on September 18, 2010. [ This web page has resources for computer programming, and creating presentation slides and documentations. ]
			\item Courses @ UCLA, April 22, 2006: \url{http://www.ee.ucla.edu/~hu/course.htm}
			\item Research: \url{http://www.ee.ucla.edu/~hu/project.htm}
			\end{enumerate}
		\end{enumerate}
	\item Stanford University: \vspace{-0.2cm}
		\begin{enumerate} \itemsep -2pt
		\item John Ousterhout, ``My Favorite Sayings,'' Department of Computer Science, Stanford University, September 09, 2009. Available at: \url{http://www.stanford.edu/~ouster/cgi-bin/sayings.php}; last accessed on September 4, 2010. [Also, see {\it Odds \& Ends}: \url{http://www.stanford.edu/~ouster/cgi-bin/misc.php} ]
		\item Jeffrey Michael Heer, Department of Computer Science: \vspace{-0.1cm}
			\begin{enumerate} \itemsep -1pt
			\item The only Ph.D. student to have ever won the Microsoft Graduate Fellowship and the IBM Ph.D. Fellowship concurrently. After he won these fellowships as a Ph.D. student at Berkeley, IBM changed the rules for its Ph.D. fellowship so that nobody else can do this anymore. This prevents other companies from competing with IBM for hiring these fellows as research interns.
			\item \url{http://hci.stanford.edu/jheer/cv/}
			\end{enumerate}
		\item Ravi Vakil, ``For potential students,'' Department of Mathematics, Stanford University. Available at: \url{http://math.stanford.edu/~vakil/potentialstudents.html}; last accessed on September 1, 2010. [ ``Great articles and books'' in mathematics: \url{http://math.stanford.edu/~vakil/greatwriting.html}. Information about getting/writing letters of recommendation: \url{http://math.stanford.edu/~vakil/recommendations.html}. ]
		\item Philip Guo, {\it Academic Home Page}, Department of Computer Science, Stanford University. Available at: \url{http://www.stanford.edu/~pgbovine/academic.htm}; last accessed on September 1, 2010. [ See resources at the bottom of the page. Also, see \url{http://www.stanford.edu/~pgbovine/writings.htm} for his non-academic/research articles. ]
		\item Stanford University, {\it Tomorrow's Professor$^{\rm SM}$ Mailing List Links}, Center for Teaching and Learning, Stanford University. Available at: \url{http://www.stanford.edu/dept/CTL/Tomprof/links.html}; last accessed on September 1, 2010.
		\item Eran Magen, {\it How I Got Into the Stanford Psychology Ph.D. Program}, Department of Psychology, Stanford University. Available at: \url{http://www.howigotintostanford.com/}; last accessed on September 1, 2010.
		\item Stanford University, {\it Tutoring and Academic Support}, [Office of the] Vice Provost for Undergraduate Education, Stanford University. Available at: \url{http://ual.stanford.edu./ARS/index.html}; last accessed on September 1, 2010.
		\item Stanford University, ``Guidelines for Advising Relationships between Faculty Advisors and Graduate Students,'' Office of the Vice Provost for Graduate Education (VPGE), Stanford University, 2009. Available online at: \url{http://vpge.stanford.edu/docs/Advisor_Guidelines.pdf}; last accessed on December 22, 2010.
		\end{enumerate}
	\item University of Washington: \vspace{-0.2cm}
		\begin{enumerate} \itemsep -2pt
		\item University of Washington, {\it 10-Year Review Self-Study}, Department of Computer Science \& Engineering, University of Washington, January 2000. Available at: \url{http://www.cs.washington.edu/homes/lazowska/selfstudy/}; last accessed on September 2, 2010. See other information on Prof. Ed Lazowska's web page: \url{http://www.cs.washington.edu/homes/lazowska/}.
		\item Yuriy Brun, {\it Yuriy Brun's Advice}, Department of Computer Science \& Engineering, University of Washington. Available at: \url{http://www.cs.washington.edu/homes/brun/advice/}; last accessed on August 28, 2010. See \url{http://www.cs.washington.edu/homes/brun/advice/PhDAdvice.pdf} for: Yuriy Brun, ``Getting a Ph.D. at the University of Southern California,'' May 20, 2010.
		\item Michael Ernst, {\it Advice for researchers and students}, Department of Computer Science \& Engineering, University of Washington. Available at: \url{http://www.cs.washington.edu/homes/mernst/advice/}; last accessed on August 28, 2010.
		\item Karin Strauss, {\it For graduate students} [ see the links on the left side of her home page ]. Available at: \url{http://www.cs.washington.edu/homes/kstrauss/}; last accessed on September 3, 2010.
		\item Wanda Pratt, {\it Advice}, Information School \& Division of Biomedical \& Health Informatics / Department of Medical Education and Biomedical Informatics / School of Medicine, University of Washington. Available at: \url{http://faculty.washington.edu/wpratt/advice.htm}; last accessed on September 3, 2010.
		\item William A. Stein, {\it Home Page}, Department of Mathematics, University of Washington. Available at: \url{http://wstein.org/}; last accessed on September 5, 2010. {\bf \color{blue} [ Has GREAT resources for junior faculty application and research grant proposals. He has provided tar balls (or zip files) and PDF files of these material. ]}
		\item University of Washington Graduate School, {\it Re-envisioning the Ph.D. project}, University of Washington Graduate School, University of Washington. Available at: \url{http://www.grad.washington.edu/envision/index.html}; last accessed on August 28, 2010. [This research is about issues concerning Ph.D. programs, such as: how to improve the quality of Ph.D. programs and student outcomes, and the lifestyle (including social life) of Ph.D. students; and funding issues.]
		\end{enumerate}
	\item Duke University: \vspace{-0.2cm}
		\begin{enumerate} \itemsep -2pt
		\item Xiaowei Yang, {\it Advice Collection}, Department of Computer Science, Duke University. Available at: \url{http://www.cs.duke.edu/~xwy/advices.html}; last accessed on August 28, 2010.
		\end{enumerate}
	\item Columbia University: \vspace{-0.2cm}
		\begin{enumerate} \itemsep -2pt
		\item Department of Economics: \vspace{-0.1cm}
			\begin{enumerate} \itemsep -1pt
			\item Donald R. Davis, ``Ph.D. Thesis Research: Where do I Start?'', Department of Economics, Columbia University, February 2001. Available online at: \url{http://www.columbia.edu/~drd28/Thesis%20Research.pdf}; last accessed on January 9, 2010.
			\end{enumerate}
		\end{enumerate}
	\item Princeton University: \vspace{-0.2cm}
		\begin{enumerate} \itemsep -2pt
		\item Boaz Barak, Department of Computer Science: \vspace{-0.1cm}
			\begin{enumerate} \itemsep -1pt
			\item He won the ACM Doctoral Dissertation Award.
			\item \url{http://awards.acm.org/doctoral_dissertation/}
			\item \url{http://www.cs.princeton.edu/~boaz/}
			\end{enumerate}
		\end{enumerate}
	\item The University of Texas at Austin: \vspace{-0.2cm}
		\begin{enumerate} \itemsep -2pt
		\item Department of Computer Science: \vspace{-0.1cm}
			\begin{enumerate} \itemsep -1pt
			\item Mike Dahlin, ``Advice to systems researchers,'' Department of Computer Science, The University of Texas at Austin. Available online at: \url{http://www.cs.utexas.edu/users/dahlin/advice.html}; last accessed on January 9, 2010.
			\end{enumerate}
		\end{enumerate}
	\item University of Pennsylvania: \vspace{-0.2cm}
		\begin{enumerate} \itemsep -2pt
		\item Stephanie Weirich, {\it Advice for Graduate Studies}, Department of Computer and Information Science, School of Engineering and Applied Science, University of Pennsylvania. Available at: \url{http://www.seas.upenn.edu/~sweirich/phdadvice.htm}; last accessed on September 5, 2010.
		\end{enumerate}
	\item Northwestern University: \vspace{-0.2cm}
		\begin{enumerate} \itemsep -2pt
		\item Department of Electrical Engineering and Computer Science; Robert R. McCormick School of Engineering and Applied Science: \vspace{-0.1cm}
			\begin{enumerate} \itemsep -1pt
			\item Lance Fortnow and William Gasarch, ``Graduate Student Guide,'' in their blog {\it Computational Complexity}, Department of Electrical Engineering and Computer Science, Robert R. McCormick School of Engineering and Applied Science, Northwestern University, February 21, 2007. Available at: \url{http://blog.computationalcomplexity.org/2007/02/graduate-student-guide.html}; last accessed on September 14, 2010. [ William Gasarch is from the Department of Computer Science at the University of Maryland, College Park. THIS IS EXCELLENT!!! ]
			\end{enumerate}
		\end{enumerate}
	\item Purdue University: \vspace{-0.2cm}
		\begin{enumerate} \itemsep -2pt
		\item Douglas E. Comer, {\it A few essays about Computer Science}, Department of Computer Science, Purdue University: \vspace{-0.1cm}
			\begin{enumerate} \itemsep -1pt
			\item \url{http://www.cs.purdue.edu/homes/dec/}
			\item Look for the section, ``A few essays about Computer Science''. The essay, ``How to generate a CS research topic,'' is funny: \url{http://www.cs.purdue.edu/homes/dec/essay.topic.generator.html}.
			\item Douglas E. Comer, ``Notes On The PhD Degree,'' Department of Computer Science, Purdue University. Available at: \url{http://www.cs.purdue.edu/homes/dec/essay.phd.html}; last accessed on September 12, 2010.
			\end{enumerate}
		\item Jan Vitek, {\it Home Page: Miscellaneous}, Department of Computer Science, Purdue University. Available online at: \url{http://www.cs.purdue.edu/homes/jv/}; last accessed on September 28, 2010.
		\end{enumerate}
	\item Cornell University: \vspace{-0.2cm}
		\begin{enumerate} \itemsep -2pt
		\item CU-ADVANCE Center (research and resource center concerning diversity and gender equity): \url{http://advance.cornell.edu/}
		\item Department of Computer Science, Faculty of Computing and Information Science (CIS): \vspace{-0.1cm}
			\begin{enumerate} \itemsep -1pt
			\item Charles F. Van Loan: \url{http://www.cs.cornell.edu/cv/default.htm}
			\end{enumerate}
		\end{enumerate}
	\item Rice University: \vspace{-0.2cm}
		\begin{enumerate} \itemsep -2pt
		\item Richard G. Baraniuk, ``Seven Steps to Success in Graduate School (and Beyond),'' Department of Electrical and Computer Engineering, George R. Brown School of Engineering, Rice University. Available online at: \url{http://www.ece.rice.edu/~richb/success.html}; last accessed on January 9, 2010.
		\end{enumerate}
	\item Yale University: \vspace{-0.2cm}
		\begin{enumerate} \itemsep -2pt
		\item Stephen C. Stearns, ``Some Modest Advice for Graduate Students,'' Department of Ecology and Evolutionary Biology, Yale University. Available at: \url{http://www.yale.edu/eeb/stearns/advice.htm}; last accessed on August 28, 2010.
		\item Stephen C. Stearns, ``Designs for Learning,'' Department of Ecology and Evolutionary Biology, Yale University. Available at: \url{http://www.yale.edu/eeb/stearns/designs.htm}; last accessed on August 28, 2010.
		\end{enumerate}
	\item Harvard University: \vspace{-0.2cm}
		\begin{enumerate} \itemsep -2pt
		\item H. T. Kung, ``Useful Things to Know About Ph. D. Thesis Research,'' Harvard School of Engineering and Applied Sciences, Harvard University. (Prepared for ``What is Research'' Immigration Course, Computer Science Department, Carnegie Mellon University, 14 October 1987)
		\item Susan Athey, ``Advice for Applying to Grad School in Economics,'' Department of Economics, Harvard University. Available online at: \url{http://kuznets.fas.harvard.edu/~athey/gradadv.html}; last accessed on January 9, 2010. Prof. Athey has also provided an article, ``Negotiating Senior Job Offers,'' on her web page; this would concern senior faculty job offers.
		\item The Collaborative on Academic Careers in Higher Education, Harvard University Graduate School of Education: \url{http://isites.harvard.edu/icb/icb.do?keyword=coache&tabgroupid=icb.tabgroup104863}
		\end{enumerate}
	\item University of Wisconsin-Madison: \vspace{-0.2cm}
		\begin{enumerate} \itemsep -2pt
		\item Department of Electrical and Computer Engineering; College of Engineering: \vspace{-0.1cm}
			\begin{enumerate} \itemsep -1pt
			\item Azadeh Davoodi: \url{http://homepages.cae.wisc.edu/~adavoodi/Links.htm}
			\end{enumerate}
		\item Dorothea Salo, {\it A Tale of Graduate School Burnout}. Available at: \url{http://members.terracom.net/~dorothea/gradsch/index.html}; last accessed on August 28, 2010.
		\item Dorothea Salo, {\it Straight Talk about Graduate School}. Available at: \url{http://members.terracom.net/~dorothea/gradsch/straighttalk.html}; last accessed on August 28, 2010.
		\item Dorothea Salo, {\it What to do before applying to graduate school}. Available at: \url{http://members.terracom.net/~dorothea/gradsch/success.html}; last accessed on August 28, 2010.
		\end{enumerate}
	\item Pennsylvania State University: \vspace{-0.2cm}
		\begin{enumerate} \itemsep -2pt
		\item Tao Xie and Yuan Xie, {\it Advice Collection}, Department of Computer Science at North Carolina State University, and Department of Computer Science and Engineering at Pennsylvania State University. Available at: \url{http://www.cse.psu.edu/~yuanxie/advice.htm}; last accessed on August 25, 2010. Also, see \url{http://people.engr.ncsu.edu/txie/advice/index.html} and \url{http://people.engr.ncsu.edu/txie/advice.htm}.
		\item Office of Engineering Diversity; Penn State College of Engineering: \vspace{-0.1cm}
			\begin{enumerate} \itemsep -1pt
			\item Office of Engineering Diversity, ``Tips for Graduate Students,'' Penn State College of Engineering, Pennsylvania State University, 2009. Available online at: \url{http://www.engr.psu.edu/mep/tips.html}; last accessed on December 9, 2010.: \vspace{-0.1cm}
				\begin{itemize} \itemsep -1pt
				\item Getting the most out of the relationship with your research advisor or boss: \vspace{-0.1cm}
					\begin{itemize} \itemsep -1pt
					\item {\bf Meet regularly} - you should insist on meeting once a week or at least every other week because it gives you motivation to make regular progress and it keeps your advisor aware of your work.
					\item {\bf Prepare for your meetings} - come to each meeting with: List of topics to discuss; Plan for what you hope to get out of the meeting; Summary of you have done since your last meeting; List of any upcoming deadlines; \& Notes from your previous meeting
					\item {\bf Email him/her a brief summary of EVERY meeting} - this helps avoid misunderstandings and provides a great record of your research progress. Include (where applicable): Time and plan for next meeting; New summary of what you think you are doing; To do list for yourself; To do list for your advisor; List of related work to read; List of major topics discussed; List of what you agreed on; \& List of advice that you may not follow
					\item {\bf Show your advisor the results of your work as soon as possible} - this will help your advisor understand your research and identify potential points of conflict early in the process. Summaries of related work. Anything you write about your research. Experimental results.
					\item {\bf Communicate clearly} - if you disagree with your advisor, state your objections or concerns clearly and calmly. If you feel something about your relationship is not working well, discuss it with him or her. Whenever possible, suggest steps they could take to address your concerns.
					\item {\bf Take the initiative} - you do not need to clear every activity with your advisor. He/she has a lot of work to do too. You must be responsible for your own research ideas and progress.
					\end{itemize}
				\item Getting the most out of what you read: \vspace{-0.1cm}
					\begin{itemize} \itemsep -1pt
					\item {\bf Be organized}. Keep an electronic bibliography with notes \& pointers to the paper files. Keep and file all the papers you have read or skimmed.
					\item {\bf Be efficient} - only read what you need to. Start by reading only the conclusion, scanning figures \& tables, and looking at their references. Read the other sections only if the paper seems relevant or you think it may help you get a different perspective. Skip the sections that you already understand (often the background and motivation sections).
					\item {\bf Take notes on every paper you find worth reading} - What problem are they trying to solve? What is their approach? How is it different from other approaches?
					\item {\bf Summarize what you have read on each topic} -- after you have read several papers covering some topic, note the: key problems; various formulations of the problem they are addressing; relationship among the various approaches; and alternative approaches
					\item {\bf Read Ph.D. theses} -- even though they are long they can be very helpful in quickly learning about what has been done is some field. Especially focus on: Background sections; Method sections; and Your advisor's thesis (this will give you an idea for what he/she expects from you).
					\end{itemize}
				\item Making continual progress on your research: \vspace{-0.1cm}
					\begin{itemize} \itemsep -1pt
					\item {\bf Keep a journal of your ideas} - write down everything you are thinking about even if you think it is stupid. It will help you keep track of your progress and keep you from going in circles. Do not plan to share it with anyone, so you can write freely.
					\item {\bf Set some reasonable goals with deadlines}: Identify key tasks that need to be completed; Set a reasonable date for completing them (on the order of weeks or months); Share this with your advisor or enlist your advisors help in creating the goals and deadlines; and Set some deadlines that you must keep (e.g., volunteer to give a student seminar on your research, work toward a conference paper submission deadline, etc.)
					\item {\bf Keep a to do list} - Checking off things on a to do list can feel very rewarding when you are working on a long-term project. List the small tasks that can be done in about an hour. Pick at least one that has to be completed each day.
					\item {\bf Continually update your: Problem statement}, Goals, Approach (or a list of possible approaches); One-minute version of your research (aka the elevator ride summary); and Five-minute version of your research
					\item {\bf Discuss your research with anyone who will listen} - use your fellow students, friends, family, etc. to practice discussing your research on various levels. They may have useful insights or you may find that verbalizing your ideas clarifies them for yourself.
					\item {\bf Write about your work}. Early stage: Write short idea papers and share them with your advisor and colleagues. Intermediate stage: Find workshops and conferences for submitting preliminary results. This can also help you set deadlines. Advanced stage: Target relevant journals.
					\item {\bf Avoid distractions} - it is easy to ignore your research in favor of more structured tasks such as taking classes, teaching classes, organizing student activities, creating web pages like this, etc. Minimize these kinds of activities or commitments.
					\item {\bf Confront your fears and weaknesses}: If you are afraid of public speaking, volunteer to give lots of talks. If you are afraid your ideas are stupid, discuss them with someone. If you are afraid of writing, write something about your research every day. One-minute version of your research (aka the elevator ride summary). Five-minute version of your research. List of advice that you may not follow.
					\item {\bf Balance reading, thinking, writing and hacking} - often research needs to be an iterative process across all of those tasks.
					\item {\bf Finding a thesis topic or formulating a research plan}. Pick something you find interesting - if you work on something solely because your advisor wants you to, it will be difficult to stay motivated ... Pick something your advisor finds interesting - if your advisor doesn't find it interesting he/she is unlikely to devote much time to your research. He/she will be even more motivated to help you if your project is on their critical path (although this has down sides too! )... Pick something the research community will find interesting - if you want to make yourself marketable... Make sure it addresses a real problem. Remember that your topic will evolve as you work on it. Pick something that is narrow enough that it can be done in a reasonable time frame. Have realistic expectations (i.e. Don't expect the Nobel Prize ). Don't worry that you will be stuck in this area for the rest of your career. It is very likely that you will be doing very different research after you graduate.
					\end{itemize}
				\item Characteristics to look for in a good advisor, mentor, boss, or committee member: \vspace{-0.1cm}
					\begin{itemize} \itemsep -1pt
					\item It is unreasonable to expect one person to have all of the qualities you desire. You should choose thesis committee members who are strong in the areas where your advisor is weak.
					\item {\bf Willing to meet with you regularly (about 1 hour every week or every other week). You can trust him/her to}: Give you credit for the work you do; Defend your work when you are not around; Tell you when your work is or is not good enough; Help you graduate in a reasonable time frame; and Look out for you professionally and personally
					\item {\bf Is interested in your topic. Has good personal and communication skills}: You can talk freely and easily about research ideas; Tells you when you are doing something stupid; Patient, Never feels threatened by your capabilities; Helps motivate you and keep you unstuck
					\end{itemize}
				\item {\bf Has good technical skills}. Can provide constructive criticism of papers you write or talks you give. Knows if what you are doing is good enough for a good thesis. Can help you figure out what you are not doing well; Can help you improve your skills; Can suggest related articles to read or people to talk to; Can tell you or help you discover if what you are doing has already been done; Can help you set and obtain reasonable goals. Will be around until you finish. Is well respected in his/her field. Has good connections for the type of job you would want when you graduate; Willing and able to provide financial and computing support.
				\item {\bf Avoiding the research blues}: \vspace{-0.1cm}
					\begin{itemize} \itemsep -1pt
					\item When you meet your goals, reward yourself.
					\item Don't compare yourself to senior researchers who have many more years of work and publications.
					\item Don't be afraid to leave part of your research problem for future work.
					\item Exercise.
					\item Use the student counseling services
					\item Occasionally, do something fun without feeling guilty! 
					\end{itemize}
				\end{itemize}
			\end{enumerate}
		\end{enumerate}
	\item Technion - Israel Institute of Technology: \vspace{-0.2cm}
		\begin{enumerate} \itemsep -2pt
		\item Department of Computer Science: \vspace{-0.1cm}
			\begin{enumerate} \itemsep -1pt
			\item 
			\end{enumerate}
		\end{enumerate}
	\item University of California, Santa Cruz: \vspace{-0.2cm}
		\begin{enumerate} \itemsep -2pt
		\item Department of Computer Science; Jack Baskin School of Engineering: \vspace{-0.1cm}
			\begin{enumerate} \itemsep -1pt
			\item CMPS/CMPE 200 - Research and Teaching in Computer Science and Engineering (and Applied Math and Statistics and Technology and Information Management) (Fall 2010) by Prof. Cormac Flanagan and Prof. Jose Renau (\& Prof. Alexandre Brandwajn???). Available online at: \url{http://www.soe.ucsc.edu/classes/cmps200/Fall10/}; last accessed on December 17, 2010. [ Also, see {\it CMPS 200 - Research and Teaching in Computer Science and Engineering}: \url{http://www.cs.ucsc.edu/courses/course?cmps200} ]
			\end{enumerate}
		\end{enumerate}
	\item The University of North Carolina at Chapel Hill: \vspace{-0.2cm}
		\begin{enumerate} \itemsep -2pt
		\item Ronald T. Azuma, {``So long, and thanks for the Ph.D.!'' a.k.a. ``Everything I wanted to know about C.S. graduate school at the beginning but didn't learn until later.'' The 4th guide in the Hitchhiker's guide trilogy (and if that doesn't make sense, you obviously have not read Douglas Adams}, v. 1.08, Department of Computer Science, The University of North Carolina at Chapel Hill, January 2003. Available at: \url{http://www.cs.unc.edu/~azuma/hitch4.html}; last accessed on September 3, 2010. [ Also, see {\it Guides to surviving Computer Science graduate school} at: \url{http://www.cs.unc.edu/~azuma/azuma_guides.html}. ]
		\end{enumerate}
	\item Johns Hopkins University: \vspace{-0.2cm}
		\begin{enumerate} \itemsep -2pt
		\item Adam Ruben, Department of Biological Chemistry, School of Medicine, Johns Hopkins University. Adam Ruben, ``Surviving Your Stupid, Stupid Decision to Go to Grad School,'' Broadway Books, New York, NY, 2010.
		\end{enumerate}
	\item University of British Columbia: \vspace{-0.2cm}
		\begin{enumerate} \itemsep -2pt
		\item UBC Faculty of Graduate Studies, {\it The Graduate Game Plan}, UBC Faculty of Graduate Studies, University of British Columbia. Available at: \url{http://www.grad.ubc.ca/current-students/gps-graduate-pathways-success/graduate-game-plan}; last accessed on August 28, 2010.
		\item UBC Faculty of Graduate Studies, {\it Resources for Achieving Success}, UBC Faculty of Graduate Studies, University of British Columbia. Available at: \url{http://www.grad.ubc.ca/current-students/gps-graduate-pathways-success/resources-achieving-success}; last accessed on August 28, 2010.
		\item UBC Faculty of Graduate Studies, {\it Research on the Lived Experience of Graduate Students}, UBC Faculty of Graduate Studies, University of British Columbia. Available at: \url{http://www.grad.ubc.ca/current-students/gps-graduate-pathways-success/research-lived-experience-graduate-students}; last accessed on August 28, 2010.
		\item UBC Faculty of Graduate Studies, {\it Resources for Graduate Student Career Development}, UBC Faculty of Graduate Studies, University of British Columbia. Available at: \url{http://www.grad.ubc.ca/current-students/gps-graduate-pathways-success/resources-graduate-student-career-development}; last accessed on August 28, 2010.
		\item UBC Faculty of Graduate Studies, {\it Graduate Guides}, UBC Faculty of Graduate Studies, University of British Columbia. Available at: \url{http://www.grad.ubc.ca/current-students/gps-graduate-pathways-success/graduate-guides}; last accessed on August 28, 2010.
		\item UBC Faculty of Graduate Studies, {\it Present and Publish Your Research}, UBC Faculty of Graduate Studies, University of British Columbia. Available at: \url{http://www.grad.ubc.ca/current-students/gps-graduate-pathways-success/present-publish-your-research}; last accessed on August 28, 2010.
		\end{enumerate}
	\item Oxford University: \vspace{-0.2cm}
		\begin{enumerate} \itemsep -2pt
		\item Computing Laboratory (Computer Science department): \vspace{-0.1cm}
			\begin{enumerate} \itemsep -1pt
			\item Computing Laboratory, ``D.Phil. in Computer Science,'' Oxford University. Available online at: \url{http://www.comlab.ox.ac.uk/admissions/dphil/transfer.html}; last accessed on October 29, 2010.
			\item Marta Kwiatkowska, ``How to apply for a Doctorate in the Computing Laboratory,'' Computing Laboratory, MT 2009. Available online at: \url{http://www.comlab.ox.ac.uk/admissions/dphil/howtoapply2009.pdf}; last accessed on December 17, 2010. \vspace{-0.1cm}
				\begin{itemize} \itemsep -1pt
				\item A B.S. or B.A. gives you general education; a Masters is your license to practice; and a Ph.D. is a license to teach, do research, and examine people for their Ph.D. defense.
				\item A Ph.D. ``is a research degree,'' and an ``apprentice in research''. It is ``awarded for a significant and substantial piece of research,'' and ``examined by experts and defended in a viva.''
				\item Research is about finding out about something, and discovering how to do that. It involves taking responsibility for organizing my time and taking charge of the investigation.
				\item Doing a Ph.D. is exciting as I ``carry out investigations into [the] unknown,'' and it is ``enriching to learn and master new techniques'' while doing so.
				\end{itemize}
			\end{enumerate}
		\end{enumerate}
	\item University of California, Irvine; Donald Bren School of Information and Computer Sciences: \vspace{-0.2cm}
		\begin{enumerate} \itemsep -2pt
		\item {\it UCI on iTunes U}, ``Improving your Grad School Application,'' University of California, Irvine: Donald Bren School of Information and Computer Sciences: Bren School Honors Seminar on November 12, 2008. Available at: \url{http://deimos3.apple.com/WebObjects/Core.woa/Browse/uci.edu.1983660442.01983660444.1975757832?i=2046363870}; last accessed on August 28, 2010. Also, see \url{http://www.oit.uci.edu/itunesu/} for {\it UCI on iTunes U}, and \url{http://www.ics.uci.edu/about/videos/index.php} for {\it Bren School iTunes U} content. [I can access this from the main {iTunes} site as follows: Look at the ``Find Educational Provider'' tab on the left panel (it's in the middle), and select ``Universities andColleges'' $\Longrightarrow$ Select ``UC Irvine'' $\Longrightarrow$ Under the Courses panel, select ``Donald Bren School of Information and Computer Sciences'' $\Longrightarrow$ Under the ``Community Outreach'' panel, select ``Improving your grad school application'' $\Longrightarrow$ watch this video. The video clip is about a panel discussion of professors about how to get into a top-tier graduate program in CS. It talks about things that admission officers look for, how to get strong letters of recommendation.]
		\end{enumerate}
	\item University of California, Davis: \vspace{-0.2cm}
		\begin{enumerate} \itemsep -2pt
		\item Galois Group, Department of Mathematics: \vspace{-0.1cm}
			\begin{enumerate} \itemsep -1pt
			\item University of California, Davis, {\it Useful things to know when starting graduate school... {\small ...as contributed by experienced grad students!}}, Department of Mathematics, University of California, Davis. Available at: \url{http://galois.math.ucdavis.edu/UsefulGradInfo/HelpfulAdvice/WishIdKnown}; last accessed on September 1, 2010.
			\item University of California, Davis, {\it LaTeX Tutorial}, Department of Mathematics, University of California, Davis. Available at: \url{http://galois.math.ucdavis.edu/UsefulGradInfo/GettingStarted/LaTeXTutorial}; last accessed on September 1, 2010. [ Has \LaTeX\ template for research proposal that is required for the Ph.D. qualifying exam. ]
			\item University of California, Davis, {\it Writing Your Doctoral Thesis}, Department of Mathematics, University of California, Davis. Available at: \url{http://galois.math.ucdavis.edu/UsefulGradInfo/HelpfulAdvice/WritingYourThesis}; last accessed on September 1, 2010. [ Has \LaTeX\ template for Ph.D. dissertations. ]
			\item ``If you are a UC Davis Math Grad Student, then you are a member of the Galois Group.''
			\end{enumerate}
		\end{enumerate}
	\item University of Chicago: \vspace{-0.2cm}
		\begin{enumerate} \itemsep -2pt
		\item Pedro F. Felzenszwalb, Department of Computer Science: \vspace{-0.2cm}
			\begin{enumerate} \itemsep -2pt
			\item \url{http://people.cs.uchicago.edu/~pff/}
			\item His paper, ``Digipaper: A Versatile Color Document Image Representation,'' has been cited 24 times in about 11 years since publication (as of September 1, 2010). He was an undergraduate then, and probably did this work as a junior or early in his senior year.
			\item His paper, ``Efficient Matching of Pictorial Structures,'' is probably based on his work done as a senior. As of September 1, 2010, this paper as been cited 222 times in about 10 years. From \url{http://cs.uchicago.edu/}, it states the following in its news section on September 1, 2010. ``Pedro Felzenszwalb receives Longuet-Higgins prize. The 2010 Longuet-Higgins award has been given to Pedro Felzenszwalb and Daniel Huttenlocher, for their paper "Efficient Matching of Pictorial Structures", Conference on Computer Vision and Pattern Recognition 2000. This award goes to a paper from 10 years ago that has made a fundamental impact on computer vision. Congratulations, Pedro!''
			\end{enumerate}
		\end{enumerate}
	\item University of Virginia: \vspace{-0.2cm}
		\begin{enumerate} \itemsep -2pt
		\item David Evans, {\it Advice}, Department of Computer Science, University of Virginia. Available at: \url{http://www.cs.virginia.edu/~evans/advice/}; last accessed on September 2, 2010. Also, see {\color{blue} ``advice for prospective research students''}: \url{http://www.cs.virginia.edu/~evans/advice/prospective.html}.
		\end{enumerate}
	\item University of Maryland, Baltimore County: \vspace{-0.2cm}
		\begin{enumerate} \itemsep -2pt
		\item Marie desJardins, Department of Computer Science and Electrical Engineering: \vspace{-0.1cm}
			\begin{enumerate} \itemsep -1pt
			\item \url{http://www.cs.umbc.edu/~mariedj/}
			\item Has information on ``How to Succeed in Graduate School,'' ``how to organize a workshop,'' and ``Presenting your research: Papers, talks and chats''.
			\item E.g., Marie desJardins, ``How to Succeed in Graduate School,'' Department of Computer Science and Electrical Engineering, University of Maryland, Baltimore County: \url{http://www.cs.umbc.edu/~mariedj/papers/advice-summary.html}
			\end{enumerate}
		\end{enumerate}
	\item University of Maryland, College Park: \vspace{-0.2cm}
		\begin{enumerate} \itemsep -2pt
		\item Dianne Prost O'Leary, {\it Graduate Study in the Computer and Mathematical Sciences: A Survival Guide}, Department of Computer Science, University of Maryland, College Park. Available at: \url{http://www.cs.umd.edu/~oleary/gradstudy/gradstudy.html}; last accessed on August 28, 2010. It is also available at: \url{http://www.cs.umd.edu/~oleary/gradstudy/}. See \url{http://www.cs.umd.edu/~oleary/} for more articles about ``the accessibility of computer science,'' ``8 rules for career success,'' and the disparity in gender ratios in STEM fields.
		\end{enumerate}
	\item University of Utah: \vspace{-0.2cm}
		\begin{enumerate} \itemsep -2pt
		\item Matt Might, {\it The illustrated guide to a Ph.D.}, School of Computing, University of Utah. Available at: \url{http://matt.might.net/articles/phd-school-in-pictures/}; last accessed on September 13, 2010. [
See {\it blog.might.net} for more articles about graduate school: \url{http://matt.might.net/articles/}; e.g., read ``10 easy ways to fail a Ph.D.'' at: \url{http://matt.might.net/articles/ways-to-fail-a-phd/}. His web page, \url{http://matt.might.net/}, has a sample of these articles. THIS IS EXCELLENT!!! ]
		\item Department of Electrical and Computer Engineering: \vspace{-0.1cm}
			\begin{enumerate} \itemsep -1pt
			\item Prof. Cynthia Furse: \vspace{-0.1cm}
				\begin{itemize} \itemsep -1pt
				\item \url{http://www.ece.utah.edu/~cfurse/}
				\item Cynthia Furse, {\it Graduate Student Survival 101}, February 2009. Available online at: \url{http://www.ece.utah.edu/~cfurse/Tutorials/UU%20Thesis/How%20to%20Write%20a%20Thesis.htm}; last accessed on December 10, 2010.
%	http://www.ece.utah.edu/~cfurse/Tutorials/UU%20Thesis/How%20to%20Write%20a%20Thesis.htm
				\item Cynthia Furse, {\it Dr. Furse's OnLine Tutorials}, August 2007. Available online at: \url{http://www.ece.utah.edu/~cfurse/Tutorials/tutorialsUofU.htm}; last accessed on December 10, 2010. [ Has a lot of good resources for teaching/lecturing, academic/technical writing (including thesis writing), advice for grad/Ph.D. students, making presentations and giving talks, writing grants and proposals, entrepreneurship, and resources for job hunting. ]
				\end{itemize}
			\end{enumerate}
		\end{enumerate}
	\item Indiana University: \vspace{-0.2cm}
		\begin{enumerate} \itemsep -2pt
		\item Indiana University, {\it What Every New Grad Student Should Know}, School of Informatics and Computing, Indiana University. Available at: \url{http://www.cs.indiana.edu/docproject/grad.stuff.html}; last accessed on September 1, 2010.
		\item David Chapman (Editor), {\it How to do Research At the MIT AI Lab}, AI Working Paper 316, MIT AI Lab, Massachusetts Institute of Technology, October, 1988. Available at: \url{http://www.cs.indiana.edu/docproject/grad.stuff.html}; last accessed on September 1, 2010.
		\item Marie desJardins, {\it How to Be a Good Graduate Student}, SRI International (formerly Stanford Research Institute), March 1994. Available at: \url{http://www.cs.indiana.edu/docproject/grad.stuff.html}; last accessed on September 1, 2010.
		\end{enumerate}
	\item The University of Arizona: \vspace{-0.2cm}
		\begin{enumerate} \itemsep -2pt
		\item Jonathan Sprinkle, {\it Students: So, you want to be my student}, Department of Electrical and Computer Engineering, The University of Arizona. Available at: \url{http://www2.engr.arizona.edu/~sprinkjm/Main/Students}; last accessed on September 5, 2010. ``Choose 2-3 IEEE or AIAA journal or conference publications from my website that interest you. Do not choose technical reports, or student papers. 
Write a critical review of the papers, including why the work is interesting, but most importantly where you think the work should go next. In this review, you are proving to me that you understand the purpose of research, and most importantly that you understand the technical details of the paper and how they relate to research.''
		\end{enumerate}
	\item Dartmouth College: \vspace{-0.2cm}
		\begin{enumerate} \itemsep -2pt
		\item Mark L. Tomforde, ``I've passed my quals, now what? - A guide for Ph.D. candidates in mathematics at Dartmouth College,'' Department of Mathematics, Dartmouth College, August 5, 2002. Available online at: \url{http://www.math.dartmouth.edu/graduate-students/current/guide/GradGuide.pdf}; last accessed on December 22, 2010. [ Also, available at: \url{http://www.math.dartmouth.edu/graduate-students/current/guide/} ]
		\end{enumerate}
	\item Vienna University of Technology (TU Vienna): \vspace{-0.2cm}
		\begin{enumerate} \itemsep -2pt
		\item Silvia Miksch, {\it Tips: How to Do Research}, Faculty of Informatics, Vienna University of Technology. Available at: \url{http://www.ifs.tuwien.ac.at/~silvia/research-tips/}; last accessed on September 1, 2010. It has plenty of resources about: \vspace{-0.1cm}
			\begin{enumerate} \itemsep -1pt
			\item ``How to Do Research''
			\item ``How to Write a Scientific Paper''
			\item ``How to Design a Poster''
			\item ``Tips on Organizing Conferences, Workshops, and Symposia''
			\item ``How to Review''
			\item ``Digitial Libaries''
			\item ``Tips for Writing Correct English''
			\end{enumerate}
		\end{enumerate}
	\item Tufts University: \vspace{-0.2cm}
		\begin{enumerate} \itemsep -2pt
		\item Norman Ramsey, {\it Resources for Students}, Department of Computer Science, Tufts University. Available at: \url{http://www.cs.tufts.edu/~nr/students/}; last accessed on September 2, 2010.
		\item Norman Ramsey, {\it How to get admitted to a PhD program}, Department of Computer Science, Tufts University. Available at: \url{http://www.cs.tufts.edu/~nr/students/admit.html}; last accessed on September 2, 2010.
		\end{enumerate}
	\item Portland State University: \vspace{-0.2cm}
		\begin{enumerate} \itemsep -2pt
		\item Department of Computer Science; Maseeh College of Engineering and Computer Science: \vspace{-0.1cm}
			\begin{enumerate} \itemsep -1pt
			\item PSU CS 569 (MS Students) and CS 669 (PhD students) - Scholarship Skills (Fall 2010) by Prof. Andrew Black and Tim Sheard. Available online at: \url{http://web.cecs.pdx.edu/~black/ScholarshipSkills/}; last accessed on September 29, 2010.
			\end{enumerate}
		\end{enumerate}
	\item State University of New York at Buffalo: \vspace{-0.2cm}
		\begin{enumerate} \itemsep -2pt
		\item William J. Rapaport, {\it Information for Grad Students in Computer Science \& Engineering at UB}, Department of Computer Science and Engineering, Department of Philosophy, and Center for Cognitive Science, State University of New York at Buffalo, Buffalo, NY. Available at: \url{http://www.cse.buffalo.edu/~rapaport/GRAD/}; last accessed on September 2, 2010.
		\end{enumerate}
	\item The Ohio State University: \vspace{-0.2cm}
		\begin{enumerate} \itemsep -2pt
		\item The Ohio Science and Engineering Alliance: \vspace{-0.1cm}
			\begin{enumerate} \itemsep -1pt
			\item Graduate School: \url{http://www.ohiosea.org/academic/academic_grad.html}
			\end{enumerate}
		\end{enumerate}
	\item Swarthmore College: \vspace{-0.2cm}
		\begin{enumerate} \itemsep -2pt
		\item Department of History: \vspace{-0.1cm}
			\begin{enumerate} \itemsep -1pt
			\item Timothy Burke, ``Should You Go to Graduate School?,'' in his blog {\it Easily Distracted: Culture, Politics, Academia and Other Shiny Objects}, Department of History, Swarthmore College. Available at: \url{http://weblogs.swarthmore.edu/burke/permanent-features-advice-on-academia/features/}; last accessed on September 14, 2010. [ Also, see \url{http://www.swarthmore.edu/SocSci/tburke1/gradschool.html}. ]
			\item Timothy Burke, ``From ABD to the Job Market: Advice for the Grad School Endgame,'' in his blog {\it Easily Distracted: Culture, Politics, Academia and Other Shiny Objects}, Department of History, Swarthmore College. Available at: \url{http://weblogs.swarthmore.edu/burke/permanent-features-advice-on-academia/abd/}; last accessed on September 14, 2010.
			\end{enumerate}
		\end{enumerate}
	\item Georgetown University: \vspace{-0.2cm}
		\begin{enumerate} \itemsep -2pt
		\item Department of Economics: \vspace{-0.1cm}
			\begin{enumerate} \itemsep -1pt
			\item Garance Genicot, ``Applying to Grad School in Economics,'' Department of Economics, Georgetown University. Available online at: \url{http://www9.georgetown.edu/faculty/gg58/GradSchool.html}; last accessed on January 9, 2010.
			\end{enumerate}
		\end{enumerate}
	\item Norwegian University of Science and Technology: \vspace{-0.2cm}
		\begin{enumerate} \itemsep -2pt
		\item ``PhD Studies,'' Department of Computer and Information Science; Faculty of Information Technology, Mathematics and Electrical Engineering. Available online at: \url{http://www.idi.ntnu.no/research/phd.php}; last accessed on September 29, 2010. [ Includes an outline of a Ph.D. research proposal that is required for applications to its Ph.D. program in computer science. ]
		\end{enumerate}
	\item North Carolina State University: \vspace{-0.2cm}
		\begin{enumerate} \itemsep -2pt
		\item Richard M. Felder, ``An Engineering Student Survival Guide,'' Department of Chemical and Biomolecular Engineering, North Carolina State University, 1993. Available at: \url{http://www4.ncsu.edu/unity/lockers/users/f/felder/public/Papers/survivalguide.htm}; last accessed on August 27, 2010.
		\item Matthias F. (Matt) Stallmann, ``What CSC Graduates Should Know,'' Department of Computer Science, North Carolina State University, February 9, 1996. Available online at: \url{http://people.engr.ncsu.edu/mfms/Teaching/what-grads-should-know.html}; last accessed on October 6, 2010.
		\end{enumerate}
	\item Michigan State University: \vspace{-0.2cm}
		\begin{enumerate} \itemsep -2pt
		\item The Graduate School: \vspace{-0.1cm}
			\begin{enumerate} \itemsep -1pt
			\item The Graduate School, {\it ``Setting Expectations and Resolving Conflict'' Program: Developing Communication and Conflict Management Skills to Save Time and Enhance Productivity}, The Graduate School, Michigan State University, July 12, 2010. Available online at: \url{http://grad.msu.edu/conflictresolution/}; last accessed on December 22, 2010.
			\end{enumerate}
		\item Collegiate Employment Research Institute: \vspace{-0.1cm}
			\begin{enumerate} \itemsep -1pt
			\item \url{http://www.ceri.msu.edu/}
			\item Recruiting Trends 2010-2011: \url{http://www.ceri.msu.edu/recruiting-trends-2009-2010/}
			\end{enumerate}
		\end{enumerate}
	\item San Francisco State University: \vspace{-0.2cm}
		\begin{enumerate} \itemsep -2pt
		\item Eric Hsu, Department of Mathematics: \vspace{-0.1cm}
			\begin{enumerate} \itemsep -1pt
			\item Eric Hsu, {\it Math Education Job Search Resources}, Department of Mathematics, San Francisco State University. Available at: \url{http://bfc.sfsu.edu/cgi-bin/hsu.pl?Math_Education_Job_Search_Resources}; last accessed on September 1, 2010. Also, accessible at: \url{http://math.sfsu.edu/hsu/jobs.html}.
			\item Has lots of information on applying for positions in academia.
			\item Has lists of postdoc positions and (junior) faculty openings.
			\end{enumerate}
		\end{enumerate}
	\item Colorado School of Mines: \vspace{-0.2cm}
		\begin{enumerate} \itemsep -2pt
		\item Department of Physics: \vspace{-0.1cm}
			\begin{enumerate} \itemsep -1pt
			\item Robert L. Read, ``How to be a Programmer,'' APPLICATI, 2003. Available online at: \url{http://samizdat.mines.edu/howto/}; last accessed on September 28, 2010. [ Another publisher is FEINHOCHBURG ]
			\end{enumerate}
		\end{enumerate}
	\item New Mexico Institute of Mining and Technology (New Mexico Tech): \vspace{-0.2cm}
		\begin{enumerate} \itemsep -2pt
		\item Brian Borchers, ``Recommendation Letters,'' Department of Mathematics, New Mexico Institute of Mining and Technology. Available at: \url{http://infohost.nmt.edu/~borchers/recletters.html}; last accessed on September 2, 2010.
		\end{enumerate}
	\item University of Minnesota Duluth: \vspace{-0.2cm}
		\begin{enumerate} \itemsep -2pt
		\item University of Minnesota Duluth, {\it Is Graduate School Right For You?}, Career Services, University of Minnesota Duluth. Available at: \url{http://www.d.umn.edu/careers/grad_school/right_for_you.html}; last accessed on September 2, 2010.
		\end{enumerate}
	\item The University of Waikato: \vspace{-0.2cm}
		\begin{enumerate} \itemsep -2pt
		\item Sean Oughton, {\it Graduate School Survival Guide}, Department of Mathematics, The University of Waikato. Available at: \url{http://www.math.waikato.ac.nz/~seano/grad-school-advice.html}; last accessed on September 3, 2010. [ Also, see \url{http://www.math.waikato.ac.nz/~seano/} for Sean's web page. ]
		\end{enumerate}
	\item Institute for Operations Research and the Management Sciences (INFORMS): \vspace{-0.2cm}
		\begin{enumerate} \itemsep -2pt
		\item {\it Career Center} (has some information on funding/fellowships, and academic careers): \url{http://www.informs.org/Build-Your-Career/INFORMS-Student-Union/Career-Center}
		\end{enumerate}
	\item The PhD Project: \vspace{-0.2cm}
		\begin{enumerate} \itemsep -2pt
		\item Resources for Potential/Current Doctoral Students: \vspace{-0.1cm}
			\begin{enumerate} \itemsep -1pt
			\item \url{http://www.phdproject.org/resources.html}
			\item Information about good business schools that offer Ph.D. programs, preparation for the GMAT, and the life in graduate school as a Ph.D. student.
			\item Suggested Reading: \vspace{-0.1cm}
				\begin{itemize} \itemsep -1pt
				\item \url{http://www.phdproject.org/reading.html}
				\item Has information life in graduate school as a Ph.D. student, racial diversity/issues in higher education, job searching in academia, and work-life balance for female Ph.D. students.
				\end{itemize}
			\end{enumerate}
		\end{enumerate}
	\item U.S. Department of Education: \vspace{-0.2cm}
		\begin{enumerate} \itemsep -2pt
		\item National Center for Education Statistics (NCES); Institute of Education Sciences: \vspace{-0.1cm}
			\begin{enumerate} \itemsep -1pt
			\item National Center for Education Statistics, {\it Projections of Education Statistics to 2018} [report], National Center for Education Statistics, Institute of Education Sciences, U.S. Department of Education, September 2009. Available online at: \url{http://nces.ed.gov/programs/projections/projections2018/index.asp}; last accessed on January 7, 2010.
			\end{enumerate}
		\end{enumerate}
	\item ABD Solution Company: \vspace{-0.2cm}
		\begin{enumerate} \itemsep -2pt
		\item Dr. Carter�s Educational Group, L.L.C.: \vspace{-0.1cm}
			\begin{enumerate} \itemsep -1pt
			\item Wendy Y. Carter, {\it TA-DA!}\texttrademark\ {\it Thesis and Dissertation Accomplished}, 2011. Available online at: \url{http://www.tadafinallyfinished.com/index.html}; last accessed on January 8, 2010. \vspace{-0.1cm}
				\begin{itemize} \itemsep -1pt
				\item {\it TA-DA!}\texttrademark\ Links and Resources: \url{http://www.tadafinallyfinished.com/links/index.html}
				\end{itemize}
			\item Educational Research Institute: \vspace{-0.1cm}
				\begin{itemize} \itemsep -1pt
				\item \url{http://www.educationalresearchinstitute.org/}
				\end{itemize}
			\end{enumerate}
		\end{enumerate}
	\item About.com: \vspace{-0.2cm}
		\begin{enumerate} \itemsep -2pt
		\item About.com, {\it Graduate School}. Available at: \url{http://gradschool.about.com/}; last accessed on August 25, 2010.
		\item Timothy Dzurilla, {\it Writing Graduate Application Essay: Tips to Writing a Successful Personal Statement}, About.com, Nov 1, 2007. Available at: \url{http://www.suite101.com/content/writing-graduate-application-essay-a34598}; last accessed on September 1, 2010. [ Help with writing a statement of purpose ]
		\item Naomi Rockler-Gladen, {\it How to Choose a Graduate School: Faculty, Fit, Student Culture, and Other Grad Program Considerations}, About.com, Nov 12, 2007. Available at: \url{http://www.suite101.com/content/how-to-choose-a-graduate-school-a35387}; last accessed on September 1, 2010. [ How to select a graduate program ... EXCELLENT ]
		\item Naomi Rockler-Gladen, {\it How to Choose a Graduate Advisor: Finding a Faculty Member to Direct an MA Thesis or PhD Dissertation}, About.com, Oct 30, 2007. Available at: \url{http://www.suite101.com/content/how-to-choose-a-graduate-advisor-a34468}; last accessed on September 1, 2010. [ How to pick an advisor ... EXCELLENT ]
		\end{enumerate}
	\item UndergradEcon.com, {\it Graduate School Economics}. Available online at: \url{http://www.undergradecon.com/grad_school.html}; last accessed on January 9, 2010.
	\item Sumit Gupta, {\it Articles and Information about Graduate School}. Available at: \url{http://www.4bearsonline.com/collections/grad/index.shtml}; last accessed on August 25, 2010. 
	\item William Stallings: \vspace{-0.2cm}
		\begin{enumerate} \itemsep -2pt
		\item {\it Computer Science Student Resource Site: How-To}: \url{http://www.computersciencestudent.com/SS/SS-howto.html}
		\item {\it Computer Science Student Resource Site: Computer Science Careers}: \url{http://www.computersciencestudent.com/SS/SS-career.html}
		\item {\it Computer Science Student Resource Site}: \url{http://www.computersciencestudent.com/}
		\end{enumerate}
	\item Dario Toncich, {\it Key Factors in Postgraduate Research - A Guide for Students}. Available at: \url{http://www.doctortee.net/KeyFactors.html}; last accessed on September 1, 2010.
	\end{enumerate}
%%%%%%%%%%%%%%%%%%%%%%%%%%%%%%%%%%%%%
%%%%%%%%%%%%%%%%%%%%%%%%%%%%%%%%%%%%%
%%%%%%%%%%%%%%%%%%%%%%%%%%%%%%%%%%%%%
\item Grad school admission advice: \vspace{-0.3cm}
	\begin{enumerate} \itemsep -2pt
	\item University of California, Berkeley: \vspace{-0.2cm}
		\begin{enumerate} \itemsep -2pt
		\item Department of Economics, ``Criteria,'' Department of Economics, University of California, Berkeley. Available at: \url{http://www.econ.berkeley.edu/econ/grad/admit-criteria.shtml}; last accessed on August 28, 2010.
		\end{enumerate}
	\item Harvard University: \vspace{-0.2cm}
		\begin{enumerate} \itemsep -2pt
		\item Susan Athey, ``Advice for Applying to Grad School in Economics,'' Department of Economics, Harvard University. Available at: \url{http://kuznets.fas.harvard.edu/~athey/gradadv.html}; last accessed on August 28, 2010.
		\end{enumerate}
	\item Statementofpurpose.com: \url{http://www.statementofpurpose.com/}
	\end{enumerate}
%%%%%%%%%%%%%%%%%%%%%%%%%%%%%%%%%%%%%
%%%%%%%%%%%%%%%%%%%%%%%%%%%%%%%%%%%%%
%%%%%%%%%%%%%%%%%%%%%%%%%%%%%%%%%%%%%
\item Advice concerning research: \vspace{-0.3cm}
	\begin{enumerate} \itemsep -2pt
	\item University of California, Riverside: \vspace{-0.2cm}
		\begin{enumerate} \itemsep -2pt
		\item John Baez, ``Advice for the Young Scientist,'' Department of Mathematics, University of California, Riverside, March 25, 2007. Available at: \url{http://math.ucr.edu/home/baez/advice.html}; last accessed on August 28, 2010.
		\end{enumerate}
	\item Research Information Network, RIN: \vspace{-0.2cm}
		\begin{enumerate} \itemsep -2pt
		\item Researchers resources: \url{http://www.rin.ac.uk/resources/researcher-resources}
		\item Researcher development and skills: \url{http://www.rin.ac.uk/resources/researcher-development-and-skills}
		\item Publishing: \url{http://www.rin.ac.uk/resources/publishing}
		\item Learned and professional society: \url{http://www.rin.ac.uk/resources/learned-and-professional-society}
		\end{enumerate}
	\end{enumerate}
%%%%%%%%%%%%%%%%%%%%%%%%%%%%%%%%%%%%%
%%%%%%%%%%%%%%%%%%%%%%%%%%%%%%%%%%%%%
%%%%%%%%%%%%%%%%%%%%%%%%%%%%%%%%%%%%%
\item advice about giving presentations: \vspace{-0.3cm}
	\begin{enumerate} \itemsep -2pt
	\item Cornell University, Department of Computer Science, Faculty of Computing and Information Science (CIS): \vspace{-0.2cm}
		\begin{enumerate} \itemsep -2pt
		\item Charles F. Van Loan, ``The Short Talk,'' Department of Computer Science, Cornell University. Available at: \url{http://www.cs.cornell.edu/cv/ShortTalk.htm}; last accessed on August 25, 2010.
		\end{enumerate}
	\item University of Wisconsin-Madison, Computer Sciences Department: \vspace{-0.2cm}
		\begin{enumerate} \itemsep -2pt
		\item Mark D. Hill, ``Oral Presentation Advice,'' Computer Sciences Department, University of Wisconsin-Madison, April 1992, Revised January 1997. Available at: \url{http://pages.cs.wisc.edu/~markhill/conference-talk.html}; last accessed on August 25, 2010. It includes a short summary of a presentation on this topic by Prof. David A. Patterson. David A. Patterson, ``How to Give a Bad Talk,'' Computer Science Division, Department of Electrical Engineering and Computer Sciences, University of California-Berkeley, 1983.
		\end{enumerate}
	\item University of California, Los Angeles: \vspace{-0.2cm}
		\begin{enumerate} \itemsep -2pt
		\item Terence Tao, ``Talks are not the same as papers,'' Department of Mathematics, University of California, Los Angeles. Available at: \url{http://terrytao.wordpress.com/career-advice/talks-are-not-the-same-as-papers/}; last accessed on September 1, 2010.
		\end{enumerate}
	\item North Carolina State University, Department of Chemical and Biomolecular Engineering: \vspace{-0.2cm}
		\begin{enumerate} \itemsep -2pt
		\item Richard M. Felder, ``Tips on Talks,'' Department of Chemical and Biomolecular Engineering, North Carolina State University. Available at: \url{http://www4.ncsu.edu/unity/lockers/users/f/felder/public/Papers/speakingtips.htm}; last accessed on August 28, 2010.
		\end{enumerate}
	\item Random information: \vspace{-0.2cm}
		\begin{enumerate} \itemsep -2pt
		\item The number of presentation slides is approximately the same as the number of minutes allocated for the presentation. Therefore, for a 15 minutes presentation, the speaker shall use about 15 slides for her/his presentation.
		\item 
		\end{enumerate}
	\end{enumerate}
%%%%%%%%%%%%%%%%%%%%%%%%%%%%%%%%%%%%%
%%%%%%%%%%%%%%%%%%%%%%%%%%%%%%%%%%%%%
%%%%%%%%%%%%%%%%%%%%%%%%%%%%%%%%%%%%%
\item Advice on studying: \vspace{-0.3cm}
	\begin{enumerate} \itemsep -2pt
	\item State University of New York at Buffalo, Department of Computer Science and Engineering: \vspace{-0.2cm}
		\begin{enumerate} \itemsep -2pt
		\item William J. Rapaport, ``How to Study: A Brief Guide,'' Department of Computer Science and Engineering, Department of Philosophy, and Center for Cognitive Science, State University of New York at Buffalo, Buffalo, NY. Available at: \url{http://www.cse.buffalo.edu/~rapaport/howtostudy.html}; last accessed on August 25, 2010.
		\end{enumerate}
	\item University of Oregon, Teaching and Learning Center: \vspace{-0.2cm}
		\begin{enumerate} \itemsep -2pt
		\item Ronald C. Blue, ``How to Study,'' Teaching and Learning Center, University of Oregon. Available at: \url{http://tep.uoregon.edu/resources/faqs/outsidehelp/study.html}; last accessed on August 25, 2010.
		\end{enumerate}
	\item North Carolina State University, Department of Chemical and Biomolecular Engineering: \vspace{-0.2cm}
		\begin{enumerate} \itemsep -2pt
		\item Richard M. Felder, ``Handouts for Students,'' Department of Chemical and Biomolecular Engineering, North Carolina State University. Available at: \url{http://www4.ncsu.edu/unity/lockers/users/f/felder/public/Student_handouts.html}; last accessed on August 28, 2010.
		\end{enumerate}
	\item Middle Tennessee State University: \vspace{-0.2cm}
		\begin{enumerate} \itemsep -2pt
		\item Carolyn Hopper, ``The Study Skills Help Page: Learning Strategies for Success,'' Middle Tennessee State University. Available at: \url{http://frank.mtsu.edu/~studskl/}; last accessed on August 25, 2010.
		\end{enumerate}
	\item Joseph Frank Landsberger: \vspace{-0.2cm}
		\begin{enumerate} \itemsep -2pt
		\item Joseph Frank Landsberger, {\it Study Guides and Strategies}. Available at: \url{http://www.studygs.net/}; last accessed on August 25, 2010.
		\end{enumerate}
	\end{enumerate}
%%%%%%%%%%%%%%%%%%%%%%%%%%%%%%%%%%%%%
%%%%%%%%%%%%%%%%%%%%%%%%%%%%%%%%%%%%%
%%%%%%%%%%%%%%%%%%%%%%%%%%%%%%%%%%%%%
\item Advice on test preparation and test taking: \vspace{-0.3cm}
	\begin{enumerate} \itemsep -2pt
	\item North Carolina State University: \vspace{-0.2cm}
		\begin{enumerate} \itemsep -2pt
		\item Richard M. Felder, ``Random Thoughts:  Memo,'' {\it Chemical Engineering Education}, Vol. 33, No. 2, pp. 136--137, 1999. Available at: \url{http://www4.ncsu.edu/unity/lockers/users/f/felder/public/Columns/memo.html}; last accessed on August 28, 2010.
		\item Richard M. Felder and James E. Stice, ``Tips on Test Taking,'' Department of Chemical and Biomolecular Engineering, North Carolina State University, and Deptartment of Chemical Engineering, The University of Texas at Austin. Available at: \url{http://www4.ncsu.edu/unity/lockers/users/f/felder/public/Papers/testtaking.htm}; last accessed on August 28, 2010.
		\item Richard M. Felder and Matthias F. (Matt) Stallmann, ``Tips for Test Takers,'' Department of Computer Science, North Carolina State University, February 17, 2005. Available online at: \url{http://people.engr.ncsu.edu/mfms/Teaching/tips-for-test-takers.html}; last accessed on October 6, 2010.
		\end{enumerate}
	\end{enumerate}
%%%%%%%%%%%%%%%%%%%%%%%%%%%%%%%%%%%%%
%%%%%%%%%%%%%%%%%%%%%%%%%%%%%%%%%%%%%
%%%%%%%%%%%%%%%%%%%%%%%%%%%%%%%%%%%%%
\item Advice for engineering students: \vspace{-0.3cm}
	\begin{enumerate} \itemsep -2pt
	\item University of Maryland, Baltimore County; Department of Computer Science and Electrical Engineering: \vspace{-0.2cm}
		\begin{enumerate} \itemsep -2pt
		\item Alan T. Sherman (Alan Theodore Sherman), {\it How To's and Other Generic Course Documents}, Department of Computer Science and Electrical Engineering, University of Maryland, Baltimore County, September 12, 1995. Available at: \url{http://www.csee.umbc.edu/~sherman/Courses/documents/}; last accessed on August 28, 2010. Also, see \url{http://www.csee.umbc.edu/~sherman/Courses/}.
		\item Alan T. Sherman (Alan Theodore Sherman), {\it Teaching Activites}, Department of Computer Science and Electrical Engineering, University of Maryland, Baltimore County. Available at: \url{http://www.csee.umbc.edu/~sherman/mycourses.html}; last accessed on August 28, 2010.
		\end{enumerate}
	\item North Carolina State University, Department of Chemical and Biomolecular Engineering: \vspace{-0.2cm}
		\begin{enumerate} \itemsep -2pt
		\item Richard M. Felder's column, ``Random Thoughts,'' in the journal, {\it Chemical Engineering Education}. Available at: \url{http://www4.ncsu.edu/unity/lockers/users/f/felder/public/Columns.html}; last accessed on August 28, 2010.
		\item Richard M. Felder, ``An Engineering Student Survival Guide,'' Department of Chemical and Biomolecular Engineering, North Carolina State University, 1993. Available at: \url{http://www4.ncsu.edu/unity/lockers/users/f/felder/public/Papers/survivalguide.htm}; last accessed on August 27, 2010. \vspace{-0.2cm}
			\begin{itemize} \itemsep -2pt
			\item Do not expect people to tell me how to solve certain problems, especially implementation details as a researcher.
			\item Learn to find out for myself what I need to know. That is, determine the scope of things that I need to know, the time frame and deadline(s) in which I should acquire knowledge of those skills and knowledge, and create a plan to acquire those skills and knowledge.
			\item I should learn to be more resourceful, and determine where can I get help. Particularly, resources (e.g., publications and online material), individuals, and networks of people that can provide a significant amount of help to people.
			\item Make a serious effort to solve a problem before I approach others for help. Else, they may get annoyed when I did not bother to learn how to solve problems that are actually very simple. Also, bring my (attempted/considered) solutions to the people who I seek help from. Show them my flow charts, schematics, calculations, algorithms, and heuristics. This would convince them that I have done my homework, and am not asking bane questions.
			\item To help me understand how to apply the skills and knowledge that I am acquiring in ``practical, real-world applications,'' I should look at textbooks (including alternative textbooks/books, such as handbooks and encyclopedias) for such examples. I can also look ahead further in the chapter, book, or books of subsequent classes to see how these skills and knowledge will be applied. Note that information that I skip while reading my textbook, manuals, and guides may actually contain the solution that I am looking for. Think of the questions that I asked Anders Franzen about the SMT-LIB manual during my internship at FBK while working on the {\it MathSAT} project. I did not understand the material in the manual, since I lacked a background in compiler design and formal grammar/languages. So, I had to get he to help me interpret what I was reading. This was like when I was learning about UNIX as a freshman/sophomore. I did not understand what I was reading when I looked at the UNIX manual (``man pages''). Thus, I shall learn how to read technical literature better.
			\item By reading technical and semi-technical magazines and newspapers/newsletters, I can learn about ``practical, real-world applications'' of the things that I am learning about. In addition, by talking to others (students further ahead of me in engineering education and professional engineers), I learn to see how can I apply the things that I am learning about in ``practical, real-world applications''. Furthermore, I can tap into the newsletters and technical magazines of professional organizations, such as ACM and IEEE, to find out about research opportunities/projects where I can apply what I am learning about.
			\item If my lecture slides/notes and textbooks(s) do not have adequate worked-out examples (e.g., only trivial examples) to help me understand mathematical theories and formulas, and engineering concepts, I shall seek other resources. E.g., I can look at lecture slides/notes from equivalent/similar classes that are taught at other universities. In addition, I can look at other textbooks/books on this subject. Note that for advanced topics, such as those covered by advanced graduate classes, I may only be able to find 1 or 2 books on this topic. So, I may not always have the luxury of looking at worked examples from other textbooks; e.g., I could find many textbooks for differential equations and vector analysis, but not for antenna analysis or satisfiability modulo theories.
			\item I shall improve my ability to work out problems on my own if I cannot find adequate examples for that problem or similar problems. In addition, I shall document worked solutions digitally, so that I can refer to them during revision for an exam, my prelims/quals, or when I encounter a similar problem during research and development.
			\item I shall also improve the way I revise/relearn concepts, technologies, skills, and knowledge. Documenting resources and prior solutions to problems would help me relearn things. Such documentation requires proper information management, so that I can reuse the previously acquired knowledge and skills. Remember that using \LaTeX\ on a UNIX-like operating system helps me with information management.
			\item If I do not understand how and why things work, determine if knowledge of that is required for solving problems in my research/class project, or assignment. If not, I can move on and address this when I have more free time (e.g., ``slack'' periods during the calendar year). To find out more about how certain things work, I can look into the references in my lecture slides/notes and textbooks(s), or search/google for references online. Note that learning how and why certain things work may require (advanced) knowledge that is outside the scope of my discipline or research area. Hence, it is important to know when to stop delving into (/investigating/probing) a concept/technique. Remember my problems with understanding Prof. Sanjit A. Seshia�s publications on adaptive eager encodings, in which he used ``polyhedral theory'' (I believe in the context of integer programming and combinatorial optimization) to prove a theorem regarding the satisfiability of UTVPI formulas? Ditto for lambda expressions (and lambda calculus) so that I can understand how syntax is represented for
	a given signature in first-order logic.]
			\item I shall improve my ability to convert descriptions of architectures and techniques into hardware/software implementations. It is easier to grasp the concepts in pseudocode, flowcharts, schematics, figures, and demonstrations than to learn the concepts from text and create abstractions of those concepts on my own. I shall improve my ability to create pseudocode, flowcharts, schematics, figures, and demonstrations from what I have read, especially in journal and conference papers.
			\item I shall improve my ability to perform statistical analysis on my experimental data, and analyze the figures/graphs that I have plotted.
			\item Use a book from the {\it Schaum's Outline} series from {\it McGraw-Hill} to help me learn material from introductory and intermediate classes. ``Even if you can't find a reference with exactly the type of coverage that works best for you, just reading about the same topic in two different places usually clarifies the ideas.'' [Remember how I read about the same topic in different advance engineering math textbooks to learn concepts for my classes in differential equations and vector analysis?]
			\item Working with others allows me to overcome obstacles that I may not be able to overcome on my own. While I may give up on learning certain things or overcome specific problems in individual projects, my teammates may be able to come up with solutions to problems in group projects. In addition, working in a diverse group exposes me to solutions that can be more effective and/or efficient... {\it students routinely teach one another in group work -- and as any professor will tell you, teaching something is probably the most effective way to learn it.}
			\item Try to find groups of three to four people to work on a problem. When I work in pairs, I may not expose myself to a sufficient variety of approaches. Similarly, when I work in larger groups (i.e., $> 4$), some individuals may be left out of the ``active problem-solving process''.
			\item I shall endeavor to outline solution on my own first, without being boggled or encumbered by the implementation details. Subsequently, I can work out the complete solutions with my group. If each individual does this, each group member can learn how to get started in solving problems in the project. That is, let's outline solutions to the problem, before we meet to discuss our considered solutions and develop the complete solution together.
			\item ``For group work to be fully effective, every group member should be able to explain in detail every solution obtained in a work session. Having the group members (particularly the weaker ones) go through these explanations before ending the session is a good way to make sure that the session has achieved its objectives.'' This will mitigate the tendency for the more technically challenged and reserved individuals to accept proposed solutions without understanding those solutions.
			\end{itemize}
		\item Richard M. Felder, ``How to Survive Engineering School,'' Department of Chemical and Biomolecular Engineering, North Carolina State University. Available at: \url{http://www4.ncsu.edu/unity/lockers/users/f/felder/public/Columns/Surviving-School.html}; last accessed on August 28, 2010.
		\end{enumerate}
	\end{enumerate}
%%%%%%%%%%%%%%%%%%%%%%%%%%%%%%%%%%%%%
%%%%%%%%%%%%%%%%%%%%%%%%%%%%%%%%%%%%%
%%%%%%%%%%%%%%%%%%%%%%%%%%%%%%%%%%%%%
\item Advice on teaching: \vspace{-0.3cm}
	\begin{enumerate} \itemsep -2pt
	\item Stanford University: \vspace{-0.2cm}
		\begin{enumerate} \itemsep -2pt
		\item Stanford University, {\it Teaching at Stanford}, Center for Teaching and Learning, Stanford University. Available at: \url{http://ctl.stanford.edu/teaching-at-stanford.html}; last accessed on September 1, 2010.
		\item Stanford University, {\it Handouts and Teaching Tips}, Center for Teaching and Learning, Stanford University. Available at: \url{http://ctl.stanford.edu/teachingta/handouts-and-teaching-tips.html}; last accessed on September 1, 2010.
		\item Stanford University, {\it Speaking of Teaching Newsletters}, Center for Teaching and Learning, Stanford University. Available at: \url{http://ctl.stanford.edu/speaking-of-teaching-newsletters.html}; last accessed on September 1, 2010.
		\end{enumerate}
	\item University of California, Riverside; Department of Mathematics: \vspace{-0.2cm}
		\begin{enumerate} \itemsep -2pt
		\item John Baez, ``How to Teach Stuff,'' Department of Mathematics, University of California, Riverside, January 23, 2006. Available at: \url{http://math.ucr.edu/home/baez/teaching.html}; last accessed on August 28, 2010.
		\end{enumerate}
	\item University of Oregon, Teaching and Learning Center: \vspace{-0.2cm}
		\begin{enumerate} \itemsep -2pt
		\item Teaching Effectiveness Program, {\it Teaching Resources}, Teaching and Learning Center, University of Oregon. Available at: \url{http://tep.uoregon.edu/resources/index.html}; last accessed on August 25, 2010. Also, look the ``Teaching FAQ's'': \url{http://tep.uoregon.edu/resources/faqs/}
		\item Teaching Effectiveness Program, Resources for {\it Teaching with Technology}, Teaching and Learning Center, University of Oregon. Available at: \url{http://tep.uoregon.edu/technology/index.html}; last accessed on August 25, 2010.
		\end{enumerate}
	\item North Carolina State University, Department of Chemical and Biomolecular Engineering: \vspace{-0.2cm}
		\begin{enumerate} \itemsep -2pt
		\item Richard M. Felder, {\it Student-centered Teaching and Learning}, Department of Chemical and Biomolecular Engineering, North Carolina State University. Available at: \url{http://www4.ncsu.edu/unity/lockers/users/f/felder/public/Student-Centered.html}; last accessed on August 28, 2010.
		\item Richard M. Felder, {\it Index of Learning Styles}, Department of Chemical and Biomolecular Engineering, North Carolina State University. Available at: \url{http://www4.ncsu.edu/unity/lockers/users/f/felder/public/ILSpage.html}; last accessed on August 28, 2010.
		\item Richard M. Felder, {\it Learning Styles}, Department of Chemical and Biomolecular Engineering, North Carolina State University. Available at: \url{http://www4.ncsu.edu/unity/lockers/users/f/felder/public/Learning_Styles.html}; last accessed on August 28, 2010.
		\item Richard M. Felder, {\it Richard Felder's Education-related Publications}, Department of Chemical and Biomolecular Engineering, North Carolina State University. Available at: \url{http://www4.ncsu.edu/unity/lockers/users/f/felder/public/Papers/Education_Papers.html}; last accessed on August 28, 2010.
		\end{enumerate}
	\item Joseph Frank Landsberger: \vspace{-0.2cm}
		\begin{enumerate} \itemsep -2pt
		\item Joseph Frank Landsberger, {\it Teaching Guides and Strategies}. Available at: \url{http://www.studygs.net/teaching/}; last accessed on August 25, 2010.
		\end{enumerate}
	\end{enumerate}
%%%%%%%%%%%%%%%%%%%%%%%%%%%%%%%%%%%%%
%%%%%%%%%%%%%%%%%%%%%%%%%%%%%%%%%%%%%
%%%%%%%%%%%%%%%%%%%%%%%%%%%%%%%%%%%%%
\item Resources to improve my English skills: \vspace{-0.3cm}
	\begin{enumerate} \itemsep -2pt
	\item {\it Guide to Online Schools}: \vspace{-0.2cm}
		\begin{enumerate} \itemsep -2pt
		\item {\it Guide to Online Schools} [or {\it GuideToOnlineSchools.com}], {\it Resources to Help Improve Your English Pronunciation}. Available at: \url{http://www.guidetoonlineschools.com/tips-and-tools/english-pronunciation}; last accessed on August 25, 2010.
		\end{enumerate}
	\end{enumerate}
%%%%%%%%%%%%%%%%%%%%%%%%%%%%%%%%%%%%%
%%%%%%%%%%%%%%%%%%%%%%%%%%%%%%%%%%%%%
%%%%%%%%%%%%%%%%%%%%%%%%%%%%%%%%%%%%%
\item time management: \vspace{-0.3cm}
	\begin{enumerate} \itemsep -2pt
	\item {\it Guide to Online Schools}: \vspace{-0.2cm}
		\begin{enumerate} \itemsep -2pt
		\item {\it Guide to Online Schools} [or {\it GuideToOnlineSchools.com}], {\it The Best Compilation of Time Management Resources on the Web}. Available at: \url{http://www.guidetoonlineschools.com/tips-and-tools/time-management}; last accessed on August 25, 2010.
		\end{enumerate}
	\end{enumerate}
%%%%%%%%%%%%%%%%%%%%%%%%%%%%%%%%%%%%%
%%%%%%%%%%%%%%%%%%%%%%%%%%%%%%%%%%%%%
%%%%%%%%%%%%%%%%%%%%%%%%%%%%%%%%%%%%%
\item Math and Science revision: \vspace{-0.3cm}
	\begin{enumerate} \itemsep -2pt
	\item Basic high school math: \vspace{-0.2cm}
		\begin{enumerate} \itemsep -2pt
		\item North Carolina State University, Department of Chemical and Biomolecular Engineering: \vspace{-0.1cm}
			\begin{itemize} \itemsep -1pt
			\item Kenny Felder and Gary Felder, ``Kenny's Math and Physics Help,'' 2009. Available at: \url{http://www4.ncsu.edu/unity/lockers/users/f/felder/public/kenny/home.html}; last accessed on August 28, 2010.
			\item Kenny Felder, ``Selected Other Educational Sites on the Web''. Available at: \url{http://www4.ncsu.edu/unity/lockers/users/f/felder/public/kenny/edulinks.html}; last accessed on August 28, 2010.
			\end{itemize}
		\end{enumerate}
	\end{enumerate}
%%%%%%%%%%%%%%%%%%%%%%%%%%%%%%%%%%%%%
%%%%%%%%%%%%%%%%%%%%%%%%%%%%%%%%%%%%%
%%%%%%%%%%%%%%%%%%%%%%%%%%%%%%%%%%%%%
\item good blogs about graduate school: \vspace{-0.3cm}
	\begin{enumerate} \itemsep -2pt
	\item Marc Eaddy, {\it Marc Eaddy: Confessions of an Ex-PhD Student}. Available at: \url{http://marceaddy.blogspot.com/}; last accessed on August 28, 2010.
	\item The Academic Blog Portal: \url{http://academicblogs.org/wiki/index.php/Main_Page}
	\end{enumerate}
%%%%%%%%%%%%%%%%%%%%%%%%%%%%%%%%%%%%%
%%%%%%%%%%%%%%%%%%%%%%%%%%%%%%%%%%%%%
%%%%%%%%%%%%%%%%%%%%%%%%%%%%%%%%%%%%%
\item fun stuff about grad school: \vspace{-0.3cm}
	\begin{enumerate} \itemsep -2pt
	\item Jorge Cham, {\it Piled Higher and Deeper}. Available at: \url{http://www.phdcomics.com/}; last accessed on August 28, 2010. See the latest ``PHD Comics'' at: \url{http://www.phdcomics.com/comics.php}. This comic strip pokes fun at the [fun, harsh, interesting, and absurd] realities of life in grad school.
	\item Jorge Cham, {\it Academica}. Available online at: \url{http://academia.edu/academica}; last accessed on October 3, 2010.
	\item Jorge Cham, {\it The PhD Forums}. Available online at: \url{http://www.phdcomics.com/proceedings/index.php}; last accessed on October 3, 2010. [ Has useful guidelines for surviving graduate school and is a decent resource for some technical support (e.g., with \LaTeX\ or C++). ]
	\end{enumerate}
%%%%%%%%%%%%%%%%%%%%%%%%%%%%%%%%%%%%%
%%%%%%%%%%%%%%%%%%%%%%%%%%%%%%%%%%%%%
%%%%%%%%%%%%%%%%%%%%%%%%%%%%%%%%%%%%%
\item other information about or related to grad school (and higher education): \vspace{-0.3cm}
	\begin{enumerate} \itemsep -2pt
	\item {\it Eurodoc}: union of grad student associations of each European country; see \url{http://en.wikipedia.org/wiki/EURODOC} and \url{http://www.eurodoc.net/}
	\item {\it European Association for Quality Assurance in Higher Education} (ENQA): union of accreditation board(s) of each European country; see \url{http://en.wikipedia.org/wiki/ENQA} and \url{http://www.enqa.eu/}
	\item {\it Innolyst}: \vspace{-0.2cm}
		\begin{enumerate} \itemsep -2pt
		\item Innolyst, {\it ResearchCrossroads}, Innolyst: \vspace{-0.1cm}
			\begin{enumerate} \itemsep -1pt
			\item Ernest Kuh, UC Berkeley: \url{http://www.researchcrossroads.org/Researchers/830167} or \url{http://www.researchcrossroads.org/index.php?option=com_content&view=article&id=49&Itemid=55&user_id=830167}
			\item US-based researchers who receive US government funding for their research have profiles in {\it ResearchCrossroads}. E.g., I can find the amount of public funding that my professors at USC received, and the organization that funds them. I can also read an abstract of the project that they got funded for.
			\item \url{http://www.researchcrossroads.org/}
			\end{enumerate}
		\item \url{http://www.innolyst.com/}
		\end{enumerate}
	\item A. Lee, C. Dennis, and P. Campbell. Nature's guide for mentors. Nature, 447(7146):791--797, June 14, 2007 \cite{Lee2007}.
	\item American Council of Trustees and Alumni (ACTA): \vspace{-0.2cm}
		\begin{enumerate} \itemsep -2pt
		\item The American Council of Trustees and Alumni (ACTA) is an independent, non-profit organization committed to academic freedom, excellence, and accountability at America�s colleges and universities.
		\item Launched in 1995, we are the only organization that works with alumni, donors, trustees, and education leaders across the United States to support liberal arts education, uphold high academic standards, safeguard the free exchange of ideas on campus, and ensure that the next generation receives a philosophically rich, high-quality college education at an affordable price.
		\item ACTA Publications: \url{https://www.goacta.org/publications/}. [ ACTA publications cover many aspects of issues concerning higher education institutions, and serve to provide standards of academic excellence and strategies for achieving these standards. ]
		\end{enumerate}
		\item GOOD, {\it GOOD Education}: \url{http://www.good.is/series/good-education/}
	\item Lumina Foundation for Education, {\it Publications}. Available at: \url{http://www.luminafoundation.org/publications/}; last accessed on September 4, 2010.
	\item National Center for Academic Transformation: \url{http://thencat.org/}
	\item Graduate Software Engineering 2009 (GSwE2009): \url{http://www.gswe2009.org/}
	\item CollegeMeasures.org (a joint endeavor by American Institutes for Research and Matrix Knowledge Group): \url{http://collegemeasures.org/}
	\item Association of American Colleges and Universities (AAC\&U): \vspace{-0.2cm}
		\begin{enumerate} \itemsep -2pt
		\item \url{http://www.aacu.org/}
		\item DiversityWeb: \vspace{-0.1cm}
			\begin{enumerate} \itemsep -1pt
			\item The DiversityWeb project is housed within the Office of Diversity, Equity and Global Initiatives at the Association of American Colleges and Universities (AAC\&U)
			\item \url{http://www.diversityweb.org/index.cfm}
			\end{enumerate}
		\end{enumerate}
	\item Newsweek Education: \url{http://education.newsweek.com/index.html}
	\item GRE: \cite{Green2000,Stewart2003,Kaplan2008,ETS2007,Wells2005,Lurie2002,Wu2007,Curtis2007}
	\item Council for Higher Education Accreditation: \url{http://www.chea.org/}
	\item Council of Graduate Schools (CGS): \vspace{-0.2cm}
		\begin{enumerate} \itemsep -2pt
		\item \url{http://www.cgsnet.org/}
		\item Ph.D. Completion Project: \url{http://www.phdcompletion.org/}
		\end{enumerate}
	\item Unigo: \vspace{-0.2cm}
		\begin{enumerate} \itemsep -2pt
		\item \url{http://www.unigo.com/}
		\item Offers students' perspectives on various aspects of college life, from admissions and dorm/college life to studying abroad and academics (studying skills and selecting a major)
		\end{enumerate}
	\item NAFSA / Association of International Educators (formerly, National Association of Foreign Student Advisers): \vspace{-0.2cm}
		\begin{enumerate} \itemsep -2pt
		\item \url{http://www.nafsa.org/}
		\item For Students: \vspace{-0.1cm}
			\begin{enumerate} \itemsep -1pt
			\item \url{http://www.nafsa.org/students.sec/}
			\item Resources about seeking financial aid for study abroad programs, absentee ballot procedure
			\end{enumerate}
		\item {\it Connecting Our World}: \url{http://www.connectingourworld.org/}
		\end{enumerate}
	\end{enumerate}
\end{itemize}



%%%%%%%%%%%%%%%%%%%%%%%%%%%%%%%%%%%%%%%%%
%\subsubsection{\hspace{0.1in} Zhiyang's Suggestions for Graduate School Applications}
%\label{zygradschapps}
\input{grad_school_apps_copy}







%%%%%%%%%%%%%%%%%%%%%%%%%%%%%%%%%%%%%%%%%
% Thoughts and Resources for Specific Areas and Topics
%	% This is written by Zhiyang Ong for his management of information and tasks.
%
% It includes information on professional development, including membership of professional organizations and networking societies.





%%%%%%%%%%%%%%%%%%%%%%%%%%%%%%%%%%%%%%%%%%%
\section{Heuristic for Locating Outreach Resources}
\label{heuristiclocateoutreach}

\proc{Find}$(\varphi, \tau)$ is a heuristic for locating resources for outreach activities, which includes finding information about the following: \vspace{-0.3cm}
\begin{enumerate} \itemsep -4pt
\item awards
\item career resources (including material for career guidance)
\item competitions and contests
\item educational material (e.g., suggested activities and curricular) for specific areas, such as marine sciences and electrical/computer engineering
\item fellowships
\item internships
\item scholarships
\item summer camps
\item summer programs (or summer schools); here, summer schools refer to short educational programs that last from days (e.g., a weekend for the ACM SIGDA Design Automation Summer School) to about a month (e.g., Santa Fe Institute's Complex Systems Summer Schools)
\end{enumerate}
\ \\

Its input $\tau$ is the deadline by which this search process must terminate. For example, if I have to apply for internships by next week, I would use the date of a week from now as the deadline $\tau$. In line \ref{find-pt-professional-org}, an example of a professional organization is the Institute of Electrical and Electronics Engineers (IEEE). The term ``good'' that is used in line \ref{find-pt-gd-uni} is an arbitrary measure of quality determined by the reader/user. \\

A reading group (in line \ref{find-pt-reading-grp}) is a small group of (graduate) students, which may possibly include professors and postdocs, that meet regularly (e.g., once/twice a week) to discuss papers that they have read since the previous meeting/discussion. Each individual in the reading group can be assigned a paper to read and present at the next meeting. The aim of a reading group is to improve the coverage of papers in our research area that each member has read. This is important for interdisciplinary research, since grad students working in interdisciplinary research areas have so much ground to cover. \\

Line \ref{find-pt-athletics} uses the term ``athletics department'' to refer to an administrative department at an American college or university that is in charge of managing varsity/NCAA sports teams. An example of a profession-specific networking organization (line \ref{find-pt-netwk-org}) is DVClub. In line \ref{find-pt-domain-specific-www}, a domain-specific web page is {\it SAT Live!}. An example of a corporate research laboratory (line \ref{find-pt-corporate-research-labs}) is ``Cadence Research Laboratories'' (\url{http://www.cadence.com/cadence/cadence_labs/pages/default.aspx}), and an example of a research institute (line \ref{find-pt-research-institute}) is Santa Fe Institute.




\begin{codebox}
\Procname{$\proc{Find}(\varphi, \tau)$}
\zi	\Comment {\it Input $\varphi \gets $ Item to find out about}
\zi	\Comment {\it Input $\tau \gets $ Deadline for the search process}
\zi	\Comment {\it Output $\kappa \gets $ List of resources about $\varphi$}
\zi
\li \While ( [ resources about $\varphi$ are inadequate ] AND [ $\tau$ has not yet passed ] )
	\Do
\li	Find out the professional organizations for the field of $\varphi$	\label{find-pt-professional-org}
\li	\For each professional organization in the field
		\Do
\li		Check if it has information about $\varphi$ in its web pages, publications, or mailing list archive
\li		\If (it has information about $\varphi$)
			\Then
\li			Add that information to $\kappa$
			\End
		\End
\zi
\li	\For each good (college OR university)	\label{find-pt-gd-uni}
		\Do
\li		\If ($\varphi == $ summer programs )
			\Then
\li			Search for summer programs in the web pages of departments \& schools/colleges
\li		\ElseIf ($\varphi == $ summer camps )
			\Then
\li			Search for summer camps in the web pages of departments \& schools/colleges
\li			Search for summer camps in the web pages of administrative/athletics departments	\label{find-pt-athletics}
\li		\ElseNoIf
\li			Search for $\varphi$ in the web pages of the department(s), including its news section/archive
\li			Search for $\varphi$ in the web pages of professors, postdoctoral researchers, \& students
\li			Search for $\varphi$ in the web pages of reading groups		\label{find-pt-reading-grp}
\li			Search for $\varphi$ in the web pages of student organizations
\li			Search for $\varphi$ in the mailing list archive of classes \& the department
\li			Search for $\varphi$ in the mailing list archive of research groups/labs and projects
\li			Search for $\varphi$ in the mailing list archive of reading groups
\li			Search for $\varphi$ in the mailing list archive of student organizations
			\End
\zi
\li		\If (it has information about $\varphi$)
			\Then
\li			Add that information to $\kappa$
			\End
		\End
\zi
\li	Search for $\varphi$ in the mailing list archive of open-source projects
\li	Search for $\varphi$ in the mailing list archive of profession-specific networking organizations	\label{find-pt-netwk-org}
\li	Search for $\varphi$ in the web pages of domain-specific web pages	\label{find-pt-domain-specific-www}
\li	Search for $\varphi$ in the web pages of research scientists in corporate research labs	\label{find-pt-corporate-research-labs}
\li	Search for $\varphi$ in the web pages of research scientists in research institutes	\label{find-pt-research-institute}
\li	\If ( [ mailing list archive OR web page ] has information about $\varphi$)
		\Then
\li		Add that information to $\kappa$
		\End
	\End	
\li \Return $\kappa$
\end{codebox}




%%%%%%%%%%%%%%%%%%%%%%%%%%%%%%%%%%%%%%%%%%%
\section{General Outreach Resources}
\label{generaloutreachresources}

General outreach resources: \vspace{-0.3cm}
\begin{enumerate} \itemsep -4pt
\item volunteering opportunities: \vspace{-0.3cm}
	\begin{enumerate} \itemsep -2pt
	\item Engineers Without Borders: \url{http://www.ewb-international.org/}
	\item Australian Volunteers International: \url{http://www.australianvolunteers.com/}
	\item Youth Challenge Australia: \url{http://www.youthchallenge.com.au/}
	\item Go Volunteer: \url{http://www.govolunteer.com.au/}
	\item Volunteer Search: \url{http://www.volunteersearch.gov.au/}
	\item Conservation Volunteers: \url{http://www.conservationvolunteers.com.au/volunteer}
	\item Volunteering Australia: \url{http://www.volunteeringaustralia.org/html/s01_home/home.asp}
	\item Sponsors for Educational Opportunity (SEO): \vspace{-0.2cm}
		\begin{enumerate} \itemsep -2pt
		\item Philanthropy \& Volunteerism Resources, \url{http://www.seo-usa.org/AlumniResources}
		\item Volunteer Leadership Opportunities: \url{http://www.seo-usa.org/Alumni_Volunteer}
		\end{enumerate}
	\item : \url{}
	\end{enumerate}
\item public health and preventive medicine: \vspace{-0.3cm}
	\begin{enumerate} \itemsep -2pt
	\item U.S. Department of Health \& Human Services: \vspace{-0.2cm}
		\begin{enumerate} \itemsep -2pt
		\item Agency for Healthcare Research and Quality (AHRQ): \vspace{-0.1cm}
			\begin{enumerate} \itemsep -1pt
			\item Prevention \& Care Management: Resources and Materials, \url{http://www.ahrq.gov/clinic/ppipix.htm}
			\end{enumerate}
		\end{enumerate}
	\end{enumerate}
\item career resources: \vspace{-0.3cm}
	\begin{enumerate} \itemsep -2pt
	\item CRAC: The Career Development Organisation: \vspace{-0.2cm}
		\begin{enumerate} \itemsep -2pt
		\item {\it icould}: \vspace{-0.1cm}
			\begin{enumerate} \itemsep -1pt
			\item \url{http://icould.com/about/}
			\item Resource for students, people who are commencing their careers or are making changes in their careers, career counselors, parents, educators, human resource staff, and employers.
			\item icould, {\it Stories by Life Theme}, in icould: Watch Career Stories. Available online at: \url{http://icould.com/watch-career-stories/by-life-theme/}; last accessed on December 25, 2010. [ Has articles briefly describing how people pursued their career goals or their career paths as they went through different experiences in life. This includes people who ``blossomed after school,'' changed careers or became entrepreneurs, had no plans, took risks, encountered turning points, faced adversity, have disabilities, went through financial hardship, or got laid off. It also has stories of people who volunteered, took a gap year, or pursued internships. ]
			\item icould, {\it Stories by Job Type}, in icould: Watch Career Stories. Available online at: \url{http://icould.com/watch-career-stories/by-job-type/}; last accessed on December 25, 2010. [ Includes stories of people in automotive retail, customer services, engineering, education, and many other job types. ]
			\end{enumerate}
		\end{enumerate}
	\item Jobs for the Future: \vspace{-0.2cm}
		\begin{enumerate} \itemsep -2pt
		\item \url{http://www.jff.org/}
		\item Current Projects: \url{http://www.jff.org/projects/current}
		\item Publications: \url{http://www.jff.org/publications}
		\item Policy: \url{http://www.jff.org/policy}
		\item Funders (funding agencies/organizations): \url{http://www.jff.org/funders}
		\item Programs: \url{http://www.jff.org/index.php?select=work}
		\end{enumerate}
	\item SkillsUSA: \vspace{-0.2cm}
		\begin{enumerate} \itemsep -2pt
		\item ``SkillsUSA is a partnership of students, teachers and industry working together to ensure America has a skilled work force. SkillsUSA helps each student excel.''
		\item Educators: \vspace{-0.1cm}
			\begin{enumerate} \itemsep -1pt
			\item \url{http://www.skillsusa.org/educators/index.shtml}
			\item Programs and Curricula: \url{http://www.skillsusa.org/educators/programs.shtml}
			\end{enumerate}
		\item Students: \vspace{-0.1cm}
			\begin{enumerate} \itemsep -1pt
			\item \url{http://www.skillsusa.org/students/index.shtml}
			\item Scholarships \& Financial Aid--SkillsUSA-related Scholarships: \url{http://www.skillsusa.org/students/scholarships.shtml}
			\end{enumerate}
		\item SkillsUSA competitions: \url{http://www.skillsusa.org/compete/index.shtml}
		\end{enumerate}
	\item others: \vspace{-0.2cm}
		\begin{enumerate} \itemsep -2pt
		\item public speaking and leadership: \vspace{-0.1cm}
			\begin{enumerate} \itemsep -1pt
			\item {\it Toastmasters International} is a non-profit educational organization that teaches public speaking and leadership skills through a worldwide network of meeting locations. Available online at: \url{http://www.toastmasters.org/}; last accessed on January 7, 2010.
			\end{enumerate}
		\end{enumerate}
	\end{enumerate}
\end{enumerate}




%%%%%%%%%%%%%%%%%%%%%%%%%%%%%%%%%%%%%%%%%%%
\section{Youth Outreach}
\label{youthoutreach}

Resources for youth outreach: \vspace{-0.3cm}
\begin{enumerate} \itemsep -4pt
%%%%%%%%%%%%%%%%%%%%%%%
\item educational (computer) games: \vspace{-0.3cm}
	\begin{enumerate} \itemsep -2pt
	\item Chevron Corporation: \vspace{-0.2cm}
		\begin{enumerate} \itemsep -2pt
		\item Energyville (about issues concerning energy and the environment): \url{http://www.willyoujoinus.com/energyville/}
		\end{enumerate}
	\item {\it Lego Digital Designer (LDD)}: \vspace{-0.2cm}
		\begin{enumerate} \itemsep -2pt
		\item CAD software for building Lego toys on Windows and Mac OS X platforms
		\item Free software, as in free beer
		\item \url{http://designbyme.lego.com/en-us/Default.aspx} and \url{http://ldd.lego.com/}
		\end{enumerate}
	\item Robocode: \vspace{-0.2cm}
		\begin{enumerate} \itemsep -2pt
		\item \url{http://en.wikipedia.org/wiki/Robocode} and \url{http://robocode.sourceforge.net/}
		\item Learn how to develop computer programs that will control a robot
		\end{enumerate}
	\item {\it Skill-Life}: \vspace{-0.2cm}
		\begin{enumerate} \itemsep -2pt
		\item \url{http://skill-life.com/}
		\item Use online games to teach youth life skills concerning financial literacy, nutrition, and citizenship.
		\end{enumerate}
	\item PowerUp (IBM with TryScience/New York Hall of Science): \vspace{-0.2cm}
		\begin{enumerate} \itemsep -2pt
		\item \url{http://www.powerupthegame.org/}
		\item Computer game to teach youths about energy conservation, global warming, renewable energy, and sustainable engineering
		\end{enumerate}
	\item EnergyNet: \vspace{-0.2cm}
		\begin{enumerate} \itemsep -2pt
		\item \url{http://www.energynet.net/games/}
		\item Computer game to teach youths about energy efficiency, and other topics related to energy
		\end{enumerate}
	\end{enumerate}
%%%%%%%%%%%%%%%%%%%%%%%
\item summer camps: \vspace{-0.3cm}
	\begin{enumerate} \itemsep -2pt
	\item United States Naval Academy: \vspace{-0.2cm}
		\begin{enumerate} \itemsep -2pt
		\item Naval Academy Athletic Association: \vspace{-0.1cm}
			\begin{enumerate} \itemsep -1pt
			\item Sports camps: \url{http://www.navysports.com/camps/navy-camps.html}
			\end{enumerate}
		\end{enumerate}
	\end{enumerate}
%%%%%%%%%%%%%%%%%%%%%%%
\item competitions for youths: \vspace{-0.3cm}
	\begin{enumerate} \itemsep -2pt
	\item International Geography Olympiad (for high school students): \url{http://www.geoolympiad.org/}
	\item International Linguistic Olympiad (for high school students): \url{http://en.wikipedia.org/wiki/International_Linguistics_Olympiad}
	\item International Philosophy Olympiad (for high school students): \url{http://www.philosophy-olympiad.org/}
	\item JA Worldwide: Responsible People Business Competition (for students in North and South America, and Europe), \url{http://www.responsible-business.org/}
	\item The Choral Arts Society of Washington: \vspace{-0.2cm}
		\begin{enumerate} \itemsep -2pt
		\item \url{http://www.choralarts.org/MLK-Celebration-Community-Initiative/Writing-Competition.aspx}
		\item ``As part of our MLK Celebration Community Initiative and in celebration of Black History Month, The Choral Arts Society of Washington hosts an annual writing competition for students in grades K-12.''
		\item ``Each year, students are presented with a different writing prompt and are asked to respond in poetic form.''
		\item ``Students are encouraged to be creative in their writing and to use their knowledge of Martin Luther King, Jr.'s life, the Civil Rights Movement, and current events as inspiration for their writing.''
		\end{enumerate}
	\item Vocal Arts DC (or Vocal Arts Society): \vspace{-0.2cm}
		\begin{enumerate} \itemsep -2pt
		\item Young Artists Competition: \vspace{-0.1cm}
			\begin{enumerate} \itemsep -1pt
			\item \url{http://vocalartsdc.org/youngartists.shtml}
			\item ``Each year, Vocal Arts DC holds a vocal competition open to all singers who are residents of the greater DC area, including Baltimore and Annapolis.''
			\item ``Singers are asked to submit a CD for review along with a sample recital program that the singer is prepared to sing in recital. The CDs will be reviewed in a blind audition and finalist will be selected for live auditions.''
			\item ``Two winners are selected from the finalists and are presented in the Art Song Discovery Series in four different venues across the greater DC area.''
			\end{enumerate}
		\end{enumerate}
	\item The John F. Kennedy Center for the Performing Arts: \vspace{-0.2cm}
		\begin{enumerate} \itemsep -2pt
		\item The National Symphony Orchestra (NSO): \vspace{-0.1cm}
			\begin{enumerate} \itemsep -1pt
			\item Young Soloists' Competition (High School Division; Washington metropolitan area): \url{http://www.kennedy-center.org/nso/nsoed/youngsoloists.cfm#concerts}
			\end{enumerate}
		\end{enumerate}
	\item Center for Interactive Learning and Collaboration (CILC): \vspace{-0.2cm}
		\begin{enumerate} \itemsep -2pt
		\item Kids Creating Community Content KC$^{3}$ International Contest (for students in Middle and High School): \vspace{-0.1cm}
			\begin{enumerate} \itemsep -1pt
			\item \url{http://kc3.cilc.org/} and \url{http://kc3.cilc.org/guidelines.htm}
			\item Make a short film to educate others about the uniqueness of your community, geographical region, natural/agricultural resources, local/national treasures, culture/heritage, or country.
			\end{enumerate}
		\end{enumerate}
	\end{enumerate}
%%%%%%%%%%%%%%%%%%%%%%%
\item educational resources: \vspace{-0.3cm}
	\begin{itemize} \itemsep -2pt
	\item Xcel Energy Foundation: \vspace{-0.2cm}
		\begin{enumerate} \itemsep -2pt
		\item Focus Area Grants: \vspace{-0.1cm}
			\begin{enumerate} \itemsep -1pt
			\item \url{http://www.xcelenergy.com/Minnesota/Company/Community/Xcel%20Energy%20Foundation/Pages/Focus_Area_Grants.aspx}
			\item Scope of eligible funding, and details on the grant application process
			\end{enumerate}
		\item Education Initiatives: \vspace{-0.1cm}
			\begin{enumerate} \itemsep -1pt
			\item \url{http://www.xcelenergy.com/Minnesota/Company/Community/Education%20Initiatives/Pages/Education_Initiatives.aspx}
			\item Energy Safety Calendar Program, K-6: \vspace{-0.1cm}
				\begin{itemize} \itemsep -1pt
				\item \url{http://www.xcelenergy.com/New%20Mexico/Company/Community/Education%20Initiatives/Pages/Energy_Safety_Calendar_ProgramK-6.aspx}
				\item ``The Energy Safety Calendar Program offers K-6 students in our service territory a great opportunity to learn about electricity and natural gas safety.''
				\end{itemize}
			\end{enumerate}
		\item Safety World: \vspace{-0.1cm}
			\begin{enumerate} \itemsep -1pt
			\item \url{http://www.xcelenergy.com/New%20Mexico/Company/Community/Education%20Initiatives/Pages/Safety_World.aspx}
			\item e-SMART kid: \vspace{-0.1cm}
				\begin{itemize} \itemsep -1pt
				\item \url{http://www.e-smartonline.net/xcelenergy/}
				\item Help children and youth learn about ``electricity and natural gas and how to use them safely''
				\end{itemize}
			\end{enumerate}
		\item Energy Classroom: \vspace{-0.1cm}
			\begin{enumerate} \itemsep -1pt
			\item \url{http://www.energyclassroom.com/}
			\item \url{http://www.xcelenergy.com/Minnesota/Company/Community/Pages/Energy_Classroom.aspx}
			\item Educational material for students about energy sources, energy conservation, and environmental protection
			\item For Teachers (educational material and suggested class activities): \url{http://www.energyclassroom.com/index.php?id=34&page=For_Teachers}
			\end{enumerate}
		\item Power Plant Tour Information: \url{http://www.xcelenergy.com/New%20Mexico/Company/About_Energy_and_Rates/Power%20Generation/Pages/Power_Plant_Tour_Information.aspx}
		\end{enumerate}
	\item HowStuffWorks, Inc.: \url{http://www.howstuffworks.com/}
	\item Chevron Corporation: \vspace{-0.2cm}
		\begin{enumerate} \itemsep -2pt
		\item {\it Will you join us}: \vspace{-0.1cm}
			\begin{enumerate} \itemsep -1pt
			\item Energy issues: \url{http://www.willyoujoinus.com/energyissues/}
			\item Tools and resources: \vspace{-0.1cm}
				\begin{itemize} \itemsep -1pt
				\item \url{http://www.willyoujoinus.com/toolsresources/}
				\item Helpful links (includes K-12 educational material): \url{http://www.willyoujoinus.com/toolsresources/helpfullinks/}
				\end{itemize}
			\item MPG Optimizer: \url{http://www.willyoujoinus.com/usingenergywisely/mpgoptimizer/}
			\item Energy generator: \url{http://www.willyoujoinus.com/usingenergywisely/energygenerator/}
			\end{enumerate}
		\end{enumerate}
	\item National Energy Foundation: \vspace{-0.2cm}
		\begin{enumerate} \itemsep -2pt
		\item \url{http://www.nef.org.uk/} and \url{http://www.nef1.org/}
		\item Students: \url{http://www.nef1.org/students.html}
		\item Educators: \url{http://www.nef1.org/educators.html}
		\item Schools: \vspace{-0.1cm}
			\begin{enumerate} \itemsep -1pt
			\item \url{http://www.nef.org.uk/communities/schools/index.html}
			\item Helpful links: \url{http://www.nef.org.uk/communities/schools/energylinks.html}
			\item School Resources: \url{http://www.nef.org.uk/communities/schools/resources/index.html}
			\item {\it LogiCity} is a fun interactive computer game with a difference. It's a game set in a 3D virtual city with five main activities where you are set the task of reducing the carbon footprint of an average resident. See \url{http://www.nef.org.uk/communities/schools/logicity.html}.
			\end{enumerate}
		\item Resources: \url{http://www.nef.org.uk/actonCO2/index.asp}
		\item Igniting Creative Energy - A National Student Challenge: \vspace{-0.1cm}
			\begin{enumerate} \itemsep -1pt
			\item \url{http://www.ignitingcreativeenergy.org/}
			\item Students: \url{http://www.ignitingcreativeenergy.org/students.html}
			\end{enumerate}
		\end{enumerate}
	\item StartSpot Mediaworks: \vspace{-0.2cm}
		\begin{enumerate} \itemsep -2pt
		\item StartSpot Network: \vspace{-0.1cm}
			\begin{enumerate} \itemsep -1pt
			\item HomeworkSpot: \vspace{-0.1cm}
				\begin{itemize} \itemsep -1pt
				\item \url{http://www.homeworkspot.com/}
				\item Science Fair Project Center: \url{http://www.homeworkspot.com/sciencefair/}
				\end{itemize}
			\end{enumerate}
		\end{enumerate}
	\item Super Science Fair Projects: \url{http://www.super-science-fair-projects.com/}
	\item All Science Fair Projects: Science Fair Projects with Complete Instructions, \url{http://www.all-science-fair-projects.com/}
	\item The Science Club: \vspace{-0.2cm}
		\begin{enumerate} \itemsep -2pt
		\item \url{http://scienceclub.org/}
		\item Science Fair Idea Exchange: \url{http://scienceclub.org/scifair.html}
		\end{enumerate}
	\item Oracle Education Foundation: \vspace{-0.2cm}
		\begin{enumerate} \itemsep -2pt
		\item \url{http://www.oraclefoundation.org/}
		\item ThinkQuest: \vspace{-0.1cm}
			\begin{enumerate} \itemsep -1pt
			\item \url{http://www.thinkquest.org/en/}
			\item ThinkQuest International Competition: \url{http://www.thinkquest.org/competition/}
			\item Projects: \url{http://thinkquest.org/en/projects/index.html}
			\item Library: \url{http://thinkquest.org/pls/html/think.library}
			\item Example of a computer game developed by students: Crisis! - The Game, \url{http://library.thinkquest.org/20331/game/}
			\end{enumerate}
		\end{enumerate}
	\item University of Minnesota: \vspace{-0.2cm}
		\begin{enumerate} \itemsep -2pt
		\item Institute on Community Integration; College of Education and Human Development: \vspace{-0.1cm}
			\begin{enumerate} \itemsep -1pt
			\item National Center on Secondary Education and Transition (NCSET): \vspace{-0.1cm}
				\begin{itemize} \itemsep -1pt
				\item \url{http://www.ncset.org/}
				\item NCSET Topics: \url{http://www.ncset.org/topics/default.asp}
				\item Web Sites: \url{http://www.ncset.org/websites/default.asp}
				\item The Youthhood!: \url{http://www.youthhood.org/}
				\end{itemize}
			\end{enumerate}
		\end{enumerate}
	\item Jobs for America's Graduates: \vspace{-0.2cm}
		\begin{enumerate} \itemsep -2pt
		\item \url{http://www.jag.org/}
		\item JAG Model program applications: \vspace{-0.1cm}
			\begin{enumerate} \itemsep -1pt
			\item \url{http://www.jag.org/model.htm}
			\item Programs are available for students in middle school and high school, high school dropouts, high school seniors, students in alternative education programs, and college underclassmen
			\end{enumerate}
		\item JAG Career Corner: \url{http://www.jag.org/jag_career_corner.htm}
		\item Students: \url{http://www.jag.org/students.htm}
		\item Resource library: \url{http://www.jag.org/library.htm}
		\item Performance outcomes: \url{http://www.jag.org/outcomes.htm}
		\item Funding: \url{http://www.jag.org/funding.htm}
		\end{enumerate}
	\item Alliance to Save Energy: \vspace{-0.2cm}
		\begin{enumerate} \itemsep -2pt
		\item Energy Hog campaign: \vspace{-0.1cm}
			\begin{enumerate} \itemsep -1pt
			\item \url{http://www.energyhog.org/}
			\item Adults: \url{http://www.energyhog.org/adult/adults.htm}
			\item Children: \url{http://www.energyhog.org/childrens.htm}
			\end{enumerate}
		\end{enumerate}
	\item Learning First Alliance: \vspace{-0.2cm}
		\begin{enumerate} \itemsep -2pt
		\item \url{http://www.learningfirst.org/}
		\item Issues and publications: \url{http://www.learningfirst.org/issues}
		\item Resources: \url{http://www.learningfirst.org/resources}
		\end{enumerate}
	\item NaMaYa: \url{http://www.namaya.com/}
	\item NIXTY: \url{http://nixty.com/}
	\item K12 Open Ed: \url{http://www.k12opened.com/wiki/index.php/Main_Page}
	\item Learning Is For Everyone: \url{http://www.learningis4everyone.org/}
	\item The Smithsonian Commons Prototype: \url{http://www.si.edu/commons/prototype/}
	\item Futurelab: Resources for educators and parents, \url{http://www.futurelab.org.uk/resources}
	\item Innosight Institute: Resources for education, \url{http://www.innosightinstitute.org/practices/education/}
	\item WGBH Educational Foundation: \url{http://www.wgbh.org/}
	\item Discovery Education: \vspace{-0.2cm}
		\begin{enumerate} \itemsep -2pt
		\item Classroom resources: \url{http://school.discoveryeducation.com/}
		\item Home resources: \url{http://school.discoveryeducation.com/homeworkhelp/homework_help_home.html}
		\end{enumerate}
	\item The Gilder Lehrman Institute of American History: \vspace{-0.2cm}
		\begin{enumerate} \itemsep -2pt
		\item \url{http://www.gilderlehrman.org/}
		\item Resources for teachers and schools: \url{http://www.gilderlehrman.org/teachers/}
		\item Civil War Essay Contest (for students in selected K-12 schools): \url{http://www.gilderlehrman.org/affiliate/civil_war.php}
		\end{enumerate}
	\item The GRAMMY Museum: \vspace{-0.2cm}
		\begin{enumerate} \itemsep -2pt
		\item Teacher curriculum and resources. Available online at: \url{http://www.grammymuseum.org/interior.php?section=education&page=teachercurriculum}; last accessed on November 15, 2010.
		\end{enumerate}
	\item Purdue University: \vspace{-0.2cm}
		\begin{enumerate} \itemsep -2pt
		\item Department of Entomology: \vspace{-0.1cm}
			\begin{enumerate} \itemsep -1pt
			\item Genomics Analogy Model for Educators (G.A.M.E.): \url{http://www.entm.purdue.edu/extensiongenomics/GAME/default.html}
			\end{enumerate}
		\end{enumerate}
	\item Verizon Thinkfinity: \url{http://www.thinkfinity.org/about-us}
	\item Oregon Virtual School District (ORVSD): \vspace{-0.2cm}
		\begin{enumerate} \itemsep -2pt
		\item \url{http://orvsd.org/}
		\item ``Oregon Virtual School District (ORVSD) helps integrate technology into Oregon public school classrooms by giving teachers access to free tech tools and resources online.''
		\item ``The Oregon Virtual School District is a program led by the Oregon Department of Education that, in cooperation with a consortium of virtual learning providers throughout the state, seeks to increase access and availability of online learning and teaching resources free of charge to public school teachers of Oregon. Oregon State University is providing hosting and development resources through a partnership with the OSU Open Source Lab and the OSU Business Solutions Group.''
		\end{enumerate}
	\item The Association of Educational Publishers (AEP): \vspace{-0.2cm}
		\begin{enumerate} \itemsep -2pt
		\item The AEP Awards: \vspace{-0.1cm}
			\begin{enumerate} \itemsep -1pt
			\item \url{http://www.aepweb.org/awards/index.htm}
			\item Look at the winners of previous AEP awards to determine some of the good educational resources that are available
			\end{enumerate}
		\end{enumerate}
	\item Educational Dividends: \vspace{-0.2cm}
		\begin{enumerate} \itemsep -2pt
		\item \url{http://www.educationaldividends.com/}
		\item Teachers: \vspace{-0.1cm}
			\begin{enumerate} \itemsep -1pt
			\item \url{http://www.educationaldividends.com/index.asp?menu=Teachers}
			\item Teaching Tools: \url{http://www.educationaldividends.com/teachers/tools.asp}
			\item Reference Desk: \vspace{-0.1cm}
				\begin{itemize} \itemsep -1pt
				\item \url{http://www.educationaldividends.com/teachers/reference.asp}
				\item Standards Reference Desk (resources for education standards in the US at the national, state, and local levels): \url{http://www.educationaldividends.com/teachers/standards_desk.asp}
				\item How We Learn: Learning Styles, \url{http://www.educationaldividends.com/teachers/learning_styles.asp}
				\item How We Learn: Multiple Intelligences, \url{http://www.educationaldividends.com/teachers/multiple_intelligences.asp}
				\item Statistics Desk (statistical information about education in the US): \url{http://www.educationaldividends.com/teachers/statistics_desk.asp}
				\end{itemize}
			\item Information about the teaching profession: \vspace{-0.1cm}
				\begin{itemize} \itemsep -1pt
				\item \url{http://www.educationaldividends.com/teachers/welcome.asp}
				\item Office of Occupational Statistics and Employment Projections, ``Educational Services,'' in {\it Career Guide to Industries}, 2010-11 Edition, U.S. Bureau of Labor Statistics, U.S. Department of Labor, Washington, DC, December 17, 2009. Available online at: \url{http://stats.bls.gov/oco/cg/cgs034.htm}; last accessed on December 8, 2010. [ Suggested citation: Bureau of Labor Statistics, U.S. Department of Labor, {\it Career Guide to Industries, 2010-11 Edition}, Educational Services , on the Internet at \url{http://www.bls.gov/oco/cg/cgs034.htm} (visited December 07, 2010). ]
				\item Experience Teaching: \url{http://www.educationaldividends.com/teachers/experience.asp}
				\item Continuous Improvement: \url{http://www.educationaldividends.com/teachers/toolkit.asp}
				\end{itemize}
			\end{enumerate}
		\item Personality and Career Tests: \url{http://www.educationaldividends.com/teachers/tests.asp}
		\end{enumerate}
	\item Smithsonian Institution: \vspace{-0.2cm}
		\begin{enumerate} \itemsep -2pt
		\item Educators: \url{http://www.si.edu/Educators}
		\item Smithsonian Institution Traveling Exhibition Service (SITES): \vspace{-0.1cm}
			\begin{enumerate} \itemsep -1pt
			\item For Teachers: \url{http://www.sites.si.edu/education/teachers_res2.htm}
			\end{enumerate}
		\item Smithsonian Folkways Recordings (or simply, Smithsonian Folkways): \vspace{-0.1cm}
			\begin{enumerate} \itemsep -1pt
			\item Tools for Teaching: \url{http://www.folkways.si.edu/tools_for_teaching/introduction.aspx}
			\end{enumerate}
		\item Freer Gallery of Art / Arthur M. Sackler Gallery: \vspace{-0.1cm}
			\begin{enumerate} \itemsep -1pt
			\item Resources for Educators: \url{http://www.asia.si.edu/explore/teacherResources.asp}
			\item Explore + Learn: Browse Online Resources by Area: \vspace{-0.1cm}
				\begin{itemize} \itemsep -1pt
				\item \url{http://www.asia.si.edu/explore/default.asp}
				\item Has resources for art in: \vspace{-0.1cm}
					\begin{itemize} \itemsep -1pt
					\item The Americas
					\item Ancient Egypt
					\item Ancient Near East
					\item Islamic world
					\item China
					\item Japan
					\item Korea
					\item South Asia
					\item Himalayas
					\item Southeast Asian
					\item It also has biblical manuscripts and contemporary art
					\end{itemize}
				\end{itemize}
			\item Online Exhibition Features: \url{http://www.asia.si.edu/exhibitions/online.asp}
			\item Collections: \url{http://www.asia.si.edu/collections/default.asp}
			\end{enumerate}
		\item National Museum of American History: \vspace{-0.1cm}
			\begin{enumerate} \itemsep -1pt
			\item Jerome and Dorothy Lemelson Center for the Study of Invention and Innovation: \vspace{-0.1cm}
				\begin{itemize} \itemsep -1pt
				\item Resources: \vspace{-0.1cm}
					\begin{itemize} \itemsep -1pt
					\item \url{http://invention.smithsonian.org/resources/}
					\item \url{http://invention.smithsonian.org/resources/default_sites_weblinks.aspx}
					\item Invention stories - archives, articles, audio, and video: \url{http://invention.smithsonian.org/resources/default_index.aspx}
					\end{itemize}
				\item Educational Materials: \vspace{-0.1cm}
					\begin{itemize} \itemsep -1pt
					\item \url{http://invention.smithsonian.org/resources/menu_edu_materials.aspx}
					\item Experiments: \url{http://invention.smithsonian.org/resources/menu_edu_materials.aspx?MaterialTypeID=3&MaterialTypeDesc=Experiments}
					\item Educational Materials: \url{http://invention.smithsonian.org/resources/menu_edu_materials_f.aspx?MaterialTypeDesc=Features}
					\end{itemize}
				\item Centerpieces: \vspace{-0.1cm}
					\begin{itemize} \itemsep -1pt
					\item \url{http://invention.smithsonian.org/centerpieces/}
					\item \url{http://invention.smithsonian.org/centerpieces/iap-info.aspx}
					\item Electric guitar: \url{http://invention.smithsonian.org/centerpieces/electricguitar/index.htm}
					\item Innovative Lives: \url{http://invention.smithsonian.org/centerpieces/ilives/}
					\item ``Exploring the History of Women Inventors'' by J.E. Bedi (in {\it Innovative Lives}): \url{http://invention.smithsonian.org/centerpieces/ilives/womeninventors.html}
					\item Whole Cloth: \url{http://invention.smithsonian.org/centerpieces/whole_cloth/index.html}
					\item The Quartz Watch: \url{http://invention.smithsonian.org/centerpieces/quartz/index.html}
					\item Edison Invents!: All about Thomas Edison and his invention, \url{http://invention.smithsonian.org/centerpieces/edison/default.asp}
					\end{itemize}
				\item Modern Inventors Documentation Program (MIND): \url{http://invention.smithsonian.org/resources/mind_resources.aspx}
				\item Invention at Play: \vspace{-0.1cm}
					\begin{itemize} \itemsep -1pt
					\item \url{http://inventionatplay.org/}
					\item Resources: \url{http://inventionatplay.org/resources.html}
					\item Invention Playhouse: \url{http://inventionatplay.org/playhouse_main.html}
					\item Inventors' Stories: \url{http://inventionatplay.org/inventors_main.html}
					\item Does play matter? (using play to help children learn and think): \url{http://inventionatplay.org/matter_main.html}
					\end{itemize}
				\item Spark!Lab: \vspace{-0.1cm}
					\begin{itemize} \itemsep -1pt
					\item \url{http://sparklab.si.edu/}
					\item About Spark!Lab (introduce children to the process of innovation via play and fun activities): \url{http://sparklab.si.edu/spark-about.html}
					\item Activities \& Experiments: \url{http://sparklab.si.edu/spark-experiments.html}
					\item Inventor Profiles: \url{http://sparklab.si.edu/spark-inventors.html}
					\item Resources: \url{http://sparklab.si.edu/spark-resources.html}
					\end{itemize}
				\end{itemize}
			\end{enumerate}
		\end{enumerate}
	\item Economic and Social Research Council (ESRC): \vspace{-0.2cm}
		\begin{enumerate} \itemsep -2pt
		\item {\it Social Science for Schools}; Science in Society Team: \vspace{-0.1cm}
			\begin{enumerate} \itemsep -1pt
			\item \url{http://www.esrcsocietytoday.ac.uk/ESRCInfoCentre/ssfs/}
			\item Social science resources: \url{http://www.esrcsocietytoday.ac.uk/ESRCInfoCentre/ssfs/resources/}
			\item Career guides for different disciplines in social science and economics: \url{http://www.esrcsocietytoday.ac.uk/ESRCInfoCentre/ssfs/careers/}
			\item Related online resources: \url{http://www.esrcsocietytoday.ac.uk/ESRCInfoCentre/ssfs/links/}
			\end{enumerate}
		\end{enumerate}
	\end{itemize}
\item National Council for Accreditation of Teacher Education (NCATE): \vspace{-0.3cm}
	\begin{enumerate} \itemsep -2pt
	\item \url{http://www.ncate.org/}
	\item Has resources about degree programs in education and their accreditation, as well as how to become a teacher
	\item State-specific Recognized Programs by NCATE and Specialized Professional Associations (SPAs): \vspace{-0.2cm}
		\begin{enumerate} \itemsep -2pt
		\item \url{http://www.ncate.org/tabid/165/Default.aspx}
		\item Find out about educational programs in: \vspace{-0.1cm}
			\begin{enumerate} \itemsep -1pt
			\item special education
			\item early childhood education
			\item educational leadership
			\item educational technology specialist
			\item elementary education
			\item English
			\item health education
			\item foreign languages
			\item gifted education
			\item mathematics
			\item physical education
			\item science education
			\item school psychology
			\item secondary computer science education
			\item social studies
			\item Teachers of English to Speakers of Other Languages (TESOL)
			\item technology and engineering educators
			\end{enumerate}
		\end{enumerate}
	\item Financial Aid Resources for Teacher Education Students: \url{http://www.ncate.org/Public/CurrentFutureTeachers/FinancialAidResources/tabid/351/Default.aspx}
	\end{enumerate}
%%%%%%%%%%%%%%%%%%%%%%%
\item scholarships: \vspace{-0.3cm}
	\begin{enumerate} \itemsep -2pt
	\item U.S. Department of State: \vspace{-0.2cm}
		\begin{enumerate} \itemsep -2pt
		\item Bureau of Educational and Cultural Affairs: \vspace{-0.1cm}
			\begin{enumerate} \itemsep -1pt
			\item National Security Language Initiative for Youth (NSLI-Y): \vspace{-0.1cm}
				\begin{itemize} \itemsep -1pt
				\item \url{http://exchanges.state.gov/youth/programs/nsli.html}
				\item ``The State Department�s National Security Language Initiative for Youth (NSLI-Y) provides merit-based scholarships to U.S. high school students and recent graduates interested in learning less-commonly studied foreign languages.''
				\end{itemize}
			\end{enumerate}
		\end{enumerate}
	\end{enumerate}
%%%%%%%%%%%%%%%%%%%%%%%
\item underrepresented minorities: \vspace{-0.3cm}
	\begin{enumerate} \itemsep -2pt
	\item The University of North Carolina at Chapel Hill: \vspace{-0.2cm}
		\begin{enumerate} \itemsep -2pt
		\item Gary Bishop, {\it Research}, Department of Computer Science, The University of North Carolina at Chapel Hill. Available at: \url{http://wwwx.cs.unc.edu/~gb/wp/research/}; last accessed on September 3, 2010. [ Has plenty of information and resources (including learning aids and material) to help people who are visually or mobility impaired learn. ]
		\end{enumerate}
	\item Myra Sadker Foundation: \vspace{-0.2cm}
		\begin{enumerate} \itemsep -2pt
		\item $100+$ Ideas to Promote Gender Equity in Schools and Beyond: \url{http://www.sadker.org/100ideas.html}
		\item Gender Equity Activities: \url{http://www.sadker.org/WhatYouCanDo.html}
		\item Gender Equity Activities for Concerned Citizens: \url{http://www.sadker.org/GenderEquity-citizens.html}
		\item Gender Equity Activities for Families: \url{http://www.sadker.org/GenderEquity-family.html}
		\item Gender Equity Activities for Teachers: \vspace{-0.1cm}
			\begin{enumerate} \itemsep -1pt
			\item Early Childhood: \url{http://www.sadker.org/GenderEquity-teacher1.html}
			\item Primary Grades: \url{http://www.sadker.org/GenderEquity-teacher2.html}
			\item Upper Elementary: \url{http://www.sadker.org/GenderEquity-teacher3.html}
			\item Middle and High School: \url{http://www.sadker.org/GenderEquity-teacher4.html}
			\end{enumerate}
		\item Resources for feminism and links to web pages of feminist organizations: \url{http://www.sadker.org/ReadsLinks.html}
		\end{enumerate}
	\item League of United Latin American Citizens (LULAC): \vspace{-0.3cm}
		\begin{enumerate} \itemsep -2pt
		\item LULAC National Educational Service Centers, Inc: \vspace{-0.2cm}
			\begin{enumerate} \itemsep -2pt
			\item \url{http://www.lnesc.org/}
			\item Programs: \vspace{-0.1cm}
				\begin{itemize} \itemsep -1pt
				\item Improving literacy among Latino/Latina youth
				\item Encouraging Latino/Latina youth to pursue careers in science and engineering
				\item Helping Latino/Latina youth acquire leadership skills
				\item Improving college access for Latino/Latina youth by mentoring and summer programs (e.g., Gear-Up, Upward Bound, and the PALMS Initiative)
				\item Helping Latino/Latina families acquire financial success, so that Latino/Latina youth can pursue higher education
				\item Scholarships for Latino/Latina youth
				\item \url{http://lnesc.org/index.asp?Type=B_BASIC&SEC={808B6D04-913C-483F-8A05-5BD44B03ED62}}
				\end{itemize}
			\end{enumerate}
		\end{enumerate}
	\item ASPIRA: \vspace{-0.2cm}
		\begin{enumerate} \itemsep -2pt
		\item ASPIRA Programs for Latino/Latina youth: \url{http://aspira.org/manuals/aspira-programs}
		\end{enumerate}
	\end{enumerate}
%%%%%%%%%%%%%%%%%%%%%%%
\item places to visit: \vspace{-0.3cm}
	\begin{enumerate} \itemsep -2pt
	\item Exploratorium @ The Palace of Fine Arts (San Francisco, CA): \url{http://www.exploratorium.edu/}
	\item Educational Dividends: \vspace{-0.2cm}
		\begin{enumerate} \itemsep -2pt
		\item \url{http://www.educationaldividends.com/}
		\item Suggestions for organizing field trips to explore your interests: \url{http://www.educationaldividends.com/students/student_issues.asp}
		\item Career exploration: \url{http://www.educationaldividends.com/students/career_choices.asp}
		\item Computer skills: \url{http://www.educationaldividends.com/students/technology.asp}
		\item Quizzes to help you find out what is your preferred learning style and to discover more about your personality: \url{http://www.educationaldividends.com/students/learning_quiz.asp}
		\item Resources to help you learn about various topics in science, mathematics, social science, and humanities: \url{http://www.educationaldividends.com/students/resources.asp}
		\end{enumerate}
	\end{enumerate}
%%%%%%%%%%%%%%%%%%%%%%%
\item resources for at-risk youths: \vspace{-0.3cm}
	\begin{enumerate} \itemsep -2pt
	\item At-Risk Youth: \url{http://www.at-risk.org/}
	\item Peace First: \vspace{-0.2cm}
		\begin{enumerate} \itemsep -2pt
		\item \url{http://www.peacefirst.org/site/}
		\item To help youths become ``problem-solvers, rather than witnesses, or victims of their surrounding''
		\item To reduce youth homicide rates
		\item Teach children ``critical conflict resolution skills''
		\item Help teachers improve their ``conflict resolution and classroom management skills''
		\item To encourage youths to help each other, and get them to break up fights
		\item ``The Peace First curriculum is tailored to meet the developmental needs of students in Pre-K through eighth grade. Once a week, young adult volunteers and classroom teachers work together to teach students about friendship, communication, and conflict resolution through the use of experiential activities. First graders learn about communicating their feelings, third graders work on being peacemakers in their classroom, and fifth graders explore how to resolve and deescalate conflicts.''
		\item Has programs for students/youths, teachers, principals, and volunteers.
		\end{enumerate}
	\item Americans for the Arts: \vspace{-0.2cm}
		\begin{enumerate} \itemsep -2pt
		\item YouthARTS: \vspace{-0.1cm}
			\begin{enumerate} \itemsep -1pt
			\item \url{http://www.artsusa.org/youtharts/index.asp}
			\item ``The YouthARTS site is designed to give arts agencies, juvenile justice agencies, social service organizations, and other community-based organizations detailed information about how to plan, run, provide training, and evaluate arts programs for at-risk youth.''
			\end{enumerate}
		\end{enumerate}
	\end{enumerate}
%%%%%%%%%%%%%%%%%%%%%%%
\item general music and arts education: \vspace{-0.3cm}
	\begin{enumerate} \itemsep -2pt
	\item Americans for the Arts: \vspace{-0.2cm}
		\begin{enumerate} \itemsep -2pt
		\item Americans for the Arts, ``Ten Simple Ways Parents Can Get More Art in Their Kids' Lives.'' Available online at: \url{http://www.americansforthearts.org/public_awareness/get_involved/001.asp}; last accessed on November 30, 2010.
		\item YouthARTS: \vspace{-0.1cm}
			\begin{enumerate} \itemsep -1pt
			\item \url{http://www.artsusa.org/youtharts/index.asp}
			\item ``The YouthARTS site is designed to give arts agencies, juvenile justice agencies, social service organizations, and other community-based organizations detailed information about how to plan, run, provide training, and evaluate arts programs for at-risk youth.''
			\end{enumerate}
		\end{enumerate}
	\item The John F. Kennedy Center for the Performing Arts: \vspace{-0.2cm}
		\begin{enumerate} \itemsep -2pt
		\item Kennedy Center Institute for Arts Management: \url{http://artsmanagerfba.artsmanager.org/common/Pages/About.aspx}
		\item {\sc ArtsEdge}: \vspace{-0.1cm}
			\begin{enumerate} \itemsep -1pt
			\item The National Standards for Arts Education for Grades K-4, 5-8, and 9-12: \url{http://artsedge.kennedy-center.org/educators/standards.aspx}
			\item Tips and guides for educators: \url{http://artsedge.kennedy-center.org/educators/how-to.aspx}
			\item Lesson plans for educators: \url{http://artsedge.kennedy-center.org/educators/lessons.aspx}
			\item Information for parents, guardians, foster parents, baby-sitters, and grandparents: \url{http://artsedge.kennedy-center.org/families.aspx}
			\item Information for students: \url{http://artsedge.kennedy-center.org/students.aspx}
			\item Themes for artistic, cultural, academic, and intellectual exploration: \url{http://artsedge.kennedy-center.org/themes.aspx}
			\item Multimedia: \url{http://artsedge.kennedy-center.org/multimedia.aspx}
			\end{enumerate}
		\end{enumerate}
	\end{enumerate}
\item music education: \vspace{-0.3cm}
	\begin{enumerate} \itemsep -2pt
	\item Washington Performing Arts Society (WPAS): \vspace{-0.2cm}
		\begin{enumerate} \itemsep -2pt
		\item WPAS Education \& Community -- Connections through the Arts Education Programs for All Ages: \vspace{-0.1cm}
			\begin{enumerate} \itemsep -1pt
			\item The Capitol Jazz Project: \vspace{-0.1cm}
				\begin{itemize} \itemsep -1pt
				\item \url{http://www.wpas.org/educcomm/programsforyoungpeople/capitoljazzproject.aspx}
				\item ``Washington Performing Arts Society (WPAS) and the D.C. Public Schools, in collaboration with Jazz at Lincoln Center, has launched The Capitol Jazz Project, an important step in supporting music education for all students in the District of Colombia.''
				\item ``Through the Capitol Jazz Project, students hone their listening, performing, improvising, composing, arranging, music reading, and notation skills.''
				\item ``The Capitol Jazz Project is being implemented in 6 D.C. middle schools with a total enrollment of more than 500 music students.''
				\item ``A true collaboration, The Capitol Jazz Project brings the combined resources and expertise of WPAS, Jazz at Lincoln Center, and the D.C. Public Schools to create a model music education program.''
				\end{itemize}
			\item Joseph and Goldie Feder Memorial String Competition: \vspace{-0.1cm}
				\begin{itemize} \itemsep -1pt
				\item \url{http://www.wpas.org/educcomm/programsforyoungpeople/josephandgoldiefedermemorialstringcompetition.aspx}
				\item ``The Feder String Competition inspires and nurtures D.C. area youth in grades 6 through 12 who study violin, viola, cello, and double bass.''
				\item ``Each year, 80 students compete for 30 awards and scholarships.''
				\item ``Held each spring, WPAS awards cash prizes toward private lessons, scholarships for summer study programs, and tickets for top winners and their family members to attend a WPAS concert.''
				\item ``Winners of the competition are also given special performance opportunities such as on the Kennedy Center's Millennium Stage and The Shakespeare Theatre Company's Happenings at the Harman series.''
				\end{itemize}
			\item WPAS Summer Performing Arts Academy summer programs: \vspace{-0.1cm}
				\begin{itemize} \itemsep -1pt
				\item \url{http://www.wpas.org/educcomm/programsforyoungpeople/wpassummerperformingartsacademy.aspx}
				\end{itemize}
			\end{enumerate}
		\end{enumerate}
	\item Young Concert Artists, Inc. \vspace{-0.2cm}
		\begin{enumerate} \itemsep -2pt
		\item Annaliese Soros Educational Residency Program: \url{http://www.yca.org/auditions/}
		\end{enumerate}
	\item The Choral Arts Society of Washington: \vspace{-0.2cm}
		\begin{enumerate} \itemsep -2pt
		\item Classroom Resources: \url{http://www.choralarts.org/Education/Classroom-Resources.aspx}
		\end{enumerate}
	\item League of American Orchestras: \vspace{-0.2cm}
		\begin{enumerate} \itemsep -2pt
		\item Career planning: \vspace{-0.1cm}
			\begin{enumerate} \itemsep -1pt
			\item Resources for pre-college students, college students, and graduate students: \url{http://www.americanorchestras.org/career_center/career_planning.html}
			\item Arts Administration programs: \url{http://www.americanorchestras.org/career_center/arts_admin_programs.html}
			\item Non-profit management, {\bf public policy} and leadership programs: \url{http://www.americanorchestras.org/career_center/resources_non_prof_and.html}
			\end{enumerate}
		\end{enumerate}
	\item The John F. Kennedy Center for the Performing Arts: \vspace{-0.2cm}
		\begin{enumerate} \itemsep -2pt
		\item Betty Carter's Jazz Ahead: \vspace{-0.1cm}
			\begin{enumerate} \itemsep -1pt
			\item \url{http://www.kennedy-center.org/programs/jazz/jazzahead/}
			\item ``Music residency program for young people''
			\item ``The Jazz Ahead program identifies outstanding, emerging jazz artists in their mid-teens to age thirty, and brings them together under the tutelage of experienced artist-instructors who coach and counsel them, helping to polish their performance, composing and arranging skills.''
			\item ``The two week-long residency program includes daily workshops and rehearsals with established jazz artists, and culminate in three concerts on the Kennedy Center Millennium Stage, which will be broadcast live over the internet.''
			\end{enumerate}
		\item The National Symphony Orchestra (NSO): \vspace{-0.1cm}
			\begin{enumerate} \itemsep -1pt
			\item The National Symphony Orchestra's Summer Music Institute (SMI): \vspace{-0.1cm}
				\begin{itemize} \itemsep -1pt
				\item \url{http://www.kennedy-center.org/nso/nsoed/smi/home.cfm}
				\item ``Every summer, approximately 70 students (ages 15-20) from all over the nation meet in Washington, D.C., to attend the National Symphony Orchestra's Summer Music Institute (SMI).''
				\item ``The Institute offers four weeks of private lessons, rehearsals, coaching by National Symphony Orchestra members, classes, and lectures to prepare aspiring musicians for their futures in music.''
				\end{itemize}
			\item Young Associates' Program: \vspace{-0.1cm}
				\begin{itemize} \itemsep -1pt
				\item \url{http://www.kennedy-center.org/nso/nsoed/youngassociates.html}
				\item ``The National Symphony Orchestra (NSO) is sponsoring its Young Associates' Program for high school students in grades 11 and 12 in the Washington, DC, metropolitan area who are interested in pursuing a musical career.''
				\item ``Twenty outstanding instrumentalists (pianists are not included) will be selected to attend rehearsals of the National Symphony Orchestra and take part in seminars with conductors, artists, NSO musicians, and representatives of the arts management field.''
				\item ``Through this program, the Young Associates will acquire an appreciation of the wide range of skills, knowledge, and abilities--managerial as well as musical--that are required to put together a performance by a major symphony orchestra. Selection process is by application.''
				\item ``The core of the program involves attendance at rehearsals of the National Symphony Orchestra at the Kennedy Center and observation of various guest artists. In addition to attending NSO rehearsals, students participate in workshops to explore careers in management, music education, publicity, music library, and other professions that are essential to the life of every successful orchestra.''
				\item ``Students do not play their instruments as part of the program. Students learn through listening, observation, and asking questions of professionals.''
				\end{itemize}
			\end{enumerate}
		\end{enumerate}
	\end{enumerate}
\item dance education: \vspace{-0.3cm}
	\begin{enumerate} \itemsep -2pt
	\item The Washington Ballet: \vspace{-0.2cm}
		\begin{enumerate} \itemsep -2pt
		\item The Washington School of Ballet (TWSB): \vspace{-0.1cm}
			\begin{enumerate} \itemsep -1pt
			\item Summer Intensive program (requires an audition): \url{http://www.washingtonballet.org/the-school/summer-intensive/}
			\end{enumerate}
		\item TWB's EXCEL! scholarship program (for DanceDC students): \vspace{-0.1cm}
			\begin{enumerate} \itemsep -1pt
			\item \url{http://www.washingtonballet.org/community-engagement/default.htm}
			\item \url{http://www.washingtonballet.org/community-engagement/other-programs/}
			\item Also, has need-based scholarships
			\end{enumerate}
		\end{enumerate}
	\item The John F. Kennedy Center for the Performing Arts: \vspace{-0.2cm}
		\begin{enumerate} \itemsep -2pt
		\item Exploring Ballet With Suzanne Farrell: A Three-Week Summer Ballet Intensive for Young Dancers: \vspace{-0.1cm}
			\begin{enumerate} \itemsep -1pt
			\item \url{http://www.kennedy-center.org/education/farrell/}
			\item ``In July and August, students from across the United States and around the world will participate in the eighteenth annual session of the Kennedy Center's ballet training program Exploring Ballet with Suzanne Farrell. The three-week residency for dancers ages 14 to 18 with at least five years of ballet training will be held at the Kennedy Center from August 1 - August 20, 2011.''
			\item ``During the three-week period, students take two ballet technique classes a day, six days a week, with Ms. Farrell. Students also participate in a number of cultural activities to enhance their experience in Washington, D.C., including museum visits, trips to historical landmarks, and attending performances.''
			\end{enumerate}
		\item Dance Theatre of Harlem Residency program: \vspace{-0.1cm}
			\begin{enumerate} \itemsep -1pt
			\item \url{http://www.kennedy-center.org/education/community/programs.html#artistic}
			\item ``Since 1993, the Kennedy Center's Dance Theatre of Harlem Residency program has provided ballet training for male and female students age 8-18 with identified promise in ballet taught by Dance Theatre of Harlem (DTH) instructors or former principal dancers.''
			\item ``Students are selected by audition for a twenty-class series, culminating with a public demonstration and performance on a Kennedy Center main stage.''
			\item ``Classical ballet training is taught in four class levels, from novice to advance.''
			\item ``Students must have at least one year of ballet training to qualify for the program.''
			\end{enumerate}
		\end{enumerate}
	\end{enumerate}
%%%%%%%%%%%%%%%%%%%%%%%
\item JA Worldwide (Junior Achievement): \vspace{-0.3cm}
	\begin{enumerate} \itemsep -2pt
	\item \url{http://www.ja.org/}
	\item Resources for educators: \url{http://www.ja.org/involved/involved_educat.shtml}
	\item Resources for parents: \url{http://www.ja.org/involved/involved_parents.shtml}
	\item Resources for students: \url{http://www.ja.org/involved/involved_students.shtml}
	\end{enumerate}
\item U.S. Department of State: \vspace{-0.3cm}
	\begin{enumerate} \itemsep -2pt
	\item Programs for Americans and non-Americans.
	\item Summer Work Travel - In the summer work travel program: \url{http://exchanges.state.gov/}
	\item Cultural Programs Division: \url{http://exchanges.state.gov/cultural/index.html}
	\item Youth Programs Division: \url{http://exchanges.state.gov/youth/index.html}
	\item EducationUSA: \url{http://educationusa.state.gov/}
	\item International Visitor Leadership Program: \url{http://exchanges.state.gov/ivlp/ivlp.html}
	\item Programs for non-U.S. Citizens: \url{http://exchanges.state.gov/prog-non-us.html}
	\item Programs for U.S. Citizens: \url{http://exchanges.state.gov/prog-us.html}
	\item Resources for Students: \url{http://exchanges.state.gov/student.html}
	\item Bureau of Educational and Cultural Affairs: \vspace{-0.2cm}
		\begin{enumerate} \itemsep -2pt
		\item Future Leaders Exchange (FLEX) Program: \vspace{-0.1cm}
			\begin{enumerate} \itemsep -1pt
			\item \url{http://exchanges.state.gov/youth/programs/flex.html}
			\item ``The Future Leaders Exchange (FLEX) Program gives students (ages 15-17) the chance to live with a host family and attend a U.S. high school for a year.''
			\end{enumerate}
		\item Office of Citizen Exchanges: \vspace{-0.1cm}
			\begin{enumerate} \itemsep -1pt
			\item Youth Programs Division: \vspace{-0.1cm}
				\begin{itemize} \itemsep -1pt
				\item \url{http://exchanges.state.gov/youth/index.html}
				\item Has programs for youths in various parts of the world
				\item ``The Youth Programs Division is committed to empowering the next generation and establishing long-lasting ties between the United States and other countries through exchange programs and institutional partnerships. Programs focus primarily on secondary schools and promote mutual understanding, leadership development, educational transformation and democratic ideals.''
				\end{itemize}
			\item SportsUnited: \vspace{-0.1cm}
				\begin{itemize} \itemsep -1pt
				\item \url{http://exchanges.state.gov/sports/index.html}
				\item SportsUnited is an international sports programming initiative designed to help start a dialogue at the grassroots level with non-elite boys and girls ages 7-17.
				\item The programs aid youth in discovering how success in athletics can be translated into the development of life skills and achievement in the classroom.
				\item Foreign participants are given an opportunity to establish links with U.S. sports professionals and exposure to American life and culture.
				\item Americans learn about foreign cultures and the challenges young people from overseas face today.
				\item The U.S. Department of State has programmed initiatives in: baseball, basketball, football, track and field, soccer, volleyball, wrestling, archery, boxing, swimming, fencing, table tennis, ice skating, weightlifting, water polo and managing sports community centers.
				\item Countries covered by this program are listed on the web page.
				\item Sports Envoy Program: \vspace{-0.1cm}
					\begin{itemize} \itemsep -1pt
					\item \url{http://exchanges.state.gov/sports/envoy1.html}
					\item Working with the national sports leagues and the U.S. Olympic Committee, athletes and coaches in various sports are chosen to serve as envoys or ambassadors of sport in overseas programs that include conducting clinics, visiting schools and speaking to youth.
					\item The American athletes and coaches conduct drills and team building activities, as well as engage the youth in a dialogue on the importance of an education, positive health practices and respect for diversity.
					\end{itemize}
				\item Sports Grant Competition: \vspace{-0.1cm}
					\begin{itemize} \itemsep -1pt
					\item The Bureau of Educational and Cultural Affairs (ECA) has an annual open competition under its International Sports Programming Initiative.
					\item Public and private non-profit organizations, 501(c)(3), may submit proposals to discuss approaches designed to enhance and improve the infrastructure of youth sports programs.
					\item The focus of all programs must be reaching out to non-elite youth ages 7-17 and/or their coaches/administrators.
					\item There are four themes that a proposal can address; Youth Sports Management, Training Sports Coaches, Sport and Disability, and Sport and Health.
					\item The list of eligible countries changes each year.
					\item \url{http://exchanges.state.gov/sports/index/sports-grant-competition.html}
					\end{itemize}
				\item Sports Visitor Program: \vspace{-0.1cm}
					\begin{itemize} \itemsep -1pt
					\item Nominated by our U.S. embassies overseas, selected athletes, managers and coaches are brought to the U.S. for technical sports training, sports management, conflict resolution training and exposure to valuable U.S. sports contacts and then are encouraged to return to conduct in-country clinics for youth with their newly learned skills.
					\item \url{http://exchanges.state.gov/sports/visitors.html}
					\end{itemize}
				\end{itemize}
			\end{enumerate}
		\end{enumerate}
	\end{enumerate}
\item U.S. Department of Labor: \vspace{-0.3cm}
	\begin{enumerate} \itemsep -2pt
	\item Wage and Hour Division: \vspace{-0.2cm}
		\begin{enumerate} \itemsep -2pt
		\item YouthRules!: \vspace{-0.1cm}
			\begin{enumerate} \itemsep -1pt
			\item \url{http://youthrules.dol.gov/}
			\item Has information for youths, parents, educators, and employers on how to let youth work part-time safely
			\item Teens: \url{http://youthrules.dol.gov/teens/default.htm}
			\item Parents: \url{http://youthrules.dol.gov/parents/default.htm}
			\item Educators: \url{http://youthrules.dol.gov/educators/default.htm}
			\item Employers: \url{http://youthrules.dol.gov/employers/default.htm}
			\item Resources: \url{http://youthrules.dol.gov/resources.htm}
			\item Compliance Assistance: \url{http://youthrules.dol.gov/ca.htm}
			\end{enumerate}
		\end{enumerate}
	\end{enumerate}
\item ASCL Educational Services, Inc. (Marc McCulloch): \vspace{-0.3cm}
	\begin{enumerate} \itemsep -2pt
	\item Transitions: Life Skills for Personal Success!: \vspace{-0.2cm}
		\begin{enumerate} \itemsep -2pt
		\item Curriculum \& Materials: \url{http://transitions.ascl.info/infomaterials}
		\item Soft Skills: \url{http://transitions.ascl.info/infoskills}
		\end{enumerate}
	\end{enumerate}
\item Partnership for 21st Century Skills: \vspace{-0.3cm}
	\begin{enumerate} \itemsep -2pt
	\item \url{http://www.p21.org/}
	\item Framework for 21st Century Learning: \url{http://www.p21.org/index.php?option=com_content&task=view&id=254&Itemid=119}
	\item Tools and Resources: \url{http://www.p21.org/index.php?option=com_content&task=view&id=273&Itemid=139}
	\end{enumerate}
\item National Career and Technical Education Foundation (NCTEF): \vspace{-0.3cm}
	\begin{enumerate} \itemsep -2pt
	\item States' Career Clusters Initiative (SCCI): \vspace{-0.2cm}
		\begin{enumerate} \itemsep -2pt
		\item \url{http://www.careerclusters.org/}
		\item The 16 Career Clusters: \url{http://www.careerclusters.org/16clusters.cfm}
		\item Plans of Study: \url{http://www.careerclusters.org/resources/web/pos.cfm}
		\item Knowledge and Skills Charts: \url{http://www.careerclusters.org/resources/web/ks.php}
		\item Crosswalks: \url{http://www.careerclusters.org/crosswalks.cfm}
		\item Publications: \url{http://www.careerclusters.org/publications.php}
		\item Sixteen Career Clusters and Their Pathways: \url{http://www.careerclusters.org/list16clusters.php}
		\item Career Clusters Models: \url{http://www.careerclusters.org/resources/web/16ccall.php?action=models}
		\item Career Clusters Brochure Previews: \url{http://www.careerclusters.org/resources/web/16ccall.php?action=brochures}
		\item Career Clusters Interest Survey: \url{http://www.careerclusters.org/ccinterestsurvey.php}
		\item Related Websites: \url{http://www.careerclusters.org/related.php}
		\end{enumerate}
	\end{enumerate}
\item U. S. Department of Labor: \vspace{-0.3cm}
	\begin{enumerate} \itemsep -2pt
	\item Employment and Training Administration: \vspace{-0.2cm}
		\begin{enumerate} \itemsep -2pt
		\item CareerOneStop: \vspace{-0.1cm}
			\begin{enumerate} \itemsep -1pt
			\item \url{http://www.careeronestop.org/}
			\item Students, parents, and career advisors: \url{http://www.careeronestop.org/studentsandcareeradvisors/studentsandcareeradvisors.aspx}
			\end{enumerate}
		\end{enumerate}
	\end{enumerate}
\item U. S. Department of Defense: \vspace{-0.3cm}
	\begin{enumerate} \itemsep -2pt
	\item ASVAB Career Exploration Program: \vspace{-0.2cm}
		\begin{enumerate} \itemsep -2pt
		\item \url{http://www.asvabprogram.com/}
		\item Learn about yourself: \url{http://www.asvabprogram.com/index.cfm?fuseaction=learn.main}
		\item Explore careers: \url{http://www.asvabprogram.com/index.cfm?fuseaction=explore.main}
		\item Plan for your future: \url{http://www.asvabprogram.com/index.cfm?fuseaction=plan.main}
		\item Information for educators and career counselors: \url{http://www.asvabprogram.com/index.cfm?fuseaction=edu.main}
		\item Information for parents: \url{http://www.asvabprogram.com/index.cfm?fuseaction=parents.main}
		\end{enumerate}
	\end{enumerate}
\end{enumerate}



%%%%%%%%%%%%%%%%%%%%%%%%%%%%%%%%%%%%%%%%%%%
\section{Internship Opportunities}
\label{Internship Opportunities}

Internship opportunities: \vspace{-0.3cm}
\begin{enumerate} \itemsep -4pt
\item Canada: \vspace{-0.3cm}
	\begin{enumerate} \itemsep -2pt
	\item SWAP: \vspace{-0.2cm}
		\begin{enumerate} \itemsep -2pt
		\item \url{http://www.swap.ca/}
		\item For Canadians who want to work abroad: \url{http://www.swap.ca/out_eng/index.aspx}
		\item For citizens of selected countries who want to work in Canada: \url{http://www.swap.ca/in_eng/partner_organizations.aspx}
		\end{enumerate}
	\end{enumerate}
\item Singapore: \vspace{-0.3cm}
	\begin{enumerate} \itemsep -2pt
	\item Speedwing Training (Asia) Pte Ltd: \vspace{-0.2cm}
		\begin{enumerate} \itemsep -2pt
		\item \url{http://www.speedwing.org/}
		\item For Singaporeans who want to work in the United States, Canada, Germany, and New Zealand
		\item For citizens of selected countries who want to work in Singapore
		\end{enumerate}
	\end{enumerate}
\end{enumerate}

%%%%%%%%%%%%%%%%%%%%%%%%%%%%%%%%%%%%%%%%%%%
\subsection{Internship Opportunities in Australia}
\label{internshipaus}

Internship Opportunities in Australia: \vspace{-0.3cm}
\begin{enumerate} \itemsep -4pt
\item The Association of Professional Engineers, Scientists and Managers, Australia: \url{http://www.apesma.asn.au/index.asp} --- Ask for guide to internships in your region/major; free student membership
\item Engineers Australia: \url{http://www.engineersaustralia.org.au/} --- Ask for guide to internships in your region/major; free student membership
\item CPA Australia: \url{http://www.cpaaustralia.com.au/cps/rde/xchg/cpa/hs.xsl/index.html} and \url{http://www.cpaaustralia.com.au/cps/rde/xchg/careers/site/index_ENA_HTML.htm/cps/rde/xchg/SID-3F57FECB-EEFEF50E/careers/site/204_ENA_HTML.htm}
\item Institute of Chartered Accountants in Australia: \url{http://www.charteredaccountants.com.au/}
\item 
\end{enumerate}


%%%%%%%%%%%%%%%%%%%%%%%%%%%%%%%%%%%%%%%%%%%
\subsection{Internship Opportunities in Europe}
\label{internshipeu}

Internship Opportunities in Portugal: \vspace{-0.3cm}
\begin{enumerate} \itemsep -4pt
\item Portugal: \vspace{-0.3cm}
	\begin{enumerate} \itemsep -2pt
	\item IAESTE Portugal (The International Association for the Exchange of Students for Technical Experience): \url{http://www.iaeste.pt/en/foreign-trainees/why-portugal/}
	\end{enumerate}
\item United Kingdom: \vspace{-0.3cm}
	\begin{enumerate} \itemsep -2pt
	\item Graduate Talent Pool: \url{http://graduatetalentpool.direct.gov.uk/}
	\end{enumerate}
\end{enumerate}




%%%%%%%%%%%%%%%%%%%%%%%%%%%%%%%%%%%%%%%%%%%
\subsection{Internship Opportunities in the United States}
\label{internshipsus}

Internship Opportunities in the United States: \vspace{-0.3cm}
\begin{enumerate} \itemsep -4pt
\item Use the Procedure \proc{Find}$(\varphi, \tau)$ in \S\ref{heuristiclocateoutreach} to look up internship opportunities and lists of internship opportunities.

Look at government organizations (e.g., the White House), nonprofit organizations (e.g., Engineers Without Borders), professional organizations (e.g., IEEE and ACM), colleges and universities, and companies (e.g., Intel, Google, and start-ups).

You can start your search by looking at the organizations that provide resources for underrepresented minorities as well as resources for scholarships and fellowships. These information can be found in other sections of this document.

If you do not know where to start, speak to a professor or staff member at the career center of your college/university. Alternatively, you can ask your awesome resident advisors (RAs).

My personal advice is to start your search based on your interests and skill set. You can always narrow the search space based on factors, such as geographical location, later on.

Competitive internships, especially research internships in electrical and computer engineering or computer science, weed out many students from applying via demanding job requirements. For example, if you want to apply for research internships with electronic design automation (EDA) companies and corporate research labs, you would need to have significant experience designing integrated circuits and developing EDA software. The stringent job requirements also mean that students need to plan in advance (say, about a year) about the internships that they would like to seek, and plan to acquire the necessary skill set and experiences before the application deadlines (which can be several months before the start of your internship).

Taking as many challenging classes as you can possibly cope, especially in electrical and computer engineering or computer science, would provide you with a skill set that allows you to apply for competitive internships in many fields. Apart from taking challenging classes as well as engaging in research and/or open source projects, you can try to acquire additional skills and experience in your free time to boost the competitiveness of your internship application. Certain skills and experiences, such as compiler design, are hard to acquire in your free time, so it would be ``easier'' to take classes that would help you acquire those skills and experiences.

Note that you may want to look into creating your own entrepreneurial venture, say an EDA start-up or organization in social entrepreneurship, rather than to seek an internship. Also, seeking an internship abroad is always a good addition to your resume/CV.
\item National Science Foundation: \vspace{-0.3cm}
	\begin{enumerate} \itemsep -2pt
	\item Research Experiences for Undergraduates (REU): \vspace{-0.2cm}
		\begin{enumerate} \itemsep -2pt
		\item \url{http://www.nsf.gov/crssprgm/reu/reu_search.cfm}
		\item Academic fields: \vspace{-0.1cm}
			\begin{enumerate} \itemsep -1pt
			\item Astronomical Sciences
			\item Atmospheric and Geospace Sciences
			\item Biological Sciences
			\item Chemistry
			\item Computer and Information Science and Engineering
			\item Cyberinfrastructure
			\item Department of Defense (DoD)
			\item Earth Sciences
			\item Education and Human Resources
			\item Engineering
			\item Ethics and Values Studies
			\item International Science and Engineering
			\item Materials Research
			\item Mathematical Sciences
			\item Ocean Sciences
			\item Physics
			\item Polar Programs
			\item Social, Behavioral, and Economic Sciences
			\end{enumerate}
		\end{enumerate}
	\end{enumerate}
\item Society for Industrial and Applied Mathematics: \vspace{-0.3cm}
	\begin{enumerate} \itemsep -2pt
	\item Internship and Career Information in Industry, Research Institutions, and Government Labs: \url{http://www.siam.org/careers/internships.php}
	\end{enumerate}
\item American Institute of Physics (AIP): \vspace{-0.3cm}
	\begin{enumerate} \itemsep -2pt
	\item Society of Physics Students (SPS): \vspace{-0.2cm}
		\begin{enumerate} \itemsep -2pt
		\item SPS Internships: \url{http://www.spsnational.org/programs/internships/}
		\item Research Opportunities: \url{http://www.spsnational.org/programs/research/}
		\end{enumerate}
	\end{enumerate}
%%%%%%%%%%%%%%%%%%%%%%%%%%%%%%%%%%%%%%
%%%%%%%%%%%%%%%%%%%%%%%%%%%%%%%%%%%%%%
%%%%%%%%%%%%%%%%%%%%%%%%%%%%%%%%%%%%%%
\item United States Office of Personnel Management: \vspace{-0.3cm}
	\begin{enumerate} \itemsep -2pt
	\item USAJOBS: \vspace{-0.2cm}
		\begin{enumerate} \itemsep -2pt
		\item Student Jobs: \url{http://www.usajobs.gov/studentjobs/}
		\end{enumerate}
	\end{enumerate}
%%%%%%%%%%%%%%%%%%%%%%%%%%%%%%%%%%%%%%
%%%%%%%%%%%%%%%%%%%%%%%%%%%%%%%%%%%%%%
%%%%%%%%%%%%%%%%%%%%%%%%%%%%%%%%%%%%%%
\item Americans for the Arts: \vspace{-0.3cm}
	\begin{enumerate} \itemsep -2pt
	\item Internship Program: \url{http://www.americansforthearts.org/about_us/internships.asp}
	\end{enumerate}
\item New York Women's Foundation: \vspace{-0.3cm}
	\begin{enumerate} \itemsep -2pt
	\item Internship Opportunities: \url{http://www.nywf.org/internship.html}
	\item Volunteer Opportunities: \url{http://www.nywf.org/volunteer.html}
	\end{enumerate}
\item Council on International Educational Exchange (CIEE): \url{http://www.ciee.org/hire/index.aspx}
\item The John F. Kennedy Center for the Performing Arts: \vspace{-0.3cm}
	\begin{enumerate} \itemsep -2pt
	\item Kennedy Center Arts Management Internships: \url{http://www.kennedy-center.org/education/artsmanagement/internships/}
	\end{enumerate}
\item Washington Performing Arts Society (WPAS): \vspace{-0.3cm}
	\begin{enumerate} \itemsep -2pt
	\item Internships with WPAS: \vspace{-0.2cm}
		\begin{enumerate} \itemsep -2pt
		\item \url{http://www.wpas.org/aboutwpas/opportunities/intern.aspx}
		\item ``WPAS offers internships throughout the year. Applicants should be highly motivated, creative and hard-working individuals with an interest in all aspects of arts management. It is required that applicants have previous office experience.''
		\item In addition, applicants should possess: \vspace{-0.1cm}
			\begin{enumerate} \itemsep -1pt
			\item Interest/background in music, dance or performance art
			\item Strong organizational skills
			\item Effective writing and communication skills
			\item Ability to learn quickly, handle multiple tasks, take initiative, and work independently with little supervision
			\item High energy level and ability to work well in deadline and/or pressure situations
			\item Computer literacy
			\end{enumerate}
		\item ``WPAS interns leave our offices with a better understanding of arts management, knowledge of artists in a variety of fields (classical music, world music, dance and performance art), contacts in theaters throughout the D.C. metro area, practical experience and a portfolio of work. The internship is unpaid, however stipends are occasionally granted during the performance year (September - May). Interns are also invited to attend many WPAS performances on a complimentary basis.''
		\item Types of internships: \vspace{-0.1cm}
			\begin{enumerate} \itemsep -1pt
			\item Accounting Internship
			\item Development Internship
			\item Education Internship
			\item Marketing/Public Relations Internship
			\item Office Administration Internship
			\item Programming Internship
			\end{enumerate}
		\end{enumerate}
	\end{enumerate}
\item The Washington Ballet: Internships, \url{http://www.washingtonballet.org/about-twb/auditions-employment/#internships}
\item The Choral Arts Society of Washington: \vspace{-0.3cm}
	\begin{enumerate} \itemsep -2pt
	\item Internship and Apprenticeship Program: \url{http://www.choralarts.org/About-Us/Internships-and-Apprenticeships.aspx}
	\end{enumerate}
\item League of American Orchestras: Internships, \url{http://www.americanorchestras.org/career_center/internships.html}
%%%%%%%%%%%%%%%%%%%%%%%%%%%%%%%%%%%%%%
%%%%%%%%%%%%%%%%%%%%%%%%%%%%%%%%%%%%%%
%%%%%%%%%%%%%%%%%%%%%%%%%%%%%%%%%%%%%%
\item Congressional Hispanic Caucus Institute (CHCI): \vspace{-0.3cm}
	\begin{enumerate} \itemsep -2pt
	\item CHCI United Health Foundation Scholars: \vspace{-0.2cm}
		\begin{enumerate} \itemsep -2pt
		\item \url{http://www.chci.org/scholarships/page/chci-united-health-foundation-scholars-}
		\item In addition to providing scholarship opportunities for Latino youth, the United Health Foundation decided to partner with CHCI to create a six-month internship program for students interested in the medical field.
		\item Seventeen participants enrolled in either a full-time undergraduate or graduate course of study at an accredited two- or four-year college, university, vocational or technical school were selected.
		\end{enumerate}
	\item CHCI Congressional Internship: \vspace{-0.2cm}
		\begin{enumerate} \itemsep -2pt
		\item The purpose of the Congressional Internship Program (CIP) is to expose young Latinos to the legislative process and to strengthen their professional and leadership skills, ultimately promoting the presence of Latinos on Capitol Hill.
		\item The Congressional Internship Program provides college students with a paid Congressional work placement on Capitol Hill for a period of twelve weeks (Spring/Fall) or eight weeks (Summer). This unmatched experience allows students to learn first hand about our nation's legislative process.
		\end{enumerate}
	\end{enumerate}
\item Mexican American Legal Defense and Educational Fund (MALDEF): Law Clerk Summer Internship program, \url{http://maldef.org/about/jobs/index.html}
\item Hispanic Association of Colleges and Universities (HACU): \vspace{-0.3cm}
	\begin{enumerate} \itemsep -2pt
	\item HACU National Internship Program (HNIP): \url{http://www.hacu.net/hacu/HNIP_EN.asp}
	\end{enumerate}
%%%%%%%%%%%%%%%%%%%%%%%%%%%%
\item Smithsonian Institution: \vspace{-0.3cm}
	\begin{enumerate} \itemsep -2pt
	\item Smithsonian Institution Traveling Exhibition Service (SITES): \vspace{-0.2cm}
		\begin{enumerate} \itemsep -2pt
		\item Internship programs: \url{http://www.sites.si.edu/interns/internships.htm}
		\item ``The Smithsonian Institution Traveling Exhibition Service internship programs allows people with diverse interests, strengths, and goals to experience an educational environment where they can work and learn from professionals in the museum field.''
		\item ``SITES offers internship opportunities in a variety of different areas: public relations, development (fund raising), research, and project design.''
		\end{enumerate}
	\item Smithsonian Folkways Recordings (or simply, Smithsonian Folkways): \vspace{-0.2cm}
		\begin{enumerate} \itemsep -2pt
		\item Internships: \url{http://www.folkways.si.edu/about_us/jobs.aspx}
		\end{enumerate}
	\item Freer Gallery of Art / Arthur M. Sackler Gallery: \vspace{-0.2cm}
		\begin{enumerate} \itemsep -2pt
		\item Internships: \url{http://www.asia.si.edu/research/internships.asp}
		\end{enumerate}
	\item National Museum of American History: \vspace{-0.2cm}
		\begin{enumerate} \itemsep -2pt
		\item Jerome and Dorothy Lemelson Center for the Study of Invention and Innovation: \vspace{-0.1cm}
			\begin{enumerate} \itemsep -1pt
			\item Archival Internships: \url{http://invention.smithsonian.org/resources/research_interns.aspx}
			\end{enumerate}
		\end{enumerate}
	\end{enumerate}
%%%%%%%%%%%%%%%%%%%%%%%%%%%%
\item Council on International Educational Exchange (CIEE): \vspace{-0.3cm}
	\begin{enumerate} \itemsep -2pt
	\item CIEE's Trainee Program: \vspace{-0.2cm}
		\begin{enumerate} \itemsep -2pt
		\item part of the J-1 visa category of the US government�s Exchange Visitor Program
		\item \url{http://www.ciee.org/trainee/}
		\end{enumerate}
	\item CIEE Work \& Travel USA; and Internship USA: \vspace{-0.2cm}
		\begin{enumerate} \itemsep -2pt
		\item \url{http://www.ciee.org/hire/}
		\item \url{http://www.ciee.org/wat/}
		\end{enumerate}
	\end{enumerate}
\item American Institute For Foreign Study (AIFS): \vspace{-0.3cm}
	\begin{enumerate} \itemsep -2pt
	\item Camp America Counselors and Summer Staff: \url{http://www.aifs.com/work_travel.asp}
	\item Au Pair Placement: \url{http://www.aifs.com/au_pair.asp}
	\end{enumerate}
\item U.S. Department of State: \vspace{-0.3cm}
	\begin{enumerate} \itemsep -2pt
	\item Bureau of Educational and Cultural Affairs: \vspace{-0.2cm}
		\begin{enumerate} \itemsep -2pt
		\item International cultural programs: \url{http://exchanges.state.gov/cultural/related-cultural-programs.html}
		\item Office of Global Educational Programs: \vspace{-0.1cm}
			\begin{enumerate} \itemsep -1pt
			\item Camp Counselor: \vspace{-0.1cm}
				\begin{itemize} \itemsep -1pt
				\item \url{http://exchanges.state.gov/jexchanges/programs/camp.html}
				\item Camp counselors interact with groups of American youth by overseeing their camp activities during the U.S. summer.
				\item Through the Camp Counselor program, American campers have the chance to gain knowledge of foreign cultures, while foreign participants increase their knowledge of American culture.
				\item Participants must be at least 18 years of age and may work as counselors in U.S. summer camps for up to four months. Extensions are not allowed. They receive a combination a pay and benefits equal to Americans who work in the same position.
				\end{itemize}
			\end{enumerate}
		\item Private Sector Exchange office: \vspace{-0.1cm}
			\begin{enumerate} \itemsep -1pt
			\item \url{http://exchanges.state.gov/jexchanges/index.html}
			\item The Private Sector Exchange office designates, monitors and partners with U.S. organizations, including government agencies, academic institutions, educational and cultural organizations, and corporations, that administer the Exchange Visitor Program.
			\item Au Pair program: \vspace{-0.1cm}
				\begin{itemize} \itemsep -1pt
				\item Through the Au Pair program, foreign nationals between 18 and 26 years of age participate in the home life of a host family. Au pairs provide limited childcare services for up to 12 months. An extension of 6, 9, or 12 months may be granted in certain cases.
				\item \url{http://exchanges.state.gov/jexchanges/programs/aupair.html}
				\end{itemize}
			\item Internships: \vspace{-0.1cm}
				\begin{itemize} \itemsep -1pt
				\item \url{http://exchanges.state.gov/jexchanges/programs/intern.html}
				\item Internship programs are designed to allow foreign professionals to come to the United States to gain exposure to U.S. culture and to receive training in U.S. business practices in their chosen occupational field.
				\item The maximum duration of an internship in any occupational field is 12 months.
				\item Upon completion of their exchange programs, participants are expected to return to their home countries.
				\item The State Department allows internships in the following occupational categories: \vspace{-0.1cm}
					\begin{itemize} \itemsep -1pt
					\item Agriculture, Forestry, and Fishing
					\item Arts and Culture
					\item Construction and Building Trades
					\item Education, Social Sciences, Library Science, Counseling and Social Services
					\item Health Related Occupations
					\item Hospitality and Tourism
					\item Information Media and Communications
					\item Management, Business, Commerce and Finance
					\item Public Administration and Law
					\item The Sciences, Engineering, Architecture, Mathematics, and Industrial Occupations.
					\end{itemize}
				\item An Intern must be a foreign national: \vspace{-0.1cm}
					\begin{itemize} \itemsep -1pt
					\item Who is currently enrolled in and pursuing studies at a foreign degree- or certificate-granting post-secondary academic institution outside the United States, or
					\item Who has graduated from such an institution no more than 12 months prior to his or her exchange visitor program start date.
					\end{itemize}
				\item Interns cannot work in unskilled or casual labor positions, in positions that require or involve child care or elder care, or in any kind of position that involves medical patient care or contact. Nor can interns work in positions that require more than 20 per cent clerical or office support work.
				\end{itemize}
			\item The Summer Work Travel Program: \vspace{-0.1cm}
				\begin{itemize} \itemsep -1pt
				\item \url{http://exchanges.state.gov/jexchanges/programs/swt.html}
				\item In the summer work travel program, post-secondary students may enter the United States to work and travel during their summer vacation.
				\item Participants can be admitted to the program more than once.
				\item The maximum length of the program is four months.
				\item Most of the time, participants work in unskilled service positions at resorts, hotels, restaurants, and amusement parks. However, they may also work in other types of organizations.
				\item For example, they could work in architectural firms, scientific research organizations, graphic art/publishing and other media communication businesses, advertising agencies, computer software and electronics firms, legal offices, etc.
				\item The program may not exceed four-months and must be finished during the student's summer vacation.
				\item Participants receive pay and benefits equal to an American working in the same or similar position.
				\end{itemize}
			\item Training programs: \vspace{-0.1cm}
				\begin{itemize} \itemsep -1pt
				\item \url{http://exchanges.state.gov/jexchanges/programs/trainee.html}
				\item Training programs are designed to allow foreign professionals to come to the United States to gain exposure to U.S. culture and to receive training in U.S. business practices in their chosen occupational field.
				\item Foreign nationals have had the opportunity to train with some of the finest employers in the U.S., gaining real time experience in their chosen career fields.
				\item Upon completion of their exchange programs, participants are expected to return to their home countries to utilize their newly learned skills and knowledge to advance their careers and share their experiences with their communities.
				\item The State Department allows training programs in the following occupational categories: \vspace{-0.1cm}
					\begin{itemize} \itemsep -1pt
					\item Agriculture, Forestry, and Fishing
					\item Arts and Culture
					\item Construction and Building Trades
					\item Education, Social Sciences, Library Science, Counseling and Social Services
					\item Health Related Occupations
					\item Hospitality and Tourism
					\item Information Media and Communications
					\item Management, Business, Commerce and Finance
					\item Public Administration and Law
					\item The Sciences, Engineering, Architecture, Mathematics, and Industrial Occupations.
					\end{itemize}
				\item A trainee must be a foreign national who has: \vspace{-0.1cm}
					\begin{itemize} \itemsep -1pt
					\item A degree or professional certificate from a foreign post-secondary academic institution and at least one year of prior related work experience in his or her occupational field outside the United States, or
					\item Five years of work experience outside the United States in the occupational field in which they are seeking training.
					\end{itemize}
				\end{itemize}
			\item Specialists: \vspace{-0.1cm}
				\begin{itemize} \itemsep -1pt
				\item \url{http://exchanges.state.gov/jexchanges/programs/specialist.html}
				\item This category is for a participant who is an expert in a field of specialized knowledge or skill who will demonstrate such skills in the United States. Such exchanges are to provide opportunities to increase the exchange knowledge and ideas between American and foreign specialists. The maximum duration of this program is one year.
				\item This category is for foreign nationals who are experts in a field of specialized knowledge or skill, coming to the United States for observing, consulting, or demonstrating their special skills, except: Professors and Research Scholars, Short-Term Scholars, and Alien Physicians.
				\item Individuals participating in the specialist program are: \vspace{-0.1cm}
					\begin{itemize} \itemsep -1pt
					\item Experts in a field of specialized knowledge or skill;
					\item Seeks to travel to the United States for the purpose of observing, consulting, or demonstrating their special knowledge or skills;
					\item Does not fill a permanent or long-term position of employment while in the U.S.
					\end{itemize}
				\end{itemize}
			\item International Visitor: \vspace{-0.1cm}
				\begin{itemize} \itemsep -1pt
				\item \url{http://exchanges.state.gov/jexchanges/programs/intl_visitor.html}
				\item The international visitor category enables visitors to better understand American culture and enhanced American knowledge of foreign cultures.
				\item This category is for individuals who are recognized as potential leaders in their own country, selected by the Department of State to participate in observation tours, discussions, consultation, professional meetings, conferences, workshops and travel.
				\item The maximum duration of the program is one year.
				\end{itemize}
			\item Alien Physician: \vspace{-0.1cm}
				\begin{itemize} \itemsep -1pt
				\item \url{http://exchanges.state.gov/jexchanges/programs/physician.html}
				\item The Alien Physician program is for foreign national physicians seeking entry into U.S. graduate medical education programs or training at accredited U.S. schools of medicine or other U.S. institutions.
				\item There are generally two types of exchange programs in which a foreign national physician (also referred to as a foreign/international medical graduate) participates: \vspace{-0.1cm}
					\begin{itemize} \itemsep -1pt
					\item Clinical training in the �alien physician� category
					\item Non-Clinical training in the �research scholar� category
					\end{itemize}
				\end{itemize}
			\item FORTUNE/U.S. State Department Global Women's Mentoring Partnership: \vspace{-0.1cm}
				\begin{itemize} \itemsep -1pt
				\item \url{http://exchanges.state.gov/citizens/professionals/fortunepartnership.html}
				\item This public-private partnership places talented, emerging women leaders from all over the world in mentoring programs with FORTUNE's Most Powerful Women Leaders.
				\item For three weeks, American and international participants work together in mentoring relationships to share the skills and experiences necessary for strengthening women�s leadership.
				\end{itemize}
			\item American Council of Young Political Leaders (ACYPL): \vspace{-0.1cm}
				\begin{itemize} \itemsep -1pt
				\item \url{http://exchanges.state.gov/citizens/profs/acypl.html}
				\item \url{http://www.acypl.org/}
				\item For 44 years, the American Council of Young Political Leaders (ACYPL) has designed, organized and managed unique international educational exchanges for young political leaders (ages 25-40) worldwide.
				\item ACYPL programs are designed to promote mutual understanding, respect, and friendship and to cultivate long-lasting relationships among young people who are poised to become tomorrow's global leaders and policy makers.
				\item American participants are nominated by members of Congress, governors, political party leaders, and ACYPL alumni, while international delegates are selected from countries where ACYPL is currently conducting programs by international program partners with the U.S. Embassy input.
				\end{itemize}
			\item Edward R. Murrow Program for Journalists: \vspace{-0.1cm}
				\begin{itemize} \itemsep -1pt
				\item \url{http://exchanges.state.gov/ivlp/murrow.html}
				\item The Edward R. Murrow Program for Journalists invites rising international journalists to travel to the United States and examine journalistic principles and practices.
				\end{itemize}
			\end{enumerate}
		\item Office of Citizen Exchanges: \vspace{-0.1cm}
			\begin{enumerate} \itemsep -1pt
			\item Youth Programs Division: \vspace{-0.1cm}
				\begin{itemize} \itemsep -1pt
				\item \url{http://exchanges.state.gov/youth/index.html}
				\item The Youth Programs Division is committed to empowering the successor generation and establishing long-lasting ties between the United States and other countries through exchange programs and institutional partnerships.
				\item Programs focus primarily on secondary schools and promote mutual understanding, leadership development, educational transformation, and democratic ideals.
				\item Year-Long Programs, Short Term Programs, and Virtual Partnerships: \url{http://exchanges.state.gov/youth/programs-by-type.html}
				\item Programs for Young Americans, and Programs for International Students and Teachers: \url{http://exchanges.state.gov/youth/programs-by-participants.html}
				\item Opportunities for American Hosts: Families and Schools, \url{http://exchanges.state.gov/youth/opps-for-am-hosts.html}
				\item Programs for High School Students: \url{http://exchanges.state.gov/youth/programs.html}
				\end{itemize}
			\item Professional Exchanges Division: \vspace{-0.1cm}
				\begin{itemize} \itemsep -1pt
				\item \url{http://exchanges.state.gov/citizens/profs.html}
				\item The Professional Exchanges division provides grants to U.S. nonprofit organizations to carry out exchange programs that support the professional development of foreign participants. The purpose of each exchange program is to engage with foreign leaders in critical professions, to demonstrate respect for foreign cultures, and to promote mutual understanding between the people of the United States and other countries.
				\item Professional exchanges typically last several years and include internships, study tours or workshops in the United States and in the host country. Participants come from a variety of professions including education administrators, public servants, journalists, labor union officials, entrepreneurs, environmental leaders, jurists, lawyers, and civic leaders.
				\item ECA grant opportunities: \vspace{-0.1cm}
					\begin{itemize} \itemsep -1pt
					\item Open Funding Opportunities: Requests For Grant Proposals (RFGPs), \url{http://exchanges.state.gov/grants/open2.html}
					\item Grants.gov: \url{http://www.grants.gov/}
					\end{itemize}
				\item Grants by Region: \vspace{-0.1cm}
					\begin{itemize} \itemsep -1pt
					\item \url{http://exchanges.state.gov/citizens/professionals/grant-region.html}
					\item Africa 
					\item East Asia and the Pacific 
					\item Europe and Eurasia 
					\item North Africa and the Middle East 
					\item South and Central Asia 
					\item Western Hemisphere 
					\item Multi-regional
					\end{itemize}
				\end{itemize}
			\end{enumerate}
		\end{enumerate}
	\end{enumerate}
\end{enumerate}






%%%%%%%%%%%%%%%%%%%%%%%%%%%%%%%%%%%%%%%%%%%
\section{Resources on Studying Abroad}
\label{resourcesonstudyingabroad}

Resources on studying abroad: \vspace{-0.3cm}
\begin{enumerate} \itemsep -4pt
\item Council on International Educational Exchange (CIEE): \vspace{-0.3cm}
	\begin{enumerate} \itemsep -2pt
	\item Study abroad programs for high school students from the United States: \vspace{-0.2cm}
		\begin{enumerate} \itemsep -2pt
		\item \url{http://www.ciee.org/hsabroad/index.html}
		\item \url{http://www.ciee.org/hsabroad/high-school-study-abroad/index.html}
		\item These programs include:: \vspace{-0.1cm}
			\begin{enumerate} \itemsep -1pt
			\item High School Abroad programs (for U.S. high school students)
			\item Summer High School Abroad programs (for U.S. high school students)
			\item Gap Year Abroad programs (for recent U.S. high school graduates)
			\end{enumerate}
		\end{enumerate}
	\end{enumerate}
\item U.S. Department of State: \vspace{-0.3cm}
	\begin{enumerate} \itemsep -2pt
	\item Bureau of Educational and Cultural Affairs: \vspace{-0.2cm}
		\begin{enumerate} \itemsep -2pt
		\item Office of Global Educational Programs: \vspace{-0.1cm}
			\begin{enumerate} \itemsep -1pt
			\item EducationUSA: \vspace{-0.1cm}
				\begin{itemize} \itemsep -1pt
				\item EducationUSA is a network of more than 400 student advising centers, which offer accurate, comprehensive, objective and timely information about educational opportunities in the United States and guidance to qualified individuals on how best to access those opportunities. This includes information about application procedures, standardized test requirements, student visas, financial aid, and the full range of accredited U.S. higher education institutions.
				\item \url{http://exchanges.state.gov/globalexchanges/index/educationusa.html}
				\item \url{http://www.educationusa.state.gov/} and \url{http://www.educationusa.info/centers.php}
				\end{itemize}
			\item Open Doors: \vspace{-0.1cm}
				\begin{itemize} \itemsep -1pt
				\item The Educational Information and Resources Branch funds Open Doors, a census of foreign students and scholars in the U.S. and of U.S. students studying abroad published annually by the Institute for International Education.
				\item Open Doors data is used by U.S. embassies, the Departments of State, Commerce, and Education, and U.S. colleges and universities to inform policy decisions about educational exchanges, trade in educational services, and study abroad activity.
				\item \url{http://exchanges.state.gov/globalexchanges/index/open_doors.html}
				\item \url{http://www.opendoors.iienetwork.org/}
				\end{itemize}
			\end{enumerate}
		\item EducationUSA: \vspace{-0.1cm}
			\begin{enumerate} \itemsep -1pt
			\item \url{http://educationusa.state.gov/}
			\item For U.S. (college) students who want to study/work abroad: \url{http://www.educationusa.info/pages/students/forus.php}
			\end{enumerate}
		\end{enumerate}
	\end{enumerate}
\item IES Abroad (formerly Institute of European Studies / Institute for the International Education of Students): \vspace{-0.3cm}
	\begin{enumerate} \itemsep -2pt
	\item \url{https://www.iesabroad.org/} and \url{https://www.iesabroad.org/IES/home.html}
	\end{enumerate}
\item Global Learning Semesters, Inc.: \vspace{-0.3cm}
	\begin{enumerate} \itemsep -2pt
	\item Summer in the Mediterranean: \vspace{-0.2cm}
		\begin{enumerate} \itemsep -2pt
		\item \url{http://www.globalsemesters.com/Mediterranean.html}
		\item Has programs in the following areas: \vspace{-0.1cm}
			\begin{enumerate} \itemsep -1pt
			\item Art \& Photography
			\item Early Christianity
			\item Greek Heritage
			\item International Marketing
			\item Music
			\end{enumerate}
		\end{enumerate}
	\end{enumerate}
\item American Institute For Foreign Study (AIFS): \vspace{-0.3cm}
	\begin{enumerate} \itemsep -2pt
	\item \url{http://www.aifs.com/}
	\item College Study Abroad: \url{http://www.aifsabroad.com/}
	\item For high school students: \vspace{-0.2cm}
		\begin{enumerate} \itemsep -2pt
		\item Gifted Education: \url{http://www.aifs.com/gifted_education.asp}
		\item High School Study and Travel: \url{http://www.aifs.com/highschool_study_travel.asp}
		\item Academic Year in America (AYA): \url{http://www.academicyear.org/?source=AIFS}
		\end{enumerate}
	\end{enumerate}
\end{enumerate}





%%%%%%%%%%%%%%%%%%%%%%%%%%%%%%%%%%%%%%%%%%%
\section{College Preparation}
\label{collegepreparation}

College preparation: \vspace{-0.3cm}
\begin{enumerate} \itemsep -4pt
\item {\it Guide to Online Schools} [or {\it GuideToOnlineSchools.com}], {\it The Top 53 College Preparation Resources for Students}. Available at: \url{http://www.guidetoonlineschools.com/tips-and-tools/college-prep-resources}; last accessed on August 25, 2010.
\item U.S. Department of Education's resources for parents to help their children learn: \url{http://www2.ed.gov/parents/academic/help/hyc.html} and \url{http://www2.ed.gov/parents/academic/help/homework/index.html}
\item The College Board: \vspace{-0.3cm}
	\begin{enumerate} \itemsep -2pt
	\item Information about SATs, college preparation, and financial aid
	\item {\it Trends in Higher Education} series 201X: \url{http://trends.collegeboard.org/}
	\item \url{http://www.collegeboard.com/}
	\end{enumerate}
\item {\it Accreditation.org}: \vspace{-0.3cm}
	\begin{enumerate} \itemsep -2pt
	\item Information about the accreditation of engineering degree programs around the world
	\item \url{http://www.accreditation.org/}
	\end{enumerate}
\item {\it New York Times}: \vspace{-0.3cm}
	\begin{enumerate} \itemsep -2pt
	\item The Learning Network: \url{http://learning.blogs.nytimes.com/category/test-yourself/}
	\item New York Times Magazine: \vspace{-0.2cm}
		\begin{enumerate} \itemsep -2pt
		\item The Sep 20, 2010 issue has many articles covering how technology can be used to improve education in K-12 programs. Available online at: \url{http://www.nytimes.com/indexes/2010/09/19/magazine/index.html?ref=magazine}; last accessed on September 20, 2010.
		\item ``New York Times Magazine Features Technology in Education,'' in {\it CCC Blog}, Computing Community Consortium (CCC), Computing Research Association (CRA), Sep 20, 2010. Available online at: \url{http://www.cccblog.org/2010/09/20/new-york-times-magazine-features-technology-in-education/}; last accessed on September 20, 2010.
		\item Articles in this issue discuss: \vspace{-0.1cm}
			\begin{enumerate} \itemsep -1pt
			\item How journalists can make use of technology to automate certain tasks, and improve their productivity and effectiveness in covering news stories
			\item How children can create computer games that introduces them to careers in computing and helps them to develop skills in computational thinking
			\item How to learn things without a lot of rote learning, to have fun while learning, and to use technology to make learning more fun
			\end{enumerate}
		\end{enumerate}
	\end{enumerate}
\item University of Southern California, USC: \vspace{-0.3cm}
	\begin{enumerate} \itemsep -2pt
	\item USC Office of Continuing Education and Summer Programs: \vspace{-0.2cm}
		\begin{enumerate} \itemsep -2pt
		\item \url{http://cesp.usc.edu/}
		\item These programs allow students in K-12 to earn credit at USC, and exposes them to different majors/professions, like medicine, engineering, creative writing, or film making.
		\item Students can benefit from these programs, and learn about different academic disciplines before applying to college. This would help them in their college applications.
		\item Underrepresented minority students can get scholarships to attend these programs. So, if parents have financial difficulty paying for the programs, they can seek financial aid for this.
		\item Also, current undergraduates can also sign up for programs to learn about marketing, finance, and entrepreneurship. They can also do summer research with USC researchers.
		\end{enumerate}
	\item Summer sports programs for youths: \vspace{-0.2cm}
		\begin{enumerate} \itemsep -2pt
		\item SC Futbol Academy (USC Soccer Camps): \url{http://www.usctrojans.com/sports/w-soccer/spec-rel/021610aaa.html}
		\item Mick Haley's USC Girls Volleyball Camp: \url{http://www.usctrojans.com/sports/w-volley/spec-rel/volley-camp.html}
		\item Salo Swim Camp: \url{http://www.saloswimcamp.com/on-line/default.asp}
		\item USC NYSP Trojan KidSCamp: \url{http://sait.usc.edu/recsports/site_content/youth_sports/nysptk.html}
		\item After School Sports Connection, ASSC (operates in fall, spring, and summer): \url{http://sait.usc.edu/recsports/site_content/youth_sports/assc.html}
		\end{enumerate}
	\end{enumerate}
\item Telluride Association: \vspace{-0.3cm}
	\begin{enumerate} \itemsep -2pt
	\item Telluride Association Summer Program (TASP) [ for high school students ]: \url{http://www.tellurideassociation.org/programs/high_school_students/tasp/tasp_general_info.html}
	\item Telluride Association Sophomore Seminar (TASS) [ for high school students ]: \url{http://www.tellurideassociation.org/programs/high_school_students/tass/tass_general_info.html}
	\item Resources for high school educators to nominate summer program applicants: \url{http://www.tellurideassociation.org/programs/high_school_students/hs_resources/hs_resources_general_information.html}
	\end{enumerate}
\item MathNerds: \vspace{-0.3cm}
	\begin{enumerate} \itemsep -2pt
	\item \url{http://www.mathnerds.com/}
	\item ``Provides free, discovery-based, mathematical guidance via an international, volunteer network of mathematicians.''
	\item If you have a mathematical problem to solve, you can ask mathematicans at {\it MathNerds} for help.
	\item They would require you to discuss your attempted approaches/solutions.
	\item If you have not made attempts to solve the problem, they will not give you much guidance.
	\item In addition, they cannot solve problems for you.
	\item They provide guidance for mathematical problems from K-12 material through undergraduate mathematics and statistics classes.
	\item They also provide help for selected topics in advanced mathematics classes (for graduate students).
	\item Other resources: \url{http://www.mathnerds.com/links/links.aspx}
	\end{enumerate}
\item Hobsons: \vspace{-0.3cm}
	\begin{enumerate} \itemsep -2pt
	\item CollegeView (Hobsons' college recruiting services): \url{http://www.collegeview.com/index.jsp}
	\end{enumerate}
\item Sponsors for Educational Opportunity (SEO): \vspace{-0.3cm}
	\begin{enumerate} \itemsep -2pt
	\item Resources: \url{http://www.seo-usa.org/ScholarsResources}
	\end{enumerate}
\item U.S. Department of Education: \vspace{-0.3cm}
	\begin{enumerate} \itemsep -2pt
	\item Students.gov: \url{http://www.students.gov/STUGOVWebApp/index.jsp}
	\item college.gov: \url{http://www.college.gov/wps/portal}
	\end{enumerate}
\item U.S. Department of State: \vspace{-0.3cm}
	\begin{enumerate} \itemsep -2pt
	\item Bureau of Educational and Cultural Affairs: \vspace{-0.2cm}
		\begin{enumerate} \itemsep -2pt
		\item EducationUSA: \vspace{-0.1cm}
			\begin{enumerate} \itemsep -1pt
			\item Information for international students: \url{http://www.educationusa.info/students.php}
			\end{enumerate}
		\end{enumerate}
	\end{enumerate}
\item Congressional Hispanic Caucus Institute (CHCI): \vspace{-0.3cm}
	\begin{enumerate} \itemsep -2pt
	\item CHCI Education Center: \vspace{-0.2cm}
		\begin{enumerate} \itemsep -2pt
		\item \url{http://www.chci.org/education_center/}
		\item Has resources on college planning, financial aid, scholarships, college internships, and housing.
		\item For Parents: \url{http://www.chci.org/education_center/page/for-parents}
		\item For Students: \url{http://www.chci.org/education_center/page/for-students}
		\end{enumerate}
	\end{enumerate}
\item My College Options: \vspace{-0.3cm}
	\begin{enumerate} \itemsep -2pt
	\item \url{http://www.mycollegeoptions.org/}
	\item ``My College Options is a FREE college planning service, offering assistance to students, parents, high schools, counselors, and teachers nationwide.''
	\item ``It is designed to assist high school students in exploring a wide range of post-secondary opportunities, with special emphasis on the college search process.''
	\end{enumerate}
\end{enumerate}

Resources for financial aid: \vspace{-0.3cm}
\begin{enumerate} \itemsep -4pt
\item {\it Guide to Online Schools} [or {\it GuideToOnlineSchools.com}], {\it Financial Aid}. Available at: \url{http://www.guidetoonlineschools.com/financial-aid}; last accessed on August 25, 2010.
\item The Institute for College Access \& Success, {\it Links} [ Resources that provide information about student loans and student debt ]. Available at: \url{http://projectonstudentdebt.org/links.vp.html}; last accessed on September 4, 2010. [ Also, see \url{http://projectonstudentdebt.org/advice.vp.html} and \url{http://ticas.org/about.vp.html}. ]
\end{enumerate}


Information about colleges and universities: \vspace{-0.3cm}
\begin{enumerate} \itemsep -4pt
\item The Institute for College Access \& Success, {\it College InSight}. Available at: \url{http://college-insight.org/}; last accessed on September 4, 2010.
\item 
\end{enumerate}



%%%%%%%%%%%%%%%%%%%%%%%%%%%%%%%%%%%%%%%%%%%
\section{Outreach for Students in Colleges and Universities}
\label{outreachcollege}

Resources to reach out to students in colleges and universities: \vspace{-0.3cm}
\begin{enumerate} \itemsep -4pt
%%%%%%%%%%%%%%%%%%%%%%%%%%%%%
\item Film contests: \vspace{-0.3cm}
	\begin{enumerate} \itemsep -2pt
	\item Ed Wood Film Festival [@ USC]: \vspace{-0.2cm}
		\begin{enumerate} \itemsep -2pt
		\item Celebrating independent filmmaking at USC and named for the famous director, the Ed Wood Film Festival is put on by a committee of Residential Education staff members at New Residential College, chaired by the Cinema Floor RA's.
		\item Teams of students come together to obtain the year's secret theme in which to write, shoot, and edit their very own short film within 24 hours. A week later, the films are shown at USC's Norris Cinema and a panel of judges selects the Festival winners in a variety of categories.
		\item \url{http://sait.usc.edu/resed/Programs.aspx}
		\end{enumerate}
	\item Reel LA: Parkside International Film Festival [or USC Reel LA Film Festival at USC]; see \url{http://www-scf.usc.edu/~pirc/areagov/government.php}
	\end{enumerate}
%%%%%%%%%%%%%%%%%%%%%%%%%%%%%
\item residential education: \vspace{-0.3cm}
	\begin{enumerate} \itemsep -2pt
	\item Telluride Association: \vspace{-0.2cm}
		\begin{enumerate} \itemsep -2pt
		\item Information about how to reside at the Cornell Branch (also known as Telluride House or CBTA) and the Michigan Branch of Telluride Association, which are ``residential colleges'': \url{http://www.tellurideassociation.org/programs/university_students.html}
		\item Awards for residents at the Cornell or Michigan Branch: \url{http://www.tellurideassociation.org/programs/university_students/us_awards.html}
		\end{enumerate}
	\end{enumerate}
%%%%%%%%%%%%%%%%%%%%%%%%%%%%%
\item MathNerds: \vspace{-0.3cm}
	\begin{enumerate} \itemsep -2pt
	\item \url{http://www.mathnerds.com/}
	\item ``Provides free, discovery-based, mathematical guidance via an international, volunteer network of mathematicians.''
	\item If you have a mathematical problem to solve, you can ask mathematicans at {\it MathNerds} for help.
	\item They would require you to discuss your attempted approaches/solutions.
	\item If you have not made attempts to solve the problem, they will not give you much guidance.
	\item In addition, they cannot solve problems for you.
	\item They provide guidance for mathematical problems from K-12 material through undergraduate mathematics and statistics classes.
	\item They also provide help for selected topics in advanced mathematics classes (for graduate students).
	\end{enumerate}
%%%%%%%%%%%%%%%%%%%%%%%%%%%%%
\item Invent Now: \vspace{-0.3cm}
	\begin{enumerate} \itemsep -2pt
	\item 
	\end{enumerate}
\item Journal of Young Investigators (JYI): \vspace{-0.3cm}
	\begin{enumerate} \itemsep -2pt
	\item \url{http://www.jyi.org/}
	\item ``peer-reviewed journal for undergraduates''
	\item ``JYI's web journal (which is also called JYI) is dedicated to the presentation of undergraduate research in science, mathematics, and engineering. It publishes the best submissions from undergraduates, with an emphasis on both the quality of research and the manner in which it is communicated. The journal, JYI, also allows students to experience the other side of the scientific publication process: the review process. Students working with their faculty advisors review the work of their peers and determine whether that work is acceptable for publication in JYI.''
	\end{enumerate}
\item The Recording Academy: \vspace{-0.3cm}
	\begin{enumerate} \itemsep -2pt
	\item GRAMMY U: \vspace{-0.2cm}
		\begin{enumerate} \itemsep -2pt
		\item \url{http://www.grammy365.com/grammy-u}
		\item GRAMMY U is a unique and fast-growing community of full-time college students, primarily between the ages of 17 and 25,  who are pursuing a career in the recording industry.
		\item The Recording Academy created GRAMMY U to help prepare college students for their careers in the music industry through networking, educational programs and performance opportunities.
		\item GRAMMY U is designed to enhance students' current academic curriculum with access to recording industry professionals to give an ``out of classroom'' perspective on the recording industry.
		\end{enumerate}
	\end{enumerate}
%%%%%%%%%%%%%%%%%%%%%%%%%%%%%
\item --- --- --- --- --- --- --- --- --- --- --- --- --- --- --- --- --- --- --- --- --- --- --- --- --- --- --- --- --- --- ---
\item \colorbox{blue}{\bf Help for Underrepresented Minorities}
% Help for Underrepresented Minorities
\item INROADS, Inc.: \vspace{-0.3cm}
	\begin{enumerate} \itemsep -2pt
	\item Internships: \url{http://www.inroads.org/interns/internWhatItTakes.jsp}
	\end{enumerate}
\item The PhD Project: \vspace{-0.3cm}
	\begin{enumerate} \itemsep -2pt
	\item \url{http://www.phdproject.org/index.html}
	\item Program and informational network to encourage ``African-Americans, Hispanic-Americans and Native Americans'' to pursue Ph.D. programs in business and seek careers in academia.
	\item Annual PhD Project Conference: \vspace{-0.2cm}
		\begin{enumerate} \itemsep -2pt
		\item Conference: \vspace{-0.1cm}
			\begin{enumerate} \itemsep -1pt
			\item \url{http://www.phdproject.org/conference.html}
			\item \url{http://www.phdproject.org/conference_application.html}
			\item For prospective Ph.D. students in business to learn more about Ph.D. programs in business, the Ph.D. application process, and life in graduate school.
			\item Registration Policy: \vspace{-0.1cm}
				\begin{itemize} \itemsep -1pt
				\item If you are selected to attend the conference you will be required to pay a \$200 registration fee which can be processed via credit card during the registration process. All travel and conferences expenses will paid by The PhD Project (total conference expenses for hotel, meals, materials, and transportation are valued at approximately \$1,500 per invited attendee.) Your investment of the \$200 registration fee will be refunded if you enter a full-time, AACSB accredited business doctoral program within 3 years of attending the conference. 
				\item If you previously attended a PhD Project Conference, you may submit an application to be reviewed, however if you are selected to attend, The PhD Project will only cover hotel costs (shared room with another participant). You will be required to pay the registration and travel costs
				\end{itemize}
			\end{enumerate}
		\item Resources for Potential/Current Doctoral Students: \vspace{-0.1cm}
			\begin{enumerate} \itemsep -1pt
			\item \url{http://www.phdproject.org/resources.html}
			\item Information about good business schools that offer Ph.D. programs, preparation for the GMAT, and the life in graduate school as a Ph.D. student.
			\item Suggested Reading: \vspace{-0.1cm}
				\begin{itemize} \itemsep -1pt
				\item \url{http://www.phdproject.org/reading.html}
				\item Has information life in graduate school as a Ph.D. student, racial diversity/issues in higher education, job searching in academia, and work-life balance for female Ph.D. students.
				\end{itemize}
			\end{enumerate}
		\item The PhD Project Doctoral Student Association (DSA): \vspace{-0.1cm}
			\begin{enumerate} \itemsep -1pt
			\item The PhD Project network: \vspace{-0.1cm}
				\begin{itemize} \itemsep -1pt
				\item \url{http://www.myphdnetwork.org/}
				\item ``There are 5 discipline specific associations covering the major areas of business education: Accounting, Finance, Information Systems, Management, Marketing.''
				\end{itemize}
			\end{enumerate}
		\end{enumerate}
	\end{enumerate}
\item MS-to-Ph.D. program for underrepresented minorities at Fisk and Vanderbilt in certain areas of
science (including astronomy, material science, and physics)
\item Outreach programs for underrepresented minorities to help them get into medical (and/or graduate) schools. Search for ``PREP (Post-baccalaureate Research Education Programs),'' which have stipends. E.g., Georgetown University School of Medicine, and George Washington University's medical school
\item New York University: \vspace{-0.3cm}
	\begin{enumerate} \itemsep -2pt
	\item Leonard N. Stern School of Business: \vspace{-0.2cm}
		\begin{enumerate} \itemsep -2pt
		\item Stern Pre-Doctoral program: \url{http://www.stern.nyu.edu/AcademicPrograms/PhD/Pre-Doctoral/index.htm}
		\end{enumerate}
	\end{enumerate}
\end{enumerate}



%%%%%%%%%%%%%%%%%%%%%%%%%%%%%%%%%%%%%%%%%%%
\section{Science \& Engineering Outreach}
\label{stemoutreach}

%%%%%%%%%%%%%%%%%%%%%%%%%%%%%%%%%%%%%%%%%%%
\subsection{Precollege Science \& Engineering Outreach}
\label{stemoutreachk12}

Science and engineering outreach to high-school (and middle-school) students, and their parents, teachers, and career counselors: \vspace{-0.3cm}
\begin{enumerate} \itemsep -4pt
\item {\it MentorNet}: \vspace{-0.3cm}
	\begin{enumerate} \itemsep -2pt
	\item \url{http://www.mentornet.net/}
	\item Enables people to network with scientists, engineers, and professors in Science, Technology, Engineering, and Mathematics (STEM)
	\item Is very supportive of minorities, so that more minorities (particularly underrepresented minorities) can be attracted to STEM careers
	\end{enumerate}
\item International Science Olympiad (for high school students): \vspace{-0.3cm}
	\begin{enumerate} \itemsep -2pt
	\item International Olympiad in Informatics: \url{http://en.wikipedia.org/wiki/International_Olympiad_in_Informatics} and \url{http://www.ioinformatics.org/index.shtml}
	\item International Mathematical Olympiad: \url{http://www.imo-official.org/}
	\item International Physics Olympiad: \url{http://www.jyu.fi/tdk/kastdk/olympiads/}
	\item International Chemistry Olympiad: \url{http://www.icho.sk/}
	\item International Biology Olympiad: \url{http://www.ibo-info.org/}
	\item \url{http://scienceolympiads.org/}
	\end{enumerate}
\item International Astronomy Olympiad: \url{http://www.issp.ac.ru/iao/}
\item International Earth Science Olympiad: \url{http://en.wikipedia.org/wiki/International_Earth_Science_Olympiad}
\item International Junior Science Olympiad (for students younger than 15 years old): \url{http://www.ijso-official.org/home}
\item Teen Leadership Institute Science, Technology, Engineering, and Math (STEM) programs @ YWCA Greater Pittsburgh; see \url{http://www.ywcapgh.org/STEM_Programs.asp}
\item For Inspiration and Recognition of Science and Technology (FIRST): \url{http://www.usfirst.org/} (including resources and guides to mentoring); scholarships @ \url{http://www.usfirst.org/aboutus/content.aspx?id=508}; and robotics programs @ \url{http://www.usfirst.org/roboticsprograms/frc/default.aspx?=966}
\item Mac Hyman, ``Good Choices for Great Careers in the Mathematical Sciences,'' talk given at 2008 SIAM Annual Meeting. Available at: \url{http://client.blueskybroadcast.com/siam08/hyman/index.html}; last accessed on August 25, 2010. Also, see \url{http://meetings.siam.org/program.cfm?CONFCODE=AN08}, \url{http://www.siam.org/meetings/an08/program.php}, and \url{http://www.siam.org/meetings/an08/}.
\item {\it RoboCup}\texttrademark\ competitions: \vspace{-0.2cm}
	\begin{enumerate} \itemsep -2pt
	\item Junior category for K-12 students involves contests the these areas of challenges: \vspace{-0.1cm}
		\begin{enumerate} \itemsep -1pt
		\item soccer
		\item dance
		\item rescue operations
		\end{enumerate}
	\item \url{http://www.robocup.org/}
	\end{enumerate}
\item {\it Curriki}, which is an online educational resource for teachers, students, and parents in K-12: \url{http://www.curriki.org/xwiki/bin/view/Main/About}
%%%%%%%%%%%%%%%%%%%%%%%%%%%%%%%%%%%%%%%%
%%%%%%%%%%%%%%%%%%%%%%%%%%%%%%%%%%%%%%%%
\item Electrical and computer engineering and/or computer science: \vspace{-0.2cm}
	\begin{enumerate} \itemsep -2pt
	\item {\it TopCoder} coding and design contests: \vspace{-0.2cm}
		\begin{enumerate} \itemsep -2pt
		\item High School category
		\item \url{http://www.topcoder.com/}
		\end{enumerate}
	\item Student Cluster Competition (SCC): \vspace{-0.2cm}
		\begin{enumerate} \itemsep -2pt
		\item SCC is held at each (annual) SC conference, which is the International Conference for High Performance Computing, Networking, Storage, and Analysis. IEEE Computer Society and the Association for Computing Machinery are the sponsors for this conference.
		\item During SC10, teams consisting of six students, undergraduate and/or high school, will showcase the amazing power of clusters and the ability to utilize open source software to solve interesting and important problems. They will compete in real-time on the exhibit floor to run a workload of real-world applications on clusters of their own design while never exceeding the dictated power limit.
		\item During SC10 in New Orleans, teams will assemble, test and tune their machines and run the HPCC benchmarks until the starting bell rings on Monday night at the Exhibit Opening Gala where they will be given the competition data sets. In full view of conference attendees, teams will execute the prescribed workload while showing progress and science visualization output on large high-resolution displays in their areas. Teams race to correctly complete the greatest number of application runs during the competition period until the close of the exhibit floor on Wednesday evening.
		\item \url{http://sc10.supercomputing.org/?pg=studentcluster.html}
		\end{enumerate}
	\item Institute of Electrical and Electronics Engineers, IEEE: \vspace{-0.3cm}
		\begin{enumerate} \itemsep -2pt
		\item {\it IEEE Educational Activities} recommended resources: \url{http://www.ieee.org/education_careers/education/preuniversity/resources/index.html}
		\item Engineering Projects in Community Service (EPICS) in IEEE: \vspace{-0.2cm}
			\begin{enumerate} \itemsep -2pt
			\item High school students collaborate with college students in engineering projects to benefit the community
			\item \url{http://www.ieee.org/education_careers/education/preuniversity/epics_high.html}
			\end{enumerate}
		\item Talk given by John Cohn at the IEEE International Symposium on Circuits and Systems (ISCAS), May 18-21, 2008. The talk is titled, ``Kids these days. How we can inspire the next generation of Engineers and Scientists?'' See \url{http://ewh.ieee.org/soc/icss/IEEE-ISCAS-08-Tue-Keynote-JC/IEEE-ISCAS-08-Tue-Keynote-JC.HTML}. [ Alternatively, go to: IEEE Circuits and Systems Society, \url{http://www.ieee-cas.org/}: Select the ``Resources'' tab in the menu bar, and select the ``ISCAS Keynote Videos'' option. Click on the video link with the appropriate title. ]
		\end{enumerate}
	\item Association for Computing Machinery (ACM): \vspace{-0.2cm}
		\begin{enumerate} \itemsep -2pt
		\item Sanjeev Arora, Boaz Barak, and Luca Trevisan, ``Survey Papers and Essays,'' in {\it Theory Matters Wiki: Theoretical Computer Science (TCS) Advocacy Wiki}, SIGACT Committee for the Advancement of Theoretical Computer Science, ACM Special Interest Group on Algorithms and Computation Theory (SIGACT), Association for Computing Machinery, February 25, 2010. Available at: \url{http://theorymatters.org/pmwiki/pmwiki.php?n=Main.SurveyCollection}; last accessed on September 14, 2010.
		\end{enumerate}
	\item WGBH Educational Foundation: \vspace{-0.2cm}
		\begin{enumerate} \itemsep -2pt
		\item Dot Diva / New Image for Computing (NIC) initiative: \vspace{-0.1cm}
			\begin{enumerate} \itemsep -1pt
			\item \url{http://dotdiva.org/}
			\item Resource for parents and teachers: \url{http://dotdiva.org/parents.html}
			\end{enumerate}
		\end{enumerate}
	\item Silicon Valley StRUT: \vspace{-0.2cm}
		\begin{enumerate} \itemsep -2pt
		\item Students Recycling Used Technology, StRUT, Competition; StRUT Competition consists of: \vspace{-0.1cm}
			\begin{enumerate} \itemsep -1pt
			\item Disassemble and Reassemble A Computer 
			\item Create and Present a Powerpoint Presentation 
			\item Computer Parts Identification and Challenge Test  
			\item Team Quiz Bowl on Computer Technology and Related Subjects
			\item \url{http://www.svstrut.org/cms/content/section/1/5/}
			\item Teacher Resources: \url{http://www.svstrut.org/cms/component/option,com_weblinks/catid,11/Itemid,10/}
			\item [ Resources to Support ] Curriculum for Engineering and Computer Technology Education: \url{http://www.svstrut.org/cms/content/view/8/18/}
			\end{enumerate}
		\item \url{http://www.svstrut.org/cms/}
		\end{enumerate}
	\item Google Code Jam (programming contest): \url{http://code.google.com/codejam/} and \url{http://en.wikipedia.org/wiki/Google_Code_Jam}
	\item University of Illinois at Urbana-Champaign (UIUC): \vspace{-0.2cm}
		\begin{enumerate} \itemsep -2pt
		\item College of Engineering; Department of Computer Science: \vspace{-0.1cm}
			\begin{enumerate} \itemsep -1pt
			\item Outreach \& Diversity: \url{http://cs.illinois.edu/outreach}
			\item ChicTech: \url{http://cs.illinois.edu/outreach/chictech}
			\item Technical Ambassadors: \url{http://cs.illinois.edu/outreach/tac}
			\item Games4Girls: \url{http://cs.illinois.edu/outreach/games4girls}
			\item Workshops \& Camps: \url{http://cs.illinois.edu/outreach/k12}
			\item \url{http://cs.illinois.edu/outreach}
			\end{enumerate}
		\end{enumerate}
	\item Carnegie Mellon University: \vspace{-0.2cm}
		\begin{enumerate} \itemsep -2pt
		\item women@SCS School of Computer Science, Carnegie Mellon University: \vspace{-0.1cm}
			\begin{enumerate} \itemsep -1pt
			\item Papers: \url{http://women.cs.cmu.edu/Resources/Papers/}
			\item Alumnae Interviews / Profiles: \url{http://women.cs.cmu.edu/Who/Alumnae/alumInterviews.php}
			\item Job and Research Opportunities: \url{http://www.women.cs.cmu.edu/Resources/JobsResearch/}
			\item Career Advice: \url{http://women.cs.cmu.edu/Resources/JobsResearch/careeradvice.php}
			\item Other Sites: \url{http://www.women.cs.cmu.edu/Miscellaneous/Other/}
			\end{enumerate}
		\end{enumerate}
	\item {\it Quora}: \vspace{-0.2cm}
		\begin{enumerate} \itemsep -2pt
		\item ``If a 10-year-old wanted to start programming today, what language path would be the most valuable moving forward?'' Available online at: \url{http://www.quora.com/If-a-10-year-old-wanted-to-start-programming-today-what-language-path-would-be-the-most-valuable-moving-forward}; last accessed on November 23, 2010.
		\end{enumerate}
	\end{enumerate}
%%%%%%%%%%%%%%%%%%%%%%%%%%%%%%%%%%%%%%%%
%%%%%%%%%%%%%%%%%%%%%%%%%%%%%%%%%%%%%%%%
\item Engineering Education Service Center (EESC): \vspace{-0.3cm}
	\begin{enumerate} \itemsep -2pt
	\item Has lists of: \vspace{-0.2cm}
		\begin{enumerate} \itemsep -2pt
		\item Educational material: \vspace{-0.1cm}
			\begin{enumerate} \itemsep -1pt
			\item books
			\item DVDs
			\item resource kits for teachers
			\end{enumerate}
		\item engineering camps (for the summer in the United States): \url{http://www.engineeringedu.com/camps/}
		\item {\it Women in Engineering} programs at US engineering schools: \url{http://www.engineeringedu.com/wie.html}
		\item US engineering schools: \url{http://www.engineeringedu.com/engrschools.htm}
		\item competitions for youths, including high school students: \url{http://www.engineeringedu.com/competitions.html}
		\item online resources
		\item list of professional organizations in engineering (or engineering societies): \url{http://www.engineeringedu.com/soc1.html}
		\item scholarships: \url{http://www.engineeringedu.com/scholars.html}
		\end{enumerate}
	\item It has resources for K-12 students, and their teachers and parents. It also has information for girls who are seeking careers in engineering; in addition, it provides their parents and teachers with information to guide the girls.
	\item It runs a workshop (in the US) for mother-daughter pairs to encourage girls to pursue careers in engineering.
	\item \url{http://www.engineeringedu.com/}
	\end{enumerate}
\item TryNano.org: \vspace{-0.3cm}
	\begin{enumerate} \itemsep -2pt
	\item Information about educational opportunities and careers in nanotechnology and nanoscience
	\item \url{TryNano.org}
	\end{enumerate}
\item {\it Mathematical Association of America} (MAA): \vspace{-0.3cm}
	\begin{enumerate} \itemsep -2pt
	\item Middle/High School Students: \url{http://www.maa.org/students/middle_high/}
	\item Parents: \url{http://www.maa.org/students/Parents.html}
	\item MAA American Mathematics Competitions: \vspace{-0.2cm}
		\begin{enumerate} \itemsep -2pt
		\item {\it Students} [resources]. Available at: \url{http://amc.maa.org/a-activities/a4-for-students/s-index.shtml}; last accessed on September 2, 2010.
		\item It includes tips to help students do well in math contests and Olympiads, a reading list for students interested in mathematics, problems from past math contests and Olympiads, and other resources from the World Wide Web.
		\end{enumerate}
	\item {\it Fun Math Sites}. Available at: \url{http://www.maa.org/students/funsites.html}; last accessed on September 2, 2010.
	\item Special Interest Group on Mathematics and the Arts (SIGMAA-ARTS): Resources, see \url{http://myweb.cwpost.liu.edu/aburns/sigmaa-arts/resources.html}.
	\item Special Interest Group of the MAA on Quantitative Literacy (SIGMAA QL): \url{http://sigmaa.maa.org/ql/}
	\end{enumerate}
\item eGFI (Engineering, Go For It!): \vspace{-0.3cm}
	\begin{enumerate} \itemsep -2pt
	\item Provides information for students, parents, and teachers about educational pathways and careers in engineering
	\item \url{http://egfi-k12.org/}
	\end{enumerate}
\item {\it Sloan Career Cornerstone Center}: \vspace{-0.3cm}
	\begin{enumerate} \itemsep -2pt
	\item Career exploration resources in STEM (science, technology, engineering, mathematics, computing, and healthcare)
	\item \url{http://www.careercornerstone.org/}
	\end{enumerate}
\item {\it TryEngineering}: \vspace{-0.3cm}
	\begin{enumerate} \itemsep -2pt
	\item Career exploration resources for engineering
	\item \url{http://www.tryengineering.org/}
	\end{enumerate}
\item {\it Junior Engineering Technical Society, JETS}: \vspace{-0.3cm}
	\begin{enumerate} \itemsep -2pt
	\item Career exploration resources for engineering
	\item \url{http://www.jets.org/}
	\end{enumerate}
\item {\it American Society of Mechanical Engineers, ASME}: \vspace{-0.3cm}
	\begin{enumerate} \itemsep -2pt
	\item K-12 Student Resources: \url{http://www.asme.org/Communities/Students/K12/} and \url{http://www.asme.org/Education/PreCollege/EngineeringResources/}
	\item Engineering Camps: \url{http://www.asme.org/Communities/Students/K12/Camps.cfm}
	\end{enumerate}
\item BESTRobotics, Inc.: \vspace{-0.3cm}
	\begin{enumerate} \itemsep -2pt
	\item BEST (Boosting Engineering, Science, and Technology) competition: \vspace{-0.2cm}
		\begin{enumerate} \itemsep -2pt
		\item \url{http://best.eng.auburn.edu/}
		\item Hosted at Auburn University's Samuel Ginn College of Engineering
		\item BEST World Championship: \url{http://best.eng.auburn.edu/world-championship/}
		\end{enumerate}
	\end{enumerate}
\item {\it American Society of Civil Engineers, ASCE}: \vspace{-0.3cm}
	\begin{enumerate} \itemsep -2pt
	\item Outreach resource for K-12 students, and their parents and teachers
	\item \url{http://content.asce.org/asceville/index.html}
	\end{enumerate}
\item {\it Science.gov} (USA.gov for Science): Internship and Fellowship Opportunities in Science (for high school students); see \url{http://www.science.gov/internships/k-12.html}
\item {\it iTunes U}: \vspace{-0.3cm}
	\begin{enumerate} \itemsep -2pt
	\item {\it iTunes} is required to listen to or watch these lectures, talks, and presentations.
	\item Access {\it iTunes U} at: \url{http://deimos3.apple.com/indigo/main/main.html?v0=WWW-AMUS-ITUNESU070521-N48LX}
	\item WGBH's Teachers' Domain -- Boston's PBS Station: Video presentation on ``Engineering for the Red Planet''; see \url{http://deimos3.apple.com/WebObjects/Core.woa/Browse/wgbh.org.1416254059.01416254061.1416793683?i=1951581658}. Also, check out its video clip on ``Carbon Fiber Car of the Future''.
	\item {\it iTunes U} is a set of webcast and podcasts, where you can easily find audio and video clips for lectures, seminars, announcements, virtual tours, and so on. For example, some professors from schools like MIT or Berkeley will provide audio/video clips of their lectures on {\it iTunes U}.
	\item This can help in exploring different majors during the college application process, or before a college student declares her/his major(s). If a student is not sure if she/he wants to double major in deaf studies and linguistics, this student can check out some linguistics lectures from her/his (preferred) college/university, if it uses {\it iTunes U}, or those from other universities.
	\end{enumerate}
\item High School Ace's College Prep Guide: \url{http://highschoolace.com/ace/colleges.cfm}
\item {\it Dr. Sally Ride} (America�s first woman in space): \vspace{-0.3cm}
	\begin{enumerate} \itemsep -2pt
	\item {\it Sally Ride Science}'s resources for educators: \url{https://www.sallyridescience.com/for_educators}
	\item Sally Ride Science Educator Institutes (to educate K-12 teachers about science): \url{https://www.sallyridescience.com/for_educators/institutes}
	\item {\it Sally Ride Science Academy} helps teachers to increase their students' interest in science: \url{https://www.sallyridescience.com/academy}
	\item {\it Sally Ride Science}'s resources for teachers: \url{https://www.sallyridescience.com/resources}
	\item {\it Sally Ride Science Festivals} are events for girls from the $5^{th}$ grade to the $8^{th}$ grade: \url{https://www.sallyridescience.com/festivals}
	\item {\it Sally Ride Science Camps} are summer camps for girls from the $4^{th}$ grade to the $9^{th}$ grade: \url{http://www.sallyridecamps.com/}
	\item GRAIL MoonKAM: \vspace{-0.2cm}
		\begin{enumerate} \itemsep -2pt
		\item ``GRAIL MoonKAM (Moon Knowledge Acquired by Middle school students) is GRAIL's signature education and public outreach program.''
		\item ``GRAIL MoonKAM will engage middle schools across the country in the GRAIL mission and lunar exploration.''
		\item \url{https://www.grailmoonkam.com/}
		\end{enumerate}
	\item EarthKAM: \vspace{-0.2cm}
		\begin{enumerate} \itemsep -2pt
		\item EarthKAM (Earth Knowledge Acquired by Middle school students) is a NASA educational outreach program enabling students, teachers and the public to learn about Earth from the unique perspective of space.
		\item \url{https://earthkam.ucsd.edu/}
		\end{enumerate}
	\end{enumerate}
\item Andrew Rader Studios: \vspace{-0.3cm}
	\begin{enumerate} \itemsep -2pt
	\item Chem4Kids.com: \url{http://www.chem4kids.com/}
	\end{enumerate}
\item {\it American Association for the Advancement of Science, AAAS}: \vspace{-0.3cm}
	\begin{enumerate} \itemsep -2pt
	\item ENTRY POINT! for Students With Disabilities (in STEM): \url{http://www.aaas.org/careercenter/fellowships/} and \url{http://ehrweb.aaas.org/entrypoint/}
	\item AAAS Mass Media Science \& Engineering Fellows Program (for STEM grad students to intern in mass media companies): \url{http://www.aaas.org/programs/education/MassMedia/}
	\item Diversity Issues: \url{http://sciencecareers.sciencemag.org/career_magazine/diversity_issues/}
	\item Internships involving science and journalism, human rights, scientific freedom, responsibility, or law: \url{http://www.aaas.org/careercenter/} and \url{http://www.aaas.org/careercenter/internships/scienceminority.shtml} (AAAS Minority Science Writers Internship)
	\item Kinetic City: \url{http://www.kineticcity.com/}
	\end{enumerate}
\item {\it NASA} resources for students: \url{http://www.nasa.gov/audience/forstudents/index.html} and \url{http://www.nasa.gov/offices/education/programs/national/summer/education_resources/index.html} (NASA Summer of Innovation)
\item National Academy of Engineering, NAE: \vspace{-0.3cm}
	\begin{enumerate} \itemsep -2pt
	\item NAE Grand Challenges: \vspace{-0.2cm}
		\begin{enumerate} \itemsep -2pt
		\item Includes a list of NAE Grand Challenges, which indicate some of the challenges faced by people on a global scale that can be partially solved by engineers. This can be used to get children and youths to be excited about engineering.
		\item NAE Grand Challenges: \vspace{-0.1cm}
			\begin{enumerate} \itemsep -1pt
			\item Make solar energy economical
			\item Provide energy from fusion
			\item Develop carbon sequestration methods
			\item Manage the nitrogen cycle
			\item Provide access to clean water
			\item Restore and improve urban infrastructure
			\item Advance health informatics
			\item Engineer better medicines
			\item Reverse-engineer the brain
			\item Prevent nuclear terror
			\item Secure cyberspace
			\item Enhance virtual reality
			\item Advance personalized learning
			\item Engineer the tools of scientific discovery
			\end{enumerate}
		\item \url{http://www.engineeringchallenges.org/}
		\item NAE Grand Challenge K12 Partners Program: \vspace{-0.1cm}
			\begin{enumerate} \itemsep -1pt
			\item \url{http://www.grandchallengek12.org/about}
			\item 5-Part Make it Happen Plan: \url{http://www.grandchallengek12.org/5-part-plan}
			\end{enumerate}
		\end{enumerate}
	\item {\it National Academy of Engineering}'s technological literacy program for people (students, parents, and educators) to learn more about technology: \url{http://www.nae.edu/nae/techlithome.nsf}
	\item Greatest Engineering Achievements: \url{http://www.greatachievements.org/}
	\end{enumerate}
\item National Science Foundation: \vspace{-0.3cm}
	\begin{enumerate} \itemsep -2pt
	\item Broadening Participation in Computing (BPC): \vspace{-0.2cm}
		\begin{enumerate} \itemsep -2pt
		\item \url{http://www.bpcportal.org/}
		\item \url{http://www.bpcportal.org/bpc/shared/home.jhtml;jsessionid=0MIUYDR5U4ARXABAVRSSFEQ?_requestid=9445}
		\item \url{http://www.nsf.gov/funding/pgm_summ.jsp?pims_id=13510}
		\item \url{http://www.nsf.gov/funding/pgm_summ.jsp?pims_id=13510&org=NSF&sel_org=NSF&from=fund}
		\item ``Broadening Participation in Computing (BPC) is a NSF sponsored program with the goal of significantly increasing the number of underrepresented graduates in the computing disciplines, with an emphasis on women, persons with disabilities, and minorities (African Americans, Hispanics, American Indians, Alaska Natives, Native Hawaiians, and Pacific Islanders).''
		\item Broadening Participation in Computing Digital Library: \vspace{-0.1cm}
			\begin{enumerate} \itemsep -1pt
			\item \url{http://www.bpcportal.org/bpc/interdiscipline/dl_index.jhtml;jsessionid=ROYEHJV1UQYWNABAVRSSFEQ?comm=BPC}
			\item Includes resources for different target populations: \vspace{-0.1cm}
				\begin{itemize} \itemsep -1pt
				\item Women
				\item African Americans
				\item Hispanic Americans, or Latinas and Latinos
				\item People with disabilities
				\item Native Americans
				\end{itemize}
			\item It also includes resources for different topics, such as mentoring, recruitment, retention, and work-life balance.
			\end{enumerate}
		\item Alliances (other professional organizations): \url{http://www.bpcportal.org/bpc/comm/projects.jhtml}
		\end{enumerate}
	\item The National Science Digital Library (NSDL): \vspace{-0.2cm}
		\begin{enumerate} \itemsep -2pt
		\item \url{http://www.nsdl.org/} and \url{http://www.nsdl.org/browse/}
		\item ``The National Science Digital Library is a national network dedicated to advancing STEM teaching and learning for all learners, in both formal and informal settings, and the locus of activity for the National Science Foundation's National STEM Distributed Learning program.''
		\item Outreach materials: \vspace{-0.1cm}
			\begin{enumerate} \itemsep -1pt
			\item \url{http://www.nsdl.org/pd/?pager=materials}
			\item Has outreach materials for educators in K-12 and higher educational institutions.
			\end{enumerate}
		\item Resources for K-12 Teachers: \url{http://nsdl.org/resources_for/k12_teachers/}
		\item Resources for Librarians: \url{http://nsdl.org/resources_for/librarians/}
		\item Billingual Resources: \url{http://www.nsdlnetwork.org/collections/billingual-resources}
		\item NSDL on {\it iTunes U}: \url{http://www.nsdl.org/iTunesU/}
		\item Collections: \url{http://www.nsdl.org/browse/?subject=All}
		\item NSDL Pathways: \vspace{-0.1cm}
			\begin{enumerate} \itemsep -1pt
			\item \url{http://nsdl.org/about/?pager=pathways}
			\item ``Pathways are large projects that are aggregators and stewards of resources and services to broad categories of users---either discipline-focused (e.g. chemistry), or audience-focused (e.g. middle school educators), or resources of a specific type or format (e.g. multimedia content).''
			\item ``They are digital library portals developed and managed in partnership with organizations and institutions that have a history and expertise in serving their portal's target audiences.''
			\item ``They contribute metadata (descriptive information) about their resources to NSDL to make their resources searchable and discoverable via the NSDL.org portal, in addition to their own portals.''
			\end{enumerate}
		\item {\bf NSDL Science Literacy Maps}: \vspace{-0.1cm}
			\begin{enumerate} \itemsep -1pt
			\item \url{http://strandmaps.nsdl.org/}
			\item ``{\it NSDL Science Literacy Maps} are a tool for teachers and students to find resources that relate to specific science and math concepts. The maps illustrate connections between concepts as well as how concepts build upon one another across grade levels.''
			\end{enumerate}
		\item NSDL Professional Development: \url{http://www.nsdl.org/pd/}
		\item NSDL Technical Network Services: \vspace{-0.1cm}
			\begin{enumerate} \itemsep -1pt
			\item \url{http://www.nsdl.org/about/?pager=tns}
			\item \url{http://nsdlnetwork.org/}
			\item \url{http://nsdlnetwork.org/content/book/page/953/about-nsdl-technical-network-services}
			\end{enumerate}
		\item NSDL Resource Center: \url{http://nsdlnetwork.org/content/book/951/page/954/about-nsdl-resource-center}
		\end{enumerate}
	\end{enumerate}
\item {\it American Chemical Society} Science for Kids program (for students in K-12): \url{http://portal.acs.org/portal/acs/corg/content?_nfpb=true&_pageLabel=PP_TRANSITIONMAIN&node_id=878&use_sec=false&sec_url_var=region1&__uuid=984d4ee7-4214-4d35-9899-bc2f91dee58b}
\item {\it California Digital Educator Consortium}, ``Digital Educator,'' Digital Learning Center: \url{http://www.digitaleducator.com/}
\item Kenny Felder, ``Selected Other Educational Sites on the Web''. Available at: \url{http://www4.ncsu.edu/unity/lockers/users/f/felder/public/kenny/edulinks.html}; last accessed on August 28, 2010.
\item FHSST (Free High School Science Texts); free textbooks for grades 10-12 in Physics, Chemistry, and Mathematics. Available at: \url{http://www.fhsst.org/}; last accessed on August 28, 2010.
\item John Baez, {\it Usenet Physics FAQ}, Department of Mathematics, University of California, Riverside, September 2009. Available at: \url{http://math.ucr.edu/home/baez/physics/}; last accessed on August 28, 2010.
\item {\it American Society for Engineering Education}: \vspace{-0.3cm}
	\begin{enumerate} \itemsep -2pt
	\item Science and Engineering Apprenticeship Program (SEAP): \vspace{-0.2cm}
		\begin{enumerate} \itemsep -2pt
		\item ``The Science and Engineering Apprenticeship Program (SEAP) provides an opportunity for students to participate in research at a Department of Navy (DoN) laboratory during the summer.''
		\item ``The goals of SEAP are to encourage participating students to pursue science and engineering careers, to further their education via mentoring by laboratory personnel and their participation in research, and to make them aware of DoN Research and technology efforts, which can lead to employment within the DoN.''
		\item ``High school students who have completed at least Grade 9. A graduating senior is eligible to apply.''
		\item ``Must be 16 years of age for most laboratories. Some laboratories may accept a 15 year old applicant. Please check individual lab description for more details.''
		\item ``Applicants must be US citizens and participation by Permanent Resident Aliens is limited. Please check individual lab descriptions for participation of Permanent Resident Aliens.''
		\item \url{http://seap.asee.org/}
		\end{enumerate}
	\end{enumerate}
\item robots.net, {\it Robot Competitions} (list of robot competitions and contests) : \url{http://robots.net/rcfaq.html}
\item International Council on Systems Engineering (INCOSE): \vspace{-0.3cm}
	\begin{enumerate} \itemsep -2pt
	\item Careers in Systems Engineering: \url{http://www.incose.org/educationcareers/careersinsystemseng.aspx}
	\item Frequently Asked Questions for Students [about Systems Engineering]: \url{http://www.incose.org/educationcareers/faqsforstudents.aspx}
	\item What is Systems Engineering?: \url{http://www.incose.org/practice/whatissystemseng.aspx}
	\end{enumerate}
\item {\it National Society of Professional Engineers}: \vspace{-0.3cm}
	\begin{enumerate} \itemsep -2pt
	\item A Sightseer's Guide to Engineering: \url{http://www.engineeringsights.org/}
	\end{enumerate}
\item {\it Engineers Dedicated to a Better Tomorrow (a.k.a., DedicatedEngineers)}: \vspace{-0.3cm}
	\begin{enumerate} \itemsep -2pt
	\item The ``K-12 Crowd'' (Students, Teachers, Guidance Counselors and Parents): \url{http://www.dedicatedengineers.org/intro_for_K-12.htm}
	\item \url{http://www.dedicatedengineers.org/}
	\end{enumerate}
\item National Engineers Week Foundation: \vspace{-0.3cm}
	\begin{enumerate} \itemsep -2pt
	\item Discover Engineering: \url{http://www.discoverengineering.org/}
	\item Introduce A Girl to Engineering: \url{http://www.eweek.org/EngineersWeek/IntroduceAGirl.aspx}
	\item All About Engineering: \url{http://www.eweek.org/AboutEngineering/AboutEngineering.aspx}
	\end{enumerate}
\item University of California: \vspace{-0.3cm}
	\begin{enumerate} \itemsep -2pt
	\item The Coalition For Science After School: \vspace{-0.2cm}
		\begin{enumerate} \itemsep -2pt
		\item \url{http://afterschoolscience.org/}
		\item ``Promoting high-quality afterschool science'' ... ``The Coalition for Science After School envisions the day when young people from all backgrounds have access to high-quality science, technology, engineering and mathematics (STEM) learning beyond the classroom.''
		\item Tools for advocates--Championing afterschool science: \url{http://afterschoolscience.org/tools/}
		\item Program resources--Enhancing the quality of afterschool opportunities: \url{http://afterschoolscience.org/resources/}
		\item The National After School Science Directory: \vspace{-0.1cm}
			\begin{enumerate} \itemsep -1pt
			\item \url{http://afterschoolscience.org/directory/}
			\item ``The National After School Science Directory is a searchable database designed to increase access to high-quality science, technology, engineering and math (STEM) education beyond the classroom for youth and families across the nation. The Directory houses thousands of STEM opportunities, submitted by science centers, museums, schools and other youth-serving organizations. Search our Directory to view opportunities to connect the America's youth to high-quality STEM learning experiences.''
			\end{enumerate}
		\item Become an advocate: \url{http://afterschoolscience.org/tools/advocate.php}
		\item Funders (funding organizations/agencies): \url{http://afterschoolscience.org/tools/funders.php}
		\end{enumerate}
	\end{enumerate}
\item Harvey Mudd College: \vspace{-0.3cm}
	\begin{enumerate} \itemsep -2pt
	\item Francis Edward Su, {\it Math Fun Facts!}, Department of Mathematics, Harvey Mudd College: \url{http://www.math.hmc.edu/funfacts/}
	\end{enumerate}
\item Clay Mathematics Institute: \vspace{-0.3cm}
	\begin{enumerate} \itemsep -2pt
	\item Program in Mathematics for Young Scientists, PROMYS: \vspace{-0.2cm}
		\begin{enumerate} \itemsep -2pt
		\item \url{http://www.claymath.org/programs/outreach/PROMYS/}
		\item \url{http://math.bu.edu/people/promys/}
		\item \url{http://www.promys.org/}
		\end{enumerate}
	\item Ross Program (for pre-college students): \vspace{-0.2cm}
		\begin{enumerate} \itemsep -2pt
		\item \url{http://www.claymath.org/programs/outreach/ross/}
		\item \url{http://www.math.ohio-state.edu/ross/}
		\end{enumerate}
	\item CMI Summer Schools: \url{http://www.claymath.org/programs/summer_school/}
	\end{enumerate}
\item Consortium for Ocean Leadership: \vspace{-0.3cm}
	\begin{enumerate} \itemsep -2pt
	\item Oceans of Opportunity (for African American students in K-12, and colleges and universities -- includes undergraduates and grad students): \url{http://www.oceanleadership.org/education/diversity/oceans-of-opportunity/}
	\item The JOIDES Resolution (The JR) scientific research vessel [ Deep Earth Academy ]: \vspace{-0.2cm}
		\begin{enumerate} \itemsep -2pt
		\item Fun \& Games: \url{http://joidesresolution.org/node/53}
		\item Discovery Center: \url{http://joidesresolution.org/node/44}
		\item Just for Kids Blog: \url{http://joidesresolution.org/node/366}
		\end{enumerate}
	\item National Ocean Sciences Bowl (high school academic competition that provides a forum for talented students to test their knowledge of the marine sciences including biology, chemistry, physics, and geology): \vspace{-0.2cm}
		\begin{enumerate} \itemsep -2pt
		\item \url{http://www.nosb.org/}
		\item Career Resources: \url{http://www.nosb.org/ocean-careers/career-resources/}
		\end{enumerate}
	\item Integrated Ocean Drilling Program (IODP), IODP United States Implementing Organization (IODP-USIO): \vspace{-0.2cm}
		\begin{enumerate} \itemsep -2pt
		\item U.S.-sponsored Teacher at Sea Program (for US teachers to participate in seagoing research experiences aboard the JOIDES Resolution): \url{http://www.iodp-usio.org/Education/TAS.html}
		\end{enumerate}
	\item Careers: \url{http://www.oceanleadership.org/education/deep-earth-academy/students/careers/}
	\end{enumerate}
\item The Oceanography Society: \vspace{-0.3cm}
	\begin{enumerate} \itemsep -2pt
	\item Careers in Oceanography: Profiles, \url{http://www.tos.org/resources/career_profiles.html}
	\item Links [includes links to educational material for students in K-12]: \url{http://www.tos.org/resources/links.html}
	\end{enumerate}
\item American Geophysical Union: \vspace{-0.3cm}
	\begin{enumerate} \itemsep -2pt
	\item Bright Students Training as Research Scientists (Bright STaRS): \vspace{-0.2cm}
		\begin{enumerate} \itemsep -2pt
		\item \url{http://www.agu.org/education/diversity_programs/bstars.shtml}
		\item ``High school students participating in after-school and summer research experiences in the Earth and space sciences are invited to participate in the AGU Bright STaRS program. The Bright STaRS program provides a dedicated forum for $\sim$50 students to present their own research results to the scientific community and learn about exciting research, education, and career opportunities in the geosciences.''
		\end{enumerate}
	\end{enumerate}
\item American Geological Institute, AGI: \vspace{-0.3cm}
	\begin{enumerate} \itemsep -2pt
	\item AGI Education Department: \url{http://www.agiweb.org/geoeducation.html}
	\end{enumerate}
\item Society for Science \& the Public (SSP): \vspace{-0.3cm}
	\begin{enumerate} \itemsep -2pt
	\item Intel International Science \& Engineering Fair (Intel ISEF), which is a pre-college science competition: \url{http://www.societyforscience.org/isef/}
	\item Broadcom MASTERS\texttrademark\ competition (which stands for Broadcom Math, Applied Science, Technology and Engineering for Rising Stars): \vspace{-0.2cm}
		\begin{enumerate} \itemsep -2pt
		\item Is a U.S. ``national science, technology, engineering, and math competition for America's $6^{th}$, $7^{th}$, and $8^{th}$ graders.''
		\item \url{http://www.societyforscience.org/masters} or \url{http://www.broadcomfoundation.org/masters/}
		\end{enumerate} 
	\item Science resources: \url{http://www.societyforscience.org/resources}
	\item Science News: \url{http://www.sciencenews.org/}
	\item Science News for Kids (for ``children of ages 9-14, their teachers and their parents''): \url{http://www.societyforscience.org/sciencenewsforkids} and \url{http://www.sciencenewsforkids.org/}
	\end{enumerate}
\item Institute for Operations Research and the Management Sciences (INFORMS): \vspace{-0.3cm}
	\begin{enumerate} \itemsep -2pt
	\item Operations Research: The Science of Better, \url{http://www.scienceofbetter.org/}
	\end{enumerate}
\item Technion - Israel Institute of Technology: \vspace{-0.3cm}
	\begin{enumerate} \itemsep -2pt
	\item SciTech - the summer camp for talented students ($11^{th}$ and $12^{th}$ graders from all over the world): \url{http://www.scitech.technion.ac.il/}
	\end{enumerate}
\item USA Science \& Engineering Festival: \url{http://www.usasciencefestival.org/}
\item Girl Scouts: \vspace{-0.3cm}
	\begin{enumerate} \itemsep -2pt
	\item Girl Scouts of Western New York: \vspace{-0.2cm}
		\begin{enumerate} \itemsep -2pt
		\item STEM Resource Guide: \url{http://www.gswny.org/Data/Documents/STEM%2520Resource%2520Guide%25202010-Oct-11.pdf}
		\item Also, see \url{http://www.gswny.org/Programs/Awards/Gold/}; scroll to the bottom of the page and look under the subsection heading, ``Tell Us About Your Gold Award Project''
		\end{enumerate}
	\item Science, Technology, Engineering and Math (STEM): \url{http://www.girlscouts.org/program/program_opportunities/science/}
	\end{enumerate}
\item American Museum of Science and Energy (AMSE): \vspace{-0.3cm}
	\begin{enumerate} \itemsep -2pt
	\item \url{http://www.amse.org/}
	\item Owned by the US Department of Energy, and managed under Oak Ridge National Laboratory
	\item Educators: \url{http://www.amse.org/content.aspx?article=1140&parent=30}
	\item Educational Programs: \url{http://www.amse.org/content.aspx?article=1139&parent=30}
	\item Home school programs: \url{http://www.amse.org/content.aspx?article=1169&parent=30}
	\item Online resources: \url{http://www.amse.org/content.aspx?article=1170&parent=30}
	\end{enumerate}
\item Center for Energy Workforce Development (CEWD): \vspace{-0.3cm}
	\begin{enumerate} \itemsep -2pt
	\item Teachers and guidance counselors: \vspace{-0.2cm}
		\begin{enumerate} \itemsep -2pt
		\item \url{http://www.cewd.org/educators_index.asp}
		\item Lesson plans for teachers: \url{http://www.cewd.org/educators_lessonplans.asp}
		\end{enumerate}
	\item Parents: \url{http://www.cewd.org/parents_index.asp}
	\end{enumerate}
\item TryScience: \url{http://tryscience.net/tryscinetmain.nsf/Welcome?OpenPage}
\item The Dana Foundation: \vspace{-0.3cm}
	\begin{enumerate} \itemsep -2pt
	\item Brainy Kids: \vspace{-0.2cm}
		\begin{enumerate} \itemsep -2pt
		\item \url{http://www.dana.org/resources/brainykids/}
		\item Fun: \vspace{-0.1cm}
			\begin{enumerate} \itemsep -1pt
			\item \url{http://dana.org/resources/brainykids/detail.aspx?folder_id=104}
			\item Has interactive online games, activities, and fun quizzes on: \vspace{-0.1cm}
				\begin{itemize} \itemsep -1pt
				\item biology
				\item health
				\item neuroscience
				\item astronomy
				\item chemistry
				\item ecology
				\end{itemize}
			\end{enumerate}
		\item The Lab: \vspace{-0.1cm}
			\begin{enumerate} \itemsep -1pt
			\item \url{http://dana.org/resources/brainykids/detail.aspx?folder_id=106}
			\item Has maps of the brain, virtual dissections, resources for science fairs, and virtual microscopes
			\end{enumerate}
		\item Lesson Plans: \vspace{-0.1cm}
			\begin{enumerate} \itemsep -1pt
			\item \url{http://dana.org/resources/brainykids/detail.aspx?folder_id=108}
			\item Includes resources that cover the history of science and technology, lesson plans for K-12 science teachers, and science news for youths.
			\end{enumerate}
		\item The Mindboggling Workbook: \vspace{-0.1cm}
			\begin{enumerate} \itemsep -1pt
			\item \url{http://www.dana.org/uploadedFiles/The_Dana_Alliances/mindboggling_workbook.pdf}
			\item ``A fun-filled activity book about the brain for children in grades K-3 (ages 5-9). Provides an introduction to how the brain works, what the brain does, its importance, and how to take care of it.''
			\end{enumerate}
		\end{enumerate}
	\end{enumerate}
\item University of New Mexico: \vspace{-0.3cm}
	\begin{enumerate} \itemsep -2pt
	\item Department of Mathematics and Statistics: \vspace{-0.2cm}
		\begin{enumerate} \itemsep -2pt
		\item UNM - PNM Statewide Mathematics Contest (sponsored by the PNM Foundation): \url{http://mathcontest.unm.edu/}
		\end{enumerate}
	\end{enumerate}
\item Center for Energy Workforce (CEWD): \vspace{-0.3cm}
	\begin{enumerate} \itemsep -2pt
	\item Get Into Energy: \vspace{-0.2cm}
		\begin{enumerate} \itemsep -2pt
		\item \url{http://www.getintoenergy.com/index.asp} and \url{http://www.getintoenergy.com/careers.asp}
		\item Fun educational resources for students: \url{http://www.getintoenergy.com/students.asp}
		\item Career Quiz: \vspace{-0.1cm}
			\begin{enumerate} \itemsep -1pt
			\item \url{http://www.getintoenergy.com/search/careerquizj.asp}
			\item Help you find out more about career options in the energy field
			\end{enumerate}
		\item Career Resources: \vspace{-0.1cm}
			\begin{enumerate} \itemsep -1pt
			\item \url{http://www.getintoenergy.com/careerresources.asp}
			\item Has information on: \vspace{-0.1cm}
				\begin{itemize} \itemsep -1pt
				\item Training Programs (technical schools and colleges)
				\item Work-based Programs (apprenticeships and internships)
				\item Featured Employers
				\end{itemize}
			\end{enumerate}
		\item Skills Needed in the Energy Field: \vspace{-0.1cm}
			\begin{enumerate} \itemsep -1pt
			\item \url{http://www.getintoenergy.com/skills.asp}
			\item List skills for different kinds of jobs in the energy field
			\end{enumerate}
		\item Information for parents: \url{http://www.getintoenergy.com/Parents.asp}
		\item Information for teachers and guidance counselors: \url{http://www.getintoenergy.com/Educators.asp}
		\end{enumerate}
	\end{enumerate}
\item University of Utah: \vspace{-0.3cm}
	\begin{enumerate} \itemsep -2pt
	\item Department of Electrical and Computer Engineering: \vspace{-0.2cm}
		\begin{enumerate} \itemsep -2pt
		\item Prof. Cynthia Furse: \vspace{-0.1cm}
			\begin{enumerate} \itemsep -1pt
			\item Cynthia Furse, {\it K-12 Engineering Outreach}, August 2007. Available online at: \url{http://www.ece.utah.edu/~cfurse/K12.html}; last accessed on December 10, 2010.
			\item Cynthia Furse, {\it U Dream. U Design. U Create.}, Department of Electrical and Computer Engineering, University of Utah. Available online at: \url{http://www.ece.utah.edu/~cfurse/NSF/}; last accessed on December 10, 2010.
			\end{enumerate}
		\end{enumerate}
	\end{enumerate}
\item Society for Industrial and Applied Mathematics: \vspace{-0.3cm}
	\begin{enumerate} \itemsep -2pt
	\item Public Awareness: \vspace{-0.2cm}
		\begin{enumerate} \itemsep -2pt
		\item Math Competitions, \url{http://www.siam.org/publicawareness/competitions.php}
		\item Moody's Mega Math Challenge (M3 Challenge) is an applied mathematics competition for high school students. Available online at: \url{http://m3challenge.siam.org/}; last accessed on December 13, 2010.
		\item {\it Math Matters, Apply It!}: \url{http://www.siam.org/careers/matters.php}
		\item Nuggets: \url{http://www.siam.org/publicawareness/nuggets.php}
		\end{enumerate}
	\item Society for Industrial and Applied Mathematics, ``Unveiling Why Do Math,'' May 27, 2010. Available online at: \url{http://www.siam.org/about/news-siam.php?id=1741}; last accessed on December 13, 2010.
	\end{enumerate}
\item International Federation of Operational Research Societies (IFORS): \vspace{-0.3cm}
	\begin{enumerate} \itemsep -2pt
	\item Association of European Operational Research Societies (EURO): \vspace{-0.2cm}
		\begin{enumerate} \itemsep -2pt
		\item {\it What is Operational Research?}: \url{http://www.euro-online.org/display.php?pageid=197&}
		\item Applications of OR in music, literature, and aesthetics: \url{http://www.euro-online.org/display.php?pageid=211&}
		\item 24 Hours Operations Research: \url{http://www.24hor.org/}
		\item Branding OR: \url{http://www.euro-online.org/display.php?pageid=198&}
		\end{enumerate}
	\end{enumerate}
\item American Institute of Aeronautics and Astronautics (AIAA): \vspace{-0.3cm}
	\begin{enumerate} \itemsep -2pt
	\item Students \& Educators: \url{http://www.aiaa.org/content.cfm?pageid=5}
	\item Ask An Engineer: \url{http://www.aiaa.org/content.cfm?pageid=214}
	\item Kid's Place: \vspace{-0.2cm}
		\begin{enumerate} \itemsep -2pt
		\item \url{http://www.aiaa.org/content.cfm?pageid=473}
		\item Enjoy games, puzzles, fun experiments, teen-recommended books and movies, and more.
		\end{enumerate}
	\item History of Flight Timeline: \url{http://www.aiaa.org/content.cfm?pageid=260}
	\item Ask Polaris: \vspace{-0.2cm}
		\begin{enumerate} \itemsep -2pt
		\item \url{http://www.askpolaris.org/}
		\item Resource for career exploration in aerospace engineering and related fields
		\end{enumerate}
	\end{enumerate}
\item Massachusetts Institute of Technology: \vspace{-0.3cm}
	\begin{enumerate} \itemsep -2pt
	\item MIT School of Engineering: \vspace{-0.2cm}
		\begin{enumerate} \itemsep -2pt
		\item Lemelson-MIT Program: \vspace{-0.1cm}
			\begin{enumerate} \itemsep -1pt
			\item \url{http://web.mit.edu/invent/}
			\item Inventor's Handbook: \url{http://web.mit.edu/invent/h-main.html}
			\item Games \& Trivia; \url{http://web.mit.edu/invent/g-main.html}
			\item Links \& Resources: \url{http://web.mit.edu/invent/r-main.html}
			\end{enumerate}
		\end{enumerate}
	\end{enumerate}
\item BT Group plc: \vspace{-0.3cm}
	\begin{enumerate} \itemsep -2pt
	\item British Telecommunications plc (BT): \vspace{-0.2cm}
		\begin{enumerate} \itemsep -2pt
		\item BT Young Scientist \& Technology Exhibition: \vspace{-0.1cm}
			\begin{enumerate} \itemsep -1pt
			\item \url{http://www.btyoungscientist.com/}
			\item \url{http://www.btyoungscientist.com/all-you-need-to-know/}
			\item Science and technology fair for high/secondary school students in Ireland
			\end{enumerate}
		\end{enumerate}
	\end{enumerate}
\item NHS Medical Careers: \vspace{-0.3cm}
	\begin{enumerate} \itemsep -2pt
	\item \url{http://www.medicalcareers.nhs.uk/Default.aspx}
	\item Provides information about careers in medicine for prospective medical students, medical students, medical school graduates (or young medical professionals), (medical speciality) trainers, and medical specialists.
	\end{enumerate}
\item British Science Association: \vspace{-0.3cm}
	\begin{enumerate} \itemsep -2pt
	\item British Science Festival: \vspace{-0.2cm}
		\begin{enumerate} \itemsep -2pt
		\item \url{http://www.britishscienceassociation.org/web/BritishScienceFestival/AboutFestival/index.htm}
		\item Festival Student Bursaries: \url{http://www.britishscienceassociation.org/web/BritishScienceFestival/StudentBursaries/index.htm}
		\end{enumerate}
	\item National Science \& Engineering Week: \url{http://www.britishscienceassociation.org/web/NSEW/index.htm}
	\item Clubs, CREST Awards and Fairs (programs and activities for children and youth, 5-19 years of age): \url{http://www.britishscienceassociation.org/web/ccaf/index.htm}
	\item National Science \& Engineering Competition: \url{http://www.britishscienceassociation.org/web/NSEC/index.htm} and \url{http://www.thebigbangfair.co.uk/nsec/}
	\end{enumerate}
\item Research Councils UK (RCUK): \vspace{-0.3cm}
	\begin{enumerate} \itemsep -2pt
	\item \url{http://www.rcuk.ac.uk/per/Pages/Schools.aspx}
	\item Schoolscience: \vspace{-0.2cm}
		\begin{enumerate} \itemsep -2pt
		\item \url{http://www.schoolscience.co.uk/}
		\item For students and educators in K-12 to enrich the learning experiences of science topics, and help students connect classroom material to the real world.
		\item Teacher Zone - professional resources for teachers: \url{http://www.schoolscience.co.uk/teacher_zone.cfm}
		\item Interactive Learning Resources: \url{http://www.schoolscience.co.uk/interactives.cfm}
		\item Free Resources: \url{http://www.schoolscience.co.uk/freebies.cfm}
		\item Competitions: \url{http://www.schoolscience.co.uk/competitions.cfm}
		\item Research focus: \url{http://www.schoolscience.co.uk/research_focus.cfm}
		\item Resources on the World Wide Web: \url{http://www.schoolscience.co.uk/sciencelink.cfm}
		\end{enumerate}
	\item Researchers in Residence (RinR): \vspace{-0.2cm}
		\begin{enumerate} \itemsep -2pt
		\item \url{http://www.researchersinresidence.ac.uk/cms/schools-colleges/}
		\item For students in middle and high schools to job shadow (observe first-hand) a Ph.D. student or postdoctoral researcher in her/his research activities for up to a week, so that students can learn what doing research in her/his research area is like. In addition, the researcher would explain in laypeople's terms what her/his research is about. It can be considered as an externship program.
		\end{enumerate}
	\item Nuffield Bursaries: \vspace{-0.2cm}
		\begin{enumerate} \itemsep -2pt
		\item \url{http://www.nuffieldfoundation.org/capacity-building}
		\item \url{http://www.nuffieldfoundation.org/science-bursaries-schools-and-colleges}
		\item For high school juniors/seniors to pursue a research internship in science and engineering.
		\end{enumerate}
	\item CREST (Creativity in Science and Technology): \vspace{-0.2cm}
		\begin{enumerate} \itemsep -2pt
		\item \url{http://www.britishscienceassociation.org/web/ccaf/CREST/index.htm}
		\item Program to help students get engaged in a science or engineering project, where they learn how to solve real problems in science or engineering.
		\end{enumerate}
	\end{enumerate}
\item Nuffield Foundation: \vspace{-0.3cm}
	\begin{enumerate} \itemsep -2pt
	\item Science bursaries for schools and colleges: \url{http://www.nuffieldfoundation.org/science-bursaries-schools-and-colleges}
	\item Students: \url{http://www.nuffieldfoundation.org/students}
	\item Twenty First Century Science: \vspace{-0.2cm}
		\begin{enumerate} \itemsep -2pt
		\item \url{http://www.21stcenturyscience.org/}
		\item ``Twenty First Century Science is a set of GCSE science courses giving all 14-16-year-olds a worthwhile and inspiring experience of science. The strength of the programme is that it meets the needs, through flexible options, of those who will go on to be professional scientists and of those who will not.''
		\item The Courses: \url{http://www.21stcenturyscience.org/the-courses/}
		\item Assessment overview: \url{http://www.21stcenturyscience.org/assess/}
		\item Teaching resources: \url{http://www.21stcenturyscience.org/resources/}
		\end{enumerate}
	\item Science in Society: \vspace{-0.2cm}
		\begin{enumerate} \itemsep -2pt
		\item \url{http://www.scienceinsocietyadvanced.org/}
		\item ``Science in Society is an interesting and topical GCE advanced level course. It aims to develop the knowledge and skills that are needed for students to understand how science works, analyse contemporary issues involving science and technology and communicate their scientific appreciation and understanding to others.''
		\end{enumerate}
	\item Parents: \url{http://www.nuffieldfoundation.org/parents}
	\item Education: \url{http://www.nuffieldfoundation.org/education}
	\item Teachers (has excellent resources for science and mathematics): \url{http://www.nuffieldfoundation.org/teachers}
	\item Capacity building: \url{http://www.nuffieldfoundation.org/capacity-building}
	\end{enumerate}
\item The Story of Stuff Project (by Annie Leonard): \vspace{-0.3cm}
	\begin{enumerate} \itemsep -2pt
	\item \url{http://www.storyofstuff.com/}
	\item ``The Story of Stuff Project was created by Annie Leonard to leverage and extend the film's impact. We amplify public discourse on a series of environmental, social and economic concerns and facilitate the growing Story of Stuff community's involvement in strategic efforts to build a more sustainable and just world.''
	\item Resources: \vspace{-0.2cm}
		\begin{enumerate} \itemsep -2pt
		\item \url{http://www.storyofstuff.com/resources.php}
		\item The Story of Stuff Project PDFs: \url{http://www.storyofstuff.com/dl-pdfs.php}
		\item Teaching Tools: \url{http://www.storyofstuff.com/teach.php}
		\item More About Stuff: \url{http://www.storyofstuff.com/aboutstuff.php}
		\item Recommended Reading \& Bibliography: \url{http://www.storyofstuff.com/reading.php}
		\item Get Involved: \url{http://www.storyofstuff.com/getinvolved.php}
		\item Curricula: \url{http://storyofstuff.org/curricula.php}
		\end{enumerate}
	\end{enumerate}
\item Facing the Future: \vspace{-0.3cm}
	\begin{enumerate} \itemsep -2pt
	\item \url{http://www.facingthefuture.org/}
	\item ``{\it Facing the Future} engages students in learning by making academics relevant to their lives. We empower students to think critically, develop a global perspective, and participate in positive solutions for a sustainable future.''
	\item Curriculum Alignment with Education Standards: \url{http://www.facingthefuture.org/Curriculum/AlignmentwithEducationStandards/tabid/116/Default.aspx}
	\item Global Sustainability Curriculum Finder: \url{http://www.facingthefuture.org/Curriculum/FindCurriculumthatisRightforYou/tabid/68/Default.aspx}
	\item Download FREE Global Issues and Sustainability Curriculum: \url{http://www.facingthefuture.org/Curriculum/DownloadFreeCurriculum/tabid/114/Default.aspx}
	\item Classroom Examples: How Engaging Curriculum Can Help Address Classroom Challenges, \url{http://www.facingthefuture.org/ForEducators/ClassroomExamples/tabid/213/Default.aspx}
	\item Our Impact on Student Achievement: \url{http://www.facingthefuture.org/ForEducators/OurImpactonStudentAchievement/tabid/73/Default.aspx}
	\item Action Project Database: \url{http://www.facingthefuture.org/ServiceLearning/ActionProjectDatabase/tabid/94/Default.aspx}
	\item Service Learning Examples: \url{http://www.facingthefuture.org/ServiceLearning/ExamplesofStudentsTakingAction/tabid/147/Default.aspx}
	\item Curriculum: \url{http://www.facingthefuture.org/Curriculum/CurriculumHome/tabid/113/Default.aspx}
	\end{enumerate}
\item U.S. Department of Energy: \vspace{-0.3cm}
	\begin{enumerate} \itemsep -2pt
	\item Office of Science: \vspace{-0.2cm}
		\begin{enumerate} \itemsep -2pt
		\item U.S. Department of Energy (DOE) National Science Bowl\textregistered: \vspace{-0.1cm}
			\begin{enumerate} \itemsep -1pt
			\item \url{http://www.scied.science.doe.gov/nsb/default.htm}
			\item ``The U.S. Department of Energy (DOE) National Science Bowl\textregistered\ is a nationwide academic competition that tests students' knowledge in all areas of science. High school and middle school students are quizzed in a fast paced question-and-answer format similar to Jeopardy. Competing teams from diverse backgrounds are comprised of four students, one alternate, and a teacher who serves as an advisor and coach.''
			\end{enumerate}
		\item Argonne National Laboratory: \vspace{-0.1cm}
			\begin{enumerate} \itemsep -1pt
			\item Division of Educational Programs: \vspace{-0.1cm}
				\begin{itemize} \itemsep -1pt
				\item Newton BBS Ask A Scientist: \url{http://www.newton.dep.anl.gov/aas.htm}
				\end{itemize}
			\end{enumerate}
		\end{enumerate}
	\item Office of Energy Efficiency and Renewable Energy (EERE): \vspace{-0.2cm}
		\begin{enumerate} \itemsep -2pt
		\item Kids Saving Energy: \vspace{-0.1cm}
			\begin{enumerate} \itemsep -1pt
			\item \url{http://www.eere.energy.gov/kids/index.html}
			\item K-12 Lesson Plans \& Activities: \url{http://www1.eere.energy.gov/education/lessonplans/}
			\item Energy Savers: \url{http://www.energysavers.gov/}
			\item Games and activities: \url{http://www.eere.energy.gov/kids/games.html}
			\item Smart home: \url{http://www.eere.energy.gov/kids/smart_home.html}
			\item About renewable energy: \url{http://www.eere.energy.gov/kids/renergy.html}
			\end{enumerate}
		\end{enumerate}
	\item Contest \& Competitions: \url{http://www.energy.gov/contests&competitions.htm}
	\end{enumerate}
\item United States Department of Defense (DoD): \vspace{-0.3cm}
	\begin{enumerate} \itemsep -2pt
	\item National Defense Education Program; Defense Advanced Research Projects Agency (DARPA): \vspace{-0.2cm}
		\begin{enumerate} \itemsep -2pt
		\item Resource for Students: \url{http://www.ndep.us/GetInvoStu.aspx}
		\item Resource for Educators: \url{http://www.ndep.us/GetInvoTea.aspx}
		\end{enumerate}
	\end{enumerate}
\item Project Lead The Way: \vspace{-0.3cm}
	\begin{enumerate} \itemsep -2pt
	\item \url{http://www.pltw.org/}
	\item Getting started: \url{http://www.pltw.org/getting-started/getting-started}
	\item Program support: \url{http://www.pltw.org/program-support/program-support}
	\item Grants available to schools and teachers: \url{http://www.pltw.org/pltw-in-the-news/grants-available-schools-teachers-and-classrooms}
	\item Students: \url{http://www.pltw.org/students/students}
	\item Educators and Administrators: \url{http://www.pltw.org/educators-administrators/educators-administrators-overview}
	\item Parents: \url{http://www.pltw.org/parents/parents}
	\end{enumerate}
\item National Science Teachers Association: \vspace{-0.3cm}
	\begin{enumerate} \itemsep -2pt
	\item \url{http://www.exploravision.org/}
	\item Science competition for K-12 students
	\end{enumerate}
\item American Mathematical Society: \vspace{-0.3cm}
	\begin{enumerate} \itemsep -2pt
	\item Some career resources for mathematics: \url{http://e-math.ams.org/samplings/samplings}
	\end{enumerate}
\item American Institute of Physics (AIP): \vspace{-0.3cm}
	\begin{enumerate} \itemsep -2pt
	\item Physics Success Stories: \url{http://www.aip.org/success/}
	\item Physics is for you; Career Services Division: \vspace{-0.2cm}
		\begin{enumerate} \itemsep -2pt
		\item \url{http://www.aip.org/careersvc/pify/}
		\item Physicists at work: \url{http://www.aip.org/careersvc/pify/yellow.html}
		\end{enumerate}
	\item Society of Physics Students (SPS): \vspace{-0.2cm}
		\begin{enumerate} \itemsep -2pt
		\item Careers Using Physics (CUP): \vspace{-0.1cm}
			\begin{enumerate} \itemsep -1pt
			\item \url{http://www.spsnational.org/cup/}
			\item Advice: \url{http://www.spsnational.org/cup/advice/index.html}
			\item Resources: \url{http://www.spsnational.org/cup/resources.html}
			\item Preparing to Teach: \url{http://www.spsnational.org/cup/teach/index.html}
			\end{enumerate}
		\end{enumerate}
	\item ComPADRE Digital Library: \vspace{-0.2cm}
		\begin{enumerate} \itemsep -2pt
		\item \url{http://www.compadre.org/}
		\item The Physics Career Resource: \url{http://www.compadre.org/careers/}
		\end{enumerate}
	\item Career guidance for high school and undergraduate students: \url{http://www.aip.org/statistics/trends/career.html}
	\item Gayle A. Buck, Jack G. Hehn, and Diandra L. Leslie-Pelecky (Editors), ``The Role of Physics Departments in Preparing K-12 Teachers,'' American Institute of Physics. Available online at: \url{http://www.aip.org/education/teacherprep/}; last accessed on January 9, 2010.
	\item American Geophysical Union: \vspace{-0.2cm}
		\begin{enumerate} \itemsep -2pt
		\item Students \& Teachers: \url{http://www.agu.org/education/students_teachers.shtml}
		\item Diversity Programs: \url{http://www.agu.org/education/diversity_programs/}
		\end{enumerate}
	\end{enumerate}
\item Institute for Operations Research and the Management Sciences (INFORMS): \vspace{-0.3cm}
	\begin{enumerate} \itemsep -2pt
	\item Career FAQ's: \url{http://www.informs.org/Build-Your-Career/INFORMS-Student-Union/Career-Center/Career-FAQ-s}
	\end{enumerate}
\item American Institute of Mathematics: \vspace{-0.3cm}
	\begin{enumerate} \itemsep -2pt
	\item Math Teachers' Circle Network: \vspace{-0.2cm}
		\begin{enumerate} \itemsep -2pt
		\item Classroom Materials: \url{http://www.mathteacherscircle.org/resources/classroommaterials.html}
		\item Helpful Resources: \url{http://www.mathteacherscircle.org/resources/general.html}
		\end{enumerate}
	\item Resources for the Math Community: \vspace{-0.2cm}
		\begin{enumerate} \itemsep -2pt
		\item \url{http://www.aimath.org/mathcommunity/}
		\item David W. Farmer, ``The AIM REU: individual projects with a common theme,'' in the {\it Proceedings of the Conference on Promoting Undergraduate Research in Mathematics}, American Mathematical Society, 2006. Available online at: \url{http://www.aimath.org/mathcommunity/farmerREU.pdf}; last accessed on January 9, 2010. [ ``AIM Research Experience for Undergraduates (REU)'' ]
		\item Sally Koutsoliotas and David W. Farmer, ``Preparing students to give talks,'' American Institute of Mathematics. Available online at: \url{http://www.aimath.org/mathcommunity/studenttalks.pdf}; last accessed on January 9, 2010. [ ``Preparing students to give talks'' ]
		\end{enumerate}
	\end{enumerate}
\item Invent Now: \vspace{-0.3cm}
	\begin{enumerate} \itemsep -2pt
	\item Camp Invention: \vspace{-0.2cm}
		\begin{enumerate} \itemsep -2pt
		\item ``Summer enrichment program for children entering grades one through six.''
		\item ``The Camp Invention program instills vital 21st century life skills such as problem-solving and teamwork through hands-on fun!''
		\item Parents: \url{http://www.invent.org/camp/parents.aspx}
		\item Teachers: \url{http://www.invent.org/camp/teachers.aspx}
		\end{enumerate}
	\end{enumerate}
\item Massachusetts Institute of Technology: \vspace{-0.3cm}
	\begin{enumerate} \itemsep -2pt
	\item MIT School of Engineering: \vspace{-0.2cm}
		\begin{enumerate} \itemsep -2pt
		\item Lemelson-MIT Program: \vspace{-0.1cm}
			\begin{enumerate} \itemsep -1pt
			\item \url{http://web.mit.edu/invent/}
			\item Invention Dimension (for children): \url{http://web.mit.edu/invent/invent-main.html}
			\end{enumerate}
		\end{enumerate}
	\end{enumerate}
\item The Lemelson Foundation: \vspace{-0.3cm}
	\begin{enumerate} \itemsep -2pt
	\item \url{http://web.mit.edu/invent/w-foundation.html}
	\item Programs \& Grants: \url{http://www.lemelson.org/programs-grants}
	\item Grantmaking: \url{http://www.lemelson.org/grantmaking}
	\end{enumerate}
\item Smithsonian Institution: \vspace{-0.3cm}
	\begin{enumerate} \itemsep -2pt
	\item Smithsonian Kids: \url{http://www.si.edu/Kids}
	\item National Museum of American History: \vspace{-0.2cm}
		\begin{enumerate} \itemsep -2pt
		\item Lemelson Center for the Study of Invention and Innovation: \vspace{-0.1cm}
			\begin{enumerate} \itemsep -1pt
			\item \url{http://inventionatplay.org/index.html}
			\item Resources: \url{http://inventionatplay.org/resources.html}
			\end{enumerate}
		\end{enumerate}
	\end{enumerate}
%%%%%%%%%%%%%%%%%%%%%%%%%%%%%%%%%%%%%%%%
%%%%%%%%%%%%%%%%%%%%%%%%%%%%%%%%%%%%%%%%
\item Scholarships: \vspace{-0.3cm}
	\begin{enumerate} \itemsep -2pt
	\item IEEE Presidents' Scholarship: \url{http://www.ieee.org/education_careers/education/preuniversity/scholarship.html}
	\item ACM/SIGDA {\it P. O. Pistilli scholarship}: \vspace{-0.1cm}
		\begin{enumerate} \itemsep -1pt
		\item Supported by the Design Automation Conference which ACM/SIGDA sponsors, the objective of the P. O. Pistilli Scholarship is to increase the pool of professionals in Electrical Engineering and Computer Science from underrepresented groups (Women, African American, Hispanic, American Indian, and Disabled).
		\item Scholarships of \$4000 per year, renewable for up to 5 years, are awarded annually to 2-7 high school seniors from the above mentioned under represented groups who have a 3.00 GPA or better (on a 4.00 scale), have demonstrated high achievement in math and science courses, have expressed a strong desire to pursue careers in electrical engineering, computer engineering, or computer science, and who have demonstrated substantial financial need.
		\item U.S. citizenship is not required, but applicants must be U.S. residents when they apply and must plan to attend an accredited US college or university.
		\item \url{http://www.sigda.org/pistilli.html}
		\end{enumerate}
	\item Engineering Education Service Center (EESC): \url{http://www.engineeringedu.com/scholars.html}
	\item ASME-ASME Auxiliary FIRST Clarke Scholarships: \url{http://www.asme.org/Education/College/FinancialAid/High_School_Seniors.cfm} and \url{http://www.asme.org/Education/College/FinancialAid/Auxiliary_FIRST_Clarke.cfm}
	\item International Petroleum Institute�s High School Scholarships (for individuals entering a college program in engineering): \url{http://www.asme-ipti.org/public/pagscholarshipprograms.aspx}
	\item American Institute of Chemical Engineers (AIChE): \vspace{-0.2cm}
		\begin{enumerate} \itemsep -2pt
		\item Fuels and Petrochemicals Division Scholarship (for high school students entering undergraduate programs in engineering or science that are related to fuels and petrochemicals): \url{http://www.aiche.org/Students/Awards/F_PDScholarship.aspx}
		\item Minority Scholarship Awards for Incoming College Freshmen (for underrepresented minorities entering an undergraduate chemical engineering program): \url{http://www.aiche.org/Students/Awards/MinorityScholarshipAwardsIncomingFreshmen.aspx}
		\end{enumerate}
	\item Sallie Mae Fund: \vspace{-0.3cm}
		\begin{enumerate} \itemsep -2pt
		\item \url{http://www.thesalliemaefund.org/smfnew/index.html}
		\item List of scholarship resources: \url{http://www.thesalliemaefund.org/smfnew/sections/search.html}
		\item Top 10 Tips for Planning and Paying for College: \url{http://www.thesalliemaefund.org/smfnew/fin_aid/index.html}
		\item Scholarships: \url{http://www.thesalliemaefund.org/smfnew/scholarship/index.html} and \url{http://www.thesalliemaefund.org/smfnew/sections/apply.html}
		\item Important information for parents about saving for college and getting financial aid: \vspace{-0.2cm}
			\begin{enumerate} \itemsep -2pt
			\item \url{http://www.thesalliemaefund.org/smfnew/sections/download.html}
			\item This information is also available in Spanish. Summaries are also available in other languages such as: \vspace{-0.1cm}
				\begin{itemize} \itemsep -1pt
				\item French
				\item German
				\item Italian
				\item Korean
				\item Russian
				\item Simplified and Traditional Chinese
				\item Tagalog
				\item Vietnamese
				\end{itemize}
			\item Top 10 Tips for Planning and Paying for College: \url{http://www.thesalliemaefund.org/smfnew/fin_aid/index.html}
			\end{enumerate}
		\item Kids2College program: \url{http://www.thesalliemaefund.org/smfnew/initiatives/kidscollege.html}
		\item For African-American individuals entering college: \vspace{-0.2cm}
			\begin{enumerate} \itemsep -2pt
			\item Black College Dollars: \url{http://www.thesalliemaefund.org/smfnew/scholarship_directory/index.html}
			\item \url{http://www.thesalliemaefund.org/smfnew/initiatives/aa.html}
			\end{enumerate}
		\item For Hispanic Americans, or Latinos/Latinas: \vspace{-0.2cm}
			\begin{enumerate} \itemsep -2pt
			\item \url{http://www.thesalliemaefund.org/smfnew/pdf/Scholarship_Directory.pdf}
			\item Latino College Dollars: \url{http://www.latinocollegedollars.org/}
			\end{enumerate}
		\end{enumerate}
	\item {\it American Chemical Society}: \vspace{-0.3cm}
		\begin{enumerate} \itemsep -2pt
		\item ACS Scholars Program (for underrepresented minorities in, or entering, an undergraduate program in chemistry, biochemistry, or chemical engineering): \url{http://portal.acs.org/portal/acs/corg/content?_nfpb=true&_pageLabel=PP_SUPERARTICLE&node_id=1650&use_sec=false&sec_url_var=region1&__uuid=b3b583cf-18ae-4fb0-9375-33f75a0ccf49}
		\item Project SEED Scholarships (for high school seniors who have worked at least one summer at a science institute under the Project SEED program): \url{http://portal.acs.org/portal/acs/corg/content?_nfpb=true&_pageLabel=PP_SUPERARTICLE&node_id=2031&use_sec=false&sec_url_var=region1&__uuid=99bc6a62-3e78-4b2a-be3f-50b28f7ff265}
		\end{enumerate}
	\item The Posse Foundation: \url{http://www.possefoundation.org/}
	\item Hispanic Scholarship Fund (HSF) scholarship programs for high school students: \url{http://www.hsf.net/innerContent.aspx?id=426}
	\item Asian \& Pacific Islander American Scholarship Fund (APIASF): scholarships for individuals entering college as freshmen; see \url{http://www.apiasf.org/scholarship_apiasf.html}
	\item Nationally Coveted College Scholarships, Graduate School Fellowships \& Postdoctoral Awards: \url{http://scholarships.fatomei.com/}
	\item {\it SPIE} Scholarship Program (for high school students entering college to study optics, photonics, imaging, optoelectronics, or related program): \url{http://spie.org//x1733.xml?WT.svl=mddm14}
	\item Susan G. Komen for the Cure\textregistered: The Komen College Scholarship Program, \url{http://ww5.komen.org/ResearchGrants/CollegeScholarshipAward.html}
	\item National Society of Professional Engineers's list of scholarships for high school students: \url{http://www.nspe.org/Students/Scholarships/index.html}
	\item AWM Essay Contest: Biographies of Contemporary Women in Mathematics; see \url{http://www.awm-math.org/biographies/contest.html}
	\item National Engineers Week Future City Competition (students from $6^{th}$--$8^{th}$ grades): \url{http://www.futurecity.org/}
	\item National Ocean Sciences Bowl: \vspace{-0.2cm}
		\begin{enumerate} \itemsep -2pt
		\item \url{http://www.nosb.org/ocean-careers/}
		\item National Ocean Scholar Program (for high school seniors who are current/past participants of the Bowl, and are seeking a career in the ocean sciences or a marine-related field): \url{http://www.nosb.org/ocean-careers/national-ocean-scholar-program/}
		\end{enumerate}
	\item National Center for Women \& Information Technology (NCWIT): \vspace{-0.2cm}
		\begin{enumerate} \itemsep -2pt
		\item NCWIT Award for Aspirations in Computing (for young women in high school): \url{http://www.ncwit.org/work.awards.aspiration.html}
		\end{enumerate}
	\end{enumerate}
%%%%%%%%%%%%%%%%%%%%%%%%%%%%%%%%%%%%%%%%
%%%%%%%%%%%%%%%%%%%%%%%%%%%%%%%%%%%%%%%%
\item Resources for teachers/educators: \vspace{-0.3cm}
	\begin{enumerate} \itemsep -2pt
	\item Google: \vspace{-0.2cm}
		\begin{enumerate} \itemsep -2pt
		\item Google Teacher Academy (for teachers to learn how to use Google technologies to facilitate teaching): \url{http://www.google.com/educators/gta.html}
		\item Classroom activities (suggestions): \url{http://www.google.com/educators/activities.html}
		\end{enumerate}
	\item IEEE Teacher In-Service Program (TISP): \vspace{-0.2cm}
		\begin{enumerate} \itemsep -2pt
		\item \url{http://www.ieee.org/education_careers/education/preuniversity/tispt/index.html}
		\item Lesson Plans for Pre-university Instructors: \url{http://www.ieee.org/education_careers/education/preuniversity/resources/index.html}
		\end{enumerate}
	\item Global Challenge Award: \url{http://www.globalchallengeaward.org/display/public/Home}
	\item Teachers' Domain (to teach students about science, engineering, and the arts): \url{http://www.teachersdomain.org/}
	\item {\it TeachEngineering} digital library: \vspace{-0.2cm}
		\begin{enumerate} \itemsep -2pt
		\item The {\it TeachEngineering} digital library provides teacher-tested, standards-based engineering content for K-12 teachers engineering content for K12 teachers to use in science and math classrooms. Engineering lessons connect real-world experiences with curricular content already taught in K-12 classrooms. Mapped to educational content standards, {\it TeachEngineering}'s comprehensive curricula are hands-on, free, and relevant to children's daily lives.
		\item \url{http://www.teachengineering.com/index.php}
		\end{enumerate}
	\item Engineering Pathway: \url{http://www.engineeringpathway.com/ep/index.jhtml}
	\item {\it American Society of Mechanical Engineers, ASME}: \url{http://www.asme.org/Education/PreCollege/TeacherResources/}
	\item {\it National Science Foundation} resources for the K-12 classroom: \url{http://nsf.gov/news/classroom/engineering.jsp}
	\item {\it NASA}: \url{http://www.nasa.gov/audience/foreducators/index.html}
	\item The Mathematical Association of America: \vspace{-0.2cm}
		\begin{enumerate} \itemsep -2pt
		\item Pre-College Programs: \url{http://www.maa.org/funding/pre_college.html}. Also, see \url{http://www.maa.org/funding/undergraduate.html}.
		\item Special Interest Group of the Mathematical Association of America on the use of the World-Wide Web in Undergraduate Mathematics Instruction (Web SIGMAA). Available at: \url{http://math.chapman.edu/websigmaa/index.php/Main_Page}; last accessed on September 2, 2010.
		\item SIGMAA TAHSM (Teaching Advanced High School Mathematics). Available at: \url{http://sigmaa.maa.org/tahsm/}; last accessed on September 2, 2010.
		\item Special Interest Group on Statistics Education: \url{http://sigmaa.maa.org/stat-ed/}
		\end{enumerate}
	\item Math for America: \vspace{-0.2cm}
		\begin{enumerate} \itemsep -2pt
		\item M$f$A Master Teacher Fellowship program: \vspace{-0.1cm}
			\begin{enumerate} \itemsep -1pt
			\item The Math for America Master Teacher Fellowship program rewards exceptional public secondary school math teachers with a four-year Fellowship.
			\item M$f$A Master Teacher Fellowships are currently available in Berkeley, Boston and New York City.
			\item \url{http://www.mathforamerica.org/web/guest/master-teachers}
			\end{enumerate}
		\item M$f$A Early Career Fellows: \vspace{-0.1cm}
			\begin{enumerate} \itemsep -1pt
			\item The Math for America Early Career Fellowship is awarded to public secondary school math teachers early in their careers.
			\item The M$f$A Early Career Fellowship requires a commitment of four years and is available in New York City. 
			\item \url{http://www.mathforamerica.org/early-career-fellows}
			\end{enumerate}
		\item M$f$A Fellows: \vspace{-0.1cm}
			\begin{enumerate} \itemsep -1pt
			\item \url{http://www.mathforamerica.org/web/guest/mfa-fellows}
			\end{enumerate}
		\item Teachers resources: \url{http://www.mathforamerica.org/web/guest/teacher-resources} and \url{http://www.mathforamerica.org/teacher-resources/classroom} (classroom resources)
		\item Resources for professional development (teachers): \url{http://www.mathforamerica.org/teacher-resources/professional}
		\item \url{http://www.mathforamerica.org/home}
		\end{enumerate}
	\item Association for Symbolic Logic (ASL): \vspace{-0.2cm}
		\begin{enumerate} \itemsep -2pt
		\item Guidelines on Logic Education: \url{http://www.ucalgary.ca/aslcle/guidelines}
		\item Educational Logic Software: \url{http://www.ucalgary.ca/aslcle/logic-courseware}
		\end{enumerate}
	\item Consortium for Ocean Leadership: \vspace{-0.2cm}
		\begin{enumerate} \itemsep -2pt
		\item Educational Resources: \url{http://www.oceanleadership.org/gulf-oil-spill/educational-resources/}
		\item The JOIDES Resolution (The JR) scientific research vessel [ Deep Earth Academy ]: \vspace{-0.1cm}
			\begin{enumerate} \itemsep -1pt
			\item Teacher Resources (to teach students about geology and physical geography): \url{http://joidesresolution.org/node/46}
			\item Teachers at Sea/On-board Education Officer (for teachers to go on scientific expeditions on board): \url{http://joidesresolution.org/node/453}
			\end{enumerate}
		\item Integrated Ocean Drilling Program (IODP) -- IODP United States Implementing Organization (IODP-USIO): \vspace{-0.1cm}
			\begin{enumerate} \itemsep -1pt
			\item Teaching Materials: \url{http://www.iodp-usio.org/Education/educ.html}
			\end{enumerate}
		\item Deep Earth Academy (includes suggested ``curriculum and classroom activities for kindergarten through college level''): \vspace{-0.1cm}
			\begin{enumerate} \itemsep -1pt
			\item \url{http://www.oceanleadership.org/education/deep-earth-academy/}
			\item For Educators: \url{http://www.oceanleadership.org/education/deep-earth-academy/educators/}
			\end{enumerate}
		\end{enumerate}
	\item Virginia Institute of Marine Science (College of William and Mary): \vspace{-0.2cm}
		\begin{enumerate} \itemsep -2pt
		\item Bridge Ocean Education Teacher Resource Center: \url{http://web.vims.edu/bridge/?svr=www#}
		\end{enumerate}
	\item American Geological Institute: \vspace{-0.2cm}
		\begin{enumerate} \itemsep -2pt
		\item Awards for teachers: \url{http://www.agiweb.org/education/awards/index.html}
		\item Edward C. Roy, Jr. Award For Excellence in K-8 Earth Science Teaching (for middle school teachers in the US who are teaching earth science): \url{http://www.agiweb.org/education/awards/ed-roy/}
		\item Presidential Awards for Excellence in Mathematics \& Science Teaching, PAEMST (for kindergarten and K-12 teachers in the US who are teaching students about STEM fields): \url{http://www.agiweb.org/education/awards/paemst.html}
		\item National Association of Geoscience Teachers (NAGT) Outstanding Earth Science Teacher Award: \url{http://www.agiweb.org/education/awards/nagt.html}
		\item American Association of Petroleum Geologists' (AAPG) National Earth Science Teacher of the Year Award: \url{http://www.agiweb.org/education/awards/aapg.html}
		\item Curriculum Materials and Activities: \url{http://www.agiweb.org/education/curriculum/index.html}
		\item K-12 Professional Development Programs: \url{http://www.agiweb.org/education/pd/index.html}
		\item Educational Resources: \url{http://www.agiweb.org/education/resource/index.html}
		\end{enumerate}
	\item Institute for Broadening Participation: \vspace{-0.2cm}
		\begin{enumerate} \itemsep -2pt
		\item PathwaysToScience.org: \vspace{-0.1cm}
			\begin{enumerate} \itemsep -1pt
			\item For K-12 teachers (resources to encourage students to be interested in STEM): \url{http://www.pathwaystoscience.org/Teachers.asp}
			\end{enumerate}
		\end{enumerate}
	\item National Science Foundation: \vspace{-0.2cm}
		\begin{enumerate} \itemsep -2pt
		\item The National Science Digital Library (NSDL): \vspace{-0.1cm}
			\begin{enumerate} \itemsep -1pt
			\item Resources for K-12 Teachers: \url{http://nsdl.org/resources_for/k12_teachers/}
			\end{enumerate}
		\end{enumerate}
	\item National Academy of Engineering, NAE: \vspace{-0.2cm}
		\begin{enumerate} \itemsep -2pt
		\item NAE Grand Challenges: \vspace{-0.1cm}
			\begin{enumerate} \itemsep -1pt
			\item Includes a list of NAE Grand Challenges, which indicate some of the challenges faced by people on a global scale that can be partially solved by engineers. This can be used to get children and youths to be excited about engineering. 
			\item NAE Grand Challenges: \vspace{-0.1cm}
				\begin{itemize} \itemsep -1pt
				\item Make solar energy economical
				\item Provide energy from fusion
				\item Develop carbon sequestration methods
				\item Manage the nitrogen cycle
				\item Provide access to clean water
				\item Restore and improve urban infrastructure
				\item Advance health informatics
				\item Engineer better medicines
				\item Reverse-engineer the brain
				\item Prevent nuclear terror
				\item Secure cyberspace
				\item Enhance virtual reality
				\item Advance personalized learning
				\item Engineer the tools of scientific discovery
				\end{itemize}
			\item \url{http://www.engineeringchallenges.org/}
			\end{enumerate}
		\item NAE Grand Challenge K12 Partners Program: \vspace{-0.1cm}
			\begin{enumerate} \itemsep -1pt
			\item Can be used by schools/teachers to raise awareness of global challenges among students and to encourage students to plan career paths to tackle these challenges
			\item 5-Part Make it Happen Plan (includes suggested activities for students in elementary school to learn about basic science and engineering concepts that are relevant to solve the NAE grand challenges): \url{http://www.grandchallengek12.org/5-part-plan}
			\item \url{http://www.grandchallengek12.org/about}
			\end{enumerate}
		\item {\it National Academy of Engineering}'s technological literacy program for people (students, parents, and educators) to learn more about technology: \url{http://www.nae.edu/nae/techlithome.nsf}
		\end{enumerate}
	\item Women in Technology (WIT): \vspace{-0.2cm}
		\begin{enumerate} \itemsep -2pt
		\item Girls In Technology (GIT): \vspace{-0.1cm}
			\begin{enumerate} \itemsep -1pt
			\item Get Involved: \vspace{-0.1cm}
				\begin{itemize} \itemsep -1pt
				\item \url{http://www.girlsintechnology.org/getinvolved.cfm}
				\item Teacher: teach girls about IT as an after-school activity or in a summer camp session
				\item Assistant Teacher: Assist instructors in GIT sessions, after-school activities, or summer camp sessions
				\item Develop Curriculum: Develop a curriculum for a supported GIT educational program
				\item Mentor: Mentor a girl in one of [GIT's] supported programs
				\item Job Shadow: ``Let a girl shadow you at work''
				\item Guest Speaker: ``Speak to a group of girls on a topic both you and they enjoy, such as computers, technology, education, how to take apart computers, how to build a web site, etc.''
				\end{itemize}
			\end{enumerate}
		\end{enumerate}
	\item Organization for Economic Co-operation and Development (OECD): \vspace{-0.2cm}
		\begin{enumerate} \itemsep -2pt
		\item Programme for International Student Assessment (PISA): \vspace{-0.1cm}
			\begin{enumerate} \itemsep -1pt
			\item {\it PISA 2009 Results}. Available online at: \url{http://www.oecd.org/document/61/0,3343,en_32252351_32235731_46567613_1_1_1_1,00.html}; last accessed on December 10, 2010. [ Includes suggestions to improve learning outcomes, as well as education policies and practices. ]
			\end{enumerate}
		\end{enumerate}
	\item American Institute of Aeronautics and Astronautics (AIAA): \vspace{-0.2cm}
		\begin{enumerate} \itemsep -2pt
		\item K-12 Educators: \url{http://www.aiaa.org/content.cfm?pageid=208}
		\end{enumerate}
	\item Research Councils UK (RCUK): \vspace{-0.2cm}
		\begin{enumerate} \itemsep -2pt
		\item Biotechnology and Biological Sciences Research Council (BBSRC): \vspace{-0.1cm}
			\begin{enumerate} \itemsep -1pt
			\item Resources for schools and young people: \url{http://www.bbsrc.ac.uk/society/schools/schools-index.aspx}
			\item Teaching resources: publications and web-based activities: \vspace{-0.1cm}
				\begin{itemize} \itemsep -1pt
				\item Primary (ages 5-12) resources: \url{http://www.bbsrc.ac.uk/society/schools/primary/primary-index.aspx}
				\item Secondary (ages 12-16) and post-16 resources: \url{http://www.bbsrc.ac.uk/society/schools/secondary/secondary-index.aspx}
				\end{itemize}
			\end{enumerate}
		\end{enumerate}
	\item Nuffield Foundation: \vspace{-0.2cm}
		\begin{enumerate} \itemsep -2pt
		\item Education: \url{http://www.nuffieldfoundation.org/education}
		\item Teachers: \vspace{-0.1cm}
			\begin{enumerate} \itemsep -1pt
			\item (Excellent) resources in science and mathematics: \url{http://www.nuffieldfoundation.org/teachers}
			\item \url{http://www.nuffieldfoundation.org/teachers-0}
			\end{enumerate}
		\end{enumerate}
	\item Wellcome Trust: \vspace{-0.2cm}
		\begin{enumerate} \itemsep -2pt
		\item Education resources: \url{http://www.wellcome.ac.uk/Education-resources/index.htm}
		\item {\it yourgenome.org}: \vspace{-0.1cm}
			\begin{enumerate} \itemsep -1pt
			\item \url{http://www.yourgenome.org/}
			\item Resources for teachers about genomics: \url{http://www.yourgenome.org/landing_teachers.shtml}
			\end{enumerate}
		\item Network of Science Learning Centers (Science Learning Centers): \vspace{-0.1cm}
			\begin{enumerate} \itemsep -1pt
			\item \url{https://www.sciencelearningcentres.org.uk/}
			\item Awards and Bursaries: \vspace{-0.1cm}
				\begin{itemize} \itemsep -1pt
				\item \url{https://www.sciencelearningcentres.org.uk/centres/national/awards-and-bursaries}
				\item \url{https://www.sciencelearningcentres.org.uk/about/impact-awards}
				\end{itemize}
			\item Resource collections: \url{https://www.sciencelearningcentres.org.uk/resources}
			\item Curriculum resources for primary, secondary, and tertiary education: \url{https://www.sciencelearningcentres.org.uk/curriculum}
			\end{enumerate}
		\end{enumerate}
	\end{enumerate}
%%%%%%%%%%%%%%%%%%%%%%%%%%%%%%%%%%%%%%%%
%%%%%%%%%%%%%%%%%%%%%%%%%%%%%%%%%%%%%%%%
\item Underrepresented minorities: \vspace{-0.3cm}
	\begin{enumerate} \itemsep -2pt
	\item University of Washington: \vspace{-0.2cm}
		\begin{enumerate} \itemsep -2pt
		\item Department of Computer Science and Engineering: \vspace{-0.1cm}
			\begin{enumerate} \itemsep -1pt
			\item {\it AccessComputing}: \vspace{-0.1cm}
				\begin{itemize} \itemsep -1pt
				\item \url{http://www.washington.edu/accesscomputing/}
				\item Has resources to help students with disabilities to pursue ``undergraduate and graduate degrees and careers in computing fields''.
				\item It runs the ``Summer Academy for Advancing Deaf \& Hard of Hearing in Computing'' for youths who are hearing impaired: \url{http://www.washington.edu/accesscomputing/dhh/academy/index.html}
				\end{itemize}
			\end{enumerate}
		\end{enumerate}
	%%%%%%%%%%%%%%%%%%%%%%%%%
	\item Engineer Girl: \vspace{-0.2cm}
		\begin{enumerate} \itemsep -2pt
		\item Resources for students, parents, and teachers to encourage girls to explore careers and educational opportunities in engineering
		\item Created by the National Academy of Sciences and The National Academy of Engineering
		\item Contests for K-12 students: \url{http://www.engineergirl.org/?id=4436}
		\item \url{http://www.engineergirl.org/}
		\end{enumerate}
	\item Engineering Your Life: \url{http://www.engineeryourlife.org/}
	\item GirlGeeks: \url{http://www.girlgeeks.org/}
	\item {\it Women in Science, Technology, Engineering, and Mathematics ON THE AIR!}: \vspace{-0.2cm}
		\begin{enumerate} \itemsep -2pt
		\item Audio resources that describe stories about women in science, technology, engineering, and mathematics (STEM) fields
		\item \url{http://www.womeninscience.org/}
		\end{enumerate}
	\item {\it Women Scientists in History}: \url{http://www.hypatiamaze.org/}
	\item Association for Women in Mathematics (AWM): \vspace{-0.2cm}
		\begin{enumerate} \itemsep -2pt
		\item \url{http://www.awm-math.org/}
		\item Education: \vspace{-0.1cm}
			\begin{enumerate} \itemsep -1pt
			\item \url{http://sites.google.com/site/awmmath/awm-resources/education}
			\item Includes information for students in middle school, high school, college and university (including graduate students). It also includes information for parents and teachers/educators.
			\end{enumerate}
		\item Women in Math, Science, and Society: \url{http://sites.google.com/site/awmmath/women-in-math-science-and-society}
		\item Essay contest on biographies of contemporary women in mathematics: \url{http://sites.google.com/site/awmmath/programs/essay-contest}
		\end{enumerate}
	\item Women in Technology (WIT): \vspace{-0.2cm}
		\begin{enumerate} \itemsep -2pt
		\item Girls in Technology: \vspace{-0.1cm}
			\begin{enumerate} \itemsep -1pt
			\item \url{http://www.girlsintechnology.org/}
			\item WIT Education Foundation: provides educational programs for girls in technology
			\item TeamBusiness Fundraiser: ``A combined fundraiser and program for girls in Grades 9-12 across the Metro DC area. Each year, up to forty girls participate with mentors and WIT volunteers in a full-day business simulation workshop conducted by TeamBusiness USA. The teams competed as companies, learning how to run a technology company in a fun and exciting simulation environment.''
			\item Hispanic Youth Foundation: ``In 2005, GIT established a partnership with the Hispanic Youth Foundation (HYF) and provided a grant to fund HYF�s innovative Laptops for Learning Dollars program, providing laptops and Internet connections for elementary and middle school students and their families in Arlington County and the City of Manassas.''
			\item Empower Girls -- CLCP Clubs: ``Empower Girls after-school programs were held at Hybla Valley Elementary School and Sacramento Community Center. GIT/WITEF provided funding to run these programs in conjunction with the Fairfax County Computer Learning Center Partnership (CLCP). The selected centers serve economically challenged communities in Fairfax County.''
			\end{enumerate}
		\end{enumerate}
	%%%%%%%%%%%%%%%%%%%%%%%%%
	\item National Society of Black Engineers (NSBE) competitions for high school/K-12 students: \url{http://www.nsbe.org/Programs/Competitions/NSBE-Jr-.aspx}
	\item The Society of Mexican American Engineers and Scientists (MAES): MAES PreCollege Outreach Programs, \url{http://www.maes-natl.org/index.php?module=ContentExpress&func=display&ceid=16&meid=236}
	\item {\it Center for the Advancement of Hispanics in Science and Engineering Education} (CAHSEE): \vspace{-0.2cm}
		\begin{enumerate} \itemsep -2pt
		\item STEM - The Science, Technology, Engineering \& Mathematics Institute (for students from grades 5 through 11): \url{http://www.cahsee.org/2programs/stem.asp.htm}
		\item YEP - Young Educators Program (fellows would learn how to train students in the aforementioned STEM Institute): \url{http://www.cahsee.org/2programs/yep.asp.htm}
		\item CAYSA - Central American Young Scholar Awards: \url{http://www.cahsee.org/2programs/caysa.asp.htm}. ``The CAYSA ceremonies honor more than 60 Washington, D.C. area high school seniors of Central American descent who have demonstrated remarkable success throughout all four years of high school. Students must be of Central American descent and have at least a 3.0 gpa.''
		\item Scholarships: \url{http://www.cahsee.org/6resources/scholarships.asp.htm}
		\item \url{http://www.cahsee.org/about/about.asp.htm}
		\end{enumerate}
	%%%%%%%%%%%%%%%%%%%%%%%%%
	\item International Computer Science Institute (UC Berkeley): \vspace{-0.2cm}
		\begin{enumerate} \itemsep -2pt
		\item Berkeley Foundation for Opportunities in Information Technology, BFOIT: \vspace{-0.1cm}
			\begin{enumerate} \itemsep -1pt
			\item BFOIT Programs for women and underrepresented minorities (African Americans and Chicanos/Latinos) in middle/high school who are interested in electrical/computer engineering and computer science careers: \url{http://www.bfoit.org/programs.html}
			\end{enumerate}
		\end{enumerate}
	\item Institute for Broadening Participation: \vspace{-0.2cm}
		\begin{enumerate} \itemsep -2pt
		\item PathwaysToScience.org: \vspace{-0.1cm}
			\begin{enumerate} \itemsep -1pt
			\item PathwaysToScience.org is a portal website supporting pathways to the STEM fields: science, technology, engineering, and mathematics.
			\item Particular emphasis is placed on connecting traditionally underrepresented groups with STEM programs and resources, including funding and mentoring opportunities. 
			\item For K-12 students: \url{http://www.pathwaystoscience.org/K12.asp}
			\item STEM Resources by Institution (colleges, universities, and US national research laboratories): \url{http://www.pathwaystoscience.org/Institution.asp}
			\item profiles of people and programs in STEM: \vspace{-0.3cm}
				\begin{itemize} \itemsep -2pt
				\item \url{http://www.pathwaystoscience.org/Profiles.asp}
				\item Find out about the career paths of underrepresented minorities in STEM
				\item Find out about programs that are offered by institutions for underrepresented minorities in STEM
				\end{itemize}
			\item Directory of partners (organizations that cooperate with or support the Institute for Broadening Participation): \url{http://www.pathwaystoscience.org/Partners.asp}
			\item Additional resources: \url{http://www.pathwaystoscience.org/Ideaexchange.asp}
			\end{enumerate}
		\item Maine Pathways to STEM (Science, Technology, Engineering \& Mathematics): \vspace{-0.1cm}
			\begin{enumerate} \itemsep -1pt
			\item \url{http://www.mainestem.org/}
			\item K-12 Teachers \& University Faculty: \url{http://www.mainestem.org/METeachersFaculty.asp}
			\item K-12 STEM Resources: \url{http://www.mainestem.org/MEK12.asp}
			\end{enumerate}
		\end{enumerate}
	\item Building Engineering and Science Talent, BEST: \vspace{-0.2cm}
		\begin{enumerate} \itemsep -2pt
		\item \url{http://www.bestworkforce.org/}
		\item Publications: \url{http://www.bestworkforce.org/publications.htm}
		\item List of programs to help underrepresented minority students in K-12 schools explore careers in STEM: \url{http://www.bestworkforce.org/links.htm}
		\end{enumerate}
	\item American Indian Science and Engineering Society (AISES): \vspace{-0.2cm}
		\begin{enumerate} \itemsep -2pt
		\item Pre-college programs: \vspace{-0.1cm}
			\begin{enumerate} \itemsep -1pt
			\item \url{http://www.aises.org/Programs}
			\item Resources: \url{http://www.aises.org/Programs/Resources}
			\end{enumerate}
		\end{enumerate}
	\end{enumerate}
\end{enumerate}







%%%%%%%%%%%%%%%%%%%%%%%%%%%%%%%%%%%%%%%%%%%
\subsection{Science \& Engineering Outreach for Undergraduates, Grad Students, \& Postdocs}
\label{stemoutreachcollegegradsch}


Science, mathematics, and engineering outreach to undergraduates, graduate students, and postdocs: \vspace{-0.3cm}
\begin{enumerate} \itemsep -4pt
\item Mac Hyman, ``Good Choices for Great Careers in the Mathematical Sciences,'' talk given at 2008 SIAM Annual Meeting. Available at: \url{http://client.blueskybroadcast.com/siam08/hyman/index.html}; last accessed on August 25, 2010. Also, see \url{http://meetings.siam.org/program.cfm?CONFCODE=AN08}, \url{http://www.siam.org/meetings/an08/program.php}, and \url{http://www.siam.org/meetings/an08/}.
\item {\it Accreditation.org}: \vspace{-0.3cm}
	\begin{enumerate} \itemsep -2pt
	\item Information about the accreditation of engineering degree programs around the world
	\item \url{http://www.accreditation.org/}
	\end{enumerate}
\item John Baez, ``How to Learn Math and Physics,'' Department of Mathematics, University of California, Riverside, December 24, 2007. Available at: \url{http://math.ucr.edu/home/baez/books.html}; last accessed on August 28, 2010.
\item {\it MentorNet}: \vspace{-0.3cm}
	\begin{enumerate} \itemsep -2pt
	\item \url{http://www.mentornet.net/}
	\item Enables people to network with scientists, engineers, and professors in Science, Technology, Engineering, and Mathematics (STEM)
	\item Is very supportive of minorities, so that more minorities (particularly underrepresented minorities) can be attracted to STEM careers
	\end{enumerate}
\item {\it The Indus Entrepreneurs (TiE)} for networking among high-tech entrepreneurs, start-up co-founders, venture capitalists, and angel investors: \url{http://www.tie.org/}
\item National Academy of Engineering, NAE: \vspace{-0.3cm}
	\begin{enumerate} \itemsep -2pt
	\item Includes a list of NAE Grand Challenges, which can provide some suggestions for research trajectories
	\item Summit Series on the Grand Challenges: Includes the National Grand Challenges Summits
	\item \url{http://www.engineeringchallenges.org/}
	\end{enumerate}
\item {\it National Society of Professional Engineers}: \vspace{-0.3cm}
	\begin{enumerate} \itemsep -2pt
	\item Student Resources: \vspace{-0.2cm}
		\begin{enumerate} \itemsep -2pt
		\item \url{http://www.nspe.org/Students/Resources/index.html}
		\item An Employment Guidelines Checklist for the Engineer Job Applicant: \url{http://www.nspe.org/Students/Resources/checklist.html}
		\end{enumerate}
	\item Career Center: \url{http://www.nspe.org/CareerCenter/index.html}
	\item A Sightseer's Guide to Engineering: \url{http://www.engineeringsights.org/}
	\end{enumerate}
\item {\it JustGarciaHill} ``Study Skills for Budding Scientists'': \url{http://www.justgarciahill.org/index.php/science-study-skills.html}
\item {\it NASA} resources for students: \vspace{-0.3cm}
	\begin{enumerate} \itemsep -2pt
	\item \url{http://www.nasa.gov/audience/forstudents/index.html}
	\item NASA University Student Launch Initiative, or USLI: \url{http://www.nasa.gov/offices/education/programs/descriptions/University_Student_Launch_Initiative.html}
	\end{enumerate}
\item {\it iTunes U}: \vspace{-0.3cm}
	\begin{enumerate} \itemsep -2pt
	\item {\it iTunes} is required to listen to or watch these lectures, talks, and presentations.
	\item Access {\it iTunes U} at: \url{http://www.apple.com/education/itunes-u/} or \url{http://deimos3.apple.com/indigo/main/main.html?v0=WWW-AMUS-ITUNESU070521-N48LX}
	\item {\it iTunes U} is a set of webcast and podcasts, where you can easily find audio and video clips for lectures, seminars, announcements, virtual tours, and so on. For example, some professors from schools like MIT or Berkeley will provide audio/video clips of their lectures on {\it iTunes U}.
	\item This can help in exploring different majors before a college student declares her/his major(s). If a student is not sure if she/he wants to double major in deaf studies and linguistics, this student can check out some linguistics lectures from her/his (preferred) college/university, if it uses {\it iTunes U}, or those from other universities.
	\end{enumerate}
\item Harvey Mudd College: \vspace{-0.3cm}
	\begin{enumerate} \itemsep -2pt
	\item Francis Edward Su, {\it Math Fun Facts!}, Department of Mathematics, Harvey Mudd College: \url{http://www.math.hmc.edu/funfacts/}
	\end{enumerate}
\item Engineering Pathway: \url{http://www.engineeringpathway.com/ep/index.jhtml}
\item Rochester Institute of Technology, ``Biology \& Biotechnology Paid Co-op/Internships for 2011,'' Department of Biological Sciences, Rochester Institute of Technology: \url{http://people.rit.edu/gtfsbi/Symp/summer.htm}
\item {\it Mathematical Association of America (MAA)} information on educational pathways and career opportunities: \vspace{-0.3cm}
	\begin{enumerate} \itemsep -2pt
	\item Undergraduate Students: \url{http://www.maa.org/students/undergrad/}
	\item Graduate Students: \url{http://www.maa.org/students/grad/}
	\item Underrepresented Groups: \url{http://www.maa.org/programs/underrep.html}
	\item Mathematical Association of America (MAA) MathFest (for students in mathematics): \url{http://www.maa.org/mathfest/}
	\item MAA Online Columns: \url{http://www.maa.org/news/columns.html}
	\end{enumerate}
\item New Zealand Institute of Mathematics and its Applications (NZIMA): \vspace{-0.3cm}
	\begin{enumerate} \itemsep -2pt
	\item {\it MathsReach}: Careers (information about careers based on a higher education in mathematics or related field): \url{http://www.mathsreach.org/Careers}
	\end{enumerate}
\item {\it Engineers Dedicated to a Better Tomorrow (a.k.a., DedicatedEngineers)}: \vspace{-0.3cm}
	\begin{enumerate} \itemsep -2pt
	\item [Resources for] College Students and Faculty/Staff Members: \url{http://www.dedicatedengineers.org/intro_for_college.htm}
	\item \url{http://www.dedicatedengineers.org/}
	\end{enumerate}
\item American Institute of Physics: \vspace{-0.3cm}
	\begin{enumerate} \itemsep -2pt
	\item GradschoolShopper.com: \vspace{-0.2cm}
		\begin{enumerate} \itemsep -2pt
		\item \url{http://www.gradschoolshopper.com/}
		\item ``Find information on graduate programs in physics, astronomy, and other physical sciences''
		\end{enumerate}
	\item Career guidance for high school and undergraduate students: \url{http://www.aip.org/statistics/trends/career.html}
	\item American Geophysical Union: \vspace{-0.2cm}
		\begin{enumerate} \itemsep -2pt
		\item Diversity Programs: \url{http://www.agu.org/education/diversity_programs/}
		\end{enumerate}
	\end{enumerate}
\item {\it icademic.org} resources for the life sciences and engineering: \url{http://www.icademic.org/}
\item The Oceanography Society: \vspace{-0.3cm}
	\begin{enumerate} \itemsep -2pt
	\item Hands-On Oceanography: peer-reviewed activities appropriate for undergraduate and/or graduate classes in oceanography, \url{http://www.tos.org/hands-on/index.html}
	\end{enumerate}
%%%%%%%%%%%%%%%%%%%%%%%%%%%%%%%%%%%%%%%
%%%%%%%%%%%%%%%%%%%%%%%%%%%%%%%%%%%%%%%
\item outreach activities (including mentoring) to students in K-12: \vspace{-0.3cm}
	\begin{enumerate} \itemsep -2pt
	\item Research Councils UK (RCUK): \vspace{-0.2cm}
		\begin{enumerate} \itemsep -2pt
		\item Researchers in Residence (RinR): \vspace{-0.1cm}
			\begin{enumerate} \itemsep -1pt
			\item \url{http://www.researchersinresidence.ac.uk/cms/}
			\item \url{http://www.researchersinresidence.ac.uk/cms/researchers/}
			\item Mentor middle and high school students who are job shadowing (observing you first-hand) in your research activities for up to a week, so that they can learn what doing research in your research area is like. You should explain in laypeople's terms what your research is about. That is, be a mentor for the externships of middle and high school students.
			\end{enumerate}
		\end{enumerate}
	\end{enumerate}
%%%%%%%%%%%%%%%%%%%%%%%%%%%%%%%%%%%%%%%
%%%%%%%%%%%%%%%%%%%%%%%%%%%%%%%%%%%%%%%
\item competitions: \vspace{-0.3cm}
	\begin{enumerate} \itemsep -2pt
	\item Invent Now, Inc.: \vspace{-0.2cm}
		\begin{enumerate} \itemsep -2pt
		\item Collegiate Inventors Competition: \url{http://www.invent.org/collegiate/} [ Resources for {\color{blue} Patent Search Strategy} are available. \colorbox{blue}{\bf This is the ultimate competition for US students in science and engineering.} ]
		\end{enumerate}
	\item INFORMS Doing Good with Good OR - Student Competition: \vspace{-0.2cm}
		\begin{enumerate} \itemsep -2pt
		\item Doing Good with Good OR-Student Competition is held each year to identify and honor outstanding projects in the field of operations research and the management sciences conducted by a student or student group that have a significant societal impact.
		\item \url{http://www.informs.org/Recognize-Excellence/INFORMS-Prizes-Awards/Doing-Good-with-Good-OR}
		\end{enumerate}
	\item AWM Essay Contest: Biographies of Contemporary Women in Mathematics; see \url{http://www.awm-math.org/biographies/contest.html}
	\item American Society of Mechanical Engineers (ASME): \vspace{-0.2cm}
		\begin{enumerate} \itemsep -2pt
		\item Student Design Competition: \url{http://www.asme.org/Events/Contests/DesignContest/Student_Design_Competition.cfm}
		\item ASME Student Mechanism and Robot Design Competition: \url{http://www.asme.org/Events/Contests/Student_Mechanism_Robot_2.cfm}
		\end{enumerate}
	\item American Institute of Chemical Engineers (AIChE) competitions: \url{http://www.aiche.org/Students/Awards/index.aspx}
	\item Association for Unmanned Vehicle Systems International (AUVSI): \vspace{-0.2cm}
		\begin{enumerate} \itemsep -2pt
		\item AUVSI Student Competitions: \vspace{-0.1cm}
			\begin{enumerate} \itemsep -1pt
			\item \url{http://www.auvsi.org/AUVSI/AUVSI/Home/Default.aspx}, or \url{http://www.auvsi.org/}
			\item Annual Intelligent Ground Vehicle Competition (IGVC): \url{http://www.igvc.org/}
			\item Annual Student Unmanned Air System (SUAS) Competition: \url{http://65.210.16.57/studentcomp2010/default.html}
			\item International Aerial Robotics Competition (IARC): \url{http://iarc.angel-strike.com/}
			\item AUVSI and ONR's International Autonomous Surface Vehicle (ASV) Competition [ASVC]
			\item AUVSI Foundation and ONR's (U.S. Office of Naval Research) 4th International RoboBoats Competition: \url{http://www.auvsifoundation.org/AUVSI/FOUNDATION/Competitions/ASVCompetition/Default.aspx?C=00000000-0000-0000-0000-000000000000}
			\item AUVSI Foundation and ONR's (U.S. Office of Naval Research) International RoboSub Competition (or AUVSI and ONR's International Autonomous Underwater Vehicle Competition): \url{http://www.auvsifoundation.org/AUVSI/FOUNDATION/Competitions/AUVCompetition/Default.aspx}
			\item ONR: U.S. Office of Naval Research
			\end{enumerate}
		\end{enumerate}
	\item American Institute of Aeronautics and Astronautics (AIAA): \vspace{-0.2cm}
		\begin{enumerate} \itemsep -2pt
		\item Design Competitions: \url{http://www.aiaa.org/content.cfm?pageid=210}
		\end{enumerate}
	\item National Aeronautics and Space Administration: \vspace{-0.2cm}
		\begin{enumerate} \itemsep -2pt
		\item NASA's Langley Research Center: \vspace{-0.1cm}
			\begin{enumerate} \itemsep -1pt
			\item SpaceTech Engineering Design Challenge: \url{http://spacetech.larc.nasa.gov}
			\end{enumerate}
		\end{enumerate}
	\item American Concrete Institute (ACI): \vspace{-0.2cm}
		\begin{enumerate} \itemsep -2pt
		\item Competitions: \url{http://www.concrete.org/STUDENTS/st_competitions.htm}
		\end{enumerate}
	\end{enumerate}
%%%%%%%%%%%%%%%%%%%%%%%%%%%%%%%%%%%%%%%
%%%%%%%%%%%%%%%%%%%%%%%%%%%%%%%%%%%%%%%
\item underrepresented minorities: \vspace{-0.3cm}
	\begin{enumerate} \itemsep -2pt
	\item The Society of Women Engineers: \url{http://societyofwomenengineers.swe.org/}
	\item Association for Women in Science (AWIS): \url{http://www.awis.org/} and \url{http://www.awis.affiniscape.com/displaycommon.cfm?an=1&subarticlenbr=19}
	\item Association for Women in Mathematics (AWM): \vspace{-0.2cm}
		\begin{enumerate} \itemsep -2pt
		\item \url{http://www.awm-math.org/}
		\item Education: \vspace{-0.1cm}
			\begin{enumerate} \itemsep -1pt
			\item \url{http://sites.google.com/site/awmmath/awm-resources/education}
			\item Includes information for students in middle school, high school, college and university (including graduate students). It also includes information for parents and teachers/educators.
			\end{enumerate}
		\item Career advice and opportunities: \url{http://sites.google.com/site/awmmath/awm-resources/career}
		\item Women in Math, Science, and Society: \url{http://sites.google.com/site/awmmath/women-in-math-science-and-society}
		\item Essay contest on biographies of contemporary women in mathematics: \url{http://sites.google.com/site/awmmath/programs/essay-contest}
		\end{enumerate}
	\item Sigma Delta Epsilon-Graduate Women in Science (GWIS): \url{http://www.gwis.org/}
	\item Society of Hispanic Professional Engineers (SHPE): \vspace{-0.2cm}
		\begin{enumerate} \itemsep -2pt
		\item Advancing Hispanic Excellence in Technology, Engineering, Math and Science (AHETEMS) Foundation: \url{http://www.ahetems.org/}
		\item AHETEMS Scholarship Program: \url{http://www.ahetems.org/scholarships/}
		\item Graduate \& Young Professional Fellowship Program (encourage young professionals to engage in {\bf public policy}): \url{http://www.ahetems.org/graduate/graduate-young-professional-fellowship-program/}
		\item SHPE/GEM Fellowship (for graduate students in STEM at GEM Member Universities): \url{http://www.ahetems.org/graduate/shpe-gem-graduate-award/}. See \url{http://www.gemfellowship.org/gem-universities/university-members} for a list of GEM member universities.
		\item Internship opportunities: \url{http://www.ahetems.org/scholar-internships/}
		\item \url{http://oneshpe.shpe.org/wps/portal/national}
		\end{enumerate}
	\item National Society of Black Engineers (NSBE): \vspace{-0.2cm}
		\begin{enumerate} \itemsep -2pt
		\item Scholarships: \url{http://www.nsbe.org/Programs/Scholarships.aspx}
		\item Competitions for undergraduates and graduate students: \url{http://www.nsbe.org/Programs/Competitions/Collegiate.aspx}
		\item \url{http://www.nsbe.org/}
		\end{enumerate}
	\item The Society of Mexican American Engineers and Scientists (MAES): \vspace{-0.2cm}
		\begin{enumerate} \itemsep -2pt
		\item MAES Undergraduate and Graduate Outreach Programs (including ``GRE/Graduate Application Fee Waivers''): \url{http://www.maes-natl.org/index.php?module=ContentExpress&func=display&ceid=90&meid=238}
		\item Scholarships \& Awards: \url{http://www.maes-natl.org/index.php?meid=328}
		\item MAES Scholarship Program: \url{http://www.maes-natl.org/index.php?module=ContentExpress&func=display&ceid=518&meid=241}
		\end{enumerate}
	\item SACNAS (Society for Advancement of Chicanos and Native Americans in Science): \vspace{-0.2cm}
		\begin{enumerate} \itemsep -2pt
		\item Scholarships: \url{http://www.sacnas.org/webadindex.cfm?webadcategory_id=7}
		\item Fellowships: \url{http://www.sacnas.org/webadIndex.cfm?webadcategory_id=5}
		\end{enumerate}
	\item {\it Center for the Advancement of Hispanics in Science and Engineering Education} (CAHSEE): \vspace{-0.2cm}
		\begin{enumerate} \itemsep -2pt
		\item YESP - Young Engineers \& Scientists Program: \url{http://www.cahsee.org/2programs/yesp.asp.htm}. ``This program places talented Hispanic college students in the research labs of government agencies.''
		\item Scholarships: \url{http://www.cahsee.org/6resources/scholarships.asp.htm}
		\end{enumerate}
	\item American Geophysical Union: \vspace{-0.2cm}
		\begin{enumerate} \itemsep -2pt
		\item Has a list of organizations for specific underrepresented ethnic-minority groups in the geosciences and physics: \vspace{-0.1cm}
			\begin{enumerate} \itemsep -1pt
			\item \url{http://www.agu.org/education/diversity_programs/}
			\item These organizations may have information about scholarships, fellowships, and educational material for K-12 and college students.
			\end{enumerate}
		\end{enumerate}
	\item Institute for Broadening Participation: \vspace{-0.2cm}
		\begin{enumerate} \itemsep -2pt
		\item Minorities Striving and Pursuing Higher Degrees of Success in Earth System Science (MS PHD'S\textregistered) initiative: \vspace{-0.1cm}
			\begin{enumerate} \itemsep -1pt
			\item \url{http://www.msphds.org/}
			\item Prospective Students/Mentees: \url{http://www.msphds.org/prospective.asp}
			\item For MS PHD'S Students: \url{http://www.msphds.org/students.asp}
			\end{enumerate}
		\item PathwaysToScience.org: \vspace{-0.1cm}
			\begin{enumerate} \itemsep -1pt
			\item Resources for undergraduate students: \url{http://www.pathwaystoscience.org/Undergrads.asp}
			\item Resources for graduate students: \url{http://www.pathwaystoscience.org/Grad.asp}
			\item Resources for postdocs: \url{http://www.pathwaystoscience.org/Postdocs_portal.asp}
			\item STEM Resources by Institution (colleges, universities, and US national research laboratories): \url{http://www.pathwaystoscience.org/Institution.asp}
			\item Additional resources: \url{http://www.pathwaystoscience.org/Ideaexchange.asp}
			\end{enumerate}
		\item National Alliance for Doctoral Studies in the Mathematical Sciences: \vspace{-0.1cm}
			\begin{enumerate} \itemsep -1pt
			\item \url{http://www.mathalliance.org/}
			\item Student/Alliance Scholars: \url{http://www.mathalliance.org/scholars.asp}
			\item Alliance Mentors / Alliance Undergraduate Mentors: \url{http://www.mathalliance.org/mentors.asp}
			\item Alliance Programs: \url{http://www.mathalliance.org/programs.asp}
			\end{enumerate}
		\item Alliances for Graduate Education and the Professoriate (AGEP): \vspace{-0.1cm}
			\begin{enumerate} \itemsep -1pt
			\item \url{http://www.agep.us/}
			\end{enumerate}
		\item Maine Pathways to STEM (Science, Technology, Engineering \& Mathematics): \vspace{-0.1cm}
			\begin{enumerate} \itemsep -1pt
			\item \url{http://www.mainestem.org/}
			\item K-12 Teachers \& University Faculty: \url{http://www.mainestem.org/METeachersFaculty.asp}
			\item Graduate \& Undergraduate Students: \url{http://www.mainestem.org/MEUndergradGrad.asp}
			\end{enumerate}
		\end{enumerate}
	\item ARTSI (Advancing Robotics Technology for Societal Impact) Alliance: \vspace{-0.2cm}
		\begin{enumerate} \itemsep -2pt
		\item \url{http://artsialliance.org/}
		\item ``The ARTSI (Advancing Robotics Technology for Societal Impact) Alliance is a collaborative education and research project centered around robotics for healthcare, the arts, and entrepreneurship.  Spelman College, a historically black college (HBCU) for women is leading the alliance in partnership with several other HBCUs and Research I (R1) institutions.''
		\item Summer REU (Research Experience for Undergraduates) program: \url{http://artsialliance.org/Summer-REU-Program}
		\end{enumerate}
	\item Women in Technology (WIT): \vspace{-0.2cm}
		\begin{enumerate} \itemsep -2pt
		\item \url{http://www.womenintechnology.org/index.asp}
		\item WIT Mentor-Prot{\'{e}}g{\'{e}} Program: \url{http://www.womenintechnology.org/content.asp?contentid=59}
		\item {\bf \color{blue} WIT Career Transition Resource Guide}: \url{http://www.womenintechnology.org/content.asp?contentid=146}
		\item Girls In Technology (GIT): \vspace{-0.1cm}
			\begin{enumerate} \itemsep -1pt
			\item Get Involved: \vspace{-0.1cm}
				\begin{itemize} \itemsep -1pt
				\item \url{http://www.girlsintechnology.org/getinvolved.cfm}
				\item Teacher: teach girls about IT as an after-school activity or in a summer camp session
				\item Assistant Teacher: Assist instructors in GIT sessions, after-school activities, or summer camp sessions
				\item Develop Curriculum: Develop a curriculum for a supported GIT educational program
				\item Mentor: Mentor a girl in one of [GIT's] supported programs
				\item Job Shadow: ``Let a girl shadow you at work''
				\item Guest Speaker: ``Speak to a group of girls on a topic both you and they enjoy, such as computers, technology, education, how to take apart computers, how to build a web site, etc.''
				\end{itemize}
			\end{enumerate}
		\end{enumerate}
	\item Arizona State University: \vspace{-0.2cm}
		\begin{enumerate} \itemsep -2pt
		\item {\it Career}WISE: \vspace{-0.1cm}
			\begin{enumerate} \itemsep -1pt
			\item \url{http://careerwise.asu.edu/}
			\item Helpful resources for female graduate/Ph.D. students in science and engineering.
			\end{enumerate}
		\end{enumerate}
	\item American Indian Science and Engineering Society (AISES): \vspace{-0.2cm}
		\begin{enumerate} \itemsep -2pt
		\item Programs for undergraduates and grad students (including scholarships and internships): \vspace{-0.1cm}
			\begin{enumerate} \itemsep -1pt
			\item \url{http://www.aises.org/Programs}
			\item Resources: \url{http://www.aises.org/Programs/Resources}
			\end{enumerate}
		\end{enumerate}
	\end{enumerate}
\end{enumerate}




%%%%%%%%%%%%%%%%%%%%%%%%%%%%%%%%%%%%%%%%%%%
\subsection{Other Science and Engineering Outreach}
\label{otherstemoutreach}

Other Science and Engineering Outreach: \vspace{-0.3cm}
\begin{enumerate} \itemsep -4pt
\item Frontiers of Engineering (networking event for mid-career engineers): \url{http://www.naefrontiers.org/}
\item Consortium for Ocean Leadership: \vspace{-0.3cm}
	\begin{enumerate} \itemsep -2pt
	\item Resources for scientists in the marine sciences to use in outreach activities: \url{http://www.oceanleadership.org/education/deep-earth-academy/scientists/}
	\end{enumerate}
\item The Oceanography Society: \vspace{-0.3cm}
	\begin{enumerate} \itemsep -2pt
	\item Education and Public Outreach (EPO): A Guide for Scientists [material that scientists and professors can use for outreach activities], \url{http://www.tos.org/epo_guide/index.html}
	\end{enumerate}
\item The Joy McCann Foundation: \vspace{-0.3cm}
	\begin{enumerate} \itemsep -2pt
	\item McCann Scholar (for professors in medicine, science, and nursing): \url{http://www.mccannfoundation.org/scholars.htm}
	\item The Joy McCann Professorship for Women in Medicine: \url{http://www.mccannfoundation.org/medicine.htm}
	\end{enumerate}
\item U.S. National Academies: \vspace{-0.3cm}
	\begin{enumerate} \itemsep -2pt
	\item International Activities of the U.S. National Academies -- Science, Engineering \& Medicine: Working toward a better world: \vspace{-0.2cm}
		\begin{enumerate} \itemsep -2pt
		\item \url{http://sites.nationalacademies.org/International/}
		\item Solving the grand challenges: \vspace{-0.1cm}
			\begin{enumerate} \itemsep -1pt
			\item Energy and the Environment
			\item Global Health
			\item Water Resources
			\item Agriculture and Food Security
			\item International Security
			\item Population
			\end{enumerate}
		\item Help other countries build/improve their capacities: \vspace{-0.1cm}
			\begin{enumerate} \itemsep -1pt
			\item Cooperative Program with Pakistan 
			\item African Science Academies 
			\item Visiting Math Lecturer Program in Cambodia 
			\item Humanitarian Relief Efforts
			\item Improved Road Safety
			\item Science-based Decision Making for Sustainability
			\item Science Academies' Input to G8 Summits
			\end{enumerate}
		\item Scientific Cooperation: \vspace{-0.1cm}
			\begin{enumerate} \itemsep -1pt
			\item Building Bridges in the Middle East
			\item Cooperation with Iran
			\item Human Rights
			\item Frontiers of Science and Engineering Symposia
			\item Travel Grants
			\item International Conference on Women's Issues in Transportation
			\end{enumerate}
		\item Advising the U.S. Government: \vspace{-0.1cm}
			\begin{enumerate} \itemsep -1pt
			\item Science \& Technology in Foreign Policy
			\item Health 
			\item Science and Security
			\end{enumerate}
		\end{enumerate}
	\end{enumerate}
\item National Academy of Engineering: \vspace{-0.3cm}
	\begin{enumerate} \itemsep -2pt
	\item The Charles Stark Draper Prize (``to recognize innovative engineering achievements and their reduction to practice in ways that have led to important benefits and significant improvement in the well being and freedom of humanity''): \url{http://www.draperprize.org/}
	\item NAE Grand Challenge Scholars Program: \url{http://www.grandchallengescholars.org/}
	\end{enumerate}
\item United States Department of Defense (DoD): \vspace{-0.3cm}
	\begin{enumerate} \itemsep -2pt
	\item National Defense Education Program; Defense Advanced Research Projects Agency (DARPA): \vspace{-0.2cm}
		\begin{enumerate} \itemsep -2pt
		\item Resource for scientists and engineers to mentor youths, so that they would look into pursuing careers in science and engineering: \url{http://www.ndep.us/GetInvoSci.aspx}
		\item STEM Learning Modules (SLM): \vspace{-0.1cm}
			\begin{enumerate} \itemsep -1pt
			\item \url{http://www.ndep.us/ProgSLM.aspx}
			\item Help educators develop programs in science and engineering in K-12 institutions, so that youths would be encouraged to explore careers in science and engineering
			\end{enumerate}
		\end{enumerate}
	\end{enumerate}
\item Hewlett-Packard Development Company: \vspace{-0.3cm}
	\begin{enumerate} \itemsep -2pt
	\item HP Catalyst Initiative (grants for STEM education in colleges and universities): \url{http://www.hp.com/hpinfo/socialinnovation/catalyst.html}
	\item HP EdTech Innovators Award (for higher educational institutions that integrate IT into the curricular): \url{http://www.hp.com/hpinfo/socialinnovation/edtech.html}
	\end{enumerate}
\item The William and Flora Hewlett Foundation (Hewlett Foundation): \vspace{-0.3cm}
	\begin{enumerate} \itemsep -2pt
	\item Funding Programs: \url{http://www.hewlett.org/programs}
	\item Grantseekers: \url{http://www.hewlett.org/grants/grantseekers}
	\end{enumerate}
\item The Sloan Consortium (Sloan-C): \vspace{-0.3cm}
	\begin{enumerate} \itemsep -2pt
	\item Sloan-C Awards (for recognizing outstanding work in the field of online education) and Sloan-C Fellows: \url{http://sloanconsortium.org/aboutus/awards}
	\item Mayadas Leadership Award in Online Education: \url{http://sloanconsortium.org/mayadas_award}
	\end{enumerate}
\item W.K. Kellogg Foundation: \vspace{-0.3cm}
	\begin{enumerate} \itemsep -2pt
	\item Grant database: \url{http://www.wkkf.org/grants/grants-database.aspx}
	\end{enumerate}
\item Hewlett-Packard Company: \vspace{-0.3cm}
	\begin{enumerate} \itemsep -2pt
	\item HP community investment for education, economic development, and the environment: \url{http://www.hp.com/hpinfo/socialinnovation/focus.html}
	\item Entrepreneurship education: \vspace{-0.2cm}
		\begin{enumerate} \itemsep -2pt
		\item \url{http://www.hp.com/hpinfo/globalcitizenship/society/social/entrepreneurship.html}
		\item HP Graduate Entrepreneurship Training through IT (GET-IT)
		\item HP Entrepreneurship Learning Program (HELP)
		\end{enumerate}
	\item HP Innovations in Education grants: \url{http://www.hp.com/hpinfo/globalcitizenship/society/social/innovations.html}
	\end{enumerate}
\item General Electric Company: \vspace{-0.3cm}
	\begin{enumerate} \itemsep -2pt
	\item GE Foundation: \vspace{-0.2cm}
		\begin{enumerate} \itemsep -2pt
		\item Developing Futures\texttrademark\ in Education program (which encompasses the GE College Bound Program): \url{http://www.ge.com/foundation/developing_futures_in_education/index.jsp}
		\item Environment, health and safety, and health industry training programs (outside the US): \url{http://www.ge.com/foundation/international_programs/training.jsp}
		\item Student, education and scholarship initiatives: \url{http://www.ge.com/foundation/international_programs/education_initiatives.jsp}
		\end{enumerate}
	\end{enumerate}
\item The GRAMMY Foundation: \vspace{-0.3cm}
	\begin{enumerate} \itemsep -2pt
	\item GRAMMY Foundation Grants: \vspace{-0.2cm}
		\begin{enumerate} \itemsep -2pt
		\item \url{http://www2.grammy.com/GRAMMY_Foundation/Grants/}
		\item It funds {\bf Scientific Research Projects} as well as {\it Archiving And Preservation Projects}.
		\item Concerning scientific research projects: ``The GRAMMY Foundation Grant Program awards grants to organizations and individuals to support research on the impact of music on the human condition. Examples might include the study of the effects of music on mood, cognition and healing, as well as the medical and occupational well-being of music professionals and the creative process underlying music.'' [ E.g., look at music therapy as a possible research topic/area. ]
		\end{enumerate}
	\end{enumerate}
\item The Dana Foundation: \vspace{-0.3cm}
	\begin{enumerate} \itemsep -2pt
	\item \url{http://www.dana.org/grants/}
	\item Has grants for: \vspace{-0.2cm}
		\begin{enumerate} \itemsep -2pt
		\item Brain and Immuno-Imaging
		\item Clinical Neuroscience
		\item Human Immunology
		\item Neuroimmunology of Brain Infections and Cancers
		\end{enumerate}
		\item Deadlines and Requests for Proposals (RFP): \url{http://www.dana.org/grants/deadlines.aspx}
	\end{enumerate}
%%%%%%%%%%%%%%%%%%%%%%%%%%%%%%%%%%%%%%%
% underrepresented minorities
\item Institute for Broadening Participation: \vspace{-0.3cm}
	\begin{enumerate} \itemsep -2pt
	\item PathwaysToScience.org: \vspace{-0.2cm}
		\begin{enumerate} \itemsep -2pt
		\item Resources for faculty and administrators (to facilitate STEM outreach activities as well as the recruitment of underrepresented minorities to the student body and faculty): \url{http://www.pathwaystoscience.org/Faculty.asp}
		\end{enumerate}
	\end{enumerate}
\item National Center for Women \& Information Technology (NCWIT): \vspace{-0.3cm}
	\begin{enumerate} \itemsep -2pt
	\item NCWIT Academic Alliance Seed Fund (for developing and implementing initiatives in colleges and universities to recruit and retain women in computing and information technology): \url{http://www.ncwit.org/work.awards.seed.html}
	\item NCWIT Symons Innovator Award (for outstanding women who have successfully built and funded an IT business): \url{http://www.ncwit.org/work.awards.innovator.html}
	\end{enumerate}
\item Women in Technology (WIT): \vspace{-0.3cm}
	\begin{enumerate} \itemsep -2pt
	\item Girls In Technology (GIT): \vspace{-0.2cm}
		\begin{enumerate} \itemsep -2pt
		\item Get Involved: \vspace{-0.1cm}
			\begin{itemize} \itemsep -1pt
			\item \url{http://www.girlsintechnology.org/getinvolved.cfm}
			\item Teacher: teach girls about IT as an after-school activity or in a summer camp session
			\item Assistant Teacher: Assist instructors in GIT sessions, after-school activities, or summer camp sessions
			\item Develop Curriculum: Develop a curriculum for a supported GIT educational program
			\item Mentor: Mentor a girl in one of [GIT's] supported programs
			\item Job Shadow: ``Let a girl shadow you at work''
			\item Guest Speaker: ``Speak to a group of girls on a topic both you and they enjoy, such as computers, technology, education, how to take apart computers, how to build a web site, etc.''
			\end{itemize}
		\end{enumerate}
	\end{enumerate}
\item European Platform of Women Scientists (EPWS): \vspace{-0.3cm}
	\begin{enumerate} \itemsep -2pt
	\item \url{http://www.epws.org/}
	\item Members: \url{http://www.epws.org/index.php?option=com_content&task=blogcategory&id=134&Itemid=4652}
	\end{enumerate}
\end{enumerate}





Commercializing academic research into products and services via start-ups: \vspace{-0.3cm}
\begin{enumerate} \itemsep -4pt
\item Ben Franklin Technology Partners (BFTP): \vspace{-0.3cm}
	\begin{enumerate} \itemsep -2pt
	\item Innovation Works (IW): \vspace{-0.2cm}
		\begin{enumerate} \itemsep -2pt
		\item For universities in the Pittsburgh metropolitan area
		\item University Innovation Grants (UIGs) / University Grants: \vspace{-0.1cm}
			\begin{enumerate} \itemsep -1pt
			\item For technology validation, market research, prototype development, and intellectual property evaluation
			\item Available online at: \url{http://www.innovationworks.org/OurPrograms/UniversityGrants/tabid/115/Default.aspx}; last accessed on November 14, 2010.
			\end{enumerate}
		\end{enumerate}
	\end{enumerate}
\end{enumerate}









%%%%%%%%%%%%%%%%%%%%%%%%%%%%%%%%%%%%%%%%%%%
\subsection{Electrical and Computer Engineering \& Computer Science Outreach}
\label{ececsoutreach}

Electrical and computer engineering, and computer science outreach: \vspace{-0.3cm}
\begin{enumerate} \itemsep -4pt
\item IEEE: \vspace{-0.3cm}
	\begin{enumerate} \itemsep -2pt
	\item {\it IEEE-USA Salary Service} provides a survey of jobs in electrical and computer engineering: \url{http://www.ieeeusa.org/careers/salary/}
	\item {\it IEEE Santa Clara Valley Section PACE}: Professional Activities Committee for Engineers (PACE); see \url{http://www.ewh.ieee.org/r6/scv/PACE/}
	\item {\it IEEE Santa Clara Valley Section}: \url{http://ewh.ieee.org/r6/scv/} and \url{http://www.ieee.org/scv}
	\item 
	\end{enumerate}
\item Association for Computing Machinery, ACM: \vspace{-0.3cm}
	\begin{enumerate} \itemsep -2pt
	\item Sanjeev Arora, Boaz Barak, and Luca Trevisan, ``Survey Papers and Essays,'' in {\it Theory Matters Wiki: Theoretical Computer Science (TCS) Advocacy Wiki}, SIGACT Committee for the Advancement of Theoretical Computer Science, ACM Special Interest Group on Algorithms and Computation Theory (SIGACT), Association for Computing Machinery, February 25, 2010. Available at: \url{http://theorymatters.org/pmwiki/pmwiki.php?n=Main.SurveyCollection}; last accessed on September 14, 2010.
	\item Online Resources for Graduating Students: \url{http://www.acm.org/membership/student/resources-for-grads}
	\end{enumerate}
\item VLSI design and verification: \vspace{-0.3cm}
	\begin{enumerate} \itemsep -2pt
	\item {\it DVClub} for individuals interested in VLSI verification: \url{http://www.dvclub.org/}
	\item {\it DeepChip.com}: \url{http://www.deepchip.com}
	\end{enumerate}
%%%%%%%%%%%%%%%%%%%%%%%%%%%%%%%
\item undergraduates: \vspace{-0.3cm}
	\begin{enumerate} \itemsep -2pt
	\item {\it Humanitarian FOSS Project}: \vspace{-0.2cm}
		\begin{enumerate} \itemsep -2pt
		\item Where FOSS refers to Free and Open Source Software
		\item For computer science and engineering students
		\item \url{http://www.hfoss.org/}
		\end{enumerate}
	\item {\it SIGDA Design Automation Summer School}: \vspace{-0.2cm}
		\begin{enumerate} \itemsep -2pt
		\item {\it NSF�SRC�SIGDA�DAC Design Automation Summer School}
		\item \url{http://www.sigda.org/dass.html}
		\item Travel grants are provided to defray travel and accommodation expenses
		\end{enumerate}
	\item {\it Young Student Support Program at DAC}: \vspace{-0.2cm}
		\begin{enumerate} \itemsep -2pt
		\item Also known as {\it DAC Young Student Support Program}
		\item \url{http://www.sigda.org/youngstudent.html}
		\item Travel grants are provided to defray travel and accommodation expenses
		\end{enumerate}
	\item {\it ACM Student Research Competition at Design Automation Conference}: \vspace{-0.2cm}
		\begin{enumerate} \itemsep -2pt
		\item Sponsored by {\it Microsoft Research}
		\item \url{http://www.sigda.org/studentcomp.html}
		\item Also, see {\it ACM Student Research Competition} @ \url{http://src.acm.org/}.
		\end{enumerate}
	\item Job database for positions in the Video Game, Animation, VFX, and Software/Technology industries: \url{http://www.creativeheads.net/}
	\end{enumerate}
%%%%%%%%%%%%%%%%%%%%%%%%%%%%%%
\item graduate students: \vspace{-0.3cm}
	\begin{enumerate} \itemsep -2pt
	\item {\it SIGDA Design Automation Summer School}: \vspace{-0.2cm}
		\begin{enumerate} \itemsep -2pt
		\item {\it NSF�SRC�SIGDA�DAC Design Automation Summer School}
		\item \url{http://www.sigda.org/dass.html}
		\item Travel grants are provided to defray travel and accommodation expenses
		\end{enumerate}
	\item {\it Young Student Support Program at DAC}: \vspace{-0.2cm}
		\begin{enumerate} \itemsep -2pt
		\item Also known as {\it DAC Young Student Support Program}
		\item \url{http://www.sigda.org/youngstudent.html}
		\item Travel grants are provided to defray travel and accommodation expenses
		\end{enumerate}
	\item {\it ACM Student Research Competition at Design Automation Conference}: \vspace{-0.2cm}
		\begin{enumerate} \itemsep -2pt
		\item Sponsored by {\it Microsoft Research}
		\item \url{http://www.sigda.org/studentcomp.html}
		\item Also, see {\it ACM Student Research Competition} @ \url{http://src.acm.org/}.
		\end{enumerate}
	\item {\it SIGDA University Booth at DAC}: \vspace{-0.2cm}
		\begin{enumerate} \itemsep -2pt
		\item Or, {\it SIGDA/DAC University Booth}
		\item \url{http://www.sigda.org/ubooth.html}
		\end{enumerate}
	\item {\it SIGDA Ph.D. Forum at DAC}: \vspace{-0.2cm}
		\begin{enumerate} \itemsep -2pt
		\item \url{http://www.sigda.org/phdforum.html}
		\item \url{http://www.sigda.org/daforum/}
		\end{enumerate}
	\item {\it DAC Graduate Scholarship}: \vspace{-0.2cm}
		\begin{enumerate} \itemsep -2pt
		\item {\it A. Richard Newton Graduate Scholarships} to Support Graduate Research and Study
		\item \url{http://www.sigda.org/gradscholarship.html}
		\end{enumerate}
	\end{enumerate}
%%%%%%%%%%%%%%%%%%%%%%%%%%%%%%
\item competitions, and programming contests and challenges: \vspace{-0.3cm}
	\begin{itemize} \itemsep -2pt
	\item {\it SIGDA CADathlon at ICCAD}: \vspace{-0.2cm}
		\begin{enumerate} \itemsep -2pt
		\item \url{http://www.sigda.org/programs/cadathlon/}
		\item \url{http://www.sigda.org/cadathlon.html}
		\item Travel grants are provided to defray travel and accommodation expenses
		\end{enumerate}
	\item ISPD Programming Contest: \url{http://www.ispd.cc/contests/}
	\item ACM International Workshop on Timing Issues in the Specification and Synthesis of Digital Systems (TAU Workshop): \vspace{-0.2cm}
		\begin{enumerate} \itemsep -2pt
		\item Power Grid Simulation Contest: \url{http://www.tauworkshop.com/PREVIOUS/contest_2011.html}
		\end{enumerate}
	\item IEEE Computer Society Simulator Design competition: \url{http://www.computer.org/portal/web/competition}
	\item {\it DAC/ISSCC Student Design Contest}: \vspace{-0.2cm}
		\begin{enumerate} \itemsep -2pt
		\item \url{http://www.dac.com}
		\end{enumerate}
	\item {\it ACM/IEEE International Conference on Formal Methods and Models for Codesign -- Design Contest}: \vspace{-0.2cm}
		\begin{enumerate} \itemsep -2pt
		\item MEMOCODE Hardware/Software Co-Design Contest (MEMOCODE HW/SW co-design contest)
		\item \url{http://www-memocode2010.imag.fr/}
		\item \url{http://memocode2010.csail.mit.edu/redmine/wiki/memocode2010/Results}
		\end{enumerate}
	\item {\it International Low Power Design Contest}: \vspace{-0.2cm}
		\begin{enumerate} \itemsep -2pt
		\item ACM/IEEE International Symposium on Low Power Electronics and Design (ISLPED) -- Design Contest
		\item The International Symposium on Low Power Electronics and Design is holding the International Low Power Design Contest to provide a forum for universities and research organizations to showcase original ``power-aware'' designs and to highlight the innovations and design choices targeted at low power.
		\item The goal is to encourage and highlight design-oriented approaches to power reduction.
		\item \url{http://www.islped.org/2010/index.html}
		\end{enumerate}
	\item {\it University LSI Design Contest @ ASP-DAC}: \vspace{-0.2cm}
		\begin{enumerate} \itemsep -2pt
		\item Application areas or types of circuits of the original LSI circuit designs include (but are not limited to): \vspace{-0.1cm}
			\begin{enumerate} \itemsep -1pt
			\item Analog, RF and Mixed-Signal Circuits
			\item Digital Signal Processing
			\item Microprocessors
			\item Custom ASIC
			\end{enumerate} 
		\item Methods or technology used for implementation include: \vspace{-0.1cm}
			\begin{enumerate} \itemsep -1pt
			\item Full Custom and Cell-Based LSIs
			\item Gate Arrays
			\item FPGA/PLDs.
			\end{enumerate}
		\item \url{http://www.aspdac.com/aspdac2011/cfd/}
		\end{enumerate}
	\item IEEE Programming Challenge at IWLS: \url{http://www.iwls.org/challenge/}
	\item IEEE Asian Solid-State Circuits Conference (A-SSCC) Student Design Contest: \url{http://a-sscc2010.a-sscc.org/contest.html}
	\item {\it VLSI Conference 2011 - Design Contest}: \vspace{-0.2cm}
		\begin{enumerate} \itemsep -2pt
		\item Design/project fields include (but not limited to): \vspace{-0.1cm}
			\begin{enumerate} \itemsep -1pt
			\item Digital Integrated Circuits
			\item Analog Integrated Circuits
			\item FPGA based designs
			\item Computer Architectures/ Processors
			\item Reconfigurable Computing Systems
			\item SoC / Platform-based designs
			\item Embedded Systems
			\item MEMS/Optics/Bio-Chips
			\item Innovative Design Methodologies and Verification Techniques.
			\end{enumerate}
		\item \url{http://vlsiconference.com/vlsi2011/submissions_design_contest.html}
		\end{enumerate}
	\item {\it Satisfiability Modulo Theories Competition} (SMT-COMP): \vspace{-0.2cm}
		\begin{enumerate} \itemsep -2pt
		\item Competition for SMT solvers
		\item \url{http://www.smtcomp.org/2010/}
		\end{enumerate}
	\item {\it SAT Competition 201X}, where $X > 0$ \& $X {\it mod} 2 = 1$: \vspace{-0.2cm}
		\begin{enumerate} \itemsep -2pt
		\item The purpose of the competition is to identify new challenging benchmarks and to promote new solvers for the propositional satisfiability problem (SAT) as well as to compare them with state-of-the-art solvers.
		\item \url{http://www.satcompetition.org/}
		\end{enumerate}
	\item {\it SAT-Race 201X}, where $X > 0$ \& $X {\it mod} 2 = 0$: \vspace{-0.2cm}
		\begin{enumerate} \itemsep -2pt
		\item SAT-Race 201X is a competitive event for solvers of the Boolean Satisfiability (SAT) problem. 
		\item In contrast to the SAT Competitions, the focus of SAT-Race is on application benchmarks only.
		\item \url{http://baldur.iti.uka.de/sat-race-2010/}
		\end{enumerate}
	\item Hardware Model Checking Competition (HWMCC): \url{http://fmv.jku.at/hwmcc10/}
	\item {\it CADE ATP System Competition} (CASC): \vspace{-0.2cm}
		\begin{enumerate} \itemsep -2pt
		\item It is a yearly competition of fully automated theorem provers for classical first order logic.
		\item \url{http://www.cs.miami.edu/~tptp/CASC/}
		\end{enumerate}
	\item Apple Design Awards: \url{http://developer.apple.com/wwdc/ada/index.html}
	\item {\it International Constraint Solver Competition}: \vspace{-0.2cm}
		\begin{enumerate} \itemsep -2pt
		\item Also known as: \vspace{-0.2cm}
			\begin{enumerate} \itemsep -2pt
			\item International Constraint Solver Competition (CSP, Max-CSP and Weighted-CSP competition)
			\item International CSP Solver Competition (CSP, Max-CSP and Weighted-CSP competition)
			\end{enumerate}
		\item The Fourth International Constraint Solver Competition (CSC'2009) is organized to improve our knowledge of what is behind the efficiency of constraint satisfaction algorithms, heuristics, solving strategies, and constraint systems.
		\item \url{http://cpai.ucc.ie/}
		\end{enumerate}
	\item International Conference on Field-Programmable Technology (FPT 201X): \vspace{-0.2cm}
		\begin{enumerate} \itemsep -2pt
		\item FPT Design Competition: \url{http://cas.ee.ic.ac.uk/people/as999/FPTDesignComp/}
		\end{enumerate}
	\item International Microwave Symposium: Student Design Competitions -- Jan (includes AMS circuit simulation, and AMS/RF EDA); \url{http://ims2011.org/Technical_Program/Student_Design_Competitions.html}
	\item {\it QBFEVAL'1X}: \vspace{-0.2cm}
		\begin{enumerate} \itemsep -2pt
		\item QBF Solver competition for solvers to determine Quantified Boolean Formula (QBF) satisfiability.
		\item QBFLIB is a collection of instances, solvers, and tools related to Quantified Boolean Formula (QBF) satisfiability. See \url{http://www.qbflib.org/}.
		\item \url{http://www.qbflib.org/index_eval.php}
		\end{enumerate}
	\item {\it Pseudo-Boolean Competition 201X}: \vspace{-0.2cm}
		\begin{enumerate} \itemsep -2pt
		\item Competition for pseudo-Boolean solvers.
		\item \url{http://www.cril.univ-artois.fr/PB10/}
		\end{enumerate}
	\item {\it Answer Set Programming System Competition}: \vspace{-0.2cm}
		\begin{enumerate} \itemsep -2pt
		\item \url{http://dtai.cs.kuleuven.be/events/ASP-competition/}
		\end{enumerate}
	\item {\it Max-SAT Evaluation, Max-SAT 201X}: \vspace{-0.2cm}
		\begin{enumerate} \itemsep -2pt
		\item Competition for Max-SAT solvers
		\item \url{http://www.maxsat.udl.cat/}
		\item \url{http://www.maxsat.udl.cat/09/}
		\end{enumerate}
	\item {\it IEEEXtreme 24 Hour Programming Challenge}: \vspace{-0.2cm}
		\begin{enumerate} \itemsep -2pt
		\item Programming contest for college students
		\item \url{http://portal.ieee.org/web/membership/students/scholarshipsawardscontests/ieeextreme.html}
		\end{enumerate}
	\item {\it ACM International Collegiate Programming Contest} (ACM-ICPC or ICPC): \vspace{-0.2cm}
		\begin{enumerate} \itemsep -2pt
		\item Programming contest for college students
		\item Official web page: \url{http://cm.baylor.edu/welcome.icpc}
		\item Other web resources: \vspace{-0.1cm}
			\begin{enumerate} \itemsep -1pt
			\item {\it Wikipedia}: \url{http://en.wikipedia.org/wiki/ACM_International_Collegiate_Programming_Contest}
			\item {\it }: \url{}
			\item {\it }: \url{}
			\item {\it Valladolid Online Judge Site}: \url{http://acm.uva.es/}
			\item {\it ACMSolver :: Art of Programming Contest, Tips and Tricks for C, C++, Java}: \url{http://www.acmsolver.org/}
			\end{enumerate}
		\item 
		\end{enumerate}
	\item {\it TopCoder} coding and design contests: \vspace{-0.2cm}
		\begin{enumerate} \itemsep -2pt
		\item The contests cover various fields, such as: \vspace{-0.1cm}
			\begin{enumerate} \itemsep -1pt
			\item Algorithm
			\item Conceptualization
			\item Specification
			\item Architecture
			\item Component Design
			\item Component Development
			\item Assembly
			\item Test Scenarios
			\item Test Suites
			\item UI Prototype
			\item Rich Internet Application (RIA) Build
			\item Bug Race
			\item Marathon Match
			\item High School (for high school students)
			\item Copilot Opportunities
			\end{enumerate}
		\item \url{http://www.topcoder.com/}
		\end{enumerate}
	\item IEEE Presidents' Change the World competition: \vspace{-0.2cm}
		\begin{enumerate} \itemsep -2pt
		\item The IEEE Presidents� Change the World Competition recognizes students who develop unique solutions to real-world problems using engineering, science, computing and leadership skills to benefit their community, the world at large, or both. 
		\item \url{http://www.ieeechangetheworld.org/}
		\end{enumerate}
	\item Google Code Jam (programming contest): \url{http://code.google.com/codejam/} and \url{http://en.wikipedia.org/wiki/Google_Code_Jam}
	\item {\it RoboCup}\texttrademark\ competitions: \vspace{-0.2cm}
		\begin{enumerate} \itemsep -2pt
		\item Has different categories, including soccer, rescue operations, and home applications.
		\item \url{http://www.robocup.org/}
		\end{enumerate}
	\item ICFP Programming Contest (ICFP refers to International Conference on Functional Programming): \url{http://icfpcontest.org/}
	\item Student Cluster Competition (SCC): \vspace{-0.2cm}
		\begin{enumerate} \itemsep -2pt
		\item SCC is held at each (annual) SC conference, which is the International Conference for High Performance Computing, Networking, Storage, and Analysis. IEEE Computer Society and the Association for Computing Machinery are the sponsors for this conference.
		\item During SC10, teams consisting of six students, undergraduate and/or high school, will showcase the amazing power of clusters and the ability to utilize open source software to solve interesting and important problems. They will compete in real-time on the exhibit floor to run a workload of real-world applications on clusters of their own design while never exceeding the dictated power limit.
		\item During SC10 in New Orleans, teams will assemble, test and tune their machines and run the HPCC benchmarks until the starting bell rings on Monday night at the Exhibit Opening Gala where they will be given the competition data sets. In full view of conference attendees, teams will execute the prescribed workload while showing progress and science visualization output on large high-resolution displays in their areas. Teams race to correctly complete the greatest number of application runs during the competition period until the close of the exhibit floor on Wednesday evening.
		\item \url{http://sc10.supercomputing.org/?pg=studentcluster.html}
		\end{enumerate}
	\item Cypress Semiconductor Corporation: \vspace{-0.2cm}
		\begin{enumerate} \itemsep -2pt
		\item ARM Cortex-M3 PSoC\textregistered\ 5 Design Challenge: \url{http://www.cypress.com/?id=3271}
		\end{enumerate}
	\item Mentor Graphics: \vspace{-0.2cm}
		\begin{enumerate} \itemsep -2pt
		\item PCB Technology Leadership Awards (PCB design contest): \url{http://www.mentor.com/products/pcb-system-design/tla/index.cfm?v=mentorgraphics&p=handout:tla&a=print_card&g=sdd&s=1x1&c=ocid_2203&cmpid=3911}, or \url{http://www.mentor.com/go/tla}
		\end{enumerate}
	\item INFORMS Data Mining Contest: \vspace{-0.2cm}
		\begin{enumerate} \itemsep -2pt
		\item \url{http://ifors.org/web/call-for-participation-informs-data-mining-contest-2010/}
		\item \url{http://kaggle.com/informs2010}
		\end{enumerate}
	\item INFORMS Doing Good with Good OR - Student Competition: \vspace{-0.2cm}
		\begin{enumerate} \itemsep -2pt
		\item Doing Good with Good OR-Student Competition is held each year to identify and honor outstanding projects in the field of operations research and the management sciences conducted by a student or student group that have a significant societal impact.
		\item \url{http://www.informs.org/Recognize-Excellence/INFORMS-Prizes-Awards/Doing-Good-with-Good-OR}
		\end{enumerate}
	\item HPC Challenge Award Competition: \url{http://www.hpcchallenge.org/}
	\item Sphere Online Judge, SPOJ (programming contest): \url{http://www.spoj.pl/}
	\item High Performance and Scientific Computing Contest (Argonne National Laboratory, U.S. Department of Energy, DOE): \url{https://wiki.alcf.anl.gov/index.php/HPSC_Contest_Information}
	\item Argonne National Laboratory, ANL; Mathematics and Computer Science Division: \vspace{-0.2cm}
		\begin{enumerate} \itemsep -2pt
		\item J. H. Wilkinson Prize for Numerical Software (for developers of numerical software): \url{http://www.mcs.anl.gov/research/opportunities/wilkinsonprize/index.php}
		\end{enumerate}
	\item Society for Industrial and Applied Mathematics, SIAM: \vspace{-0.2cm}
		\begin{enumerate} \itemsep -2pt
		\item SIAM/ACM Prize in Computational Science and Engineering: \url{http://www.siam.org/prizes/sponsored/cse.php}. [ For developers of mathematical and computational tools and methods for the solution of science and engineering. Or, for developers of computational science and engineering software. ]
		\end{enumerate}
	\end{itemize}
	\item Sun HPC Software Programming Challenge (Oracle Corporation): \url{http://wikis.sun.com/display/HPCContest/Home}
%%%%%%%%%%%%%%%%%%%%%%%%%%%%%%
\item News media: \vspace{-0.3cm}
	\begin{itemize} \itemsep -2pt
	\item --- --- --- --- --- --- --- --- --- --- --- --- --- --- --- --- --- --- --- --- --- --- --- --- --- --- --- --- --- --- ---
	\item \colorbox{blue}{\bf News media for Electronic Design Automation}
	% News media for Electronic Design Automation
	\item {\it EDACafe}: \url{http://www.edacafe.com/}
	\item {\it SIGDA E-Newsletter} (SIGDA Electronic Newsletter): \url{http://www.sigda.org/newsletter/}
	\item {\it DeepChip.com}: \url{http://www.deepchip.com}
	\item --- --- --- --- --- --- --- --- --- --- --- --- --- --- --- --- --- --- --- --- --- --- --- --- --- --- --- --- --- --- ---
	\item \colorbox{blue}{\bf News media for Electrical and Computer Engineering}
	% News media for Electrical and Computer Engineering
	\item {\it EE Times} (Electronic Engineering Times): \url{http://www.eetimes.com/}
	\item {\it EDN} (Electrical Design News): \url{http://www.edn.com/}
	\item {\it IEEE Spectrum}: \url{http://spectrum.ieee.org/}
	\item {\it The Institute} (from IEEE): \url{http://www.theinstitute.ieee.org}
	\item {\it IEEE-USA Today's Engineer}: \url{http://www.todaysengineer.org/}
	\item {\it DeepChip.com}: \url{http://www.deepchip.com}
	\item --- --- --- --- --- --- --- --- --- --- --- --- --- --- --- --- --- --- --- --- --- --- --- --- --- --- --- --- --- --- ---
	\item \colorbox{blue}{\bf News media for Computer Science and Engineering, Information Systems, and IT}
	% News media for Computer Science and Engineering, Information Systems, and IT
	\item {\it ACM TechNews}: \url{http://technews.acm.org/}
	\item {\it TechCareers}: \url{http://www.techcareers.com/}
	\item {\it }: \url{}
	\item {\it }: \url{}
	\item {\it }: \url{}
	\item {\it }: \url{}
	\item {\it }: \url{}
	\item {\it }: \url{}
	\item {\it }: \url{}
	\item --- --- --- --- --- --- --- --- --- --- --- --- --- --- --- --- --- --- --- --- --- --- --- --- --- --- --- --- --- --- ---
	\item \colorbox{blue}{\bf Other News Media}
	% Other News Media
	\item {\it iTunes U}
	\item {\it YouTube EDU}
	\end{itemize}
%%%%%%%%%%%%%%%%%%%%%%%%%%%%%%
\item underrepresented minorities: \vspace{-0.3cm}
	\begin{enumerate} \itemsep -2pt
	\item women: \vspace{-0.2cm}
		\begin{enumerate} \itemsep -2pt
		\item IEEE Women in Engineering (WIE): \url{http://www.ieee.org/membership_services/membership/women/index.html?WT.mc_id=WIE_nav1}
		\item ACM-W: \url{http://women.acm.org/}
		\item Computer Research Association's Committee on the Status of Women in Computing Research (CRA-W): \vspace{-0.1cm}
			\begin{enumerate} \itemsep -1pt
			\item \url{http://www.cra-w.org/}
			\item Computing Research Association's Committee on the Status of Women (CRA-W) and the Coalition to Diversify Computing (CDC), {\it CompArch Summer School on Parallel Programming and Architectures}. Available at: \url{http://www.princeton.edu/~archss/}; last accessed on September 3, 2010.
			\end{enumerate}
		\item National Center for Women \& Information Technology: \url{http://www.ncwit.org/}
		\item African-American Women in Technology organization (AAWIT): \url{http://www.aawit.net/09/index.cfm}
		\item Grace Hopper Celebration of Women in Computing (conference for female IT students, professors, and professionals): \url{http://gracehopper.org/} or \url{http://gracehopper.org/2010/}
		\item Anita Borg Institute for Women and Technology: \vspace{-0.1cm}
			\begin{enumerate} \itemsep -1pt
			\item Has many programs for female students and professionals: \url{http://anitaborg.org/}
			\end{enumerate}
		\end{enumerate}
	\end{enumerate}
\end{enumerate}







%%%%%%%%%%%%%%%%%%%%%%%%%%%%%%%%%%%%%%%%%%%
\section{Scholarships, Fellowships, Awards, and Financial Aid}
\label{scholarshipsfinaidawards}

Resources for scholarships, fellowships, and financial aid: \vspace{-0.3cm}
\begin{enumerate} \itemsep -4pt
\item --- --- --- --- --- --- --- --- --- --- --- --- --- --- --- --- --- --- --- --- --- --- --- --- --- --- --- --- --- --- ---
\item \colorbox{blue}{\bf Lists of Scholarships and Fellowships}
% Lists of Scholarships and Fellowships
\item List of scholarships: \vspace{-0.3cm}
	\begin{enumerate} \itemsep -2pt
	\item Engineering Education Service Center, EESC (Engineering): \url{http://www.engineeringedu.com/scholars.html}
	\item High Performance and Embedded Architecture and Compilation, HiPEAC (Computer Science and Engineering): \url{http://www.hipeac.net/all_jobs_op}
	\item Office of Doctoral Programs at USC Viterbi School of Engineering, {\bf University of Southern California}. External Fellowships and other support: \url{http://viterbi.usc.edu/students/phd/fellowships-and-other-support/external-fellowships.htm}. USC Fellowships: \url{http://viterbi.usc.edu/students/phd/fellowships-and-other-support/usc-fellowships.htm}
	\item Columbia College, {\bf Columbia University} in the City of New York: \url{http://www.college.columbia.edu/students/fellowships/catalog}
	\item {\bf New York University} School of Law: \url{http://www.law.nyu.edu/financialaid/supplementalaid/fellowships/index.htm}
	\item Swedish Institute: \vspace{-0.2cm}
		\begin{enumerate} \itemsep -2pt
		\item The Swedish Institute, a government agency, administers over 500 scholarships each year for students and researchers coming to Sweden to pursue their objectives at a Swedish university.
		\item Study in Sweden: scholarships, \url{http://www.studyinsweden.se/Scholarships/}
		\item Swedish Institute (SI): \url{http://www.si.se/English/Navigation/Scholarships-and-exchanges/} [ Has special programs for Pakistanis and Turkish citizens ]
		\end{enumerate}
	\item The Swedish Foundation for International Cooperation in Research and Higher Education (STINT): \vspace{-0.2cm}
		\begin{enumerate} \itemsep -2pt
		\item \url{http://www.stint.se/en}
		\item Scholarships and grants: \url{http://www.stint.se/en/scholarships_and_grants}
		\end{enumerate}
	\item Center for the Advancement of Hispanics in Science and Engineering Education (CAHSEE): \url{http://www.cahsee.org/6resources/scholarships.asp.htm}
	\item University of Wisconsin-Madison: \vspace{-0.2cm}
		\begin{enumerate} \itemsep -2pt
		\item Grants Information Collection: A Cooperating Collection of the Foundation Center Library Network, \url{http://grants.library.wisc.edu/}
		\end{enumerate}
	\item {\it Find A PhD}: \url{http://www.findaphd.com/}
	\item QS World Grad School Tour Scholarships (QS Quacquarelli Symonds Limited): \url{http://graduateschool.topuniversities.com/world-grad-school-tour/scholarships}
	\item GlobalGrant (requires paid access to the list of scholarships and fellowships): \url{http://www.globalgrant.com/en/stipendier.html} and \url{http://www.globalgrant.com/}
	\item Stockholm University: \vspace{-0.2cm}
		\begin{enumerate} \itemsep -2pt
		\item \url{http://www.su.se/pub/jsp/polopoly.jsp?d=777&a=1770}
		\item \url{http://www.su.se/pub/jsp/polopoly.jsp?d=797}
		\item \url{http://www.su.se/pub/jsp/polopoly.jsp?d=788}
		\item \url{http://www.su.se/pub/jsp/polopoly.jsp?d=777&a=1769}
		\end{enumerate}
	\item NordForsk (in Norwegian): \url{http://www.nordforsk.org/index.cfm}
	\item Wallenberg Scholars (in Swedish): \url{http://www.wallenberg.com/default.aspx} or \url{http://www.wallenberg.com/in-english.aspx}
	\item Royal Institute of Technology (in Swedish): \url{http://www.kth.se/aktuellt/stipendier/stipendier-och-anslag-1.2024}
	\item European Commission: \vspace{-0.2cm}
		\begin{enumerate} \itemsep -2pt
		\item Marie Curie Fellowships: \vspace{-0.1cm}
			\begin{enumerate} \itemsep -1pt
			\item \url{http://cordis.europa.eu/fp7/people/home_en.html}
			\item \url{http://ec.europa.eu/research/mariecurieactions/}
			\item \url{http://ec.europa.eu/research/fp6/mariecurie-actions/action/fellow_en.html}
			\item \url{http://www.mariecurie.org/}
			\end{enumerate}
		\item Euraxess: \url{http://ec.europa.eu/euraxess/}
		\item \url{http://ec.europa.eu/index_en.htm}
		\end{enumerate}
	\item Science Please (for research positions in life sciences in The Netherlands and Belgium, including Ph.D. and postdoc positions): \url{http://www.scienceplease.com/} or \url{http://www.scienceplease.com/about-us}
	\item University of Gothenburg: \vspace{-0.2cm}
		\begin{enumerate} \itemsep -2pt
		\item ResearchResearch: \url{http://www.researchresearch.com/} or \url{http://www.gu.se/english/research/scholarships/ResearchResearch/}
		\item Scholarship links: \url{http://www.gu.se/english/research/scholarships/scholarship_links/}
		\item Scholarships at University of Gothenburg: \url{http://www.gu.se/english/research/scholarships/gu/}
		\end{enumerate}
	\item Princeton University; The Graduate School: \url{http://gradschool.princeton.edu/financial/}
	\item National Association for Bilingual Education: \vspace{-0.2cm}
		\begin{enumerate} \itemsep -2pt
		\item List of Scholarships: \url{http://www.nabe.org/scholarship.html}
		\end{enumerate}
	\item {\bf Pennsylvania State University}: \vspace{-0.2cm}
		\begin{enumerate} \itemsep -2pt
		\item Office of Engineering Diversity; Penn State College of Engineering: \vspace{-0.1cm}
			\begin{enumerate} \itemsep -1pt
			\item Undergraduate Student Scholarships: \url{http://www.engr.psu.edu/oed/UnderScholarships.html}
			\item Graduate Student Scholarships: \url{http://www.engr.psu.edu/oed/GradScholarships.html}
			\item High School Student Scholarships: \url{http://www.engr.psu.edu/oed/HighSchoolScholarships.html}
			\item Disabled Student Scholarships: \url{http://www.engr.psu.edu/oed/DisabScholarships.html}
			\item Corporate Office of Engineering Diversity (OED) Scholarships: \url{http://www.engr.psu.edu/oed/OEDScholarships.html}
			\end{enumerate}
		\item University Fellowships Office: \vspace{-0.1cm}
			\begin{enumerate} \itemsep -1pt
			\item \url{http://sites.google.com/site/psuufo/}
			\item Prestigious Scholarships: \url{http://sites.google.com/site/psuufo/prestigious}
			\item Penn State Scholarships: \url{http://sites.google.com/site/psuufo/internal-scholarships}
			\item Other resources: \url{http://sites.google.com/site/psuufo/resources}
			\end{enumerate}
		\end{enumerate}
	\item {\bf Peterson's} college search: \vspace{-0.2cm}
		\begin{enumerate} \itemsep -2pt
		\item {\it College Scholarship Search}: \url{http://www.petersons.com/college-search/scholarship-search.aspx}
		\end{enumerate}
	\item Society for Industrial and Applied Mathematics (SIAM): \vspace{-0.2cm}
		\begin{enumerate} \itemsep -2pt
		\item Fellowship \& Research Opportunities: \url{http://www.siam.org/students/resources/fellowship.php}
		\end{enumerate}
	\item Institute of International Education (IIE): \vspace{-0.2cm}
		\begin{enumerate} \itemsep -2pt
		\item {\it Funding for US Study Online}: \vspace{-0.1cm}
			\begin{enumerate} \itemsep -1pt
			\item \url{http://www.fundingusstudy.org/}
			\end{enumerate}
		\end{enumerate}
	\end{enumerate}
\item --- --- --- --- --- --- --- --- --- --- --- --- --- --- --- --- --- --- --- --- --- --- --- --- --- --- --- --- --- --- ---
\item \colorbox{blue}{\bf Scholarships and Fellowships in Electrical and Computer Engineering}
% Scholarships and Fellowships in Electrical and Computer Engineering
\item IEEE: \vspace{-0.3cm}
	\begin{enumerate} \itemsep -2pt
	\item IEEE Awards, Competitions, and Scholarships: \url{http://www.ieee.org/membership_services/membership/students/awards/index.html}
	\item IEEE Circuits and Systems Society Pre-Doctoral Scholarships: Announced via email from IEEE Circuits and Systems Society
	\item IEEE Power \& Energy Society: \vspace{-0.2cm}
		\begin{enumerate} \itemsep -2pt
		\item G. Ray Ekenstam Memorial Scholarship: \vspace{-0.1cm}
			\begin{enumerate} \itemsep -1pt
			\item \url{http://www.ieee-pes.org/g-ray-ekenstam-memorial-scholarship}
			\item ``The Scholarship Fund awards, on an annual basis, a scholarship to a qualified undergraduate student who seeks an electrical engineering degree in the field of power or a related discipline, from an accredited US university or college.''
			\end{enumerate}
		\end{enumerate}
	\item IEEE Reliability Society: \vspace{-0.2cm}
		\begin{enumerate} \itemsep -2pt
		\item IEEE Reliability Society Scholarship: \url{http://www.ieee.org/portal/cms_docs_relsoc/relsoc/newsflipper/RS_Scholarship_Application.pdf} [ Look under the tab/option on ``Useful Information'' in the panel on the left. ]
		\end{enumerate}
	\end{enumerate}
\item The George Michael Memorial HPC Fellowship Program: \vspace{-0.3cm}
	\begin{enumerate} \itemsep -2pt
	\item The Association of Computing Machinery (ACM), IEEE Computer Society and SC Conference series have established the High Performance Computing (HPC) Ph.D. Fellowship Program. The SC conference is the International Conference for High Performance Computing, Networking, Storage, and Analysis. IEEE Computer Society and the Association for Computing Machinery are the sponsors for this conference.
	\item Every year, up to three fellowship recipients will each receive a stipend of at least \$5,000 (U.S.) for one academic year, plus travel support to attend the SC conference.
	\item See \url{http://sc10.supercomputing.org/?searchterm=fellowship&pg=GeorgeMichaelMemorial.html}
	\end{enumerate}
\item Intel: \vspace{-0.3cm}
	\begin{enumerate} \itemsep -2pt
	\item Intel Foundation Fellowship: \vspace{-0.2cm}
		\begin{enumerate} \itemsep -2pt
		\item Intel Foundation Ph.D. Fellowship % \url{http://intelscholarships.intel.com/}
		\item \url{http://www.intel.com/education/highered/studentprograms/fellowship.htm}
		\item This awards two-year fellowships to Ph.D. candidates pursuing leading-edge work in fields related to Intel's business and research interests.
		\item Fellowships are available at select U.S. universities, by invitation only, and focus on Ph.D. students who have completed at least one year of study.
		\item The fellowship includes a cash award (tuition/fees/stipend), an Intel mentor, and the opportunity to participate in an internship at Intel.
		\end{enumerate}
	\end{enumerate}
\item IBM: \vspace{-0.3cm}
	\begin{enumerate} \itemsep -2pt
	\item \url{http://www-304.ibm.com/jct01005c/university/scholars/phdfellowship}
	\item IBM Ph.D. Fellowship Award
	\item The IBM Ph.D. Fellowship Awards is an intensely competitive program which honors exceptional Ph.D. students in many academic disciplines and areas of study, for example: computer science and engineering, electrical and mechanical engineering , physical sciences (including chemistry, material sciences, and physics), mathematical sciences (including optimization), business sciences (including financial services, communication, and learning/knowledge), and service sciences, management, and engineering.
	\item IBM Herman Goldstine Postdoctoral Fellowship in Mathematical Sciences: \url{http://domino.research.ibm.com/comm/research_projects.nsf/pages/goldstine.index.html}
	\item Josef Raviv Memorial Postdoctoral Fellowship; see \url{http://domino.research.ibm.com/comm/research.nsf/pages/d.compsci.josef.raviv.general.info.html}, \url{http://domino.research.ibm.com/comm/research.nsf/pages/d.compsci.raviv.winner.html}, and \url{http://domino.research.ibm.com/comm/research.nsf/pages/d.compsci.raviv.winner2008.html}
	\end{enumerate}
\item AMD: Ph.D. fellowship, \url{http://developer.amd.com/programs/fellowship/Pages/default.aspx}
\item Qualcomm, {\it Qualcomm Innovation Fellowship} for Ph.D. students in Electrical Engineering and Computer Science at Stanford, UC Berkeley, UCLA, UCSD, and USC: \url{http://www.qualcomm.com/innovation/research/university_relations/innovation_fellowship/qinf10.html}
\item NVIDIA: \vspace{-0.3cm}
	\begin{enumerate} \itemsep -2pt
	\item NVIDIA Fellowship Program; see \url{http://www.nvidia.com/page/fellowship_programs.html}
	\end{enumerate}
\item Automatic RF Techniques Group (ARFTG): \vspace{-0.3cm}
	\begin{enumerate} \itemsep -2pt
	\item Microwave Measurement Student Fellowship (for ``graduate students who show promise and interest in pursuing research related to improvement of radio frequency and microwave measurement techniques''): \url{http://www.arftg.org/student_fellowship.html}
	\end{enumerate}
\item Gallium Arsenide Applications Symposium (GAAS) Association: \vspace{-0.3cm}
	\begin{enumerate} \itemsep -2pt
	\item GAAS PhD Student Fellowship (for Ph.D. students who have accepted papers at the European Microwave Integrated Circuits Conference, EuMIC): \url{http://www.gaas-symposium.org/english/awards_fellowship.htm} and \url{http://www.eumweek.com/2010/EuMIC.asp?id=c}
	\end{enumerate}
\item The Institution of Engineering and Technology, IET: \vspace{-0.3cm}
	\begin{enumerate} \itemsep -2pt
	\item Hudswell International Research Scholarship: \url{http://www.theiet.org/about/scholarships-awards/ambition/postgraduate1/hudswell-what.cfm}
	\item IET Postgraduate Scholarship: \url{http://www.theiet.org/about/scholarships-awards/ambition/postgraduate1/postgrad-what.cfm}
	\end{enumerate}
\item --- --- --- --- --- --- --- --- --- --- --- --- --- --- --- --- --- --- --- --- --- --- --- --- --- --- --- --- --- --- ---
\item \colorbox{blue}{\bf Scholarships and Fellowships in Computer Science}
% Scholarships and Fellowships in Computer Science
\item ACM Special Interest Group on Symbolic and Algebraic Manipulation (SIGSAM): List of Ph.D. positions in computer algebra and symbolic computation, as listed by SIGSAM; see \url{http://www.sigsam.org/opportunities.phtml?searchterm=fellowship}
\item Carnegie Mellon University: \vspace{-0.3cm}
	\begin{enumerate} \itemsep -2pt
	\item women@SCS School of Computer Science: \vspace{-0.2cm}
		\begin{enumerate} \itemsep -2pt
		\item Individuals, Corporations \& Organizations: \url{http://women.cs.cmu.edu/Resources/Funding/}
		\end{enumerate}
	\end{enumerate}
\item IBM: \vspace{-0.3cm}
	\begin{enumerate} \itemsep -2pt
	\item \url{http://www-304.ibm.com/jct01005c/university/scholars/phdfellowship}
	\item IBM Ph.D. Fellowship Award
	\item The IBM Ph.D. Fellowship Awards is an intensely competitive program which honors exceptional Ph.D. students in many academic disciplines and areas of study, for example: computer science and engineering, electrical and mechanical engineering , physical sciences (including chemistry, material sciences, and physics), mathematical sciences (including optimization), business sciences (including financial services, communication, and learning/knowledge), and service sciences, management, and engineering.
	\item IBM Herman Goldstine Postdoctoral Fellowship in Mathematical Sciences: \url{http://domino.research.ibm.com/comm/research_projects.nsf/pages/goldstine.index.html}
	\item Josef Raviv Memorial Postdoctoral Fellowship; see \url{http://domino.research.ibm.com/comm/research.nsf/pages/d.compsci.josef.raviv.general.info.html}, \url{http://domino.research.ibm.com/comm/research.nsf/pages/d.compsci.raviv.winner.html}, and \url{http://domino.research.ibm.com/comm/research.nsf/pages/d.compsci.raviv.winner2008.html}
	\end{enumerate}
\item Computing Innovation Fellows (CIFellows); post my profile on \url{http://cifellows.org/profiles/}; also see \url{http://www.cifellows.org/}
\item Microsoft: \vspace{-0.3cm}
	\begin{enumerate} \itemsep -2pt
	\item Microsoft Research Graduate Women's Scholarship: \url{http://research.microsoft.com/en-us/collaboration/awards/fellows-women.aspx}
	\item Microsoft Research PhD Fellowship: \url{http://research.microsoft.com/en-us/collaboration/awards/apply-us.aspx}
	\end{enumerate}
\item Google: \vspace{-0.3cm}
	\begin{enumerate} \itemsep -2pt
	\item Google Fellowship Program; see \url{http://googleblog.blogspot.com/2009/05/best-and-brightest.html}
	\end{enumerate}
\item NVIDIA: \vspace{-0.3cm}
	\begin{enumerate} \itemsep -2pt
	\item NVIDIA Fellowship Program; see \url{http://www.nvidia.com/page/fellowship_programs.html}
	\end{enumerate}
\item Facebook Ph.D. Fellowship: \url{http://www.facebook.com/careers/fellowship.php}
\item Yahoo! Labs: Yahoo! Key Scientific Challenges Program, \url{http://labs.yahoo.com/ksc}
\item Qualcomm, {\it Qualcomm Innovation Fellowship} for Ph.D. students in Electrical Engineering and Computer Science at Stanford, UC Berkeley, UCLA, UCSD, and USC: \url{http://www.qualcomm.com/innovation/research/university_relations/innovation_fellowship/qinf10.html} and \url{http://www.qualcomm.com/innovation/research/university_relations/innovation_fellowship/}
\item Computing Research Association (CRA): Outstanding Undergraduate Researchers, \url{http://www.cra.org/awards/undergrad-current/}
\item {\color{blue} European Research Consortium for Informatics and Mathematics (ERCIM)}: \vspace{-0.3cm}
	\begin{enumerate} \itemsep -2pt
	\item ERCIM Alain Bensoussan Fellowship Programme (for Ph.D. degree holders in selected research areas): \url{http://fellowship.ercim.eu/} and \url{http://www.ercim.eu/news/283-fellowship-programme}; research areas are listed at: \url{http://fellowship.ercim.eu/home/topic}. Deadlines are on April 30 and September 30 annually.
	\end{enumerate}
\item {\it Theory Matters Wiki}; Theoretical Computer Science (TCS) Advocacy Wiki: \vspace{-0.3cm}
	\begin{enumerate} \itemsep -2pt
	\item Funding Opportunities and Tips: \url{http://theorymatters.org/pmwiki/pmwiki.php?n=Main.FundingOpportunities}
	\end{enumerate}
\item Kurt G{\"{o}}del Research Prize Fellowship: \vspace{-0.3cm}
	\begin{enumerate} \itemsep -2pt
	\item 2 Ph.D. (pre-doctoral) fellowships
	\item 2 post-doctoral fellowships
	\item 1 unrestricted fellowship
	\item $[$Scope of the$]$ original fellowship proposals [includes] the areas of: \vspace{-0.2cm}
		\begin{enumerate} \itemsep -2pt
		\item set theory
		\item recursion theory
		\item proof theory/intuitionism
		\item model theory
		\item computer assisted reasoning
		\item philosophy of mathematics 
		\end{enumerate}
	\item All fellowship proposals, regardless of subject area, will be judged according to: \vspace{-0.2cm}
		\begin{enumerate} \itemsep -2pt
		\item the relevance and resemblance of the research (finished and proposed) to the great insights and originality of Kurt G{\"{o}}del
		\item its general interest and clarity of motivation
		\item its rigorous scientific quality and depth. 
		\end{enumerate}
	\item \url{http://fellowship.logic.at/}
	\end{enumerate}
\item Hewlett-Packard Company: \vspace{-0.3cm}
	\begin{enumerate} \itemsep -2pt
	\item Hewlett-Packard Labs India (Bengaluru / Bangalore): \vspace{-0.2cm}
		\begin{enumerate} \itemsep -2pt
		\item {\it BITS - HP Labs India Ph.D. Fellowship} for Research related to Information Technologies: \vspace{-0.1cm}
			\begin{enumerate} \itemsep -1pt
			\item \url{http://www.hpl.hp.com/india/bits-hplindia_phd/index.html} or \url{http://www.hpl.hp.com/india/bits-hplindia_phd/}
			\item \url{http://www.hpl.hp.com/india/bits-hplindia_phd/iiitbphd.html}
			\item BITS, Pilani and HP Labs India jointly offer a unique PhD fellowship for research in Information and Communication Technologies (ICT) relevant to fast-growing markets like India.
			\item HP Labs India currently has ongoing Ph.D. Fellowships with BITS Pilani and IIIT, Bangalore: \url{http://www.hpl.hp.com/india/bits-hplindia_phd/university.html}
			\end{enumerate}
		\item Open Innovation Office: \vspace{-0.1cm}
			\begin{enumerate} \itemsep -1pt
			\item \url{http://www.hpl.hp.com/open_innovation/}
			\item HP Labs Innovation Research Program (IRP): \url{http://www.hpl.hp.com/open_innovation/irp/index.html}
			\end{enumerate}
		\end{enumerate}
	\end{enumerate}
\item Code for America (CfA): \vspace{-0.3cm}
	\begin{enumerate} \itemsep -2pt
	\item CfA Fellowship (develop web applications for local governments in the US): \url{http://codeforamerica.org/fellows/}
	\end{enumerate}
\item University of Minnesota, Twin Cities: \vspace{-0.3cm}
	\begin{enumerate} \itemsep -2pt
	\item College of Science and Engineering: \vspace{-0.2cm}
		\begin{enumerate} \itemsep -2pt
		\item Charles Babbage Institute: \vspace{-0.1cm}
			\begin{enumerate} \itemsep -1pt
			\item Adelle and Erwin Tomash Graduate Fellowship (for Ph.D. candidates doing research in the history of IT/computing - all but dissertation Ph.D. students only): \url{http://www.cbi.umn.edu/research/tfellowship.html}
			\item Arthur L. Norberg Travel Fund (short-term grants-in-aid to help scholars with travel expenses to use archival collections at the Charles Babbage Institute): \url{http://www.cbi.umn.edu/research/ntravelfund.html}
			\end{enumerate}
		\end{enumerate}
	\end{enumerate}
\item --- --- --- --- --- --- --- --- --- --- --- --- --- --- --- --- --- --- --- --- --- --- --- --- --- --- --- --- --- --- ---
\item \colorbox{blue}{\bf Scholarships and Fellowships in Biomedical Engineering}
% Scholarships and Fellowships in Biomedical Engineering
\item Whitaker International Fellows and Scholars Program: \vspace{-0.3cm}
	\begin{enumerate} \itemsep -2pt
	\item For graduate/Ph.D. students and postdocs in biomedical engineering
	\item \url{http://www.whitaker.org/home}
	\end{enumerate}
\item --- --- --- --- --- --- --- --- --- --- --- --- --- --- --- --- --- --- --- --- --- --- --- --- --- --- --- --- --- --- ---
\item \colorbox{blue}{\bf Scholarships and Fellowships in Optical Engineering}
% Scholarships and Fellowships in Optical Engineering
\item {\it SPIE} -- The International Society for Optical Engineering: \vspace{-0.3cm}
	\begin{enumerate} \itemsep -2pt
	\item ``SPIE Scholarship Program'' for undergraduates or graduate students studying optics, photonics, imaging, or optoelectronics program or related discipline (e.g., physics, electrical engineering): \url{http://spie.org//x1733.xml?WT.svl=mddm14}
	\item Other scholarships (including scholarships for students doing research in nanolithography techniques and lasers): \url{http://spie.org/x1736.xml}
	\end{enumerate}
\item {\it Kidger Optics Associates} Michael Kidger Memorial Scholarship (to a college freshman, or sophomore of optical design): \url{http://www.kidger.com/mkms_requirements.html}
\item --- --- --- --- --- --- --- --- --- --- --- --- --- --- --- --- --- --- --- --- --- --- --- --- --- --- --- --- --- --- ---
\item \colorbox{blue}{\bf Scholarships and Fellowships in Mechanical Engineering}
% Scholarships and Fellowships in Mechanical Engineering
\item American Society of Mechanical Engineers (ASME): \vspace{-0.3cm}
	\begin{enumerate} \itemsep -2pt
	\item Graduate Teaching Fellowships (for Ph.D. students in mechanical engineering): \url{http://www.asme.org/Education/College/FinancialAid/Graduate_Teaching_Fellowships.cfm}
	\item ASME Scholarships: \vspace{-0.2cm}
		\begin{enumerate} \itemsep -2pt
		\item \url{http://www.asme.org/Education/College/FinancialAid/Scholarships.cfm}
		\item US Undergraduates: \url{http://www.asme.org/Education/College/FinancialAid/US_Undergraduates.cfm}
		\item Graduate Students: \url{http://www.asme.org/Education/College/FinancialAid/Graduate_Students.cfm}
		\item International Students: \url{http://www.asme.org/Education/College/FinancialAid/International_Undergraduates.cfm}
		\end{enumerate}
	\item Auxiliary Scholarships: \vspace{-0.2cm}
		\begin{enumerate} \itemsep -2pt
		\item \url{http://www.asme.org/Education/College/FinancialAid/Auxiliary_Scholarships.cfm}
		\item Undergraduate Scholarships: \url{http://www.asme.org/Education/College/FinancialAid/Undergraduate_Scholarships.cfm}
		\item Graduate Scholarships: \url{http://www.asme.org/Education/College/FinancialAid/Graduate_Scholarships.cfm}
		\item Rice-Cullimore Scholarship (for international graduate students in the US): \url{http://www.asme.org/Education/College/FinancialAid/RiceCullimore_Scholarship.cfm}
		\end{enumerate}
	\item International Petroleum Institute�s College Scholarships (for undergraduates): \url{http://www.asme-ipti.org/public/pagscholarshipprograms.aspx}
	\item International Petroleum Institute�s Graduate Fellowship (for individuals entering a graduate program in mechanical engineering, and has an interest in the petroleum industry): \url{http://www.asme-ipti.org/public/pagscholarshipprograms.aspx} and \url{http://www.asme.org/Communities/Students/Grad/Fellowships.cfm}
	\end{enumerate}
\item --- --- --- --- --- --- --- --- --- --- --- --- --- --- --- --- --- --- --- --- --- --- --- --- --- --- --- --- --- --- ---
\item \colorbox{blue}{\bf Scholarships and Fellowships in Civil Engineering}
% Scholarships and Fellowships in Civil Engineering
\item American Society of Civil Engineers (ASCE): \vspace{-0.3cm}
	\begin{enumerate} \itemsep -2pt
	\item Jack E. Leisch Memorial National Graduate Fellowship (for graduate students in transportation/traffic engineering): \url{http://www.asce.org/Content.aspx?id=25021}
	\item Scholarships \& Fellowships (for undergraduates and graduate students): \url{http://www.asce.org/Content.aspx?id=18337}
	\end{enumerate}
\item American Concrete Institute (ACI): \vspace{-0.3cm}
	\begin{enumerate} \itemsep -2pt
	\item ACI Foundation Fellowships \& Scholarships: \url{http://www.concrete.org/STUDENTS/ST_SCHOLARSHIPS.HTM}
	\end{enumerate}
\item Institute of Transportation Engineers: \vspace{-0.3cm}
	\begin{enumerate} \itemsep -2pt
	\item Burton W. Marsh Fellowship for Graduate Study in Traffic and Transportation Engineering: \url{http://www.ite.org/education/Burton_W_MarshFellowship.asp}
	\end{enumerate}
\item --- --- --- --- --- --- --- --- --- --- --- --- --- --- --- --- --- --- --- --- --- --- --- --- --- --- --- --- --- --- ---
\item \colorbox{blue}{\bf Scholarships and Fellowships in Chemical Engineering}
% Scholarships and Fellowships in Chemical Engineering
\item American Institute of Chemical Engineers (AIChE) scholarships (includes scholarships for underrepresented minorities): \url{http://www.aiche.org/Students/Scholarships/index.aspx}
\item --- --- --- --- --- --- --- --- --- --- --- --- --- --- --- --- --- --- --- --- --- --- --- --- --- --- --- --- --- --- ---
\item \colorbox{blue}{\bf Scholarships and Fellowships in Aerospace Engineering}
% Scholarships and Fellowships in Aerospace Engineering
\item American Institute of Aeronautics and Astronautics (AIAA): \vspace{-0.3cm}
	\begin{enumerate} \itemsep -2pt
	\item AIAA Foundation Scholarships: \vspace{-0.2cm}
		\begin{enumerate} \itemsep -2pt
		\item \url{http://www.aiaa.org/content.cfm?pageid=211}
		\item For undergraduates and graduate students
		\item Named scholarships for undergraduates are: \vspace{-0.1cm}
			\begin{enumerate} \itemsep -1pt
			\item \url{http://www.aiaa.org/content.cfm?pageid=226}
			\item A. Thomas Young Scholarship
			\item L. S. ``Skip'' Fletcher Scholarship 
			\item Sam F. Iacobellis Scholarship
			\item Robert L. Crippen Scholarship
			\item E. C. ``Pete'' Aldridge Scholarship
			\item Liquid Propulsion Technical Committee Scholarship
			\item Space Transportation Technical Committee Scholarship
			\item Digital Avionics Technical Committee Scholarship (4)
			\item Next Century of Flight Scholarship (2)
			\item Leatrice Gregory Pendray Scholarship
			\end{enumerate}
		\item Awards for graduate students: \vspace{-0.1cm}
			\begin{enumerate} \itemsep -1pt
			\item \url{http://www.aiaa.org/content.cfm?pageid=227}
			\item Martin Summerfield Propellants and Combustion Graduate Award
			\item Guidance, Navigation, And Control Graduate Award
			\item Gordon C. Oates Air Breathing Propulsion Graduate Award
			\item William T. Piper, Sr. General Aviation Systems Graduate Award
			\item Orville and Wilbur Wright Graduate Award
			\item John Leland Atwood Graduate Award
			\item Open Topic Graduate Award
			\end{enumerate}
		\end{enumerate}
	\item Student Design Competition Award: \url{http://www.aiaa.org/content.cfm?pageid=401}
	\end{enumerate}
\item --- --- --- --- --- --- --- --- --- --- --- --- --- --- --- --- --- --- --- --- --- --- --- --- --- --- --- --- --- --- ---
\item \colorbox{blue}{\bf Scholarships and Fellowships in Mathematics}
% Scholarships and Fellowships in Mathematics
\item Association for Women in Mathematics (AWM): \vspace{-0.3cm}
	\begin{enumerate} \itemsep -2pt
	\item Travel grants: \url{http://sites.google.com/site/awmmath/programs/travel-grants}
	\item Alice T. Schafer Mathematics Prize for excellence in mathematics by an undergraduate woman: \url{http://sites.google.com/site/awmmath/programs/schafer-prize}
	\item The {\it Ruth I. Michler Memorial Prize} of the AWM is awarded annually to a woman recently promoted to Associate Professor or an equivalent position in the mathematical sciences: \url{http://sites.google.com/site/awmmath/programs/michler-prize}
	\end{enumerate}
\item Seth Bonder Scholarship for Applied Operations Research in Health Services: \url{http://www.informs.org/Recognize-Excellence/INFORMS-Community-Prizes-and-Awards/Seth-Bonder-Scholarship-for-Applied-Operations-Research-in-Health-Services}
\item Oberwolfach Foundation: \vspace{-0.3cm}
	\begin{enumerate} \itemsep -2pt
	\item Oberwolfach Prize (for young European mathematicians): \url{http://www.mfo.de/programme/prize/}
	\item John Todd Fellowship (or John Todd Award) [for young excellent mathematicians working in numerical analysis]: \url{http://www.mfo.de/programme/todd/}
	\end{enumerate}
\item Clay Mathematics Institute: Clay Research Award, \url{http://www.claymath.org/research_award/}
\item --- --- --- --- --- --- --- --- --- --- --- --- --- --- --- --- --- --- --- --- --- --- --- --- --- --- --- --- --- --- ---
\item \colorbox{blue}{\bf Scholarships and Fellowships in Science}
% Scholarships and Fellowships in Science
\item {\it Science.gov} (USA.gov for Science): \vspace{-0.3cm}
	\begin{enumerate} \itemsep -2pt
	\item Internship and Fellowship Opportunities in Science for Undergraduate Students: \url{http://www.science.gov/internships/undergrad.html}
	\item Graduate Students/Postdoctoral Fellowships: \url{http://www.science.gov/internships/graduate.html}
	\end{enumerate}
\item Heinz Family Philanthropies: \vspace{-0.3cm}
	\begin{enumerate} \itemsep -2pt
	\item Teresa Heinz Scholars for Environmental Research program (for Ph.D./MS students working on their thesis in environmental science/engineering) at selected universities: \url{http://www.heinzfamily.org/programs/environmentalscholars.html}
	\item \url{http://www.heinzfamily.org/}
	\end{enumerate}
\item Mayo Clinic: \vspace{-0.3cm}
	\begin{enumerate} \itemsep -2pt
	\item Postbaccalaureate Research Education Program (PREP): \url{http://www.mayo.edu/mgs/postbac-program.html}
	\end{enumerate}
\item {\it American Chemical Society (ACS)}: \vspace{-0.3cm}
	\begin{enumerate} \itemsep -2pt
	\item ACS-Hach Land Grant Undergraduate Scholarship (for chemistry undergraduates at a partner institution of ACS, and who plan to become chemistry teachers in US high schools): \url{http://portal.acs.org/portal/acs/corg/content?_nfpb=true&_pageLabel=PP_SUPERARTICLE&node_id=2243&use_sec=false&sec_url_var=region1&__uuid=eb054647-53e0-4594-81e8-8ef49159f3f4}
	\item ACS-Hach Second Career Teacher Scholarship (for graduates in chemistry or related areas who are entering an education masters program or teacher certification program): \url{http://portal.acs.org/portal/acs/corg/content?_nfpb=true&_pageLabel=PP_SUPERARTICLE&node_id=2244&use_sec=false&sec_url_var=region1&__uuid=4c27333f-4aad-481e-aaa4-f1db045d4eb4}
	\item ACS Scholars Program (for undergraduate underrepresented minorities majoring in chemistry, biochemistry, or chemical engineering): \url{http://portal.acs.org/portal/acs/corg/content?_nfpb=true&_pageLabel=PP_SUPERARTICLE&node_id=1650&use_sec=false&sec_url_var=region1&__uuid=b3b583cf-18ae-4fb0-9375-33f75a0ccf49}
	\item Scholarships: \url{http://portal.acs.org/portal/acs/corg/content?_nfpb=true&_pageLabel=PP_TRANSITIONMAIN&node_id=630&use_sec=false&sec_url_var=region1&__uuid=98e85c05-be75-4283-a97c-7a63ab4c3178}
	\end{enumerate}
\item European Molecular Biology Organization: \vspace{-0.3cm}
	\begin{enumerate} \itemsep -2pt
	\item EMBO Short-Term Fellowships (for junior researchers, including Ph.D. students): \url{http://www.embo.org/programmes/fellowships/short-term.html}
	\item EMBO Long-Term Fellowships (for junior researchers/postdocs): \url{http://www.embo.org/programmes/fellowships/long-term.html}
	\end{enumerate}
\item L'OR{\'{E}}AL: \vspace{-0.3cm}
	\begin{enumerate} \itemsep -2pt
	\item ``For Women in Science'' program: \url{http://www.lorealusa.com/forwomeninscience} or \url{http://www.lorealusa.com/_en/_us/index.aspx?direct1=00008&direct2=00008/00001}
	\item Alternatively, go to \url{http://www.lorealusa.com/_en/_us/} and select the ``For Women in Science'' tab.
	\item Check out the ``L'Or{\'{e}}al USA Fellowships for Women in Science'' (US postdocs), ``UNESCO-L'Or{\'{e}}al Fellowships for Women in Science'' (for female Ph.D. students and postdocs in the life sciences), and the ``L'Or{\'{e}}al-UNESCO Awardss for Women in Science'' (for distinguished female scientists)
	\end{enumerate}
\item American Institute of Physics (AIP): \vspace{-0.3cm}
	\begin{enumerate} \itemsep -2pt
	\item AIP and Member Society Government Science Fellowships: \vspace{-0.2cm}
		\begin{enumerate} \itemsep -2pt
		\item \url{http://www.aip.org/gov/fellowships.html}
		\item American Institute of Physics State Department Science Fellowship: \url{http://www.aip.org/gov/fellowships/sdf.html}
		\item American Institute of Physics Congressional Science Fellowship: \url{http://www.aip.org/gov/fellowships/cf.html}
		\item American Physical Society Congressional Science Fellowship: \url{http://www.aps.org/policy/fellowships/congressional.cfm}
		\item American Geophysical Union Congressional Science Fellowship: \url{http://www.agu.org/sci_pol/cong_fellowship/}
		\item Optical Society of America Congressional Science Fellowships: \url{http://www.osa.org/about_osa/public_policy/congressional_fellowships/default.aspx}
		\item For US citizens with good track records in research
		\end{enumerate}
	\item American Geophysical Union: \vspace{-0.2cm}
		\begin{enumerate} \itemsep -2pt
		\item Research Grants and Awards: \url{http://www.agu.org/about/honors/research_grants/}
		\item Student Travel Grants: \url{http://www.agu.org/education/grants/travel.shtml}
		\item Research Grants \& Awards: \url{http://www.agu.org/education/grants/research.shtml}
		\item Mass Media Fellowship: \url{http://www.agu.org/news/mass_media_fellowship/}
		\end{enumerate}
	\item Society of Physics Students (SPS): \vspace{-0.2cm}
		\begin{enumerate} \itemsep -2pt
		\item SPS Scholarships: \url{http://www.spsnational.org/programs/scholarships/}
		\item SPS Awards: \url{http://www.spsnational.org/programs/awards/}
		\end{enumerate}
	\end{enumerate}
\item Consortium for Ocean Leadership: \vspace{-0.3cm}
	\begin{enumerate} \itemsep -2pt
	\item Employment, Internships, and Opportunities [ includes funding opportunities for researchers (professors, postdocs, and grad students) ]: \url{http://www.oceanleadership.org/about-ocean-leadership/ocean-of-opportunities/}
	\item HBCU Fellowship: Ocean Leadership/IODP-USIO for Students of Historically Black Colleges and Universities, \url{http://www.oceanleadership.org/education/diversity/hbcu-fellowship/}
	\item HBCU Educator at Sea: \url{http://www.oceanleadership.org/education/diversity/hbcu-educator/}
	\item MS PHD's Professional Development Program: The Minorities Striving and Pursuing Higher Degrees of Success in the Earth System Sciences (MS PHD'S) Professional Development Program, \url{http://www.oceanleadership.org/education/diversity/ms-phds-professional-development-program/}
	\item Schlanger Ocean Drilling Fellowship Program (merit-based awards for outstanding graduate students to conduct research related to the Integrated Ocean Drilling Program): \url{http://www.oceanleadership.org/programs-and-partnerships/usssp/schlanger-fellowship/}
	\end{enumerate}
\item American Geological Institute Foundation: \vspace{-0.3cm}
	\begin{enumerate} \itemsep -2pt
	\item William L. Fisher Congressional Geoscience Fellowship (for young geoscientists to get engaged in {\bf public policy}): \url{http://www.agifoundation.org/govtaffairs.html} and \url{http://www.agifoundation.org/endowments.html}
	\item AGI Minority Participation Program: Minority Participation Program Geoscience Student Scholarships for ``underrepresented ethnic-minority (undergraduate or graduate) students in the geosciences'', \url{http://www.agiweb.org/mpp/index.html}
	\end{enumerate}
\item Lady Davis Institute/Jewish General Hospital: \vspace{-0.3cm}
	\begin{enumerate} \itemsep -2pt
	\item Awards for ``graduate students (in biomedical science) and post-doctoral fellows/clinical fellows'': \url{http://www.ladydavis.ca/en/awards}
	\end{enumerate}
\item Adolph C. and Mary Sprague Miller Institute for Basic Research in Science: \vspace{-0.3cm}
	\begin{enumerate} \itemsep -2pt
	\item Miller Fellowships (for outstanding recent Ph.D.s / postdoctoral fellowship): \url{http://millerinstitute.berkeley.edu/topage.php?nav=11&to=1} or \url{http://millerinstitute.berkeley.edu/page.php?nav=11}
	\item Visiting Miller Research Professorships (for professors and research scientists): \url{http://millerinstitute.berkeley.edu/topage.php?nav=24&to=1} or \url{http://millerinstitute.berkeley.edu/page.php?nav=24}
	\item Miller Research Professorships (for professors in the UC system): \url{http://millerinstitute.berkeley.edu/topage.php?nav=15&to=1} or \url{http://millerinstitute.berkeley.edu/page.php?nav=15}
	\item Miller Senior Fellowships (Nominations are solicited by invitation only; Senior Fellow appointments are made to tenured UC Berkeley faculty for five years, possibly renewable for a subsequent five years, but no longer.): \url{http://millerinstitute.berkeley.edu/topage.php?nav=126&to=1}
	\end{enumerate}
\item Funda{\c{c}}{\~{a}}o para a Ci{\^{e}}ncia e a Tecnologia (FCT); Minist{\'{e}}rio da Ci{\^{e}}ncia, Technologia e Ensino Superior (MCTES): International Prize Fernando Gil in Philosophy of Science, \url{http://alfa.fct.mctes.pt/apoios/premios/fernando_gil/index.phtml.pt}
\item Wellcome Trust: \vspace{-0.3cm}
	\begin{enumerate} \itemsep -2pt
	\item Wellcome Trust Sanger Institute: \vspace{-0.2cm}
		\begin{enumerate} \itemsep -2pt
		\item \url{http://www.sanger.ac.uk/workstudy/}
		\item Postdoctoral fellows (for research in genomics): \url{http://www.sanger.ac.uk/workstudy/career/postdocs/}
		\item Graduate program (for research in genomics): \url{http://www.sanger.ac.uk/workstudy/phd/}
		\item Student placements and work experience (for research in genomics): \url{http://www.sanger.ac.uk/workstudy/placements/}
		\end{enumerate}
	\end{enumerate}
\item Paul B. Beeson Career Development Awards in Aging Research Program (formerly the Beeson Physician Faculty Scholars Program): \vspace{-0.3cm}
	\begin{enumerate} \itemsep -2pt
	\item \url{http://www.beeson.org/}
	\item ``Today, the Beeson program continues to foster the independent research careers of clinically trained investigators -- a growing cadre of talented physician-scientists -- whose research and leadership are enhancing the health and quality of life of Americans, particularly older people.''
	\item About the Program: \url{http://www.beeson.org/program_hx.cfm}
	\end{enumerate}
\item American Mathematical Society: \vspace{-0.3cm}
	\begin{enumerate} \itemsep -2pt
	\item AMS Fellowships and Scholarships: \vspace{-0.2cm}
		\begin{enumerate} \itemsep -2pt
		\item \url{http://e-math.ams.org/programs/ams-fellowships/ams-fellowships}
		\item AMS Centennial Research Fellowship Program: \url{http://e-math.ams.org/programs/ams-fellowships/centennial-fellow/emp-centflyer}
		\item Waldemar J. Trijitzinsky Memorial Awards: \url{http://e-math.ams.org/programs/ams-fellowships/trjitzinsky/trjitzinsky-award}
		\item Other Sources of Funding: \url{http://e-math.ams.org/programs/funding/funding}
		\end{enumerate}
	\end{enumerate}
\item --- --- --- --- --- --- --- --- --- --- --- --- --- --- --- --- --- --- --- --- --- --- --- --- --- --- --- --- --- --- ---
\item \colorbox{blue}{\bf Scholarships and Fellowships in Medicine}
% Scholarships and Fellowships in Medicine
\item Sarnoff Medical Student Research Fellowship Program (for US medical students interested in cardiovascular research): \url{http://www.sarnoffendowment.org/}
\item Mayo Clinic: \vspace{-0.3cm}
	\begin{enumerate} \itemsep -2pt
	\item Postbaccalaureate Research Education Program (PREP): \url{http://www.mayo.edu/mgs/postbac-program.html}
	\end{enumerate}
\item Paul B. Beeson Career Development Awards in Aging Research Program (formerly the Beeson Physician Faculty Scholars Program): \vspace{-0.3cm}
	\begin{enumerate} \itemsep -2pt
	\item \url{http://www.beeson.org/}
	\item ``Today, the Beeson program continues to foster the independent research careers of clinically trained investigators -- a growing cadre of talented physician-scientists -- whose research and leadership are enhancing the health and quality of life of Americans, particularly older people.''
	\item About the Program: \url{http://www.beeson.org/program_hx.cfm}
	\end{enumerate}
\item --- --- --- --- --- --- --- --- --- --- --- --- --- --- --- --- --- --- --- --- --- --- --- --- --- --- --- --- --- --- ---
\item \colorbox{blue}{\bf Scholarships and Fellowships in Science and Engineering}
% Scholarships and Fellowships in Science and Engineering
\item National Academies: \vspace{-0.3cm}
	\begin{enumerate} \itemsep -2pt
	\item Research Associateship Programs (graduate, postdoctoral, and senior level research opportunities): \url{http://sites.nationalacademies.org/pga/rap/}
	\item Ford Foundation Fellowship Programs (predoctoral, dissertation or postdoctoral fellowships for individuals seeking academic careers in science and engineering): \url{http://sites.nationalacademies.org/PGA/FordFellowships/index.htm}
	\item \url{http://nationalacademies.org/grantprograms.html}
	\item \url{http://sites.nationalacademies.org/pga/fellowships/}
	\item List of Fellowship, Scholarship, and Grant Databases: \url{http://sites.nationalacademies.org/PGA/Fellowships/PGA_046300}
	\item List of Outside Fellowships, Scholarships, and Grants Websites: \url{http://sites.nationalacademies.org/PGA/Fellowships/PGA_046301}
	\item Awards for junior and mid-career researchers: \url{http://www.nasonline.org/site/PageServer?pagename=AWARDS_main}
	\item National Academy of Engineering, NAE: \vspace{-0.2cm}
		\begin{enumerate} \itemsep -2pt
		\item NAE Grand Challenges Scholars Program: \url{http://www.grandchallengescholars.org/}
		\end{enumerate}
	\item National Science Foundation: \vspace{-0.2cm}
		\begin{enumerate} \itemsep -2pt
		\item International Research Fellowship Program (IRFP) for junior scientists and engineers: \url{http://www.nsf.gov/funding/pgm_summ.jsp?pims_id=5179}
		\item Integrative Graduate Education and Research Traineeship Program (IGERT) for undergraduates and graduate students in STEM: \url{http://www.nsf.gov/funding/pgm_summ.jsp?pims_id=12759}
		\item National Science Foundation's Graduate Research Fellowship Program (GRFP) for students seeking research degrees in STEM: \url{http://www.nsfgrfp.org/}
		\item NSF Alliances for Graduate Education and the Professoriate (AGEP) program (to help underrepresented minorities obtain graduate degrees in STEM and prepare them for faculty positions in academia): \url{http://www.nsfagep.org/}
		\item National Science Foundation's (NSF) East Asia and Pacific Summer Institutes (EAPSI) program: \vspace{-0.1cm}
			\begin{enumerate} \itemsep -1pt
			\item \url{http://www.nsf.gov/funding/pgm_summ.jsp?pims_id=5284}
			\item The East Asia and Pacific Summer Institutes (EAPSI) provide U.S. graduate students in science and engineering: \vspace{-0.1cm}
				\begin{itemize} \itemsep -1pt
				\item first-hand research experiences in Australia, China, Japan, Korea, New Zealand, Singapore or Taiwan
				\item an introduction to the science, science policy, and scientific infrastructure of the respective location
				\item an orientation to the society, culture and language.
				\end{itemize}
			\item ``The primary goals of EAPSI are to introduce students to East Asia and Pacific science and engineering in the context of a research setting, and to help students initiate scientific relationships that will better enable future collaboration with foreign counterparts.''
			\item ``All institutes, except Japan, last approximately eight weeks from June to August. Japan lasts approximately ten weeks from June to August (specific dates are available and updated at \url{http://www.nsfsi.org/}).''
			\item Example of Ph.D. student, Jakub Szefer, from Prof. Ruby Lee's lab at Princeton University, who interned with Prof. Cheng Chen-Mou from National Taiwan University: \url{http://www.nsf.gov/discoveries/disc_summ.jsp?cntn_id=118116&org=NSF}
			\end{enumerate}
		\end{enumerate}
	\end{enumerate}
\item United States Department of Defense (DoD): \vspace{-0.3cm}
	\begin{enumerate} \itemsep -2pt
	\item National Defense Education Program; Defense Advanced Research Projects Agency (DARPA): \vspace{-0.2cm}
		\begin{enumerate} \itemsep -2pt
		\item Science, Mathematics, and Research for Transformation (SMART) scholarship program: \vspace{-0.1cm}
			\begin{itemize} \itemsep -1pt
			\item \url{http://smart.asee.org/}
			\item Co-organized by the American Society for Engineering Education
			\end{itemize}
		\item National Security Science and Engineering Faculty Fellowships (NSSEFF): \url{http://www.ndep.us/ProgNSSEFF.aspx}
		\end{enumerate}
	\end{enumerate}
\item National Society of Professional Engineers: \vspace{-0.3cm}
	\begin{enumerate} \itemsep -2pt
	\item Scholarships for undergraduates and graduate students: \url{http://www.nspe.org/Students/Scholarships/index.html}
	\item NSPE-PEC George B. Hightower, P.E. Fellowship (for an outstanding engineering graduate student): \url{http://www.nspe.org/InterestGroups/PEC/Resources/Awards/hightower_fellowship.html}
	\item PEG Management Fellowship: \vspace{-0.2cm}
		\begin{enumerate} \itemsep -2pt
		\item \url{http://www.nspe.org/InterestGroups/PEG/Resources/AwardsAndScholarships/peg_fellowship.html}
		\item ``This scholarship is designed for graduate students who are pursuing an MBA, a master's degree in engineering management, or a master's degree in public administration.''
		\end{enumerate}
	\end{enumerate}
\item Technion -- Israel Institute of Technology: \vspace{-0.3cm}
	\begin{enumerate} \itemsep -2pt
	\item Department of Mathematics: Anna and Paul Erdos postdoctoral Fellowship, \url{http://www.math.technion.ac.il/Site/people/positions.html}
	\item Lady Davis Postdoctoral Fellowship
	\item Department of Electrical Engineering: \vspace{-0.2cm}
		\begin{enumerate} \itemsep -2pt
		\item The Andrew and Erna Finci Viterbi Fellowship Program (for graduate and post-doctoral fellows), \url{http://webee.technion.ac.il/Research/Fellowship-Programs}
		\item Lady Davis Fellowship Trust: Technion Fellowships (for visiting professors, post-doctoral researchers, as well as Masters and Ph.D. students), \url{http://ldft.huji.ac.il/upload/info/}
		\item \url{http://webee.technion.ac.il/Research/Fellowship-Programs}
		\end{enumerate}
	\end{enumerate}
\item Hebrew University: \vspace{-0.3cm}
	\begin{enumerate} \itemsep -2pt
	\item Lady Davis Fellowship Trust: Technion Fellowships (for visiting professors, post-doctoral researchers, as well as Masters and Ph.D. students), \url{http://ldft.huji.ac.il/upload/info/infoHUa.html}
	\end{enumerate}
\item Hertz Foundation: \vspace{-0.3cm}
	\begin{enumerate} \itemsep -2pt
	\item The Graduate Fellowship Award: \url{http://www.hertzfoundation.org/dx/Fellowships/award.aspx}
	\item Thesis Prize: \url{http://www.hertzfoundation.org/dx/Fellowships/thesis_winners.aspx}
	\end{enumerate}
\item Krell Institute, Inc.: \vspace{-0.3cm}
	\begin{enumerate} \itemsep -2pt
	\item DOE Computational Science Graduate Fellowship: \url{http://www.krellinst.org/csgf/index.shtml}
	\end{enumerate}
\item The Winston Churchill Foundation of the United States: \vspace{-0.3cm}
	\begin{enumerate} \itemsep -2pt
	\item The Churchill Scholarship: \url{http://winstonchurchillfoundation.org/index.php?hide=1&section=eligibility}
	\end{enumerate}
\item American Society for Engineering Education: \vspace{-0.3cm}
	\begin{enumerate} \itemsep -2pt
	\item \url{http://blogs.asee.org/fellowships/}
	\item Fellowship programs: \url{http://www.asee.org/fellowship-programs}
	\item Awards: \url{http://www.asee.org/member-resources/awards/full-list-of-awards}
	\item DuPont Minorities in Engineering Award: \vspace{-0.2cm}
		\begin{enumerate} \itemsep -2pt
		\item \url{http://www.asee.org/member-resources/awards/full-list-of-awards/national-awards/special#DuPont_Minorities_in_Engineering_Award}
		\item {\bf \color{blue} ``The DuPont Minorities in Engineering Award is conferred for outstanding achievements by an engineering or engineering technology educator in increasing student diversity within engineering and engineering technology programs.''}
		\end{enumerate}
	\end{enumerate}
\item Alexander von Humboldt-Stiftung/Foundation: \vspace{-0.3cm}
	\begin{enumerate} \itemsep -2pt
	\item Feodor Lynen Research Fellowship for Postdoctoral Researchers (junior postdocs): \url{http://www.humboldt-foundation.de/web/feodor-lynen-fellowship-postdoc.html}
	\item Friedrich Wilhelm Bessel Research Award (mid-career researchers): \url{http://www.humboldt-foundation.de/web/bessel-award.html}
	\item Georg Forster Research Fellowship for Postdoctoral Researchers (for non-German junior postdocs ``with above average qualifications''): \url{http://www.humboldt-foundation.de/web/georg-forster-fellowship-postdoc.html}
	\item Humboldt Research Fellowship for Postdoctoral Researchers (junior postdocs): \url{http://www.humboldt-foundation.de/web/771.html}
	\item Sofja Kovalevskaja Award (junior postdocs): \url{http://www.humboldt-foundation.de/web/kovalevskaja-award.html}
	\item Fraunhofer-Bessel Research Award: \url{http://www.humboldt-foundation.de/web/fraunhofer-bessel-award.html}
	\item \url{http://www.humboldt-foundation.de/web/home.html}
	\end{enumerate}
\item Santa Fe Institute: Omidyar Postdoctoral Fellowship; see \url{http://www.santafe.edu/education/fellowships/omidyar-postdoctoral/}
\item Applied Materials: Applied Materials Graduate Fellowship
\item American Society of Naval Engineers (ASNE): \vspace{-0.3cm}
	\begin{enumerate} \itemsep -2pt
	\item (Undergraduate and Graduate) Scholarships: \url{http://www.navalengineers.org/awards/scholarships/Pages/ASNELandingPage.aspx}
	\end{enumerate}
\item Lindau Meeting of Nobel Laureates and Students in Lindau (Oak Ridge Associated Universities, ORAU): \vspace{-0.3cm}
	\begin{enumerate} \itemsep -2pt
	\item Graduate Student Award program: \vspace{-0.2cm}
		\begin{enumerate} \itemsep -2pt
		\item \url{http://www.orau.org/lindau/}
		\item A student nominated to participate in this program must: \vspace{-0.1cm}
			\begin{enumerate} \itemsep -1pt
			\item Be a U.S. citizen
			\item Be currently enrolled as a full-time graduate student
			\item Be currently sponsored by, or working on, and supported by projects sponsored by, the agency to which the nomination is made, such as the U.S. Department of Energy Office of Science, the National Institutes of Health or other federal agency
			\item Have completed by June 2011 two years (but not more than four years) of study toward a doctoral degree in medicine or physiology, or in a related discipline, including the basic biomedical (or life) sciences
			\end{enumerate}
		\end{enumerate}
	\end{enumerate}
\item Research Councils UK (RCUK): \vspace{-0.3cm}
	\begin{enumerate} \itemsep -2pt
	\item RCUK Academic Fellowships: \vspace{-0.2cm}
		\begin{enumerate} \itemsep -2pt
		\item \url{http://www.rcuk.ac.uk/ResearchCareers/fellowships/Pages/home.aspx}
		\item \url{http://www.rcuk.ac.uk/ResearchCareers/fellowships/Pages/about.aspx}
		\item Dorothy Hodgkin Postgraduate Awards: \vspace{-0.1cm}
			\begin{enumerate} \itemsep -1pt
			\item \url{http://www.rcuk.ac.uk/ResearchCareers/dhpa/Pages/home.aspx}
			\item ``Dorothy Hodgkin Postgraduate Awards (DHPA) is a UK scheme to bring outstanding students from India, China, Hong Kong, South Africa, Brazil, Russia and the developing world to come and study for PhDs in top rated UK research facilities.''
			\end{enumerate}
		\end{enumerate}
	\item International Funding Opportunities: \vspace{-0.2cm}
		\begin{enumerate} \itemsep -2pt
		\item \url{http://www.rcuk.ac.uk/international/funding/FundingOpps/Pages/home.aspx}
		\item Early Career Researchers: \url{http://www.rcuk.ac.uk/international/funding/FundingOpps/Pages/EarlyCareer.aspx}
		\end{enumerate}
	\item Engineering and Physical Sciences Research Council: \vspace{-0.2cm}
		\begin{enumerate} \itemsep -2pt
		\item Programs: \vspace{-0.1cm}
			\begin{enumerate} \itemsep -1pt
			\item Physical sciences: \vspace{-0.1cm}
				\begin{itemize} \itemsep -1pt
				\item Organic synthetic chemistry studentships: \url{http://www.epsrc.ac.uk/about/progs/physsci/Pages/organicstudentships.aspx}
				\item Analytical science studentships: \url{http://www.epsrc.ac.uk/about/progs/physsci/Pages/analyticalstudentships.aspx}
				\end{itemize}
			\item Mathematical sciences: \vspace{-0.1cm}
				\begin{itemize} \itemsep -1pt
				\item Fellowships (for postdoctoral research): \url{http://www.epsrc.ac.uk/about/progs/maths/Pages/fellowships.aspx}
				\end{itemize}
			\end{enumerate}
		\item Funding: \vspace{-0.1cm}
			\begin{enumerate} \itemsep -1pt
			\item \url{http://www.epsrc.ac.uk/funding/Pages/default.aspx}
			\item Grants available [has funds for (new/junior) professors and to support international collaboration]: \url{http://www.epsrc.ac.uk/funding/grants/Pages/default.aspx}
			\item Calls for proposals (open/current funding calls for applications and future/proposed calls): \url{http://www.epsrc.ac.uk/funding/calls/Pages/default.aspx}
			\item Studentships (training grants for Ph.D. and Masters students, including international students): \url{http://www.epsrc.ac.uk/funding/students/Pages/default.aspx}
			\item Fellowships (from junior scientists and engineers engaged in postdoctoral research to senior researchers): \url{http://www.epsrc.ac.uk/funding/fellows/Pages/default.aspx}
			\end{enumerate}
		\end{enumerate}
	\item Biotechnology and Biological Sciences Research Council (BBSRC): \vspace{-0.2cm}
		\begin{enumerate} \itemsep -2pt
		\item ``The UK's leading funding agency for academic research and training in the non-clinical life sciences''
		\item Funding research: \vspace{-0.1cm}
			\begin{enumerate} \itemsep -1pt
			\item \url{http://www.bbsrc.ac.uk/funding/funding-index.aspx}
			\item Fellowships (for early career scientists, for supporting individuals seeking a change in research directions or scientists who are returning to research, and senior researchers): \url{http://www.bbsrc.ac.uk/funding/fellowships/fellowships-index.aspx}
			\item Studentships (Doctoral training grants, Masters training grants, postgraduate awards, and undergraduate research grants): \url{http://www.bbsrc.ac.uk/funding/studentships/studentships-index.aspx}
			\item Special opportunities (current calls for funding): \url{http://www.bbsrc.ac.uk/funding/opportunities/opportunities-index.aspx}
			\item Apply for funding (information about the process of applying for research funds): \url{http://www.bbsrc.ac.uk/funding/apply/apply-index.aspx}
			\end{enumerate}
		\end{enumerate}
	\item Science and Technology Facilities Council: \vspace{-0.2cm}
		\begin{enumerate} \itemsep -2pt
		\item STFC Grants and Awards: \vspace{-0.1cm}
			\begin{enumerate} \itemsep -1pt
			\item \url{http://www.stfc.ac.uk/Funding+and+Grants/501.aspx}
			\item ``The Science and Technology Facilities Council offers grants and support in Particle Physics, Astronomy, Nuclear Physics and Facility Development. It also provides support for research infrastructure, training, knowledge exchange and public engagement activities through a variety of funding schemes and activities.''
			\item STFC Funding Opportunities: \url{http://www.stfc.ac.uk/Funding%20and%20Grants/598.aspx}
			\item Postgraduate Studentships: \url{http://www.stfc.ac.uk/Funding+and+Grants/637.aspx} or \url{http://www.stfc.ac.uk/Funding%20and%20Grants/636.aspx}
			\end{enumerate}
		\item Fellowship opportunities: \vspace{-0.1cm}
			\begin{enumerate} \itemsep -1pt
			\item \url{http://www.stfc.ac.uk/Funding%20and%20Grants/508.aspx}
			\item ``Fellowship opportunities in Astronomy, Solar and Planetary Science, Particle Physics, Particle Astrophysics, Nuclear Physics, Development of STFC Neutron, Laser and Synchrotron Facilities within the UK.''
			\item There are postdoctoral and advanced research fellowships.
			\end{enumerate}
		\item Innovations Partnership Schemes (IPS and mini-IPS): \url{http://www.stfc.ac.uk/19213.aspx}
		\item IPS Fellowships: \vspace{-0.1cm}
			\begin{enumerate} \itemsep -1pt
			\item \url{http://www.stfc.ac.uk/19226.aspx}
			\item The IPS fellowship is a scheme designed to support a role to develop the commercial exploitation of technologies. This is not a research orientated fellowship.
			\end{enumerate}
		\item Follow-on-Funding: \vspace{-0.1cm}
			\begin{enumerate} \itemsep -1pt
			\item \url{http://www.stfc.ac.uk/19207.aspx}
			\item ``Follow on Funding is intended to provide financial support at the very early or pre-seed stage of turning research outputs into a commercial proposition. Unlike the other research councils, in STFC, industry partners are not allowed. If you have an industry partner, please use the mini-IPS or IPS scheme.''
			\item ``STFC staff, grant funded academics and researchers at CERN and ESO are eligible to apply for follow-on-funds (see the research grants handbook for CERN and ESO eligibility). STFC staff should first investigate whether they can be funded through proof of concept funding.''
			\end{enumerate}
		\end{enumerate}
	\item Natural Environment Research Council: \vspace{-0.2cm}
		\begin{enumerate} \itemsep -2pt
		\item Grants and studentships on the web: \vspace{-0.1cm}
			\begin{enumerate} \itemsep -1pt
			\item \url{http://www.nerc.ac.uk/research/gotw.asp}
			\item Grants on the web: \url{http://gotw.nerc.ac.uk/goti.asp?c=1}
			\end{enumerate}
		\item Funding: \vspace{-0.1cm}
			\begin{enumerate} \itemsep -1pt
			\item \url{http://www.nerc.ac.uk/funding/}
			\item Postgraduate training: \vspace{-0.1cm}
				\begin{itemize} \itemsep -1pt
				\item Postgraduate eligibility (requires UK/EU citizenship): \url{http://www.nerc.ac.uk/funding/available/postgrad/eligibility.asp}
				\end{itemize}
			\item Research Fellowship Scheme [for all nationalities]: \url{http://www.nerc.ac.uk/funding/available/fellowships/}
			\item Research Experience Placements (REP) scheme [for undergraduates]: \url{http://www.nerc.ac.uk/funding/available/rep.asp}
			\item Research Grants: \vspace{-0.1cm}
				\begin{itemize} \itemsep -1pt
				\item Eligibility: \url{http://www.nerc.ac.uk/funding/available/researchgrants/eligibility.asp}
				\end{itemize}
			\end{enumerate}
		\item {\bf Other potential sources of funding}: \vspace{-0.1cm}
			\begin{enumerate} \itemsep -1pt
			\item \url{http://www.nerc.ac.uk/funding/otherfunding.asp}
			\item Look at the ``Higher Education Funding Councils'' for each country (England, Wales, Northern Ireland, and Scotland)
			\end{enumerate}
		\end{enumerate}
	\end{enumerate}
\item Nuffield Foundation: \vspace{-0.3cm}
	\begin{enumerate} \itemsep -2pt
	\item Undergraduate research bursaries in science: \url{http://www.nuffieldfoundation.org/undergraduate-research-bursaries-0}
	\item Funding for social policy projects in the UK: \vspace{-0.2cm}
		\begin{enumerate} \itemsep -2pt
		\item \url{http://www.nuffieldfoundation.org/social-policy}
		\item \url{http://www.nuffieldfoundation.org/children-and-families-law-society-education-and-open-door}
		\end{enumerate}
	\item Apply for funding: \url{http://www.nuffieldfoundation.org/apply-for-funding}
	\item Africa program: \url{http://www.nuffieldfoundation.org/africa-programme-0}
	\item Nuffield Farming Scholarships Trust: \vspace{-0.2cm}
		\begin{enumerate} \itemsep -2pt
		\item Nuffield Farming Scholarships: \url{http://www.nuffieldscholar.org/}
		\end{enumerate}
	\item The Nuffield Trust (or, The Nuffield Trust for Research and Policy Studies in Health Services): \vspace{-0.2cm}
		\begin{enumerate} \itemsep -2pt
		\item Fellowships: \vspace{-0.1cm}
			\begin{enumerate} \itemsep -1pt
			\item \url{http://www.nuffieldtrust.org.uk/fellowships/index.aspx?id=43}
			\item Rock Carling fellowship (for senior researchers in public health): \url{http://www.nuffieldtrust.org.uk/fellowships/index.aspx?id=112}
			\item John Fry Fellowship (for senior researchers in public health): \url{http://www.nuffieldtrust.org.uk/fellowships/index.aspx?id=109}
			\item Harkness Fellowships in Health Care Policy: \vspace{-0.1cm}
				\begin{itemize} \itemsep -1pt
				\item ``Since September 2009 The Nuffield Trust have been the proud co-sponsors of the prestigious Harkness Fellowships programme with The Commonwealth Fund.''
				\item ``These offer an unparalleled opportunity for the health policy analysts of the future to conduct original research and learn about healthcare in North America.''
				\item ``Mid-career health policy researchers and practitioners (including doctors, health services managers, journalists and government officials) are supported to spend 9 to 12 months in the United States conducting a policy-oriented research project and working with leading U.S. health policy experts.''
				\end{itemize}
			\end{enumerate}
		\end{enumerate}
	\end{enumerate}
\item U.S. Department of Homeland Security (DHS): \vspace{-0.3cm}
	\begin{enumerate} \itemsep -2pt
	\item DHS Scholarship and Fellowship Program: \url{http://www.orau.gov/dhsed/}
	\end{enumerate}
\item ACT, Inc.: \vspace{-0.3cm}
	\begin{enumerate} \itemsep -2pt
	\item Barry M. Goldwater Scholarship and Excellence in Education Program (for US residents who will be college upperclassmen in STEM fields in the following academic year): \url{http://www.act.org/goldwater/}
	\end{enumerate}
\item Massachusetts Institute of Technology: \vspace{-0.3cm}
	\begin{enumerate} \itemsep -2pt
	\item MIT School of Engineering: \vspace{-0.2cm}
		\begin{enumerate} \itemsep -2pt
		\item Lemelson-MIT Program: \vspace{-0.1cm}
			\begin{enumerate} \itemsep -1pt
			\item \url{http://web.mit.edu/invent/}
			\item Lemelson-MIT Awards for Invention and Innovation: \url{http://web.mit.edu/invent/a-main.html}
			\end{enumerate}
		\end{enumerate}
	\end{enumerate}
\item --- --- --- --- --- --- --- --- --- --- --- --- --- --- --- --- --- --- --- --- --- --- --- --- --- --- --- --- --- --- ---
\item \colorbox{blue}{\bf Scholarships and Fellowships in Various Fields (Including Creative Arts, Teaching, and Sports)}
% Scholarships and Fellowships in Various Fields (Including Creative Arts, Teaching, and Sports)
\item U.S. Department of Education: \vspace{-0.3cm}
	\begin{enumerate} \itemsep -2pt
	\item Robert C. Byrd Honors Scholarship Program: \vspace{-0.2cm}
		\begin{enumerate} \itemsep -2pt
		\item High school graduates who have been accepted for enrollment at institutions of higher education (IHEs), have demonstrated outstanding academic achievement, and show promise of continued academic excellence may apply to states in which they are residents.
		\item \url{http://www2.ed.gov/programs/iduesbyrd/index.html}
		\end{enumerate}
	\item \colorbox{yellow}{\bf Jacob K. Javits Fellowships Program}: \vspace{-0.1cm}
		\begin{enumerate} \itemsep -1pt
		\item This program provides fellowships to students of superior academic ability -- selected on the basis of demonstrated achievement, financial need, and exceptional promise -- to undertake study at the doctoral and Master of Fine Arts level in selected fields of arts, humanities, and social sciences.
		\item \url{http://www2.ed.gov/programs/jacobjavits/index.html}
		\end{enumerate}
	\item Close Up Fellowship Program: \vspace{-0.2cm}
		\begin{enumerate} \itemsep -2pt
		\item This program provides financial aid to enable low-income students, their teachers, and recent immigrants to come to Washington, D.C., to study the operations of the three branches of the federal government.
		\item \url{http://www2.ed.gov/programs/closeup/index.html}
		\end{enumerate}
	\item {\bf \color{blue} B.J. Stupak Olympic Scholarships}: \vspace{-0.2cm}
		\begin{enumerate} \itemsep -2pt
		\item This program provides financial assistance to athletes who are training at the U.S. Olympic Education Center or one of the U.S. Olympic training centers and who are pursuing a postsecondary education at institutions of higher education (IHEs).
		\item \url{http://www2.ed.gov/programs/olympic/index.html}
		\end{enumerate}
	\item {\bf \color{blue} Teacher Education Assistance for College and Higher Education (TEACH) Grant Program}: \vspace{-0.2cm}
		\begin{enumerate} \itemsep -2pt
		\item Through the College Cost Reduction and Access Act of 2007, Congress created the Teacher Education Assistance for College and Higher Education (TEACH) Grant Program that provides grants of up to \$4,000 per year to students who intend to teach in a public or private elementary or secondary school that serves students from low-income families.
		\item \url{http://studentaid.ed.gov/PORTALSWebApp/students/english/TEACH.jsp}
		\end{enumerate}
	\item Scholarship search engine: \url{https://studentaid2.ed.gov/getmoney/scholarship/}
	\item Financial Aid: \vspace{-0.2cm}
		\begin{enumerate} \itemsep -2pt
		\item \url{http://www2.ed.gov/finaid/landing.jhtml?src=rt}
		\item \url{http://studentaid.ed.gov/PORTALSWebApp/students/english/funding.jsp}
		\item Paying for college: \url{http://www.college.gov}
		\item Student Aid (has information for students at all levels and parents): \url{http://studentaid.ed.gov/}
		\item Student Aid Eligibility: \url{http://studentaid.ed.gov/PORTALSWebApp/students/english/aideligibility.jsp?tab=funding}
		\item Federal Student Aid: \url{http://federalstudentaid.ed.gov/}
		\item Academic Competitiveness Grant: The Academic Competitiveness Grant provides up to \$750 for the first year of undergraduate study and up to \$1,300 for the second year of undergraduate study. See \url{http://studentaid.ed.gov/PORTALSWebApp/students/english/NewPrograms.jsp}.
		\end{enumerate}
	\item Free Application for Federal Student Aid (FAFSA): \vspace{-0.2cm}
		\begin{enumerate} \itemsep -2pt
		\item Financial Aid Estimator Tool (FAFSA4caster): \url{http://www.fafsa4caster.ed.gov/F4CApp/index/index.jsf}
		\item \url{http://www.fafsa.ed.gov/}
		\end{enumerate}
	\item Federal Pell Grant Program: \url{http://www2.ed.gov/programs/fpg/index.html}
	\end{enumerate}
\item European Commission: \vspace{-0.3cm}
	\begin{enumerate} \itemsep -2pt
	\item Erasmus Programme (for Europeans): \url{http://ec.europa.eu/education/lifelong-learning-programme/doc80_en.htm}
	\item Erasmus Mundus (for non-Europeans): \url{http://ec.europa.eu/education/external-relation-programmes/doc72_en.htm}
	\end{enumerate}
\item Woodrow Wilson Foundation: \vspace{-0.3cm}
	\begin{enumerate} \itemsep -2pt
	\item {\bf \color{blue} The Woodrow Wilson-Rockefeller Brothers Fund Fellowships for Aspiring Teachers of Color (for underrepresented minorities seeking a career as a K-12 public school teacher in the US)}: \url{http://www.woodrow.org/teaching-fellowships/wwrbf/index.php}
	\item {\bf \color{blue} Woodrow Wilson Teaching Fellowship (for a MS program in teacher education, who would teach at high-need urban and rural schools or $\ge$ 3 years)}: \url{http://www.wwteachingfellowship.org/}
	\item {\bf \color{blue} Leonore Annenberg Teaching Fellowship (for a MS program in teacher education, who would teach at high-need urban and rural schools or $\ge$ 3 years)}: \url{http://www.woodrow.org/teaching-fellowships/annenberg/index.php}
	\item MMUF Travel \& Research Grants (for graduate students who participated in the Mellon Mays Undergraduate Fellowship Program): \url{http://www.woodrow.org/higher-education-fellowships/opportunity/research/index.php}
	\item MMUF Dissertation Grants (for graduate students who participated in the Mellon Mays Undergraduate Fellowship Program): \url{http://www.woodrow.org/higher-education-fellowships/opportunity/dissertation/index.php}
	\item Charlotte W. Newcombe Doctoral Dissertation Fellowship (for Ph.D. students writing their theses on ethical or religious values in all fields of the humanities and social sciences): \url{http://www.woodrow.org/higher-education-fellowships/religion_ethics/index.php}
	\item {\bf \color{blue} Woodrow Wilson Dissertation Fellowship in Women�s Studies}: \url{http://www.woodrow.org/higher-education-fellowships/women_gender/index.php}
	\item Doris Duke Conservation Fellowship program (Masters students seeking careers as practicing conservationists): \url{http://www.woodrow.org/higher-education-fellowships/conservation/index.php}
	\item Thomas R. Pickering Graduate Foreign Affairs Fellowship: \vspace{-0.2cm}
		\begin{enumerate} \itemsep -2pt
		\item Prior to joining the United States Department of State Foreign Service, this fellowship supports students entering a Masters program in the following fields: \vspace{-0.1cm}
			\begin{enumerate} \itemsep -1pt
			\item {\bf public policy}
			\item international affairs
			\item public administration
			\item academic fields such as: \vspace{-0.1cm}
				\begin{itemize} \itemsep -1pt
				\item business
				\item economics
				\item political science
				\item sociology
				\item foreign languages
				\end{itemize}
			\end{enumerate}
		\item \url{http://www.woodrow.org/higher-education-fellowships/foreign_affairs/pickering_grad/index.php}
		\end{enumerate}
	\item Thomas R. Pickering Undergraduate Foreign Affairs Fellowship (for undergraduates seeking to join the United States Department of State Foreign Service): \url{http://www.woodrow.org/higher-education-fellowships/foreign_affairs/pickering_undergrad/index.php}
	\end{enumerate}
\item Burroughs Wellcome Fund: \vspace{-0.3cm}
	\begin{enumerate} \itemsep -2pt
	\item Career Awards for Medical Scientists (post-Ph.D.): \url{http://www.bwfund.org/pages/188/Career-Awards-for-Medical-Scientists/}
	\item {\bf \color{blue} Career Award for Science and Mathematics Teachers (science or mathematics K-12 teachers in North Carolina public schools)}: \url{http://www.bwfund.org/pages/379/Career-Awards-for-Science-and-Mathematics-Teachers/}
	\end{enumerate}
\item Susan G. Komen for the Cure\textregistered: The Komen College Scholarship Program, \url{http://ww5.komen.org/ResearchGrants/CollegeScholarshipAward.html}
\item University of Kansas Madison \& Lila Self Graduate Fellowship (Ph.D. fellowships for business, economics, and STEM): \url{http://www2.ku.edu/~selfpro/}
\item Nationally Coveted College Scholarships, Graduate School Fellowships \& Postdoctoral Awards: \url{http://scholarships.fatomei.com/}
\item The Andrew W. Mellon Foundation: \vspace{-0.3cm}
	\begin{enumerate} \itemsep -2pt
	\item Fellowships \& Scholarships for undergraduates: \url{http://www.mmuf.org/undergraduates/explore-your-opportunities/fellowships-scholorships}
	\end{enumerate}
\item Siebel Scholars Foundation: \vspace{-0.3cm}
	\begin{enumerate} \itemsep -2pt
	\item For students in selected business, bioengineering, and computer science graduate programs
	\item Only available for students at selected universities.
	\item \url{http://www.siebelscholars.com/scholars}
	\item \url{http://www.siebelscholars.com/}
	\end{enumerate}
\item Aspen Institute (for leaders, e.g. in business, education, community service, and politics): \vspace{-0.3cm}
	\begin{enumerate} \itemsep -2pt
	\item Catto Fellowship Program: \url{http://www.aspeninstitute.org/leadership-programs/catto-fellowship-program}
	\item Rodel Fellowship Program: \url{http://www.aspeninstitute.org/leadership-programs/aspen-institute-rodel-fellowships-public-le-/about-rodel-fellowship-program}
	\item Henry Crown Fellowship Program: \url{http://www.aspeninstitute.org/leadership-programs/henry-crown-fellowship-program}
	\end{enumerate}
\item Smithsonian Institution: \vspace{-0.3cm}
	\begin{enumerate} \itemsep -2pt
	\item Postdoctoral Fellowships, Predoctoral Fellowships, and Graduate Student Fellowships: \vspace{-0.2cm}
		\begin{enumerate} \itemsep -2pt
		\item \url{http://www.si.edu/ofg/infotoapply.htm}
		\item \url{http://www.si.edu/ofg/fell.htm}
		\item \url{http://www.si.edu/ofg/ofgapp.htm}
		\item fields of research and study: \vspace{-0.1cm}
			\begin{enumerate} \itemsep -1pt
			\item {\bf \color{blue} American History, American Material and Folk Culture, and the History of Music and Musical Instruments}
			\item History of Science and Technology
			\item {\bf \color{blue} History of Art, Design, Crafts, and the Decorative Arts}
			\item Anthropology, Archaeology, Linguistics, and Ethnic Studies
			\item Evolutionary, Systematic, Behavioral, Environmental, and Conservation Biology
			\item Earth, Mineral, and Planetary Science
			\item Materials Characterization and Conservation
			\end{enumerate}
		\end{enumerate}
	\item Internship opportunities: \url{http://www.si.edu/ofg/internopp.htm}
	\item Research centers: \url{http://www.si.edu/research/}. [ It also has lots of information for K-12 teachers. It has resources, funding, and internship opportunities for undergraduates and graduate students pursing research in various aspects of humanities, social science, and natural science. ]
	\item Freer Gallery of Art / Arthur M. Sackler Gallery: \vspace{-0.2cm}
		\begin{enumerate} \itemsep -2pt
		\item Fellowships: \url{http://www.asia.si.edu/research/fellowships.asp}
		\end{enumerate}
	\item National Museum of American History: \vspace{-0.2cm}
		\begin{enumerate} \itemsep -2pt
		\item Jerome and Dorothy Lemelson Center for the Study of Invention and Innovation: \vspace{-0.1cm}
			\begin{enumerate} \itemsep -1pt
			\item The Lemelson Center Fellows Program (for Ph.D. students and postdocs): \url{http://invention.smithsonian.org/resources/research_fellowships.aspx}
			\end{enumerate}
		\end{enumerate}
	\end{enumerate}
\item Intercollegiate Studies Institute (ISI): \vspace{-0.3cm}
	\begin{enumerate} \itemsep -2pt
	\item William E. Simon Fellowship for Noble Purpose (for American undergraduates who are planning to use the fellowship grant for serving humanity -- in their own ways): \url{http://www.isi.org/programs/fellowships/simon.html}
	\item {\bf \color{blue} Richard M. Weaver Fellowship (for Americans who are attending a graduate program and are intending to pursue a career in academia/teaching)}: \url{http://www.isi.org/programs/fellowships/richard_weaver.html}
	\item Western Civilization Fellowships (for Americans who are attending a graduate program about Western culture/civilization): \url{http://www.isi.org/programs/fellowships/western_civilization.html}
	\item Salvatori Fellowship (for Americans who are attending a graduate program about early American history): \url{http://www.isi.org/programs/fellowships/salvatori.html}
	\item Bache Renshaw Fellowship for Doctoral Study in Education (for Americans who plan to attend doctoral programs in education): \url{http://www.isi.org/programs/fellowships/bache_renshaw.html}
	\item \url{http://www.isi.org/programs/fellowships/fellowships.html}
	\end{enumerate}
\item Le Fonds qu{\'{e}}b{\'{e}}cois de la recherche sur la nature et les technologies (The Quebec Research Fund on nature and technology): \vspace{-0.3cm}
	\begin{enumerate} \itemsep -2pt
	\item Scholarships: \url{http://www.fqrnt.gouv.qc.ca/en/bourses/index.htm}
	\end{enumerate}
\item Horatio Alger Association of Distinguished Americans, Inc.: \vspace{-0.3cm}
	\begin{enumerate} \itemsep -2pt
	\item Scholarship Programs (for US high school seniors who have faced and overcome great obstacles in their young lives): \url{https://www.horatioalger.org/scholarships/sp.cfm}
	\item Awards: \vspace{-0.2cm}
		\begin{enumerate} \itemsep -2pt
		\item \url{http://www.horatioalger.org/aboutus.cfm}
		\item Horatio Alger Award: ``dedicated community leaders who demonstrate individual initiative and a commitment to excellence; as exemplified by remarkable achievements accomplished through honesty, hard work, self-reliance and perseverance over adversity''
		\item International Horatio Alger Award: ``recipients of this award must have overcome humble beginnings and/or adversity to achieve success. They serve as outstanding role models to the international community and are committed to the Association's mission of encouraging and educating today's young people.''
		\item Norman Vincent Peale Award: ``a Member who has made exceptional humanitarian contributions to society, who has been an active participant in the Association, and who continues to exhibit courage, tenacity and integrity in the face of great challenges. ''
		\end{enumerate}
	\end{enumerate}
\item The W. Garfield Weston Foundation: \vspace{-0.3cm}
	\begin{enumerate} \itemsep -2pt
	\item Entrance Awards \& Upper Year Garfield Weston Awards (for students pursuing college or CEGEP studies in Canada): \url{http://www.garfieldwestonawards.ca/en/about}
	\end{enumerate}
\item Canadian Merit Scholarship Foundation (\url{http://www.cmsf.ca/}): Loran Award (undergraduate funding for Canadian citizens and permanent residents), \url{http://www.loranaward.ca/}
\item StartingBloc: \vspace{-0.3cm}
	\begin{enumerate} \itemsep -2pt
	\item StartingBloc Fellowship: \vspace{-0.2cm}
		\begin{enumerate} \itemsep -2pt
		\item \url{http://www.startingbloc.org/fellowship}
		\item For people who believe that economic value creation and social value creation are complementary... For people who believe in making money and doing good, and creating social and economic impact... 
		\item The Institute for Social Innovation is a ``conference'' to learn about global issues, ``corporate social responsibility, social entrepreneurship, cross sector partnerships and sustainability. Sessions are led by top academics, corporate innovators, social entrepreneurs, activists and government officials.'' 
		\end{enumerate}
	\end{enumerate}
\item The John D. and Catherine T. MacArthur Foundation: \vspace{-0.3cm}
	\begin{enumerate} \itemsep -2pt
	\item Applying for Grants: \url{http://www.macfound.org/site/c.lkLXJ8MQKrH/b.913959/k.E1BE/Applying_for_Grants.htm}
	\item Financial \& Grant Information: \url{http://www.macfound.org/site/c.lkLXJ8MQKrH/b.938093/k.9E4C/Financial__Grant_Information.htm}
	\item MacArthur Fellows Program: \url{http://www.macfound.org/site/c.lkLXJ8MQKrH/b.959463/k.9D7D/Fellows_Program.htm}
	\end{enumerate}
\item Wenner-Gren Foundations (The Wenner-Gren Center Foundation for Scientific Research, The Axel Wenner-Gren Foundation for International Exchange of Scientists and The Foundation Wenner-Grenska Samfundet): Fellowships (for Swedish postdocs), \url{http://www.swgc.org/stipendier.aspx}
\item {\'{E}}gide: \vspace{-0.3cm}
	\begin{enumerate} \itemsep -2pt
	\item EGIDE Latitudes: \url{http://www.egidelatitudes.fr/jahia/Jahia/site/egidelatitudes}
	\item Call for applications to scholarship opportunities (including a scholarship for French citizens to study abroad): \url{http://www.egide.asso.fr/jahia/Jahia/accueil/appels}
	\item Eiffel excellence scholarship programme (organized by the French Ministry of Foreign and European Affairs): \vspace{-0.2cm}
		\begin{enumerate} \itemsep -2pt
		\item \url{http://www.egide.asso.fr/jahia/Jahia/appels/eiffel}
		\item For non-French citizens pursuing advanced degrees.
		\end{enumerate}
	\end{enumerate}
\item Gottlieb Daimler and Karl Benz Foundation: \vspace{-0.3cm}
	\begin{enumerate} \itemsep -2pt
	\item {\bf \color{blue} Ph.D. fellowship for international students to study in Germany}; see \url{http://www.daimler-benz-stiftung.de/home/fellowship/en/start.html}
	\end{enumerate}
\item The San Diego Foundation: \vspace{-0.3cm}
	\begin{enumerate} \itemsep -2pt
	\item San Diego Foundation Community Scholarship Program: \vspace{-0.2cm}
		\begin{enumerate} \itemsep -2pt
		\item \url{http://www.sdfoundation.org/GrantsScholarships/Scholarships.aspx}
		\item Available scholarships: \url{http://www.sdfoundation.org/GrantsScholarships/Scholarships/ForStudents/AvailableScholarships.aspx}. Also, see \url{http://www.sdfoundation.org/GrantsScholarships/Scholarships/ForStudents/AvailableScholarships/CommonApplicationScholarships.aspx#twomey}
		\item It has scholarships for: \vspace{-0.1cm}
			\begin{enumerate} \itemsep -1pt
			\item graduating high school seniors
			\item current undergraduates
			\item non-traditional college students: \vspace{-0.1cm}
				\begin{itemize} \itemsep -1pt
				\item mature-age students
				\item mature student
				\item adult learner
				\item adult student
				\item adults who are returning to college
				\end{itemize}
			\item people pursuing teaching certificates
			\item students attending grad school
			\item students attending trade/vocational school
			\item foster youth
			\item students in various ethnic groups
			\item students in different geographical locations
			\item {\bf \color{blue} students pursuing education in certain fields, such as engineering, nursing, music, and arts and humanities}
			\end{enumerate}
		\item Separate Scholarships: \url{http://www.sdfoundation.org/GrantsScholarships/Scholarships/ForStudents/AvailableScholarships/SeparateScholarships.aspx}
		\item Other Scholarships and Financial Aid Resources: \url{http://www.sdfoundation.org/GrantsScholarships/Scholarships/ForStudents/AvailableScholarships/OtherScholarshipsandFinancialAidResources.aspx}
		\item Financial Aid Information: \url{http://www.sdfoundation.org/GrantsScholarships/Scholarships/ForStudents/Resources/FinancialAidInformation.aspx}
		\end{enumerate}
	\item Grant Opportunities (for non-profit organizations): \url{http://www.sdfoundation.org/GrantsScholarships/ForNonprofits/GrantOpportunities.aspx}
	\end{enumerate}
\item Ewing Marion Kauffman Foundation: \vspace{-0.3cm}
	\begin{enumerate} \itemsep -2pt
	\item Kauffman Dissertation Fellowship Program (for ``Ph.D., D.B.A., or other doctoral students at accredited U.S. universities to support dissertations in the area of entrepreneurship''): \url{http://www.kauffman.org/research-and-policy/kauffman-dissertation-fellowship-program.aspx}
	\item Kauffman Junior Faculty Fellowship in Entrepreneurship Research: \vspace{-0.2cm}
		\begin{enumerate} \itemsep -2pt
		\item \url{http://www.kauffman.org/research-and-policy/kauffman-junior-faculty-fellowship-in-entrepreneurship.aspx}
		\item ``to recognize tenured or tenure-track junior faculty members at accredited U.S. universities who are beginning to establish a record of scholarship and exhibit the potential to make significant contributions to the body of research in the field of entrepreneurship''
		\end{enumerate}
	\item Ewing Marion Kauffman Prize Medal for Distinguished Research in Entrepreneurship (for promising young scholars in the field of entrepreneurship): \url{http://www.kauffman.org/research-and-policy/kauffman-prize-medal-for-entrepreneurship-research.aspx}
	\item Kauffman Legal Fellowship Program (for post-J.D. research fellowship): \url{http://www.kauffman.org/research-and-policy/kauffman-legal-fellowship-program.aspx}
	\item Kauffman Global Scholars Program (for non-American top young entrepreneurs): \url{http://www.kauffman.org/entrepreneurship/kauffman-global-scholars-program.aspx}
	\item Entrepreneur Fellows program (for M.D.s and Ph.D.s who want to become high-tech start-up entrepreneurs): \url{http://www.kauffman.org/entrepreneurship/entrepreneur-fellows-program.aspx}
	\item Entrepreneur Postdoctoral Fellows program (for postdocs who want to become high-tech start-up entrepreneurs): \url{http://www.kauffman.org/entrepreneurship/entrepreneur-postdoctoral-fellows-program.aspx}
	\item Kauffman Fellows Program (``to educate and train future venture capitalists and future leaders of high-growth companies''): \url{http://www.kauffman.org/entrepreneurship/kauffman-fellows.aspx}
	\item Kauffman Foundation Outstanding Postdoctoral Entrepreneur Award: \url{http://www.kauffman.org/entrepreneurship/outstanding-postdoctoral-entrepreneur-award.aspx}
	\end{enumerate}
\item Killam Fellowships Program: \vspace{-0.3cm}
	\begin{enumerate} \itemsep -2pt
	\item \url{http://www.killamfellowships.com/}
	\item The Killam Fellowships Program allows undergraduate students from Canada and the United States to participate in a program of binational residential exchange.
	\item Killam Fellows spend either one semester or a full academic year as an exchange student in the host country.
	\end{enumerate}
\item Canada Council for the Arts: \vspace{-0.3cm}
	\begin{enumerate} \itemsep -2pt
	\item Killam Research Fellowship: \vspace{-0.2cm}
		\begin{enumerate} \itemsep -2pt
		\item \url{http://killam.canadacouncil.ca/welcome.asp}
		\item For researchers in the following fields, and interdisciplinary fields between these fields: \vspace{-0.1cm}
			\begin{enumerate} \itemsep -1pt
			\item humanities
			\item social sciences
			\item natural sciences
			\item health sciences
			\item engineering
			\end{enumerate}
		\item For outstanding researchers who are Canadian citizens or permanent residents
		\end{enumerate}
	\item Killam Prizes (and Killam Research Fellowships): \url{http://www.canadacouncil.ca/prizes/killam}
	\end{enumerate}
\item Killam Trusts: \vspace{-0.3cm}
	\begin{enumerate} \itemsep -2pt
	\item Killam Scholarship and Prize Programs (multiple fields in selected Canadian universities): \url{http://www.killamtrusts.ca/index.asp}
	\item Killam Award winners: \url{http://www.killamtrusts.ca/awardwinners.asp}
	\item Killam Scholarship and Prize Programs at various institutions (including universities): \url{http://www.killamtrusts.ca/uofAlberta.asp}
	\end{enumerate}
\item U.S. Department of State: \vspace{-0.3cm}
	\begin{enumerate} \itemsep -2pt
	\item Bureau of Educational and Cultural Affairs: \vspace{-0.2cm}
		\begin{enumerate} \itemsep -2pt
		\item Institute of International Education (administrator of program): \vspace{-0.1cm}
			\begin{enumerate} \itemsep -1pt
			\item Council for International Exchange of Scholars: \vspace{-0.1cm}
				\begin{itemize} \itemsep -1pt
				\item Fulbright Programs (for U.S. and non-U.S. Scholars): \url{http://www.cies.org/Fulbright_programs.htm}; \url{http://www.cies.org/about_fulb.htm}; \url{http://us.fulbrightonline.org/about.html}; \url{http://foreign.fulbrightonline.org/}; \url{http://exchanges.state.gov/academicexchanges/index/fulbright-program.html}; and \url{http://fulbright.state.gov/}
				\item Hubert H. Humphrey Fellowship Program: \vspace{-0.1cm}
					\begin{itemize} \itemsep -1pt
					\item For mid-career professionals in the following fields: economic development/finance and banking, agricultural and rural development, natural resources, environmental policy, and climate change, human resource management, communications/journalism, teaching of English as a foreign language, educational administration, planning, and policy, substance abuse education, treatment, and prevention, HIV/AIDS policy and prevention, public health policy and management, {\bf public policy} analysis and public administration, law and human rights, urban and regional planning, trafficking in persons - policy and prevention, technology policy and management, and higher education administration
					\item \url{http://www.humphreyfellowship.org/}
					\item \url{http://exchanges.state.gov/globalexchanges/humphrey-fellowship.html}
					\end{itemize}
				\end{itemize}
			\item International programs for scholars (search under each continent): \url{http://www.iie.org/en/Our-Global-Reach}
			\end{enumerate}
		\item International Documentary Filmmakers Fellowship: \vspace{-0.1cm}
			\begin{enumerate} \itemsep -1pt
			\item \url{http://exchanges.state.gov/cultural/docfilmmakers.html}
			\item \url{http://smpa.gwu.edu/doccenter/fellowship.php}
			\item For ``emerging or mid-career documentary filmmakers''
			\item Intensive six-week program at the Documentary Center, The George Washington University
			\end{enumerate}
		\item Office of English Language Programs: \vspace{-0.1cm}
			\begin{enumerate} \itemsep -1pt
			\item English Language Fellow Program (for ``highly qualified U.S. educators in the field of Teaching English to Speakers of Other Languages, TESOL''): \url{http://exchanges.state.gov/englishteaching/el-fellow.html}
			\item English Language Specialist Program: \vspace{-0.1cm}
				\begin{itemize} \itemsep -1pt
				\item \url{http://exchanges.state.gov/englishteaching/el-specialist.html}
				\item U.S. academics in the fields of Teaching English as a Foreign Language (TEFL) / Teaching English as a Second Language (TESL) and Applied Linguistics
				\end{itemize}
			\item E-Teacher Scholarship Program (for English teaching professionals living outside of the United States): \url{http://exchanges.state.gov/englishteaching/eteacher.html}
			\item English Access Microscholarship Program (Access): \vspace{-0.1cm}
				\begin{itemize} \itemsep -1pt
				\item \url{http://exchanges.state.gov/englishteaching/eam.html}
				\item The English Access Microscholarship Program (Access) provides a foundation of English language skills to non-elite, 14 - 18 year old students through afterschool classes and intensive summer learning activities.
				\end{itemize}
			\item \url{http://exchanges.state.gov/englishteaching/index.html}
			\end{enumerate}
		\item Office of Global Educational Programs: \vspace{-0.1cm}
			\begin{enumerate} \itemsep -1pt
			\item Community College Initiative: \vspace{-0.1cm}
				\begin{itemize} \itemsep -1pt
				\item For ``individuals from Brazil, Egypt, Ghana, Indonesia, Pakistan, South Africa, Turkey, and selected countries in Central America to spend one year studying at community colleges in the United States and earn a vocational certificate.''
				\item ``The program provides academic instruction in selected fields including agriculture, applied engineering, business management and administration, health professions, information technology, media, and tourism and hospitality management, while also immersing participants in U.S. society and cultural life.''
				\item ``Participants are recruited from historically underserved populations and may not have had opportunities for formal job training or higher education. Most participants are in their early- to mid-twenties and many already have work experience.''
				\item \url{http://exchanges.state.gov/globalexchanges/community-colleges-initiative.html}
				\end{itemize}
			\item {\bf \color{blue} Benjamin A. Gilman International Scholarship Program}: \vspace{-0.1cm}
				\begin{itemize} \itemsep -1pt
				\item ``The Benjamin A. Gilman International Scholarship Program provides scholarships to U.S. undergraduates with financial need for study abroad, including students from diverse backgrounds and students going to non-traditional study abroad destinations.''
				\item ``The applicant must be receiving a Federal Pell Grant or provide proof that he/she will be receiving a Pell Grant at the time of application or during the term of his/her study abroad.''
				\item \url{http://exchanges.state.gov/globalexchanges/gilman-scholarship-program.html}
				\end{itemize}
			\item Global Undergraduate Exchange Program (Global UGRAD Program): \vspace{-0.1cm}
				\begin{itemize} \itemsep -1pt
				\item \url{http://exchanges.state.gov/academicexchanges/guep.html}
				\item The Global Undergraduate Exchange Program (also known as the Global UGRAD Program) provides one semester and academic year scholarships to outstanding undergraduate students from underrepresented sectors in East Asia, Eurasia and Central Asia, the Near East and South Asia and the Western Hemisphere for non-degree full-time study combined with community service, internships and cultural enrichment.
				\end{itemize}
			\item Professors and Research Scholars: \url{http://exchanges.state.gov/jexchanges/programs/professor.html}
			\item Short-Term Scholar: \url{http://exchanges.state.gov/jexchanges/programs/shortterm.html}
			\item Student, College/University: \vspace{-0.1cm}
				\begin{itemize} \itemsep -1pt
				\item \url{http://exchanges.state.gov/jexchanges/programs/ucstudent.html}
				\item The College/University Student Program gives foreign students the opportunity to study at an American degree-granting post-secondary accredited educational institution, including colleges and universities. Students may participate in degree and non-degree programs. They must pursue a full-time course of study and maintain satisfactory advancement toward the completion of their academic program.
				\end{itemize}
			\item Study of the United States Institutes for Scholars: \vspace{-0.1cm}
				\begin{itemize} \itemsep -1pt
				\item Study of the United States Institutes for Scholars  are designed to strengthen curricula and improve the quality of teaching about the United States in academic institutions overseas.
				\item Foreign university faculty, secondary educators and other scholars spend approximately four weeks at host universities where they take part in a series of lectures, seminar discussions and site visits related to each institute's theme.
				\item They learn about American educational philosophies, explore new teaching methods and pursue related research interests.
				\item Interests of these institutes: \vspace{-0.1cm}
					\begin{itemize} \itemsep -1pt
					\item American Politics and Political Thought
					\item Contemporary American Literature
					\item Journalism and Media
					\item Religious Pluralism in the United States
					\item Secondary School Educators
					\item U.S. Culture and Society
					\item U.S. Foreign Policy
					\item U.S. National Security
					\end{itemize}
				\item \url{http://exchanges.state.gov/academicexchanges/scholars.html}
				\end{itemize}
			\item Study of the United States Institutes for Student Leaders: \vspace{-0.1cm}
				\begin{itemize} \itemsep -1pt
				\item Study of the United States Institutes for Student Leaders are five-to-six-week academic programs for foreign undergraduate leaders.
				\item Hosted by U.S. academic institutions throughout the United States, the Student Leader Institutes include an intensive academic component, an educational tour of other regions of the country, local community service activities and a unique opportunity for participants to get to know their American peers.
				\item \url{http://exchanges.state.gov/academicexchanges/students.html}
				\item Interests of the institutes: \vspace{-0.1cm}
					\begin{itemize} \itemsep -1pt
					\item Comparative {\bf Public Policy} for Pakistani Student Leaders
					\item Energy and the Environment
					\item Global Environmental Issues
					\item New Media
					\item Religious Pluralism in the U.S.
					\item Social Entrepreneurship
					\item U.S. Foreign Policy for East Asian Student Leaders
					\item Western Hemisphere Student Leaders 
					\item Women's Leadership
					\end{itemize}
				\end{itemize}
			\item Edmund S. Muskie Graduate Fellowship: \vspace{-0.1cm}
				\begin{itemize} \itemsep -1pt
				\item \url{http://exchanges.state.gov/academicexchanges/muskie.html}
				\item The Edmund S. Muskie Graduate Fellowship Program (Muskie) confers fellowships for Master's degree-level study in the U.S. in the fields of business administration, economics, education, environmental policy and management, international affairs, journalism/mass communications, law, library and information science, public administration, public health and {\bf public policy} for students and professionals from Eurasia.
				\item Candidates are recruited through a merit-based competition administered by the International Research \& Exchanges Board (IREX).
				\item U.S. host campuses are also selected through a competition process and generally provide tuition waivers of fifty percent.
				\item Approximately 145 fellowships are awarded each academic year.
				\end{itemize}
			\item Critical Language Scholarship Program: \vspace{-0.1cm}
				\begin{itemize} \itemsep -1pt
				\item \url{http://exchanges.state.gov/academicexchanges/sli2.html}
				\item The Critical Language Scholarship (CLS) Program provides overseas foreign language instruction and cultural enrichment experiences in 13 critical need languages for U.S. students in higher education.
				\item The CLS Program is part of a U.S. government effort to expand dramatically the number of Americans studying and mastering critical need foreign languages.
				\item Undergraduate, master's and doctoral-level students of diverse disciplines and majors are encouraged to apply for the seven-to-10-week-long programs.
				\item Participants are expected to continue their language study beyond the scholarship period, and later apply their critical language skills in their future professional careers.
				\end{itemize}
			\item Critical Language Enhancement Award (CLEA): \vspace{-0.1cm}
				\begin{itemize} \itemsep -1pt
				\item \url{http://exchanges.state.gov/academicexchanges/clea2.html}
				\item The Critical Language Enhancement Award (CLEA) provides funding to eligible Fulbright U.S. Student Program Grantees who intend to use one of the following languages for their Fulbright project: \vspace{-0.1cm}
					\begin{itemize} \itemsep -1pt
					\item Arabic (all dialiects)
					\item Azeri
					\item Bangla/Bengali
					\item Bhasa Indonesia
					\item Chinese (Mandarin Only)
					\item Farsi
					\item Gujarati
					\item Hindi
					\item Korean
					\item Marathi
					\item Pashto
					\item Punjabi
					\item Russian
					\item Turkish
					\item Urdu
					\end{itemize}
				\end{itemize}
			\end{enumerate}
		\item Office of International Visitors: \vspace{-0.1cm}
			\begin{enumerate} \itemsep -1pt
			\item International Visitor Leadership Program (IVLP): \vspace{-0.1cm}
				\begin{itemize} \itemsep -1pt
				\item \url{http://exchanges.state.gov/ivlp/index.html}
				\item \url{http://exchanges.state.gov/ivlp/ivlp.html}
				\item The Office of International Visitors manages and funds the International Visitor Leadership Program (IVLP).
				\item Launched in 1940, the IVLP is a professional exchange program that seeks to build mutual understanding between the U.S. and other nations through carefully designed short-term visits to the U.S. for current and emerging foreign leaders.
				\item These visits reflect the International Visitors' professional interests and support the foreign policy goals of the United States.
				\end{itemize}
			\end{enumerate}
		\item Program Search (find international exchange programs sponsored by the Bureau of Educational and Cultural Affairs): \url{http://exchanges.state.gov/index/search.html}
		\end{enumerate}
	\end{enumerate}
\item Mexican American Legal Defense and Educational Fund (MALDEF): \vspace{-0.3cm}
	\begin{enumerate} \itemsep -2pt
	\item Scholarship Resources: \url{http://maldef.org/leadership/scholarships/}
	\item MALDEF Law School Scholarship Program: \vspace{-0.2cm}
		\begin{enumerate} \itemsep -2pt
		\item MALDEF's Law School Scholarship Program provides several scholarships in varying amounts to deserving law students with a commitment to advancing the civil rights of Latinos.
		\item MALDEF's Law School Scholarship Program is open to all law students who will be enrolled full-time in an American-accredited law school in 2010-2011.
		\item Scholarships are awarded to students based on their commitment to serve the Latino community through law; their past achievement and potential for achievement; and their financial need.
		\item \url{http://maldef.org/leadership/scholarships/law_school_scholarship_program/index.html}
		\end{enumerate}
	\item Undergraduate Scholarship Resource Guide: \url{http://maldef.org/leadership/scholarships/resources/index.html}
	\end{enumerate}
\item Ashoka: \vspace{-0.3cm}
	\begin{enumerate} \itemsep -2pt
	\item Ashoka Fellows (to promote and support social entrepreneurship): \url{http://www.ashoka.org/fellows}
	\end{enumerate}
\item Heinz Family Foundation: \vspace{-0.3cm}
	\begin{enumerate} \itemsep -2pt
	\item Heinz Award Criteria: \vspace{-0.2cm}
		\begin{enumerate} \itemsep -2pt
		\item \url{http://heinzawards.net/awards/criteria}
		\item The Heinz Endowments
		\item Attributes and qualities of awardees: \vspace{-0.1cm}
			\begin{enumerate} \itemsep -1pt
			\item an enormous capacity to love
			\item smile
			\item take risks
			\item question
			\item work hard
			\item believe in the power of the individual to improve the lives of others
			\end{enumerate}
		\item ``Candidates [should] possess a remarkable mix of vision, optimism, creativity and hard work which, when combined, produce tangible achievements of lasting good.''
		\item Nominees must exhibit the following personal characteristics: \vspace{-0.1cm}
			\begin{enumerate} \itemsep -1pt
			\item A passion for excellence that goes beyond intellectual curiosity;
			\item A concern for humanity rooted in a deep sensitivity for the well-being of others; 
			\item A knowledge of self which acknowledges weaknesses but relies on individual strengths;
			\item A gritty determination that will see a job through to completion despite the inevitable setbacks;
			\item A broad vision which extends far beyond the particular and embraces something universal.
			\end{enumerate}
		\item Work of the candidates for a Heinz Award must meet the following criteria: \vspace{-0.1cm}
			\begin{enumerate} \itemsep -1pt
			\item Be significant and not a ``quick fix.''
			\item Have an enduring and meaningful impact.
			\item Be creative and innovative, and
			\item Be sufficiently tangible to serve as a model for replication elsewhere.
			\end{enumerate}
		\item ``In addition, candidates should be actively working in the field in which they are nominated with the hope that, in receiving this award, their potential for future societal contribution will be enhanced.''
		\end{enumerate}
	\item Categories: \vspace{-0.2cm}
		\begin{enumerate} \itemsep -2pt
		\item Arts \& Humanities
		\item Environment
		\item Human Condition
		\item {\bf Public Policy}
		\item Technology, Economy, + Employment
		\end{enumerate}
	\end{enumerate}
\item Echoing Green: \vspace{-0.3cm}
	\begin{enumerate} \itemsep -2pt
	\item Echoing Green Fellowship: \vspace{-0.2cm}
		\begin{enumerate} \itemsep -2pt
		\item \url{http://www.echoinggreen.org/fellowship}
		\item Has information on eligibility, the benefits of the fellowship, and application cycle and dates.
		\end{enumerate}
	\item Echoing Green Fellows: \url{http://www.echoinggreen.org/fellows}
	\end{enumerate}
\item Ben Franklin Technology Partners (BFTP): \vspace{-0.3cm}
	\begin{enumerate} \itemsep -2pt
	\item Innovation Works (IW): \vspace{-0.2cm}
		\begin{enumerate} \itemsep -2pt
		\item AlphaLab: \vspace{-0.1cm}
			\begin{enumerate} \itemsep -1pt
			\item ``An immersive environment where entrepreneurs can tap IW's onsite experts for business and market advice and exchange ideas with other entrepreneurs launching in similar markets''
			\end{enumerate}
		\end{enumerate}
	\end{enumerate}
\item Carnegie Corporation of New York: \vspace{-0.3cm}
	\begin{enumerate} \itemsep -2pt
	\item Carnegie Scholars Program (not available in 2010): \url{http://carnegie.org/programs/carnegie-scholars/}
	\end{enumerate}
\item New York Women's Foundation: \vspace{-0.3cm}
	\begin{enumerate} \itemsep -2pt
	\item Finch Scholar Program (with the Finch College Alumnae Association): \vspace{-0.2cm}
		\begin{enumerate} \itemsep -2pt
		\item \url{http://www.nywf.org/internship.html} and \url{http://www.finchcollege.org/}
		\item ``Our partnership with the Finch Scholar Program allows us to provide practical community service experience to an outstanding local student enrolled in college. The internship affords the Finch Scholar opportunities to work in meaningful ways in a nonprofit organization with exposure to social change philanthropy, participatory grantmaking, advocacy and {\bf public policy}. Generally, we offer one scholarship per year with a stipend.''
		\item \url{http://www.finchcollege.org/newFinchScholarPrgm.html}
		\item \url{http://www.finchcollege.org/newscholarships.html}
		\end{enumerate}
	\end{enumerate}
\item The Rockefeller Foundation: \vspace{-0.3cm}
	\begin{enumerate} \itemsep -2pt
	\item The Bellagio Center: \vspace{-0.2cm}
		\begin{enumerate} \itemsep -2pt
		\item \url{http://www.rockefellerfoundation.org/bellagio-center}
		\item Residency Programs: \vspace{-0.1cm}
			\begin{enumerate} \itemsep -1pt
			\item \url{http://www.rockefellerfoundation.org/bellagio-center/residency-programs}
			\item ``The Bellagio Residency program offers scholars, artists, thought leaders, policymakers and practitioners a serene setting conducive to focused, goal-oriented work, and the unparalleled opportunity to establish new connections with fellow residents, across a stimulating array of disciplines and geographies.  The Bellagio Center community generates new knowledge to solve some of the most complex problems facing our world and creates art that inspires reflection, understanding, and imagination.''
			\item Scholarly Residencies: \vspace{-0.1cm}
				\begin{itemize} \itemsep -1pt
				\item ``Researchers in the humanities, natural sciences, social sciences and other academic disciplines''
				\item ``The Center typically offers one-month residencies for no more than 12 scholars and scientists at a time. Individuals in any discipline and from any part of the world are welcome to apply. The Center maintains a core focus on projects consistent with the Foundation's mission to expand opportunities for poor or vulnerable people and to help see that the benefits of globalization are shared more widely. It also seeks to include beyond that core a wide variety of projects from all academic disciplines.''
				\item \url{http://www.rockefellerfoundation.org/bellagio-center/residency-programs/scholarly-residencies}
				\end{itemize}
			\item Creative Artist Residencies: \vspace{-0.1cm}
				\begin{itemize} \itemsep -1pt
				\item ``Artists, composers, writers''
				\item ``Bellagio creative artist residencies for composers, novelists, playwrights, poets, video/filmmakers and visual artists provide time for disciplined work, individual reflection, and collegial engagement, uninterrupted by the usual professional and personal demands. The Center typically offers one-month stays for no more than three to five creative artists at a time. Artists of significant achievement from any country are welcome to apply.''
				\item \url{http://www.rockefellerfoundation.org/bellagio-center/residency-programs/creative-artist-residencies}
				\end{itemize}
			\item Practitioner Residencies: \vspace{-0.1cm}
				\begin{itemize} \itemsep -1pt
				\item ``Policymakers, nonprofit leaders, journalists and public advocates''
				\item ``The Center offers residencies to professionals in fields relevant to the Rockefeller Foundation's issue areas. The Center maintains a core focus on projects consistent with our mission, to expand opportunities for poor or vulnerable people and to help see that the benefits of globalization are shared more widely.   We seek practitioner applicants with demonstrated leadership qualities and the capacity to contribute to the intellectual life at the Center.''
				\item \url{http://www.rockefellerfoundation.org/bellagio-center/residency-programs/practitioner-residencies}
				\end{itemize}
			\end{enumerate}
		\item {\bf \color{blue} Creative Arts Fellowships}: \vspace{-0.1cm}
			\begin{enumerate} \itemsep -1pt
			\item ``This high-profile program hosts visual artists at the Bellagio Center for three-month residencies that inspire creativity and promote interaction between the arts and other fields. Creative Arts Fellows, like other participants in Bellagio residency programs, have the time and space to work independently during the day. They also enjoy and benefit from a lively community of scholars, writers, policymakers and other artists who gather in the evening for dinner and occasional presentations.  The combination of private work space, an extended stay, a generous stipend and a unique group of fellow residents makes a Creative Arts Fellowship at the Bellagio Center a remarkable opportunity.''
			\item \url{http://www.rockefellerfoundation.org/bellagio-center/creative-arts-fellowships}
			\end{enumerate}
		\end{enumerate}
	\end{enumerate}
\item Wellcome Trust: \vspace{-0.3cm}
	\begin{enumerate} \itemsep -2pt
	\item Wellcome Trust Book Prize: \vspace{-0.2cm}
		\begin{enumerate} \itemsep -2pt
		\item \url{http://www.wellcomebookprize.org/About-the-prize/index.htm}
		\item ``The Wellcome Trust Book Prize celebrates the best of medicine in literature by awarding 25 000 each year for the finest fiction or non-fiction book centered around medicine.''
		\end{enumerate}
	\end{enumerate}
\item The Kennedy Memorial Trust: \vspace{-0.3cm}
	\begin{enumerate} \itemsep -2pt
	\item \url{http://www.kennedytrust.org.uk/}
	\item Kennedy Scholarship: \url{http://www.kennedytrust.org.uk/display.aspx?Id=1165&pid=0}
	\item Frank Knox Fellowships: \url{http://www.kennedytrust.org.uk/display.aspx?Id=1175&pid=0}
	\end{enumerate}
\item Foreign \& Commonwealth Office / United Kingdom: \vspace{-0.3cm}
	\begin{enumerate} \itemsep -2pt
	\item Chevening scholarships: \vspace{-0.2cm}
		\begin{enumerate} \itemsep -2pt
		\item \url{http://www.fco.gov.uk/en/about-us/what-we-do/scholarships/}
		\item ``The Chevening programme, has, over 26 years, provided more than 30,000 Scholarships at Higher Education Institutions (HEIs) in the UK for postgraduate students or researchers from countries across the world.''
		\end{enumerate}
	\item {\bf Marshall Scholarships} finance young Americans of high ability to study for a graduate degree in the United Kingdom: \url{http://www.marshallscholarship.org/}
	\end{enumerate}
\item Ministry of Education, Culture, Sports, Science and Technology (MEXT) / Japan: \vspace{-0.3cm}
	\begin{enumerate} \itemsep -2pt
	\item \url{http://www.mext.go.jp/english/}
	\item Monbukagakusho Scholarship: \vspace{-0.2cm}
		\begin{enumerate} \itemsep -2pt
		\item \url{http://en.wikipedia.org/wiki/Monbukagakusho_Scholarship}
		\item \url{http://project.monbusho.org/old/} and \url{http://www.monbusho.org/}
		\end{enumerate}
	\end{enumerate}
\item Institute of International Education (IIE): \vspace{-0.3cm}
	\begin{enumerate} \itemsep -2pt
	\item GE Foundation Scholar-Leaders Program: \vspace{-0.2cm}
		\begin{enumerate} \itemsep -2pt
		\item \url{http://www.iie.org/en/Programs/GE-Foundation-Scholar-Leaders-Program}
		\item ``The GE Foundation Scholar-Leaders Program began in 1987 in Mexico and now supports outstanding students in higher education in fourteen countries around the world. The program initially provided traditional financial support for university education, but has developed into an exciting Leadership Development Program to complement the student's academic curriculum.''
		\item Eligibility: ``Students in their first year of study in engineering, technology, business, finance, management, or economics attending a participating university. GE Foundation Scholar-Leaders qualification requirements vary by region.''
		\end{enumerate}
	\end{enumerate}
\item British Council: \vspace{-0.3cm}
	\begin{enumerate} \itemsep -2pt
	\item Shine! 2011: International Student Awards: \vspace{-0.2cm}
		\begin{enumerate} \itemsep -2pt
		\item \url{http://www.educationuk.org/shine}
		\item For international students in the United Kingdom
		\end{enumerate}
	\item Funding your studies: \vspace{-0.2cm}
		\begin{enumerate} \itemsep -2pt
		\item \url{http://www.britishcouncil.org/learning-funding-your-studies.htm}
		\item Education UK: \url{http://www.educationuk.org/pls/hot_bc/page_pls_user_advice?x=&y=&a=0&d=4460}
		\item 9/11 Scholarship Fund: \vspace{-0.1cm}
			\begin{enumerate} \itemsep -1pt
			\item \url{http://www.britishcouncil.org/911scholarships.htm}
			\item ``The 9/11 Scholarship Fund supports international students who were directly affected by the 2001 terrorist events in the US. Find out more how each scholarship offers the opportunity to study at a UK college or university every year.''
			\end{enumerate}
		\end{enumerate}
	\item {\it Youth in Action} European program: \url{http://www.britishcouncil.org/youthinaction}
	\item British Council Arts Group: \vspace{-0.2cm}
		\begin{enumerate} \itemsep -2pt
		\item Support and funding overview: \url{http://www.britishcouncil.org/arts-support-and-funding-overview.htm}
		\item Visual arts support and funding: \url{http://www.britishcouncil.org/arts-visual-arts-funding.htm}
		\item Drama and dance support and funding: \url{http://www.britishcouncil.org/arts-performing-arts-funding.htm}
		\item Literature support and funding: \url{http://www.britishcouncil.org/arts-literature-support-and-funding.htm}
		\item Film support and funding: \url{http://www.britishcouncil.org/arts-film-funding.htm}
		\item Music support and funding: \url{http://www.britishcouncil.org/arts-music-funding.htm}
		\item Architecture, design, fashion support and funding: \url{http://www.britishcouncil.org/arts-adf-funding.htm}
		\item International Short Film Festival Support Scheme: \url{http://www.britishcouncil.org/arts-film-short-films-scheme.htm}
		\end{enumerate}
	\end{enumerate}
\item Alfred P. Sloan Foundation: \vspace{-0.3cm}
	\begin{enumerate} \itemsep -2pt
	\item Sloan Research Fellowships: \vspace{-0.2cm}
		\begin{enumerate} \itemsep -2pt
		\item \url{http://www.sloan.org/fellowships}
		\item Hold a Ph.D. (or equivalent) in chemistry, physics, mathematics, computer science, economics, neuroscience or computational and evolutionary molecular biology, or in a related interdisciplinary field;
		\item Be members of the regular faculty (i.e., tenure track) of a degree-granting college or university in the United States or Canada; and
		\item Normally, be no more than six years from completion of the most recent Ph.D. or equivalent as of the year of their nomination.
		\end{enumerate}
	\end{enumerate}
\item --- --- --- --- --- --- --- --- --- --- --- --- --- --- --- --- --- --- --- --- --- --- --- --- --- --- --- --- --- --- ---
\item \colorbox{blue}{\bf Scholarships and Fellowships in Business (including Finance, Entrepreneurship, and Accounting)}
% Scholarships and Fellowships in Business (including Finance, Entrepreneurship, and Accounting)
\item IREX: \vspace{-0.3cm}
	\begin{enumerate} \itemsep -2pt
	\item Opportunities ``for individuals, organizations, universities, and alumni'': \url{http://www.irex.org/apply}
	\item Edmund S. Muskie Graduate Fellowship Program: \vspace{-0.2cm}
		\begin{enumerate} \itemsep -2pt
		\item : \url{http://www.irex.org/application/edmund-s-muskie-graduate-fellowship-program-application}
		\item ``The Muskie Program is open to graduate students and professionals from Armenia, Azerbaijan, Belarus, Georgia, Kazakhstan, Kyrgyzstan, Moldova, Russia, Tajikistan, Turkmenistan, Ukraine and Uzbekistan for one-year non-degree, one-year degree, or two-year degree study in the United States.''
		\item ``Eligible fields of study for the Muskie Program are: business administration, economics, education, environmental management, international affairs, journalism and mass communication, law, library and information science, public administration, public health, and {\bf public policy}.''
		\end{enumerate}
	\end{enumerate}
\item Sponsors for Educational Opportunity (SEO): \vspace{-0.3cm}
	\begin{enumerate} \itemsep -2pt
	\item Alternative Investment Fellowship Program: \vspace{-0.2cm}
		\begin{enumerate} \itemsep -2pt
		\item \url{http://www.seo-usa.org/Fellowship}
		\item Eligibility: \vspace{-0.1cm}
			\begin{enumerate} \itemsep -1pt
			\item \url{http://www.seo-usa.org/FellowshipEligibility}
			\item The program is open to professionals traditionally underrepresented in alternative investments who are in the first year (or second year with a third-year offer) of an analyst program at an investment bank.
			\item Corporate finance, M\&A, leveraged finance and structured finance analysts are preferred.
			\item Management consultants will also be considered.
			\end{enumerate}
		\end{enumerate}
	\item The SEO Scholars Program: \vspace{-0.2cm}
		\begin{enumerate} \itemsep -2pt
		\item \url{http://www.seo-usa.org/Scholars}
		\item The SEO Scholars Program is a rigorous out-of-school academic enrichment program that prepares motivated New York City public high school students of color to gain admission to and succeed at competitive colleges and universities throughout the country.  Numerous studies confirm that rigorous academics are the single most important factor for low-income and minority students in gaining college admission and earning a degree.  However, U.S. Department of Education research shows that ``A'' work in low-income schools equals ``C'' work in affluent schools.
		\item Admissions: \url{http://www.seo-usa.org/ScholarsAdmissions}
		\item Roadmap To Success: \url{http://www.seo-usa.org/ScholarsRoadmapToSuccess}
		\item Enrichment Programs: \url{http://www.seo-usa.org/ScholarsEnrichmentPrograms}
		\item Volunteering: \url{http://www.seo-usa.org/ScholarsVolunteering}
		\item Andrew Golkin Fund: \vspace{-0.1cm}
			\begin{enumerate} \itemsep -1pt
			\item \url{http://www.seo-usa.org/ScholarsAndrewGolkinFund}
			\item \url{http://www.seo-usa.org/andrewgolkinfund/index.html}
			\end{enumerate}
		\item Franklin H. and Shirley B. Williams Scholarship Fund: \url{http://www.seo-usa.org/ScholarsFHSBW}
		\item The Advantages of Attending a Competitive College: \url{http://www.seo-usa.org/ScholarsAdvantages}
		\end{enumerate}
	\item Career program: \vspace{-0.2cm}
		\begin{enumerate} \itemsep -2pt
		\item \url{http://www.seo-usa.org/Career}
		\item The SEO Career Program places students of color interested in finance, philanthropy, business and corporate law in internships with competitive pay, rigorous training, support through mentors, and broad access to industry professionals. 
		\item Sponsors for Educational Opportunity (SEO) is the nation's premiere summer internship program for talented underrepresented students of color that can lead to full-time job offers.
		\item SEO offers internship opportunities in the following areas: \vspace{-0.1cm}
			\begin{enumerate} \itemsep -1pt
			\item Corporate Financial Leadership: \url{http://www.seo-usa.org/Career/Corporate_Financial_Leadership}
			\item Banking/Asset Management Areas: \vspace{-0.1cm}
				\begin{itemize} \itemsep -1pt
				\item Investment Banking: \url{http://www.seo-usa.org/Career/Investment_Banking}
				\item Sales \& Trading: \url{http://www.seo-usa.org/Career/Sales_&_Trading}
				\item Investment Research: \url{http://www.seo-usa.org/Career/Investment_Research}
				\item Transaction Services: \url{http://www.seo-usa.org/Career/Transaction_Services}
				\item Asset Management: \url{http://www.seo-usa.org/Career/Asset_Management}
				\item Accounting/Finance: \url{http://www.seo-usa.org/Career/Accounting/Finance}
				\item Information Technology: \url{http://www.seo-usa.org/Career/Information_Technology}
				\end{itemize}
			\item Corporate Law: \url{http://www.seo-usa.org/Career/Corporate_Law}
			\item Nonprofit: \url{http://www.seo-usa.org/Career/Nonprofit}
			\item SEO-U: Freshmen and Sophomore Training: \url{http://www.seo-usa.org/Career/SEO-U:Freshmen_&_Sophomore_Training}
			\end{enumerate}
		\item Application Deadlines: \url{http://www.seo-usa.org/CareerApplicationDeadlines}
		\item Eligibility Information: \url{http://www.seo-usa.org/CareerEligibilityInfo}
		\item Application Tips: \url{http://www.seo-usa.org/CareerApplicationTips}
		\item Interview Tips: \url{http://www.seo-usa.org/CareerInterviewTips}
		\end{enumerate}
	\end{enumerate}
\item --- --- --- --- --- --- --- --- --- --- --- --- --- --- --- --- --- --- --- --- --- --- --- --- --- --- --- --- --- --- ---
\item \colorbox{blue}{\bf Scholarships for Studying Abroad}
% Scholarships for Studying Abroad
\item U.S. Department of State: \vspace{-0.3cm}
	\begin{enumerate} \itemsep -2pt
	\item Bureau of Educational and Cultural Affairs: \vspace{-0.2cm}
		\begin{enumerate} \itemsep -2pt
		\item Benjamin A. Gilman International Scholarship: \vspace{-0.1cm}
			\begin{enumerate} \itemsep -1pt
			\item \url{http://exchanges.state.gov/globalexchanges/gilman-scholarship-program.html}
			\item ``The Benjamin A. Gilman International Scholarship Program provides scholarships to U.S. undergraduates with financial need for study abroad, including students from diverse backgrounds and students going to non-traditional study abroad destinations.  Established under the International Academic Opportunity Act of 2000, Gilman Scholarships provide up to \$5,000 for American students to pursue overseas study for college credit.''
			\item Critical Need Languages: Students studying critical need languages are eligible for up to \$3,000 in additional funding as part of the Gilman Critical Need Language Supplement program. Those critical need languages include: \vspace{-0.1cm}
				\begin{itemize} \itemsep -1pt
				\item Arabic
				\item Chinese
				\item Korean
				\item Russian
				\item Turkic (Azerbaijani, Kazakh, Kyrgyz, Turkish, Turkmen, Uzbek)
				\item Persian (Farsi, Dari, Kurdish, Pashto, Tajiki)
				\item Indic (Hindi, Urdu, Nepali, Sinhala, Bengali, Punjabi, Marathi, Gujurati, Sindhi)
				\end{itemize}
			\item \url{http://www.iie.org/en/Programs/Gilman-Scholarship-Program}
			\item \url{http://www.iie.org/en/Programs/Gilman-Scholarship-Program/About-the-Program}
			\end{enumerate}
		\end{enumerate}
	\end{enumerate}
\item Council on International Educational Exchange (CIEE): \vspace{-0.3cm}
	\begin{enumerate} \itemsep -2pt
	\item CIEE Scholarships: \url{http://www.ciee.org/study/scholarships/index.aspx}
	\end{enumerate}
\item IES Abroad (formerly Institute of European Studies / Institute for the International Education of Students): \vspace{-0.3cm}
	\begin{enumerate} \itemsep -2pt
	\item Scholarships and Financial Aid: \url{https://www.iesabroad.org/IES/Scholarships_and_Aid/financialAid.html}
	\item IES Abroad Need-Based Financial Aid: \url{https://www.iesabroad.org/IES/Scholarships_and_Aid/Need-Based/needBasedFinancialAid.html}
	\item IES Abroad Merit-Based Scholarships: \url{https://www.iesabroad.org/IES/Scholarships_and_Aid/Merit_Based/meritBasedFinancialAid.html}
	\item IES Abroad Public University Grants: \url{https://www.iesabroad.org/IES/Scholarships_and_Aid/publicScholarship.html}
	\end{enumerate}
\item American Institute For Foreign Study (AIFS): \vspace{-0.3cm}
	\begin{enumerate} \itemsep -2pt
	\item AIFS Study Abroad Programs: \vspace{-0.2cm}
		\begin{enumerate} \itemsep -2pt
		\item \url{http://www.aifsabroad.com/programs.asp}
		\item AIFS Study Abroad Scholarships: \url{http://www.aifsabroad.com/scholarships.asp}
		\end{enumerate}
	\end{enumerate}
\item --- --- --- --- --- --- --- --- --- --- --- --- --- --- --- --- --- --- --- --- --- --- --- --- --- --- --- --- --- --- ---
\item \colorbox{blue}{\bf Scholarships and Fellowships in Public Policy and Public Health}
% Scholarships and Fellowships in Public Policy and Public Health
\item The Commonwealth Fund: \vspace{-0.3cm}
	\begin{enumerate} \itemsep -2pt
	\item Commonwealth Fund fellowship programs: \vspace{-0.2cm}
		\begin{enumerate} \itemsep -2pt
		\item \url{http://www.commonwealthfund.org/Fellowships.aspx}
		\item ``Commonwealth Fund fellowship programs are designed to give promising young researchers the opportunity for in-depth study of various health care policy topics, working with investigators, policy analysts, government officials, and others in a number of U.S. and international settings.''
		\item The Commonwealth Fund/Harvard University Fellowship in Minority Health Policy: \url{http://www.commonwealthfund.org/Fellowships/Minority-Health-Policy-Fellowship.aspx}
		\item Harkness Fellowships in Health Care Policy and Practice: \url{http://www.commonwealthfund.org/Fellowships/Harkness-Fellowships.aspx}
		\item Australian-American Health Policy Fellowship: \url{http://www.commonwealthfund.org/Fellowships/Australian-American-Health-Policy-Fellowships.aspx}
		\item Ian Axford (New Zealand) Fellowships in Public Policy: \url{http://www.commonwealthfund.org/Fellowships/Ian-Axford-Fellowships.aspx}
		\end{enumerate}
	\end{enumerate}
\item American Institute of Aeronautics and Astronautics (AIAA): \vspace{-0.3cm}
	\begin{enumerate} \itemsep -2pt
	\item Federal Government Fellows Program: \vspace{-0.2cm}
		\begin{enumerate} \itemsep -2pt
		\item \url{http://www.aiaa.org/content.cfm?pageid=731}
		\item Shaping U.S. {\bf public policy} concerning aerospace research and the aerospace industry
		\end{enumerate}
	\end{enumerate}
\item IEEE-USA: \vspace{-0.3cm}
	\begin{enumerate} \itemsep -2pt
	\item Congressional Fellowship
	\item Engineering \& Diplomacy (State Department) Fellowship
	\item For IEEE-USA members to support the creation and modification of technology-related public policies
	\item \url{http://ieeeusa.org/policy/govfel/default.asp}
	\end{enumerate}
\item American Mathematical Society: \vspace{-0.3cm}
	\begin{enumerate} \itemsep -2pt
	\item Fellowships and Awards (Policy and Advocacy: Government Relations \& Programs): \vspace{-0.2cm}
		\begin{enumerate} \itemsep -2pt
		\item \url{http://e-math.ams.org/policy/government/fellow-awards/fellow-awards}
		\item Mass Media Fellowships: \url{http://e-math.ams.org/programs/ams-fellowships/media-fellow/massmediafellow}
		\item AMS-AAAS Congressional Fellowship: \url{http://e-math.ams.org/programs/ams-fellowships/ams-aaas/ams-aaas-congressional-fellowship}
		\end{enumerate}
	\end{enumerate}
\item American Association for the Advancement of Science: \vspace{-0.3cm}
	\begin{enumerate} \itemsep -2pt
	\item AAAS Science \& Technology Policy Fellowships: \url{http://fellowships.aaas.org/index.shtml}
	\end{enumerate}
\item --- --- --- --- --- --- --- --- --- --- --- --- --- --- --- --- --- --- --- --- --- --- --- --- --- --- --- --- --- --- ---
\item \colorbox{blue}{\bf Scholarships and Fellowships in Social Science and Humanities}
% Scholarships and Fellowships in Social Science and Humanities
\item United States Institute of Peace (USIP): \vspace{-0.3cm}
	\begin{enumerate} \itemsep -2pt
	\item Jennings Randolph Peace Scholarship Dissertation Program (for Ph.D. students working on topics related to peace, conflict, and international security): \url{http://www.usip.org/grants-fellowships/jennings-randolph-peace-scholarship-dissertation-program}
	\end{enumerate}
\item Library of Congress: \vspace{-0.3cm}
	\begin{enumerate} \itemsep -2pt
	\item Kluge Fellowships: \vspace{-0.2cm}
		\begin{enumerate} \itemsep -2pt
		\item Research in the humanities and social sciences, especially interdisciplinary, cross-cultural or multilingual
		\item Open to scholars worldwide with a Ph.D. or other terminal advanced degree conferred within seven years of the July 15 deadline
		\item \url{http://www.loc.gov/loc/kluge/fellowships/kluge.html}
		\end{enumerate}
	\item J. Franklin Jameson Fellowship Research in American History (junior postdocs): \url{http://www.loc.gov/loc/kluge/fellowships/jameson.html}
	\item Kislak Short Term Fellowship Opportunities in American Studies (students, postdocs, and faculty): \url{http://www.loc.gov/loc/kluge/fellowships/kislakshort.html}
	\item Kislak Fellowship in American Studies (Ph.D. requirement): \url{http://www.loc.gov/loc/kluge/fellowships/kislak.html}
	\end{enumerate}
\item American Historical Association (AHA): \vspace{-0.3cm}
	\begin{enumerate} \itemsep -2pt
	\item AHA Research Grants: \url{http://www.historians.org/prizes/Grants.htm}
	\item Fellowships: \url{http://www.historians.org/prizes/Fellowships.htm}
	\end{enumerate}
\item American Sociological Association: \vspace{-0.3cm}
	\begin{enumerate} \itemsep -2pt
	\item ASA Dissertation Award: \url{http://www.asanet.org/about/awards/dissertation.cfm}
	\end{enumerate}
\item American Psychological Association: \vspace{-0.3cm}
	\begin{enumerate} \itemsep -2pt
	\item Scholarships, Grants, and Awards: \url{http://www.apa.org/about/awards/index.aspx}
	\end{enumerate}
\item American Anthropological Association (AAA): \vspace{-0.3cm}
	\begin{enumerate} \itemsep -2pt
	\item AAA Minority Dissertation Fellowship Program (for minority Ph.D. candidates in anthropology): \url{http://www.aaanet.org/cmtes/minority/Minfellow.cfm}
	\item Margaret Mead Award (for young scholars in anthropology): \url{http://www.aaanet.org/about/Prizes-Awards/AAA-Margaret-Mead-Award.cfm}
	\item COSWA Award: \vspace{-0.2cm}
		\begin{enumerate} \itemsep -2pt
		\item The COSWA Award (formerly the Squeaky Wheel Award), sponsored by the Committee on the Status of Women in Anthropology (COSWA), recognizes individuals who have demonstrated the courage to bring to light and investigate practices in anthropology that are potentially discriminatory to women, or have acted to improve the status of women in anthropology through activities that raise awareness of women's contribution to anthropology or identify barriers to full participation by women in anthropology.
		\item \url{http://www.aaanet.org/about/Prizes-Awards/COSWA-Award.cfm}
		\end{enumerate}
	\item David M. Schneider Award (for Ph.D. students in anthropology): \url{http://www.aaanet.org/about/Prizes-Awards/David-Schneider-Award.cfm}
	\item Links to ``Section Prizes \& Awards'': \url{http://www.aaanet.org/about/Prizes-Awards/section_awards.cfm}
	\item List of national (US) and international ``Grants and Fellowships'': \url{http://www.aaanet.org/profdev/fellowships/}
	\item \url{http://www.aaanet.org/}
	\end{enumerate}
\item National Academy of Social Insurance: \vspace{-0.3cm}
	\begin{enumerate} \itemsep -2pt
	\item John Heinz Dissertation Award (Ph.D. students writing their thesis on the planning and implementation of social insurance): \url{http://www.nasi.org/studentopps/heinz}
	\end{enumerate}
\item National Endowment for the Humanities's Division of Research Programs, grants and fellowship opportunities: \url{http://www.neh.gov/grants/}
\item {\it The Henry Luce Foundation}'s Luce Scholars Program to help US graduates learn more about Asia and Asian culture(s): \url{http://www.hluce.org/lsprogram.aspx}
\item Institute for Humane Studies at George Mason University: \vspace{-0.3cm}
	\begin{enumerate} \itemsep -2pt
	\item Humane Studies Fellowships: \vspace{-0.2cm}
		\begin{enumerate} \itemsep -2pt
		\item \url{http://www.theihs.org/programs/humane-studies-fellowships}
		\item Humane Studies Fellowships are awarded to graduate students and outstanding undergraduates planning academic careers with liberty-advancing research interests.
		\item The fellowships are open to students in a range of fields, such as economics, philosophy, law, political science, anthropology, and literature.
		\end{enumerate}
	\end{enumerate}
\item The Gilder Lehrman Institute of American History: Gilder Lehrman History Scholars \& Gilder Lehrman One-Week Scholars (for sophomores or juniors majoring in American history or American Studies), \url{http://www.gilderlehrman.org/education/hs_program_details.php}
\item Myra Sadker Foundation: \vspace{-0.3cm}
	\begin{enumerate} \itemsep -2pt
	\item \url{http://www.sadker.org/awards.html}
	\item Teacher Award: Designed to promote and support teacher projects (K-12) that help students learn about and respect group differences, promote fairness, and in other ways build upon the values and contributions of Myra Sadker's work. Each project should have a gender dimension.
	\item Student Award: Designed to encourage student ideas, activities and projects (K-12) that promote respect for group differences, fairness, and in other ways build upon the values and contributions of Myra Sadker's work. Each project should have a gender dimension. 
	\item Doctoral Dissertation Award: Designed to promote and support graduate students engaged in educational equity research. Doctoral level dissertations that explore or promote educational equity and fairness based on gender, race, ethnicity, religion, class, sexual orientation, or other such variables will be considered for support. Each dissertation should have a gender dimension.
	\end{enumerate}
\item IREX: \vspace{-0.3cm}
	\begin{enumerate} \itemsep -2pt
	\item Opportunities ``for individuals, organizations, universities, and alumni'': \url{http://www.irex.org/apply}
	\item Edmund S. Muskie Graduate Fellowship Program: \vspace{-0.2cm}
		\begin{enumerate} \itemsep -2pt
		\item : \url{http://www.irex.org/application/edmund-s-muskie-graduate-fellowship-program-application}
		\item ``The Muskie Program is open to graduate students and professionals from Armenia, Azerbaijan, Belarus, Georgia, Kazakhstan, Kyrgyzstan, Moldova, Russia, Tajikistan, Turkmenistan, Ukraine and Uzbekistan for one-year non-degree, one-year degree, or two-year degree study in the United States.''
		\item ``Eligible fields of study for the Muskie Program are: business administration, economics, education, environmental management, international affairs, journalism and mass communication, law, library and information science, public administration, public health, and {\bf public policy}.''
		\end{enumerate}
	\item Legal Education and Development (LEAD) Fellowship: \vspace{-0.2cm}
		\begin{enumerate} \itemsep -2pt
		\item \url{http://www.irex.org/application/legal-education-and-development-lead-fellowship-application}
		\item Legal Education and Development Fellowship Program (LEAD) in Tajikistan
		\item Eligibility: \vspace{-0.1cm}
			\begin{enumerate} \itemsep -1pt
			\item Is a citizen, national, or permanent resident qualified to hold a valid passport issued by Tajikistan;
			\item Is the recipient of an undergraduate degree in law (four- or five-year study) by the time of the application;
			\item Is able to begin the academic exchange program in the United States in the summer of 2011;
			\item Is able to receive and maintain a United States J-1 visa.
			\end{enumerate}
		\end{enumerate}
	\item Community Solutions Program: \vspace{-0.2cm}
		\begin{enumerate} \itemsep -2pt
		\item \url{http://www.irex.org/application/community-solutions-information-applicants}
		\item ``a professional development program for the best and brightest global community leaders working in Transparency \& Accountability, Tolerance/Conflict Resolution, Environmental Issues, and Women's Issues''
		\item ``Competition for the Community Solutions Program is merit-based and open to community leaders, ages 25-38 at the time of application''
		\end{enumerate}
	\item Crimea Undergraduate Exchange Program (Crimea UGRAD) Application: \vspace{-0.2cm}
		\begin{enumerate} \itemsep -2pt
		\item \url{http://www.irex.org/application/crimea-undergraduate-exchange-program-crimea-ugrad-application}
		\item ``The Crimea UGRAD Program is open to undergraduate students from the Autonomous Republic of Crimea for one academic year of non-degree study in a US university or community college.''
		\end{enumerate}
	\end{enumerate}
\item {\it Demos}: \vspace{-0.3cm}
	\begin{enumerate} \itemsep -2pt
	\item The Ed Baker Fellowship in Democratic Values: \vspace{-0.2cm}
		\begin{enumerate} \itemsep -2pt
		\item \url{http://www.demos.org/edbakerfellowship.cfm}
		\item ``Based in our New York offices, Ed Baker Fellows will give voice to strong democratic values within a wide range of potential issues, including voting rights, citizen engagement, immigration policy and civic inclusion, campaign finance reform and money in politics, and media reform, among others.''
		\end{enumerate}
	\item Fellows Program: \vspace{-0.2cm}
		\begin{enumerate} \itemsep -2pt
		\item \url{http://www.demos.org/fellowsapp.cfm}
		\item \url{http://www.demos.org/program.cfm?currentprogramID=5A196E48-3FF4-6C82-50CBCA5825B661BA}
		\item ``The Fellows Program at Demos provides support and community for writers and thinkers with well-defined projects that aim to advance the values at the core of Demos' programs and mission: a robust and inclusive democracy; shared prosperity; strong \& effective public governance; and global interdependence.''
		\end{enumerate}
	\end{enumerate}
\item Research Councils UK (RCUK): \vspace{-0.3cm}
	\begin{enumerate} \itemsep -2pt
	\item Economic and Social Research Council (ESRC): \vspace{-0.2cm}
		\begin{enumerate} \itemsep -2pt
		\item Academic (funding opportunities for students, postdocs, and professors): \url{http://www.esrcsocietytoday.ac.uk/ESRCInfoCentre/index_academic.aspx}
		\item Professorial Fellowships (for leading senior social scientists): \url{http://www.esrcsocietytoday.ac.uk/ESRCInfoCentre/opportunities/professorial/}
		\item Funding opportunities: \vspace{-0.1cm}
			\begin{enumerate} \itemsep -1pt
			\item \url{http://www.esrcsocietytoday.ac.uk/ESRCInfoCentre/index_government.aspx}
			\item \url{http://www.esrcsocietytoday.ac.uk/ESRCInfoCentre/opportunities/}
			\item ESRC Research Funding Guide / ESRC's Funding Rules: \url{http://www.esrcsocietytoday.ac.uk/ESRCInfoCentre/opportunities/research_funding}
			\item Eligibility for Research Council Funding: \url{http://www.esrcsocietytoday.ac.uk/ESRCInfoCentre/opportunities/eligibility}
			\item Current Funding Opportunities: \url{http://www.esrcsocietytoday.ac.uk/ESRCInfoCentre/opportunities/current_funding_opportunities/}
			\item Forthcoming funding opportunities: \url{http://www.esrcsocietytoday.ac.uk/ESRCInfoCentre/opportunities/forthcoming_opportunities/}
			\item Placement Fellows Scheme: \url{http://www.esrcsocietytoday.ac.uk/ESRCInfoCentre/opportunities/placement/}
			\item Professorial Fellowships: \url{http://www.esrcsocietytoday.ac.uk/ESRCInfoCentre/opportunities/professorial/}
			\item Early Career Researchers (including Postdoctoral Fellowships, International Training, and Networking Opportunities): \url{http://www.esrcsocietytoday.ac.uk/ESRCInfoCentre/opportunities/earlycareer/}
			\item Postgraduate and Career Development Opportunities: \url{http://www.esrcsocietytoday.ac.uk/ESRCInfoCentre/opportunities/postgraduate/}
			\item International Funding Opportunities: \url{http://www.esrcsocietytoday.ac.uk/ESRCInfoCentre/opportunities/international/}
			\item Joint Funding Opportunities: \url{http://www.esrcsocietytoday.ac.uk/ESRCInfoCentre/opportunities/jointfunding/}
			\item Annual competitions: \url{http://www.esrcsocietytoday.ac.uk/ESRCInfoCentre/opportunities/annual/index.aspx#3}
			\end{enumerate}
		\end{enumerate}
	\item Arts and Humanities Research Council (AHRC): \vspace{-0.2cm}
		\begin{enumerate} \itemsep -2pt
		\item Funding Opportunities: \vspace{-0.1cm}
			\begin{enumerate} \itemsep -1pt
			\item \url{http://www.ahrc.ac.uk/FundingOpportunities/Pages/default.aspx}
			\item Fellowships: \url{http://www.ahrc.ac.uk/FundingOpportunities/Pages/Fellowships.aspx}
			\item Fellowships - route for early career researchers: \url{http://www.ahrc.ac.uk/FundingOpportunities/Pages/Fellowshipserc.aspx}
			\item Placement Fellowship based in the Department for Culture, Media and Sport (DCMS) - Climate Change: \url{http://www.ahrc.ac.uk/FundingOpportunities/Pages/PlacementFellowshipDCMS-Climatechange.aspx}
			\item Placement Fellowship based in the Department for Culture, Media and Sport (DCMS) - Health and Wellbeing: \url{http://www.ahrc.ac.uk/FundingOpportunities/Pages/PlacementFellowshipDCMShealthandwellbeing.aspx}
			\item Research Grants - route for early career researchers: \url{http://www.ahrc.ac.uk/FundingOpportunities/Pages/RG-EarlyCareers.aspx}
			\item Research Grants - Speculative Research: \url{http://www.ahrc.ac.uk/FundingOpportunities/Pages/RG-SpeculativeResearch.aspx}
			\item Research Grants - Standard Route: \url{http://www.ahrc.ac.uk/FundingOpportunities/Pages/RG-StandardRoute.aspx}
			\item Postgraduate Funding (for Masters and Ph.D. students): \url{http://www.ahrc.ac.uk/FundingOpportunities/Pages/summaryinformationforprospectivepostgraduatestudents.aspx}
			\item Browse Funding Opportunities: \url{http://www.ahrc.ac.uk/FundingOpportunities/Pages/BrowseOpportunities.aspx}
			\end{enumerate}
		\end{enumerate}
	\end{enumerate}
\item World Bank Institute (WBI): \vspace{-0.3cm}
	\begin{enumerate} \itemsep -2pt
	\item Or The World Bank Group
	\item Scholarships: \url{http://wbi.worldbank.org/wbi/scholarships} or \url{http://www.worldbank.org/wbi/scholarships/home.html}
	\end{enumerate}
\item --- --- --- --- --- --- --- --- --- --- --- --- --- --- --- --- --- --- --- --- --- --- --- --- --- --- --- --- --- --- ---
\item \colorbox{blue}{\bf Fellowships in Art and Music}
% Fellowships in Art and Music
\item The Kresge Foundation: \vspace{-0.3cm}
	\begin{enumerate} \itemsep -2pt
	\item \url{http://www.kresge.org/index.php/what/detroit_program/kresge_arts_in_detroit/}
	\item Kresge Artist Fellowships: \vspace{-0.2cm}
		\begin{enumerate} \itemsep -2pt
		\item ``Kresge Artist Fellowships seek to advance the art forms and professional careers of artists from the visual, performing and literary arts as well as elevate the profile of the artistic community and encourage creative expression in the region. Each year, Kresge will provide funding for 18 fellowships of \$25,000 each, which are awarded to artists living and working in metropolitan Detroit.''
		\item ``The fellowships recognize creative vision and commitment to excellence within a wide range of artistic disciplines, including artists who have been classically and academically trained, self taught artists and artists whose art forms have been passed down through cultural and traditional heritage.''
		\item ``Kresge Arts in Detroit is committed to supporting artists from diverse cultural backgrounds at all stages of their professional careers.''
		\item \url{http://kresge.collegeforcreativestudies.edu/}
		\item \url{http://kresge.collegeforcreativestudies.edu/kaf_guidelines.html}
		\item Information Sessions: \url{http://kresge.collegeforcreativestudies.edu/kaf_sessions.html}
		\end{enumerate}
	\item Kresge Eminent Artist Award: \vspace{-0.2cm}
		\begin{enumerate} \itemsep -2pt
		\item ``Kresge Eminent Artist Award recognizes an exceptional artist for his or her professional achievements and contributions to the cultural community, and encourages that individual's pursuit of a chosen art form as well as an ongoing commitment to metropolitan Detroit. Each year, one highly accomplished individual will be presented with the award which includes a \$50,000 prize.''
		\item \url{http://kresge.collegeforcreativestudies.edu/eminent-artist-award.html}
		\end{enumerate}
	\end{enumerate}
\item Guggenheim Fellowships from the {\it John Simon Guggenheim Memorial Foundation}: \url{http://www.gf.org/applicants}
\item The John F. Kennedy Center for the Performing Arts: \vspace{-0.3cm}
	\begin{enumerate} \itemsep -2pt
	\item DeVos Institute of Arts Management at the Kennedy Center: \vspace{-0.2cm}
		\begin{enumerate} \itemsep -2pt
		\item DeVos Institute Programs: \vspace{-0.1cm}
			\begin{enumerate} \itemsep -1pt
			\item Kennedy Center Fellowship Program: \vspace{-0.1cm}
				\begin{itemize} \itemsep -1pt
				\item \url{http://www.kennedy-center.org/education/artsmanagement/fellowships.cfm}
				\item \url{http://www.kennedy-center.org/education/artsmanagement/fellowships/home.html}
				\item ``The Kennedy Center Fellowship Program began in 2001, and provides comprehensive study to 10 arts managers at the Kennedy Center with coursework in strategic planning, marketing, and development; three practical work rotations in Center departments; and a series of professional development seminars. The paid fellowships are full-time and last nine months from September through May.''
				\end{itemize}
			\item DeVos Institute Summer International Fellowship Program at the Kennedy Center: \vspace{-0.1cm}
				\begin{itemize} \itemsep -1pt
				\item \url{http://www.kennedy-center.org/education/artsmanagement/fellowships.cfm}
				\item \url{http://www.kennedy-center.org/education/artsmanagement/international_faq.cfm}
				\item ``The Summer International Fellowship Program provides practical experience to 15 mid-to-high level arts leaders currently working in international nonprofit performing arts organizations. This full-time, four-week intensive program takes place at the Kennedy Center each July; Fellows attend each summer for three consecutive years. While at the Center, the fellows take classes and refine strategic plans for their home organizations.''
				\end{itemize}
			\item U.S. Department of State International Exchange Programs: \vspace{-0.1cm}
				\begin{itemize} \itemsep -1pt
				\item \url{http://www.kennedy-center.org/education/state/}
				\item ``The U.S. Department of State and The Kennedy Center have teamed to produce international exchange opportunities through the Performing Artists Cultural Visitors Program and International Cultural Fellows Mentoring Program.''
				\item Performing Artists Cultural Visitors Program: \url{http://www.kennedy-center.org/education/state/cultural/}
				\item International Cultural Fellows Mentoring Program: \url{http://www.kennedy-center.org/education/state/fellows/}
				\item ``Visitors, comprised of modern and hip-hop dancers, theater technicians/designers/actors, as well as classical and jazz musicians, engage with American colleagues in the creation and performance of their discipline in Washington, D.C. and in another American city.''
				\item ``The Fellows, comprised of arts managers and presenters from outside the United States, attend arts management seminars led by Kennedy Center staff, travel to another American city to study with a mentor organization, and visit New York City to meet with experts in their field.''
				\end{itemize}
			\end{enumerate}
		\end{enumerate}
	\item The National Symphony Orchestra (NSO): \vspace{-0.2cm}
		\begin{enumerate} \itemsep -2pt
		\item National Symphony Orchestra Youth Fellowship Program: \vspace{-0.1cm}
			\begin{itemize} \itemsep -1pt
			\item \url{http://www.kennedy-center.org/nso/nsoed/youthfellowship.cfm}
			\item \url{http://www.kennedy-center.org/explorer/artists/?entity_id=10811&source_type=B}
			\item ``Now in its 30th season, the National Symphony Orchestra Youth Fellowship Program is an orchestral training project for high school musicians.''
			\item ``From its inception in 1980-81 to the present, the program provides Washington metropolitan area high school students with scholarships to study privately with NSO members, as well as opportunities to observe NSO rehearsals; attend concerts; and to participate in seminars, discussions, and master classes with musicians, conductors, and NSO and Kennedy Center management.''
			\item ``There are 20 students in the NSO Youth Fellowship Program for 2009-10.''
			\item ``Participation by ethnic minorities is encouraged.''
			\item ``Priority is given to students entering 10th grade in order to provide as sustained a training as possible.''
			\end{itemize}
		\end{enumerate}
	\end{enumerate}
\item League of American Orchestras: \vspace{-0.3cm}
	\begin{enumerate} \itemsep -2pt
	\item Fellowships: \vspace{-0.2cm}
		\begin{enumerate} \itemsep -2pt
		\item \url{http://www.americanorchestras.org/learning_and_leadership/fellowships.html}
		\item Orchestra Management Fellowship Program: \vspace{-0.1cm}
			\begin{enumerate} \itemsep -1pt
			\item \url{http://www.americanorchestras.org/learning_and_leadership/omfp.html}
			\item ``This year-long, highly competitive program is designed to launch executive careers in orchestra management.''
			\item ``Along with an intense course of study, fellows undertake a series of residencies with orchestras of various sizes across the U.S. receiving invaluable work experience and the support of host orchestra staff, in particular that of the orchestra�s executive director.''
			\item ``Fellows also participate in other League leadership seminars throughout the year and receive a comprehensive overview of the classical music industry.''
			\end{enumerate}
		\item ``The League's Fellowship programs identify and prepare the future leaders of tomorrow, today.''
		\item ``Long-term curricula, developed for conductors, executive directors, and managers looking to advance, provide intensive education, hands-on learning, and valuable networking opportunities.''
		\end{enumerate}
	\end{enumerate}
\item Americans for the Arts: \vspace{-0.3cm}
	\begin{enumerate} \itemsep -2pt
	\item Event scholarships (scholarships to attend events): \url{http://www.artsusa.org/events/scholarships.asp}
	\item \url{http://www.artsusa.org/news/annual_awards/default.asp}
	\item Alene Valkanas State Arts Advocacy Award\url{http://www.artsusa.org/news/annual_awards/alene_valkanas/default.asp}
	\item Arts Education Award (awarded to institutions): \url{http://www.artsusa.org/news/annual_awards/arts_education/default.asp}
	\item Emerging Leader Award: \url{http://www.artsusa.org/news/annual_awards/emerging_leader/default.asp}
	\item Michael Newton Award for United Arts Funds Leadership (management and fundraising): \url{http://www.artsusa.org/news/annual_awards/michael_newton/default.asp}
	\item Selina Roberts Ottum Award (contributions to the field of the arts): \url{http://www.artsusa.org/news/annual_awards/selina_roberts_ottum/default.asp}
	\item United States Urban Arts Federation (USUAF): \vspace{-0.2cm}
		\begin{enumerate} \itemsep -2pt
		\item Ray Hanley Innovation Award: \url{http://www.artsusa.org/networks/usuaf/hanley.asp}
		\end{enumerate}
	\end{enumerate}
\item NEA National Heritage Fellowship (for master folk and traditional artists): \url{http://www.nea.gov/honors/heritage/index.html}
\item NEA Jazz Masters Fellowship (jazz artists): \url{http://www.arts.gov/honors/jazz/index.html}
\item Fellowships for Creative Writers [or NEA Literature Fellowships: Creative Writing]: \url{http://www.nea.gov/grants/apply/Lit/index.html} or \url{http://www.arts.gov/grants/apply/Lit/index.html}
\item Carnegie Investment Bank: Carnegie Art Award (for distinguished artists born or living in the Nordic countries), \url{http://www.carnegie.se/sv/ArtAward/About-Carnegie-Art-Award/}, \url{http://www.carnegie.se/artaward/}, and \url{http://www.carnegie.se/en/about/Operations/Carnegie-Art-Award/}
\item Robert McCann Foundation: \vspace{-0.3cm}
	\begin{enumerate} \itemsep -2pt
	\item Funding for artists and designers ``from all Scottish colleges and art schools'' to: \vspace{-0.2cm}
		\begin{enumerate} \itemsep -2pt
		\item extend their training in an area of specialization; OR
		\item finance a project ``in the craft industries associated with film and television''
		\end{enumerate}
	\item \url{http://robertmccannfoundation.com/how.html}
	\end{enumerate}
\item Alexander von Humboldt-Stiftung/Foundation: \vspace{-0.3cm}
	\begin{enumerate} \itemsep -2pt
	\item Hezekiah Wardwell Fellowship (for musicians or musicologists from Spain): \url{http://www.humboldt-foundation.de/web/wardwell-en.html}
	\end{enumerate}
\item Canada Council for the Arts: \vspace{-0.3cm}
	\begin{enumerate} \itemsep -2pt
	\item Endowments and Prizes: \vspace{-0.2cm}
		\begin{enumerate} \itemsep -2pt
		\item \url{http://www.canadacouncil.ca/prizes/}
		\item Prizes and fellowships for Canadian artists and scholars to recognize their contributions to the arts, humanities, and sciences
		\item Categories of prizes and fellowships: \vspace{-0.1cm}
			\begin{enumerate} \itemsep -1pt
			\item dance
			\item inter-arts
			\item media arts
			\item music
			\item theatre
			\item visual arts
			\item writing and publishing
			\end{enumerate}
		\end{enumerate}
	\item Grant Programs: \url{http://www.canadacouncil.ca/grants/}
	\end{enumerate}
\item Institute for Humane Studies at George Mason University: \vspace{-0.3cm}
	\begin{enumerate} \itemsep -2pt
	\item Film \& Fiction Scholarships: \vspace{-0.2cm}
		\begin{enumerate} \itemsep -2pt
		\item Students pursuing MFAs in a variety of areas are eligible: film directing, production, screenwriting, playwriting, fiction, and literary-nonfiction writing
		\item \url{http://www.theihs.org/node/448}
		\end{enumerate}
	\end{enumerate}
\item --- --- --- --- --- --- --- --- --- --- --- --- --- --- --- --- --- --- --- --- --- --- --- --- --- --- --- --- --- --- ---
\item \colorbox{blue}{\bf Scholarships and Fellowships for Underrepresented Minorities}
% Scholarships and Fellowships for Underrepresented Minorities
\item Lists of scholarships and fellowships for underrepresented minorities: \vspace{-0.3cm}
	\begin{enumerate} \itemsep -2pt
	\item Chris Enstrom, ``Cashing in on Diversity Grants and Scholarships,'' in Graduating Engineer \& Computer Careers. Available at: \url{http://www.graduatingengineer.com/higher-education/20061129/Cashing-in-on-Diversity-Grants-and-Scholarships-}; last accessed on August 25, 2010.
	\end{enumerate}
\item Gates Millennium Scholars (GMS) scholarship (for underrepresented minorities in the US): \url{http://www.gmsp.org/}
\item Society of Women Engineers (SWE): SWE Scholarships and other scholarships, \url{http://societyofwomenengineers.swe.org/index.php?option=com_content&task=view&id=222&Itemid=111}
\item Coalition to Diversify Computing: \url{http://www.cdc-computing.org/scholarships/}
\item IES Abroad (formerly Institute of European Studies / Institute for the International Education of Students): \vspace{-0.3cm}
	\begin{enumerate} \itemsep -2pt
	\item Diversity Abroad: \vspace{-0.2cm}
		\begin{enumerate} \itemsep -2pt
		\item \url{https://www.iesabroad.org/IES/Diversity/diversity.html}
		\item Programs to improve student diversity in study abroad programs
		\item IES Abroad Diversity Scholarships: \vspace{-0.1cm}
			\begin{enumerate} \itemsep -1pt
			\item IES Abroad Merit-Based Scholarship for Under-represented Students: \url{https://www.iesabroad.org/IES/Scholarships_and_Aid/Diversity_Scholarships/diversityScholarship.html}
			\item IES Abroad Merit-Based David Porter Diversity Scholarship (Up to \$5,000!): \url{https://www.iesabroad.org/IES/Scholarships_and_Aid/Merit_Based/davidPorterScholarship.html}
			\item HBCU Scholarships: \url{https://www.iesabroad.org/IES/Scholarships_and_Aid/Diversity_Scholarships/hbcuScholarship.html}
			\item HACU-IES Abroad Merit/Need-Based Scholarship: \url{https://www.iesabroad.org/IES/Scholarships_and_Aid/Diversity_Scholarships/HACUScholarship.html}
			\end{enumerate}
		\end{enumerate}
	\end{enumerate}
\item MassMutual Scholars Program: \vspace{-0.3cm}
	\begin{enumerate} \itemsep -2pt
	\item Applicants must be undergraduates of African American/Black, Asian/Pacific Islander or Hispanic decent in the US.
	\item Reside or plan to attend an institution in one of the following metropolitan areas: \vspace{-0.2cm}
		\begin{enumerate} \itemsep -2pt
		\item Atlanta, GA
		\item Chicago, IL
		\item Central New Jersey
		\item Denver, CO
		\item Houston, TX
		\item Miami, FL
		\item Los Angeles, CA
		\item San Antonio, TX
		\item San Francisco, CA
		\end{enumerate}
	\item Be majoring in business, economics, finance, financial planning, management, marketing or sales.
	\item \url{http://www.hsf.net/massmutual.aspx}
	\item \url{http://www.apiasf.org/scholarship_apiasf_massmutual.html}
	\end{enumerate}
\item {\it NASA}'s Minority University Research and Education Program (MUREP): \vspace{-0.3cm}
	\begin{enumerate} \itemsep -2pt
	\item \url{http://www.nasa.gov/offices/education/programs/national/murep/home/index.html}
	\item \url{http://www.nasa.gov/offices/education/about/murep_overview.html}
	\item Jenkins Pre-doctoral Fellowship Project, JPFP: \url{http://www.nasa.gov/offices/education/programs/descriptions/Jenkins_Predoctoral_Fellowship_Project.html}
	\end{enumerate}
\item UNCF: \vspace{-0.3cm}
	\begin{enumerate} \itemsep -2pt
	\item UNCF Special Programs Corporation: \vspace{-0.2cm}
		\begin{enumerate} \itemsep -2pt
		\item Harriett G. Jenkins Pre-doctoral Fellowship Program (JPFP) for underrepresented minorities pursuing graduate degrees in STEM: \url{http://www.uncfsp.org/spknowledge/default.aspx?page=program.view&areaid=1&contentid=177&typeid=jpfp}
		\item NASA Science and Technology Institute (NSTI) Summer Scholars Program (10-week summer research scholarship): \url{http://www.uncfsp.org/spknowledge/default.aspx?page=program.view&areaid=1&contentid=172&typeid=nstiinternship}
		\item Motivating Undergraduates in Science and Technology (MUST) Program for undergraduates in STEM: \url{http://www.uncfsp.org/spknowledge/default.aspx?page=program.view&areaid=1&contentid=346&typeid=must}
		\item Institute for International {\bf Public Policy} Fellows Program: \url{http://www.uncfsp.org/IIPP}
		\item \url{http://www.uncfsp.org/spknowledge/default.aspx?page=home.default}
		\end{enumerate}
	\item UNCF scholarship resources: \url{http://www.uncf.org/forstudents/scholarship.asp}
	\item UNCF $\cdot$ Merck Science Initiative: scholarships and fellowships: \url{http://umsi.uncf.org/ScholarshipsInternshipsFellowships/tabid/151/Default.aspx}
	\end{enumerate}
\item Hispanic College Fund: \vspace{-0.3cm}
	\begin{enumerate} \itemsep -2pt
	\item Scholarships: \url{http://www.hispanicfund.org/scholarships/} and \url{http://scholarships.hispanicfund.org/applications/}
	\item NASA MUST Scholarship Program: \url{http://www.hispanicfund.org/nasa-must/}
	\item Hispanic Youth Symposium (scholarships are awarded to winners of the art competition, talent competition, and speech competition): \url{http://www.hispanicyouth.org/about-the-program}
	\item \url{http://www.hispanicfund.org/}
	\end{enumerate}
\item Hispanic Heritage Foundation (HHF): \vspace{-0.3cm}
	\begin{enumerate} \itemsep -2pt
	\item Scholarships and Resources: \url{http://www.hispanicheritage.org/youth_int.php?sec=80}
	\item \url{http://www.hispanicheritage.org/}
	\end{enumerate}
\item Hispanic Scholarship Fund (HSF): \vspace{-0.3cm}
	\begin{enumerate} \itemsep -2pt
	\item Scholarship programs for: \vspace{-0.2cm}
		\begin{enumerate} \itemsep -2pt
		\item college students
		\item community college transfer students
		\item high school students
		\item Gates Millennium Scholars
		\item See \url{http://www.hsf.net/innercontent.aspx?id=34}
		\end{enumerate}
	\item \url{http://www.hsf.net/}
	\end{enumerate}
\item League of United Latin American Citizens (LULAC): \vspace{-0.3cm}
	\begin{enumerate} \itemsep -2pt
	\item LULAC National Educational Service Centers, Inc: \vspace{-0.2cm}
		\begin{enumerate} \itemsep -2pt
		\item \url{http://www.lnesc.org/}
		\item LULAC National Scholarship Fund (LNSF): \vspace{-0.1cm}
			\begin{enumerate} \itemsep -1pt
			\item \url{http://www.lulac.org/programs/education/scholarships/}
			\item \url{http://lnesc.org/index.asp?Type=B_BASIC&SEC={3AEDB506-F425-4E58-B9F6-44867E2FD943}}
%http://lnesc.org/index.asp?Type=B_BASIC&SEC={3AEDB506-F425-4E58-B9F6-44867E2FD943}
			\item Applicants must meet the following criteria to be considered for a scholarship: \vspace{-0.1cm}
				\begin{itemize} \itemsep -1pt
				\item Must be a U.S. citizen or legal resident
				\item Must have applied to or be enrolled in a   college, university, or graduate school, including 2-year colleges, or vocational schools that lead to an associate�s degree
				\item A student will not be eligible for a scholarship if he/she is related to a scholarship committee member, the Council President, or an individual contributor to the local funds of the Council
				\end{itemize}
			\item National Scholastic Achievement Awards (for high school seniors entering college, university, or vocational school)
			\item Honors Awards (for high school seniors entering college, university, or vocational school)
			\item General Awards (Need, community involvement, and leadership activities will also be considered)
			\item General Electric Foundation/ LULAC Scholarship program: for underrepresented minorities (US freshmen) entering their sophomore year as majors in Business or Engineering with a cumulative college G.P.A. $\leq$ 3.25/4.0; these students must be enrolled in a 4-year undergraduate program.
			\end{enumerate}
		\end{enumerate}
	\end{enumerate}
\item Hispanic Association of Colleges and Universities (HACU): \vspace{-0.3cm}
	\begin{enumerate} \itemsep -2pt
	\item HACU Student Programs Overview: \vspace{-0.2cm}
		\begin{enumerate} \itemsep -2pt
		\item \url{http://www.hacu.net/hacu/HACU_Student_Programs_EN.asp?SnID=1942709283}
		\item HACU Scholarship Programs: \vspace{-0.1cm}
			\begin{enumerate} \itemsep -1pt
			\item \url{http://www.hacu.net/hacu/Scholarships_EN.asp?SnID=1942709283}
			\item Includes scholarships for students in: \vspace{-0.1cm}
				\begin{itemize} \itemsep -1pt
				\item Accounting
				\item Behavioral Health
				\item Business
				\item Clinical Psychology
				\item Computer Engineering
				\item Computer Science
				\item Dental Technician
				\item Electrical Engineering
				\item Engineering
				\item Food Merchandising
				\item Information Technology
				\item International Business
				\item Management
				\item Marketing
				\item Mass Media
				\item Mental Health
				\item Merchandising
				\item Nursing
				\item Physician Assistant
				\item (Pre) Optometry
				\item (Pre) Dental
				\item (Pre) Medicine
				\item (Pre) Pharmacy
				\item Public Health
				\item Public Relations
				\item Retail Management
				\item Sports Marketing
				\item Technology
				\end{itemize}
			\end{enumerate}
		\item ``D{\'{a}}ndole Alas a Tu {\'{E}}xito/Giving Flight to Your Success'' travel award program (Southwest Airlines' Travel Award Program): \vspace{-0.1cm}
			\begin{enumerate} \itemsep -1pt
			\item For students with financial need who have to across the United States to participate in their undergraduate or graduate degree programs
			\item \url{http://www.hacu.net/hacu/Lanzate_EN.asp?SnID=1942709283}
			\item \url{http://www.hacu.net/hacu/Lanzate1_EN.asp?SnID=1808826658}
			\end{enumerate}
		\item HACU Study Abroad Scholarship Programs: \vspace{-0.1cm}
			\begin{enumerate} \itemsep -1pt
			\item \url{http://www.hacu.net/hacu/Study_Abroad_EN.asp?SnID=1808826658}
			\item HACU-Global Learning Semesters (GLS) Program: Hispanic Study Abroad Scholars: \url{http://www.studyabroadscholars.org/index.html}
			\item HACU-American Institute for Foreign Study (AIFS) Scholarship Program: \url{http://www.aifsabroad.com/scholarships.asp#hacu}
			\item HACU-Institute for the International Education of Students (IES) Scholarship Program: \url{https://www.iesabroad.org/IES/home.html}
			\item Hispanic Study Abroad Scholars program: \url{http://www.studyabroadscholars.org/index.html}
			\end{enumerate}
		\item Scholarship Resource List: \url{http://www.hacu.net/hacu/Scholarship_Resource_List_EN.asp?SnID=1109551622}
%		\item Scholarship Resource List: \url{http://www.hacu.net/hacu/Scholarship_Resource_List_EN.asp?SnID=1942709283}		-- Redundant
		\end{enumerate}
	\end{enumerate}
\item Congressional Hispanic Caucus Institute (CHCI): \vspace{-0.3cm}
	\begin{enumerate} \itemsep -2pt
	\item CHCI Scholarship: \vspace{-0.2cm}
		\begin{enumerate} \itemsep -2pt
		\item \url{http://www.chci.org/scholarships/}
		\item CHCI's scholarship opportunities are afforded to Latino students in the United States who have a history of performing public service-oriented activities in their communities and who demonstrate a desire to continue their civic engagement in the future. There is no GPA or academic major requirement. Students with excellent leadership potential are encouraged to apply.
		\item Scholarship awards are intended to provide assistance with tuition, room and board, textbooks, and other educational expenses associated with college enrollment.
		\item Students continue to receive annual disbursements as long as they maintain good academic standing.
		\item CHCI scholarships provide recipients with a one time scholarship of: \vspace{-0.1cm}
			\begin{enumerate} \itemsep -1pt
			\item \$1,000 community college or AA/AS granting institution
			\item \$2,500 4-year academic institution
			\item \$5,000 graduate-level institution
			\end{enumerate}
		\item Eligibility Criteria: \vspace{-0.1cm}
			\begin{enumerate} \itemsep -1pt
			\item Full-time enrollment in a United States Department of Education accredited community college, four-year university, or graduate/professional program during the period for which scholarship is requested
			\item Demonstrated financial need
			\item Consistent, active participation in public and/or community service activities
			\item Strong writing skills
			\item U.S. citizenship or legal permanent residency
			\end{enumerate}
		\end{enumerate}
	\item CHCI Fellowships: \vspace{-0.2cm}
		\begin{enumerate} \itemsep -2pt
		\item \url{http://www.chci.org/fellowships/}
		\item CHCI {\bf Public Policy} Fellowship: \vspace{-0.1cm}
			\begin{enumerate} \itemsep -1pt
			\item This is a paid Fellowship Program that offers talented Latinos, who have earned a bachelor's degree within two years of the program start date, the opportunity to gain hands-on experience at the national level in public policy.
			\item Fellows have the opportunity to work in congressional offices and federal agencies, depending on their area of interest.  Some past focus areas have included international affairs, economic development, health and education policy, housing, or local government.
			\item Program Dates: August to May (10-month internship)
			\item \url{http://www.chci.org/fellowships/page/chci-public-policy-fellowship}
			\end{enumerate}
		\item CHCI Graduate Fellowship Program: \vspace{-0.1cm}
			\begin{enumerate} \itemsep -1pt
			\item The CHCI Graduate Fellowship Program seeks to enhance participants' leadership abilities, strengthen professional skills and ultimately produce more competent and competitive Latino professionals in underserved {\bf public policy} issue areas.
			\item This paid Fellowship Program offers exceptional Latinos who have earned a graduate degree or higher related to a chosen policy issue area within three years of program start date unparalleled exposure to hands-on experience in public policy.
			\item This program focuses specifically on the areas of: \vspace{-0.1cm}
				\begin{itemize} \itemsep -1pt
				\item Higher Education: CHCI Graduate Higher Education Fellowship, \url{http://www.chci.org/fellowships/page/chci-graduate-higher-education-fellowship}
				\item Secondary Education: CHCI Graduate Secondary Education Fellowship, \url{http://www.chci.org/fellowships/page/chci-graduate-secondary-education-fellowship}
				\item Health: CHCI Graduate Health Fellowship, \url{http://www.chci.org/fellowships/page/chci-graduate-health-fellowship}
				\item Housing: CHCI Graduate Housing Fellowship, \url{http://www.chci.org/fellowships/page/chci-graduate-housing-fellowship}
				\item International Affairs (includes last three months abroad in Mexico): CHCI Graduate International Affairs Fellowship, \url{http://www.chci.org/fellowships/page/chci-graduate-international-affairs-fellowship}
				\item Law: CHCI Graduate Law Fellowship, \url{http://www.chci.org/fellowships/page/chci-graduate-law-fellowship}
				\item STEM (Science, Technology, Engineering and Math): CHCI Graduate STEM Fellowship, \url{http://www.chci.org/fellowships/page/chci-graduate-stem-fellowship}
				\end{itemize}
			\item Program Dates: August to May (10-month internship)
			\item \url{http://www.chci.org/fellowships/page/chci-graduate-fellowship-program}
			\end{enumerate}
		\end{enumerate}
	\end{enumerate}
\item American Indian Graduate Center (AIGC): \vspace{-0.3cm}
	\begin{enumerate} \itemsep -2pt
	\item AIGC scholarships and fellowships: \vspace{-0.2cm}
		\begin{enumerate} \itemsep -2pt
		\item for advanced degree students in art, music, environmental studies, journalism, communications, medicine, dentistry, public health, nursing, or other health-related fields
		\item for members of Wisconsin, New Mexico or Arizona tribes.
		\item \url{http://www.aigc.com/02scholarships/scholarships.htm}
		\item AIGC Fellowship (Graduate) for Native Americans and their descendants seeking advanced degrees: \url{http://www.aigc.com/02scholarships/aigc/fellowship.htm}
		\item Rainer Scholarship (for grad students): \url{http://www.aigc.com/02scholarships/rainer.htm}
		\end{enumerate}
	\item List of resources about scholarships and fellowships: \vspace{-0.2cm}
		\begin{enumerate} \itemsep -2pt
		\item \url{http://www.aigc.com/08otherscholarship/otherscholarships.html}
		\item Scholarships: \url{http://www.aigc.com/08otherscholarship/scholarships.htm}
		\item Fellowships: \url{http://www.aigc.com/08otherscholarship/fellowships.htm}
		\end{enumerate}
	\item Gates Millennium Scholar Program (for individuals seeking basic and advanced degrees): \url{http://www.aigc.com/03gms/gms.htm}
	\end{enumerate}
\item Asian \& Pacific Islander American Scholarship Fund (APIASF) scholarship resources: \url{http://www.apiasf.org/scholarships.html}
\item American Association of University Women: \vspace{-0.3cm}
	\begin{enumerate} \itemsep -2pt
	\item \url{http://www.aauw.org/learn/fellowships_grants/index.cfm}
	\end{enumerate}
\item Sigma Delta Epsilon-Graduate Women in Science (GWIS): \url{http://www.gwis.org/programs.html}
\item Society of Hispanic Professional Engineers (SHPE): \vspace{-0.3cm}
	\begin{enumerate} \itemsep -2pt
	\item Advancing Hispanic Excellence in Technology, Engineering, Math and Science (AHETEMS) Foundation: \url{http://www.ahetems.org/}
	\item AHETEMS Scholarship Program: \url{http://www.ahetems.org/scholarships/}
	\item Graduate \& Young Professional Fellowship Program (encourage young professionals to engage in {\bf public policy}): \url{http://www.ahetems.org/graduate/graduate-young-professional-fellowship-program/}
	\item SHPE/GEM Fellowship (for graduate students in STEM at GEM Member Universities): \url{http://www.ahetems.org/graduate/shpe-gem-graduate-award/}. See \url{http://www.gemfellowship.org/gem-universities/university-members} for a list of GEM member universities.
	\end{enumerate}
\item National Society of Black Engineers (NSBE): \vspace{-0.3cm}
	\begin{enumerate} \itemsep -2pt
	\item Scholarships: \url{http://www.nsbe.org/Programs/Scholarships.aspx}
	\end{enumerate}
\item The Society of Mexican American Engineers and Scientists (MAES): \vspace{-0.3cm}
	\begin{enumerate} \itemsep -2pt
	\item Scholarships \& Awards: \url{http://www.maes-natl.org/index.php?meid=328}
	\item MAES Scholarship Program: \url{http://www.maes-natl.org/index.php?module=ContentExpress&func=display&ceid=518&meid=241}
	\end{enumerate}
\item SACNAS (Society for Advancement of Chicanos and Native Americans in Science): \vspace{-0.3cm}
	\begin{enumerate} \itemsep -2pt
	\item Scholarships: \url{http://www.sacnas.org/webadindex.cfm?webadcategory_id=7}
	\item Fellowships: \url{http://www.sacnas.org/webadIndex.cfm?webadcategory_id=5}
	\end{enumerate}
\item {\it Center for the Advancement of Hispanics in Science and Engineering Education} (CAHSEE): \vspace{-0.3cm}
	\begin{enumerate} \itemsep -2pt
	\item Scholarships: \url{http://www.cahsee.org/6resources/scholarships.asp.htm}
	\end{enumerate}
\item National Consortium for Graduate Degrees for Minorities in Engineering and Science, Inc.: \vspace{-0.3cm}
	\begin{enumerate} \itemsep -2pt
	\item National GEM Consortium: GEM Fellowship, \url{http://www.gemfellowship.org/gem-fellowship/application-requirements}
	\end{enumerate}
\item National Physical Science Consortium (NPSC): \vspace{-0.3cm}
	\begin{enumerate} \itemsep -2pt
	\item NPSC Graduate Fellowship: \url{http://www.npsc.org/}
	\end{enumerate}
\item Finch College Alumnae Association: \vspace{-0.3cm}
	\begin{enumerate} \itemsep -2pt
	\item The Finch College Alumnae Foundation Education Grant: \vspace{-0.2cm}
		\begin{enumerate} \itemsep -2pt
		\item \url{http://www.finchcollege.org/newscholarships.html}
		\item \url{http://www.finchcollege.org/newFinchGrantQandA.html}
		\item ``THE FINCH GRANT, an annual program where four community college women entering a four year college are awarded a grant of \$1500 which can be used toward any needs to completing college.  The selection is determined by a panel of college professors.''
		\end{enumerate}
	\end{enumerate}
\item : \url{}
\item : \url{}
\item : \url{}
\item : \url{}
\item : \url{}
\item \S\ref{phdandpostdocfellowships} has more information concerning scholarships and fellowships in the following areas: \vspace{-0.3cm}
	\begin{enumerate} \itemsep -2pt
	\item electronic design automation (EDA), and related areas of design automation: \vspace{-0.2cm}
		\begin{enumerate} \itemsep -2pt
		\item bio design automation (BDA)
		\item Lab-on-chip design (LoC) automation
		\item MEMS/NEMS design automation
		\end{enumerate}
	\item digital VLSI design
	\item analog and mixed-signal (AMS) VLSI design
	\item computer architecture
	\item parallel computing
	\item concurrent programming
	\item data mining
	\item theoretical computer science
	\end{enumerate}
\item Ph.D. dissertation awards: \vspace{-0.3cm}
	\begin{enumerate} \itemsep -2pt
	\item --- --- --- --- --- --- --- --- --- --- --- --- --- --- --- --- --- --- --- --- --- --- --- --- --- --- --- --- --- --- ---
	\item \colorbox{blue}{\bf Ph.D. Dissertation Awards for Computer Science}
	% Ph.D. Dissertation Awards for Computer Science
	\item ACM Doctoral Dissertation Award: \url{http://awards.acm.org/doctoral_dissertation/}
	\item ACM Outstanding Ph.D. Dissertation Award in Electronic Design Automation: \url{http://www.sigda.org/opda.html}
	\item EDAA Outstanding Dissertation Award (European Design and Automation Association, EDAA): \url{http://www.edaa.com/dissertation_award.html} and \url{http://www.esat.kuleuven.be/micas/EDAA-Award/index.php}
	\item EuroSys Roger Needham PhD Award (in the systems area): \vspace{-0.2cm}
		\begin{enumerate} \itemsep -2pt
		\item Areas in systems include: \vspace{-0.1cm}
			\begin{enumerate} \itemsep -1pt
			\item operating systems
			\item distributed systems
			\item real-time systems
			\item systems aspects of databases
			\item language runtimes
			\item \colorbox{yellow}{\bf embedded systems}
			\item computer networks
			\end{enumerate}
		\item \url{http://www.eurosys.org/phdprize/index.php}
		\end{enumerate}
	\item ACM SIGPLAN Outstanding Doctoral Dissertation Award: \url{http://www.sigplan.org/award-dissertation.htm}
	\item ACM SIGKDD Doctoral Disseration Award (in data mining and knowledge discovery): \url{http://www.sigkdd.org/awards_dissertation.php}
	\item ACM SIGMOD Jim Gray Doctoral Dissertation Award (in the database field): \url{http://www.sigmod.org/sigmod-awards/doctoral-dissertation-award}
	\item Special Interest Group of the ACM on Management Information Systems (SIGMIS): \vspace{-0.2cm}
		\begin{enumerate} \itemsep -2pt
		\item ACM SIGMIS Doctoral Dissertation Award Competition (at the International Conference on Information Systems, ICIS): \url{http://ai.arizona.edu/icis2009/program/dissertation.html} and \url{http://icis2010.aisnet.org/dissertation_award.htm}
		\end{enumerate}
	\item Association for Symbolic Logic: \vspace{-0.2cm}
		\begin{enumerate} \itemsep -2pt
		\item ``The Sacks Prize is awarded for the most outstanding doctoral dissertation in mathematical logic''.
		\item \url{http://www.aslonline.org/Sacks_nominations.html} and \url{http://www.aslonline.org/info-prizes.html}
		\end{enumerate}
	\item European Association for Computer Science Logic (EACSL): \vspace{-0.2cm}
		\begin{enumerate} \itemsep -2pt
		\item Ackermann Award (for outstanding dissertations in Logic in Computer Science): \url{http://www.eacsl.org/} and \url{http://www.eacsl.org/award.html}
		\end{enumerate}
	\item European Coordinating Committee for Artificial Intelligence (ECCAI): \vspace{-0.2cm}
		\begin{enumerate} \itemsep -2pt
		\item 201X Artificial Intelligence Dissertation Award: \url{http://www.eccai.org/diss-award/current.shtml}
		\end{enumerate}
	\item European Conference on Wireless Sensor Networks (EWSN 201X, \url{http://www.nes.uni-due.de/ewsn2011}) and CONET, the Cooperating Objects Network of Excellence: Ph.D. Thesis Award Competition, \url{http://www.cooperating-objects.eu/}
	\item --- --- --- --- --- --- --- --- --- --- --- --- --- --- --- --- --- --- --- --- --- --- --- --- --- --- --- --- --- --- ---
	\item \colorbox{blue}{\bf Ph.D. Dissertation Awards for Mathematics}
	% Ph.D. Dissertation Awards for Mathematics
	\item International Center for Scientific Research (CIRS): \vspace{-0.2cm}
		\begin{enumerate} \itemsep -2pt
		\item E. W. Beth Dissertation Prize (for outstanding dissertations in the fields of Logic, Language and Information): \url{http://www.cirs.net/prix/awards.php?id=481}
		\end{enumerate}
	\item The Association for Operations Management, APICS (Advancing Productivity, Innovation, and Competitive Success): \vspace{-0.2cm}
		\begin{enumerate} \itemsep -2pt
		\item Plossl Doctoral Dissertation Competition: The APICS Educational and Research Foundation, will annually grant one award of \$2,500 for a doctoral dissertation dealing with any topic in operations management. Sample topics include operations strategy, operations planning and control systems, supply chain management, quality management, Six Sigma, facility location, forecasting, just-in-time/lean production systems, and project management. Entrants must be candidates for the doctorate in operations management. The dissertation must be approved by the primary thesis advisor and not more than 50\% completed at time of submission. See \url{http://www.apics.org/Education/ERFoundation/Competitions/plossl.htm}.
		\end{enumerate}
	\item SIAM Richard C. DiPrima Prize: \vspace{-0.2cm}
		\begin{enumerate} \itemsep -2pt
		\item The Richard C. DiPrima Prize is awarded every two years to a junior scientist, based on an outstanding doctoral dissertation in applied mathematics.
		\item \url{http://www.siam.org/prizes/nominations/nom_diprima.php}
		\item \url{http://www.siam.org/prizes/sponsored/diprima.php}
		\end{enumerate}
	\item MOS A.W. Tucker Prize: \vspace{-0.2cm}
		\begin{enumerate} \itemsep -2pt
		\item It is awarded for an outstanding doctoral thesis in any aspect of mathematical optimization.
		\item \url{http://www.mathprog.org/?nav=tucker}
		\end{enumerate}
	\item --- --- --- --- --- --- --- --- --- --- --- --- --- --- --- --- --- --- --- --- --- --- --- --- --- --- --- --- --- --- ---
	\item \colorbox{blue}{\bf Other Ph.D. Dissertation Awards}
	% Other Ph.D. Dissertation Awards
	\item Institute for Operations Research and the Management Sciences (INFORMS): \vspace{-0.2cm}
		\begin{enumerate} \itemsep -2pt
		\item INFORMS George B. Dantzig Dissertation Award: \url{http://www.informs.org/Recognize-Excellence/INFORMS-Prizes-Awards/George-B.-Dantzig-Dissertation-Award}
		\item Best Dissertation Award (Technology Management Section, for Ph.D. theses in technology management): \url{http://www.informs.org/Recognize-Excellence/INFORMS-Community-Prizes-and-Awards2/Technology-Management-Section/Best-Dissertation-Award}
		\item TSL Dissertation Prize (Transportation Science and Logistics Section, for doctoral dissertations in the transportation science and logistics area): \url{http://www.informs.org/Recognize-Excellence/INFORMS-Community-Prizes-and-Awards2/Transportation-Science-and-Logistics-Section/TSL-Dissertation-Prize}
		\item Best Dissertation Award (Telecommunications Section, for Ph.D. theses in telecommunications): \url{http://www.informs.org/Recognize-Excellence/INFORMS-Community-Prizes-and-Awards2/Telecommunications-Section/Best-Dissertation-Award}
		\item Frank M. Bass Dissertation Paper Award (Society for Marketing Science, for the best marketing paper derived from a Ph.D. thesis published in an INFORMS-sponsored journal): \url{http://www.informs.org/Recognize-Excellence/INFORMS-Community-Prizes-and-Awards2/Society-for-Marketing-Science/Frank-M.-Bass-Dissertation-Paper-Award}
		\item SOLA - Air Products Bi-Annual Dissertation Award (Section on Location Analysis, for Ph.D. theses on location related research): \url{http://www.informs.org/Recognize-Excellence/INFORMS-Community-Prizes-and-Awards2/Section-on-Location-Analysis/SOLA-Air-Products-Bi-Annual-Dissertation-Award}
		\end{enumerate}
	\item EURO Doctoral Dissertation Award (EDDA) (in operations research): \url{http://www.euro-online.org/display.php?page=edda1}
	\end{enumerate}
\item Other awards: \vspace{-0.3cm}
	\begin{itemize} \itemsep -2pt
	\item --- --- --- --- --- --- --- --- --- --- --- --- --- --- --- --- --- --- --- --- --- --- --- --- --- --- --- --- --- --- ---
	\item \colorbox{blue}{\bf Awards for Computer Science}
	% Awards for Computer Science
	\item ACM SIGMOD Undergraduate Award: \url{http://www.sigmod.org/sigmod-awards/sigmod-awards#undergraduate}
	\item European Association of Theoretical Computer Science (EATCS): Presburger Award (for young researchers in theoretical computer science), \url{http://www.eatcs.org/index.php/presburger}.
	\item Computer Research Association: \vspace{-0.2cm}
		\begin{enumerate} \itemsep -2pt
		\item Committee on the Status of Women in Computing Research (CRA-W): \vspace{-0.1cm}
			\begin{enumerate} \itemsep -1pt
			\item Borg Early Career Award (BECA): \url{http://www.cra-w.org/borg}
			\end{enumerate}
		\end{enumerate}
	\item European Conference on Wireless Sensor Networks (EWSN 201X, \url{http://www.nes.uni-due.de/ewsn2011}) and CONET, the Cooperating Objects Network of Excellence: Ph.D. Thesis Award Competition, \url{http://www.cooperating-objects.eu/}. ``Cooperating Objects combine the strong functional aspects of embedded systems, pervasive computing and wireless sensor networks. Cooperating objects entities federate themselves into dynamic and loose networks in order to reach a common goal. This common goal will typically be related to sensing or control.''
	\item --- --- --- --- --- --- --- --- --- --- --- --- --- --- --- --- --- --- --- --- --- --- --- --- --- --- --- --- --- --- ---
	\item \colorbox{blue}{\bf Awards for Biomedical Engineering}
	% Awards for Biomedical Engineering
	\item Biomedical Engineering Society (BMES): \vspace{-0.2cm}
		\begin{enumerate} \itemsep -2pt
		\item Rita Schaffer Young Investigator Award (for junior researchers in biomedical engineering): \url{http://www.bmes.org/aws/BMES/pt/sp/awards_investigator}
		\item Graduate and Undergraduate Student Awards: \url{http://www.bmes.org/aws/BMES/pt/sp/awards_student}
		\end{enumerate}
	\item --- --- --- --- --- --- --- --- --- --- --- --- --- --- --- --- --- --- --- --- --- --- --- --- --- --- --- --- --- --- ---
	\item \colorbox{blue}{\bf Awards for Mechanical Engineering}
	% Awards for Mechanical Engineering
	\item American Society of Mechanical Engineers (ASME): \vspace{-0.2cm}
		\begin{enumerate} \itemsep -2pt
		\item Henry Hess Award (authors of research papers who are below 31 years old): \url{http://www.asme.org/Governance/Honors/SocietyAwards/Henry_Hess_Award.cfm}
		\item Pi Tau Sigma Gold Medal (outstanding junior engineers): \url{http://www.asme.org/Governance/Honors/SocietyAwards/Pi_Tau_Sigma_Gold_Medal.cfm}
		\item Marshall B. Peterson Award (researchers in tribology who are below 30 years old): \url{http://www.asme.org/Governance/Honors/SocietyAwards/Marshall_B_Peterson_Award.cfm}
		\item Y.C. Fung Young Investigator Award (for young researchers in bioengineering): \url{http://www.asme.org/Governance/Honors/SocietyAwards/YC_Fung_Young_Investigator.cfm}
		\end{enumerate}
	\item --- --- --- --- --- --- --- --- --- --- --- --- --- --- --- --- --- --- --- --- --- --- --- --- --- --- --- --- --- --- ---
	\item \colorbox{blue}{\bf Awards for Civil Engineering}
	% Awards for Civil Engineering
	\item American Society of Civil Engineers (ASCE): \vspace{-0.3cm}
		\begin{enumerate} \itemsep -2pt
		\item Edmund Friedman Young Engineer Award for Professional Achievement (for junior engineers under the age of 36): \url{http://www.asce.org/AwardsContent.aspx?id=16776}
		\item Committee on Younger Members (CYM) Awards (for junior engineers): \url{http://www.asce.org/Content.aspx?id=11311}
		\item Collingwood Prize (for civil engineering researchers under the age of 35): \url{http://www.asce.org/AwardsContent.aspx?id=15352}
		\end{enumerate}
	\item --- --- --- --- --- --- --- --- --- --- --- --- --- --- --- --- --- --- --- --- --- --- --- --- --- --- --- --- --- --- ---
	\item \colorbox{blue}{\bf Awards for Chemical Engineering}
	% Awards for Chemical Engineering
	\item American Institute of Chemical Engineers (AIChE) awards: \url{http://www.aiche.org/Students/Awards/index.aspx}
	\item --- --- --- --- --- --- --- --- --- --- --- --- --- --- --- --- --- --- --- --- --- --- --- --- --- --- --- --- --- --- ---
	\item \colorbox{blue}{\bf Awards for Systems Engineering}
	% Awards for Systems Engineering
	\item International Council on Systems Engineering (INCOSE) Stevens Doctoral Award (for Promising Research in Systems Engineering and Integration; A.B.D.s / Ph.D. candidates): \url{http://www.incose.org/about/foundation/doctoralaward.aspx}
	\item --- --- --- --- --- --- --- --- --- --- --- --- --- --- --- --- --- --- --- --- --- --- --- --- --- --- --- --- --- --- ---
	\item \colorbox{blue}{\bf Awards for Mathematics, Operations Research, \& Management Sciences}
	% Awards for Mathematics, Operations Research, and Management Sciences
	\item Institute for Operations Research and the Management Sciences (INFORMS): \vspace{-0.2cm}
		\begin{enumerate} \itemsep -2pt
		\item INFORMS Undergraduate Operations Research Prize: \url{http://www.informs.org/Recognize-Excellence/INFORMS-Prizes-Awards/INFORMS-Undergraduate-Operations-Research-Prize}
		\item Optimization Prize for Young Researchers: \url{http://www.informs.org/Recognize-Excellence/INFORMS-Community-Prizes-and-Awards2/Optimization-Society/Optimization-Prize-for-Young-Researchers}
		\item Underrepresented Minorities and Women Honoraria: \url{http://www.informs.org/Recognize-Excellence/INFORMS-Community-Prizes-and-Awards2/Simulation-Society/Underrepresented-Minorities-and-Women-Honoraria}
		\item Best Dissertation Proposal Competition (College on Organization Science, for Ph.D. proposals in organizational science): \url{http://www.informs.org/Recognize-Excellence/INFORMS-Community-Prizes-and-Awards2/College-on-Organization-Science/Best-Dissertation-Proposal-Competition}
		\item ISMS Doctoral Dissertation Proposal Competition (Society for Marketing Science, for Ph.D. proposals in marketing): \url{http://www.informs.org/Recognize-Excellence/INFORMS-Community-Prizes-and-Awards2/Society-for-Marketing-Science/ISMS-Doctoral-Dissertation-Proposal-Competition}
		\end{enumerate}
	\item Alice T. Schafer Mathematics Prize For Excellence in Mathematics by an Undergraduate Woman: \url{http://www.awm-math.org/schaferprize.html}
	\item European Prize in Combinatorics: \vspace{-0.2cm}
		\begin{enumerate} \itemsep -2pt
		\item The prize is established to recognize excellent contributions in Combinatorics by young European researchers (eligibility of EU) not older than 35. 
		\item \url{http://www.math.tu-berlin.de/EuroComb05/prize.html}
		\end{enumerate}
	\item The AMS-MAA-SIAM Frank and Brennie Morgan Prize for Outstanding Research in Mathematics by an Undergraduate Student: \url{http://www.maa.org/awards/morgan.html}; \url{http://www.ams.org/profession/prizes-awards/ams-prizes/morgan-prize}; and \url{http://www.siam.org/prizes/sponsored/morgan.php}
	\item --- --- --- --- --- --- --- --- --- --- --- --- --- --- --- --- --- --- --- --- --- --- --- --- --- --- --- --- --- --- ---
	% Lists of awards
	\item \colorbox{blue}{\bf Lists of awards}: \vspace{-0.2cm}
		\begin{enumerate} \itemsep -2pt
		\item Association for Women in Science: \url{http://www.awis.org/displaycommon.cfm?an=1&subarticlenbr=69}
		\item International Center for Scientific Research (CIRS): \url{http://www.cirs.net/indexenglish.htm}
		\end{enumerate}
	\end{itemize}
\end{enumerate}














%%%%%%%%%%%%%%%%%%%%%%%%%%%%%%%%%%%%%%%%%%%
\section{Funding Nonprofit Organizations}
\label{fundingnonprofitorg}

Funding nonprofit organizations (including colleges and universities): \vspace{-0.3cm}
\begin{enumerate} \itemsep -4pt
\item Alfred P. Sloan Foundation: \vspace{-0.3cm}
	\begin{enumerate} \itemsep -2pt
	\item Major Program Areas: \url{http://www.sloan.org/program/1}
	\item Apply for Grants: \url{http://www.sloan.org/apply}
	\end{enumerate}
\item The Commonwealth Fund: \vspace{-0.3cm}
	\begin{enumerate} \itemsep -2pt
	\item Grants \& Programs: \vspace{-0.2cm}
		\begin{enumerate} \itemsep -2pt
		\item \url{http://www.commonwealthfund.org/Grants-and-Programs.aspx}
		\item ``The Fund supports independent research on health and social issues and makes grants to improve health care practice and policy. We are dedicated to helping people become more informed about their health care and improving care for vulnerable populations such as children, the elderly, low-income families, minorities, and the uninsured.''
		\end{enumerate}
	\end{enumerate}
\item The Heinz Endowments (Howard Heinz Endowment \& Vira I. Heinz Endowment): \vspace{-0.3cm}
	\begin{enumerate} \itemsep -2pt
	\item \url{http://www.heinz.org/grants.aspx}
	\item grant-making programs (for non-profit organizations): \vspace{-0.2cm}
		\begin{enumerate} \itemsep -2pt
		\item Arts \& Culture
		\item Children, Youth \& Families
		\item Education
		\item Environment
		\item Innovation Economy
		\end{enumerate}
	\end{enumerate}
\item Ford Foundation: \vspace{-0.3cm}
	\begin{enumerate} \itemsep -2pt
	\item Grants: \vspace{-0.2cm}
		\begin{enumerate} \itemsep -2pt
		\item \url{http://www.fordfoundation.org/grants/}
		\item Individuals Seeking Fellowships: \vspace{-0.1cm}
			\begin{enumerate} \itemsep -1pt
			\item \url{http://www.fordfoundation.org/grants/individuals-seeking-fellowships}
			\item Ford Foundation Fellowship Programs: \url{http://sites.nationalacademies.org/PGA/FordFellowships/index.htm}
			\item Ford Foundation International Fellowships Program: \url{http://www.fordifp.net/}
			\end{enumerate}
		\item Organizations Seeking Grants: \url{http://www.fordfoundation.org/grants/organizations-seeking-grants}
		\item Other Philanthropic Resources: \url{http://www.fordfoundation.org/grants/other-philanthropic-resources}
		\item Grant Search Results (list of grants): \url{http://www.fordfoundation.org/grants/search}
		\end{enumerate}
	\end{enumerate}
\item The Rockefeller Foundation: \vspace{-0.3cm}
	\begin{enumerate} \itemsep -2pt
	\item Grants \& Grantees: \vspace{-0.2cm}
		\begin{enumerate} \itemsep -2pt
		\item \url{http://www.rockefellerfoundation.org/grants}
		\item What We Fund: \url{http://www.rockefellerfoundation.org/grants/what-we-fund}
		\item Resources for Grantseekers: Links to other Philanthropic Sources, \url{http://www.rockefellerfoundation.org/grants/resources-grantseekers}
		\end{enumerate}
	\end{enumerate}
\item Carnegie Corporation of New York: \vspace{-0.3cm}
	\begin{enumerate} \itemsep -2pt
	\item Grantseekers: \vspace{-0.2cm}
		\begin{enumerate} \itemsep -2pt
		\item \url{http://carnegie.org/grants/grantseekers/}
		\item What we fund: \url{http://carnegie.org/grants/grantseekers/what-we-fund/}
		\item What we don't fund: \url{http://carnegie.org/grants/grantseekers/what-we-dont-fund/}
		\end{enumerate}
		\item Grants database: \url{http://carnegie.org/grants/grants-database/} and \url{http://carnegie.org/grants/}
		\item (Past) individual foundation grants: \url{http://carnegie.org/publications/carnegie-reporter/single/view/article/item/221/}
	\end{enumerate}
\item The Kresge Foundation: \vspace{-0.3cm}
	\begin{enumerate} \itemsep -2pt
	\item fields of interest: \vspace{-0.2cm}
		\begin{enumerate} \itemsep -2pt
		\item health,
		\item the environment,
		\item community development,
		\item arts and culture,
		\item education, and
		\item human services
		\end{enumerate}
	\item Values Criteria (for grantmaking): \url{http://www.kresge.org/index.php/who/our_values_criteria/}
	\item funding methods: \vspace{-0.2cm}
		\begin{enumerate} \itemsep -2pt
		\item \url{http://www.kresge.org/index.php/how/index/}
		\item \url{http://www.kresge.org/index.php/our_funding_methods/index/}
		\end{enumerate}
	\item Challenge Grant: \vspace{-0.2cm}
		\begin{enumerate} \itemsep -2pt
		\item \url{http://www.kresge.org/index.php/our_funding_methods/challenge_grant_program/}
		\item ``The Kresge Foundation awards facilities capital as a challenge grant to help nonprofit organizations build their base of private financial support as they conduct capital campaigns to build or renovate their facilities.''
		\item ``Facilities capital challenge grants are awarded to organizations that cater specifically to the needs of poor, disadvantaged and disenfranchised in six program areas: Health Program, the Environment Program, Arts and Culture Program, Education Program, Human Services Program, and Community Development / Detroit Program.''
		\item ``Most challenge grant awards are made to U.S.-based organizations. On rare occasions, we award challenge grants to international organizations undertaking exceptional projects that align with the strategic objectives of a given program and advance Kresge's values.''
		\end{enumerate}
	\item Detroit Program: \vspace{-0.2cm}
		\begin{enumerate} \itemsep -2pt
		\item Kresge Arts Support: \url{http://www.kresge.org/index.php/what/detroit_program/kresge_arts_support/}
		\item Kresge Arts in Detroit: \url{http://www.kresge.org/index.php/what/detroit_program/kresge_arts_in_detroit/}
		\end{enumerate}
	\item Our Grants: \vspace{-0.2cm}
		\begin{enumerate} \itemsep -2pt
		\item \url{http://www.kresge.org/index.php/our_grants/index/}
		\item grants database: \url{http://maps.foundationcenter.org/grantmakers/index.php?gmkey=KRES002}
		\item Arts and Community Building: \vspace{-0.1cm}
			\begin{enumerate} \itemsep -1pt
			\item \url{http://www.kresge.org/index.php/what/arts_and_culture/arts_and_community_building#Community Arts}
			\item ``Cultural institutions and artists animate our communities, bring disparate people together to share common experiences, and help us imagine a better future. As the demographics of our communities become more diverse, artists and cultural institutions help us bridge differences and build cross-cultural understanding. As our economy struggles, creative enterprises and creative sector leaders offer hope for community renewal and new job development.''
			\item two pilot initiatives: College/Arts initiative, and the Community Arts initiative
			\item ``The pilot cities [for the Community Arts initiative] include Baltimore, Maryland; Birmingham, Alabama; Detroit, Michigan; St. Louis, Missouri; and Tucson, Arizona.''
			\item ``Grants for Arts and Community Building are by invitation only.''
			\end{enumerate}
		\end{enumerate}
	\end{enumerate}
\item New York Women's Foundation: \vspace{-0.3cm}
	\begin{enumerate} \itemsep -2pt
	\item Grant Information and Application: \vspace{-0.2cm}
		\begin{enumerate} \itemsep -2pt
		\item \url{http://www.nywf.org/grant.html}
		\item focus areas: \vspace{-0.1cm}
			\begin{enumerate} \itemsep -1pt
			\item Anti-Violence and Safety
			\item Economic Security
			\item Health, Sexual Rights and Reproductive Justice
			\end{enumerate}
		\item ``Grants usually range from \$50,000 to a maximum of \$70,000 [that last for a year, and can be renewed up to 5 years].''
		\end{enumerate}
	\end{enumerate}
\item The Foundation Center: \vspace{-0.3cm}
	\begin{enumerate} \itemsep -2pt
	\item Grantseekers: \url{http://foundationcenter.org/getstarted/}
	\item Find funders: \url{http://foundationcenter.org/findfunders/}
	\item GrantSpace$^{\rm SM}$: \vspace{-0.2cm}
		\begin{enumerate} \itemsep -2pt
		\item \url{http://grantspace.org/}
		\item ``GrantSpace$^{\rm SM}$ will help you gain the knowledge and skills you need to get grants, manage your nonprofit, and improve your community.''
		\item ``Established in 1956 and today supported by close to 550 foundations, the Foundation Center is a national nonprofit service organization recognized as the nation�s leading authority on organized philanthropy, connecting nonprofits and the grantmakers supporting them to tools they can use and information they can trust. Its audiences include grantseekers, grantmakers, researchers, policymakers, the media, and the general public. The Center maintains the most comprehensive database on U.S. grantmakers and their grants; issues a wide variety of print, electronic, and online information resources; conducts and publishes research on trends in foundation growth, giving, and practice; and offers an array of free and affordable educational programs.''
		\item Resources for Non-U.S. Grantseekers: \url{http://grantspace.org/Tools/Knowledge-Base/Resources-for-Non-U.S.-Grantseekers}
		\item Resources for Individual Grantseekers: \vspace{-0.1cm}
			\begin{enumerate} \itemsep -1pt
			\item \url{http://grantspace.org/Tools/Knowledge-Base/Individual-Grantseekers}
			\item \url{http://gtionline.foundationcenter.org/}
			\item General: \url{http://grantspace.org/Tools/Knowledge-Base/Individual-Grantseekers/General}
			\item Artists: \url{http://grantspace.org/Tools/Knowledge-Base/Individual-Grantseekers/Artists}
			\item Students: \url{http://grantspace.org/Tools/Knowledge-Base/Individual-Grantseekers/Students}
			\item Fiscal Sponsorship: \url{http://grantspace.org/Tools/Knowledge-Base/Individual-Grantseekers/Fiscal-Sponsorship}
			\item For-Profit Enterprises: \url{http://grantspace.org/Tools/Knowledge-Base/Individual-Grantseekers/For-Profit-Enterprises}
			\end{enumerate}
		\end{enumerate}
	\end{enumerate}
\item The Lemelson Foundation: \vspace{-0.3cm}
	\begin{enumerate} \itemsep -2pt
	\item \url{http://web.mit.edu/invent/w-foundation.html}
	\item Programs \& Grants: \url{http://www.lemelson.org/programs-grants}
	\item Grantmaking: \url{http://www.lemelson.org/grantmaking}
	\end{enumerate}
\item Partnership for Higher Education in Africa (PHEA): \vspace{-0.3cm}
	\begin{enumerate} \itemsep -2pt
	\item \url{http://www.foundation-partnership.org/} and \url{http://www.foundation-partnership.org/index.php?id=1}
	\item Grants Database: \url{http://www.foundation-partnership.org/index.php?id=2}
	\item Partnership Publications: \url{http://www.foundation-partnership.org/index.php?id=3}
	\end{enumerate}
\item Smithsonian Institution: \vspace{-0.3cm}
	\begin{enumerate} \itemsep -2pt
	\item Smithsonian Institution Traveling Exhibition Service (SITES): \vspace{-0.2cm}
		\begin{enumerate} \itemsep -2pt
		\item Smithsonian Community Grant program (supported by MetLife Foundation): \vspace{-0.1cm}
			\begin{enumerate} \itemsep -1pt
			\item \url{http://www.sites.si.edu/funding/grant2.htm}
			\item ``This program seeks to deepen connections between SITES' host venues and their communities by encouraging exhibitors to engage their local audiences in new and exciting ways while creating broader access to our exhibitions.''
			\item ``Under this new program, eligible SITES exhibitors may apply for up to \$5,000 for expenses related to public, educational programming produced in conjunction with a SITES exhibit. Exhibitors may choose to enhance current program offerings or to create a new program especially suited to the topic of the exhibition.''
			\end{enumerate}
		\end{enumerate}
	\end{enumerate}
\end{enumerate}
















%%%%%%%%%%%%%%%%%%%%%%%%%%%%%%%%%%%%%%%%%%%
\section{Technology-Related Public Policy}
\label{techpublicpolicy}

Resources for engagement in creating technology-related public policy: \vspace{-0.3cm}
\begin{enumerate} \itemsep -4pt
\item Yale Journal of Law \& Technology (YJOLT): \vspace{-0.3cm}
	\begin{enumerate} \itemsep -2pt
	\item \url{http://www.yjolt.org/}
	\item \url{http://wingenroth.org/}
	\end{enumerate}
\item ACM Public Policy Office: \vspace{-0.3cm}
	\begin{enumerate} \itemsep -2pt
	\item It represents ACM and its US Public Policy Council (USACM) on information technology policy issues that impact the computing field.
	\item It seeks to educate policymakers and the public about policies that will that foster innovations in computing and related disciplines in ways that benefit society.
	\item It also informs ACM's members and the public about policy developments through its weblog, Washington Update newsletter and articles in ACM publications.
	\item ACM US Public Policy Council (USACM): \url{http://usacm.acm.org/}
	\item ACM Committee on Computers and Public Policy (CCPP): \url{http://www.acm.org/public-policy/acm-committee-on-computers-and-public-policy}
	\item \url{http://www.acm.org/public-policy}
	\end{enumerate}
\item IEEE: \vspace{-0.3cm}
	\begin{enumerate} \itemsep -2pt
	\item IEEE-USA: \url{http://www.ieeeusa.org/policy/default.asp}
	\item Smart Grids: \url{http://smartgrid.ieee.org/public-policy}
	\end{enumerate}
\item Computing Community Consortium (CCC): \url{http://www.cra.org/ccc/}
\item Computing Research Association (CRA): \vspace{-0.3cm}
	\begin{enumerate} \itemsep -2pt
	\item \url{http://www.cra.org/}
	\item CRA Government Affairs: \url{http://www.cra.org/govaffairs/index.php}
	\end{enumerate}
\item EngineeringPolicy.org: \url{http://www.engineeringpolicy.org/}
\item Congressional Bi-Partisan Robotics Caucus: \url{http://www.roboticscaucus.org/}
\item Advisory Committee for the Congressional Research and Development $[$R\&D$]$ Caucus: \url{http://www.researchcaucus.org/}
\item {\it National Academies Press} (NAP), from the (US) {\it National Academies}: \url{http://www.nap.edu/}
\item {\it Coalition to Diversify Computing}: \url{http://www.cdc-computing.org/}
\item American Institute of Aeronautics and Astronautics (AIAA): \vspace{-0.3cm}
	\begin{enumerate} \itemsep -2pt
	\item \url{http://www.aiaa.org/content.cfm?pageid=7}
	\end{enumerate}
\item : \url{}
\item : \url{}
\item : \url{}
\item : \url{}
\item : \url{}
\item : \url{}
\item : \url{}
\item : \url{}
\end{enumerate}






%%%%%%%%%%%%%%%%%%%%%%%%%%%%%%%%%%%%%%%%%%%
\section{Feminist Outreach}
\label{feministoutreach}

Feminist outreach: \vspace{-0.3cm}
\begin{enumerate} \itemsep -4pt
\item Myra Sadker Foundation: \vspace{-0.3cm}
	\begin{enumerate} \itemsep -2pt
	\item $100+$ Ideas to Promote Gender Equity in Schools and Beyond: \url{http://www.sadker.org/100ideas.html}
	\item Gender Equity Activities: \url{http://www.sadker.org/WhatYouCanDo.html}
	\item Gender Equity Activities for Concerned Citizens: \url{http://www.sadker.org/GenderEquity-citizens.html}
	\item Gender Equity Activities for Families: \url{http://www.sadker.org/GenderEquity-family.html}
	\item Gender Equity Activities for Teachers: \vspace{-0.2cm}
		\begin{enumerate} \itemsep -2pt
		\item Early Childhood: \url{http://www.sadker.org/GenderEquity-teacher1.html}
		\item Primary Grades: \url{http://www.sadker.org/GenderEquity-teacher2.html}
		\item Upper Elementary: \url{http://www.sadker.org/GenderEquity-teacher3.html}
		\item Middle and High School: \url{http://www.sadker.org/GenderEquity-teacher4.html}
		\end{enumerate}
	\item Resources for feminism and links to web pages of feminist organizations: \url{http://www.sadker.org/ReadsLinks.html}
	\end{enumerate}
\item Feminist student organizations at colleges and universities: \vspace{-0.3cm}
	\begin{enumerate} \itemsep -2pt
	\item For example, at the University of Southern California, the organizations associated with feminist causes are: \vspace{-0.2cm}
		\begin{enumerate} \itemsep -2pt
		\item {\it USC Center for Women \& Men}: \url{http://www.usc.edu/student-affairs/cwm/links.html}
		\item {\it USC Women's Student Assembly}: \url{http://www-scf.usc.edu/~wsausc/Welcome.html}
		\end{enumerate}
	\end{enumerate}
\item International Women's Day: \url{http://www.internationalwomensday.com/}
\item Gender Across Borders: \vspace{-0.3cm}
	\begin{enumerate} \itemsep -2pt
	\item Feminism Resources: \url{http://www.genderacrossborders.com/feminist-resources/}
	\end{enumerate}
\item {\it V-Day}: \vspace{-0.3cm}
	\begin{enumerate} \itemsep -2pt
	\item \url{http://www.vday.org/}
	\item Organization that helps women plan and organize events to bring awareness about sexual assault, and what we can do to reduce sexual assault.
	\end{enumerate}
\item {\it Take Back The Night}: \vspace{-0.3cm}
	\begin{enumerate} \itemsep -2pt
	\item \url{http://www.takebackthenight.org/}
	\item Organization that helps women plan and organize events to bring awareness about sexual assault, and what we can do to reduce sexual assault. It also encourages sexual assault survivors to speak out about their sexual assaults, so that they would shame their perpetrators and let other women (and men) know that they is nothing to be ashamed of as sexual assault survivors. This is because the faults lie 100\% with the perpetrators, and not with the survivors.
	\end{enumerate}
\item {\it United Nations Development Fund for Women} (UNIFEM): \vspace{-0.3cm}
	\begin{enumerate} \itemsep -2pt
	\item \url{http://www.unifem.org/}
	\item Organization that addresses many challenges faced by girls and women.
	\end{enumerate}
\item {\it National Organization for Women}: \vspace{-0.3cm}
	\begin{enumerate} \itemsep -2pt
	\item \url{http://www.now.org/}
	\item Feminist organization in the US.
	\end{enumerate}
\item {\it A Woman's Nation}: \vspace{-0.3cm}
	\begin{enumerate} \itemsep -2pt
	\item \url{http://www.shriverreport.com/awn/}
	\item \url{http://awomansnation.com} or \url{http://www.shriverreport.com/}
	\end{enumerate}
\item {\it Peace Over Violence} is a non-profit, feminist, multicultural, volunteer organization dedicated to a building healthy relationships, families and communities free from sexual, domestic and interpersonal violence: \url{http://peaceoverviolence.org/}
\item SoulSpeakOut: \url{http://www.soulspeakout.org/resources/}
\item {\it Haven Hills}: \url{http://havenhills.org/}
%\item MaleSurvivor: \url{http://www.malesurvivor.org/}
\end{enumerate}












%%%%%%%%%%%%%%%%%%%%%%%%%%%%%%%%%%%%%%%%%%%
\section{Outreach: Professional Organizations}
\label{outreachproorgs}

Professional organizations: \vspace{-0.3cm}
\begin{enumerate} \itemsep -4pt
\item --- --- --- --- --- --- --- --- --- --- --- --- --- --- --- --- --- --- --- --- --- --- --- --- --- --- --- --- --- --- ---
\item \colorbox{blue}{\bf Professional Organizations for the Performance, Literary, and Visual Arts}
% Professional Organizations for the Performance, Literary, and Visual Arts
\item Americans for the Arts: \vspace{-0.3cm}
	\begin{enumerate} \itemsep -2pt
	\item \url{http://www.americansforthearts.org/get_involved/membership/default.asp}
	\item \url{http://www.artsusa.org/get_involved/membership/default.asp}
	\item Provides membership for organizations and individuals
	\item Individual membership are available for: \vspace{-0.2cm}
		\begin{enumerate} \itemsep -2pt
		\item Students
		\item Entrepreneurs (e.g., people in art management)
		\item Innovators
		\item Colleagues (artists)
		\end{enumerate}
	\item Americans for the Arts {\bf Emerging Leader Program}: \vspace{-0.2cm}
		\begin{enumerate} \itemsep -2pt
		\item \url{http://www.artsusa.org/networks/emerging_leaders/resources/default.asp}
		\item Has various resources for professional development, including mentoring
		\end{enumerate}
	\item Advocacy ({\bf public policy}): \url{http://www.artsusa.org/get_involved/advocate.asp}
	\end{enumerate}
\item --- --- --- --- --- --- --- --- --- --- --- --- --- --- --- --- --- --- --- --- --- --- --- --- --- --- --- --- --- --- ---
\item \colorbox{blue}{\bf Professional Organizations for the Musical Artists}
% Professional Organizations for the Musical Artists
\item The Recording Academy: \url{http://www.grammy365.com/join/membership-types}
\end{enumerate}













%%%%%%%%%%%%%%%%%%%%%%%%%%%%%%%%%%%%%%%%%%%
\section{Other Outreach}
\label{otheroutreach}

Other outreach: \vspace{-0.3cm}
\begin{enumerate} \itemsep -4pt
\item The Joy McCann Foundation: \vspace{-0.3cm}
	\begin{enumerate} \itemsep -2pt
	\item The Joy McCann Professorships in Law: \url{http://www.mccannfoundation.org/law.htm}
	\end{enumerate}
\item National Academy of Sciences: \vspace{-0.3cm}
	\begin{enumerate} \itemsep -2pt
	\item {\it Science \& Entertainment Exchange} program: \vspace{-0.2cm}
		\begin{enumerate} \itemsep -2pt
		\item \url{http://www.scienceandentertainmentexchange.org/}
		\item Provide science and engineering knowledge to help professionals in the entertainment industry create engaging storylines involving science and technology.
		\end{enumerate}
	\end{enumerate}
\item U.S. Department of State: \vspace{-0.3cm}
	\begin{enumerate} \itemsep -2pt
	\item Bureau of Educational and Cultural Affairs: \vspace{-0.2cm}
		\begin{enumerate} \itemsep -2pt
		\item Programs: \url{http://exchanges.state.gov/jexchanges/programs.html}
		\item Fulbright Classroom Teacher Exchange Program: \vspace{-0.1cm}
			\begin{enumerate} \itemsep -1pt
			\item \url{http://exchanges.state.gov/globalexchanges/fulbright-teacher-exchange-program.html}
			\item ``The Fulbright Classroom Teacher Exchange provides opportunities for primary and secondary teachers to exchange positions with colleagues in other countries. The participants contribute to mutual understanding by bringing international knowledge and perspectives to the U.S. and foreign classrooms, schools and communities. Full-time U.S. teachers can take part in either a year-long or semester-long direct exchange with a counterpart in another country.''
			\end{enumerate}
		\item FORTUNE/U.S. State Department Global Women's Mentoring Partnership: \vspace{-0.1cm}
			\begin{enumerate} \itemsep -1pt
			\item \url{http://exchanges.state.gov/citizens/professionals/fortunepartnership.html}
			\item ``This public-private partnership places talented, emerging women leaders from all over the world in mentoring programs with FORTUNE's Most Powerful Women Leaders.''
			\end{enumerate}
		\item Edward R. Murrow Program for Journalists: \vspace{-0.1cm}
			\begin{enumerate} \itemsep -1pt
			\item \url{http://exchanges.state.gov/ivlp/murrow.html}
			\item ``The Edward R. Murrow Program for Journalists invites rising international journalists to travel to the United States and examine journalistic principles and practices.''
			\end{enumerate}
		\item International Visitor Leadership Program: \vspace{-0.1cm}
			\begin{enumerate} \itemsep -1pt
			\item \url{http://exchanges.state.gov/ivlp/ivlp.html}
			\item ``These visits reflect the International Visitors' professional interests and support the foreign policy goals of the United States.''
			\item ``International Visitors are current or emerging leaders in government, politics, the media, education, the arts, business and other key fields.''
			\item ``International Visitors travel to the U.S. for carefully designed programs that reflect their professional interests and U.S. foreign policy goals. They travel in a variety of thematic programs, either individually or in groups, for up to three weeks. While in the U.S., International Visitors typically visit Washington, DC and three additional towns or cities that highlight the tremendous diversity of the U.S. They attend professional appointments with their American counterparts, learn about the U.S. system of government at the national, state and local levels, visit American schools, and experience American culture and social life.''
			\item ``There is no application for this program. International Visitors are selected and nominated annually by American Foreign Service Officers at U.S. Embassies around the world.''
			\end{enumerate}
		\item Au Pair: \vspace{-0.1cm}
			\begin{enumerate} \itemsep -1pt
			\item \url{http://exchanges.state.gov/jexchanges/programs/aupair.html}
			\item ``Through the Au Pair program, foreign nationals between 18 and 26 years of age participate in the home life of a host family. Au pairs provide limited childcare services for up to 12 months. An extension of 6, 9, or 12 months may be granted in certain cases.''
			\end{enumerate}
		\item Summer Work Travel: \vspace{-0.1cm}
			\begin{enumerate} \itemsep -1pt
			\item \url{http://exchanges.state.gov/jexchanges/programs/swt.html}
			\item ``In the summer work travel program, post-secondary students may enter the United States to work and travel during their summer vacation. Participants can be admitted to the program more than once. The maximum length of the program is four months.''
			\end{enumerate}
		\item Internship: \vspace{-0.1cm}
			\begin{enumerate} \itemsep -1pt
			\item \url{http://exchanges.state.gov/jexchanges/programs/intern.html}
			\item ``Internship programs are designed to allow foreign professionals to come to the United States to gain exposure to U.S. culture and to receive training in U.S. business practices in their chosen occupational field.  The maximum duration of an internship in any occupational field is 12 months. Upon completion of their exchange programs, participants are expected to return to their home countries.''
			\end{enumerate}
		\item Professional Exchanges Division: \vspace{-0.1cm}
			\begin{enumerate} \itemsep -1pt
			\item \url{http://exchanges.state.gov/citizens/profs.html}
			\item ``The Professional Exchanges division provides grants to U.S. nonprofit organizations to carry out exchange programs that support the professional development of foreign participants. The purpose of each exchange program is to engage with foreign leaders in critical professions, to demonstrate respect for foreign cultures, and to promote mutual understanding between the people of the United States and other countries.''
			\item ``Professional exchanges typically last several years and include internships, study tours or workshops in the United States and in the host country. Participants come from a variety of professions including education administrators, public servants, journalists, labor union officials, entrepreneurs, environmental leaders, jurists, lawyers, and civic leaders.''
			\end{enumerate}
		\end{enumerate}
	\end{enumerate}
\item Teach For All: \url{http://teachforallnetwork.org/}
\item --- --- --- --- --- --- --- --- --- --- --- --- --- --- --- --- --- --- --- --- --- --- --- --- --- --- --- --- --- --- ---
\item \colorbox{blue}{\bf Resources for Artists and Musicians}
% Resources for Artists and Musicians
\item League of American Orchestras and the Association of Performing Arts Presenters: \vspace{-0.3cm}
	\begin{enumerate} \itemsep -2pt
	\item {\it ArtistsfromAbroad.org}: \vspace{-0.2cm}
		\begin{enumerate} \itemsep -2pt
		\item \url{http://www.artistsfromabroad.org/}
		\item ``{\it ArtistsfromAbroad.org} features complete and up-to-date guidance on the visa process and tax treatment for foreign guest artists.''
		\end{enumerate}
	\end{enumerate}
\item Young Concert Artists, Inc. \vspace{-0.3cm}
	\begin{enumerate} \itemsep -2pt
	\item Composer Program (for American composers between 20 and 26 years of age): \url{http://www.yca.org/auditions/}
	\end{enumerate}
\item The John F. Kennedy Center for the Performing Arts: \vspace{-0.3cm}
	\begin{enumerate} \itemsep -2pt
	\item Mary Lou Williams Women in Jazz Emerging Artist Workshop: \vspace{-0.2cm}
		\begin{enumerate} \itemsep -2pt
		\item \url{http://www.kennedy-center.org/programs/jazz/womeninjazz/competition.html}
		\item ``The workshop provides female jazz artists ages 18 to 35 with an opportunity to explore and develop their artistry under the guidance of leading jazz artists and instructors. Each year, the workshop will focus on a specific instrument.''
		\item ``The 2011 Mary Lou Williams Women in Jazz Emerging Artist Workshop is open to advanced female jazz pianists who plan to pursue jazz performance as a career. Eligibility is exclusive to pianists who will be 18-35 years old on May 18, 2011 and have never recorded or been contracted to record as a leader or co-leader on a major label at the time of application. All applicants must be proficient in English.''
		\end{enumerate}
	\end{enumerate}
\item Grantmakers in the Arts (GIA): \vspace{-0.3cm}
	\begin{enumerate} \itemsep -2pt
	\item ``The mission of Grantmakers in the Arts (GIA) is to provide leadership and service to advance the use of philanthropic resources on behalf of arts and culture.''
	\item Arts Funding Topics: \url{http://www.giarts.org/arts-funding-topics}
	\end{enumerate}
\item The Dana Foundation: \vspace{-0.3cm}
	\begin{enumerate} \itemsep -2pt
	\item Arts Education program: \vspace{-0.2cm}
		\begin{enumerate} \itemsep -2pt
		\item Arts Education Grants: \vspace{-0.1cm}
			\begin{enumerate} \itemsep -1pt
			\item \url{http://www.dana.org/grants/BrowseArtsGrants.aspx}
			\item ``In 2001, The Dana Foundation created the Arts Education program with a sole focus of providing grants to support professional development for teaching artists and in-school arts specialists. The first several years of grants were to  programs in New York City, Washington, DC, Los Angeles and to organizations with a 50 mile radius of the three.''
			\item ``The Rural Initiative launched in 2006 with 6 grants awarded to organizations providing professional development in rural areas of the United States.''
			\end{enumerate}
		\end{enumerate}
	\end{enumerate}
\item writing/poetry contests: \vspace{-0.3cm}
	\begin{enumerate} \itemsep -2pt
	\item International 3-Day Novel Contest: \url{http://www.3daynovel.com/about/?contest}
	\end{enumerate}
\end{enumerate}




%%%%%%%%%%%%%%%%%%%%%%%%%%%%%%%%%%%%%%%%%%%
\section{Christian Colleges and Universities}
\label{christianunis}

Christian colleges and universities: \vspace{-0.3cm}
\begin{enumerate} \itemsep -4pt
\item List of Christian colleges and universities: \vspace{-0.3cm}
	\begin{enumerate} \itemsep -2pt
	\item Council for Christian Colleges and Universities (CCCU): \vspace{-0.2cm}
		\begin{enumerate} \itemsep -2pt
		\item \url{http://en.wikipedia.org/wiki/Council_for_Christian_Colleges_and_Universities}
		\item \url{http://www.cccu.org/}
		\end{enumerate}
	\item Christian College Consortium: \vspace{-0.2cm}
		\begin{enumerate} \itemsep -2pt
		\item \url{http://en.wikipedia.org/wiki/Christian_College_Consortium}
		\item \url{http://www.ccconsortium.org/}
		\end{enumerate}
	\end{enumerate}
\item California Baptist University, Riverside
\item Messiah College (Grantham, PA): \vspace{-0.3cm}
	\begin{enumerate} \itemsep -2pt
	\item Department of Engineering: \vspace{-0.2cm}
		\begin{enumerate} \itemsep -2pt
		\item \url{http://www.messiah.edu/departments/engineering/}
		\item B.S. programs in: \vspace{-0.1cm}
			\begin{enumerate} \itemsep -1pt
			\item Biomedical Engineering
			\item Computer Engineering
			\item Electrical Engineering
			\end{enumerate}
		\end{enumerate}
	\item Department of Information and Mathematical Sciences: \vspace{-0.2cm}
		\begin{enumerate} \itemsep -2pt
		\item \url{http://www.messiah.edu/departments/mathsci/index.html}
		\item Offers a B.A. Computer Science program
		\end{enumerate}
	\end{enumerate}
\end{enumerate}
























%%%%%%%%%%%%%%%%%%%%%%%%%%%%%%%%%%%%%%%%%
% Thoughts and Resources for Specific Areas and Topics
%	% This is written by Zhiyang Ong for his management of information and tasks.
%
% It includes information on professional development, including membership of professional organizations and networking societies.





%%%%%%%%%%%%%%%%%%%%%%%%%%%%%%%%%%%%%%%%%%%
\section{Heuristic for Locating Outreach Resources}
\label{heuristiclocateoutreach}

\proc{Find}$(\varphi, \tau)$ is a heuristic for locating resources for outreach activities, which includes finding information about the following: \vspace{-0.3cm}
\begin{enumerate} \itemsep -4pt
\item awards
\item career resources (including material for career guidance)
\item competitions and contests
\item educational material (e.g., suggested activities and curricular) for specific areas, such as marine sciences and electrical/computer engineering
\item fellowships
\item internships
\item scholarships
\item summer camps
\item summer programs (or summer schools); here, summer schools refer to short educational programs that last from days (e.g., a weekend for the ACM SIGDA Design Automation Summer School) to about a month (e.g., Santa Fe Institute's Complex Systems Summer Schools)
\end{enumerate}
\ \\

Its input $\tau$ is the deadline by which this search process must terminate. For example, if I have to apply for internships by next week, I would use the date of a week from now as the deadline $\tau$. In line \ref{find-pt-professional-org}, an example of a professional organization is the Institute of Electrical and Electronics Engineers (IEEE). The term ``good'' that is used in line \ref{find-pt-gd-uni} is an arbitrary measure of quality determined by the reader/user. \\

A reading group (in line \ref{find-pt-reading-grp}) is a small group of (graduate) students, which may possibly include professors and postdocs, that meet regularly (e.g., once/twice a week) to discuss papers that they have read since the previous meeting/discussion. Each individual in the reading group can be assigned a paper to read and present at the next meeting. The aim of a reading group is to improve the coverage of papers in our research area that each member has read. This is important for interdisciplinary research, since grad students working in interdisciplinary research areas have so much ground to cover. \\

Line \ref{find-pt-athletics} uses the term ``athletics department'' to refer to an administrative department at an American college or university that is in charge of managing varsity/NCAA sports teams. An example of a profession-specific networking organization (line \ref{find-pt-netwk-org}) is DVClub. In line \ref{find-pt-domain-specific-www}, a domain-specific web page is {\it SAT Live!}. An example of a corporate research laboratory (line \ref{find-pt-corporate-research-labs}) is ``Cadence Research Laboratories'' (\url{http://www.cadence.com/cadence/cadence_labs/pages/default.aspx}), and an example of a research institute (line \ref{find-pt-research-institute}) is Santa Fe Institute.




\begin{codebox}
\Procname{$\proc{Find}(\varphi, \tau)$}
\zi	\Comment {\it Input $\varphi \gets $ Item to find out about}
\zi	\Comment {\it Input $\tau \gets $ Deadline for the search process}
\zi	\Comment {\it Output $\kappa \gets $ List of resources about $\varphi$}
\zi
\li \While ( [ resources about $\varphi$ are inadequate ] AND [ $\tau$ has not yet passed ] )
	\Do
\li	Find out the professional organizations for the field of $\varphi$	\label{find-pt-professional-org}
\li	\For each professional organization in the field
		\Do
\li		Check if it has information about $\varphi$ in its web pages, publications, or mailing list archive
\li		\If (it has information about $\varphi$)
			\Then
\li			Add that information to $\kappa$
			\End
		\End
\zi
\li	\For each good (college OR university)	\label{find-pt-gd-uni}
		\Do
\li		\If ($\varphi == $ summer programs )
			\Then
\li			Search for summer programs in the web pages of departments \& schools/colleges
\li		\ElseIf ($\varphi == $ summer camps )
			\Then
\li			Search for summer camps in the web pages of departments \& schools/colleges
\li			Search for summer camps in the web pages of administrative/athletics departments	\label{find-pt-athletics}
\li		\ElseNoIf
\li			Search for $\varphi$ in the web pages of the department(s), including its news section/archive
\li			Search for $\varphi$ in the web pages of professors, postdoctoral researchers, \& students
\li			Search for $\varphi$ in the web pages of reading groups		\label{find-pt-reading-grp}
\li			Search for $\varphi$ in the web pages of student organizations
\li			Search for $\varphi$ in the mailing list archive of classes \& the department
\li			Search for $\varphi$ in the mailing list archive of research groups/labs and projects
\li			Search for $\varphi$ in the mailing list archive of reading groups
\li			Search for $\varphi$ in the mailing list archive of student organizations
			\End
\zi
\li		\If (it has information about $\varphi$)
			\Then
\li			Add that information to $\kappa$
			\End
		\End
\zi
\li	Search for $\varphi$ in the mailing list archive of open-source projects
\li	Search for $\varphi$ in the mailing list archive of profession-specific networking organizations	\label{find-pt-netwk-org}
\li	Search for $\varphi$ in the web pages of domain-specific web pages	\label{find-pt-domain-specific-www}
\li	Search for $\varphi$ in the web pages of research scientists in corporate research labs	\label{find-pt-corporate-research-labs}
\li	Search for $\varphi$ in the web pages of research scientists in research institutes	\label{find-pt-research-institute}
\li	\If ( [ mailing list archive OR web page ] has information about $\varphi$)
		\Then
\li		Add that information to $\kappa$
		\End
	\End	
\li \Return $\kappa$
\end{codebox}




%%%%%%%%%%%%%%%%%%%%%%%%%%%%%%%%%%%%%%%%%%%
\section{General Outreach Resources}
\label{generaloutreachresources}

General outreach resources: \vspace{-0.3cm}
\begin{enumerate} \itemsep -4pt
\item volunteering opportunities: \vspace{-0.3cm}
	\begin{enumerate} \itemsep -2pt
	\item Engineers Without Borders: \url{http://www.ewb-international.org/}
	\item Australian Volunteers International: \url{http://www.australianvolunteers.com/}
	\item Youth Challenge Australia: \url{http://www.youthchallenge.com.au/}
	\item Go Volunteer: \url{http://www.govolunteer.com.au/}
	\item Volunteer Search: \url{http://www.volunteersearch.gov.au/}
	\item Conservation Volunteers: \url{http://www.conservationvolunteers.com.au/volunteer}
	\item Volunteering Australia: \url{http://www.volunteeringaustralia.org/html/s01_home/home.asp}
	\item Sponsors for Educational Opportunity (SEO): \vspace{-0.2cm}
		\begin{enumerate} \itemsep -2pt
		\item Philanthropy \& Volunteerism Resources, \url{http://www.seo-usa.org/AlumniResources}
		\item Volunteer Leadership Opportunities: \url{http://www.seo-usa.org/Alumni_Volunteer}
		\end{enumerate}
	\item : \url{}
	\end{enumerate}
\item public health and preventive medicine: \vspace{-0.3cm}
	\begin{enumerate} \itemsep -2pt
	\item U.S. Department of Health \& Human Services: \vspace{-0.2cm}
		\begin{enumerate} \itemsep -2pt
		\item Agency for Healthcare Research and Quality (AHRQ): \vspace{-0.1cm}
			\begin{enumerate} \itemsep -1pt
			\item Prevention \& Care Management: Resources and Materials, \url{http://www.ahrq.gov/clinic/ppipix.htm}
			\end{enumerate}
		\end{enumerate}
	\end{enumerate}
\item career resources: \vspace{-0.3cm}
	\begin{enumerate} \itemsep -2pt
	\item CRAC: The Career Development Organisation: \vspace{-0.2cm}
		\begin{enumerate} \itemsep -2pt
		\item {\it icould}: \vspace{-0.1cm}
			\begin{enumerate} \itemsep -1pt
			\item \url{http://icould.com/about/}
			\item Resource for students, people who are commencing their careers or are making changes in their careers, career counselors, parents, educators, human resource staff, and employers.
			\item icould, {\it Stories by Life Theme}, in icould: Watch Career Stories. Available online at: \url{http://icould.com/watch-career-stories/by-life-theme/}; last accessed on December 25, 2010. [ Has articles briefly describing how people pursued their career goals or their career paths as they went through different experiences in life. This includes people who ``blossomed after school,'' changed careers or became entrepreneurs, had no plans, took risks, encountered turning points, faced adversity, have disabilities, went through financial hardship, or got laid off. It also has stories of people who volunteered, took a gap year, or pursued internships. ]
			\item icould, {\it Stories by Job Type}, in icould: Watch Career Stories. Available online at: \url{http://icould.com/watch-career-stories/by-job-type/}; last accessed on December 25, 2010. [ Includes stories of people in automotive retail, customer services, engineering, education, and many other job types. ]
			\end{enumerate}
		\end{enumerate}
	\item Jobs for the Future: \vspace{-0.2cm}
		\begin{enumerate} \itemsep -2pt
		\item \url{http://www.jff.org/}
		\item Current Projects: \url{http://www.jff.org/projects/current}
		\item Publications: \url{http://www.jff.org/publications}
		\item Policy: \url{http://www.jff.org/policy}
		\item Funders (funding agencies/organizations): \url{http://www.jff.org/funders}
		\item Programs: \url{http://www.jff.org/index.php?select=work}
		\end{enumerate}
	\item SkillsUSA: \vspace{-0.2cm}
		\begin{enumerate} \itemsep -2pt
		\item ``SkillsUSA is a partnership of students, teachers and industry working together to ensure America has a skilled work force. SkillsUSA helps each student excel.''
		\item Educators: \vspace{-0.1cm}
			\begin{enumerate} \itemsep -1pt
			\item \url{http://www.skillsusa.org/educators/index.shtml}
			\item Programs and Curricula: \url{http://www.skillsusa.org/educators/programs.shtml}
			\end{enumerate}
		\item Students: \vspace{-0.1cm}
			\begin{enumerate} \itemsep -1pt
			\item \url{http://www.skillsusa.org/students/index.shtml}
			\item Scholarships \& Financial Aid--SkillsUSA-related Scholarships: \url{http://www.skillsusa.org/students/scholarships.shtml}
			\end{enumerate}
		\item SkillsUSA competitions: \url{http://www.skillsusa.org/compete/index.shtml}
		\end{enumerate}
	\item others: \vspace{-0.2cm}
		\begin{enumerate} \itemsep -2pt
		\item public speaking and leadership: \vspace{-0.1cm}
			\begin{enumerate} \itemsep -1pt
			\item {\it Toastmasters International} is a non-profit educational organization that teaches public speaking and leadership skills through a worldwide network of meeting locations. Available online at: \url{http://www.toastmasters.org/}; last accessed on January 7, 2010.
			\end{enumerate}
		\end{enumerate}
	\end{enumerate}
\end{enumerate}




%%%%%%%%%%%%%%%%%%%%%%%%%%%%%%%%%%%%%%%%%%%
\section{Youth Outreach}
\label{youthoutreach}

Resources for youth outreach: \vspace{-0.3cm}
\begin{enumerate} \itemsep -4pt
%%%%%%%%%%%%%%%%%%%%%%%
\item educational (computer) games: \vspace{-0.3cm}
	\begin{enumerate} \itemsep -2pt
	\item Chevron Corporation: \vspace{-0.2cm}
		\begin{enumerate} \itemsep -2pt
		\item Energyville (about issues concerning energy and the environment): \url{http://www.willyoujoinus.com/energyville/}
		\end{enumerate}
	\item {\it Lego Digital Designer (LDD)}: \vspace{-0.2cm}
		\begin{enumerate} \itemsep -2pt
		\item CAD software for building Lego toys on Windows and Mac OS X platforms
		\item Free software, as in free beer
		\item \url{http://designbyme.lego.com/en-us/Default.aspx} and \url{http://ldd.lego.com/}
		\end{enumerate}
	\item Robocode: \vspace{-0.2cm}
		\begin{enumerate} \itemsep -2pt
		\item \url{http://en.wikipedia.org/wiki/Robocode} and \url{http://robocode.sourceforge.net/}
		\item Learn how to develop computer programs that will control a robot
		\end{enumerate}
	\item {\it Skill-Life}: \vspace{-0.2cm}
		\begin{enumerate} \itemsep -2pt
		\item \url{http://skill-life.com/}
		\item Use online games to teach youth life skills concerning financial literacy, nutrition, and citizenship.
		\end{enumerate}
	\item PowerUp (IBM with TryScience/New York Hall of Science): \vspace{-0.2cm}
		\begin{enumerate} \itemsep -2pt
		\item \url{http://www.powerupthegame.org/}
		\item Computer game to teach youths about energy conservation, global warming, renewable energy, and sustainable engineering
		\end{enumerate}
	\item EnergyNet: \vspace{-0.2cm}
		\begin{enumerate} \itemsep -2pt
		\item \url{http://www.energynet.net/games/}
		\item Computer game to teach youths about energy efficiency, and other topics related to energy
		\end{enumerate}
	\end{enumerate}
%%%%%%%%%%%%%%%%%%%%%%%
\item summer camps: \vspace{-0.3cm}
	\begin{enumerate} \itemsep -2pt
	\item United States Naval Academy: \vspace{-0.2cm}
		\begin{enumerate} \itemsep -2pt
		\item Naval Academy Athletic Association: \vspace{-0.1cm}
			\begin{enumerate} \itemsep -1pt
			\item Sports camps: \url{http://www.navysports.com/camps/navy-camps.html}
			\end{enumerate}
		\end{enumerate}
	\end{enumerate}
%%%%%%%%%%%%%%%%%%%%%%%
\item competitions for youths: \vspace{-0.3cm}
	\begin{enumerate} \itemsep -2pt
	\item International Geography Olympiad (for high school students): \url{http://www.geoolympiad.org/}
	\item International Linguistic Olympiad (for high school students): \url{http://en.wikipedia.org/wiki/International_Linguistics_Olympiad}
	\item International Philosophy Olympiad (for high school students): \url{http://www.philosophy-olympiad.org/}
	\item JA Worldwide: Responsible People Business Competition (for students in North and South America, and Europe), \url{http://www.responsible-business.org/}
	\item The Choral Arts Society of Washington: \vspace{-0.2cm}
		\begin{enumerate} \itemsep -2pt
		\item \url{http://www.choralarts.org/MLK-Celebration-Community-Initiative/Writing-Competition.aspx}
		\item ``As part of our MLK Celebration Community Initiative and in celebration of Black History Month, The Choral Arts Society of Washington hosts an annual writing competition for students in grades K-12.''
		\item ``Each year, students are presented with a different writing prompt and are asked to respond in poetic form.''
		\item ``Students are encouraged to be creative in their writing and to use their knowledge of Martin Luther King, Jr.'s life, the Civil Rights Movement, and current events as inspiration for their writing.''
		\end{enumerate}
	\item Vocal Arts DC (or Vocal Arts Society): \vspace{-0.2cm}
		\begin{enumerate} \itemsep -2pt
		\item Young Artists Competition: \vspace{-0.1cm}
			\begin{enumerate} \itemsep -1pt
			\item \url{http://vocalartsdc.org/youngartists.shtml}
			\item ``Each year, Vocal Arts DC holds a vocal competition open to all singers who are residents of the greater DC area, including Baltimore and Annapolis.''
			\item ``Singers are asked to submit a CD for review along with a sample recital program that the singer is prepared to sing in recital. The CDs will be reviewed in a blind audition and finalist will be selected for live auditions.''
			\item ``Two winners are selected from the finalists and are presented in the Art Song Discovery Series in four different venues across the greater DC area.''
			\end{enumerate}
		\end{enumerate}
	\item The John F. Kennedy Center for the Performing Arts: \vspace{-0.2cm}
		\begin{enumerate} \itemsep -2pt
		\item The National Symphony Orchestra (NSO): \vspace{-0.1cm}
			\begin{enumerate} \itemsep -1pt
			\item Young Soloists' Competition (High School Division; Washington metropolitan area): \url{http://www.kennedy-center.org/nso/nsoed/youngsoloists.cfm#concerts}
			\end{enumerate}
		\end{enumerate}
	\item Center for Interactive Learning and Collaboration (CILC): \vspace{-0.2cm}
		\begin{enumerate} \itemsep -2pt
		\item Kids Creating Community Content KC$^{3}$ International Contest (for students in Middle and High School): \vspace{-0.1cm}
			\begin{enumerate} \itemsep -1pt
			\item \url{http://kc3.cilc.org/} and \url{http://kc3.cilc.org/guidelines.htm}
			\item Make a short film to educate others about the uniqueness of your community, geographical region, natural/agricultural resources, local/national treasures, culture/heritage, or country.
			\end{enumerate}
		\end{enumerate}
	\end{enumerate}
%%%%%%%%%%%%%%%%%%%%%%%
\item educational resources: \vspace{-0.3cm}
	\begin{itemize} \itemsep -2pt
	\item Xcel Energy Foundation: \vspace{-0.2cm}
		\begin{enumerate} \itemsep -2pt
		\item Focus Area Grants: \vspace{-0.1cm}
			\begin{enumerate} \itemsep -1pt
			\item \url{http://www.xcelenergy.com/Minnesota/Company/Community/Xcel%20Energy%20Foundation/Pages/Focus_Area_Grants.aspx}
			\item Scope of eligible funding, and details on the grant application process
			\end{enumerate}
		\item Education Initiatives: \vspace{-0.1cm}
			\begin{enumerate} \itemsep -1pt
			\item \url{http://www.xcelenergy.com/Minnesota/Company/Community/Education%20Initiatives/Pages/Education_Initiatives.aspx}
			\item Energy Safety Calendar Program, K-6: \vspace{-0.1cm}
				\begin{itemize} \itemsep -1pt
				\item \url{http://www.xcelenergy.com/New%20Mexico/Company/Community/Education%20Initiatives/Pages/Energy_Safety_Calendar_ProgramK-6.aspx}
				\item ``The Energy Safety Calendar Program offers K-6 students in our service territory a great opportunity to learn about electricity and natural gas safety.''
				\end{itemize}
			\end{enumerate}
		\item Safety World: \vspace{-0.1cm}
			\begin{enumerate} \itemsep -1pt
			\item \url{http://www.xcelenergy.com/New%20Mexico/Company/Community/Education%20Initiatives/Pages/Safety_World.aspx}
			\item e-SMART kid: \vspace{-0.1cm}
				\begin{itemize} \itemsep -1pt
				\item \url{http://www.e-smartonline.net/xcelenergy/}
				\item Help children and youth learn about ``electricity and natural gas and how to use them safely''
				\end{itemize}
			\end{enumerate}
		\item Energy Classroom: \vspace{-0.1cm}
			\begin{enumerate} \itemsep -1pt
			\item \url{http://www.energyclassroom.com/}
			\item \url{http://www.xcelenergy.com/Minnesota/Company/Community/Pages/Energy_Classroom.aspx}
			\item Educational material for students about energy sources, energy conservation, and environmental protection
			\item For Teachers (educational material and suggested class activities): \url{http://www.energyclassroom.com/index.php?id=34&page=For_Teachers}
			\end{enumerate}
		\item Power Plant Tour Information: \url{http://www.xcelenergy.com/New%20Mexico/Company/About_Energy_and_Rates/Power%20Generation/Pages/Power_Plant_Tour_Information.aspx}
		\end{enumerate}
	\item HowStuffWorks, Inc.: \url{http://www.howstuffworks.com/}
	\item Chevron Corporation: \vspace{-0.2cm}
		\begin{enumerate} \itemsep -2pt
		\item {\it Will you join us}: \vspace{-0.1cm}
			\begin{enumerate} \itemsep -1pt
			\item Energy issues: \url{http://www.willyoujoinus.com/energyissues/}
			\item Tools and resources: \vspace{-0.1cm}
				\begin{itemize} \itemsep -1pt
				\item \url{http://www.willyoujoinus.com/toolsresources/}
				\item Helpful links (includes K-12 educational material): \url{http://www.willyoujoinus.com/toolsresources/helpfullinks/}
				\end{itemize}
			\item MPG Optimizer: \url{http://www.willyoujoinus.com/usingenergywisely/mpgoptimizer/}
			\item Energy generator: \url{http://www.willyoujoinus.com/usingenergywisely/energygenerator/}
			\end{enumerate}
		\end{enumerate}
	\item National Energy Foundation: \vspace{-0.2cm}
		\begin{enumerate} \itemsep -2pt
		\item \url{http://www.nef.org.uk/} and \url{http://www.nef1.org/}
		\item Students: \url{http://www.nef1.org/students.html}
		\item Educators: \url{http://www.nef1.org/educators.html}
		\item Schools: \vspace{-0.1cm}
			\begin{enumerate} \itemsep -1pt
			\item \url{http://www.nef.org.uk/communities/schools/index.html}
			\item Helpful links: \url{http://www.nef.org.uk/communities/schools/energylinks.html}
			\item School Resources: \url{http://www.nef.org.uk/communities/schools/resources/index.html}
			\item {\it LogiCity} is a fun interactive computer game with a difference. It's a game set in a 3D virtual city with five main activities where you are set the task of reducing the carbon footprint of an average resident. See \url{http://www.nef.org.uk/communities/schools/logicity.html}.
			\end{enumerate}
		\item Resources: \url{http://www.nef.org.uk/actonCO2/index.asp}
		\item Igniting Creative Energy - A National Student Challenge: \vspace{-0.1cm}
			\begin{enumerate} \itemsep -1pt
			\item \url{http://www.ignitingcreativeenergy.org/}
			\item Students: \url{http://www.ignitingcreativeenergy.org/students.html}
			\end{enumerate}
		\end{enumerate}
	\item StartSpot Mediaworks: \vspace{-0.2cm}
		\begin{enumerate} \itemsep -2pt
		\item StartSpot Network: \vspace{-0.1cm}
			\begin{enumerate} \itemsep -1pt
			\item HomeworkSpot: \vspace{-0.1cm}
				\begin{itemize} \itemsep -1pt
				\item \url{http://www.homeworkspot.com/}
				\item Science Fair Project Center: \url{http://www.homeworkspot.com/sciencefair/}
				\end{itemize}
			\end{enumerate}
		\end{enumerate}
	\item Super Science Fair Projects: \url{http://www.super-science-fair-projects.com/}
	\item All Science Fair Projects: Science Fair Projects with Complete Instructions, \url{http://www.all-science-fair-projects.com/}
	\item The Science Club: \vspace{-0.2cm}
		\begin{enumerate} \itemsep -2pt
		\item \url{http://scienceclub.org/}
		\item Science Fair Idea Exchange: \url{http://scienceclub.org/scifair.html}
		\end{enumerate}
	\item Oracle Education Foundation: \vspace{-0.2cm}
		\begin{enumerate} \itemsep -2pt
		\item \url{http://www.oraclefoundation.org/}
		\item ThinkQuest: \vspace{-0.1cm}
			\begin{enumerate} \itemsep -1pt
			\item \url{http://www.thinkquest.org/en/}
			\item ThinkQuest International Competition: \url{http://www.thinkquest.org/competition/}
			\item Projects: \url{http://thinkquest.org/en/projects/index.html}
			\item Library: \url{http://thinkquest.org/pls/html/think.library}
			\item Example of a computer game developed by students: Crisis! - The Game, \url{http://library.thinkquest.org/20331/game/}
			\end{enumerate}
		\end{enumerate}
	\item University of Minnesota: \vspace{-0.2cm}
		\begin{enumerate} \itemsep -2pt
		\item Institute on Community Integration; College of Education and Human Development: \vspace{-0.1cm}
			\begin{enumerate} \itemsep -1pt
			\item National Center on Secondary Education and Transition (NCSET): \vspace{-0.1cm}
				\begin{itemize} \itemsep -1pt
				\item \url{http://www.ncset.org/}
				\item NCSET Topics: \url{http://www.ncset.org/topics/default.asp}
				\item Web Sites: \url{http://www.ncset.org/websites/default.asp}
				\item The Youthhood!: \url{http://www.youthhood.org/}
				\end{itemize}
			\end{enumerate}
		\end{enumerate}
	\item Jobs for America's Graduates: \vspace{-0.2cm}
		\begin{enumerate} \itemsep -2pt
		\item \url{http://www.jag.org/}
		\item JAG Model program applications: \vspace{-0.1cm}
			\begin{enumerate} \itemsep -1pt
			\item \url{http://www.jag.org/model.htm}
			\item Programs are available for students in middle school and high school, high school dropouts, high school seniors, students in alternative education programs, and college underclassmen
			\end{enumerate}
		\item JAG Career Corner: \url{http://www.jag.org/jag_career_corner.htm}
		\item Students: \url{http://www.jag.org/students.htm}
		\item Resource library: \url{http://www.jag.org/library.htm}
		\item Performance outcomes: \url{http://www.jag.org/outcomes.htm}
		\item Funding: \url{http://www.jag.org/funding.htm}
		\end{enumerate}
	\item Alliance to Save Energy: \vspace{-0.2cm}
		\begin{enumerate} \itemsep -2pt
		\item Energy Hog campaign: \vspace{-0.1cm}
			\begin{enumerate} \itemsep -1pt
			\item \url{http://www.energyhog.org/}
			\item Adults: \url{http://www.energyhog.org/adult/adults.htm}
			\item Children: \url{http://www.energyhog.org/childrens.htm}
			\end{enumerate}
		\end{enumerate}
	\item Learning First Alliance: \vspace{-0.2cm}
		\begin{enumerate} \itemsep -2pt
		\item \url{http://www.learningfirst.org/}
		\item Issues and publications: \url{http://www.learningfirst.org/issues}
		\item Resources: \url{http://www.learningfirst.org/resources}
		\end{enumerate}
	\item NaMaYa: \url{http://www.namaya.com/}
	\item NIXTY: \url{http://nixty.com/}
	\item K12 Open Ed: \url{http://www.k12opened.com/wiki/index.php/Main_Page}
	\item Learning Is For Everyone: \url{http://www.learningis4everyone.org/}
	\item The Smithsonian Commons Prototype: \url{http://www.si.edu/commons/prototype/}
	\item Futurelab: Resources for educators and parents, \url{http://www.futurelab.org.uk/resources}
	\item Innosight Institute: Resources for education, \url{http://www.innosightinstitute.org/practices/education/}
	\item WGBH Educational Foundation: \url{http://www.wgbh.org/}
	\item Discovery Education: \vspace{-0.2cm}
		\begin{enumerate} \itemsep -2pt
		\item Classroom resources: \url{http://school.discoveryeducation.com/}
		\item Home resources: \url{http://school.discoveryeducation.com/homeworkhelp/homework_help_home.html}
		\end{enumerate}
	\item The Gilder Lehrman Institute of American History: \vspace{-0.2cm}
		\begin{enumerate} \itemsep -2pt
		\item \url{http://www.gilderlehrman.org/}
		\item Resources for teachers and schools: \url{http://www.gilderlehrman.org/teachers/}
		\item Civil War Essay Contest (for students in selected K-12 schools): \url{http://www.gilderlehrman.org/affiliate/civil_war.php}
		\end{enumerate}
	\item The GRAMMY Museum: \vspace{-0.2cm}
		\begin{enumerate} \itemsep -2pt
		\item Teacher curriculum and resources. Available online at: \url{http://www.grammymuseum.org/interior.php?section=education&page=teachercurriculum}; last accessed on November 15, 2010.
		\end{enumerate}
	\item Purdue University: \vspace{-0.2cm}
		\begin{enumerate} \itemsep -2pt
		\item Department of Entomology: \vspace{-0.1cm}
			\begin{enumerate} \itemsep -1pt
			\item Genomics Analogy Model for Educators (G.A.M.E.): \url{http://www.entm.purdue.edu/extensiongenomics/GAME/default.html}
			\end{enumerate}
		\end{enumerate}
	\item Verizon Thinkfinity: \url{http://www.thinkfinity.org/about-us}
	\item Oregon Virtual School District (ORVSD): \vspace{-0.2cm}
		\begin{enumerate} \itemsep -2pt
		\item \url{http://orvsd.org/}
		\item ``Oregon Virtual School District (ORVSD) helps integrate technology into Oregon public school classrooms by giving teachers access to free tech tools and resources online.''
		\item ``The Oregon Virtual School District is a program led by the Oregon Department of Education that, in cooperation with a consortium of virtual learning providers throughout the state, seeks to increase access and availability of online learning and teaching resources free of charge to public school teachers of Oregon. Oregon State University is providing hosting and development resources through a partnership with the OSU Open Source Lab and the OSU Business Solutions Group.''
		\end{enumerate}
	\item The Association of Educational Publishers (AEP): \vspace{-0.2cm}
		\begin{enumerate} \itemsep -2pt
		\item The AEP Awards: \vspace{-0.1cm}
			\begin{enumerate} \itemsep -1pt
			\item \url{http://www.aepweb.org/awards/index.htm}
			\item Look at the winners of previous AEP awards to determine some of the good educational resources that are available
			\end{enumerate}
		\end{enumerate}
	\item Educational Dividends: \vspace{-0.2cm}
		\begin{enumerate} \itemsep -2pt
		\item \url{http://www.educationaldividends.com/}
		\item Teachers: \vspace{-0.1cm}
			\begin{enumerate} \itemsep -1pt
			\item \url{http://www.educationaldividends.com/index.asp?menu=Teachers}
			\item Teaching Tools: \url{http://www.educationaldividends.com/teachers/tools.asp}
			\item Reference Desk: \vspace{-0.1cm}
				\begin{itemize} \itemsep -1pt
				\item \url{http://www.educationaldividends.com/teachers/reference.asp}
				\item Standards Reference Desk (resources for education standards in the US at the national, state, and local levels): \url{http://www.educationaldividends.com/teachers/standards_desk.asp}
				\item How We Learn: Learning Styles, \url{http://www.educationaldividends.com/teachers/learning_styles.asp}
				\item How We Learn: Multiple Intelligences, \url{http://www.educationaldividends.com/teachers/multiple_intelligences.asp}
				\item Statistics Desk (statistical information about education in the US): \url{http://www.educationaldividends.com/teachers/statistics_desk.asp}
				\end{itemize}
			\item Information about the teaching profession: \vspace{-0.1cm}
				\begin{itemize} \itemsep -1pt
				\item \url{http://www.educationaldividends.com/teachers/welcome.asp}
				\item Office of Occupational Statistics and Employment Projections, ``Educational Services,'' in {\it Career Guide to Industries}, 2010-11 Edition, U.S. Bureau of Labor Statistics, U.S. Department of Labor, Washington, DC, December 17, 2009. Available online at: \url{http://stats.bls.gov/oco/cg/cgs034.htm}; last accessed on December 8, 2010. [ Suggested citation: Bureau of Labor Statistics, U.S. Department of Labor, {\it Career Guide to Industries, 2010-11 Edition}, Educational Services , on the Internet at \url{http://www.bls.gov/oco/cg/cgs034.htm} (visited December 07, 2010). ]
				\item Experience Teaching: \url{http://www.educationaldividends.com/teachers/experience.asp}
				\item Continuous Improvement: \url{http://www.educationaldividends.com/teachers/toolkit.asp}
				\end{itemize}
			\end{enumerate}
		\item Personality and Career Tests: \url{http://www.educationaldividends.com/teachers/tests.asp}
		\end{enumerate}
	\item Smithsonian Institution: \vspace{-0.2cm}
		\begin{enumerate} \itemsep -2pt
		\item Educators: \url{http://www.si.edu/Educators}
		\item Smithsonian Institution Traveling Exhibition Service (SITES): \vspace{-0.1cm}
			\begin{enumerate} \itemsep -1pt
			\item For Teachers: \url{http://www.sites.si.edu/education/teachers_res2.htm}
			\end{enumerate}
		\item Smithsonian Folkways Recordings (or simply, Smithsonian Folkways): \vspace{-0.1cm}
			\begin{enumerate} \itemsep -1pt
			\item Tools for Teaching: \url{http://www.folkways.si.edu/tools_for_teaching/introduction.aspx}
			\end{enumerate}
		\item Freer Gallery of Art / Arthur M. Sackler Gallery: \vspace{-0.1cm}
			\begin{enumerate} \itemsep -1pt
			\item Resources for Educators: \url{http://www.asia.si.edu/explore/teacherResources.asp}
			\item Explore + Learn: Browse Online Resources by Area: \vspace{-0.1cm}
				\begin{itemize} \itemsep -1pt
				\item \url{http://www.asia.si.edu/explore/default.asp}
				\item Has resources for art in: \vspace{-0.1cm}
					\begin{itemize} \itemsep -1pt
					\item The Americas
					\item Ancient Egypt
					\item Ancient Near East
					\item Islamic world
					\item China
					\item Japan
					\item Korea
					\item South Asia
					\item Himalayas
					\item Southeast Asian
					\item It also has biblical manuscripts and contemporary art
					\end{itemize}
				\end{itemize}
			\item Online Exhibition Features: \url{http://www.asia.si.edu/exhibitions/online.asp}
			\item Collections: \url{http://www.asia.si.edu/collections/default.asp}
			\end{enumerate}
		\item National Museum of American History: \vspace{-0.1cm}
			\begin{enumerate} \itemsep -1pt
			\item Jerome and Dorothy Lemelson Center for the Study of Invention and Innovation: \vspace{-0.1cm}
				\begin{itemize} \itemsep -1pt
				\item Resources: \vspace{-0.1cm}
					\begin{itemize} \itemsep -1pt
					\item \url{http://invention.smithsonian.org/resources/}
					\item \url{http://invention.smithsonian.org/resources/default_sites_weblinks.aspx}
					\item Invention stories - archives, articles, audio, and video: \url{http://invention.smithsonian.org/resources/default_index.aspx}
					\end{itemize}
				\item Educational Materials: \vspace{-0.1cm}
					\begin{itemize} \itemsep -1pt
					\item \url{http://invention.smithsonian.org/resources/menu_edu_materials.aspx}
					\item Experiments: \url{http://invention.smithsonian.org/resources/menu_edu_materials.aspx?MaterialTypeID=3&MaterialTypeDesc=Experiments}
					\item Educational Materials: \url{http://invention.smithsonian.org/resources/menu_edu_materials_f.aspx?MaterialTypeDesc=Features}
					\end{itemize}
				\item Centerpieces: \vspace{-0.1cm}
					\begin{itemize} \itemsep -1pt
					\item \url{http://invention.smithsonian.org/centerpieces/}
					\item \url{http://invention.smithsonian.org/centerpieces/iap-info.aspx}
					\item Electric guitar: \url{http://invention.smithsonian.org/centerpieces/electricguitar/index.htm}
					\item Innovative Lives: \url{http://invention.smithsonian.org/centerpieces/ilives/}
					\item ``Exploring the History of Women Inventors'' by J.E. Bedi (in {\it Innovative Lives}): \url{http://invention.smithsonian.org/centerpieces/ilives/womeninventors.html}
					\item Whole Cloth: \url{http://invention.smithsonian.org/centerpieces/whole_cloth/index.html}
					\item The Quartz Watch: \url{http://invention.smithsonian.org/centerpieces/quartz/index.html}
					\item Edison Invents!: All about Thomas Edison and his invention, \url{http://invention.smithsonian.org/centerpieces/edison/default.asp}
					\end{itemize}
				\item Modern Inventors Documentation Program (MIND): \url{http://invention.smithsonian.org/resources/mind_resources.aspx}
				\item Invention at Play: \vspace{-0.1cm}
					\begin{itemize} \itemsep -1pt
					\item \url{http://inventionatplay.org/}
					\item Resources: \url{http://inventionatplay.org/resources.html}
					\item Invention Playhouse: \url{http://inventionatplay.org/playhouse_main.html}
					\item Inventors' Stories: \url{http://inventionatplay.org/inventors_main.html}
					\item Does play matter? (using play to help children learn and think): \url{http://inventionatplay.org/matter_main.html}
					\end{itemize}
				\item Spark!Lab: \vspace{-0.1cm}
					\begin{itemize} \itemsep -1pt
					\item \url{http://sparklab.si.edu/}
					\item About Spark!Lab (introduce children to the process of innovation via play and fun activities): \url{http://sparklab.si.edu/spark-about.html}
					\item Activities \& Experiments: \url{http://sparklab.si.edu/spark-experiments.html}
					\item Inventor Profiles: \url{http://sparklab.si.edu/spark-inventors.html}
					\item Resources: \url{http://sparklab.si.edu/spark-resources.html}
					\end{itemize}
				\end{itemize}
			\end{enumerate}
		\end{enumerate}
	\item Economic and Social Research Council (ESRC): \vspace{-0.2cm}
		\begin{enumerate} \itemsep -2pt
		\item {\it Social Science for Schools}; Science in Society Team: \vspace{-0.1cm}
			\begin{enumerate} \itemsep -1pt
			\item \url{http://www.esrcsocietytoday.ac.uk/ESRCInfoCentre/ssfs/}
			\item Social science resources: \url{http://www.esrcsocietytoday.ac.uk/ESRCInfoCentre/ssfs/resources/}
			\item Career guides for different disciplines in social science and economics: \url{http://www.esrcsocietytoday.ac.uk/ESRCInfoCentre/ssfs/careers/}
			\item Related online resources: \url{http://www.esrcsocietytoday.ac.uk/ESRCInfoCentre/ssfs/links/}
			\end{enumerate}
		\end{enumerate}
	\end{itemize}
\item National Council for Accreditation of Teacher Education (NCATE): \vspace{-0.3cm}
	\begin{enumerate} \itemsep -2pt
	\item \url{http://www.ncate.org/}
	\item Has resources about degree programs in education and their accreditation, as well as how to become a teacher
	\item State-specific Recognized Programs by NCATE and Specialized Professional Associations (SPAs): \vspace{-0.2cm}
		\begin{enumerate} \itemsep -2pt
		\item \url{http://www.ncate.org/tabid/165/Default.aspx}
		\item Find out about educational programs in: \vspace{-0.1cm}
			\begin{enumerate} \itemsep -1pt
			\item special education
			\item early childhood education
			\item educational leadership
			\item educational technology specialist
			\item elementary education
			\item English
			\item health education
			\item foreign languages
			\item gifted education
			\item mathematics
			\item physical education
			\item science education
			\item school psychology
			\item secondary computer science education
			\item social studies
			\item Teachers of English to Speakers of Other Languages (TESOL)
			\item technology and engineering educators
			\end{enumerate}
		\end{enumerate}
	\item Financial Aid Resources for Teacher Education Students: \url{http://www.ncate.org/Public/CurrentFutureTeachers/FinancialAidResources/tabid/351/Default.aspx}
	\end{enumerate}
%%%%%%%%%%%%%%%%%%%%%%%
\item scholarships: \vspace{-0.3cm}
	\begin{enumerate} \itemsep -2pt
	\item U.S. Department of State: \vspace{-0.2cm}
		\begin{enumerate} \itemsep -2pt
		\item Bureau of Educational and Cultural Affairs: \vspace{-0.1cm}
			\begin{enumerate} \itemsep -1pt
			\item National Security Language Initiative for Youth (NSLI-Y): \vspace{-0.1cm}
				\begin{itemize} \itemsep -1pt
				\item \url{http://exchanges.state.gov/youth/programs/nsli.html}
				\item ``The State Department�s National Security Language Initiative for Youth (NSLI-Y) provides merit-based scholarships to U.S. high school students and recent graduates interested in learning less-commonly studied foreign languages.''
				\end{itemize}
			\end{enumerate}
		\end{enumerate}
	\end{enumerate}
%%%%%%%%%%%%%%%%%%%%%%%
\item underrepresented minorities: \vspace{-0.3cm}
	\begin{enumerate} \itemsep -2pt
	\item The University of North Carolina at Chapel Hill: \vspace{-0.2cm}
		\begin{enumerate} \itemsep -2pt
		\item Gary Bishop, {\it Research}, Department of Computer Science, The University of North Carolina at Chapel Hill. Available at: \url{http://wwwx.cs.unc.edu/~gb/wp/research/}; last accessed on September 3, 2010. [ Has plenty of information and resources (including learning aids and material) to help people who are visually or mobility impaired learn. ]
		\end{enumerate}
	\item Myra Sadker Foundation: \vspace{-0.2cm}
		\begin{enumerate} \itemsep -2pt
		\item $100+$ Ideas to Promote Gender Equity in Schools and Beyond: \url{http://www.sadker.org/100ideas.html}
		\item Gender Equity Activities: \url{http://www.sadker.org/WhatYouCanDo.html}
		\item Gender Equity Activities for Concerned Citizens: \url{http://www.sadker.org/GenderEquity-citizens.html}
		\item Gender Equity Activities for Families: \url{http://www.sadker.org/GenderEquity-family.html}
		\item Gender Equity Activities for Teachers: \vspace{-0.1cm}
			\begin{enumerate} \itemsep -1pt
			\item Early Childhood: \url{http://www.sadker.org/GenderEquity-teacher1.html}
			\item Primary Grades: \url{http://www.sadker.org/GenderEquity-teacher2.html}
			\item Upper Elementary: \url{http://www.sadker.org/GenderEquity-teacher3.html}
			\item Middle and High School: \url{http://www.sadker.org/GenderEquity-teacher4.html}
			\end{enumerate}
		\item Resources for feminism and links to web pages of feminist organizations: \url{http://www.sadker.org/ReadsLinks.html}
		\end{enumerate}
	\item League of United Latin American Citizens (LULAC): \vspace{-0.3cm}
		\begin{enumerate} \itemsep -2pt
		\item LULAC National Educational Service Centers, Inc: \vspace{-0.2cm}
			\begin{enumerate} \itemsep -2pt
			\item \url{http://www.lnesc.org/}
			\item Programs: \vspace{-0.1cm}
				\begin{itemize} \itemsep -1pt
				\item Improving literacy among Latino/Latina youth
				\item Encouraging Latino/Latina youth to pursue careers in science and engineering
				\item Helping Latino/Latina youth acquire leadership skills
				\item Improving college access for Latino/Latina youth by mentoring and summer programs (e.g., Gear-Up, Upward Bound, and the PALMS Initiative)
				\item Helping Latino/Latina families acquire financial success, so that Latino/Latina youth can pursue higher education
				\item Scholarships for Latino/Latina youth
				\item \url{http://lnesc.org/index.asp?Type=B_BASIC&SEC={808B6D04-913C-483F-8A05-5BD44B03ED62}}
				\end{itemize}
			\end{enumerate}
		\end{enumerate}
	\item ASPIRA: \vspace{-0.2cm}
		\begin{enumerate} \itemsep -2pt
		\item ASPIRA Programs for Latino/Latina youth: \url{http://aspira.org/manuals/aspira-programs}
		\end{enumerate}
	\end{enumerate}
%%%%%%%%%%%%%%%%%%%%%%%
\item places to visit: \vspace{-0.3cm}
	\begin{enumerate} \itemsep -2pt
	\item Exploratorium @ The Palace of Fine Arts (San Francisco, CA): \url{http://www.exploratorium.edu/}
	\item Educational Dividends: \vspace{-0.2cm}
		\begin{enumerate} \itemsep -2pt
		\item \url{http://www.educationaldividends.com/}
		\item Suggestions for organizing field trips to explore your interests: \url{http://www.educationaldividends.com/students/student_issues.asp}
		\item Career exploration: \url{http://www.educationaldividends.com/students/career_choices.asp}
		\item Computer skills: \url{http://www.educationaldividends.com/students/technology.asp}
		\item Quizzes to help you find out what is your preferred learning style and to discover more about your personality: \url{http://www.educationaldividends.com/students/learning_quiz.asp}
		\item Resources to help you learn about various topics in science, mathematics, social science, and humanities: \url{http://www.educationaldividends.com/students/resources.asp}
		\end{enumerate}
	\end{enumerate}
%%%%%%%%%%%%%%%%%%%%%%%
\item resources for at-risk youths: \vspace{-0.3cm}
	\begin{enumerate} \itemsep -2pt
	\item At-Risk Youth: \url{http://www.at-risk.org/}
	\item Peace First: \vspace{-0.2cm}
		\begin{enumerate} \itemsep -2pt
		\item \url{http://www.peacefirst.org/site/}
		\item To help youths become ``problem-solvers, rather than witnesses, or victims of their surrounding''
		\item To reduce youth homicide rates
		\item Teach children ``critical conflict resolution skills''
		\item Help teachers improve their ``conflict resolution and classroom management skills''
		\item To encourage youths to help each other, and get them to break up fights
		\item ``The Peace First curriculum is tailored to meet the developmental needs of students in Pre-K through eighth grade. Once a week, young adult volunteers and classroom teachers work together to teach students about friendship, communication, and conflict resolution through the use of experiential activities. First graders learn about communicating their feelings, third graders work on being peacemakers in their classroom, and fifth graders explore how to resolve and deescalate conflicts.''
		\item Has programs for students/youths, teachers, principals, and volunteers.
		\end{enumerate}
	\item Americans for the Arts: \vspace{-0.2cm}
		\begin{enumerate} \itemsep -2pt
		\item YouthARTS: \vspace{-0.1cm}
			\begin{enumerate} \itemsep -1pt
			\item \url{http://www.artsusa.org/youtharts/index.asp}
			\item ``The YouthARTS site is designed to give arts agencies, juvenile justice agencies, social service organizations, and other community-based organizations detailed information about how to plan, run, provide training, and evaluate arts programs for at-risk youth.''
			\end{enumerate}
		\end{enumerate}
	\end{enumerate}
%%%%%%%%%%%%%%%%%%%%%%%
\item general music and arts education: \vspace{-0.3cm}
	\begin{enumerate} \itemsep -2pt
	\item Americans for the Arts: \vspace{-0.2cm}
		\begin{enumerate} \itemsep -2pt
		\item Americans for the Arts, ``Ten Simple Ways Parents Can Get More Art in Their Kids' Lives.'' Available online at: \url{http://www.americansforthearts.org/public_awareness/get_involved/001.asp}; last accessed on November 30, 2010.
		\item YouthARTS: \vspace{-0.1cm}
			\begin{enumerate} \itemsep -1pt
			\item \url{http://www.artsusa.org/youtharts/index.asp}
			\item ``The YouthARTS site is designed to give arts agencies, juvenile justice agencies, social service organizations, and other community-based organizations detailed information about how to plan, run, provide training, and evaluate arts programs for at-risk youth.''
			\end{enumerate}
		\end{enumerate}
	\item The John F. Kennedy Center for the Performing Arts: \vspace{-0.2cm}
		\begin{enumerate} \itemsep -2pt
		\item Kennedy Center Institute for Arts Management: \url{http://artsmanagerfba.artsmanager.org/common/Pages/About.aspx}
		\item {\sc ArtsEdge}: \vspace{-0.1cm}
			\begin{enumerate} \itemsep -1pt
			\item The National Standards for Arts Education for Grades K-4, 5-8, and 9-12: \url{http://artsedge.kennedy-center.org/educators/standards.aspx}
			\item Tips and guides for educators: \url{http://artsedge.kennedy-center.org/educators/how-to.aspx}
			\item Lesson plans for educators: \url{http://artsedge.kennedy-center.org/educators/lessons.aspx}
			\item Information for parents, guardians, foster parents, baby-sitters, and grandparents: \url{http://artsedge.kennedy-center.org/families.aspx}
			\item Information for students: \url{http://artsedge.kennedy-center.org/students.aspx}
			\item Themes for artistic, cultural, academic, and intellectual exploration: \url{http://artsedge.kennedy-center.org/themes.aspx}
			\item Multimedia: \url{http://artsedge.kennedy-center.org/multimedia.aspx}
			\end{enumerate}
		\end{enumerate}
	\end{enumerate}
\item music education: \vspace{-0.3cm}
	\begin{enumerate} \itemsep -2pt
	\item Washington Performing Arts Society (WPAS): \vspace{-0.2cm}
		\begin{enumerate} \itemsep -2pt
		\item WPAS Education \& Community -- Connections through the Arts Education Programs for All Ages: \vspace{-0.1cm}
			\begin{enumerate} \itemsep -1pt
			\item The Capitol Jazz Project: \vspace{-0.1cm}
				\begin{itemize} \itemsep -1pt
				\item \url{http://www.wpas.org/educcomm/programsforyoungpeople/capitoljazzproject.aspx}
				\item ``Washington Performing Arts Society (WPAS) and the D.C. Public Schools, in collaboration with Jazz at Lincoln Center, has launched The Capitol Jazz Project, an important step in supporting music education for all students in the District of Colombia.''
				\item ``Through the Capitol Jazz Project, students hone their listening, performing, improvising, composing, arranging, music reading, and notation skills.''
				\item ``The Capitol Jazz Project is being implemented in 6 D.C. middle schools with a total enrollment of more than 500 music students.''
				\item ``A true collaboration, The Capitol Jazz Project brings the combined resources and expertise of WPAS, Jazz at Lincoln Center, and the D.C. Public Schools to create a model music education program.''
				\end{itemize}
			\item Joseph and Goldie Feder Memorial String Competition: \vspace{-0.1cm}
				\begin{itemize} \itemsep -1pt
				\item \url{http://www.wpas.org/educcomm/programsforyoungpeople/josephandgoldiefedermemorialstringcompetition.aspx}
				\item ``The Feder String Competition inspires and nurtures D.C. area youth in grades 6 through 12 who study violin, viola, cello, and double bass.''
				\item ``Each year, 80 students compete for 30 awards and scholarships.''
				\item ``Held each spring, WPAS awards cash prizes toward private lessons, scholarships for summer study programs, and tickets for top winners and their family members to attend a WPAS concert.''
				\item ``Winners of the competition are also given special performance opportunities such as on the Kennedy Center's Millennium Stage and The Shakespeare Theatre Company's Happenings at the Harman series.''
				\end{itemize}
			\item WPAS Summer Performing Arts Academy summer programs: \vspace{-0.1cm}
				\begin{itemize} \itemsep -1pt
				\item \url{http://www.wpas.org/educcomm/programsforyoungpeople/wpassummerperformingartsacademy.aspx}
				\end{itemize}
			\end{enumerate}
		\end{enumerate}
	\item Young Concert Artists, Inc. \vspace{-0.2cm}
		\begin{enumerate} \itemsep -2pt
		\item Annaliese Soros Educational Residency Program: \url{http://www.yca.org/auditions/}
		\end{enumerate}
	\item The Choral Arts Society of Washington: \vspace{-0.2cm}
		\begin{enumerate} \itemsep -2pt
		\item Classroom Resources: \url{http://www.choralarts.org/Education/Classroom-Resources.aspx}
		\end{enumerate}
	\item League of American Orchestras: \vspace{-0.2cm}
		\begin{enumerate} \itemsep -2pt
		\item Career planning: \vspace{-0.1cm}
			\begin{enumerate} \itemsep -1pt
			\item Resources for pre-college students, college students, and graduate students: \url{http://www.americanorchestras.org/career_center/career_planning.html}
			\item Arts Administration programs: \url{http://www.americanorchestras.org/career_center/arts_admin_programs.html}
			\item Non-profit management, {\bf public policy} and leadership programs: \url{http://www.americanorchestras.org/career_center/resources_non_prof_and.html}
			\end{enumerate}
		\end{enumerate}
	\item The John F. Kennedy Center for the Performing Arts: \vspace{-0.2cm}
		\begin{enumerate} \itemsep -2pt
		\item Betty Carter's Jazz Ahead: \vspace{-0.1cm}
			\begin{enumerate} \itemsep -1pt
			\item \url{http://www.kennedy-center.org/programs/jazz/jazzahead/}
			\item ``Music residency program for young people''
			\item ``The Jazz Ahead program identifies outstanding, emerging jazz artists in their mid-teens to age thirty, and brings them together under the tutelage of experienced artist-instructors who coach and counsel them, helping to polish their performance, composing and arranging skills.''
			\item ``The two week-long residency program includes daily workshops and rehearsals with established jazz artists, and culminate in three concerts on the Kennedy Center Millennium Stage, which will be broadcast live over the internet.''
			\end{enumerate}
		\item The National Symphony Orchestra (NSO): \vspace{-0.1cm}
			\begin{enumerate} \itemsep -1pt
			\item The National Symphony Orchestra's Summer Music Institute (SMI): \vspace{-0.1cm}
				\begin{itemize} \itemsep -1pt
				\item \url{http://www.kennedy-center.org/nso/nsoed/smi/home.cfm}
				\item ``Every summer, approximately 70 students (ages 15-20) from all over the nation meet in Washington, D.C., to attend the National Symphony Orchestra's Summer Music Institute (SMI).''
				\item ``The Institute offers four weeks of private lessons, rehearsals, coaching by National Symphony Orchestra members, classes, and lectures to prepare aspiring musicians for their futures in music.''
				\end{itemize}
			\item Young Associates' Program: \vspace{-0.1cm}
				\begin{itemize} \itemsep -1pt
				\item \url{http://www.kennedy-center.org/nso/nsoed/youngassociates.html}
				\item ``The National Symphony Orchestra (NSO) is sponsoring its Young Associates' Program for high school students in grades 11 and 12 in the Washington, DC, metropolitan area who are interested in pursuing a musical career.''
				\item ``Twenty outstanding instrumentalists (pianists are not included) will be selected to attend rehearsals of the National Symphony Orchestra and take part in seminars with conductors, artists, NSO musicians, and representatives of the arts management field.''
				\item ``Through this program, the Young Associates will acquire an appreciation of the wide range of skills, knowledge, and abilities--managerial as well as musical--that are required to put together a performance by a major symphony orchestra. Selection process is by application.''
				\item ``The core of the program involves attendance at rehearsals of the National Symphony Orchestra at the Kennedy Center and observation of various guest artists. In addition to attending NSO rehearsals, students participate in workshops to explore careers in management, music education, publicity, music library, and other professions that are essential to the life of every successful orchestra.''
				\item ``Students do not play their instruments as part of the program. Students learn through listening, observation, and asking questions of professionals.''
				\end{itemize}
			\end{enumerate}
		\end{enumerate}
	\end{enumerate}
\item dance education: \vspace{-0.3cm}
	\begin{enumerate} \itemsep -2pt
	\item The Washington Ballet: \vspace{-0.2cm}
		\begin{enumerate} \itemsep -2pt
		\item The Washington School of Ballet (TWSB): \vspace{-0.1cm}
			\begin{enumerate} \itemsep -1pt
			\item Summer Intensive program (requires an audition): \url{http://www.washingtonballet.org/the-school/summer-intensive/}
			\end{enumerate}
		\item TWB's EXCEL! scholarship program (for DanceDC students): \vspace{-0.1cm}
			\begin{enumerate} \itemsep -1pt
			\item \url{http://www.washingtonballet.org/community-engagement/default.htm}
			\item \url{http://www.washingtonballet.org/community-engagement/other-programs/}
			\item Also, has need-based scholarships
			\end{enumerate}
		\end{enumerate}
	\item The John F. Kennedy Center for the Performing Arts: \vspace{-0.2cm}
		\begin{enumerate} \itemsep -2pt
		\item Exploring Ballet With Suzanne Farrell: A Three-Week Summer Ballet Intensive for Young Dancers: \vspace{-0.1cm}
			\begin{enumerate} \itemsep -1pt
			\item \url{http://www.kennedy-center.org/education/farrell/}
			\item ``In July and August, students from across the United States and around the world will participate in the eighteenth annual session of the Kennedy Center's ballet training program Exploring Ballet with Suzanne Farrell. The three-week residency for dancers ages 14 to 18 with at least five years of ballet training will be held at the Kennedy Center from August 1 - August 20, 2011.''
			\item ``During the three-week period, students take two ballet technique classes a day, six days a week, with Ms. Farrell. Students also participate in a number of cultural activities to enhance their experience in Washington, D.C., including museum visits, trips to historical landmarks, and attending performances.''
			\end{enumerate}
		\item Dance Theatre of Harlem Residency program: \vspace{-0.1cm}
			\begin{enumerate} \itemsep -1pt
			\item \url{http://www.kennedy-center.org/education/community/programs.html#artistic}
			\item ``Since 1993, the Kennedy Center's Dance Theatre of Harlem Residency program has provided ballet training for male and female students age 8-18 with identified promise in ballet taught by Dance Theatre of Harlem (DTH) instructors or former principal dancers.''
			\item ``Students are selected by audition for a twenty-class series, culminating with a public demonstration and performance on a Kennedy Center main stage.''
			\item ``Classical ballet training is taught in four class levels, from novice to advance.''
			\item ``Students must have at least one year of ballet training to qualify for the program.''
			\end{enumerate}
		\end{enumerate}
	\end{enumerate}
%%%%%%%%%%%%%%%%%%%%%%%
\item JA Worldwide (Junior Achievement): \vspace{-0.3cm}
	\begin{enumerate} \itemsep -2pt
	\item \url{http://www.ja.org/}
	\item Resources for educators: \url{http://www.ja.org/involved/involved_educat.shtml}
	\item Resources for parents: \url{http://www.ja.org/involved/involved_parents.shtml}
	\item Resources for students: \url{http://www.ja.org/involved/involved_students.shtml}
	\end{enumerate}
\item U.S. Department of State: \vspace{-0.3cm}
	\begin{enumerate} \itemsep -2pt
	\item Programs for Americans and non-Americans.
	\item Summer Work Travel - In the summer work travel program: \url{http://exchanges.state.gov/}
	\item Cultural Programs Division: \url{http://exchanges.state.gov/cultural/index.html}
	\item Youth Programs Division: \url{http://exchanges.state.gov/youth/index.html}
	\item EducationUSA: \url{http://educationusa.state.gov/}
	\item International Visitor Leadership Program: \url{http://exchanges.state.gov/ivlp/ivlp.html}
	\item Programs for non-U.S. Citizens: \url{http://exchanges.state.gov/prog-non-us.html}
	\item Programs for U.S. Citizens: \url{http://exchanges.state.gov/prog-us.html}
	\item Resources for Students: \url{http://exchanges.state.gov/student.html}
	\item Bureau of Educational and Cultural Affairs: \vspace{-0.2cm}
		\begin{enumerate} \itemsep -2pt
		\item Future Leaders Exchange (FLEX) Program: \vspace{-0.1cm}
			\begin{enumerate} \itemsep -1pt
			\item \url{http://exchanges.state.gov/youth/programs/flex.html}
			\item ``The Future Leaders Exchange (FLEX) Program gives students (ages 15-17) the chance to live with a host family and attend a U.S. high school for a year.''
			\end{enumerate}
		\item Office of Citizen Exchanges: \vspace{-0.1cm}
			\begin{enumerate} \itemsep -1pt
			\item Youth Programs Division: \vspace{-0.1cm}
				\begin{itemize} \itemsep -1pt
				\item \url{http://exchanges.state.gov/youth/index.html}
				\item Has programs for youths in various parts of the world
				\item ``The Youth Programs Division is committed to empowering the next generation and establishing long-lasting ties between the United States and other countries through exchange programs and institutional partnerships. Programs focus primarily on secondary schools and promote mutual understanding, leadership development, educational transformation and democratic ideals.''
				\end{itemize}
			\item SportsUnited: \vspace{-0.1cm}
				\begin{itemize} \itemsep -1pt
				\item \url{http://exchanges.state.gov/sports/index.html}
				\item SportsUnited is an international sports programming initiative designed to help start a dialogue at the grassroots level with non-elite boys and girls ages 7-17.
				\item The programs aid youth in discovering how success in athletics can be translated into the development of life skills and achievement in the classroom.
				\item Foreign participants are given an opportunity to establish links with U.S. sports professionals and exposure to American life and culture.
				\item Americans learn about foreign cultures and the challenges young people from overseas face today.
				\item The U.S. Department of State has programmed initiatives in: baseball, basketball, football, track and field, soccer, volleyball, wrestling, archery, boxing, swimming, fencing, table tennis, ice skating, weightlifting, water polo and managing sports community centers.
				\item Countries covered by this program are listed on the web page.
				\item Sports Envoy Program: \vspace{-0.1cm}
					\begin{itemize} \itemsep -1pt
					\item \url{http://exchanges.state.gov/sports/envoy1.html}
					\item Working with the national sports leagues and the U.S. Olympic Committee, athletes and coaches in various sports are chosen to serve as envoys or ambassadors of sport in overseas programs that include conducting clinics, visiting schools and speaking to youth.
					\item The American athletes and coaches conduct drills and team building activities, as well as engage the youth in a dialogue on the importance of an education, positive health practices and respect for diversity.
					\end{itemize}
				\item Sports Grant Competition: \vspace{-0.1cm}
					\begin{itemize} \itemsep -1pt
					\item The Bureau of Educational and Cultural Affairs (ECA) has an annual open competition under its International Sports Programming Initiative.
					\item Public and private non-profit organizations, 501(c)(3), may submit proposals to discuss approaches designed to enhance and improve the infrastructure of youth sports programs.
					\item The focus of all programs must be reaching out to non-elite youth ages 7-17 and/or their coaches/administrators.
					\item There are four themes that a proposal can address; Youth Sports Management, Training Sports Coaches, Sport and Disability, and Sport and Health.
					\item The list of eligible countries changes each year.
					\item \url{http://exchanges.state.gov/sports/index/sports-grant-competition.html}
					\end{itemize}
				\item Sports Visitor Program: \vspace{-0.1cm}
					\begin{itemize} \itemsep -1pt
					\item Nominated by our U.S. embassies overseas, selected athletes, managers and coaches are brought to the U.S. for technical sports training, sports management, conflict resolution training and exposure to valuable U.S. sports contacts and then are encouraged to return to conduct in-country clinics for youth with their newly learned skills.
					\item \url{http://exchanges.state.gov/sports/visitors.html}
					\end{itemize}
				\end{itemize}
			\end{enumerate}
		\end{enumerate}
	\end{enumerate}
\item U.S. Department of Labor: \vspace{-0.3cm}
	\begin{enumerate} \itemsep -2pt
	\item Wage and Hour Division: \vspace{-0.2cm}
		\begin{enumerate} \itemsep -2pt
		\item YouthRules!: \vspace{-0.1cm}
			\begin{enumerate} \itemsep -1pt
			\item \url{http://youthrules.dol.gov/}
			\item Has information for youths, parents, educators, and employers on how to let youth work part-time safely
			\item Teens: \url{http://youthrules.dol.gov/teens/default.htm}
			\item Parents: \url{http://youthrules.dol.gov/parents/default.htm}
			\item Educators: \url{http://youthrules.dol.gov/educators/default.htm}
			\item Employers: \url{http://youthrules.dol.gov/employers/default.htm}
			\item Resources: \url{http://youthrules.dol.gov/resources.htm}
			\item Compliance Assistance: \url{http://youthrules.dol.gov/ca.htm}
			\end{enumerate}
		\end{enumerate}
	\end{enumerate}
\item ASCL Educational Services, Inc. (Marc McCulloch): \vspace{-0.3cm}
	\begin{enumerate} \itemsep -2pt
	\item Transitions: Life Skills for Personal Success!: \vspace{-0.2cm}
		\begin{enumerate} \itemsep -2pt
		\item Curriculum \& Materials: \url{http://transitions.ascl.info/infomaterials}
		\item Soft Skills: \url{http://transitions.ascl.info/infoskills}
		\end{enumerate}
	\end{enumerate}
\item Partnership for 21st Century Skills: \vspace{-0.3cm}
	\begin{enumerate} \itemsep -2pt
	\item \url{http://www.p21.org/}
	\item Framework for 21st Century Learning: \url{http://www.p21.org/index.php?option=com_content&task=view&id=254&Itemid=119}
	\item Tools and Resources: \url{http://www.p21.org/index.php?option=com_content&task=view&id=273&Itemid=139}
	\end{enumerate}
\item National Career and Technical Education Foundation (NCTEF): \vspace{-0.3cm}
	\begin{enumerate} \itemsep -2pt
	\item States' Career Clusters Initiative (SCCI): \vspace{-0.2cm}
		\begin{enumerate} \itemsep -2pt
		\item \url{http://www.careerclusters.org/}
		\item The 16 Career Clusters: \url{http://www.careerclusters.org/16clusters.cfm}
		\item Plans of Study: \url{http://www.careerclusters.org/resources/web/pos.cfm}
		\item Knowledge and Skills Charts: \url{http://www.careerclusters.org/resources/web/ks.php}
		\item Crosswalks: \url{http://www.careerclusters.org/crosswalks.cfm}
		\item Publications: \url{http://www.careerclusters.org/publications.php}
		\item Sixteen Career Clusters and Their Pathways: \url{http://www.careerclusters.org/list16clusters.php}
		\item Career Clusters Models: \url{http://www.careerclusters.org/resources/web/16ccall.php?action=models}
		\item Career Clusters Brochure Previews: \url{http://www.careerclusters.org/resources/web/16ccall.php?action=brochures}
		\item Career Clusters Interest Survey: \url{http://www.careerclusters.org/ccinterestsurvey.php}
		\item Related Websites: \url{http://www.careerclusters.org/related.php}
		\end{enumerate}
	\end{enumerate}
\item U. S. Department of Labor: \vspace{-0.3cm}
	\begin{enumerate} \itemsep -2pt
	\item Employment and Training Administration: \vspace{-0.2cm}
		\begin{enumerate} \itemsep -2pt
		\item CareerOneStop: \vspace{-0.1cm}
			\begin{enumerate} \itemsep -1pt
			\item \url{http://www.careeronestop.org/}
			\item Students, parents, and career advisors: \url{http://www.careeronestop.org/studentsandcareeradvisors/studentsandcareeradvisors.aspx}
			\end{enumerate}
		\end{enumerate}
	\end{enumerate}
\item U. S. Department of Defense: \vspace{-0.3cm}
	\begin{enumerate} \itemsep -2pt
	\item ASVAB Career Exploration Program: \vspace{-0.2cm}
		\begin{enumerate} \itemsep -2pt
		\item \url{http://www.asvabprogram.com/}
		\item Learn about yourself: \url{http://www.asvabprogram.com/index.cfm?fuseaction=learn.main}
		\item Explore careers: \url{http://www.asvabprogram.com/index.cfm?fuseaction=explore.main}
		\item Plan for your future: \url{http://www.asvabprogram.com/index.cfm?fuseaction=plan.main}
		\item Information for educators and career counselors: \url{http://www.asvabprogram.com/index.cfm?fuseaction=edu.main}
		\item Information for parents: \url{http://www.asvabprogram.com/index.cfm?fuseaction=parents.main}
		\end{enumerate}
	\end{enumerate}
\end{enumerate}



%%%%%%%%%%%%%%%%%%%%%%%%%%%%%%%%%%%%%%%%%%%
\section{Internship Opportunities}
\label{Internship Opportunities}

Internship opportunities: \vspace{-0.3cm}
\begin{enumerate} \itemsep -4pt
\item Canada: \vspace{-0.3cm}
	\begin{enumerate} \itemsep -2pt
	\item SWAP: \vspace{-0.2cm}
		\begin{enumerate} \itemsep -2pt
		\item \url{http://www.swap.ca/}
		\item For Canadians who want to work abroad: \url{http://www.swap.ca/out_eng/index.aspx}
		\item For citizens of selected countries who want to work in Canada: \url{http://www.swap.ca/in_eng/partner_organizations.aspx}
		\end{enumerate}
	\end{enumerate}
\item Singapore: \vspace{-0.3cm}
	\begin{enumerate} \itemsep -2pt
	\item Speedwing Training (Asia) Pte Ltd: \vspace{-0.2cm}
		\begin{enumerate} \itemsep -2pt
		\item \url{http://www.speedwing.org/}
		\item For Singaporeans who want to work in the United States, Canada, Germany, and New Zealand
		\item For citizens of selected countries who want to work in Singapore
		\end{enumerate}
	\end{enumerate}
\end{enumerate}

%%%%%%%%%%%%%%%%%%%%%%%%%%%%%%%%%%%%%%%%%%%
\subsection{Internship Opportunities in Australia}
\label{internshipaus}

Internship Opportunities in Australia: \vspace{-0.3cm}
\begin{enumerate} \itemsep -4pt
\item The Association of Professional Engineers, Scientists and Managers, Australia: \url{http://www.apesma.asn.au/index.asp} --- Ask for guide to internships in your region/major; free student membership
\item Engineers Australia: \url{http://www.engineersaustralia.org.au/} --- Ask for guide to internships in your region/major; free student membership
\item CPA Australia: \url{http://www.cpaaustralia.com.au/cps/rde/xchg/cpa/hs.xsl/index.html} and \url{http://www.cpaaustralia.com.au/cps/rde/xchg/careers/site/index_ENA_HTML.htm/cps/rde/xchg/SID-3F57FECB-EEFEF50E/careers/site/204_ENA_HTML.htm}
\item Institute of Chartered Accountants in Australia: \url{http://www.charteredaccountants.com.au/}
\item 
\end{enumerate}


%%%%%%%%%%%%%%%%%%%%%%%%%%%%%%%%%%%%%%%%%%%
\subsection{Internship Opportunities in Europe}
\label{internshipeu}

Internship Opportunities in Portugal: \vspace{-0.3cm}
\begin{enumerate} \itemsep -4pt
\item Portugal: \vspace{-0.3cm}
	\begin{enumerate} \itemsep -2pt
	\item IAESTE Portugal (The International Association for the Exchange of Students for Technical Experience): \url{http://www.iaeste.pt/en/foreign-trainees/why-portugal/}
	\end{enumerate}
\item United Kingdom: \vspace{-0.3cm}
	\begin{enumerate} \itemsep -2pt
	\item Graduate Talent Pool: \url{http://graduatetalentpool.direct.gov.uk/}
	\end{enumerate}
\end{enumerate}




%%%%%%%%%%%%%%%%%%%%%%%%%%%%%%%%%%%%%%%%%%%
\subsection{Internship Opportunities in the United States}
\label{internshipsus}

Internship Opportunities in the United States: \vspace{-0.3cm}
\begin{enumerate} \itemsep -4pt
\item Use the Procedure \proc{Find}$(\varphi, \tau)$ in \S\ref{heuristiclocateoutreach} to look up internship opportunities and lists of internship opportunities.

Look at government organizations (e.g., the White House), nonprofit organizations (e.g., Engineers Without Borders), professional organizations (e.g., IEEE and ACM), colleges and universities, and companies (e.g., Intel, Google, and start-ups).

You can start your search by looking at the organizations that provide resources for underrepresented minorities as well as resources for scholarships and fellowships. These information can be found in other sections of this document.

If you do not know where to start, speak to a professor or staff member at the career center of your college/university. Alternatively, you can ask your awesome resident advisors (RAs).

My personal advice is to start your search based on your interests and skill set. You can always narrow the search space based on factors, such as geographical location, later on.

Competitive internships, especially research internships in electrical and computer engineering or computer science, weed out many students from applying via demanding job requirements. For example, if you want to apply for research internships with electronic design automation (EDA) companies and corporate research labs, you would need to have significant experience designing integrated circuits and developing EDA software. The stringent job requirements also mean that students need to plan in advance (say, about a year) about the internships that they would like to seek, and plan to acquire the necessary skill set and experiences before the application deadlines (which can be several months before the start of your internship).

Taking as many challenging classes as you can possibly cope, especially in electrical and computer engineering or computer science, would provide you with a skill set that allows you to apply for competitive internships in many fields. Apart from taking challenging classes as well as engaging in research and/or open source projects, you can try to acquire additional skills and experience in your free time to boost the competitiveness of your internship application. Certain skills and experiences, such as compiler design, are hard to acquire in your free time, so it would be ``easier'' to take classes that would help you acquire those skills and experiences.

Note that you may want to look into creating your own entrepreneurial venture, say an EDA start-up or organization in social entrepreneurship, rather than to seek an internship. Also, seeking an internship abroad is always a good addition to your resume/CV.
\item National Science Foundation: \vspace{-0.3cm}
	\begin{enumerate} \itemsep -2pt
	\item Research Experiences for Undergraduates (REU): \vspace{-0.2cm}
		\begin{enumerate} \itemsep -2pt
		\item \url{http://www.nsf.gov/crssprgm/reu/reu_search.cfm}
		\item Academic fields: \vspace{-0.1cm}
			\begin{enumerate} \itemsep -1pt
			\item Astronomical Sciences
			\item Atmospheric and Geospace Sciences
			\item Biological Sciences
			\item Chemistry
			\item Computer and Information Science and Engineering
			\item Cyberinfrastructure
			\item Department of Defense (DoD)
			\item Earth Sciences
			\item Education and Human Resources
			\item Engineering
			\item Ethics and Values Studies
			\item International Science and Engineering
			\item Materials Research
			\item Mathematical Sciences
			\item Ocean Sciences
			\item Physics
			\item Polar Programs
			\item Social, Behavioral, and Economic Sciences
			\end{enumerate}
		\end{enumerate}
	\end{enumerate}
\item Society for Industrial and Applied Mathematics: \vspace{-0.3cm}
	\begin{enumerate} \itemsep -2pt
	\item Internship and Career Information in Industry, Research Institutions, and Government Labs: \url{http://www.siam.org/careers/internships.php}
	\end{enumerate}
\item American Institute of Physics (AIP): \vspace{-0.3cm}
	\begin{enumerate} \itemsep -2pt
	\item Society of Physics Students (SPS): \vspace{-0.2cm}
		\begin{enumerate} \itemsep -2pt
		\item SPS Internships: \url{http://www.spsnational.org/programs/internships/}
		\item Research Opportunities: \url{http://www.spsnational.org/programs/research/}
		\end{enumerate}
	\end{enumerate}
%%%%%%%%%%%%%%%%%%%%%%%%%%%%%%%%%%%%%%
%%%%%%%%%%%%%%%%%%%%%%%%%%%%%%%%%%%%%%
%%%%%%%%%%%%%%%%%%%%%%%%%%%%%%%%%%%%%%
\item United States Office of Personnel Management: \vspace{-0.3cm}
	\begin{enumerate} \itemsep -2pt
	\item USAJOBS: \vspace{-0.2cm}
		\begin{enumerate} \itemsep -2pt
		\item Student Jobs: \url{http://www.usajobs.gov/studentjobs/}
		\end{enumerate}
	\end{enumerate}
%%%%%%%%%%%%%%%%%%%%%%%%%%%%%%%%%%%%%%
%%%%%%%%%%%%%%%%%%%%%%%%%%%%%%%%%%%%%%
%%%%%%%%%%%%%%%%%%%%%%%%%%%%%%%%%%%%%%
\item Americans for the Arts: \vspace{-0.3cm}
	\begin{enumerate} \itemsep -2pt
	\item Internship Program: \url{http://www.americansforthearts.org/about_us/internships.asp}
	\end{enumerate}
\item New York Women's Foundation: \vspace{-0.3cm}
	\begin{enumerate} \itemsep -2pt
	\item Internship Opportunities: \url{http://www.nywf.org/internship.html}
	\item Volunteer Opportunities: \url{http://www.nywf.org/volunteer.html}
	\end{enumerate}
\item Council on International Educational Exchange (CIEE): \url{http://www.ciee.org/hire/index.aspx}
\item The John F. Kennedy Center for the Performing Arts: \vspace{-0.3cm}
	\begin{enumerate} \itemsep -2pt
	\item Kennedy Center Arts Management Internships: \url{http://www.kennedy-center.org/education/artsmanagement/internships/}
	\end{enumerate}
\item Washington Performing Arts Society (WPAS): \vspace{-0.3cm}
	\begin{enumerate} \itemsep -2pt
	\item Internships with WPAS: \vspace{-0.2cm}
		\begin{enumerate} \itemsep -2pt
		\item \url{http://www.wpas.org/aboutwpas/opportunities/intern.aspx}
		\item ``WPAS offers internships throughout the year. Applicants should be highly motivated, creative and hard-working individuals with an interest in all aspects of arts management. It is required that applicants have previous office experience.''
		\item In addition, applicants should possess: \vspace{-0.1cm}
			\begin{enumerate} \itemsep -1pt
			\item Interest/background in music, dance or performance art
			\item Strong organizational skills
			\item Effective writing and communication skills
			\item Ability to learn quickly, handle multiple tasks, take initiative, and work independently with little supervision
			\item High energy level and ability to work well in deadline and/or pressure situations
			\item Computer literacy
			\end{enumerate}
		\item ``WPAS interns leave our offices with a better understanding of arts management, knowledge of artists in a variety of fields (classical music, world music, dance and performance art), contacts in theaters throughout the D.C. metro area, practical experience and a portfolio of work. The internship is unpaid, however stipends are occasionally granted during the performance year (September - May). Interns are also invited to attend many WPAS performances on a complimentary basis.''
		\item Types of internships: \vspace{-0.1cm}
			\begin{enumerate} \itemsep -1pt
			\item Accounting Internship
			\item Development Internship
			\item Education Internship
			\item Marketing/Public Relations Internship
			\item Office Administration Internship
			\item Programming Internship
			\end{enumerate}
		\end{enumerate}
	\end{enumerate}
\item The Washington Ballet: Internships, \url{http://www.washingtonballet.org/about-twb/auditions-employment/#internships}
\item The Choral Arts Society of Washington: \vspace{-0.3cm}
	\begin{enumerate} \itemsep -2pt
	\item Internship and Apprenticeship Program: \url{http://www.choralarts.org/About-Us/Internships-and-Apprenticeships.aspx}
	\end{enumerate}
\item League of American Orchestras: Internships, \url{http://www.americanorchestras.org/career_center/internships.html}
%%%%%%%%%%%%%%%%%%%%%%%%%%%%%%%%%%%%%%
%%%%%%%%%%%%%%%%%%%%%%%%%%%%%%%%%%%%%%
%%%%%%%%%%%%%%%%%%%%%%%%%%%%%%%%%%%%%%
\item Congressional Hispanic Caucus Institute (CHCI): \vspace{-0.3cm}
	\begin{enumerate} \itemsep -2pt
	\item CHCI United Health Foundation Scholars: \vspace{-0.2cm}
		\begin{enumerate} \itemsep -2pt
		\item \url{http://www.chci.org/scholarships/page/chci-united-health-foundation-scholars-}
		\item In addition to providing scholarship opportunities for Latino youth, the United Health Foundation decided to partner with CHCI to create a six-month internship program for students interested in the medical field.
		\item Seventeen participants enrolled in either a full-time undergraduate or graduate course of study at an accredited two- or four-year college, university, vocational or technical school were selected.
		\end{enumerate}
	\item CHCI Congressional Internship: \vspace{-0.2cm}
		\begin{enumerate} \itemsep -2pt
		\item The purpose of the Congressional Internship Program (CIP) is to expose young Latinos to the legislative process and to strengthen their professional and leadership skills, ultimately promoting the presence of Latinos on Capitol Hill.
		\item The Congressional Internship Program provides college students with a paid Congressional work placement on Capitol Hill for a period of twelve weeks (Spring/Fall) or eight weeks (Summer). This unmatched experience allows students to learn first hand about our nation's legislative process.
		\end{enumerate}
	\end{enumerate}
\item Mexican American Legal Defense and Educational Fund (MALDEF): Law Clerk Summer Internship program, \url{http://maldef.org/about/jobs/index.html}
\item Hispanic Association of Colleges and Universities (HACU): \vspace{-0.3cm}
	\begin{enumerate} \itemsep -2pt
	\item HACU National Internship Program (HNIP): \url{http://www.hacu.net/hacu/HNIP_EN.asp}
	\end{enumerate}
%%%%%%%%%%%%%%%%%%%%%%%%%%%%
\item Smithsonian Institution: \vspace{-0.3cm}
	\begin{enumerate} \itemsep -2pt
	\item Smithsonian Institution Traveling Exhibition Service (SITES): \vspace{-0.2cm}
		\begin{enumerate} \itemsep -2pt
		\item Internship programs: \url{http://www.sites.si.edu/interns/internships.htm}
		\item ``The Smithsonian Institution Traveling Exhibition Service internship programs allows people with diverse interests, strengths, and goals to experience an educational environment where they can work and learn from professionals in the museum field.''
		\item ``SITES offers internship opportunities in a variety of different areas: public relations, development (fund raising), research, and project design.''
		\end{enumerate}
	\item Smithsonian Folkways Recordings (or simply, Smithsonian Folkways): \vspace{-0.2cm}
		\begin{enumerate} \itemsep -2pt
		\item Internships: \url{http://www.folkways.si.edu/about_us/jobs.aspx}
		\end{enumerate}
	\item Freer Gallery of Art / Arthur M. Sackler Gallery: \vspace{-0.2cm}
		\begin{enumerate} \itemsep -2pt
		\item Internships: \url{http://www.asia.si.edu/research/internships.asp}
		\end{enumerate}
	\item National Museum of American History: \vspace{-0.2cm}
		\begin{enumerate} \itemsep -2pt
		\item Jerome and Dorothy Lemelson Center for the Study of Invention and Innovation: \vspace{-0.1cm}
			\begin{enumerate} \itemsep -1pt
			\item Archival Internships: \url{http://invention.smithsonian.org/resources/research_interns.aspx}
			\end{enumerate}
		\end{enumerate}
	\end{enumerate}
%%%%%%%%%%%%%%%%%%%%%%%%%%%%
\item Council on International Educational Exchange (CIEE): \vspace{-0.3cm}
	\begin{enumerate} \itemsep -2pt
	\item CIEE's Trainee Program: \vspace{-0.2cm}
		\begin{enumerate} \itemsep -2pt
		\item part of the J-1 visa category of the US government�s Exchange Visitor Program
		\item \url{http://www.ciee.org/trainee/}
		\end{enumerate}
	\item CIEE Work \& Travel USA; and Internship USA: \vspace{-0.2cm}
		\begin{enumerate} \itemsep -2pt
		\item \url{http://www.ciee.org/hire/}
		\item \url{http://www.ciee.org/wat/}
		\end{enumerate}
	\end{enumerate}
\item American Institute For Foreign Study (AIFS): \vspace{-0.3cm}
	\begin{enumerate} \itemsep -2pt
	\item Camp America Counselors and Summer Staff: \url{http://www.aifs.com/work_travel.asp}
	\item Au Pair Placement: \url{http://www.aifs.com/au_pair.asp}
	\end{enumerate}
\item U.S. Department of State: \vspace{-0.3cm}
	\begin{enumerate} \itemsep -2pt
	\item Bureau of Educational and Cultural Affairs: \vspace{-0.2cm}
		\begin{enumerate} \itemsep -2pt
		\item International cultural programs: \url{http://exchanges.state.gov/cultural/related-cultural-programs.html}
		\item Office of Global Educational Programs: \vspace{-0.1cm}
			\begin{enumerate} \itemsep -1pt
			\item Camp Counselor: \vspace{-0.1cm}
				\begin{itemize} \itemsep -1pt
				\item \url{http://exchanges.state.gov/jexchanges/programs/camp.html}
				\item Camp counselors interact with groups of American youth by overseeing their camp activities during the U.S. summer.
				\item Through the Camp Counselor program, American campers have the chance to gain knowledge of foreign cultures, while foreign participants increase their knowledge of American culture.
				\item Participants must be at least 18 years of age and may work as counselors in U.S. summer camps for up to four months. Extensions are not allowed. They receive a combination a pay and benefits equal to Americans who work in the same position.
				\end{itemize}
			\end{enumerate}
		\item Private Sector Exchange office: \vspace{-0.1cm}
			\begin{enumerate} \itemsep -1pt
			\item \url{http://exchanges.state.gov/jexchanges/index.html}
			\item The Private Sector Exchange office designates, monitors and partners with U.S. organizations, including government agencies, academic institutions, educational and cultural organizations, and corporations, that administer the Exchange Visitor Program.
			\item Au Pair program: \vspace{-0.1cm}
				\begin{itemize} \itemsep -1pt
				\item Through the Au Pair program, foreign nationals between 18 and 26 years of age participate in the home life of a host family. Au pairs provide limited childcare services for up to 12 months. An extension of 6, 9, or 12 months may be granted in certain cases.
				\item \url{http://exchanges.state.gov/jexchanges/programs/aupair.html}
				\end{itemize}
			\item Internships: \vspace{-0.1cm}
				\begin{itemize} \itemsep -1pt
				\item \url{http://exchanges.state.gov/jexchanges/programs/intern.html}
				\item Internship programs are designed to allow foreign professionals to come to the United States to gain exposure to U.S. culture and to receive training in U.S. business practices in their chosen occupational field.
				\item The maximum duration of an internship in any occupational field is 12 months.
				\item Upon completion of their exchange programs, participants are expected to return to their home countries.
				\item The State Department allows internships in the following occupational categories: \vspace{-0.1cm}
					\begin{itemize} \itemsep -1pt
					\item Agriculture, Forestry, and Fishing
					\item Arts and Culture
					\item Construction and Building Trades
					\item Education, Social Sciences, Library Science, Counseling and Social Services
					\item Health Related Occupations
					\item Hospitality and Tourism
					\item Information Media and Communications
					\item Management, Business, Commerce and Finance
					\item Public Administration and Law
					\item The Sciences, Engineering, Architecture, Mathematics, and Industrial Occupations.
					\end{itemize}
				\item An Intern must be a foreign national: \vspace{-0.1cm}
					\begin{itemize} \itemsep -1pt
					\item Who is currently enrolled in and pursuing studies at a foreign degree- or certificate-granting post-secondary academic institution outside the United States, or
					\item Who has graduated from such an institution no more than 12 months prior to his or her exchange visitor program start date.
					\end{itemize}
				\item Interns cannot work in unskilled or casual labor positions, in positions that require or involve child care or elder care, or in any kind of position that involves medical patient care or contact. Nor can interns work in positions that require more than 20 per cent clerical or office support work.
				\end{itemize}
			\item The Summer Work Travel Program: \vspace{-0.1cm}
				\begin{itemize} \itemsep -1pt
				\item \url{http://exchanges.state.gov/jexchanges/programs/swt.html}
				\item In the summer work travel program, post-secondary students may enter the United States to work and travel during their summer vacation.
				\item Participants can be admitted to the program more than once.
				\item The maximum length of the program is four months.
				\item Most of the time, participants work in unskilled service positions at resorts, hotels, restaurants, and amusement parks. However, they may also work in other types of organizations.
				\item For example, they could work in architectural firms, scientific research organizations, graphic art/publishing and other media communication businesses, advertising agencies, computer software and electronics firms, legal offices, etc.
				\item The program may not exceed four-months and must be finished during the student's summer vacation.
				\item Participants receive pay and benefits equal to an American working in the same or similar position.
				\end{itemize}
			\item Training programs: \vspace{-0.1cm}
				\begin{itemize} \itemsep -1pt
				\item \url{http://exchanges.state.gov/jexchanges/programs/trainee.html}
				\item Training programs are designed to allow foreign professionals to come to the United States to gain exposure to U.S. culture and to receive training in U.S. business practices in their chosen occupational field.
				\item Foreign nationals have had the opportunity to train with some of the finest employers in the U.S., gaining real time experience in their chosen career fields.
				\item Upon completion of their exchange programs, participants are expected to return to their home countries to utilize their newly learned skills and knowledge to advance their careers and share their experiences with their communities.
				\item The State Department allows training programs in the following occupational categories: \vspace{-0.1cm}
					\begin{itemize} \itemsep -1pt
					\item Agriculture, Forestry, and Fishing
					\item Arts and Culture
					\item Construction and Building Trades
					\item Education, Social Sciences, Library Science, Counseling and Social Services
					\item Health Related Occupations
					\item Hospitality and Tourism
					\item Information Media and Communications
					\item Management, Business, Commerce and Finance
					\item Public Administration and Law
					\item The Sciences, Engineering, Architecture, Mathematics, and Industrial Occupations.
					\end{itemize}
				\item A trainee must be a foreign national who has: \vspace{-0.1cm}
					\begin{itemize} \itemsep -1pt
					\item A degree or professional certificate from a foreign post-secondary academic institution and at least one year of prior related work experience in his or her occupational field outside the United States, or
					\item Five years of work experience outside the United States in the occupational field in which they are seeking training.
					\end{itemize}
				\end{itemize}
			\item Specialists: \vspace{-0.1cm}
				\begin{itemize} \itemsep -1pt
				\item \url{http://exchanges.state.gov/jexchanges/programs/specialist.html}
				\item This category is for a participant who is an expert in a field of specialized knowledge or skill who will demonstrate such skills in the United States. Such exchanges are to provide opportunities to increase the exchange knowledge and ideas between American and foreign specialists. The maximum duration of this program is one year.
				\item This category is for foreign nationals who are experts in a field of specialized knowledge or skill, coming to the United States for observing, consulting, or demonstrating their special skills, except: Professors and Research Scholars, Short-Term Scholars, and Alien Physicians.
				\item Individuals participating in the specialist program are: \vspace{-0.1cm}
					\begin{itemize} \itemsep -1pt
					\item Experts in a field of specialized knowledge or skill;
					\item Seeks to travel to the United States for the purpose of observing, consulting, or demonstrating their special knowledge or skills;
					\item Does not fill a permanent or long-term position of employment while in the U.S.
					\end{itemize}
				\end{itemize}
			\item International Visitor: \vspace{-0.1cm}
				\begin{itemize} \itemsep -1pt
				\item \url{http://exchanges.state.gov/jexchanges/programs/intl_visitor.html}
				\item The international visitor category enables visitors to better understand American culture and enhanced American knowledge of foreign cultures.
				\item This category is for individuals who are recognized as potential leaders in their own country, selected by the Department of State to participate in observation tours, discussions, consultation, professional meetings, conferences, workshops and travel.
				\item The maximum duration of the program is one year.
				\end{itemize}
			\item Alien Physician: \vspace{-0.1cm}
				\begin{itemize} \itemsep -1pt
				\item \url{http://exchanges.state.gov/jexchanges/programs/physician.html}
				\item The Alien Physician program is for foreign national physicians seeking entry into U.S. graduate medical education programs or training at accredited U.S. schools of medicine or other U.S. institutions.
				\item There are generally two types of exchange programs in which a foreign national physician (also referred to as a foreign/international medical graduate) participates: \vspace{-0.1cm}
					\begin{itemize} \itemsep -1pt
					\item Clinical training in the �alien physician� category
					\item Non-Clinical training in the �research scholar� category
					\end{itemize}
				\end{itemize}
			\item FORTUNE/U.S. State Department Global Women's Mentoring Partnership: \vspace{-0.1cm}
				\begin{itemize} \itemsep -1pt
				\item \url{http://exchanges.state.gov/citizens/professionals/fortunepartnership.html}
				\item This public-private partnership places talented, emerging women leaders from all over the world in mentoring programs with FORTUNE's Most Powerful Women Leaders.
				\item For three weeks, American and international participants work together in mentoring relationships to share the skills and experiences necessary for strengthening women�s leadership.
				\end{itemize}
			\item American Council of Young Political Leaders (ACYPL): \vspace{-0.1cm}
				\begin{itemize} \itemsep -1pt
				\item \url{http://exchanges.state.gov/citizens/profs/acypl.html}
				\item \url{http://www.acypl.org/}
				\item For 44 years, the American Council of Young Political Leaders (ACYPL) has designed, organized and managed unique international educational exchanges for young political leaders (ages 25-40) worldwide.
				\item ACYPL programs are designed to promote mutual understanding, respect, and friendship and to cultivate long-lasting relationships among young people who are poised to become tomorrow's global leaders and policy makers.
				\item American participants are nominated by members of Congress, governors, political party leaders, and ACYPL alumni, while international delegates are selected from countries where ACYPL is currently conducting programs by international program partners with the U.S. Embassy input.
				\end{itemize}
			\item Edward R. Murrow Program for Journalists: \vspace{-0.1cm}
				\begin{itemize} \itemsep -1pt
				\item \url{http://exchanges.state.gov/ivlp/murrow.html}
				\item The Edward R. Murrow Program for Journalists invites rising international journalists to travel to the United States and examine journalistic principles and practices.
				\end{itemize}
			\end{enumerate}
		\item Office of Citizen Exchanges: \vspace{-0.1cm}
			\begin{enumerate} \itemsep -1pt
			\item Youth Programs Division: \vspace{-0.1cm}
				\begin{itemize} \itemsep -1pt
				\item \url{http://exchanges.state.gov/youth/index.html}
				\item The Youth Programs Division is committed to empowering the successor generation and establishing long-lasting ties between the United States and other countries through exchange programs and institutional partnerships.
				\item Programs focus primarily on secondary schools and promote mutual understanding, leadership development, educational transformation, and democratic ideals.
				\item Year-Long Programs, Short Term Programs, and Virtual Partnerships: \url{http://exchanges.state.gov/youth/programs-by-type.html}
				\item Programs for Young Americans, and Programs for International Students and Teachers: \url{http://exchanges.state.gov/youth/programs-by-participants.html}
				\item Opportunities for American Hosts: Families and Schools, \url{http://exchanges.state.gov/youth/opps-for-am-hosts.html}
				\item Programs for High School Students: \url{http://exchanges.state.gov/youth/programs.html}
				\end{itemize}
			\item Professional Exchanges Division: \vspace{-0.1cm}
				\begin{itemize} \itemsep -1pt
				\item \url{http://exchanges.state.gov/citizens/profs.html}
				\item The Professional Exchanges division provides grants to U.S. nonprofit organizations to carry out exchange programs that support the professional development of foreign participants. The purpose of each exchange program is to engage with foreign leaders in critical professions, to demonstrate respect for foreign cultures, and to promote mutual understanding between the people of the United States and other countries.
				\item Professional exchanges typically last several years and include internships, study tours or workshops in the United States and in the host country. Participants come from a variety of professions including education administrators, public servants, journalists, labor union officials, entrepreneurs, environmental leaders, jurists, lawyers, and civic leaders.
				\item ECA grant opportunities: \vspace{-0.1cm}
					\begin{itemize} \itemsep -1pt
					\item Open Funding Opportunities: Requests For Grant Proposals (RFGPs), \url{http://exchanges.state.gov/grants/open2.html}
					\item Grants.gov: \url{http://www.grants.gov/}
					\end{itemize}
				\item Grants by Region: \vspace{-0.1cm}
					\begin{itemize} \itemsep -1pt
					\item \url{http://exchanges.state.gov/citizens/professionals/grant-region.html}
					\item Africa 
					\item East Asia and the Pacific 
					\item Europe and Eurasia 
					\item North Africa and the Middle East 
					\item South and Central Asia 
					\item Western Hemisphere 
					\item Multi-regional
					\end{itemize}
				\end{itemize}
			\end{enumerate}
		\end{enumerate}
	\end{enumerate}
\end{enumerate}






%%%%%%%%%%%%%%%%%%%%%%%%%%%%%%%%%%%%%%%%%%%
\section{Resources on Studying Abroad}
\label{resourcesonstudyingabroad}

Resources on studying abroad: \vspace{-0.3cm}
\begin{enumerate} \itemsep -4pt
\item Council on International Educational Exchange (CIEE): \vspace{-0.3cm}
	\begin{enumerate} \itemsep -2pt
	\item Study abroad programs for high school students from the United States: \vspace{-0.2cm}
		\begin{enumerate} \itemsep -2pt
		\item \url{http://www.ciee.org/hsabroad/index.html}
		\item \url{http://www.ciee.org/hsabroad/high-school-study-abroad/index.html}
		\item These programs include:: \vspace{-0.1cm}
			\begin{enumerate} \itemsep -1pt
			\item High School Abroad programs (for U.S. high school students)
			\item Summer High School Abroad programs (for U.S. high school students)
			\item Gap Year Abroad programs (for recent U.S. high school graduates)
			\end{enumerate}
		\end{enumerate}
	\end{enumerate}
\item U.S. Department of State: \vspace{-0.3cm}
	\begin{enumerate} \itemsep -2pt
	\item Bureau of Educational and Cultural Affairs: \vspace{-0.2cm}
		\begin{enumerate} \itemsep -2pt
		\item Office of Global Educational Programs: \vspace{-0.1cm}
			\begin{enumerate} \itemsep -1pt
			\item EducationUSA: \vspace{-0.1cm}
				\begin{itemize} \itemsep -1pt
				\item EducationUSA is a network of more than 400 student advising centers, which offer accurate, comprehensive, objective and timely information about educational opportunities in the United States and guidance to qualified individuals on how best to access those opportunities. This includes information about application procedures, standardized test requirements, student visas, financial aid, and the full range of accredited U.S. higher education institutions.
				\item \url{http://exchanges.state.gov/globalexchanges/index/educationusa.html}
				\item \url{http://www.educationusa.state.gov/} and \url{http://www.educationusa.info/centers.php}
				\end{itemize}
			\item Open Doors: \vspace{-0.1cm}
				\begin{itemize} \itemsep -1pt
				\item The Educational Information and Resources Branch funds Open Doors, a census of foreign students and scholars in the U.S. and of U.S. students studying abroad published annually by the Institute for International Education.
				\item Open Doors data is used by U.S. embassies, the Departments of State, Commerce, and Education, and U.S. colleges and universities to inform policy decisions about educational exchanges, trade in educational services, and study abroad activity.
				\item \url{http://exchanges.state.gov/globalexchanges/index/open_doors.html}
				\item \url{http://www.opendoors.iienetwork.org/}
				\end{itemize}
			\end{enumerate}
		\item EducationUSA: \vspace{-0.1cm}
			\begin{enumerate} \itemsep -1pt
			\item \url{http://educationusa.state.gov/}
			\item For U.S. (college) students who want to study/work abroad: \url{http://www.educationusa.info/pages/students/forus.php}
			\end{enumerate}
		\end{enumerate}
	\end{enumerate}
\item IES Abroad (formerly Institute of European Studies / Institute for the International Education of Students): \vspace{-0.3cm}
	\begin{enumerate} \itemsep -2pt
	\item \url{https://www.iesabroad.org/} and \url{https://www.iesabroad.org/IES/home.html}
	\end{enumerate}
\item Global Learning Semesters, Inc.: \vspace{-0.3cm}
	\begin{enumerate} \itemsep -2pt
	\item Summer in the Mediterranean: \vspace{-0.2cm}
		\begin{enumerate} \itemsep -2pt
		\item \url{http://www.globalsemesters.com/Mediterranean.html}
		\item Has programs in the following areas: \vspace{-0.1cm}
			\begin{enumerate} \itemsep -1pt
			\item Art \& Photography
			\item Early Christianity
			\item Greek Heritage
			\item International Marketing
			\item Music
			\end{enumerate}
		\end{enumerate}
	\end{enumerate}
\item American Institute For Foreign Study (AIFS): \vspace{-0.3cm}
	\begin{enumerate} \itemsep -2pt
	\item \url{http://www.aifs.com/}
	\item College Study Abroad: \url{http://www.aifsabroad.com/}
	\item For high school students: \vspace{-0.2cm}
		\begin{enumerate} \itemsep -2pt
		\item Gifted Education: \url{http://www.aifs.com/gifted_education.asp}
		\item High School Study and Travel: \url{http://www.aifs.com/highschool_study_travel.asp}
		\item Academic Year in America (AYA): \url{http://www.academicyear.org/?source=AIFS}
		\end{enumerate}
	\end{enumerate}
\end{enumerate}





%%%%%%%%%%%%%%%%%%%%%%%%%%%%%%%%%%%%%%%%%%%
\section{College Preparation}
\label{collegepreparation}

College preparation: \vspace{-0.3cm}
\begin{enumerate} \itemsep -4pt
\item {\it Guide to Online Schools} [or {\it GuideToOnlineSchools.com}], {\it The Top 53 College Preparation Resources for Students}. Available at: \url{http://www.guidetoonlineschools.com/tips-and-tools/college-prep-resources}; last accessed on August 25, 2010.
\item U.S. Department of Education's resources for parents to help their children learn: \url{http://www2.ed.gov/parents/academic/help/hyc.html} and \url{http://www2.ed.gov/parents/academic/help/homework/index.html}
\item The College Board: \vspace{-0.3cm}
	\begin{enumerate} \itemsep -2pt
	\item Information about SATs, college preparation, and financial aid
	\item {\it Trends in Higher Education} series 201X: \url{http://trends.collegeboard.org/}
	\item \url{http://www.collegeboard.com/}
	\end{enumerate}
\item {\it Accreditation.org}: \vspace{-0.3cm}
	\begin{enumerate} \itemsep -2pt
	\item Information about the accreditation of engineering degree programs around the world
	\item \url{http://www.accreditation.org/}
	\end{enumerate}
\item {\it New York Times}: \vspace{-0.3cm}
	\begin{enumerate} \itemsep -2pt
	\item The Learning Network: \url{http://learning.blogs.nytimes.com/category/test-yourself/}
	\item New York Times Magazine: \vspace{-0.2cm}
		\begin{enumerate} \itemsep -2pt
		\item The Sep 20, 2010 issue has many articles covering how technology can be used to improve education in K-12 programs. Available online at: \url{http://www.nytimes.com/indexes/2010/09/19/magazine/index.html?ref=magazine}; last accessed on September 20, 2010.
		\item ``New York Times Magazine Features Technology in Education,'' in {\it CCC Blog}, Computing Community Consortium (CCC), Computing Research Association (CRA), Sep 20, 2010. Available online at: \url{http://www.cccblog.org/2010/09/20/new-york-times-magazine-features-technology-in-education/}; last accessed on September 20, 2010.
		\item Articles in this issue discuss: \vspace{-0.1cm}
			\begin{enumerate} \itemsep -1pt
			\item How journalists can make use of technology to automate certain tasks, and improve their productivity and effectiveness in covering news stories
			\item How children can create computer games that introduces them to careers in computing and helps them to develop skills in computational thinking
			\item How to learn things without a lot of rote learning, to have fun while learning, and to use technology to make learning more fun
			\end{enumerate}
		\end{enumerate}
	\end{enumerate}
\item University of Southern California, USC: \vspace{-0.3cm}
	\begin{enumerate} \itemsep -2pt
	\item USC Office of Continuing Education and Summer Programs: \vspace{-0.2cm}
		\begin{enumerate} \itemsep -2pt
		\item \url{http://cesp.usc.edu/}
		\item These programs allow students in K-12 to earn credit at USC, and exposes them to different majors/professions, like medicine, engineering, creative writing, or film making.
		\item Students can benefit from these programs, and learn about different academic disciplines before applying to college. This would help them in their college applications.
		\item Underrepresented minority students can get scholarships to attend these programs. So, if parents have financial difficulty paying for the programs, they can seek financial aid for this.
		\item Also, current undergraduates can also sign up for programs to learn about marketing, finance, and entrepreneurship. They can also do summer research with USC researchers.
		\end{enumerate}
	\item Summer sports programs for youths: \vspace{-0.2cm}
		\begin{enumerate} \itemsep -2pt
		\item SC Futbol Academy (USC Soccer Camps): \url{http://www.usctrojans.com/sports/w-soccer/spec-rel/021610aaa.html}
		\item Mick Haley's USC Girls Volleyball Camp: \url{http://www.usctrojans.com/sports/w-volley/spec-rel/volley-camp.html}
		\item Salo Swim Camp: \url{http://www.saloswimcamp.com/on-line/default.asp}
		\item USC NYSP Trojan KidSCamp: \url{http://sait.usc.edu/recsports/site_content/youth_sports/nysptk.html}
		\item After School Sports Connection, ASSC (operates in fall, spring, and summer): \url{http://sait.usc.edu/recsports/site_content/youth_sports/assc.html}
		\end{enumerate}
	\end{enumerate}
\item Telluride Association: \vspace{-0.3cm}
	\begin{enumerate} \itemsep -2pt
	\item Telluride Association Summer Program (TASP) [ for high school students ]: \url{http://www.tellurideassociation.org/programs/high_school_students/tasp/tasp_general_info.html}
	\item Telluride Association Sophomore Seminar (TASS) [ for high school students ]: \url{http://www.tellurideassociation.org/programs/high_school_students/tass/tass_general_info.html}
	\item Resources for high school educators to nominate summer program applicants: \url{http://www.tellurideassociation.org/programs/high_school_students/hs_resources/hs_resources_general_information.html}
	\end{enumerate}
\item MathNerds: \vspace{-0.3cm}
	\begin{enumerate} \itemsep -2pt
	\item \url{http://www.mathnerds.com/}
	\item ``Provides free, discovery-based, mathematical guidance via an international, volunteer network of mathematicians.''
	\item If you have a mathematical problem to solve, you can ask mathematicans at {\it MathNerds} for help.
	\item They would require you to discuss your attempted approaches/solutions.
	\item If you have not made attempts to solve the problem, they will not give you much guidance.
	\item In addition, they cannot solve problems for you.
	\item They provide guidance for mathematical problems from K-12 material through undergraduate mathematics and statistics classes.
	\item They also provide help for selected topics in advanced mathematics classes (for graduate students).
	\item Other resources: \url{http://www.mathnerds.com/links/links.aspx}
	\end{enumerate}
\item Hobsons: \vspace{-0.3cm}
	\begin{enumerate} \itemsep -2pt
	\item CollegeView (Hobsons' college recruiting services): \url{http://www.collegeview.com/index.jsp}
	\end{enumerate}
\item Sponsors for Educational Opportunity (SEO): \vspace{-0.3cm}
	\begin{enumerate} \itemsep -2pt
	\item Resources: \url{http://www.seo-usa.org/ScholarsResources}
	\end{enumerate}
\item U.S. Department of Education: \vspace{-0.3cm}
	\begin{enumerate} \itemsep -2pt
	\item Students.gov: \url{http://www.students.gov/STUGOVWebApp/index.jsp}
	\item college.gov: \url{http://www.college.gov/wps/portal}
	\end{enumerate}
\item U.S. Department of State: \vspace{-0.3cm}
	\begin{enumerate} \itemsep -2pt
	\item Bureau of Educational and Cultural Affairs: \vspace{-0.2cm}
		\begin{enumerate} \itemsep -2pt
		\item EducationUSA: \vspace{-0.1cm}
			\begin{enumerate} \itemsep -1pt
			\item Information for international students: \url{http://www.educationusa.info/students.php}
			\end{enumerate}
		\end{enumerate}
	\end{enumerate}
\item Congressional Hispanic Caucus Institute (CHCI): \vspace{-0.3cm}
	\begin{enumerate} \itemsep -2pt
	\item CHCI Education Center: \vspace{-0.2cm}
		\begin{enumerate} \itemsep -2pt
		\item \url{http://www.chci.org/education_center/}
		\item Has resources on college planning, financial aid, scholarships, college internships, and housing.
		\item For Parents: \url{http://www.chci.org/education_center/page/for-parents}
		\item For Students: \url{http://www.chci.org/education_center/page/for-students}
		\end{enumerate}
	\end{enumerate}
\item My College Options: \vspace{-0.3cm}
	\begin{enumerate} \itemsep -2pt
	\item \url{http://www.mycollegeoptions.org/}
	\item ``My College Options is a FREE college planning service, offering assistance to students, parents, high schools, counselors, and teachers nationwide.''
	\item ``It is designed to assist high school students in exploring a wide range of post-secondary opportunities, with special emphasis on the college search process.''
	\end{enumerate}
\end{enumerate}

Resources for financial aid: \vspace{-0.3cm}
\begin{enumerate} \itemsep -4pt
\item {\it Guide to Online Schools} [or {\it GuideToOnlineSchools.com}], {\it Financial Aid}. Available at: \url{http://www.guidetoonlineschools.com/financial-aid}; last accessed on August 25, 2010.
\item The Institute for College Access \& Success, {\it Links} [ Resources that provide information about student loans and student debt ]. Available at: \url{http://projectonstudentdebt.org/links.vp.html}; last accessed on September 4, 2010. [ Also, see \url{http://projectonstudentdebt.org/advice.vp.html} and \url{http://ticas.org/about.vp.html}. ]
\end{enumerate}


Information about colleges and universities: \vspace{-0.3cm}
\begin{enumerate} \itemsep -4pt
\item The Institute for College Access \& Success, {\it College InSight}. Available at: \url{http://college-insight.org/}; last accessed on September 4, 2010.
\item 
\end{enumerate}



%%%%%%%%%%%%%%%%%%%%%%%%%%%%%%%%%%%%%%%%%%%
\section{Outreach for Students in Colleges and Universities}
\label{outreachcollege}

Resources to reach out to students in colleges and universities: \vspace{-0.3cm}
\begin{enumerate} \itemsep -4pt
%%%%%%%%%%%%%%%%%%%%%%%%%%%%%
\item Film contests: \vspace{-0.3cm}
	\begin{enumerate} \itemsep -2pt
	\item Ed Wood Film Festival [@ USC]: \vspace{-0.2cm}
		\begin{enumerate} \itemsep -2pt
		\item Celebrating independent filmmaking at USC and named for the famous director, the Ed Wood Film Festival is put on by a committee of Residential Education staff members at New Residential College, chaired by the Cinema Floor RA's.
		\item Teams of students come together to obtain the year's secret theme in which to write, shoot, and edit their very own short film within 24 hours. A week later, the films are shown at USC's Norris Cinema and a panel of judges selects the Festival winners in a variety of categories.
		\item \url{http://sait.usc.edu/resed/Programs.aspx}
		\end{enumerate}
	\item Reel LA: Parkside International Film Festival [or USC Reel LA Film Festival at USC]; see \url{http://www-scf.usc.edu/~pirc/areagov/government.php}
	\end{enumerate}
%%%%%%%%%%%%%%%%%%%%%%%%%%%%%
\item residential education: \vspace{-0.3cm}
	\begin{enumerate} \itemsep -2pt
	\item Telluride Association: \vspace{-0.2cm}
		\begin{enumerate} \itemsep -2pt
		\item Information about how to reside at the Cornell Branch (also known as Telluride House or CBTA) and the Michigan Branch of Telluride Association, which are ``residential colleges'': \url{http://www.tellurideassociation.org/programs/university_students.html}
		\item Awards for residents at the Cornell or Michigan Branch: \url{http://www.tellurideassociation.org/programs/university_students/us_awards.html}
		\end{enumerate}
	\end{enumerate}
%%%%%%%%%%%%%%%%%%%%%%%%%%%%%
\item MathNerds: \vspace{-0.3cm}
	\begin{enumerate} \itemsep -2pt
	\item \url{http://www.mathnerds.com/}
	\item ``Provides free, discovery-based, mathematical guidance via an international, volunteer network of mathematicians.''
	\item If you have a mathematical problem to solve, you can ask mathematicans at {\it MathNerds} for help.
	\item They would require you to discuss your attempted approaches/solutions.
	\item If you have not made attempts to solve the problem, they will not give you much guidance.
	\item In addition, they cannot solve problems for you.
	\item They provide guidance for mathematical problems from K-12 material through undergraduate mathematics and statistics classes.
	\item They also provide help for selected topics in advanced mathematics classes (for graduate students).
	\end{enumerate}
%%%%%%%%%%%%%%%%%%%%%%%%%%%%%
\item Invent Now: \vspace{-0.3cm}
	\begin{enumerate} \itemsep -2pt
	\item 
	\end{enumerate}
\item Journal of Young Investigators (JYI): \vspace{-0.3cm}
	\begin{enumerate} \itemsep -2pt
	\item \url{http://www.jyi.org/}
	\item ``peer-reviewed journal for undergraduates''
	\item ``JYI's web journal (which is also called JYI) is dedicated to the presentation of undergraduate research in science, mathematics, and engineering. It publishes the best submissions from undergraduates, with an emphasis on both the quality of research and the manner in which it is communicated. The journal, JYI, also allows students to experience the other side of the scientific publication process: the review process. Students working with their faculty advisors review the work of their peers and determine whether that work is acceptable for publication in JYI.''
	\end{enumerate}
\item The Recording Academy: \vspace{-0.3cm}
	\begin{enumerate} \itemsep -2pt
	\item GRAMMY U: \vspace{-0.2cm}
		\begin{enumerate} \itemsep -2pt
		\item \url{http://www.grammy365.com/grammy-u}
		\item GRAMMY U is a unique and fast-growing community of full-time college students, primarily between the ages of 17 and 25,  who are pursuing a career in the recording industry.
		\item The Recording Academy created GRAMMY U to help prepare college students for their careers in the music industry through networking, educational programs and performance opportunities.
		\item GRAMMY U is designed to enhance students' current academic curriculum with access to recording industry professionals to give an ``out of classroom'' perspective on the recording industry.
		\end{enumerate}
	\end{enumerate}
%%%%%%%%%%%%%%%%%%%%%%%%%%%%%
\item --- --- --- --- --- --- --- --- --- --- --- --- --- --- --- --- --- --- --- --- --- --- --- --- --- --- --- --- --- --- ---
\item \colorbox{blue}{\bf Help for Underrepresented Minorities}
% Help for Underrepresented Minorities
\item INROADS, Inc.: \vspace{-0.3cm}
	\begin{enumerate} \itemsep -2pt
	\item Internships: \url{http://www.inroads.org/interns/internWhatItTakes.jsp}
	\end{enumerate}
\item The PhD Project: \vspace{-0.3cm}
	\begin{enumerate} \itemsep -2pt
	\item \url{http://www.phdproject.org/index.html}
	\item Program and informational network to encourage ``African-Americans, Hispanic-Americans and Native Americans'' to pursue Ph.D. programs in business and seek careers in academia.
	\item Annual PhD Project Conference: \vspace{-0.2cm}
		\begin{enumerate} \itemsep -2pt
		\item Conference: \vspace{-0.1cm}
			\begin{enumerate} \itemsep -1pt
			\item \url{http://www.phdproject.org/conference.html}
			\item \url{http://www.phdproject.org/conference_application.html}
			\item For prospective Ph.D. students in business to learn more about Ph.D. programs in business, the Ph.D. application process, and life in graduate school.
			\item Registration Policy: \vspace{-0.1cm}
				\begin{itemize} \itemsep -1pt
				\item If you are selected to attend the conference you will be required to pay a \$200 registration fee which can be processed via credit card during the registration process. All travel and conferences expenses will paid by The PhD Project (total conference expenses for hotel, meals, materials, and transportation are valued at approximately \$1,500 per invited attendee.) Your investment of the \$200 registration fee will be refunded if you enter a full-time, AACSB accredited business doctoral program within 3 years of attending the conference. 
				\item If you previously attended a PhD Project Conference, you may submit an application to be reviewed, however if you are selected to attend, The PhD Project will only cover hotel costs (shared room with another participant). You will be required to pay the registration and travel costs
				\end{itemize}
			\end{enumerate}
		\item Resources for Potential/Current Doctoral Students: \vspace{-0.1cm}
			\begin{enumerate} \itemsep -1pt
			\item \url{http://www.phdproject.org/resources.html}
			\item Information about good business schools that offer Ph.D. programs, preparation for the GMAT, and the life in graduate school as a Ph.D. student.
			\item Suggested Reading: \vspace{-0.1cm}
				\begin{itemize} \itemsep -1pt
				\item \url{http://www.phdproject.org/reading.html}
				\item Has information life in graduate school as a Ph.D. student, racial diversity/issues in higher education, job searching in academia, and work-life balance for female Ph.D. students.
				\end{itemize}
			\end{enumerate}
		\item The PhD Project Doctoral Student Association (DSA): \vspace{-0.1cm}
			\begin{enumerate} \itemsep -1pt
			\item The PhD Project network: \vspace{-0.1cm}
				\begin{itemize} \itemsep -1pt
				\item \url{http://www.myphdnetwork.org/}
				\item ``There are 5 discipline specific associations covering the major areas of business education: Accounting, Finance, Information Systems, Management, Marketing.''
				\end{itemize}
			\end{enumerate}
		\end{enumerate}
	\end{enumerate}
\item MS-to-Ph.D. program for underrepresented minorities at Fisk and Vanderbilt in certain areas of
science (including astronomy, material science, and physics)
\item Outreach programs for underrepresented minorities to help them get into medical (and/or graduate) schools. Search for ``PREP (Post-baccalaureate Research Education Programs),'' which have stipends. E.g., Georgetown University School of Medicine, and George Washington University's medical school
\item New York University: \vspace{-0.3cm}
	\begin{enumerate} \itemsep -2pt
	\item Leonard N. Stern School of Business: \vspace{-0.2cm}
		\begin{enumerate} \itemsep -2pt
		\item Stern Pre-Doctoral program: \url{http://www.stern.nyu.edu/AcademicPrograms/PhD/Pre-Doctoral/index.htm}
		\end{enumerate}
	\end{enumerate}
\end{enumerate}



%%%%%%%%%%%%%%%%%%%%%%%%%%%%%%%%%%%%%%%%%%%
\section{Science \& Engineering Outreach}
\label{stemoutreach}

%%%%%%%%%%%%%%%%%%%%%%%%%%%%%%%%%%%%%%%%%%%
\subsection{Precollege Science \& Engineering Outreach}
\label{stemoutreachk12}

Science and engineering outreach to high-school (and middle-school) students, and their parents, teachers, and career counselors: \vspace{-0.3cm}
\begin{enumerate} \itemsep -4pt
\item {\it MentorNet}: \vspace{-0.3cm}
	\begin{enumerate} \itemsep -2pt
	\item \url{http://www.mentornet.net/}
	\item Enables people to network with scientists, engineers, and professors in Science, Technology, Engineering, and Mathematics (STEM)
	\item Is very supportive of minorities, so that more minorities (particularly underrepresented minorities) can be attracted to STEM careers
	\end{enumerate}
\item International Science Olympiad (for high school students): \vspace{-0.3cm}
	\begin{enumerate} \itemsep -2pt
	\item International Olympiad in Informatics: \url{http://en.wikipedia.org/wiki/International_Olympiad_in_Informatics} and \url{http://www.ioinformatics.org/index.shtml}
	\item International Mathematical Olympiad: \url{http://www.imo-official.org/}
	\item International Physics Olympiad: \url{http://www.jyu.fi/tdk/kastdk/olympiads/}
	\item International Chemistry Olympiad: \url{http://www.icho.sk/}
	\item International Biology Olympiad: \url{http://www.ibo-info.org/}
	\item \url{http://scienceolympiads.org/}
	\end{enumerate}
\item International Astronomy Olympiad: \url{http://www.issp.ac.ru/iao/}
\item International Earth Science Olympiad: \url{http://en.wikipedia.org/wiki/International_Earth_Science_Olympiad}
\item International Junior Science Olympiad (for students younger than 15 years old): \url{http://www.ijso-official.org/home}
\item Teen Leadership Institute Science, Technology, Engineering, and Math (STEM) programs @ YWCA Greater Pittsburgh; see \url{http://www.ywcapgh.org/STEM_Programs.asp}
\item For Inspiration and Recognition of Science and Technology (FIRST): \url{http://www.usfirst.org/} (including resources and guides to mentoring); scholarships @ \url{http://www.usfirst.org/aboutus/content.aspx?id=508}; and robotics programs @ \url{http://www.usfirst.org/roboticsprograms/frc/default.aspx?=966}
\item Mac Hyman, ``Good Choices for Great Careers in the Mathematical Sciences,'' talk given at 2008 SIAM Annual Meeting. Available at: \url{http://client.blueskybroadcast.com/siam08/hyman/index.html}; last accessed on August 25, 2010. Also, see \url{http://meetings.siam.org/program.cfm?CONFCODE=AN08}, \url{http://www.siam.org/meetings/an08/program.php}, and \url{http://www.siam.org/meetings/an08/}.
\item {\it RoboCup}\texttrademark\ competitions: \vspace{-0.2cm}
	\begin{enumerate} \itemsep -2pt
	\item Junior category for K-12 students involves contests the these areas of challenges: \vspace{-0.1cm}
		\begin{enumerate} \itemsep -1pt
		\item soccer
		\item dance
		\item rescue operations
		\end{enumerate}
	\item \url{http://www.robocup.org/}
	\end{enumerate}
\item {\it Curriki}, which is an online educational resource for teachers, students, and parents in K-12: \url{http://www.curriki.org/xwiki/bin/view/Main/About}
%%%%%%%%%%%%%%%%%%%%%%%%%%%%%%%%%%%%%%%%
%%%%%%%%%%%%%%%%%%%%%%%%%%%%%%%%%%%%%%%%
\item Electrical and computer engineering and/or computer science: \vspace{-0.2cm}
	\begin{enumerate} \itemsep -2pt
	\item {\it TopCoder} coding and design contests: \vspace{-0.2cm}
		\begin{enumerate} \itemsep -2pt
		\item High School category
		\item \url{http://www.topcoder.com/}
		\end{enumerate}
	\item Student Cluster Competition (SCC): \vspace{-0.2cm}
		\begin{enumerate} \itemsep -2pt
		\item SCC is held at each (annual) SC conference, which is the International Conference for High Performance Computing, Networking, Storage, and Analysis. IEEE Computer Society and the Association for Computing Machinery are the sponsors for this conference.
		\item During SC10, teams consisting of six students, undergraduate and/or high school, will showcase the amazing power of clusters and the ability to utilize open source software to solve interesting and important problems. They will compete in real-time on the exhibit floor to run a workload of real-world applications on clusters of their own design while never exceeding the dictated power limit.
		\item During SC10 in New Orleans, teams will assemble, test and tune their machines and run the HPCC benchmarks until the starting bell rings on Monday night at the Exhibit Opening Gala where they will be given the competition data sets. In full view of conference attendees, teams will execute the prescribed workload while showing progress and science visualization output on large high-resolution displays in their areas. Teams race to correctly complete the greatest number of application runs during the competition period until the close of the exhibit floor on Wednesday evening.
		\item \url{http://sc10.supercomputing.org/?pg=studentcluster.html}
		\end{enumerate}
	\item Institute of Electrical and Electronics Engineers, IEEE: \vspace{-0.3cm}
		\begin{enumerate} \itemsep -2pt
		\item {\it IEEE Educational Activities} recommended resources: \url{http://www.ieee.org/education_careers/education/preuniversity/resources/index.html}
		\item Engineering Projects in Community Service (EPICS) in IEEE: \vspace{-0.2cm}
			\begin{enumerate} \itemsep -2pt
			\item High school students collaborate with college students in engineering projects to benefit the community
			\item \url{http://www.ieee.org/education_careers/education/preuniversity/epics_high.html}
			\end{enumerate}
		\item Talk given by John Cohn at the IEEE International Symposium on Circuits and Systems (ISCAS), May 18-21, 2008. The talk is titled, ``Kids these days. How we can inspire the next generation of Engineers and Scientists?'' See \url{http://ewh.ieee.org/soc/icss/IEEE-ISCAS-08-Tue-Keynote-JC/IEEE-ISCAS-08-Tue-Keynote-JC.HTML}. [ Alternatively, go to: IEEE Circuits and Systems Society, \url{http://www.ieee-cas.org/}: Select the ``Resources'' tab in the menu bar, and select the ``ISCAS Keynote Videos'' option. Click on the video link with the appropriate title. ]
		\end{enumerate}
	\item Association for Computing Machinery (ACM): \vspace{-0.2cm}
		\begin{enumerate} \itemsep -2pt
		\item Sanjeev Arora, Boaz Barak, and Luca Trevisan, ``Survey Papers and Essays,'' in {\it Theory Matters Wiki: Theoretical Computer Science (TCS) Advocacy Wiki}, SIGACT Committee for the Advancement of Theoretical Computer Science, ACM Special Interest Group on Algorithms and Computation Theory (SIGACT), Association for Computing Machinery, February 25, 2010. Available at: \url{http://theorymatters.org/pmwiki/pmwiki.php?n=Main.SurveyCollection}; last accessed on September 14, 2010.
		\end{enumerate}
	\item WGBH Educational Foundation: \vspace{-0.2cm}
		\begin{enumerate} \itemsep -2pt
		\item Dot Diva / New Image for Computing (NIC) initiative: \vspace{-0.1cm}
			\begin{enumerate} \itemsep -1pt
			\item \url{http://dotdiva.org/}
			\item Resource for parents and teachers: \url{http://dotdiva.org/parents.html}
			\end{enumerate}
		\end{enumerate}
	\item Silicon Valley StRUT: \vspace{-0.2cm}
		\begin{enumerate} \itemsep -2pt
		\item Students Recycling Used Technology, StRUT, Competition; StRUT Competition consists of: \vspace{-0.1cm}
			\begin{enumerate} \itemsep -1pt
			\item Disassemble and Reassemble A Computer 
			\item Create and Present a Powerpoint Presentation 
			\item Computer Parts Identification and Challenge Test  
			\item Team Quiz Bowl on Computer Technology and Related Subjects
			\item \url{http://www.svstrut.org/cms/content/section/1/5/}
			\item Teacher Resources: \url{http://www.svstrut.org/cms/component/option,com_weblinks/catid,11/Itemid,10/}
			\item [ Resources to Support ] Curriculum for Engineering and Computer Technology Education: \url{http://www.svstrut.org/cms/content/view/8/18/}
			\end{enumerate}
		\item \url{http://www.svstrut.org/cms/}
		\end{enumerate}
	\item Google Code Jam (programming contest): \url{http://code.google.com/codejam/} and \url{http://en.wikipedia.org/wiki/Google_Code_Jam}
	\item University of Illinois at Urbana-Champaign (UIUC): \vspace{-0.2cm}
		\begin{enumerate} \itemsep -2pt
		\item College of Engineering; Department of Computer Science: \vspace{-0.1cm}
			\begin{enumerate} \itemsep -1pt
			\item Outreach \& Diversity: \url{http://cs.illinois.edu/outreach}
			\item ChicTech: \url{http://cs.illinois.edu/outreach/chictech}
			\item Technical Ambassadors: \url{http://cs.illinois.edu/outreach/tac}
			\item Games4Girls: \url{http://cs.illinois.edu/outreach/games4girls}
			\item Workshops \& Camps: \url{http://cs.illinois.edu/outreach/k12}
			\item \url{http://cs.illinois.edu/outreach}
			\end{enumerate}
		\end{enumerate}
	\item Carnegie Mellon University: \vspace{-0.2cm}
		\begin{enumerate} \itemsep -2pt
		\item women@SCS School of Computer Science, Carnegie Mellon University: \vspace{-0.1cm}
			\begin{enumerate} \itemsep -1pt
			\item Papers: \url{http://women.cs.cmu.edu/Resources/Papers/}
			\item Alumnae Interviews / Profiles: \url{http://women.cs.cmu.edu/Who/Alumnae/alumInterviews.php}
			\item Job and Research Opportunities: \url{http://www.women.cs.cmu.edu/Resources/JobsResearch/}
			\item Career Advice: \url{http://women.cs.cmu.edu/Resources/JobsResearch/careeradvice.php}
			\item Other Sites: \url{http://www.women.cs.cmu.edu/Miscellaneous/Other/}
			\end{enumerate}
		\end{enumerate}
	\item {\it Quora}: \vspace{-0.2cm}
		\begin{enumerate} \itemsep -2pt
		\item ``If a 10-year-old wanted to start programming today, what language path would be the most valuable moving forward?'' Available online at: \url{http://www.quora.com/If-a-10-year-old-wanted-to-start-programming-today-what-language-path-would-be-the-most-valuable-moving-forward}; last accessed on November 23, 2010.
		\end{enumerate}
	\end{enumerate}
%%%%%%%%%%%%%%%%%%%%%%%%%%%%%%%%%%%%%%%%
%%%%%%%%%%%%%%%%%%%%%%%%%%%%%%%%%%%%%%%%
\item Engineering Education Service Center (EESC): \vspace{-0.3cm}
	\begin{enumerate} \itemsep -2pt
	\item Has lists of: \vspace{-0.2cm}
		\begin{enumerate} \itemsep -2pt
		\item Educational material: \vspace{-0.1cm}
			\begin{enumerate} \itemsep -1pt
			\item books
			\item DVDs
			\item resource kits for teachers
			\end{enumerate}
		\item engineering camps (for the summer in the United States): \url{http://www.engineeringedu.com/camps/}
		\item {\it Women in Engineering} programs at US engineering schools: \url{http://www.engineeringedu.com/wie.html}
		\item US engineering schools: \url{http://www.engineeringedu.com/engrschools.htm}
		\item competitions for youths, including high school students: \url{http://www.engineeringedu.com/competitions.html}
		\item online resources
		\item list of professional organizations in engineering (or engineering societies): \url{http://www.engineeringedu.com/soc1.html}
		\item scholarships: \url{http://www.engineeringedu.com/scholars.html}
		\end{enumerate}
	\item It has resources for K-12 students, and their teachers and parents. It also has information for girls who are seeking careers in engineering; in addition, it provides their parents and teachers with information to guide the girls.
	\item It runs a workshop (in the US) for mother-daughter pairs to encourage girls to pursue careers in engineering.
	\item \url{http://www.engineeringedu.com/}
	\end{enumerate}
\item TryNano.org: \vspace{-0.3cm}
	\begin{enumerate} \itemsep -2pt
	\item Information about educational opportunities and careers in nanotechnology and nanoscience
	\item \url{TryNano.org}
	\end{enumerate}
\item {\it Mathematical Association of America} (MAA): \vspace{-0.3cm}
	\begin{enumerate} \itemsep -2pt
	\item Middle/High School Students: \url{http://www.maa.org/students/middle_high/}
	\item Parents: \url{http://www.maa.org/students/Parents.html}
	\item MAA American Mathematics Competitions: \vspace{-0.2cm}
		\begin{enumerate} \itemsep -2pt
		\item {\it Students} [resources]. Available at: \url{http://amc.maa.org/a-activities/a4-for-students/s-index.shtml}; last accessed on September 2, 2010.
		\item It includes tips to help students do well in math contests and Olympiads, a reading list for students interested in mathematics, problems from past math contests and Olympiads, and other resources from the World Wide Web.
		\end{enumerate}
	\item {\it Fun Math Sites}. Available at: \url{http://www.maa.org/students/funsites.html}; last accessed on September 2, 2010.
	\item Special Interest Group on Mathematics and the Arts (SIGMAA-ARTS): Resources, see \url{http://myweb.cwpost.liu.edu/aburns/sigmaa-arts/resources.html}.
	\item Special Interest Group of the MAA on Quantitative Literacy (SIGMAA QL): \url{http://sigmaa.maa.org/ql/}
	\end{enumerate}
\item eGFI (Engineering, Go For It!): \vspace{-0.3cm}
	\begin{enumerate} \itemsep -2pt
	\item Provides information for students, parents, and teachers about educational pathways and careers in engineering
	\item \url{http://egfi-k12.org/}
	\end{enumerate}
\item {\it Sloan Career Cornerstone Center}: \vspace{-0.3cm}
	\begin{enumerate} \itemsep -2pt
	\item Career exploration resources in STEM (science, technology, engineering, mathematics, computing, and healthcare)
	\item \url{http://www.careercornerstone.org/}
	\end{enumerate}
\item {\it TryEngineering}: \vspace{-0.3cm}
	\begin{enumerate} \itemsep -2pt
	\item Career exploration resources for engineering
	\item \url{http://www.tryengineering.org/}
	\end{enumerate}
\item {\it Junior Engineering Technical Society, JETS}: \vspace{-0.3cm}
	\begin{enumerate} \itemsep -2pt
	\item Career exploration resources for engineering
	\item \url{http://www.jets.org/}
	\end{enumerate}
\item {\it American Society of Mechanical Engineers, ASME}: \vspace{-0.3cm}
	\begin{enumerate} \itemsep -2pt
	\item K-12 Student Resources: \url{http://www.asme.org/Communities/Students/K12/} and \url{http://www.asme.org/Education/PreCollege/EngineeringResources/}
	\item Engineering Camps: \url{http://www.asme.org/Communities/Students/K12/Camps.cfm}
	\end{enumerate}
\item BESTRobotics, Inc.: \vspace{-0.3cm}
	\begin{enumerate} \itemsep -2pt
	\item BEST (Boosting Engineering, Science, and Technology) competition: \vspace{-0.2cm}
		\begin{enumerate} \itemsep -2pt
		\item \url{http://best.eng.auburn.edu/}
		\item Hosted at Auburn University's Samuel Ginn College of Engineering
		\item BEST World Championship: \url{http://best.eng.auburn.edu/world-championship/}
		\end{enumerate}
	\end{enumerate}
\item {\it American Society of Civil Engineers, ASCE}: \vspace{-0.3cm}
	\begin{enumerate} \itemsep -2pt
	\item Outreach resource for K-12 students, and their parents and teachers
	\item \url{http://content.asce.org/asceville/index.html}
	\end{enumerate}
\item {\it Science.gov} (USA.gov for Science): Internship and Fellowship Opportunities in Science (for high school students); see \url{http://www.science.gov/internships/k-12.html}
\item {\it iTunes U}: \vspace{-0.3cm}
	\begin{enumerate} \itemsep -2pt
	\item {\it iTunes} is required to listen to or watch these lectures, talks, and presentations.
	\item Access {\it iTunes U} at: \url{http://deimos3.apple.com/indigo/main/main.html?v0=WWW-AMUS-ITUNESU070521-N48LX}
	\item WGBH's Teachers' Domain -- Boston's PBS Station: Video presentation on ``Engineering for the Red Planet''; see \url{http://deimos3.apple.com/WebObjects/Core.woa/Browse/wgbh.org.1416254059.01416254061.1416793683?i=1951581658}. Also, check out its video clip on ``Carbon Fiber Car of the Future''.
	\item {\it iTunes U} is a set of webcast and podcasts, where you can easily find audio and video clips for lectures, seminars, announcements, virtual tours, and so on. For example, some professors from schools like MIT or Berkeley will provide audio/video clips of their lectures on {\it iTunes U}.
	\item This can help in exploring different majors during the college application process, or before a college student declares her/his major(s). If a student is not sure if she/he wants to double major in deaf studies and linguistics, this student can check out some linguistics lectures from her/his (preferred) college/university, if it uses {\it iTunes U}, or those from other universities.
	\end{enumerate}
\item High School Ace's College Prep Guide: \url{http://highschoolace.com/ace/colleges.cfm}
\item {\it Dr. Sally Ride} (America�s first woman in space): \vspace{-0.3cm}
	\begin{enumerate} \itemsep -2pt
	\item {\it Sally Ride Science}'s resources for educators: \url{https://www.sallyridescience.com/for_educators}
	\item Sally Ride Science Educator Institutes (to educate K-12 teachers about science): \url{https://www.sallyridescience.com/for_educators/institutes}
	\item {\it Sally Ride Science Academy} helps teachers to increase their students' interest in science: \url{https://www.sallyridescience.com/academy}
	\item {\it Sally Ride Science}'s resources for teachers: \url{https://www.sallyridescience.com/resources}
	\item {\it Sally Ride Science Festivals} are events for girls from the $5^{th}$ grade to the $8^{th}$ grade: \url{https://www.sallyridescience.com/festivals}
	\item {\it Sally Ride Science Camps} are summer camps for girls from the $4^{th}$ grade to the $9^{th}$ grade: \url{http://www.sallyridecamps.com/}
	\item GRAIL MoonKAM: \vspace{-0.2cm}
		\begin{enumerate} \itemsep -2pt
		\item ``GRAIL MoonKAM (Moon Knowledge Acquired by Middle school students) is GRAIL's signature education and public outreach program.''
		\item ``GRAIL MoonKAM will engage middle schools across the country in the GRAIL mission and lunar exploration.''
		\item \url{https://www.grailmoonkam.com/}
		\end{enumerate}
	\item EarthKAM: \vspace{-0.2cm}
		\begin{enumerate} \itemsep -2pt
		\item EarthKAM (Earth Knowledge Acquired by Middle school students) is a NASA educational outreach program enabling students, teachers and the public to learn about Earth from the unique perspective of space.
		\item \url{https://earthkam.ucsd.edu/}
		\end{enumerate}
	\end{enumerate}
\item Andrew Rader Studios: \vspace{-0.3cm}
	\begin{enumerate} \itemsep -2pt
	\item Chem4Kids.com: \url{http://www.chem4kids.com/}
	\end{enumerate}
\item {\it American Association for the Advancement of Science, AAAS}: \vspace{-0.3cm}
	\begin{enumerate} \itemsep -2pt
	\item ENTRY POINT! for Students With Disabilities (in STEM): \url{http://www.aaas.org/careercenter/fellowships/} and \url{http://ehrweb.aaas.org/entrypoint/}
	\item AAAS Mass Media Science \& Engineering Fellows Program (for STEM grad students to intern in mass media companies): \url{http://www.aaas.org/programs/education/MassMedia/}
	\item Diversity Issues: \url{http://sciencecareers.sciencemag.org/career_magazine/diversity_issues/}
	\item Internships involving science and journalism, human rights, scientific freedom, responsibility, or law: \url{http://www.aaas.org/careercenter/} and \url{http://www.aaas.org/careercenter/internships/scienceminority.shtml} (AAAS Minority Science Writers Internship)
	\item Kinetic City: \url{http://www.kineticcity.com/}
	\end{enumerate}
\item {\it NASA} resources for students: \url{http://www.nasa.gov/audience/forstudents/index.html} and \url{http://www.nasa.gov/offices/education/programs/national/summer/education_resources/index.html} (NASA Summer of Innovation)
\item National Academy of Engineering, NAE: \vspace{-0.3cm}
	\begin{enumerate} \itemsep -2pt
	\item NAE Grand Challenges: \vspace{-0.2cm}
		\begin{enumerate} \itemsep -2pt
		\item Includes a list of NAE Grand Challenges, which indicate some of the challenges faced by people on a global scale that can be partially solved by engineers. This can be used to get children and youths to be excited about engineering.
		\item NAE Grand Challenges: \vspace{-0.1cm}
			\begin{enumerate} \itemsep -1pt
			\item Make solar energy economical
			\item Provide energy from fusion
			\item Develop carbon sequestration methods
			\item Manage the nitrogen cycle
			\item Provide access to clean water
			\item Restore and improve urban infrastructure
			\item Advance health informatics
			\item Engineer better medicines
			\item Reverse-engineer the brain
			\item Prevent nuclear terror
			\item Secure cyberspace
			\item Enhance virtual reality
			\item Advance personalized learning
			\item Engineer the tools of scientific discovery
			\end{enumerate}
		\item \url{http://www.engineeringchallenges.org/}
		\item NAE Grand Challenge K12 Partners Program: \vspace{-0.1cm}
			\begin{enumerate} \itemsep -1pt
			\item \url{http://www.grandchallengek12.org/about}
			\item 5-Part Make it Happen Plan: \url{http://www.grandchallengek12.org/5-part-plan}
			\end{enumerate}
		\end{enumerate}
	\item {\it National Academy of Engineering}'s technological literacy program for people (students, parents, and educators) to learn more about technology: \url{http://www.nae.edu/nae/techlithome.nsf}
	\item Greatest Engineering Achievements: \url{http://www.greatachievements.org/}
	\end{enumerate}
\item National Science Foundation: \vspace{-0.3cm}
	\begin{enumerate} \itemsep -2pt
	\item Broadening Participation in Computing (BPC): \vspace{-0.2cm}
		\begin{enumerate} \itemsep -2pt
		\item \url{http://www.bpcportal.org/}
		\item \url{http://www.bpcportal.org/bpc/shared/home.jhtml;jsessionid=0MIUYDR5U4ARXABAVRSSFEQ?_requestid=9445}
		\item \url{http://www.nsf.gov/funding/pgm_summ.jsp?pims_id=13510}
		\item \url{http://www.nsf.gov/funding/pgm_summ.jsp?pims_id=13510&org=NSF&sel_org=NSF&from=fund}
		\item ``Broadening Participation in Computing (BPC) is a NSF sponsored program with the goal of significantly increasing the number of underrepresented graduates in the computing disciplines, with an emphasis on women, persons with disabilities, and minorities (African Americans, Hispanics, American Indians, Alaska Natives, Native Hawaiians, and Pacific Islanders).''
		\item Broadening Participation in Computing Digital Library: \vspace{-0.1cm}
			\begin{enumerate} \itemsep -1pt
			\item \url{http://www.bpcportal.org/bpc/interdiscipline/dl_index.jhtml;jsessionid=ROYEHJV1UQYWNABAVRSSFEQ?comm=BPC}
			\item Includes resources for different target populations: \vspace{-0.1cm}
				\begin{itemize} \itemsep -1pt
				\item Women
				\item African Americans
				\item Hispanic Americans, or Latinas and Latinos
				\item People with disabilities
				\item Native Americans
				\end{itemize}
			\item It also includes resources for different topics, such as mentoring, recruitment, retention, and work-life balance.
			\end{enumerate}
		\item Alliances (other professional organizations): \url{http://www.bpcportal.org/bpc/comm/projects.jhtml}
		\end{enumerate}
	\item The National Science Digital Library (NSDL): \vspace{-0.2cm}
		\begin{enumerate} \itemsep -2pt
		\item \url{http://www.nsdl.org/} and \url{http://www.nsdl.org/browse/}
		\item ``The National Science Digital Library is a national network dedicated to advancing STEM teaching and learning for all learners, in both formal and informal settings, and the locus of activity for the National Science Foundation's National STEM Distributed Learning program.''
		\item Outreach materials: \vspace{-0.1cm}
			\begin{enumerate} \itemsep -1pt
			\item \url{http://www.nsdl.org/pd/?pager=materials}
			\item Has outreach materials for educators in K-12 and higher educational institutions.
			\end{enumerate}
		\item Resources for K-12 Teachers: \url{http://nsdl.org/resources_for/k12_teachers/}
		\item Resources for Librarians: \url{http://nsdl.org/resources_for/librarians/}
		\item Billingual Resources: \url{http://www.nsdlnetwork.org/collections/billingual-resources}
		\item NSDL on {\it iTunes U}: \url{http://www.nsdl.org/iTunesU/}
		\item Collections: \url{http://www.nsdl.org/browse/?subject=All}
		\item NSDL Pathways: \vspace{-0.1cm}
			\begin{enumerate} \itemsep -1pt
			\item \url{http://nsdl.org/about/?pager=pathways}
			\item ``Pathways are large projects that are aggregators and stewards of resources and services to broad categories of users---either discipline-focused (e.g. chemistry), or audience-focused (e.g. middle school educators), or resources of a specific type or format (e.g. multimedia content).''
			\item ``They are digital library portals developed and managed in partnership with organizations and institutions that have a history and expertise in serving their portal's target audiences.''
			\item ``They contribute metadata (descriptive information) about their resources to NSDL to make their resources searchable and discoverable via the NSDL.org portal, in addition to their own portals.''
			\end{enumerate}
		\item {\bf NSDL Science Literacy Maps}: \vspace{-0.1cm}
			\begin{enumerate} \itemsep -1pt
			\item \url{http://strandmaps.nsdl.org/}
			\item ``{\it NSDL Science Literacy Maps} are a tool for teachers and students to find resources that relate to specific science and math concepts. The maps illustrate connections between concepts as well as how concepts build upon one another across grade levels.''
			\end{enumerate}
		\item NSDL Professional Development: \url{http://www.nsdl.org/pd/}
		\item NSDL Technical Network Services: \vspace{-0.1cm}
			\begin{enumerate} \itemsep -1pt
			\item \url{http://www.nsdl.org/about/?pager=tns}
			\item \url{http://nsdlnetwork.org/}
			\item \url{http://nsdlnetwork.org/content/book/page/953/about-nsdl-technical-network-services}
			\end{enumerate}
		\item NSDL Resource Center: \url{http://nsdlnetwork.org/content/book/951/page/954/about-nsdl-resource-center}
		\end{enumerate}
	\end{enumerate}
\item {\it American Chemical Society} Science for Kids program (for students in K-12): \url{http://portal.acs.org/portal/acs/corg/content?_nfpb=true&_pageLabel=PP_TRANSITIONMAIN&node_id=878&use_sec=false&sec_url_var=region1&__uuid=984d4ee7-4214-4d35-9899-bc2f91dee58b}
\item {\it California Digital Educator Consortium}, ``Digital Educator,'' Digital Learning Center: \url{http://www.digitaleducator.com/}
\item Kenny Felder, ``Selected Other Educational Sites on the Web''. Available at: \url{http://www4.ncsu.edu/unity/lockers/users/f/felder/public/kenny/edulinks.html}; last accessed on August 28, 2010.
\item FHSST (Free High School Science Texts); free textbooks for grades 10-12 in Physics, Chemistry, and Mathematics. Available at: \url{http://www.fhsst.org/}; last accessed on August 28, 2010.
\item John Baez, {\it Usenet Physics FAQ}, Department of Mathematics, University of California, Riverside, September 2009. Available at: \url{http://math.ucr.edu/home/baez/physics/}; last accessed on August 28, 2010.
\item {\it American Society for Engineering Education}: \vspace{-0.3cm}
	\begin{enumerate} \itemsep -2pt
	\item Science and Engineering Apprenticeship Program (SEAP): \vspace{-0.2cm}
		\begin{enumerate} \itemsep -2pt
		\item ``The Science and Engineering Apprenticeship Program (SEAP) provides an opportunity for students to participate in research at a Department of Navy (DoN) laboratory during the summer.''
		\item ``The goals of SEAP are to encourage participating students to pursue science and engineering careers, to further their education via mentoring by laboratory personnel and their participation in research, and to make them aware of DoN Research and technology efforts, which can lead to employment within the DoN.''
		\item ``High school students who have completed at least Grade 9. A graduating senior is eligible to apply.''
		\item ``Must be 16 years of age for most laboratories. Some laboratories may accept a 15 year old applicant. Please check individual lab description for more details.''
		\item ``Applicants must be US citizens and participation by Permanent Resident Aliens is limited. Please check individual lab descriptions for participation of Permanent Resident Aliens.''
		\item \url{http://seap.asee.org/}
		\end{enumerate}
	\end{enumerate}
\item robots.net, {\it Robot Competitions} (list of robot competitions and contests) : \url{http://robots.net/rcfaq.html}
\item International Council on Systems Engineering (INCOSE): \vspace{-0.3cm}
	\begin{enumerate} \itemsep -2pt
	\item Careers in Systems Engineering: \url{http://www.incose.org/educationcareers/careersinsystemseng.aspx}
	\item Frequently Asked Questions for Students [about Systems Engineering]: \url{http://www.incose.org/educationcareers/faqsforstudents.aspx}
	\item What is Systems Engineering?: \url{http://www.incose.org/practice/whatissystemseng.aspx}
	\end{enumerate}
\item {\it National Society of Professional Engineers}: \vspace{-0.3cm}
	\begin{enumerate} \itemsep -2pt
	\item A Sightseer's Guide to Engineering: \url{http://www.engineeringsights.org/}
	\end{enumerate}
\item {\it Engineers Dedicated to a Better Tomorrow (a.k.a., DedicatedEngineers)}: \vspace{-0.3cm}
	\begin{enumerate} \itemsep -2pt
	\item The ``K-12 Crowd'' (Students, Teachers, Guidance Counselors and Parents): \url{http://www.dedicatedengineers.org/intro_for_K-12.htm}
	\item \url{http://www.dedicatedengineers.org/}
	\end{enumerate}
\item National Engineers Week Foundation: \vspace{-0.3cm}
	\begin{enumerate} \itemsep -2pt
	\item Discover Engineering: \url{http://www.discoverengineering.org/}
	\item Introduce A Girl to Engineering: \url{http://www.eweek.org/EngineersWeek/IntroduceAGirl.aspx}
	\item All About Engineering: \url{http://www.eweek.org/AboutEngineering/AboutEngineering.aspx}
	\end{enumerate}
\item University of California: \vspace{-0.3cm}
	\begin{enumerate} \itemsep -2pt
	\item The Coalition For Science After School: \vspace{-0.2cm}
		\begin{enumerate} \itemsep -2pt
		\item \url{http://afterschoolscience.org/}
		\item ``Promoting high-quality afterschool science'' ... ``The Coalition for Science After School envisions the day when young people from all backgrounds have access to high-quality science, technology, engineering and mathematics (STEM) learning beyond the classroom.''
		\item Tools for advocates--Championing afterschool science: \url{http://afterschoolscience.org/tools/}
		\item Program resources--Enhancing the quality of afterschool opportunities: \url{http://afterschoolscience.org/resources/}
		\item The National After School Science Directory: \vspace{-0.1cm}
			\begin{enumerate} \itemsep -1pt
			\item \url{http://afterschoolscience.org/directory/}
			\item ``The National After School Science Directory is a searchable database designed to increase access to high-quality science, technology, engineering and math (STEM) education beyond the classroom for youth and families across the nation. The Directory houses thousands of STEM opportunities, submitted by science centers, museums, schools and other youth-serving organizations. Search our Directory to view opportunities to connect the America's youth to high-quality STEM learning experiences.''
			\end{enumerate}
		\item Become an advocate: \url{http://afterschoolscience.org/tools/advocate.php}
		\item Funders (funding organizations/agencies): \url{http://afterschoolscience.org/tools/funders.php}
		\end{enumerate}
	\end{enumerate}
\item Harvey Mudd College: \vspace{-0.3cm}
	\begin{enumerate} \itemsep -2pt
	\item Francis Edward Su, {\it Math Fun Facts!}, Department of Mathematics, Harvey Mudd College: \url{http://www.math.hmc.edu/funfacts/}
	\end{enumerate}
\item Clay Mathematics Institute: \vspace{-0.3cm}
	\begin{enumerate} \itemsep -2pt
	\item Program in Mathematics for Young Scientists, PROMYS: \vspace{-0.2cm}
		\begin{enumerate} \itemsep -2pt
		\item \url{http://www.claymath.org/programs/outreach/PROMYS/}
		\item \url{http://math.bu.edu/people/promys/}
		\item \url{http://www.promys.org/}
		\end{enumerate}
	\item Ross Program (for pre-college students): \vspace{-0.2cm}
		\begin{enumerate} \itemsep -2pt
		\item \url{http://www.claymath.org/programs/outreach/ross/}
		\item \url{http://www.math.ohio-state.edu/ross/}
		\end{enumerate}
	\item CMI Summer Schools: \url{http://www.claymath.org/programs/summer_school/}
	\end{enumerate}
\item Consortium for Ocean Leadership: \vspace{-0.3cm}
	\begin{enumerate} \itemsep -2pt
	\item Oceans of Opportunity (for African American students in K-12, and colleges and universities -- includes undergraduates and grad students): \url{http://www.oceanleadership.org/education/diversity/oceans-of-opportunity/}
	\item The JOIDES Resolution (The JR) scientific research vessel [ Deep Earth Academy ]: \vspace{-0.2cm}
		\begin{enumerate} \itemsep -2pt
		\item Fun \& Games: \url{http://joidesresolution.org/node/53}
		\item Discovery Center: \url{http://joidesresolution.org/node/44}
		\item Just for Kids Blog: \url{http://joidesresolution.org/node/366}
		\end{enumerate}
	\item National Ocean Sciences Bowl (high school academic competition that provides a forum for talented students to test their knowledge of the marine sciences including biology, chemistry, physics, and geology): \vspace{-0.2cm}
		\begin{enumerate} \itemsep -2pt
		\item \url{http://www.nosb.org/}
		\item Career Resources: \url{http://www.nosb.org/ocean-careers/career-resources/}
		\end{enumerate}
	\item Integrated Ocean Drilling Program (IODP), IODP United States Implementing Organization (IODP-USIO): \vspace{-0.2cm}
		\begin{enumerate} \itemsep -2pt
		\item U.S.-sponsored Teacher at Sea Program (for US teachers to participate in seagoing research experiences aboard the JOIDES Resolution): \url{http://www.iodp-usio.org/Education/TAS.html}
		\end{enumerate}
	\item Careers: \url{http://www.oceanleadership.org/education/deep-earth-academy/students/careers/}
	\end{enumerate}
\item The Oceanography Society: \vspace{-0.3cm}
	\begin{enumerate} \itemsep -2pt
	\item Careers in Oceanography: Profiles, \url{http://www.tos.org/resources/career_profiles.html}
	\item Links [includes links to educational material for students in K-12]: \url{http://www.tos.org/resources/links.html}
	\end{enumerate}
\item American Geophysical Union: \vspace{-0.3cm}
	\begin{enumerate} \itemsep -2pt
	\item Bright Students Training as Research Scientists (Bright STaRS): \vspace{-0.2cm}
		\begin{enumerate} \itemsep -2pt
		\item \url{http://www.agu.org/education/diversity_programs/bstars.shtml}
		\item ``High school students participating in after-school and summer research experiences in the Earth and space sciences are invited to participate in the AGU Bright STaRS program. The Bright STaRS program provides a dedicated forum for $\sim$50 students to present their own research results to the scientific community and learn about exciting research, education, and career opportunities in the geosciences.''
		\end{enumerate}
	\end{enumerate}
\item American Geological Institute, AGI: \vspace{-0.3cm}
	\begin{enumerate} \itemsep -2pt
	\item AGI Education Department: \url{http://www.agiweb.org/geoeducation.html}
	\end{enumerate}
\item Society for Science \& the Public (SSP): \vspace{-0.3cm}
	\begin{enumerate} \itemsep -2pt
	\item Intel International Science \& Engineering Fair (Intel ISEF), which is a pre-college science competition: \url{http://www.societyforscience.org/isef/}
	\item Broadcom MASTERS\texttrademark\ competition (which stands for Broadcom Math, Applied Science, Technology and Engineering for Rising Stars): \vspace{-0.2cm}
		\begin{enumerate} \itemsep -2pt
		\item Is a U.S. ``national science, technology, engineering, and math competition for America's $6^{th}$, $7^{th}$, and $8^{th}$ graders.''
		\item \url{http://www.societyforscience.org/masters} or \url{http://www.broadcomfoundation.org/masters/}
		\end{enumerate} 
	\item Science resources: \url{http://www.societyforscience.org/resources}
	\item Science News: \url{http://www.sciencenews.org/}
	\item Science News for Kids (for ``children of ages 9-14, their teachers and their parents''): \url{http://www.societyforscience.org/sciencenewsforkids} and \url{http://www.sciencenewsforkids.org/}
	\end{enumerate}
\item Institute for Operations Research and the Management Sciences (INFORMS): \vspace{-0.3cm}
	\begin{enumerate} \itemsep -2pt
	\item Operations Research: The Science of Better, \url{http://www.scienceofbetter.org/}
	\end{enumerate}
\item Technion - Israel Institute of Technology: \vspace{-0.3cm}
	\begin{enumerate} \itemsep -2pt
	\item SciTech - the summer camp for talented students ($11^{th}$ and $12^{th}$ graders from all over the world): \url{http://www.scitech.technion.ac.il/}
	\end{enumerate}
\item USA Science \& Engineering Festival: \url{http://www.usasciencefestival.org/}
\item Girl Scouts: \vspace{-0.3cm}
	\begin{enumerate} \itemsep -2pt
	\item Girl Scouts of Western New York: \vspace{-0.2cm}
		\begin{enumerate} \itemsep -2pt
		\item STEM Resource Guide: \url{http://www.gswny.org/Data/Documents/STEM%2520Resource%2520Guide%25202010-Oct-11.pdf}
		\item Also, see \url{http://www.gswny.org/Programs/Awards/Gold/}; scroll to the bottom of the page and look under the subsection heading, ``Tell Us About Your Gold Award Project''
		\end{enumerate}
	\item Science, Technology, Engineering and Math (STEM): \url{http://www.girlscouts.org/program/program_opportunities/science/}
	\end{enumerate}
\item American Museum of Science and Energy (AMSE): \vspace{-0.3cm}
	\begin{enumerate} \itemsep -2pt
	\item \url{http://www.amse.org/}
	\item Owned by the US Department of Energy, and managed under Oak Ridge National Laboratory
	\item Educators: \url{http://www.amse.org/content.aspx?article=1140&parent=30}
	\item Educational Programs: \url{http://www.amse.org/content.aspx?article=1139&parent=30}
	\item Home school programs: \url{http://www.amse.org/content.aspx?article=1169&parent=30}
	\item Online resources: \url{http://www.amse.org/content.aspx?article=1170&parent=30}
	\end{enumerate}
\item Center for Energy Workforce Development (CEWD): \vspace{-0.3cm}
	\begin{enumerate} \itemsep -2pt
	\item Teachers and guidance counselors: \vspace{-0.2cm}
		\begin{enumerate} \itemsep -2pt
		\item \url{http://www.cewd.org/educators_index.asp}
		\item Lesson plans for teachers: \url{http://www.cewd.org/educators_lessonplans.asp}
		\end{enumerate}
	\item Parents: \url{http://www.cewd.org/parents_index.asp}
	\end{enumerate}
\item TryScience: \url{http://tryscience.net/tryscinetmain.nsf/Welcome?OpenPage}
\item The Dana Foundation: \vspace{-0.3cm}
	\begin{enumerate} \itemsep -2pt
	\item Brainy Kids: \vspace{-0.2cm}
		\begin{enumerate} \itemsep -2pt
		\item \url{http://www.dana.org/resources/brainykids/}
		\item Fun: \vspace{-0.1cm}
			\begin{enumerate} \itemsep -1pt
			\item \url{http://dana.org/resources/brainykids/detail.aspx?folder_id=104}
			\item Has interactive online games, activities, and fun quizzes on: \vspace{-0.1cm}
				\begin{itemize} \itemsep -1pt
				\item biology
				\item health
				\item neuroscience
				\item astronomy
				\item chemistry
				\item ecology
				\end{itemize}
			\end{enumerate}
		\item The Lab: \vspace{-0.1cm}
			\begin{enumerate} \itemsep -1pt
			\item \url{http://dana.org/resources/brainykids/detail.aspx?folder_id=106}
			\item Has maps of the brain, virtual dissections, resources for science fairs, and virtual microscopes
			\end{enumerate}
		\item Lesson Plans: \vspace{-0.1cm}
			\begin{enumerate} \itemsep -1pt
			\item \url{http://dana.org/resources/brainykids/detail.aspx?folder_id=108}
			\item Includes resources that cover the history of science and technology, lesson plans for K-12 science teachers, and science news for youths.
			\end{enumerate}
		\item The Mindboggling Workbook: \vspace{-0.1cm}
			\begin{enumerate} \itemsep -1pt
			\item \url{http://www.dana.org/uploadedFiles/The_Dana_Alliances/mindboggling_workbook.pdf}
			\item ``A fun-filled activity book about the brain for children in grades K-3 (ages 5-9). Provides an introduction to how the brain works, what the brain does, its importance, and how to take care of it.''
			\end{enumerate}
		\end{enumerate}
	\end{enumerate}
\item University of New Mexico: \vspace{-0.3cm}
	\begin{enumerate} \itemsep -2pt
	\item Department of Mathematics and Statistics: \vspace{-0.2cm}
		\begin{enumerate} \itemsep -2pt
		\item UNM - PNM Statewide Mathematics Contest (sponsored by the PNM Foundation): \url{http://mathcontest.unm.edu/}
		\end{enumerate}
	\end{enumerate}
\item Center for Energy Workforce (CEWD): \vspace{-0.3cm}
	\begin{enumerate} \itemsep -2pt
	\item Get Into Energy: \vspace{-0.2cm}
		\begin{enumerate} \itemsep -2pt
		\item \url{http://www.getintoenergy.com/index.asp} and \url{http://www.getintoenergy.com/careers.asp}
		\item Fun educational resources for students: \url{http://www.getintoenergy.com/students.asp}
		\item Career Quiz: \vspace{-0.1cm}
			\begin{enumerate} \itemsep -1pt
			\item \url{http://www.getintoenergy.com/search/careerquizj.asp}
			\item Help you find out more about career options in the energy field
			\end{enumerate}
		\item Career Resources: \vspace{-0.1cm}
			\begin{enumerate} \itemsep -1pt
			\item \url{http://www.getintoenergy.com/careerresources.asp}
			\item Has information on: \vspace{-0.1cm}
				\begin{itemize} \itemsep -1pt
				\item Training Programs (technical schools and colleges)
				\item Work-based Programs (apprenticeships and internships)
				\item Featured Employers
				\end{itemize}
			\end{enumerate}
		\item Skills Needed in the Energy Field: \vspace{-0.1cm}
			\begin{enumerate} \itemsep -1pt
			\item \url{http://www.getintoenergy.com/skills.asp}
			\item List skills for different kinds of jobs in the energy field
			\end{enumerate}
		\item Information for parents: \url{http://www.getintoenergy.com/Parents.asp}
		\item Information for teachers and guidance counselors: \url{http://www.getintoenergy.com/Educators.asp}
		\end{enumerate}
	\end{enumerate}
\item University of Utah: \vspace{-0.3cm}
	\begin{enumerate} \itemsep -2pt
	\item Department of Electrical and Computer Engineering: \vspace{-0.2cm}
		\begin{enumerate} \itemsep -2pt
		\item Prof. Cynthia Furse: \vspace{-0.1cm}
			\begin{enumerate} \itemsep -1pt
			\item Cynthia Furse, {\it K-12 Engineering Outreach}, August 2007. Available online at: \url{http://www.ece.utah.edu/~cfurse/K12.html}; last accessed on December 10, 2010.
			\item Cynthia Furse, {\it U Dream. U Design. U Create.}, Department of Electrical and Computer Engineering, University of Utah. Available online at: \url{http://www.ece.utah.edu/~cfurse/NSF/}; last accessed on December 10, 2010.
			\end{enumerate}
		\end{enumerate}
	\end{enumerate}
\item Society for Industrial and Applied Mathematics: \vspace{-0.3cm}
	\begin{enumerate} \itemsep -2pt
	\item Public Awareness: \vspace{-0.2cm}
		\begin{enumerate} \itemsep -2pt
		\item Math Competitions, \url{http://www.siam.org/publicawareness/competitions.php}
		\item Moody's Mega Math Challenge (M3 Challenge) is an applied mathematics competition for high school students. Available online at: \url{http://m3challenge.siam.org/}; last accessed on December 13, 2010.
		\item {\it Math Matters, Apply It!}: \url{http://www.siam.org/careers/matters.php}
		\item Nuggets: \url{http://www.siam.org/publicawareness/nuggets.php}
		\end{enumerate}
	\item Society for Industrial and Applied Mathematics, ``Unveiling Why Do Math,'' May 27, 2010. Available online at: \url{http://www.siam.org/about/news-siam.php?id=1741}; last accessed on December 13, 2010.
	\end{enumerate}
\item International Federation of Operational Research Societies (IFORS): \vspace{-0.3cm}
	\begin{enumerate} \itemsep -2pt
	\item Association of European Operational Research Societies (EURO): \vspace{-0.2cm}
		\begin{enumerate} \itemsep -2pt
		\item {\it What is Operational Research?}: \url{http://www.euro-online.org/display.php?pageid=197&}
		\item Applications of OR in music, literature, and aesthetics: \url{http://www.euro-online.org/display.php?pageid=211&}
		\item 24 Hours Operations Research: \url{http://www.24hor.org/}
		\item Branding OR: \url{http://www.euro-online.org/display.php?pageid=198&}
		\end{enumerate}
	\end{enumerate}
\item American Institute of Aeronautics and Astronautics (AIAA): \vspace{-0.3cm}
	\begin{enumerate} \itemsep -2pt
	\item Students \& Educators: \url{http://www.aiaa.org/content.cfm?pageid=5}
	\item Ask An Engineer: \url{http://www.aiaa.org/content.cfm?pageid=214}
	\item Kid's Place: \vspace{-0.2cm}
		\begin{enumerate} \itemsep -2pt
		\item \url{http://www.aiaa.org/content.cfm?pageid=473}
		\item Enjoy games, puzzles, fun experiments, teen-recommended books and movies, and more.
		\end{enumerate}
	\item History of Flight Timeline: \url{http://www.aiaa.org/content.cfm?pageid=260}
	\item Ask Polaris: \vspace{-0.2cm}
		\begin{enumerate} \itemsep -2pt
		\item \url{http://www.askpolaris.org/}
		\item Resource for career exploration in aerospace engineering and related fields
		\end{enumerate}
	\end{enumerate}
\item Massachusetts Institute of Technology: \vspace{-0.3cm}
	\begin{enumerate} \itemsep -2pt
	\item MIT School of Engineering: \vspace{-0.2cm}
		\begin{enumerate} \itemsep -2pt
		\item Lemelson-MIT Program: \vspace{-0.1cm}
			\begin{enumerate} \itemsep -1pt
			\item \url{http://web.mit.edu/invent/}
			\item Inventor's Handbook: \url{http://web.mit.edu/invent/h-main.html}
			\item Games \& Trivia; \url{http://web.mit.edu/invent/g-main.html}
			\item Links \& Resources: \url{http://web.mit.edu/invent/r-main.html}
			\end{enumerate}
		\end{enumerate}
	\end{enumerate}
\item BT Group plc: \vspace{-0.3cm}
	\begin{enumerate} \itemsep -2pt
	\item British Telecommunications plc (BT): \vspace{-0.2cm}
		\begin{enumerate} \itemsep -2pt
		\item BT Young Scientist \& Technology Exhibition: \vspace{-0.1cm}
			\begin{enumerate} \itemsep -1pt
			\item \url{http://www.btyoungscientist.com/}
			\item \url{http://www.btyoungscientist.com/all-you-need-to-know/}
			\item Science and technology fair for high/secondary school students in Ireland
			\end{enumerate}
		\end{enumerate}
	\end{enumerate}
\item NHS Medical Careers: \vspace{-0.3cm}
	\begin{enumerate} \itemsep -2pt
	\item \url{http://www.medicalcareers.nhs.uk/Default.aspx}
	\item Provides information about careers in medicine for prospective medical students, medical students, medical school graduates (or young medical professionals), (medical speciality) trainers, and medical specialists.
	\end{enumerate}
\item British Science Association: \vspace{-0.3cm}
	\begin{enumerate} \itemsep -2pt
	\item British Science Festival: \vspace{-0.2cm}
		\begin{enumerate} \itemsep -2pt
		\item \url{http://www.britishscienceassociation.org/web/BritishScienceFestival/AboutFestival/index.htm}
		\item Festival Student Bursaries: \url{http://www.britishscienceassociation.org/web/BritishScienceFestival/StudentBursaries/index.htm}
		\end{enumerate}
	\item National Science \& Engineering Week: \url{http://www.britishscienceassociation.org/web/NSEW/index.htm}
	\item Clubs, CREST Awards and Fairs (programs and activities for children and youth, 5-19 years of age): \url{http://www.britishscienceassociation.org/web/ccaf/index.htm}
	\item National Science \& Engineering Competition: \url{http://www.britishscienceassociation.org/web/NSEC/index.htm} and \url{http://www.thebigbangfair.co.uk/nsec/}
	\end{enumerate}
\item Research Councils UK (RCUK): \vspace{-0.3cm}
	\begin{enumerate} \itemsep -2pt
	\item \url{http://www.rcuk.ac.uk/per/Pages/Schools.aspx}
	\item Schoolscience: \vspace{-0.2cm}
		\begin{enumerate} \itemsep -2pt
		\item \url{http://www.schoolscience.co.uk/}
		\item For students and educators in K-12 to enrich the learning experiences of science topics, and help students connect classroom material to the real world.
		\item Teacher Zone - professional resources for teachers: \url{http://www.schoolscience.co.uk/teacher_zone.cfm}
		\item Interactive Learning Resources: \url{http://www.schoolscience.co.uk/interactives.cfm}
		\item Free Resources: \url{http://www.schoolscience.co.uk/freebies.cfm}
		\item Competitions: \url{http://www.schoolscience.co.uk/competitions.cfm}
		\item Research focus: \url{http://www.schoolscience.co.uk/research_focus.cfm}
		\item Resources on the World Wide Web: \url{http://www.schoolscience.co.uk/sciencelink.cfm}
		\end{enumerate}
	\item Researchers in Residence (RinR): \vspace{-0.2cm}
		\begin{enumerate} \itemsep -2pt
		\item \url{http://www.researchersinresidence.ac.uk/cms/schools-colleges/}
		\item For students in middle and high schools to job shadow (observe first-hand) a Ph.D. student or postdoctoral researcher in her/his research activities for up to a week, so that students can learn what doing research in her/his research area is like. In addition, the researcher would explain in laypeople's terms what her/his research is about. It can be considered as an externship program.
		\end{enumerate}
	\item Nuffield Bursaries: \vspace{-0.2cm}
		\begin{enumerate} \itemsep -2pt
		\item \url{http://www.nuffieldfoundation.org/capacity-building}
		\item \url{http://www.nuffieldfoundation.org/science-bursaries-schools-and-colleges}
		\item For high school juniors/seniors to pursue a research internship in science and engineering.
		\end{enumerate}
	\item CREST (Creativity in Science and Technology): \vspace{-0.2cm}
		\begin{enumerate} \itemsep -2pt
		\item \url{http://www.britishscienceassociation.org/web/ccaf/CREST/index.htm}
		\item Program to help students get engaged in a science or engineering project, where they learn how to solve real problems in science or engineering.
		\end{enumerate}
	\end{enumerate}
\item Nuffield Foundation: \vspace{-0.3cm}
	\begin{enumerate} \itemsep -2pt
	\item Science bursaries for schools and colleges: \url{http://www.nuffieldfoundation.org/science-bursaries-schools-and-colleges}
	\item Students: \url{http://www.nuffieldfoundation.org/students}
	\item Twenty First Century Science: \vspace{-0.2cm}
		\begin{enumerate} \itemsep -2pt
		\item \url{http://www.21stcenturyscience.org/}
		\item ``Twenty First Century Science is a set of GCSE science courses giving all 14-16-year-olds a worthwhile and inspiring experience of science. The strength of the programme is that it meets the needs, through flexible options, of those who will go on to be professional scientists and of those who will not.''
		\item The Courses: \url{http://www.21stcenturyscience.org/the-courses/}
		\item Assessment overview: \url{http://www.21stcenturyscience.org/assess/}
		\item Teaching resources: \url{http://www.21stcenturyscience.org/resources/}
		\end{enumerate}
	\item Science in Society: \vspace{-0.2cm}
		\begin{enumerate} \itemsep -2pt
		\item \url{http://www.scienceinsocietyadvanced.org/}
		\item ``Science in Society is an interesting and topical GCE advanced level course. It aims to develop the knowledge and skills that are needed for students to understand how science works, analyse contemporary issues involving science and technology and communicate their scientific appreciation and understanding to others.''
		\end{enumerate}
	\item Parents: \url{http://www.nuffieldfoundation.org/parents}
	\item Education: \url{http://www.nuffieldfoundation.org/education}
	\item Teachers (has excellent resources for science and mathematics): \url{http://www.nuffieldfoundation.org/teachers}
	\item Capacity building: \url{http://www.nuffieldfoundation.org/capacity-building}
	\end{enumerate}
\item The Story of Stuff Project (by Annie Leonard): \vspace{-0.3cm}
	\begin{enumerate} \itemsep -2pt
	\item \url{http://www.storyofstuff.com/}
	\item ``The Story of Stuff Project was created by Annie Leonard to leverage and extend the film's impact. We amplify public discourse on a series of environmental, social and economic concerns and facilitate the growing Story of Stuff community's involvement in strategic efforts to build a more sustainable and just world.''
	\item Resources: \vspace{-0.2cm}
		\begin{enumerate} \itemsep -2pt
		\item \url{http://www.storyofstuff.com/resources.php}
		\item The Story of Stuff Project PDFs: \url{http://www.storyofstuff.com/dl-pdfs.php}
		\item Teaching Tools: \url{http://www.storyofstuff.com/teach.php}
		\item More About Stuff: \url{http://www.storyofstuff.com/aboutstuff.php}
		\item Recommended Reading \& Bibliography: \url{http://www.storyofstuff.com/reading.php}
		\item Get Involved: \url{http://www.storyofstuff.com/getinvolved.php}
		\item Curricula: \url{http://storyofstuff.org/curricula.php}
		\end{enumerate}
	\end{enumerate}
\item Facing the Future: \vspace{-0.3cm}
	\begin{enumerate} \itemsep -2pt
	\item \url{http://www.facingthefuture.org/}
	\item ``{\it Facing the Future} engages students in learning by making academics relevant to their lives. We empower students to think critically, develop a global perspective, and participate in positive solutions for a sustainable future.''
	\item Curriculum Alignment with Education Standards: \url{http://www.facingthefuture.org/Curriculum/AlignmentwithEducationStandards/tabid/116/Default.aspx}
	\item Global Sustainability Curriculum Finder: \url{http://www.facingthefuture.org/Curriculum/FindCurriculumthatisRightforYou/tabid/68/Default.aspx}
	\item Download FREE Global Issues and Sustainability Curriculum: \url{http://www.facingthefuture.org/Curriculum/DownloadFreeCurriculum/tabid/114/Default.aspx}
	\item Classroom Examples: How Engaging Curriculum Can Help Address Classroom Challenges, \url{http://www.facingthefuture.org/ForEducators/ClassroomExamples/tabid/213/Default.aspx}
	\item Our Impact on Student Achievement: \url{http://www.facingthefuture.org/ForEducators/OurImpactonStudentAchievement/tabid/73/Default.aspx}
	\item Action Project Database: \url{http://www.facingthefuture.org/ServiceLearning/ActionProjectDatabase/tabid/94/Default.aspx}
	\item Service Learning Examples: \url{http://www.facingthefuture.org/ServiceLearning/ExamplesofStudentsTakingAction/tabid/147/Default.aspx}
	\item Curriculum: \url{http://www.facingthefuture.org/Curriculum/CurriculumHome/tabid/113/Default.aspx}
	\end{enumerate}
\item U.S. Department of Energy: \vspace{-0.3cm}
	\begin{enumerate} \itemsep -2pt
	\item Office of Science: \vspace{-0.2cm}
		\begin{enumerate} \itemsep -2pt
		\item U.S. Department of Energy (DOE) National Science Bowl\textregistered: \vspace{-0.1cm}
			\begin{enumerate} \itemsep -1pt
			\item \url{http://www.scied.science.doe.gov/nsb/default.htm}
			\item ``The U.S. Department of Energy (DOE) National Science Bowl\textregistered\ is a nationwide academic competition that tests students' knowledge in all areas of science. High school and middle school students are quizzed in a fast paced question-and-answer format similar to Jeopardy. Competing teams from diverse backgrounds are comprised of four students, one alternate, and a teacher who serves as an advisor and coach.''
			\end{enumerate}
		\item Argonne National Laboratory: \vspace{-0.1cm}
			\begin{enumerate} \itemsep -1pt
			\item Division of Educational Programs: \vspace{-0.1cm}
				\begin{itemize} \itemsep -1pt
				\item Newton BBS Ask A Scientist: \url{http://www.newton.dep.anl.gov/aas.htm}
				\end{itemize}
			\end{enumerate}
		\end{enumerate}
	\item Office of Energy Efficiency and Renewable Energy (EERE): \vspace{-0.2cm}
		\begin{enumerate} \itemsep -2pt
		\item Kids Saving Energy: \vspace{-0.1cm}
			\begin{enumerate} \itemsep -1pt
			\item \url{http://www.eere.energy.gov/kids/index.html}
			\item K-12 Lesson Plans \& Activities: \url{http://www1.eere.energy.gov/education/lessonplans/}
			\item Energy Savers: \url{http://www.energysavers.gov/}
			\item Games and activities: \url{http://www.eere.energy.gov/kids/games.html}
			\item Smart home: \url{http://www.eere.energy.gov/kids/smart_home.html}
			\item About renewable energy: \url{http://www.eere.energy.gov/kids/renergy.html}
			\end{enumerate}
		\end{enumerate}
	\item Contest \& Competitions: \url{http://www.energy.gov/contests&competitions.htm}
	\end{enumerate}
\item United States Department of Defense (DoD): \vspace{-0.3cm}
	\begin{enumerate} \itemsep -2pt
	\item National Defense Education Program; Defense Advanced Research Projects Agency (DARPA): \vspace{-0.2cm}
		\begin{enumerate} \itemsep -2pt
		\item Resource for Students: \url{http://www.ndep.us/GetInvoStu.aspx}
		\item Resource for Educators: \url{http://www.ndep.us/GetInvoTea.aspx}
		\end{enumerate}
	\end{enumerate}
\item Project Lead The Way: \vspace{-0.3cm}
	\begin{enumerate} \itemsep -2pt
	\item \url{http://www.pltw.org/}
	\item Getting started: \url{http://www.pltw.org/getting-started/getting-started}
	\item Program support: \url{http://www.pltw.org/program-support/program-support}
	\item Grants available to schools and teachers: \url{http://www.pltw.org/pltw-in-the-news/grants-available-schools-teachers-and-classrooms}
	\item Students: \url{http://www.pltw.org/students/students}
	\item Educators and Administrators: \url{http://www.pltw.org/educators-administrators/educators-administrators-overview}
	\item Parents: \url{http://www.pltw.org/parents/parents}
	\end{enumerate}
\item National Science Teachers Association: \vspace{-0.3cm}
	\begin{enumerate} \itemsep -2pt
	\item \url{http://www.exploravision.org/}
	\item Science competition for K-12 students
	\end{enumerate}
\item American Mathematical Society: \vspace{-0.3cm}
	\begin{enumerate} \itemsep -2pt
	\item Some career resources for mathematics: \url{http://e-math.ams.org/samplings/samplings}
	\end{enumerate}
\item American Institute of Physics (AIP): \vspace{-0.3cm}
	\begin{enumerate} \itemsep -2pt
	\item Physics Success Stories: \url{http://www.aip.org/success/}
	\item Physics is for you; Career Services Division: \vspace{-0.2cm}
		\begin{enumerate} \itemsep -2pt
		\item \url{http://www.aip.org/careersvc/pify/}
		\item Physicists at work: \url{http://www.aip.org/careersvc/pify/yellow.html}
		\end{enumerate}
	\item Society of Physics Students (SPS): \vspace{-0.2cm}
		\begin{enumerate} \itemsep -2pt
		\item Careers Using Physics (CUP): \vspace{-0.1cm}
			\begin{enumerate} \itemsep -1pt
			\item \url{http://www.spsnational.org/cup/}
			\item Advice: \url{http://www.spsnational.org/cup/advice/index.html}
			\item Resources: \url{http://www.spsnational.org/cup/resources.html}
			\item Preparing to Teach: \url{http://www.spsnational.org/cup/teach/index.html}
			\end{enumerate}
		\end{enumerate}
	\item ComPADRE Digital Library: \vspace{-0.2cm}
		\begin{enumerate} \itemsep -2pt
		\item \url{http://www.compadre.org/}
		\item The Physics Career Resource: \url{http://www.compadre.org/careers/}
		\end{enumerate}
	\item Career guidance for high school and undergraduate students: \url{http://www.aip.org/statistics/trends/career.html}
	\item Gayle A. Buck, Jack G. Hehn, and Diandra L. Leslie-Pelecky (Editors), ``The Role of Physics Departments in Preparing K-12 Teachers,'' American Institute of Physics. Available online at: \url{http://www.aip.org/education/teacherprep/}; last accessed on January 9, 2010.
	\item American Geophysical Union: \vspace{-0.2cm}
		\begin{enumerate} \itemsep -2pt
		\item Students \& Teachers: \url{http://www.agu.org/education/students_teachers.shtml}
		\item Diversity Programs: \url{http://www.agu.org/education/diversity_programs/}
		\end{enumerate}
	\end{enumerate}
\item Institute for Operations Research and the Management Sciences (INFORMS): \vspace{-0.3cm}
	\begin{enumerate} \itemsep -2pt
	\item Career FAQ's: \url{http://www.informs.org/Build-Your-Career/INFORMS-Student-Union/Career-Center/Career-FAQ-s}
	\end{enumerate}
\item American Institute of Mathematics: \vspace{-0.3cm}
	\begin{enumerate} \itemsep -2pt
	\item Math Teachers' Circle Network: \vspace{-0.2cm}
		\begin{enumerate} \itemsep -2pt
		\item Classroom Materials: \url{http://www.mathteacherscircle.org/resources/classroommaterials.html}
		\item Helpful Resources: \url{http://www.mathteacherscircle.org/resources/general.html}
		\end{enumerate}
	\item Resources for the Math Community: \vspace{-0.2cm}
		\begin{enumerate} \itemsep -2pt
		\item \url{http://www.aimath.org/mathcommunity/}
		\item David W. Farmer, ``The AIM REU: individual projects with a common theme,'' in the {\it Proceedings of the Conference on Promoting Undergraduate Research in Mathematics}, American Mathematical Society, 2006. Available online at: \url{http://www.aimath.org/mathcommunity/farmerREU.pdf}; last accessed on January 9, 2010. [ ``AIM Research Experience for Undergraduates (REU)'' ]
		\item Sally Koutsoliotas and David W. Farmer, ``Preparing students to give talks,'' American Institute of Mathematics. Available online at: \url{http://www.aimath.org/mathcommunity/studenttalks.pdf}; last accessed on January 9, 2010. [ ``Preparing students to give talks'' ]
		\end{enumerate}
	\end{enumerate}
\item Invent Now: \vspace{-0.3cm}
	\begin{enumerate} \itemsep -2pt
	\item Camp Invention: \vspace{-0.2cm}
		\begin{enumerate} \itemsep -2pt
		\item ``Summer enrichment program for children entering grades one through six.''
		\item ``The Camp Invention program instills vital 21st century life skills such as problem-solving and teamwork through hands-on fun!''
		\item Parents: \url{http://www.invent.org/camp/parents.aspx}
		\item Teachers: \url{http://www.invent.org/camp/teachers.aspx}
		\end{enumerate}
	\end{enumerate}
\item Massachusetts Institute of Technology: \vspace{-0.3cm}
	\begin{enumerate} \itemsep -2pt
	\item MIT School of Engineering: \vspace{-0.2cm}
		\begin{enumerate} \itemsep -2pt
		\item Lemelson-MIT Program: \vspace{-0.1cm}
			\begin{enumerate} \itemsep -1pt
			\item \url{http://web.mit.edu/invent/}
			\item Invention Dimension (for children): \url{http://web.mit.edu/invent/invent-main.html}
			\end{enumerate}
		\end{enumerate}
	\end{enumerate}
\item The Lemelson Foundation: \vspace{-0.3cm}
	\begin{enumerate} \itemsep -2pt
	\item \url{http://web.mit.edu/invent/w-foundation.html}
	\item Programs \& Grants: \url{http://www.lemelson.org/programs-grants}
	\item Grantmaking: \url{http://www.lemelson.org/grantmaking}
	\end{enumerate}
\item Smithsonian Institution: \vspace{-0.3cm}
	\begin{enumerate} \itemsep -2pt
	\item Smithsonian Kids: \url{http://www.si.edu/Kids}
	\item National Museum of American History: \vspace{-0.2cm}
		\begin{enumerate} \itemsep -2pt
		\item Lemelson Center for the Study of Invention and Innovation: \vspace{-0.1cm}
			\begin{enumerate} \itemsep -1pt
			\item \url{http://inventionatplay.org/index.html}
			\item Resources: \url{http://inventionatplay.org/resources.html}
			\end{enumerate}
		\end{enumerate}
	\end{enumerate}
%%%%%%%%%%%%%%%%%%%%%%%%%%%%%%%%%%%%%%%%
%%%%%%%%%%%%%%%%%%%%%%%%%%%%%%%%%%%%%%%%
\item Scholarships: \vspace{-0.3cm}
	\begin{enumerate} \itemsep -2pt
	\item IEEE Presidents' Scholarship: \url{http://www.ieee.org/education_careers/education/preuniversity/scholarship.html}
	\item ACM/SIGDA {\it P. O. Pistilli scholarship}: \vspace{-0.1cm}
		\begin{enumerate} \itemsep -1pt
		\item Supported by the Design Automation Conference which ACM/SIGDA sponsors, the objective of the P. O. Pistilli Scholarship is to increase the pool of professionals in Electrical Engineering and Computer Science from underrepresented groups (Women, African American, Hispanic, American Indian, and Disabled).
		\item Scholarships of \$4000 per year, renewable for up to 5 years, are awarded annually to 2-7 high school seniors from the above mentioned under represented groups who have a 3.00 GPA or better (on a 4.00 scale), have demonstrated high achievement in math and science courses, have expressed a strong desire to pursue careers in electrical engineering, computer engineering, or computer science, and who have demonstrated substantial financial need.
		\item U.S. citizenship is not required, but applicants must be U.S. residents when they apply and must plan to attend an accredited US college or university.
		\item \url{http://www.sigda.org/pistilli.html}
		\end{enumerate}
	\item Engineering Education Service Center (EESC): \url{http://www.engineeringedu.com/scholars.html}
	\item ASME-ASME Auxiliary FIRST Clarke Scholarships: \url{http://www.asme.org/Education/College/FinancialAid/High_School_Seniors.cfm} and \url{http://www.asme.org/Education/College/FinancialAid/Auxiliary_FIRST_Clarke.cfm}
	\item International Petroleum Institute�s High School Scholarships (for individuals entering a college program in engineering): \url{http://www.asme-ipti.org/public/pagscholarshipprograms.aspx}
	\item American Institute of Chemical Engineers (AIChE): \vspace{-0.2cm}
		\begin{enumerate} \itemsep -2pt
		\item Fuels and Petrochemicals Division Scholarship (for high school students entering undergraduate programs in engineering or science that are related to fuels and petrochemicals): \url{http://www.aiche.org/Students/Awards/F_PDScholarship.aspx}
		\item Minority Scholarship Awards for Incoming College Freshmen (for underrepresented minorities entering an undergraduate chemical engineering program): \url{http://www.aiche.org/Students/Awards/MinorityScholarshipAwardsIncomingFreshmen.aspx}
		\end{enumerate}
	\item Sallie Mae Fund: \vspace{-0.3cm}
		\begin{enumerate} \itemsep -2pt
		\item \url{http://www.thesalliemaefund.org/smfnew/index.html}
		\item List of scholarship resources: \url{http://www.thesalliemaefund.org/smfnew/sections/search.html}
		\item Top 10 Tips for Planning and Paying for College: \url{http://www.thesalliemaefund.org/smfnew/fin_aid/index.html}
		\item Scholarships: \url{http://www.thesalliemaefund.org/smfnew/scholarship/index.html} and \url{http://www.thesalliemaefund.org/smfnew/sections/apply.html}
		\item Important information for parents about saving for college and getting financial aid: \vspace{-0.2cm}
			\begin{enumerate} \itemsep -2pt
			\item \url{http://www.thesalliemaefund.org/smfnew/sections/download.html}
			\item This information is also available in Spanish. Summaries are also available in other languages such as: \vspace{-0.1cm}
				\begin{itemize} \itemsep -1pt
				\item French
				\item German
				\item Italian
				\item Korean
				\item Russian
				\item Simplified and Traditional Chinese
				\item Tagalog
				\item Vietnamese
				\end{itemize}
			\item Top 10 Tips for Planning and Paying for College: \url{http://www.thesalliemaefund.org/smfnew/fin_aid/index.html}
			\end{enumerate}
		\item Kids2College program: \url{http://www.thesalliemaefund.org/smfnew/initiatives/kidscollege.html}
		\item For African-American individuals entering college: \vspace{-0.2cm}
			\begin{enumerate} \itemsep -2pt
			\item Black College Dollars: \url{http://www.thesalliemaefund.org/smfnew/scholarship_directory/index.html}
			\item \url{http://www.thesalliemaefund.org/smfnew/initiatives/aa.html}
			\end{enumerate}
		\item For Hispanic Americans, or Latinos/Latinas: \vspace{-0.2cm}
			\begin{enumerate} \itemsep -2pt
			\item \url{http://www.thesalliemaefund.org/smfnew/pdf/Scholarship_Directory.pdf}
			\item Latino College Dollars: \url{http://www.latinocollegedollars.org/}
			\end{enumerate}
		\end{enumerate}
	\item {\it American Chemical Society}: \vspace{-0.3cm}
		\begin{enumerate} \itemsep -2pt
		\item ACS Scholars Program (for underrepresented minorities in, or entering, an undergraduate program in chemistry, biochemistry, or chemical engineering): \url{http://portal.acs.org/portal/acs/corg/content?_nfpb=true&_pageLabel=PP_SUPERARTICLE&node_id=1650&use_sec=false&sec_url_var=region1&__uuid=b3b583cf-18ae-4fb0-9375-33f75a0ccf49}
		\item Project SEED Scholarships (for high school seniors who have worked at least one summer at a science institute under the Project SEED program): \url{http://portal.acs.org/portal/acs/corg/content?_nfpb=true&_pageLabel=PP_SUPERARTICLE&node_id=2031&use_sec=false&sec_url_var=region1&__uuid=99bc6a62-3e78-4b2a-be3f-50b28f7ff265}
		\end{enumerate}
	\item The Posse Foundation: \url{http://www.possefoundation.org/}
	\item Hispanic Scholarship Fund (HSF) scholarship programs for high school students: \url{http://www.hsf.net/innerContent.aspx?id=426}
	\item Asian \& Pacific Islander American Scholarship Fund (APIASF): scholarships for individuals entering college as freshmen; see \url{http://www.apiasf.org/scholarship_apiasf.html}
	\item Nationally Coveted College Scholarships, Graduate School Fellowships \& Postdoctoral Awards: \url{http://scholarships.fatomei.com/}
	\item {\it SPIE} Scholarship Program (for high school students entering college to study optics, photonics, imaging, optoelectronics, or related program): \url{http://spie.org//x1733.xml?WT.svl=mddm14}
	\item Susan G. Komen for the Cure\textregistered: The Komen College Scholarship Program, \url{http://ww5.komen.org/ResearchGrants/CollegeScholarshipAward.html}
	\item National Society of Professional Engineers's list of scholarships for high school students: \url{http://www.nspe.org/Students/Scholarships/index.html}
	\item AWM Essay Contest: Biographies of Contemporary Women in Mathematics; see \url{http://www.awm-math.org/biographies/contest.html}
	\item National Engineers Week Future City Competition (students from $6^{th}$--$8^{th}$ grades): \url{http://www.futurecity.org/}
	\item National Ocean Sciences Bowl: \vspace{-0.2cm}
		\begin{enumerate} \itemsep -2pt
		\item \url{http://www.nosb.org/ocean-careers/}
		\item National Ocean Scholar Program (for high school seniors who are current/past participants of the Bowl, and are seeking a career in the ocean sciences or a marine-related field): \url{http://www.nosb.org/ocean-careers/national-ocean-scholar-program/}
		\end{enumerate}
	\item National Center for Women \& Information Technology (NCWIT): \vspace{-0.2cm}
		\begin{enumerate} \itemsep -2pt
		\item NCWIT Award for Aspirations in Computing (for young women in high school): \url{http://www.ncwit.org/work.awards.aspiration.html}
		\end{enumerate}
	\end{enumerate}
%%%%%%%%%%%%%%%%%%%%%%%%%%%%%%%%%%%%%%%%
%%%%%%%%%%%%%%%%%%%%%%%%%%%%%%%%%%%%%%%%
\item Resources for teachers/educators: \vspace{-0.3cm}
	\begin{enumerate} \itemsep -2pt
	\item Google: \vspace{-0.2cm}
		\begin{enumerate} \itemsep -2pt
		\item Google Teacher Academy (for teachers to learn how to use Google technologies to facilitate teaching): \url{http://www.google.com/educators/gta.html}
		\item Classroom activities (suggestions): \url{http://www.google.com/educators/activities.html}
		\end{enumerate}
	\item IEEE Teacher In-Service Program (TISP): \vspace{-0.2cm}
		\begin{enumerate} \itemsep -2pt
		\item \url{http://www.ieee.org/education_careers/education/preuniversity/tispt/index.html}
		\item Lesson Plans for Pre-university Instructors: \url{http://www.ieee.org/education_careers/education/preuniversity/resources/index.html}
		\end{enumerate}
	\item Global Challenge Award: \url{http://www.globalchallengeaward.org/display/public/Home}
	\item Teachers' Domain (to teach students about science, engineering, and the arts): \url{http://www.teachersdomain.org/}
	\item {\it TeachEngineering} digital library: \vspace{-0.2cm}
		\begin{enumerate} \itemsep -2pt
		\item The {\it TeachEngineering} digital library provides teacher-tested, standards-based engineering content for K-12 teachers engineering content for K12 teachers to use in science and math classrooms. Engineering lessons connect real-world experiences with curricular content already taught in K-12 classrooms. Mapped to educational content standards, {\it TeachEngineering}'s comprehensive curricula are hands-on, free, and relevant to children's daily lives.
		\item \url{http://www.teachengineering.com/index.php}
		\end{enumerate}
	\item Engineering Pathway: \url{http://www.engineeringpathway.com/ep/index.jhtml}
	\item {\it American Society of Mechanical Engineers, ASME}: \url{http://www.asme.org/Education/PreCollege/TeacherResources/}
	\item {\it National Science Foundation} resources for the K-12 classroom: \url{http://nsf.gov/news/classroom/engineering.jsp}
	\item {\it NASA}: \url{http://www.nasa.gov/audience/foreducators/index.html}
	\item The Mathematical Association of America: \vspace{-0.2cm}
		\begin{enumerate} \itemsep -2pt
		\item Pre-College Programs: \url{http://www.maa.org/funding/pre_college.html}. Also, see \url{http://www.maa.org/funding/undergraduate.html}.
		\item Special Interest Group of the Mathematical Association of America on the use of the World-Wide Web in Undergraduate Mathematics Instruction (Web SIGMAA). Available at: \url{http://math.chapman.edu/websigmaa/index.php/Main_Page}; last accessed on September 2, 2010.
		\item SIGMAA TAHSM (Teaching Advanced High School Mathematics). Available at: \url{http://sigmaa.maa.org/tahsm/}; last accessed on September 2, 2010.
		\item Special Interest Group on Statistics Education: \url{http://sigmaa.maa.org/stat-ed/}
		\end{enumerate}
	\item Math for America: \vspace{-0.2cm}
		\begin{enumerate} \itemsep -2pt
		\item M$f$A Master Teacher Fellowship program: \vspace{-0.1cm}
			\begin{enumerate} \itemsep -1pt
			\item The Math for America Master Teacher Fellowship program rewards exceptional public secondary school math teachers with a four-year Fellowship.
			\item M$f$A Master Teacher Fellowships are currently available in Berkeley, Boston and New York City.
			\item \url{http://www.mathforamerica.org/web/guest/master-teachers}
			\end{enumerate}
		\item M$f$A Early Career Fellows: \vspace{-0.1cm}
			\begin{enumerate} \itemsep -1pt
			\item The Math for America Early Career Fellowship is awarded to public secondary school math teachers early in their careers.
			\item The M$f$A Early Career Fellowship requires a commitment of four years and is available in New York City. 
			\item \url{http://www.mathforamerica.org/early-career-fellows}
			\end{enumerate}
		\item M$f$A Fellows: \vspace{-0.1cm}
			\begin{enumerate} \itemsep -1pt
			\item \url{http://www.mathforamerica.org/web/guest/mfa-fellows}
			\end{enumerate}
		\item Teachers resources: \url{http://www.mathforamerica.org/web/guest/teacher-resources} and \url{http://www.mathforamerica.org/teacher-resources/classroom} (classroom resources)
		\item Resources for professional development (teachers): \url{http://www.mathforamerica.org/teacher-resources/professional}
		\item \url{http://www.mathforamerica.org/home}
		\end{enumerate}
	\item Association for Symbolic Logic (ASL): \vspace{-0.2cm}
		\begin{enumerate} \itemsep -2pt
		\item Guidelines on Logic Education: \url{http://www.ucalgary.ca/aslcle/guidelines}
		\item Educational Logic Software: \url{http://www.ucalgary.ca/aslcle/logic-courseware}
		\end{enumerate}
	\item Consortium for Ocean Leadership: \vspace{-0.2cm}
		\begin{enumerate} \itemsep -2pt
		\item Educational Resources: \url{http://www.oceanleadership.org/gulf-oil-spill/educational-resources/}
		\item The JOIDES Resolution (The JR) scientific research vessel [ Deep Earth Academy ]: \vspace{-0.1cm}
			\begin{enumerate} \itemsep -1pt
			\item Teacher Resources (to teach students about geology and physical geography): \url{http://joidesresolution.org/node/46}
			\item Teachers at Sea/On-board Education Officer (for teachers to go on scientific expeditions on board): \url{http://joidesresolution.org/node/453}
			\end{enumerate}
		\item Integrated Ocean Drilling Program (IODP) -- IODP United States Implementing Organization (IODP-USIO): \vspace{-0.1cm}
			\begin{enumerate} \itemsep -1pt
			\item Teaching Materials: \url{http://www.iodp-usio.org/Education/educ.html}
			\end{enumerate}
		\item Deep Earth Academy (includes suggested ``curriculum and classroom activities for kindergarten through college level''): \vspace{-0.1cm}
			\begin{enumerate} \itemsep -1pt
			\item \url{http://www.oceanleadership.org/education/deep-earth-academy/}
			\item For Educators: \url{http://www.oceanleadership.org/education/deep-earth-academy/educators/}
			\end{enumerate}
		\end{enumerate}
	\item Virginia Institute of Marine Science (College of William and Mary): \vspace{-0.2cm}
		\begin{enumerate} \itemsep -2pt
		\item Bridge Ocean Education Teacher Resource Center: \url{http://web.vims.edu/bridge/?svr=www#}
		\end{enumerate}
	\item American Geological Institute: \vspace{-0.2cm}
		\begin{enumerate} \itemsep -2pt
		\item Awards for teachers: \url{http://www.agiweb.org/education/awards/index.html}
		\item Edward C. Roy, Jr. Award For Excellence in K-8 Earth Science Teaching (for middle school teachers in the US who are teaching earth science): \url{http://www.agiweb.org/education/awards/ed-roy/}
		\item Presidential Awards for Excellence in Mathematics \& Science Teaching, PAEMST (for kindergarten and K-12 teachers in the US who are teaching students about STEM fields): \url{http://www.agiweb.org/education/awards/paemst.html}
		\item National Association of Geoscience Teachers (NAGT) Outstanding Earth Science Teacher Award: \url{http://www.agiweb.org/education/awards/nagt.html}
		\item American Association of Petroleum Geologists' (AAPG) National Earth Science Teacher of the Year Award: \url{http://www.agiweb.org/education/awards/aapg.html}
		\item Curriculum Materials and Activities: \url{http://www.agiweb.org/education/curriculum/index.html}
		\item K-12 Professional Development Programs: \url{http://www.agiweb.org/education/pd/index.html}
		\item Educational Resources: \url{http://www.agiweb.org/education/resource/index.html}
		\end{enumerate}
	\item Institute for Broadening Participation: \vspace{-0.2cm}
		\begin{enumerate} \itemsep -2pt
		\item PathwaysToScience.org: \vspace{-0.1cm}
			\begin{enumerate} \itemsep -1pt
			\item For K-12 teachers (resources to encourage students to be interested in STEM): \url{http://www.pathwaystoscience.org/Teachers.asp}
			\end{enumerate}
		\end{enumerate}
	\item National Science Foundation: \vspace{-0.2cm}
		\begin{enumerate} \itemsep -2pt
		\item The National Science Digital Library (NSDL): \vspace{-0.1cm}
			\begin{enumerate} \itemsep -1pt
			\item Resources for K-12 Teachers: \url{http://nsdl.org/resources_for/k12_teachers/}
			\end{enumerate}
		\end{enumerate}
	\item National Academy of Engineering, NAE: \vspace{-0.2cm}
		\begin{enumerate} \itemsep -2pt
		\item NAE Grand Challenges: \vspace{-0.1cm}
			\begin{enumerate} \itemsep -1pt
			\item Includes a list of NAE Grand Challenges, which indicate some of the challenges faced by people on a global scale that can be partially solved by engineers. This can be used to get children and youths to be excited about engineering. 
			\item NAE Grand Challenges: \vspace{-0.1cm}
				\begin{itemize} \itemsep -1pt
				\item Make solar energy economical
				\item Provide energy from fusion
				\item Develop carbon sequestration methods
				\item Manage the nitrogen cycle
				\item Provide access to clean water
				\item Restore and improve urban infrastructure
				\item Advance health informatics
				\item Engineer better medicines
				\item Reverse-engineer the brain
				\item Prevent nuclear terror
				\item Secure cyberspace
				\item Enhance virtual reality
				\item Advance personalized learning
				\item Engineer the tools of scientific discovery
				\end{itemize}
			\item \url{http://www.engineeringchallenges.org/}
			\end{enumerate}
		\item NAE Grand Challenge K12 Partners Program: \vspace{-0.1cm}
			\begin{enumerate} \itemsep -1pt
			\item Can be used by schools/teachers to raise awareness of global challenges among students and to encourage students to plan career paths to tackle these challenges
			\item 5-Part Make it Happen Plan (includes suggested activities for students in elementary school to learn about basic science and engineering concepts that are relevant to solve the NAE grand challenges): \url{http://www.grandchallengek12.org/5-part-plan}
			\item \url{http://www.grandchallengek12.org/about}
			\end{enumerate}
		\item {\it National Academy of Engineering}'s technological literacy program for people (students, parents, and educators) to learn more about technology: \url{http://www.nae.edu/nae/techlithome.nsf}
		\end{enumerate}
	\item Women in Technology (WIT): \vspace{-0.2cm}
		\begin{enumerate} \itemsep -2pt
		\item Girls In Technology (GIT): \vspace{-0.1cm}
			\begin{enumerate} \itemsep -1pt
			\item Get Involved: \vspace{-0.1cm}
				\begin{itemize} \itemsep -1pt
				\item \url{http://www.girlsintechnology.org/getinvolved.cfm}
				\item Teacher: teach girls about IT as an after-school activity or in a summer camp session
				\item Assistant Teacher: Assist instructors in GIT sessions, after-school activities, or summer camp sessions
				\item Develop Curriculum: Develop a curriculum for a supported GIT educational program
				\item Mentor: Mentor a girl in one of [GIT's] supported programs
				\item Job Shadow: ``Let a girl shadow you at work''
				\item Guest Speaker: ``Speak to a group of girls on a topic both you and they enjoy, such as computers, technology, education, how to take apart computers, how to build a web site, etc.''
				\end{itemize}
			\end{enumerate}
		\end{enumerate}
	\item Organization for Economic Co-operation and Development (OECD): \vspace{-0.2cm}
		\begin{enumerate} \itemsep -2pt
		\item Programme for International Student Assessment (PISA): \vspace{-0.1cm}
			\begin{enumerate} \itemsep -1pt
			\item {\it PISA 2009 Results}. Available online at: \url{http://www.oecd.org/document/61/0,3343,en_32252351_32235731_46567613_1_1_1_1,00.html}; last accessed on December 10, 2010. [ Includes suggestions to improve learning outcomes, as well as education policies and practices. ]
			\end{enumerate}
		\end{enumerate}
	\item American Institute of Aeronautics and Astronautics (AIAA): \vspace{-0.2cm}
		\begin{enumerate} \itemsep -2pt
		\item K-12 Educators: \url{http://www.aiaa.org/content.cfm?pageid=208}
		\end{enumerate}
	\item Research Councils UK (RCUK): \vspace{-0.2cm}
		\begin{enumerate} \itemsep -2pt
		\item Biotechnology and Biological Sciences Research Council (BBSRC): \vspace{-0.1cm}
			\begin{enumerate} \itemsep -1pt
			\item Resources for schools and young people: \url{http://www.bbsrc.ac.uk/society/schools/schools-index.aspx}
			\item Teaching resources: publications and web-based activities: \vspace{-0.1cm}
				\begin{itemize} \itemsep -1pt
				\item Primary (ages 5-12) resources: \url{http://www.bbsrc.ac.uk/society/schools/primary/primary-index.aspx}
				\item Secondary (ages 12-16) and post-16 resources: \url{http://www.bbsrc.ac.uk/society/schools/secondary/secondary-index.aspx}
				\end{itemize}
			\end{enumerate}
		\end{enumerate}
	\item Nuffield Foundation: \vspace{-0.2cm}
		\begin{enumerate} \itemsep -2pt
		\item Education: \url{http://www.nuffieldfoundation.org/education}
		\item Teachers: \vspace{-0.1cm}
			\begin{enumerate} \itemsep -1pt
			\item (Excellent) resources in science and mathematics: \url{http://www.nuffieldfoundation.org/teachers}
			\item \url{http://www.nuffieldfoundation.org/teachers-0}
			\end{enumerate}
		\end{enumerate}
	\item Wellcome Trust: \vspace{-0.2cm}
		\begin{enumerate} \itemsep -2pt
		\item Education resources: \url{http://www.wellcome.ac.uk/Education-resources/index.htm}
		\item {\it yourgenome.org}: \vspace{-0.1cm}
			\begin{enumerate} \itemsep -1pt
			\item \url{http://www.yourgenome.org/}
			\item Resources for teachers about genomics: \url{http://www.yourgenome.org/landing_teachers.shtml}
			\end{enumerate}
		\item Network of Science Learning Centers (Science Learning Centers): \vspace{-0.1cm}
			\begin{enumerate} \itemsep -1pt
			\item \url{https://www.sciencelearningcentres.org.uk/}
			\item Awards and Bursaries: \vspace{-0.1cm}
				\begin{itemize} \itemsep -1pt
				\item \url{https://www.sciencelearningcentres.org.uk/centres/national/awards-and-bursaries}
				\item \url{https://www.sciencelearningcentres.org.uk/about/impact-awards}
				\end{itemize}
			\item Resource collections: \url{https://www.sciencelearningcentres.org.uk/resources}
			\item Curriculum resources for primary, secondary, and tertiary education: \url{https://www.sciencelearningcentres.org.uk/curriculum}
			\end{enumerate}
		\end{enumerate}
	\end{enumerate}
%%%%%%%%%%%%%%%%%%%%%%%%%%%%%%%%%%%%%%%%
%%%%%%%%%%%%%%%%%%%%%%%%%%%%%%%%%%%%%%%%
\item Underrepresented minorities: \vspace{-0.3cm}
	\begin{enumerate} \itemsep -2pt
	\item University of Washington: \vspace{-0.2cm}
		\begin{enumerate} \itemsep -2pt
		\item Department of Computer Science and Engineering: \vspace{-0.1cm}
			\begin{enumerate} \itemsep -1pt
			\item {\it AccessComputing}: \vspace{-0.1cm}
				\begin{itemize} \itemsep -1pt
				\item \url{http://www.washington.edu/accesscomputing/}
				\item Has resources to help students with disabilities to pursue ``undergraduate and graduate degrees and careers in computing fields''.
				\item It runs the ``Summer Academy for Advancing Deaf \& Hard of Hearing in Computing'' for youths who are hearing impaired: \url{http://www.washington.edu/accesscomputing/dhh/academy/index.html}
				\end{itemize}
			\end{enumerate}
		\end{enumerate}
	%%%%%%%%%%%%%%%%%%%%%%%%%
	\item Engineer Girl: \vspace{-0.2cm}
		\begin{enumerate} \itemsep -2pt
		\item Resources for students, parents, and teachers to encourage girls to explore careers and educational opportunities in engineering
		\item Created by the National Academy of Sciences and The National Academy of Engineering
		\item Contests for K-12 students: \url{http://www.engineergirl.org/?id=4436}
		\item \url{http://www.engineergirl.org/}
		\end{enumerate}
	\item Engineering Your Life: \url{http://www.engineeryourlife.org/}
	\item GirlGeeks: \url{http://www.girlgeeks.org/}
	\item {\it Women in Science, Technology, Engineering, and Mathematics ON THE AIR!}: \vspace{-0.2cm}
		\begin{enumerate} \itemsep -2pt
		\item Audio resources that describe stories about women in science, technology, engineering, and mathematics (STEM) fields
		\item \url{http://www.womeninscience.org/}
		\end{enumerate}
	\item {\it Women Scientists in History}: \url{http://www.hypatiamaze.org/}
	\item Association for Women in Mathematics (AWM): \vspace{-0.2cm}
		\begin{enumerate} \itemsep -2pt
		\item \url{http://www.awm-math.org/}
		\item Education: \vspace{-0.1cm}
			\begin{enumerate} \itemsep -1pt
			\item \url{http://sites.google.com/site/awmmath/awm-resources/education}
			\item Includes information for students in middle school, high school, college and university (including graduate students). It also includes information for parents and teachers/educators.
			\end{enumerate}
		\item Women in Math, Science, and Society: \url{http://sites.google.com/site/awmmath/women-in-math-science-and-society}
		\item Essay contest on biographies of contemporary women in mathematics: \url{http://sites.google.com/site/awmmath/programs/essay-contest}
		\end{enumerate}
	\item Women in Technology (WIT): \vspace{-0.2cm}
		\begin{enumerate} \itemsep -2pt
		\item Girls in Technology: \vspace{-0.1cm}
			\begin{enumerate} \itemsep -1pt
			\item \url{http://www.girlsintechnology.org/}
			\item WIT Education Foundation: provides educational programs for girls in technology
			\item TeamBusiness Fundraiser: ``A combined fundraiser and program for girls in Grades 9-12 across the Metro DC area. Each year, up to forty girls participate with mentors and WIT volunteers in a full-day business simulation workshop conducted by TeamBusiness USA. The teams competed as companies, learning how to run a technology company in a fun and exciting simulation environment.''
			\item Hispanic Youth Foundation: ``In 2005, GIT established a partnership with the Hispanic Youth Foundation (HYF) and provided a grant to fund HYF�s innovative Laptops for Learning Dollars program, providing laptops and Internet connections for elementary and middle school students and their families in Arlington County and the City of Manassas.''
			\item Empower Girls -- CLCP Clubs: ``Empower Girls after-school programs were held at Hybla Valley Elementary School and Sacramento Community Center. GIT/WITEF provided funding to run these programs in conjunction with the Fairfax County Computer Learning Center Partnership (CLCP). The selected centers serve economically challenged communities in Fairfax County.''
			\end{enumerate}
		\end{enumerate}
	%%%%%%%%%%%%%%%%%%%%%%%%%
	\item National Society of Black Engineers (NSBE) competitions for high school/K-12 students: \url{http://www.nsbe.org/Programs/Competitions/NSBE-Jr-.aspx}
	\item The Society of Mexican American Engineers and Scientists (MAES): MAES PreCollege Outreach Programs, \url{http://www.maes-natl.org/index.php?module=ContentExpress&func=display&ceid=16&meid=236}
	\item {\it Center for the Advancement of Hispanics in Science and Engineering Education} (CAHSEE): \vspace{-0.2cm}
		\begin{enumerate} \itemsep -2pt
		\item STEM - The Science, Technology, Engineering \& Mathematics Institute (for students from grades 5 through 11): \url{http://www.cahsee.org/2programs/stem.asp.htm}
		\item YEP - Young Educators Program (fellows would learn how to train students in the aforementioned STEM Institute): \url{http://www.cahsee.org/2programs/yep.asp.htm}
		\item CAYSA - Central American Young Scholar Awards: \url{http://www.cahsee.org/2programs/caysa.asp.htm}. ``The CAYSA ceremonies honor more than 60 Washington, D.C. area high school seniors of Central American descent who have demonstrated remarkable success throughout all four years of high school. Students must be of Central American descent and have at least a 3.0 gpa.''
		\item Scholarships: \url{http://www.cahsee.org/6resources/scholarships.asp.htm}
		\item \url{http://www.cahsee.org/about/about.asp.htm}
		\end{enumerate}
	%%%%%%%%%%%%%%%%%%%%%%%%%
	\item International Computer Science Institute (UC Berkeley): \vspace{-0.2cm}
		\begin{enumerate} \itemsep -2pt
		\item Berkeley Foundation for Opportunities in Information Technology, BFOIT: \vspace{-0.1cm}
			\begin{enumerate} \itemsep -1pt
			\item BFOIT Programs for women and underrepresented minorities (African Americans and Chicanos/Latinos) in middle/high school who are interested in electrical/computer engineering and computer science careers: \url{http://www.bfoit.org/programs.html}
			\end{enumerate}
		\end{enumerate}
	\item Institute for Broadening Participation: \vspace{-0.2cm}
		\begin{enumerate} \itemsep -2pt
		\item PathwaysToScience.org: \vspace{-0.1cm}
			\begin{enumerate} \itemsep -1pt
			\item PathwaysToScience.org is a portal website supporting pathways to the STEM fields: science, technology, engineering, and mathematics.
			\item Particular emphasis is placed on connecting traditionally underrepresented groups with STEM programs and resources, including funding and mentoring opportunities. 
			\item For K-12 students: \url{http://www.pathwaystoscience.org/K12.asp}
			\item STEM Resources by Institution (colleges, universities, and US national research laboratories): \url{http://www.pathwaystoscience.org/Institution.asp}
			\item profiles of people and programs in STEM: \vspace{-0.3cm}
				\begin{itemize} \itemsep -2pt
				\item \url{http://www.pathwaystoscience.org/Profiles.asp}
				\item Find out about the career paths of underrepresented minorities in STEM
				\item Find out about programs that are offered by institutions for underrepresented minorities in STEM
				\end{itemize}
			\item Directory of partners (organizations that cooperate with or support the Institute for Broadening Participation): \url{http://www.pathwaystoscience.org/Partners.asp}
			\item Additional resources: \url{http://www.pathwaystoscience.org/Ideaexchange.asp}
			\end{enumerate}
		\item Maine Pathways to STEM (Science, Technology, Engineering \& Mathematics): \vspace{-0.1cm}
			\begin{enumerate} \itemsep -1pt
			\item \url{http://www.mainestem.org/}
			\item K-12 Teachers \& University Faculty: \url{http://www.mainestem.org/METeachersFaculty.asp}
			\item K-12 STEM Resources: \url{http://www.mainestem.org/MEK12.asp}
			\end{enumerate}
		\end{enumerate}
	\item Building Engineering and Science Talent, BEST: \vspace{-0.2cm}
		\begin{enumerate} \itemsep -2pt
		\item \url{http://www.bestworkforce.org/}
		\item Publications: \url{http://www.bestworkforce.org/publications.htm}
		\item List of programs to help underrepresented minority students in K-12 schools explore careers in STEM: \url{http://www.bestworkforce.org/links.htm}
		\end{enumerate}
	\item American Indian Science and Engineering Society (AISES): \vspace{-0.2cm}
		\begin{enumerate} \itemsep -2pt
		\item Pre-college programs: \vspace{-0.1cm}
			\begin{enumerate} \itemsep -1pt
			\item \url{http://www.aises.org/Programs}
			\item Resources: \url{http://www.aises.org/Programs/Resources}
			\end{enumerate}
		\end{enumerate}
	\end{enumerate}
\end{enumerate}







%%%%%%%%%%%%%%%%%%%%%%%%%%%%%%%%%%%%%%%%%%%
\subsection{Science \& Engineering Outreach for Undergraduates, Grad Students, \& Postdocs}
\label{stemoutreachcollegegradsch}


Science, mathematics, and engineering outreach to undergraduates, graduate students, and postdocs: \vspace{-0.3cm}
\begin{enumerate} \itemsep -4pt
\item Mac Hyman, ``Good Choices for Great Careers in the Mathematical Sciences,'' talk given at 2008 SIAM Annual Meeting. Available at: \url{http://client.blueskybroadcast.com/siam08/hyman/index.html}; last accessed on August 25, 2010. Also, see \url{http://meetings.siam.org/program.cfm?CONFCODE=AN08}, \url{http://www.siam.org/meetings/an08/program.php}, and \url{http://www.siam.org/meetings/an08/}.
\item {\it Accreditation.org}: \vspace{-0.3cm}
	\begin{enumerate} \itemsep -2pt
	\item Information about the accreditation of engineering degree programs around the world
	\item \url{http://www.accreditation.org/}
	\end{enumerate}
\item John Baez, ``How to Learn Math and Physics,'' Department of Mathematics, University of California, Riverside, December 24, 2007. Available at: \url{http://math.ucr.edu/home/baez/books.html}; last accessed on August 28, 2010.
\item {\it MentorNet}: \vspace{-0.3cm}
	\begin{enumerate} \itemsep -2pt
	\item \url{http://www.mentornet.net/}
	\item Enables people to network with scientists, engineers, and professors in Science, Technology, Engineering, and Mathematics (STEM)
	\item Is very supportive of minorities, so that more minorities (particularly underrepresented minorities) can be attracted to STEM careers
	\end{enumerate}
\item {\it The Indus Entrepreneurs (TiE)} for networking among high-tech entrepreneurs, start-up co-founders, venture capitalists, and angel investors: \url{http://www.tie.org/}
\item National Academy of Engineering, NAE: \vspace{-0.3cm}
	\begin{enumerate} \itemsep -2pt
	\item Includes a list of NAE Grand Challenges, which can provide some suggestions for research trajectories
	\item Summit Series on the Grand Challenges: Includes the National Grand Challenges Summits
	\item \url{http://www.engineeringchallenges.org/}
	\end{enumerate}
\item {\it National Society of Professional Engineers}: \vspace{-0.3cm}
	\begin{enumerate} \itemsep -2pt
	\item Student Resources: \vspace{-0.2cm}
		\begin{enumerate} \itemsep -2pt
		\item \url{http://www.nspe.org/Students/Resources/index.html}
		\item An Employment Guidelines Checklist for the Engineer Job Applicant: \url{http://www.nspe.org/Students/Resources/checklist.html}
		\end{enumerate}
	\item Career Center: \url{http://www.nspe.org/CareerCenter/index.html}
	\item A Sightseer's Guide to Engineering: \url{http://www.engineeringsights.org/}
	\end{enumerate}
\item {\it JustGarciaHill} ``Study Skills for Budding Scientists'': \url{http://www.justgarciahill.org/index.php/science-study-skills.html}
\item {\it NASA} resources for students: \vspace{-0.3cm}
	\begin{enumerate} \itemsep -2pt
	\item \url{http://www.nasa.gov/audience/forstudents/index.html}
	\item NASA University Student Launch Initiative, or USLI: \url{http://www.nasa.gov/offices/education/programs/descriptions/University_Student_Launch_Initiative.html}
	\end{enumerate}
\item {\it iTunes U}: \vspace{-0.3cm}
	\begin{enumerate} \itemsep -2pt
	\item {\it iTunes} is required to listen to or watch these lectures, talks, and presentations.
	\item Access {\it iTunes U} at: \url{http://www.apple.com/education/itunes-u/} or \url{http://deimos3.apple.com/indigo/main/main.html?v0=WWW-AMUS-ITUNESU070521-N48LX}
	\item {\it iTunes U} is a set of webcast and podcasts, where you can easily find audio and video clips for lectures, seminars, announcements, virtual tours, and so on. For example, some professors from schools like MIT or Berkeley will provide audio/video clips of their lectures on {\it iTunes U}.
	\item This can help in exploring different majors before a college student declares her/his major(s). If a student is not sure if she/he wants to double major in deaf studies and linguistics, this student can check out some linguistics lectures from her/his (preferred) college/university, if it uses {\it iTunes U}, or those from other universities.
	\end{enumerate}
\item Harvey Mudd College: \vspace{-0.3cm}
	\begin{enumerate} \itemsep -2pt
	\item Francis Edward Su, {\it Math Fun Facts!}, Department of Mathematics, Harvey Mudd College: \url{http://www.math.hmc.edu/funfacts/}
	\end{enumerate}
\item Engineering Pathway: \url{http://www.engineeringpathway.com/ep/index.jhtml}
\item Rochester Institute of Technology, ``Biology \& Biotechnology Paid Co-op/Internships for 2011,'' Department of Biological Sciences, Rochester Institute of Technology: \url{http://people.rit.edu/gtfsbi/Symp/summer.htm}
\item {\it Mathematical Association of America (MAA)} information on educational pathways and career opportunities: \vspace{-0.3cm}
	\begin{enumerate} \itemsep -2pt
	\item Undergraduate Students: \url{http://www.maa.org/students/undergrad/}
	\item Graduate Students: \url{http://www.maa.org/students/grad/}
	\item Underrepresented Groups: \url{http://www.maa.org/programs/underrep.html}
	\item Mathematical Association of America (MAA) MathFest (for students in mathematics): \url{http://www.maa.org/mathfest/}
	\item MAA Online Columns: \url{http://www.maa.org/news/columns.html}
	\end{enumerate}
\item New Zealand Institute of Mathematics and its Applications (NZIMA): \vspace{-0.3cm}
	\begin{enumerate} \itemsep -2pt
	\item {\it MathsReach}: Careers (information about careers based on a higher education in mathematics or related field): \url{http://www.mathsreach.org/Careers}
	\end{enumerate}
\item {\it Engineers Dedicated to a Better Tomorrow (a.k.a., DedicatedEngineers)}: \vspace{-0.3cm}
	\begin{enumerate} \itemsep -2pt
	\item [Resources for] College Students and Faculty/Staff Members: \url{http://www.dedicatedengineers.org/intro_for_college.htm}
	\item \url{http://www.dedicatedengineers.org/}
	\end{enumerate}
\item American Institute of Physics: \vspace{-0.3cm}
	\begin{enumerate} \itemsep -2pt
	\item GradschoolShopper.com: \vspace{-0.2cm}
		\begin{enumerate} \itemsep -2pt
		\item \url{http://www.gradschoolshopper.com/}
		\item ``Find information on graduate programs in physics, astronomy, and other physical sciences''
		\end{enumerate}
	\item Career guidance for high school and undergraduate students: \url{http://www.aip.org/statistics/trends/career.html}
	\item American Geophysical Union: \vspace{-0.2cm}
		\begin{enumerate} \itemsep -2pt
		\item Diversity Programs: \url{http://www.agu.org/education/diversity_programs/}
		\end{enumerate}
	\end{enumerate}
\item {\it icademic.org} resources for the life sciences and engineering: \url{http://www.icademic.org/}
\item The Oceanography Society: \vspace{-0.3cm}
	\begin{enumerate} \itemsep -2pt
	\item Hands-On Oceanography: peer-reviewed activities appropriate for undergraduate and/or graduate classes in oceanography, \url{http://www.tos.org/hands-on/index.html}
	\end{enumerate}
%%%%%%%%%%%%%%%%%%%%%%%%%%%%%%%%%%%%%%%
%%%%%%%%%%%%%%%%%%%%%%%%%%%%%%%%%%%%%%%
\item outreach activities (including mentoring) to students in K-12: \vspace{-0.3cm}
	\begin{enumerate} \itemsep -2pt
	\item Research Councils UK (RCUK): \vspace{-0.2cm}
		\begin{enumerate} \itemsep -2pt
		\item Researchers in Residence (RinR): \vspace{-0.1cm}
			\begin{enumerate} \itemsep -1pt
			\item \url{http://www.researchersinresidence.ac.uk/cms/}
			\item \url{http://www.researchersinresidence.ac.uk/cms/researchers/}
			\item Mentor middle and high school students who are job shadowing (observing you first-hand) in your research activities for up to a week, so that they can learn what doing research in your research area is like. You should explain in laypeople's terms what your research is about. That is, be a mentor for the externships of middle and high school students.
			\end{enumerate}
		\end{enumerate}
	\end{enumerate}
%%%%%%%%%%%%%%%%%%%%%%%%%%%%%%%%%%%%%%%
%%%%%%%%%%%%%%%%%%%%%%%%%%%%%%%%%%%%%%%
\item competitions: \vspace{-0.3cm}
	\begin{enumerate} \itemsep -2pt
	\item Invent Now, Inc.: \vspace{-0.2cm}
		\begin{enumerate} \itemsep -2pt
		\item Collegiate Inventors Competition: \url{http://www.invent.org/collegiate/} [ Resources for {\color{blue} Patent Search Strategy} are available. \colorbox{blue}{\bf This is the ultimate competition for US students in science and engineering.} ]
		\end{enumerate}
	\item INFORMS Doing Good with Good OR - Student Competition: \vspace{-0.2cm}
		\begin{enumerate} \itemsep -2pt
		\item Doing Good with Good OR-Student Competition is held each year to identify and honor outstanding projects in the field of operations research and the management sciences conducted by a student or student group that have a significant societal impact.
		\item \url{http://www.informs.org/Recognize-Excellence/INFORMS-Prizes-Awards/Doing-Good-with-Good-OR}
		\end{enumerate}
	\item AWM Essay Contest: Biographies of Contemporary Women in Mathematics; see \url{http://www.awm-math.org/biographies/contest.html}
	\item American Society of Mechanical Engineers (ASME): \vspace{-0.2cm}
		\begin{enumerate} \itemsep -2pt
		\item Student Design Competition: \url{http://www.asme.org/Events/Contests/DesignContest/Student_Design_Competition.cfm}
		\item ASME Student Mechanism and Robot Design Competition: \url{http://www.asme.org/Events/Contests/Student_Mechanism_Robot_2.cfm}
		\end{enumerate}
	\item American Institute of Chemical Engineers (AIChE) competitions: \url{http://www.aiche.org/Students/Awards/index.aspx}
	\item Association for Unmanned Vehicle Systems International (AUVSI): \vspace{-0.2cm}
		\begin{enumerate} \itemsep -2pt
		\item AUVSI Student Competitions: \vspace{-0.1cm}
			\begin{enumerate} \itemsep -1pt
			\item \url{http://www.auvsi.org/AUVSI/AUVSI/Home/Default.aspx}, or \url{http://www.auvsi.org/}
			\item Annual Intelligent Ground Vehicle Competition (IGVC): \url{http://www.igvc.org/}
			\item Annual Student Unmanned Air System (SUAS) Competition: \url{http://65.210.16.57/studentcomp2010/default.html}
			\item International Aerial Robotics Competition (IARC): \url{http://iarc.angel-strike.com/}
			\item AUVSI and ONR's International Autonomous Surface Vehicle (ASV) Competition [ASVC]
			\item AUVSI Foundation and ONR's (U.S. Office of Naval Research) 4th International RoboBoats Competition: \url{http://www.auvsifoundation.org/AUVSI/FOUNDATION/Competitions/ASVCompetition/Default.aspx?C=00000000-0000-0000-0000-000000000000}
			\item AUVSI Foundation and ONR's (U.S. Office of Naval Research) International RoboSub Competition (or AUVSI and ONR's International Autonomous Underwater Vehicle Competition): \url{http://www.auvsifoundation.org/AUVSI/FOUNDATION/Competitions/AUVCompetition/Default.aspx}
			\item ONR: U.S. Office of Naval Research
			\end{enumerate}
		\end{enumerate}
	\item American Institute of Aeronautics and Astronautics (AIAA): \vspace{-0.2cm}
		\begin{enumerate} \itemsep -2pt
		\item Design Competitions: \url{http://www.aiaa.org/content.cfm?pageid=210}
		\end{enumerate}
	\item National Aeronautics and Space Administration: \vspace{-0.2cm}
		\begin{enumerate} \itemsep -2pt
		\item NASA's Langley Research Center: \vspace{-0.1cm}
			\begin{enumerate} \itemsep -1pt
			\item SpaceTech Engineering Design Challenge: \url{http://spacetech.larc.nasa.gov}
			\end{enumerate}
		\end{enumerate}
	\item American Concrete Institute (ACI): \vspace{-0.2cm}
		\begin{enumerate} \itemsep -2pt
		\item Competitions: \url{http://www.concrete.org/STUDENTS/st_competitions.htm}
		\end{enumerate}
	\end{enumerate}
%%%%%%%%%%%%%%%%%%%%%%%%%%%%%%%%%%%%%%%
%%%%%%%%%%%%%%%%%%%%%%%%%%%%%%%%%%%%%%%
\item underrepresented minorities: \vspace{-0.3cm}
	\begin{enumerate} \itemsep -2pt
	\item The Society of Women Engineers: \url{http://societyofwomenengineers.swe.org/}
	\item Association for Women in Science (AWIS): \url{http://www.awis.org/} and \url{http://www.awis.affiniscape.com/displaycommon.cfm?an=1&subarticlenbr=19}
	\item Association for Women in Mathematics (AWM): \vspace{-0.2cm}
		\begin{enumerate} \itemsep -2pt
		\item \url{http://www.awm-math.org/}
		\item Education: \vspace{-0.1cm}
			\begin{enumerate} \itemsep -1pt
			\item \url{http://sites.google.com/site/awmmath/awm-resources/education}
			\item Includes information for students in middle school, high school, college and university (including graduate students). It also includes information for parents and teachers/educators.
			\end{enumerate}
		\item Career advice and opportunities: \url{http://sites.google.com/site/awmmath/awm-resources/career}
		\item Women in Math, Science, and Society: \url{http://sites.google.com/site/awmmath/women-in-math-science-and-society}
		\item Essay contest on biographies of contemporary women in mathematics: \url{http://sites.google.com/site/awmmath/programs/essay-contest}
		\end{enumerate}
	\item Sigma Delta Epsilon-Graduate Women in Science (GWIS): \url{http://www.gwis.org/}
	\item Society of Hispanic Professional Engineers (SHPE): \vspace{-0.2cm}
		\begin{enumerate} \itemsep -2pt
		\item Advancing Hispanic Excellence in Technology, Engineering, Math and Science (AHETEMS) Foundation: \url{http://www.ahetems.org/}
		\item AHETEMS Scholarship Program: \url{http://www.ahetems.org/scholarships/}
		\item Graduate \& Young Professional Fellowship Program (encourage young professionals to engage in {\bf public policy}): \url{http://www.ahetems.org/graduate/graduate-young-professional-fellowship-program/}
		\item SHPE/GEM Fellowship (for graduate students in STEM at GEM Member Universities): \url{http://www.ahetems.org/graduate/shpe-gem-graduate-award/}. See \url{http://www.gemfellowship.org/gem-universities/university-members} for a list of GEM member universities.
		\item Internship opportunities: \url{http://www.ahetems.org/scholar-internships/}
		\item \url{http://oneshpe.shpe.org/wps/portal/national}
		\end{enumerate}
	\item National Society of Black Engineers (NSBE): \vspace{-0.2cm}
		\begin{enumerate} \itemsep -2pt
		\item Scholarships: \url{http://www.nsbe.org/Programs/Scholarships.aspx}
		\item Competitions for undergraduates and graduate students: \url{http://www.nsbe.org/Programs/Competitions/Collegiate.aspx}
		\item \url{http://www.nsbe.org/}
		\end{enumerate}
	\item The Society of Mexican American Engineers and Scientists (MAES): \vspace{-0.2cm}
		\begin{enumerate} \itemsep -2pt
		\item MAES Undergraduate and Graduate Outreach Programs (including ``GRE/Graduate Application Fee Waivers''): \url{http://www.maes-natl.org/index.php?module=ContentExpress&func=display&ceid=90&meid=238}
		\item Scholarships \& Awards: \url{http://www.maes-natl.org/index.php?meid=328}
		\item MAES Scholarship Program: \url{http://www.maes-natl.org/index.php?module=ContentExpress&func=display&ceid=518&meid=241}
		\end{enumerate}
	\item SACNAS (Society for Advancement of Chicanos and Native Americans in Science): \vspace{-0.2cm}
		\begin{enumerate} \itemsep -2pt
		\item Scholarships: \url{http://www.sacnas.org/webadindex.cfm?webadcategory_id=7}
		\item Fellowships: \url{http://www.sacnas.org/webadIndex.cfm?webadcategory_id=5}
		\end{enumerate}
	\item {\it Center for the Advancement of Hispanics in Science and Engineering Education} (CAHSEE): \vspace{-0.2cm}
		\begin{enumerate} \itemsep -2pt
		\item YESP - Young Engineers \& Scientists Program: \url{http://www.cahsee.org/2programs/yesp.asp.htm}. ``This program places talented Hispanic college students in the research labs of government agencies.''
		\item Scholarships: \url{http://www.cahsee.org/6resources/scholarships.asp.htm}
		\end{enumerate}
	\item American Geophysical Union: \vspace{-0.2cm}
		\begin{enumerate} \itemsep -2pt
		\item Has a list of organizations for specific underrepresented ethnic-minority groups in the geosciences and physics: \vspace{-0.1cm}
			\begin{enumerate} \itemsep -1pt
			\item \url{http://www.agu.org/education/diversity_programs/}
			\item These organizations may have information about scholarships, fellowships, and educational material for K-12 and college students.
			\end{enumerate}
		\end{enumerate}
	\item Institute for Broadening Participation: \vspace{-0.2cm}
		\begin{enumerate} \itemsep -2pt
		\item Minorities Striving and Pursuing Higher Degrees of Success in Earth System Science (MS PHD'S\textregistered) initiative: \vspace{-0.1cm}
			\begin{enumerate} \itemsep -1pt
			\item \url{http://www.msphds.org/}
			\item Prospective Students/Mentees: \url{http://www.msphds.org/prospective.asp}
			\item For MS PHD'S Students: \url{http://www.msphds.org/students.asp}
			\end{enumerate}
		\item PathwaysToScience.org: \vspace{-0.1cm}
			\begin{enumerate} \itemsep -1pt
			\item Resources for undergraduate students: \url{http://www.pathwaystoscience.org/Undergrads.asp}
			\item Resources for graduate students: \url{http://www.pathwaystoscience.org/Grad.asp}
			\item Resources for postdocs: \url{http://www.pathwaystoscience.org/Postdocs_portal.asp}
			\item STEM Resources by Institution (colleges, universities, and US national research laboratories): \url{http://www.pathwaystoscience.org/Institution.asp}
			\item Additional resources: \url{http://www.pathwaystoscience.org/Ideaexchange.asp}
			\end{enumerate}
		\item National Alliance for Doctoral Studies in the Mathematical Sciences: \vspace{-0.1cm}
			\begin{enumerate} \itemsep -1pt
			\item \url{http://www.mathalliance.org/}
			\item Student/Alliance Scholars: \url{http://www.mathalliance.org/scholars.asp}
			\item Alliance Mentors / Alliance Undergraduate Mentors: \url{http://www.mathalliance.org/mentors.asp}
			\item Alliance Programs: \url{http://www.mathalliance.org/programs.asp}
			\end{enumerate}
		\item Alliances for Graduate Education and the Professoriate (AGEP): \vspace{-0.1cm}
			\begin{enumerate} \itemsep -1pt
			\item \url{http://www.agep.us/}
			\end{enumerate}
		\item Maine Pathways to STEM (Science, Technology, Engineering \& Mathematics): \vspace{-0.1cm}
			\begin{enumerate} \itemsep -1pt
			\item \url{http://www.mainestem.org/}
			\item K-12 Teachers \& University Faculty: \url{http://www.mainestem.org/METeachersFaculty.asp}
			\item Graduate \& Undergraduate Students: \url{http://www.mainestem.org/MEUndergradGrad.asp}
			\end{enumerate}
		\end{enumerate}
	\item ARTSI (Advancing Robotics Technology for Societal Impact) Alliance: \vspace{-0.2cm}
		\begin{enumerate} \itemsep -2pt
		\item \url{http://artsialliance.org/}
		\item ``The ARTSI (Advancing Robotics Technology for Societal Impact) Alliance is a collaborative education and research project centered around robotics for healthcare, the arts, and entrepreneurship.  Spelman College, a historically black college (HBCU) for women is leading the alliance in partnership with several other HBCUs and Research I (R1) institutions.''
		\item Summer REU (Research Experience for Undergraduates) program: \url{http://artsialliance.org/Summer-REU-Program}
		\end{enumerate}
	\item Women in Technology (WIT): \vspace{-0.2cm}
		\begin{enumerate} \itemsep -2pt
		\item \url{http://www.womenintechnology.org/index.asp}
		\item WIT Mentor-Prot{\'{e}}g{\'{e}} Program: \url{http://www.womenintechnology.org/content.asp?contentid=59}
		\item {\bf \color{blue} WIT Career Transition Resource Guide}: \url{http://www.womenintechnology.org/content.asp?contentid=146}
		\item Girls In Technology (GIT): \vspace{-0.1cm}
			\begin{enumerate} \itemsep -1pt
			\item Get Involved: \vspace{-0.1cm}
				\begin{itemize} \itemsep -1pt
				\item \url{http://www.girlsintechnology.org/getinvolved.cfm}
				\item Teacher: teach girls about IT as an after-school activity or in a summer camp session
				\item Assistant Teacher: Assist instructors in GIT sessions, after-school activities, or summer camp sessions
				\item Develop Curriculum: Develop a curriculum for a supported GIT educational program
				\item Mentor: Mentor a girl in one of [GIT's] supported programs
				\item Job Shadow: ``Let a girl shadow you at work''
				\item Guest Speaker: ``Speak to a group of girls on a topic both you and they enjoy, such as computers, technology, education, how to take apart computers, how to build a web site, etc.''
				\end{itemize}
			\end{enumerate}
		\end{enumerate}
	\item Arizona State University: \vspace{-0.2cm}
		\begin{enumerate} \itemsep -2pt
		\item {\it Career}WISE: \vspace{-0.1cm}
			\begin{enumerate} \itemsep -1pt
			\item \url{http://careerwise.asu.edu/}
			\item Helpful resources for female graduate/Ph.D. students in science and engineering.
			\end{enumerate}
		\end{enumerate}
	\item American Indian Science and Engineering Society (AISES): \vspace{-0.2cm}
		\begin{enumerate} \itemsep -2pt
		\item Programs for undergraduates and grad students (including scholarships and internships): \vspace{-0.1cm}
			\begin{enumerate} \itemsep -1pt
			\item \url{http://www.aises.org/Programs}
			\item Resources: \url{http://www.aises.org/Programs/Resources}
			\end{enumerate}
		\end{enumerate}
	\end{enumerate}
\end{enumerate}




%%%%%%%%%%%%%%%%%%%%%%%%%%%%%%%%%%%%%%%%%%%
\subsection{Other Science and Engineering Outreach}
\label{otherstemoutreach}

Other Science and Engineering Outreach: \vspace{-0.3cm}
\begin{enumerate} \itemsep -4pt
\item Frontiers of Engineering (networking event for mid-career engineers): \url{http://www.naefrontiers.org/}
\item Consortium for Ocean Leadership: \vspace{-0.3cm}
	\begin{enumerate} \itemsep -2pt
	\item Resources for scientists in the marine sciences to use in outreach activities: \url{http://www.oceanleadership.org/education/deep-earth-academy/scientists/}
	\end{enumerate}
\item The Oceanography Society: \vspace{-0.3cm}
	\begin{enumerate} \itemsep -2pt
	\item Education and Public Outreach (EPO): A Guide for Scientists [material that scientists and professors can use for outreach activities], \url{http://www.tos.org/epo_guide/index.html}
	\end{enumerate}
\item The Joy McCann Foundation: \vspace{-0.3cm}
	\begin{enumerate} \itemsep -2pt
	\item McCann Scholar (for professors in medicine, science, and nursing): \url{http://www.mccannfoundation.org/scholars.htm}
	\item The Joy McCann Professorship for Women in Medicine: \url{http://www.mccannfoundation.org/medicine.htm}
	\end{enumerate}
\item U.S. National Academies: \vspace{-0.3cm}
	\begin{enumerate} \itemsep -2pt
	\item International Activities of the U.S. National Academies -- Science, Engineering \& Medicine: Working toward a better world: \vspace{-0.2cm}
		\begin{enumerate} \itemsep -2pt
		\item \url{http://sites.nationalacademies.org/International/}
		\item Solving the grand challenges: \vspace{-0.1cm}
			\begin{enumerate} \itemsep -1pt
			\item Energy and the Environment
			\item Global Health
			\item Water Resources
			\item Agriculture and Food Security
			\item International Security
			\item Population
			\end{enumerate}
		\item Help other countries build/improve their capacities: \vspace{-0.1cm}
			\begin{enumerate} \itemsep -1pt
			\item Cooperative Program with Pakistan 
			\item African Science Academies 
			\item Visiting Math Lecturer Program in Cambodia 
			\item Humanitarian Relief Efforts
			\item Improved Road Safety
			\item Science-based Decision Making for Sustainability
			\item Science Academies' Input to G8 Summits
			\end{enumerate}
		\item Scientific Cooperation: \vspace{-0.1cm}
			\begin{enumerate} \itemsep -1pt
			\item Building Bridges in the Middle East
			\item Cooperation with Iran
			\item Human Rights
			\item Frontiers of Science and Engineering Symposia
			\item Travel Grants
			\item International Conference on Women's Issues in Transportation
			\end{enumerate}
		\item Advising the U.S. Government: \vspace{-0.1cm}
			\begin{enumerate} \itemsep -1pt
			\item Science \& Technology in Foreign Policy
			\item Health 
			\item Science and Security
			\end{enumerate}
		\end{enumerate}
	\end{enumerate}
\item National Academy of Engineering: \vspace{-0.3cm}
	\begin{enumerate} \itemsep -2pt
	\item The Charles Stark Draper Prize (``to recognize innovative engineering achievements and their reduction to practice in ways that have led to important benefits and significant improvement in the well being and freedom of humanity''): \url{http://www.draperprize.org/}
	\item NAE Grand Challenge Scholars Program: \url{http://www.grandchallengescholars.org/}
	\end{enumerate}
\item United States Department of Defense (DoD): \vspace{-0.3cm}
	\begin{enumerate} \itemsep -2pt
	\item National Defense Education Program; Defense Advanced Research Projects Agency (DARPA): \vspace{-0.2cm}
		\begin{enumerate} \itemsep -2pt
		\item Resource for scientists and engineers to mentor youths, so that they would look into pursuing careers in science and engineering: \url{http://www.ndep.us/GetInvoSci.aspx}
		\item STEM Learning Modules (SLM): \vspace{-0.1cm}
			\begin{enumerate} \itemsep -1pt
			\item \url{http://www.ndep.us/ProgSLM.aspx}
			\item Help educators develop programs in science and engineering in K-12 institutions, so that youths would be encouraged to explore careers in science and engineering
			\end{enumerate}
		\end{enumerate}
	\end{enumerate}
\item Hewlett-Packard Development Company: \vspace{-0.3cm}
	\begin{enumerate} \itemsep -2pt
	\item HP Catalyst Initiative (grants for STEM education in colleges and universities): \url{http://www.hp.com/hpinfo/socialinnovation/catalyst.html}
	\item HP EdTech Innovators Award (for higher educational institutions that integrate IT into the curricular): \url{http://www.hp.com/hpinfo/socialinnovation/edtech.html}
	\end{enumerate}
\item The William and Flora Hewlett Foundation (Hewlett Foundation): \vspace{-0.3cm}
	\begin{enumerate} \itemsep -2pt
	\item Funding Programs: \url{http://www.hewlett.org/programs}
	\item Grantseekers: \url{http://www.hewlett.org/grants/grantseekers}
	\end{enumerate}
\item The Sloan Consortium (Sloan-C): \vspace{-0.3cm}
	\begin{enumerate} \itemsep -2pt
	\item Sloan-C Awards (for recognizing outstanding work in the field of online education) and Sloan-C Fellows: \url{http://sloanconsortium.org/aboutus/awards}
	\item Mayadas Leadership Award in Online Education: \url{http://sloanconsortium.org/mayadas_award}
	\end{enumerate}
\item W.K. Kellogg Foundation: \vspace{-0.3cm}
	\begin{enumerate} \itemsep -2pt
	\item Grant database: \url{http://www.wkkf.org/grants/grants-database.aspx}
	\end{enumerate}
\item Hewlett-Packard Company: \vspace{-0.3cm}
	\begin{enumerate} \itemsep -2pt
	\item HP community investment for education, economic development, and the environment: \url{http://www.hp.com/hpinfo/socialinnovation/focus.html}
	\item Entrepreneurship education: \vspace{-0.2cm}
		\begin{enumerate} \itemsep -2pt
		\item \url{http://www.hp.com/hpinfo/globalcitizenship/society/social/entrepreneurship.html}
		\item HP Graduate Entrepreneurship Training through IT (GET-IT)
		\item HP Entrepreneurship Learning Program (HELP)
		\end{enumerate}
	\item HP Innovations in Education grants: \url{http://www.hp.com/hpinfo/globalcitizenship/society/social/innovations.html}
	\end{enumerate}
\item General Electric Company: \vspace{-0.3cm}
	\begin{enumerate} \itemsep -2pt
	\item GE Foundation: \vspace{-0.2cm}
		\begin{enumerate} \itemsep -2pt
		\item Developing Futures\texttrademark\ in Education program (which encompasses the GE College Bound Program): \url{http://www.ge.com/foundation/developing_futures_in_education/index.jsp}
		\item Environment, health and safety, and health industry training programs (outside the US): \url{http://www.ge.com/foundation/international_programs/training.jsp}
		\item Student, education and scholarship initiatives: \url{http://www.ge.com/foundation/international_programs/education_initiatives.jsp}
		\end{enumerate}
	\end{enumerate}
\item The GRAMMY Foundation: \vspace{-0.3cm}
	\begin{enumerate} \itemsep -2pt
	\item GRAMMY Foundation Grants: \vspace{-0.2cm}
		\begin{enumerate} \itemsep -2pt
		\item \url{http://www2.grammy.com/GRAMMY_Foundation/Grants/}
		\item It funds {\bf Scientific Research Projects} as well as {\it Archiving And Preservation Projects}.
		\item Concerning scientific research projects: ``The GRAMMY Foundation Grant Program awards grants to organizations and individuals to support research on the impact of music on the human condition. Examples might include the study of the effects of music on mood, cognition and healing, as well as the medical and occupational well-being of music professionals and the creative process underlying music.'' [ E.g., look at music therapy as a possible research topic/area. ]
		\end{enumerate}
	\end{enumerate}
\item The Dana Foundation: \vspace{-0.3cm}
	\begin{enumerate} \itemsep -2pt
	\item \url{http://www.dana.org/grants/}
	\item Has grants for: \vspace{-0.2cm}
		\begin{enumerate} \itemsep -2pt
		\item Brain and Immuno-Imaging
		\item Clinical Neuroscience
		\item Human Immunology
		\item Neuroimmunology of Brain Infections and Cancers
		\end{enumerate}
		\item Deadlines and Requests for Proposals (RFP): \url{http://www.dana.org/grants/deadlines.aspx}
	\end{enumerate}
%%%%%%%%%%%%%%%%%%%%%%%%%%%%%%%%%%%%%%%
% underrepresented minorities
\item Institute for Broadening Participation: \vspace{-0.3cm}
	\begin{enumerate} \itemsep -2pt
	\item PathwaysToScience.org: \vspace{-0.2cm}
		\begin{enumerate} \itemsep -2pt
		\item Resources for faculty and administrators (to facilitate STEM outreach activities as well as the recruitment of underrepresented minorities to the student body and faculty): \url{http://www.pathwaystoscience.org/Faculty.asp}
		\end{enumerate}
	\end{enumerate}
\item National Center for Women \& Information Technology (NCWIT): \vspace{-0.3cm}
	\begin{enumerate} \itemsep -2pt
	\item NCWIT Academic Alliance Seed Fund (for developing and implementing initiatives in colleges and universities to recruit and retain women in computing and information technology): \url{http://www.ncwit.org/work.awards.seed.html}
	\item NCWIT Symons Innovator Award (for outstanding women who have successfully built and funded an IT business): \url{http://www.ncwit.org/work.awards.innovator.html}
	\end{enumerate}
\item Women in Technology (WIT): \vspace{-0.3cm}
	\begin{enumerate} \itemsep -2pt
	\item Girls In Technology (GIT): \vspace{-0.2cm}
		\begin{enumerate} \itemsep -2pt
		\item Get Involved: \vspace{-0.1cm}
			\begin{itemize} \itemsep -1pt
			\item \url{http://www.girlsintechnology.org/getinvolved.cfm}
			\item Teacher: teach girls about IT as an after-school activity or in a summer camp session
			\item Assistant Teacher: Assist instructors in GIT sessions, after-school activities, or summer camp sessions
			\item Develop Curriculum: Develop a curriculum for a supported GIT educational program
			\item Mentor: Mentor a girl in one of [GIT's] supported programs
			\item Job Shadow: ``Let a girl shadow you at work''
			\item Guest Speaker: ``Speak to a group of girls on a topic both you and they enjoy, such as computers, technology, education, how to take apart computers, how to build a web site, etc.''
			\end{itemize}
		\end{enumerate}
	\end{enumerate}
\item European Platform of Women Scientists (EPWS): \vspace{-0.3cm}
	\begin{enumerate} \itemsep -2pt
	\item \url{http://www.epws.org/}
	\item Members: \url{http://www.epws.org/index.php?option=com_content&task=blogcategory&id=134&Itemid=4652}
	\end{enumerate}
\end{enumerate}





Commercializing academic research into products and services via start-ups: \vspace{-0.3cm}
\begin{enumerate} \itemsep -4pt
\item Ben Franklin Technology Partners (BFTP): \vspace{-0.3cm}
	\begin{enumerate} \itemsep -2pt
	\item Innovation Works (IW): \vspace{-0.2cm}
		\begin{enumerate} \itemsep -2pt
		\item For universities in the Pittsburgh metropolitan area
		\item University Innovation Grants (UIGs) / University Grants: \vspace{-0.1cm}
			\begin{enumerate} \itemsep -1pt
			\item For technology validation, market research, prototype development, and intellectual property evaluation
			\item Available online at: \url{http://www.innovationworks.org/OurPrograms/UniversityGrants/tabid/115/Default.aspx}; last accessed on November 14, 2010.
			\end{enumerate}
		\end{enumerate}
	\end{enumerate}
\end{enumerate}









%%%%%%%%%%%%%%%%%%%%%%%%%%%%%%%%%%%%%%%%%%%
\subsection{Electrical and Computer Engineering \& Computer Science Outreach}
\label{ececsoutreach}

Electrical and computer engineering, and computer science outreach: \vspace{-0.3cm}
\begin{enumerate} \itemsep -4pt
\item IEEE: \vspace{-0.3cm}
	\begin{enumerate} \itemsep -2pt
	\item {\it IEEE-USA Salary Service} provides a survey of jobs in electrical and computer engineering: \url{http://www.ieeeusa.org/careers/salary/}
	\item {\it IEEE Santa Clara Valley Section PACE}: Professional Activities Committee for Engineers (PACE); see \url{http://www.ewh.ieee.org/r6/scv/PACE/}
	\item {\it IEEE Santa Clara Valley Section}: \url{http://ewh.ieee.org/r6/scv/} and \url{http://www.ieee.org/scv}
	\item 
	\end{enumerate}
\item Association for Computing Machinery, ACM: \vspace{-0.3cm}
	\begin{enumerate} \itemsep -2pt
	\item Sanjeev Arora, Boaz Barak, and Luca Trevisan, ``Survey Papers and Essays,'' in {\it Theory Matters Wiki: Theoretical Computer Science (TCS) Advocacy Wiki}, SIGACT Committee for the Advancement of Theoretical Computer Science, ACM Special Interest Group on Algorithms and Computation Theory (SIGACT), Association for Computing Machinery, February 25, 2010. Available at: \url{http://theorymatters.org/pmwiki/pmwiki.php?n=Main.SurveyCollection}; last accessed on September 14, 2010.
	\item Online Resources for Graduating Students: \url{http://www.acm.org/membership/student/resources-for-grads}
	\end{enumerate}
\item VLSI design and verification: \vspace{-0.3cm}
	\begin{enumerate} \itemsep -2pt
	\item {\it DVClub} for individuals interested in VLSI verification: \url{http://www.dvclub.org/}
	\item {\it DeepChip.com}: \url{http://www.deepchip.com}
	\end{enumerate}
%%%%%%%%%%%%%%%%%%%%%%%%%%%%%%%
\item undergraduates: \vspace{-0.3cm}
	\begin{enumerate} \itemsep -2pt
	\item {\it Humanitarian FOSS Project}: \vspace{-0.2cm}
		\begin{enumerate} \itemsep -2pt
		\item Where FOSS refers to Free and Open Source Software
		\item For computer science and engineering students
		\item \url{http://www.hfoss.org/}
		\end{enumerate}
	\item {\it SIGDA Design Automation Summer School}: \vspace{-0.2cm}
		\begin{enumerate} \itemsep -2pt
		\item {\it NSF�SRC�SIGDA�DAC Design Automation Summer School}
		\item \url{http://www.sigda.org/dass.html}
		\item Travel grants are provided to defray travel and accommodation expenses
		\end{enumerate}
	\item {\it Young Student Support Program at DAC}: \vspace{-0.2cm}
		\begin{enumerate} \itemsep -2pt
		\item Also known as {\it DAC Young Student Support Program}
		\item \url{http://www.sigda.org/youngstudent.html}
		\item Travel grants are provided to defray travel and accommodation expenses
		\end{enumerate}
	\item {\it ACM Student Research Competition at Design Automation Conference}: \vspace{-0.2cm}
		\begin{enumerate} \itemsep -2pt
		\item Sponsored by {\it Microsoft Research}
		\item \url{http://www.sigda.org/studentcomp.html}
		\item Also, see {\it ACM Student Research Competition} @ \url{http://src.acm.org/}.
		\end{enumerate}
	\item Job database for positions in the Video Game, Animation, VFX, and Software/Technology industries: \url{http://www.creativeheads.net/}
	\end{enumerate}
%%%%%%%%%%%%%%%%%%%%%%%%%%%%%%
\item graduate students: \vspace{-0.3cm}
	\begin{enumerate} \itemsep -2pt
	\item {\it SIGDA Design Automation Summer School}: \vspace{-0.2cm}
		\begin{enumerate} \itemsep -2pt
		\item {\it NSF�SRC�SIGDA�DAC Design Automation Summer School}
		\item \url{http://www.sigda.org/dass.html}
		\item Travel grants are provided to defray travel and accommodation expenses
		\end{enumerate}
	\item {\it Young Student Support Program at DAC}: \vspace{-0.2cm}
		\begin{enumerate} \itemsep -2pt
		\item Also known as {\it DAC Young Student Support Program}
		\item \url{http://www.sigda.org/youngstudent.html}
		\item Travel grants are provided to defray travel and accommodation expenses
		\end{enumerate}
	\item {\it ACM Student Research Competition at Design Automation Conference}: \vspace{-0.2cm}
		\begin{enumerate} \itemsep -2pt
		\item Sponsored by {\it Microsoft Research}
		\item \url{http://www.sigda.org/studentcomp.html}
		\item Also, see {\it ACM Student Research Competition} @ \url{http://src.acm.org/}.
		\end{enumerate}
	\item {\it SIGDA University Booth at DAC}: \vspace{-0.2cm}
		\begin{enumerate} \itemsep -2pt
		\item Or, {\it SIGDA/DAC University Booth}
		\item \url{http://www.sigda.org/ubooth.html}
		\end{enumerate}
	\item {\it SIGDA Ph.D. Forum at DAC}: \vspace{-0.2cm}
		\begin{enumerate} \itemsep -2pt
		\item \url{http://www.sigda.org/phdforum.html}
		\item \url{http://www.sigda.org/daforum/}
		\end{enumerate}
	\item {\it DAC Graduate Scholarship}: \vspace{-0.2cm}
		\begin{enumerate} \itemsep -2pt
		\item {\it A. Richard Newton Graduate Scholarships} to Support Graduate Research and Study
		\item \url{http://www.sigda.org/gradscholarship.html}
		\end{enumerate}
	\end{enumerate}
%%%%%%%%%%%%%%%%%%%%%%%%%%%%%%
\item competitions, and programming contests and challenges: \vspace{-0.3cm}
	\begin{itemize} \itemsep -2pt
	\item {\it SIGDA CADathlon at ICCAD}: \vspace{-0.2cm}
		\begin{enumerate} \itemsep -2pt
		\item \url{http://www.sigda.org/programs/cadathlon/}
		\item \url{http://www.sigda.org/cadathlon.html}
		\item Travel grants are provided to defray travel and accommodation expenses
		\end{enumerate}
	\item ISPD Programming Contest: \url{http://www.ispd.cc/contests/}
	\item ACM International Workshop on Timing Issues in the Specification and Synthesis of Digital Systems (TAU Workshop): \vspace{-0.2cm}
		\begin{enumerate} \itemsep -2pt
		\item Power Grid Simulation Contest: \url{http://www.tauworkshop.com/PREVIOUS/contest_2011.html}
		\end{enumerate}
	\item IEEE Computer Society Simulator Design competition: \url{http://www.computer.org/portal/web/competition}
	\item {\it DAC/ISSCC Student Design Contest}: \vspace{-0.2cm}
		\begin{enumerate} \itemsep -2pt
		\item \url{http://www.dac.com}
		\end{enumerate}
	\item {\it ACM/IEEE International Conference on Formal Methods and Models for Codesign -- Design Contest}: \vspace{-0.2cm}
		\begin{enumerate} \itemsep -2pt
		\item MEMOCODE Hardware/Software Co-Design Contest (MEMOCODE HW/SW co-design contest)
		\item \url{http://www-memocode2010.imag.fr/}
		\item \url{http://memocode2010.csail.mit.edu/redmine/wiki/memocode2010/Results}
		\end{enumerate}
	\item {\it International Low Power Design Contest}: \vspace{-0.2cm}
		\begin{enumerate} \itemsep -2pt
		\item ACM/IEEE International Symposium on Low Power Electronics and Design (ISLPED) -- Design Contest
		\item The International Symposium on Low Power Electronics and Design is holding the International Low Power Design Contest to provide a forum for universities and research organizations to showcase original ``power-aware'' designs and to highlight the innovations and design choices targeted at low power.
		\item The goal is to encourage and highlight design-oriented approaches to power reduction.
		\item \url{http://www.islped.org/2010/index.html}
		\end{enumerate}
	\item {\it University LSI Design Contest @ ASP-DAC}: \vspace{-0.2cm}
		\begin{enumerate} \itemsep -2pt
		\item Application areas or types of circuits of the original LSI circuit designs include (but are not limited to): \vspace{-0.1cm}
			\begin{enumerate} \itemsep -1pt
			\item Analog, RF and Mixed-Signal Circuits
			\item Digital Signal Processing
			\item Microprocessors
			\item Custom ASIC
			\end{enumerate} 
		\item Methods or technology used for implementation include: \vspace{-0.1cm}
			\begin{enumerate} \itemsep -1pt
			\item Full Custom and Cell-Based LSIs
			\item Gate Arrays
			\item FPGA/PLDs.
			\end{enumerate}
		\item \url{http://www.aspdac.com/aspdac2011/cfd/}
		\end{enumerate}
	\item IEEE Programming Challenge at IWLS: \url{http://www.iwls.org/challenge/}
	\item IEEE Asian Solid-State Circuits Conference (A-SSCC) Student Design Contest: \url{http://a-sscc2010.a-sscc.org/contest.html}
	\item {\it VLSI Conference 2011 - Design Contest}: \vspace{-0.2cm}
		\begin{enumerate} \itemsep -2pt
		\item Design/project fields include (but not limited to): \vspace{-0.1cm}
			\begin{enumerate} \itemsep -1pt
			\item Digital Integrated Circuits
			\item Analog Integrated Circuits
			\item FPGA based designs
			\item Computer Architectures/ Processors
			\item Reconfigurable Computing Systems
			\item SoC / Platform-based designs
			\item Embedded Systems
			\item MEMS/Optics/Bio-Chips
			\item Innovative Design Methodologies and Verification Techniques.
			\end{enumerate}
		\item \url{http://vlsiconference.com/vlsi2011/submissions_design_contest.html}
		\end{enumerate}
	\item {\it Satisfiability Modulo Theories Competition} (SMT-COMP): \vspace{-0.2cm}
		\begin{enumerate} \itemsep -2pt
		\item Competition for SMT solvers
		\item \url{http://www.smtcomp.org/2010/}
		\end{enumerate}
	\item {\it SAT Competition 201X}, where $X > 0$ \& $X {\it mod} 2 = 1$: \vspace{-0.2cm}
		\begin{enumerate} \itemsep -2pt
		\item The purpose of the competition is to identify new challenging benchmarks and to promote new solvers for the propositional satisfiability problem (SAT) as well as to compare them with state-of-the-art solvers.
		\item \url{http://www.satcompetition.org/}
		\end{enumerate}
	\item {\it SAT-Race 201X}, where $X > 0$ \& $X {\it mod} 2 = 0$: \vspace{-0.2cm}
		\begin{enumerate} \itemsep -2pt
		\item SAT-Race 201X is a competitive event for solvers of the Boolean Satisfiability (SAT) problem. 
		\item In contrast to the SAT Competitions, the focus of SAT-Race is on application benchmarks only.
		\item \url{http://baldur.iti.uka.de/sat-race-2010/}
		\end{enumerate}
	\item Hardware Model Checking Competition (HWMCC): \url{http://fmv.jku.at/hwmcc10/}
	\item {\it CADE ATP System Competition} (CASC): \vspace{-0.2cm}
		\begin{enumerate} \itemsep -2pt
		\item It is a yearly competition of fully automated theorem provers for classical first order logic.
		\item \url{http://www.cs.miami.edu/~tptp/CASC/}
		\end{enumerate}
	\item Apple Design Awards: \url{http://developer.apple.com/wwdc/ada/index.html}
	\item {\it International Constraint Solver Competition}: \vspace{-0.2cm}
		\begin{enumerate} \itemsep -2pt
		\item Also known as: \vspace{-0.2cm}
			\begin{enumerate} \itemsep -2pt
			\item International Constraint Solver Competition (CSP, Max-CSP and Weighted-CSP competition)
			\item International CSP Solver Competition (CSP, Max-CSP and Weighted-CSP competition)
			\end{enumerate}
		\item The Fourth International Constraint Solver Competition (CSC'2009) is organized to improve our knowledge of what is behind the efficiency of constraint satisfaction algorithms, heuristics, solving strategies, and constraint systems.
		\item \url{http://cpai.ucc.ie/}
		\end{enumerate}
	\item International Conference on Field-Programmable Technology (FPT 201X): \vspace{-0.2cm}
		\begin{enumerate} \itemsep -2pt
		\item FPT Design Competition: \url{http://cas.ee.ic.ac.uk/people/as999/FPTDesignComp/}
		\end{enumerate}
	\item International Microwave Symposium: Student Design Competitions -- Jan (includes AMS circuit simulation, and AMS/RF EDA); \url{http://ims2011.org/Technical_Program/Student_Design_Competitions.html}
	\item {\it QBFEVAL'1X}: \vspace{-0.2cm}
		\begin{enumerate} \itemsep -2pt
		\item QBF Solver competition for solvers to determine Quantified Boolean Formula (QBF) satisfiability.
		\item QBFLIB is a collection of instances, solvers, and tools related to Quantified Boolean Formula (QBF) satisfiability. See \url{http://www.qbflib.org/}.
		\item \url{http://www.qbflib.org/index_eval.php}
		\end{enumerate}
	\item {\it Pseudo-Boolean Competition 201X}: \vspace{-0.2cm}
		\begin{enumerate} \itemsep -2pt
		\item Competition for pseudo-Boolean solvers.
		\item \url{http://www.cril.univ-artois.fr/PB10/}
		\end{enumerate}
	\item {\it Answer Set Programming System Competition}: \vspace{-0.2cm}
		\begin{enumerate} \itemsep -2pt
		\item \url{http://dtai.cs.kuleuven.be/events/ASP-competition/}
		\end{enumerate}
	\item {\it Max-SAT Evaluation, Max-SAT 201X}: \vspace{-0.2cm}
		\begin{enumerate} \itemsep -2pt
		\item Competition for Max-SAT solvers
		\item \url{http://www.maxsat.udl.cat/}
		\item \url{http://www.maxsat.udl.cat/09/}
		\end{enumerate}
	\item {\it IEEEXtreme 24 Hour Programming Challenge}: \vspace{-0.2cm}
		\begin{enumerate} \itemsep -2pt
		\item Programming contest for college students
		\item \url{http://portal.ieee.org/web/membership/students/scholarshipsawardscontests/ieeextreme.html}
		\end{enumerate}
	\item {\it ACM International Collegiate Programming Contest} (ACM-ICPC or ICPC): \vspace{-0.2cm}
		\begin{enumerate} \itemsep -2pt
		\item Programming contest for college students
		\item Official web page: \url{http://cm.baylor.edu/welcome.icpc}
		\item Other web resources: \vspace{-0.1cm}
			\begin{enumerate} \itemsep -1pt
			\item {\it Wikipedia}: \url{http://en.wikipedia.org/wiki/ACM_International_Collegiate_Programming_Contest}
			\item {\it }: \url{}
			\item {\it }: \url{}
			\item {\it Valladolid Online Judge Site}: \url{http://acm.uva.es/}
			\item {\it ACMSolver :: Art of Programming Contest, Tips and Tricks for C, C++, Java}: \url{http://www.acmsolver.org/}
			\end{enumerate}
		\item 
		\end{enumerate}
	\item {\it TopCoder} coding and design contests: \vspace{-0.2cm}
		\begin{enumerate} \itemsep -2pt
		\item The contests cover various fields, such as: \vspace{-0.1cm}
			\begin{enumerate} \itemsep -1pt
			\item Algorithm
			\item Conceptualization
			\item Specification
			\item Architecture
			\item Component Design
			\item Component Development
			\item Assembly
			\item Test Scenarios
			\item Test Suites
			\item UI Prototype
			\item Rich Internet Application (RIA) Build
			\item Bug Race
			\item Marathon Match
			\item High School (for high school students)
			\item Copilot Opportunities
			\end{enumerate}
		\item \url{http://www.topcoder.com/}
		\end{enumerate}
	\item IEEE Presidents' Change the World competition: \vspace{-0.2cm}
		\begin{enumerate} \itemsep -2pt
		\item The IEEE Presidents� Change the World Competition recognizes students who develop unique solutions to real-world problems using engineering, science, computing and leadership skills to benefit their community, the world at large, or both. 
		\item \url{http://www.ieeechangetheworld.org/}
		\end{enumerate}
	\item Google Code Jam (programming contest): \url{http://code.google.com/codejam/} and \url{http://en.wikipedia.org/wiki/Google_Code_Jam}
	\item {\it RoboCup}\texttrademark\ competitions: \vspace{-0.2cm}
		\begin{enumerate} \itemsep -2pt
		\item Has different categories, including soccer, rescue operations, and home applications.
		\item \url{http://www.robocup.org/}
		\end{enumerate}
	\item ICFP Programming Contest (ICFP refers to International Conference on Functional Programming): \url{http://icfpcontest.org/}
	\item Student Cluster Competition (SCC): \vspace{-0.2cm}
		\begin{enumerate} \itemsep -2pt
		\item SCC is held at each (annual) SC conference, which is the International Conference for High Performance Computing, Networking, Storage, and Analysis. IEEE Computer Society and the Association for Computing Machinery are the sponsors for this conference.
		\item During SC10, teams consisting of six students, undergraduate and/or high school, will showcase the amazing power of clusters and the ability to utilize open source software to solve interesting and important problems. They will compete in real-time on the exhibit floor to run a workload of real-world applications on clusters of their own design while never exceeding the dictated power limit.
		\item During SC10 in New Orleans, teams will assemble, test and tune their machines and run the HPCC benchmarks until the starting bell rings on Monday night at the Exhibit Opening Gala where they will be given the competition data sets. In full view of conference attendees, teams will execute the prescribed workload while showing progress and science visualization output on large high-resolution displays in their areas. Teams race to correctly complete the greatest number of application runs during the competition period until the close of the exhibit floor on Wednesday evening.
		\item \url{http://sc10.supercomputing.org/?pg=studentcluster.html}
		\end{enumerate}
	\item Cypress Semiconductor Corporation: \vspace{-0.2cm}
		\begin{enumerate} \itemsep -2pt
		\item ARM Cortex-M3 PSoC\textregistered\ 5 Design Challenge: \url{http://www.cypress.com/?id=3271}
		\end{enumerate}
	\item Mentor Graphics: \vspace{-0.2cm}
		\begin{enumerate} \itemsep -2pt
		\item PCB Technology Leadership Awards (PCB design contest): \url{http://www.mentor.com/products/pcb-system-design/tla/index.cfm?v=mentorgraphics&p=handout:tla&a=print_card&g=sdd&s=1x1&c=ocid_2203&cmpid=3911}, or \url{http://www.mentor.com/go/tla}
		\end{enumerate}
	\item INFORMS Data Mining Contest: \vspace{-0.2cm}
		\begin{enumerate} \itemsep -2pt
		\item \url{http://ifors.org/web/call-for-participation-informs-data-mining-contest-2010/}
		\item \url{http://kaggle.com/informs2010}
		\end{enumerate}
	\item INFORMS Doing Good with Good OR - Student Competition: \vspace{-0.2cm}
		\begin{enumerate} \itemsep -2pt
		\item Doing Good with Good OR-Student Competition is held each year to identify and honor outstanding projects in the field of operations research and the management sciences conducted by a student or student group that have a significant societal impact.
		\item \url{http://www.informs.org/Recognize-Excellence/INFORMS-Prizes-Awards/Doing-Good-with-Good-OR}
		\end{enumerate}
	\item HPC Challenge Award Competition: \url{http://www.hpcchallenge.org/}
	\item Sphere Online Judge, SPOJ (programming contest): \url{http://www.spoj.pl/}
	\item High Performance and Scientific Computing Contest (Argonne National Laboratory, U.S. Department of Energy, DOE): \url{https://wiki.alcf.anl.gov/index.php/HPSC_Contest_Information}
	\item Argonne National Laboratory, ANL; Mathematics and Computer Science Division: \vspace{-0.2cm}
		\begin{enumerate} \itemsep -2pt
		\item J. H. Wilkinson Prize for Numerical Software (for developers of numerical software): \url{http://www.mcs.anl.gov/research/opportunities/wilkinsonprize/index.php}
		\end{enumerate}
	\item Society for Industrial and Applied Mathematics, SIAM: \vspace{-0.2cm}
		\begin{enumerate} \itemsep -2pt
		\item SIAM/ACM Prize in Computational Science and Engineering: \url{http://www.siam.org/prizes/sponsored/cse.php}. [ For developers of mathematical and computational tools and methods for the solution of science and engineering. Or, for developers of computational science and engineering software. ]
		\end{enumerate}
	\end{itemize}
	\item Sun HPC Software Programming Challenge (Oracle Corporation): \url{http://wikis.sun.com/display/HPCContest/Home}
%%%%%%%%%%%%%%%%%%%%%%%%%%%%%%
\item News media: \vspace{-0.3cm}
	\begin{itemize} \itemsep -2pt
	\item --- --- --- --- --- --- --- --- --- --- --- --- --- --- --- --- --- --- --- --- --- --- --- --- --- --- --- --- --- --- ---
	\item \colorbox{blue}{\bf News media for Electronic Design Automation}
	% News media for Electronic Design Automation
	\item {\it EDACafe}: \url{http://www.edacafe.com/}
	\item {\it SIGDA E-Newsletter} (SIGDA Electronic Newsletter): \url{http://www.sigda.org/newsletter/}
	\item {\it DeepChip.com}: \url{http://www.deepchip.com}
	\item --- --- --- --- --- --- --- --- --- --- --- --- --- --- --- --- --- --- --- --- --- --- --- --- --- --- --- --- --- --- ---
	\item \colorbox{blue}{\bf News media for Electrical and Computer Engineering}
	% News media for Electrical and Computer Engineering
	\item {\it EE Times} (Electronic Engineering Times): \url{http://www.eetimes.com/}
	\item {\it EDN} (Electrical Design News): \url{http://www.edn.com/}
	\item {\it IEEE Spectrum}: \url{http://spectrum.ieee.org/}
	\item {\it The Institute} (from IEEE): \url{http://www.theinstitute.ieee.org}
	\item {\it IEEE-USA Today's Engineer}: \url{http://www.todaysengineer.org/}
	\item {\it DeepChip.com}: \url{http://www.deepchip.com}
	\item --- --- --- --- --- --- --- --- --- --- --- --- --- --- --- --- --- --- --- --- --- --- --- --- --- --- --- --- --- --- ---
	\item \colorbox{blue}{\bf News media for Computer Science and Engineering, Information Systems, and IT}
	% News media for Computer Science and Engineering, Information Systems, and IT
	\item {\it ACM TechNews}: \url{http://technews.acm.org/}
	\item {\it TechCareers}: \url{http://www.techcareers.com/}
	\item {\it }: \url{}
	\item {\it }: \url{}
	\item {\it }: \url{}
	\item {\it }: \url{}
	\item {\it }: \url{}
	\item {\it }: \url{}
	\item {\it }: \url{}
	\item --- --- --- --- --- --- --- --- --- --- --- --- --- --- --- --- --- --- --- --- --- --- --- --- --- --- --- --- --- --- ---
	\item \colorbox{blue}{\bf Other News Media}
	% Other News Media
	\item {\it iTunes U}
	\item {\it YouTube EDU}
	\end{itemize}
%%%%%%%%%%%%%%%%%%%%%%%%%%%%%%
\item underrepresented minorities: \vspace{-0.3cm}
	\begin{enumerate} \itemsep -2pt
	\item women: \vspace{-0.2cm}
		\begin{enumerate} \itemsep -2pt
		\item IEEE Women in Engineering (WIE): \url{http://www.ieee.org/membership_services/membership/women/index.html?WT.mc_id=WIE_nav1}
		\item ACM-W: \url{http://women.acm.org/}
		\item Computer Research Association's Committee on the Status of Women in Computing Research (CRA-W): \vspace{-0.1cm}
			\begin{enumerate} \itemsep -1pt
			\item \url{http://www.cra-w.org/}
			\item Computing Research Association's Committee on the Status of Women (CRA-W) and the Coalition to Diversify Computing (CDC), {\it CompArch Summer School on Parallel Programming and Architectures}. Available at: \url{http://www.princeton.edu/~archss/}; last accessed on September 3, 2010.
			\end{enumerate}
		\item National Center for Women \& Information Technology: \url{http://www.ncwit.org/}
		\item African-American Women in Technology organization (AAWIT): \url{http://www.aawit.net/09/index.cfm}
		\item Grace Hopper Celebration of Women in Computing (conference for female IT students, professors, and professionals): \url{http://gracehopper.org/} or \url{http://gracehopper.org/2010/}
		\item Anita Borg Institute for Women and Technology: \vspace{-0.1cm}
			\begin{enumerate} \itemsep -1pt
			\item Has many programs for female students and professionals: \url{http://anitaborg.org/}
			\end{enumerate}
		\end{enumerate}
	\end{enumerate}
\end{enumerate}







%%%%%%%%%%%%%%%%%%%%%%%%%%%%%%%%%%%%%%%%%%%
\section{Scholarships, Fellowships, Awards, and Financial Aid}
\label{scholarshipsfinaidawards}

Resources for scholarships, fellowships, and financial aid: \vspace{-0.3cm}
\begin{enumerate} \itemsep -4pt
\item --- --- --- --- --- --- --- --- --- --- --- --- --- --- --- --- --- --- --- --- --- --- --- --- --- --- --- --- --- --- ---
\item \colorbox{blue}{\bf Lists of Scholarships and Fellowships}
% Lists of Scholarships and Fellowships
\item List of scholarships: \vspace{-0.3cm}
	\begin{enumerate} \itemsep -2pt
	\item Engineering Education Service Center, EESC (Engineering): \url{http://www.engineeringedu.com/scholars.html}
	\item High Performance and Embedded Architecture and Compilation, HiPEAC (Computer Science and Engineering): \url{http://www.hipeac.net/all_jobs_op}
	\item Office of Doctoral Programs at USC Viterbi School of Engineering, {\bf University of Southern California}. External Fellowships and other support: \url{http://viterbi.usc.edu/students/phd/fellowships-and-other-support/external-fellowships.htm}. USC Fellowships: \url{http://viterbi.usc.edu/students/phd/fellowships-and-other-support/usc-fellowships.htm}
	\item Columbia College, {\bf Columbia University} in the City of New York: \url{http://www.college.columbia.edu/students/fellowships/catalog}
	\item {\bf New York University} School of Law: \url{http://www.law.nyu.edu/financialaid/supplementalaid/fellowships/index.htm}
	\item Swedish Institute: \vspace{-0.2cm}
		\begin{enumerate} \itemsep -2pt
		\item The Swedish Institute, a government agency, administers over 500 scholarships each year for students and researchers coming to Sweden to pursue their objectives at a Swedish university.
		\item Study in Sweden: scholarships, \url{http://www.studyinsweden.se/Scholarships/}
		\item Swedish Institute (SI): \url{http://www.si.se/English/Navigation/Scholarships-and-exchanges/} [ Has special programs for Pakistanis and Turkish citizens ]
		\end{enumerate}
	\item The Swedish Foundation for International Cooperation in Research and Higher Education (STINT): \vspace{-0.2cm}
		\begin{enumerate} \itemsep -2pt
		\item \url{http://www.stint.se/en}
		\item Scholarships and grants: \url{http://www.stint.se/en/scholarships_and_grants}
		\end{enumerate}
	\item Center for the Advancement of Hispanics in Science and Engineering Education (CAHSEE): \url{http://www.cahsee.org/6resources/scholarships.asp.htm}
	\item University of Wisconsin-Madison: \vspace{-0.2cm}
		\begin{enumerate} \itemsep -2pt
		\item Grants Information Collection: A Cooperating Collection of the Foundation Center Library Network, \url{http://grants.library.wisc.edu/}
		\end{enumerate}
	\item {\it Find A PhD}: \url{http://www.findaphd.com/}
	\item QS World Grad School Tour Scholarships (QS Quacquarelli Symonds Limited): \url{http://graduateschool.topuniversities.com/world-grad-school-tour/scholarships}
	\item GlobalGrant (requires paid access to the list of scholarships and fellowships): \url{http://www.globalgrant.com/en/stipendier.html} and \url{http://www.globalgrant.com/}
	\item Stockholm University: \vspace{-0.2cm}
		\begin{enumerate} \itemsep -2pt
		\item \url{http://www.su.se/pub/jsp/polopoly.jsp?d=777&a=1770}
		\item \url{http://www.su.se/pub/jsp/polopoly.jsp?d=797}
		\item \url{http://www.su.se/pub/jsp/polopoly.jsp?d=788}
		\item \url{http://www.su.se/pub/jsp/polopoly.jsp?d=777&a=1769}
		\end{enumerate}
	\item NordForsk (in Norwegian): \url{http://www.nordforsk.org/index.cfm}
	\item Wallenberg Scholars (in Swedish): \url{http://www.wallenberg.com/default.aspx} or \url{http://www.wallenberg.com/in-english.aspx}
	\item Royal Institute of Technology (in Swedish): \url{http://www.kth.se/aktuellt/stipendier/stipendier-och-anslag-1.2024}
	\item European Commission: \vspace{-0.2cm}
		\begin{enumerate} \itemsep -2pt
		\item Marie Curie Fellowships: \vspace{-0.1cm}
			\begin{enumerate} \itemsep -1pt
			\item \url{http://cordis.europa.eu/fp7/people/home_en.html}
			\item \url{http://ec.europa.eu/research/mariecurieactions/}
			\item \url{http://ec.europa.eu/research/fp6/mariecurie-actions/action/fellow_en.html}
			\item \url{http://www.mariecurie.org/}
			\end{enumerate}
		\item Euraxess: \url{http://ec.europa.eu/euraxess/}
		\item \url{http://ec.europa.eu/index_en.htm}
		\end{enumerate}
	\item Science Please (for research positions in life sciences in The Netherlands and Belgium, including Ph.D. and postdoc positions): \url{http://www.scienceplease.com/} or \url{http://www.scienceplease.com/about-us}
	\item University of Gothenburg: \vspace{-0.2cm}
		\begin{enumerate} \itemsep -2pt
		\item ResearchResearch: \url{http://www.researchresearch.com/} or \url{http://www.gu.se/english/research/scholarships/ResearchResearch/}
		\item Scholarship links: \url{http://www.gu.se/english/research/scholarships/scholarship_links/}
		\item Scholarships at University of Gothenburg: \url{http://www.gu.se/english/research/scholarships/gu/}
		\end{enumerate}
	\item Princeton University; The Graduate School: \url{http://gradschool.princeton.edu/financial/}
	\item National Association for Bilingual Education: \vspace{-0.2cm}
		\begin{enumerate} \itemsep -2pt
		\item List of Scholarships: \url{http://www.nabe.org/scholarship.html}
		\end{enumerate}
	\item {\bf Pennsylvania State University}: \vspace{-0.2cm}
		\begin{enumerate} \itemsep -2pt
		\item Office of Engineering Diversity; Penn State College of Engineering: \vspace{-0.1cm}
			\begin{enumerate} \itemsep -1pt
			\item Undergraduate Student Scholarships: \url{http://www.engr.psu.edu/oed/UnderScholarships.html}
			\item Graduate Student Scholarships: \url{http://www.engr.psu.edu/oed/GradScholarships.html}
			\item High School Student Scholarships: \url{http://www.engr.psu.edu/oed/HighSchoolScholarships.html}
			\item Disabled Student Scholarships: \url{http://www.engr.psu.edu/oed/DisabScholarships.html}
			\item Corporate Office of Engineering Diversity (OED) Scholarships: \url{http://www.engr.psu.edu/oed/OEDScholarships.html}
			\end{enumerate}
		\item University Fellowships Office: \vspace{-0.1cm}
			\begin{enumerate} \itemsep -1pt
			\item \url{http://sites.google.com/site/psuufo/}
			\item Prestigious Scholarships: \url{http://sites.google.com/site/psuufo/prestigious}
			\item Penn State Scholarships: \url{http://sites.google.com/site/psuufo/internal-scholarships}
			\item Other resources: \url{http://sites.google.com/site/psuufo/resources}
			\end{enumerate}
		\end{enumerate}
	\item {\bf Peterson's} college search: \vspace{-0.2cm}
		\begin{enumerate} \itemsep -2pt
		\item {\it College Scholarship Search}: \url{http://www.petersons.com/college-search/scholarship-search.aspx}
		\end{enumerate}
	\item Society for Industrial and Applied Mathematics (SIAM): \vspace{-0.2cm}
		\begin{enumerate} \itemsep -2pt
		\item Fellowship \& Research Opportunities: \url{http://www.siam.org/students/resources/fellowship.php}
		\end{enumerate}
	\item Institute of International Education (IIE): \vspace{-0.2cm}
		\begin{enumerate} \itemsep -2pt
		\item {\it Funding for US Study Online}: \vspace{-0.1cm}
			\begin{enumerate} \itemsep -1pt
			\item \url{http://www.fundingusstudy.org/}
			\end{enumerate}
		\end{enumerate}
	\end{enumerate}
\item --- --- --- --- --- --- --- --- --- --- --- --- --- --- --- --- --- --- --- --- --- --- --- --- --- --- --- --- --- --- ---
\item \colorbox{blue}{\bf Scholarships and Fellowships in Electrical and Computer Engineering}
% Scholarships and Fellowships in Electrical and Computer Engineering
\item IEEE: \vspace{-0.3cm}
	\begin{enumerate} \itemsep -2pt
	\item IEEE Awards, Competitions, and Scholarships: \url{http://www.ieee.org/membership_services/membership/students/awards/index.html}
	\item IEEE Circuits and Systems Society Pre-Doctoral Scholarships: Announced via email from IEEE Circuits and Systems Society
	\item IEEE Power \& Energy Society: \vspace{-0.2cm}
		\begin{enumerate} \itemsep -2pt
		\item G. Ray Ekenstam Memorial Scholarship: \vspace{-0.1cm}
			\begin{enumerate} \itemsep -1pt
			\item \url{http://www.ieee-pes.org/g-ray-ekenstam-memorial-scholarship}
			\item ``The Scholarship Fund awards, on an annual basis, a scholarship to a qualified undergraduate student who seeks an electrical engineering degree in the field of power or a related discipline, from an accredited US university or college.''
			\end{enumerate}
		\end{enumerate}
	\item IEEE Reliability Society: \vspace{-0.2cm}
		\begin{enumerate} \itemsep -2pt
		\item IEEE Reliability Society Scholarship: \url{http://www.ieee.org/portal/cms_docs_relsoc/relsoc/newsflipper/RS_Scholarship_Application.pdf} [ Look under the tab/option on ``Useful Information'' in the panel on the left. ]
		\end{enumerate}
	\end{enumerate}
\item The George Michael Memorial HPC Fellowship Program: \vspace{-0.3cm}
	\begin{enumerate} \itemsep -2pt
	\item The Association of Computing Machinery (ACM), IEEE Computer Society and SC Conference series have established the High Performance Computing (HPC) Ph.D. Fellowship Program. The SC conference is the International Conference for High Performance Computing, Networking, Storage, and Analysis. IEEE Computer Society and the Association for Computing Machinery are the sponsors for this conference.
	\item Every year, up to three fellowship recipients will each receive a stipend of at least \$5,000 (U.S.) for one academic year, plus travel support to attend the SC conference.
	\item See \url{http://sc10.supercomputing.org/?searchterm=fellowship&pg=GeorgeMichaelMemorial.html}
	\end{enumerate}
\item Intel: \vspace{-0.3cm}
	\begin{enumerate} \itemsep -2pt
	\item Intel Foundation Fellowship: \vspace{-0.2cm}
		\begin{enumerate} \itemsep -2pt
		\item Intel Foundation Ph.D. Fellowship % \url{http://intelscholarships.intel.com/}
		\item \url{http://www.intel.com/education/highered/studentprograms/fellowship.htm}
		\item This awards two-year fellowships to Ph.D. candidates pursuing leading-edge work in fields related to Intel's business and research interests.
		\item Fellowships are available at select U.S. universities, by invitation only, and focus on Ph.D. students who have completed at least one year of study.
		\item The fellowship includes a cash award (tuition/fees/stipend), an Intel mentor, and the opportunity to participate in an internship at Intel.
		\end{enumerate}
	\end{enumerate}
\item IBM: \vspace{-0.3cm}
	\begin{enumerate} \itemsep -2pt
	\item \url{http://www-304.ibm.com/jct01005c/university/scholars/phdfellowship}
	\item IBM Ph.D. Fellowship Award
	\item The IBM Ph.D. Fellowship Awards is an intensely competitive program which honors exceptional Ph.D. students in many academic disciplines and areas of study, for example: computer science and engineering, electrical and mechanical engineering , physical sciences (including chemistry, material sciences, and physics), mathematical sciences (including optimization), business sciences (including financial services, communication, and learning/knowledge), and service sciences, management, and engineering.
	\item IBM Herman Goldstine Postdoctoral Fellowship in Mathematical Sciences: \url{http://domino.research.ibm.com/comm/research_projects.nsf/pages/goldstine.index.html}
	\item Josef Raviv Memorial Postdoctoral Fellowship; see \url{http://domino.research.ibm.com/comm/research.nsf/pages/d.compsci.josef.raviv.general.info.html}, \url{http://domino.research.ibm.com/comm/research.nsf/pages/d.compsci.raviv.winner.html}, and \url{http://domino.research.ibm.com/comm/research.nsf/pages/d.compsci.raviv.winner2008.html}
	\end{enumerate}
\item AMD: Ph.D. fellowship, \url{http://developer.amd.com/programs/fellowship/Pages/default.aspx}
\item Qualcomm, {\it Qualcomm Innovation Fellowship} for Ph.D. students in Electrical Engineering and Computer Science at Stanford, UC Berkeley, UCLA, UCSD, and USC: \url{http://www.qualcomm.com/innovation/research/university_relations/innovation_fellowship/qinf10.html}
\item NVIDIA: \vspace{-0.3cm}
	\begin{enumerate} \itemsep -2pt
	\item NVIDIA Fellowship Program; see \url{http://www.nvidia.com/page/fellowship_programs.html}
	\end{enumerate}
\item Automatic RF Techniques Group (ARFTG): \vspace{-0.3cm}
	\begin{enumerate} \itemsep -2pt
	\item Microwave Measurement Student Fellowship (for ``graduate students who show promise and interest in pursuing research related to improvement of radio frequency and microwave measurement techniques''): \url{http://www.arftg.org/student_fellowship.html}
	\end{enumerate}
\item Gallium Arsenide Applications Symposium (GAAS) Association: \vspace{-0.3cm}
	\begin{enumerate} \itemsep -2pt
	\item GAAS PhD Student Fellowship (for Ph.D. students who have accepted papers at the European Microwave Integrated Circuits Conference, EuMIC): \url{http://www.gaas-symposium.org/english/awards_fellowship.htm} and \url{http://www.eumweek.com/2010/EuMIC.asp?id=c}
	\end{enumerate}
\item The Institution of Engineering and Technology, IET: \vspace{-0.3cm}
	\begin{enumerate} \itemsep -2pt
	\item Hudswell International Research Scholarship: \url{http://www.theiet.org/about/scholarships-awards/ambition/postgraduate1/hudswell-what.cfm}
	\item IET Postgraduate Scholarship: \url{http://www.theiet.org/about/scholarships-awards/ambition/postgraduate1/postgrad-what.cfm}
	\end{enumerate}
\item --- --- --- --- --- --- --- --- --- --- --- --- --- --- --- --- --- --- --- --- --- --- --- --- --- --- --- --- --- --- ---
\item \colorbox{blue}{\bf Scholarships and Fellowships in Computer Science}
% Scholarships and Fellowships in Computer Science
\item ACM Special Interest Group on Symbolic and Algebraic Manipulation (SIGSAM): List of Ph.D. positions in computer algebra and symbolic computation, as listed by SIGSAM; see \url{http://www.sigsam.org/opportunities.phtml?searchterm=fellowship}
\item Carnegie Mellon University: \vspace{-0.3cm}
	\begin{enumerate} \itemsep -2pt
	\item women@SCS School of Computer Science: \vspace{-0.2cm}
		\begin{enumerate} \itemsep -2pt
		\item Individuals, Corporations \& Organizations: \url{http://women.cs.cmu.edu/Resources/Funding/}
		\end{enumerate}
	\end{enumerate}
\item IBM: \vspace{-0.3cm}
	\begin{enumerate} \itemsep -2pt
	\item \url{http://www-304.ibm.com/jct01005c/university/scholars/phdfellowship}
	\item IBM Ph.D. Fellowship Award
	\item The IBM Ph.D. Fellowship Awards is an intensely competitive program which honors exceptional Ph.D. students in many academic disciplines and areas of study, for example: computer science and engineering, electrical and mechanical engineering , physical sciences (including chemistry, material sciences, and physics), mathematical sciences (including optimization), business sciences (including financial services, communication, and learning/knowledge), and service sciences, management, and engineering.
	\item IBM Herman Goldstine Postdoctoral Fellowship in Mathematical Sciences: \url{http://domino.research.ibm.com/comm/research_projects.nsf/pages/goldstine.index.html}
	\item Josef Raviv Memorial Postdoctoral Fellowship; see \url{http://domino.research.ibm.com/comm/research.nsf/pages/d.compsci.josef.raviv.general.info.html}, \url{http://domino.research.ibm.com/comm/research.nsf/pages/d.compsci.raviv.winner.html}, and \url{http://domino.research.ibm.com/comm/research.nsf/pages/d.compsci.raviv.winner2008.html}
	\end{enumerate}
\item Computing Innovation Fellows (CIFellows); post my profile on \url{http://cifellows.org/profiles/}; also see \url{http://www.cifellows.org/}
\item Microsoft: \vspace{-0.3cm}
	\begin{enumerate} \itemsep -2pt
	\item Microsoft Research Graduate Women's Scholarship: \url{http://research.microsoft.com/en-us/collaboration/awards/fellows-women.aspx}
	\item Microsoft Research PhD Fellowship: \url{http://research.microsoft.com/en-us/collaboration/awards/apply-us.aspx}
	\end{enumerate}
\item Google: \vspace{-0.3cm}
	\begin{enumerate} \itemsep -2pt
	\item Google Fellowship Program; see \url{http://googleblog.blogspot.com/2009/05/best-and-brightest.html}
	\end{enumerate}
\item NVIDIA: \vspace{-0.3cm}
	\begin{enumerate} \itemsep -2pt
	\item NVIDIA Fellowship Program; see \url{http://www.nvidia.com/page/fellowship_programs.html}
	\end{enumerate}
\item Facebook Ph.D. Fellowship: \url{http://www.facebook.com/careers/fellowship.php}
\item Yahoo! Labs: Yahoo! Key Scientific Challenges Program, \url{http://labs.yahoo.com/ksc}
\item Qualcomm, {\it Qualcomm Innovation Fellowship} for Ph.D. students in Electrical Engineering and Computer Science at Stanford, UC Berkeley, UCLA, UCSD, and USC: \url{http://www.qualcomm.com/innovation/research/university_relations/innovation_fellowship/qinf10.html} and \url{http://www.qualcomm.com/innovation/research/university_relations/innovation_fellowship/}
\item Computing Research Association (CRA): Outstanding Undergraduate Researchers, \url{http://www.cra.org/awards/undergrad-current/}
\item {\color{blue} European Research Consortium for Informatics and Mathematics (ERCIM)}: \vspace{-0.3cm}
	\begin{enumerate} \itemsep -2pt
	\item ERCIM Alain Bensoussan Fellowship Programme (for Ph.D. degree holders in selected research areas): \url{http://fellowship.ercim.eu/} and \url{http://www.ercim.eu/news/283-fellowship-programme}; research areas are listed at: \url{http://fellowship.ercim.eu/home/topic}. Deadlines are on April 30 and September 30 annually.
	\end{enumerate}
\item {\it Theory Matters Wiki}; Theoretical Computer Science (TCS) Advocacy Wiki: \vspace{-0.3cm}
	\begin{enumerate} \itemsep -2pt
	\item Funding Opportunities and Tips: \url{http://theorymatters.org/pmwiki/pmwiki.php?n=Main.FundingOpportunities}
	\end{enumerate}
\item Kurt G{\"{o}}del Research Prize Fellowship: \vspace{-0.3cm}
	\begin{enumerate} \itemsep -2pt
	\item 2 Ph.D. (pre-doctoral) fellowships
	\item 2 post-doctoral fellowships
	\item 1 unrestricted fellowship
	\item $[$Scope of the$]$ original fellowship proposals [includes] the areas of: \vspace{-0.2cm}
		\begin{enumerate} \itemsep -2pt
		\item set theory
		\item recursion theory
		\item proof theory/intuitionism
		\item model theory
		\item computer assisted reasoning
		\item philosophy of mathematics 
		\end{enumerate}
	\item All fellowship proposals, regardless of subject area, will be judged according to: \vspace{-0.2cm}
		\begin{enumerate} \itemsep -2pt
		\item the relevance and resemblance of the research (finished and proposed) to the great insights and originality of Kurt G{\"{o}}del
		\item its general interest and clarity of motivation
		\item its rigorous scientific quality and depth. 
		\end{enumerate}
	\item \url{http://fellowship.logic.at/}
	\end{enumerate}
\item Hewlett-Packard Company: \vspace{-0.3cm}
	\begin{enumerate} \itemsep -2pt
	\item Hewlett-Packard Labs India (Bengaluru / Bangalore): \vspace{-0.2cm}
		\begin{enumerate} \itemsep -2pt
		\item {\it BITS - HP Labs India Ph.D. Fellowship} for Research related to Information Technologies: \vspace{-0.1cm}
			\begin{enumerate} \itemsep -1pt
			\item \url{http://www.hpl.hp.com/india/bits-hplindia_phd/index.html} or \url{http://www.hpl.hp.com/india/bits-hplindia_phd/}
			\item \url{http://www.hpl.hp.com/india/bits-hplindia_phd/iiitbphd.html}
			\item BITS, Pilani and HP Labs India jointly offer a unique PhD fellowship for research in Information and Communication Technologies (ICT) relevant to fast-growing markets like India.
			\item HP Labs India currently has ongoing Ph.D. Fellowships with BITS Pilani and IIIT, Bangalore: \url{http://www.hpl.hp.com/india/bits-hplindia_phd/university.html}
			\end{enumerate}
		\item Open Innovation Office: \vspace{-0.1cm}
			\begin{enumerate} \itemsep -1pt
			\item \url{http://www.hpl.hp.com/open_innovation/}
			\item HP Labs Innovation Research Program (IRP): \url{http://www.hpl.hp.com/open_innovation/irp/index.html}
			\end{enumerate}
		\end{enumerate}
	\end{enumerate}
\item Code for America (CfA): \vspace{-0.3cm}
	\begin{enumerate} \itemsep -2pt
	\item CfA Fellowship (develop web applications for local governments in the US): \url{http://codeforamerica.org/fellows/}
	\end{enumerate}
\item University of Minnesota, Twin Cities: \vspace{-0.3cm}
	\begin{enumerate} \itemsep -2pt
	\item College of Science and Engineering: \vspace{-0.2cm}
		\begin{enumerate} \itemsep -2pt
		\item Charles Babbage Institute: \vspace{-0.1cm}
			\begin{enumerate} \itemsep -1pt
			\item Adelle and Erwin Tomash Graduate Fellowship (for Ph.D. candidates doing research in the history of IT/computing - all but dissertation Ph.D. students only): \url{http://www.cbi.umn.edu/research/tfellowship.html}
			\item Arthur L. Norberg Travel Fund (short-term grants-in-aid to help scholars with travel expenses to use archival collections at the Charles Babbage Institute): \url{http://www.cbi.umn.edu/research/ntravelfund.html}
			\end{enumerate}
		\end{enumerate}
	\end{enumerate}
\item --- --- --- --- --- --- --- --- --- --- --- --- --- --- --- --- --- --- --- --- --- --- --- --- --- --- --- --- --- --- ---
\item \colorbox{blue}{\bf Scholarships and Fellowships in Biomedical Engineering}
% Scholarships and Fellowships in Biomedical Engineering
\item Whitaker International Fellows and Scholars Program: \vspace{-0.3cm}
	\begin{enumerate} \itemsep -2pt
	\item For graduate/Ph.D. students and postdocs in biomedical engineering
	\item \url{http://www.whitaker.org/home}
	\end{enumerate}
\item --- --- --- --- --- --- --- --- --- --- --- --- --- --- --- --- --- --- --- --- --- --- --- --- --- --- --- --- --- --- ---
\item \colorbox{blue}{\bf Scholarships and Fellowships in Optical Engineering}
% Scholarships and Fellowships in Optical Engineering
\item {\it SPIE} -- The International Society for Optical Engineering: \vspace{-0.3cm}
	\begin{enumerate} \itemsep -2pt
	\item ``SPIE Scholarship Program'' for undergraduates or graduate students studying optics, photonics, imaging, or optoelectronics program or related discipline (e.g., physics, electrical engineering): \url{http://spie.org//x1733.xml?WT.svl=mddm14}
	\item Other scholarships (including scholarships for students doing research in nanolithography techniques and lasers): \url{http://spie.org/x1736.xml}
	\end{enumerate}
\item {\it Kidger Optics Associates} Michael Kidger Memorial Scholarship (to a college freshman, or sophomore of optical design): \url{http://www.kidger.com/mkms_requirements.html}
\item --- --- --- --- --- --- --- --- --- --- --- --- --- --- --- --- --- --- --- --- --- --- --- --- --- --- --- --- --- --- ---
\item \colorbox{blue}{\bf Scholarships and Fellowships in Mechanical Engineering}
% Scholarships and Fellowships in Mechanical Engineering
\item American Society of Mechanical Engineers (ASME): \vspace{-0.3cm}
	\begin{enumerate} \itemsep -2pt
	\item Graduate Teaching Fellowships (for Ph.D. students in mechanical engineering): \url{http://www.asme.org/Education/College/FinancialAid/Graduate_Teaching_Fellowships.cfm}
	\item ASME Scholarships: \vspace{-0.2cm}
		\begin{enumerate} \itemsep -2pt
		\item \url{http://www.asme.org/Education/College/FinancialAid/Scholarships.cfm}
		\item US Undergraduates: \url{http://www.asme.org/Education/College/FinancialAid/US_Undergraduates.cfm}
		\item Graduate Students: \url{http://www.asme.org/Education/College/FinancialAid/Graduate_Students.cfm}
		\item International Students: \url{http://www.asme.org/Education/College/FinancialAid/International_Undergraduates.cfm}
		\end{enumerate}
	\item Auxiliary Scholarships: \vspace{-0.2cm}
		\begin{enumerate} \itemsep -2pt
		\item \url{http://www.asme.org/Education/College/FinancialAid/Auxiliary_Scholarships.cfm}
		\item Undergraduate Scholarships: \url{http://www.asme.org/Education/College/FinancialAid/Undergraduate_Scholarships.cfm}
		\item Graduate Scholarships: \url{http://www.asme.org/Education/College/FinancialAid/Graduate_Scholarships.cfm}
		\item Rice-Cullimore Scholarship (for international graduate students in the US): \url{http://www.asme.org/Education/College/FinancialAid/RiceCullimore_Scholarship.cfm}
		\end{enumerate}
	\item International Petroleum Institute�s College Scholarships (for undergraduates): \url{http://www.asme-ipti.org/public/pagscholarshipprograms.aspx}
	\item International Petroleum Institute�s Graduate Fellowship (for individuals entering a graduate program in mechanical engineering, and has an interest in the petroleum industry): \url{http://www.asme-ipti.org/public/pagscholarshipprograms.aspx} and \url{http://www.asme.org/Communities/Students/Grad/Fellowships.cfm}
	\end{enumerate}
\item --- --- --- --- --- --- --- --- --- --- --- --- --- --- --- --- --- --- --- --- --- --- --- --- --- --- --- --- --- --- ---
\item \colorbox{blue}{\bf Scholarships and Fellowships in Civil Engineering}
% Scholarships and Fellowships in Civil Engineering
\item American Society of Civil Engineers (ASCE): \vspace{-0.3cm}
	\begin{enumerate} \itemsep -2pt
	\item Jack E. Leisch Memorial National Graduate Fellowship (for graduate students in transportation/traffic engineering): \url{http://www.asce.org/Content.aspx?id=25021}
	\item Scholarships \& Fellowships (for undergraduates and graduate students): \url{http://www.asce.org/Content.aspx?id=18337}
	\end{enumerate}
\item American Concrete Institute (ACI): \vspace{-0.3cm}
	\begin{enumerate} \itemsep -2pt
	\item ACI Foundation Fellowships \& Scholarships: \url{http://www.concrete.org/STUDENTS/ST_SCHOLARSHIPS.HTM}
	\end{enumerate}
\item Institute of Transportation Engineers: \vspace{-0.3cm}
	\begin{enumerate} \itemsep -2pt
	\item Burton W. Marsh Fellowship for Graduate Study in Traffic and Transportation Engineering: \url{http://www.ite.org/education/Burton_W_MarshFellowship.asp}
	\end{enumerate}
\item --- --- --- --- --- --- --- --- --- --- --- --- --- --- --- --- --- --- --- --- --- --- --- --- --- --- --- --- --- --- ---
\item \colorbox{blue}{\bf Scholarships and Fellowships in Chemical Engineering}
% Scholarships and Fellowships in Chemical Engineering
\item American Institute of Chemical Engineers (AIChE) scholarships (includes scholarships for underrepresented minorities): \url{http://www.aiche.org/Students/Scholarships/index.aspx}
\item --- --- --- --- --- --- --- --- --- --- --- --- --- --- --- --- --- --- --- --- --- --- --- --- --- --- --- --- --- --- ---
\item \colorbox{blue}{\bf Scholarships and Fellowships in Aerospace Engineering}
% Scholarships and Fellowships in Aerospace Engineering
\item American Institute of Aeronautics and Astronautics (AIAA): \vspace{-0.3cm}
	\begin{enumerate} \itemsep -2pt
	\item AIAA Foundation Scholarships: \vspace{-0.2cm}
		\begin{enumerate} \itemsep -2pt
		\item \url{http://www.aiaa.org/content.cfm?pageid=211}
		\item For undergraduates and graduate students
		\item Named scholarships for undergraduates are: \vspace{-0.1cm}
			\begin{enumerate} \itemsep -1pt
			\item \url{http://www.aiaa.org/content.cfm?pageid=226}
			\item A. Thomas Young Scholarship
			\item L. S. ``Skip'' Fletcher Scholarship 
			\item Sam F. Iacobellis Scholarship
			\item Robert L. Crippen Scholarship
			\item E. C. ``Pete'' Aldridge Scholarship
			\item Liquid Propulsion Technical Committee Scholarship
			\item Space Transportation Technical Committee Scholarship
			\item Digital Avionics Technical Committee Scholarship (4)
			\item Next Century of Flight Scholarship (2)
			\item Leatrice Gregory Pendray Scholarship
			\end{enumerate}
		\item Awards for graduate students: \vspace{-0.1cm}
			\begin{enumerate} \itemsep -1pt
			\item \url{http://www.aiaa.org/content.cfm?pageid=227}
			\item Martin Summerfield Propellants and Combustion Graduate Award
			\item Guidance, Navigation, And Control Graduate Award
			\item Gordon C. Oates Air Breathing Propulsion Graduate Award
			\item William T. Piper, Sr. General Aviation Systems Graduate Award
			\item Orville and Wilbur Wright Graduate Award
			\item John Leland Atwood Graduate Award
			\item Open Topic Graduate Award
			\end{enumerate}
		\end{enumerate}
	\item Student Design Competition Award: \url{http://www.aiaa.org/content.cfm?pageid=401}
	\end{enumerate}
\item --- --- --- --- --- --- --- --- --- --- --- --- --- --- --- --- --- --- --- --- --- --- --- --- --- --- --- --- --- --- ---
\item \colorbox{blue}{\bf Scholarships and Fellowships in Mathematics}
% Scholarships and Fellowships in Mathematics
\item Association for Women in Mathematics (AWM): \vspace{-0.3cm}
	\begin{enumerate} \itemsep -2pt
	\item Travel grants: \url{http://sites.google.com/site/awmmath/programs/travel-grants}
	\item Alice T. Schafer Mathematics Prize for excellence in mathematics by an undergraduate woman: \url{http://sites.google.com/site/awmmath/programs/schafer-prize}
	\item The {\it Ruth I. Michler Memorial Prize} of the AWM is awarded annually to a woman recently promoted to Associate Professor or an equivalent position in the mathematical sciences: \url{http://sites.google.com/site/awmmath/programs/michler-prize}
	\end{enumerate}
\item Seth Bonder Scholarship for Applied Operations Research in Health Services: \url{http://www.informs.org/Recognize-Excellence/INFORMS-Community-Prizes-and-Awards/Seth-Bonder-Scholarship-for-Applied-Operations-Research-in-Health-Services}
\item Oberwolfach Foundation: \vspace{-0.3cm}
	\begin{enumerate} \itemsep -2pt
	\item Oberwolfach Prize (for young European mathematicians): \url{http://www.mfo.de/programme/prize/}
	\item John Todd Fellowship (or John Todd Award) [for young excellent mathematicians working in numerical analysis]: \url{http://www.mfo.de/programme/todd/}
	\end{enumerate}
\item Clay Mathematics Institute: Clay Research Award, \url{http://www.claymath.org/research_award/}
\item --- --- --- --- --- --- --- --- --- --- --- --- --- --- --- --- --- --- --- --- --- --- --- --- --- --- --- --- --- --- ---
\item \colorbox{blue}{\bf Scholarships and Fellowships in Science}
% Scholarships and Fellowships in Science
\item {\it Science.gov} (USA.gov for Science): \vspace{-0.3cm}
	\begin{enumerate} \itemsep -2pt
	\item Internship and Fellowship Opportunities in Science for Undergraduate Students: \url{http://www.science.gov/internships/undergrad.html}
	\item Graduate Students/Postdoctoral Fellowships: \url{http://www.science.gov/internships/graduate.html}
	\end{enumerate}
\item Heinz Family Philanthropies: \vspace{-0.3cm}
	\begin{enumerate} \itemsep -2pt
	\item Teresa Heinz Scholars for Environmental Research program (for Ph.D./MS students working on their thesis in environmental science/engineering) at selected universities: \url{http://www.heinzfamily.org/programs/environmentalscholars.html}
	\item \url{http://www.heinzfamily.org/}
	\end{enumerate}
\item Mayo Clinic: \vspace{-0.3cm}
	\begin{enumerate} \itemsep -2pt
	\item Postbaccalaureate Research Education Program (PREP): \url{http://www.mayo.edu/mgs/postbac-program.html}
	\end{enumerate}
\item {\it American Chemical Society (ACS)}: \vspace{-0.3cm}
	\begin{enumerate} \itemsep -2pt
	\item ACS-Hach Land Grant Undergraduate Scholarship (for chemistry undergraduates at a partner institution of ACS, and who plan to become chemistry teachers in US high schools): \url{http://portal.acs.org/portal/acs/corg/content?_nfpb=true&_pageLabel=PP_SUPERARTICLE&node_id=2243&use_sec=false&sec_url_var=region1&__uuid=eb054647-53e0-4594-81e8-8ef49159f3f4}
	\item ACS-Hach Second Career Teacher Scholarship (for graduates in chemistry or related areas who are entering an education masters program or teacher certification program): \url{http://portal.acs.org/portal/acs/corg/content?_nfpb=true&_pageLabel=PP_SUPERARTICLE&node_id=2244&use_sec=false&sec_url_var=region1&__uuid=4c27333f-4aad-481e-aaa4-f1db045d4eb4}
	\item ACS Scholars Program (for undergraduate underrepresented minorities majoring in chemistry, biochemistry, or chemical engineering): \url{http://portal.acs.org/portal/acs/corg/content?_nfpb=true&_pageLabel=PP_SUPERARTICLE&node_id=1650&use_sec=false&sec_url_var=region1&__uuid=b3b583cf-18ae-4fb0-9375-33f75a0ccf49}
	\item Scholarships: \url{http://portal.acs.org/portal/acs/corg/content?_nfpb=true&_pageLabel=PP_TRANSITIONMAIN&node_id=630&use_sec=false&sec_url_var=region1&__uuid=98e85c05-be75-4283-a97c-7a63ab4c3178}
	\end{enumerate}
\item European Molecular Biology Organization: \vspace{-0.3cm}
	\begin{enumerate} \itemsep -2pt
	\item EMBO Short-Term Fellowships (for junior researchers, including Ph.D. students): \url{http://www.embo.org/programmes/fellowships/short-term.html}
	\item EMBO Long-Term Fellowships (for junior researchers/postdocs): \url{http://www.embo.org/programmes/fellowships/long-term.html}
	\end{enumerate}
\item L'OR{\'{E}}AL: \vspace{-0.3cm}
	\begin{enumerate} \itemsep -2pt
	\item ``For Women in Science'' program: \url{http://www.lorealusa.com/forwomeninscience} or \url{http://www.lorealusa.com/_en/_us/index.aspx?direct1=00008&direct2=00008/00001}
	\item Alternatively, go to \url{http://www.lorealusa.com/_en/_us/} and select the ``For Women in Science'' tab.
	\item Check out the ``L'Or{\'{e}}al USA Fellowships for Women in Science'' (US postdocs), ``UNESCO-L'Or{\'{e}}al Fellowships for Women in Science'' (for female Ph.D. students and postdocs in the life sciences), and the ``L'Or{\'{e}}al-UNESCO Awardss for Women in Science'' (for distinguished female scientists)
	\end{enumerate}
\item American Institute of Physics (AIP): \vspace{-0.3cm}
	\begin{enumerate} \itemsep -2pt
	\item AIP and Member Society Government Science Fellowships: \vspace{-0.2cm}
		\begin{enumerate} \itemsep -2pt
		\item \url{http://www.aip.org/gov/fellowships.html}
		\item American Institute of Physics State Department Science Fellowship: \url{http://www.aip.org/gov/fellowships/sdf.html}
		\item American Institute of Physics Congressional Science Fellowship: \url{http://www.aip.org/gov/fellowships/cf.html}
		\item American Physical Society Congressional Science Fellowship: \url{http://www.aps.org/policy/fellowships/congressional.cfm}
		\item American Geophysical Union Congressional Science Fellowship: \url{http://www.agu.org/sci_pol/cong_fellowship/}
		\item Optical Society of America Congressional Science Fellowships: \url{http://www.osa.org/about_osa/public_policy/congressional_fellowships/default.aspx}
		\item For US citizens with good track records in research
		\end{enumerate}
	\item American Geophysical Union: \vspace{-0.2cm}
		\begin{enumerate} \itemsep -2pt
		\item Research Grants and Awards: \url{http://www.agu.org/about/honors/research_grants/}
		\item Student Travel Grants: \url{http://www.agu.org/education/grants/travel.shtml}
		\item Research Grants \& Awards: \url{http://www.agu.org/education/grants/research.shtml}
		\item Mass Media Fellowship: \url{http://www.agu.org/news/mass_media_fellowship/}
		\end{enumerate}
	\item Society of Physics Students (SPS): \vspace{-0.2cm}
		\begin{enumerate} \itemsep -2pt
		\item SPS Scholarships: \url{http://www.spsnational.org/programs/scholarships/}
		\item SPS Awards: \url{http://www.spsnational.org/programs/awards/}
		\end{enumerate}
	\end{enumerate}
\item Consortium for Ocean Leadership: \vspace{-0.3cm}
	\begin{enumerate} \itemsep -2pt
	\item Employment, Internships, and Opportunities [ includes funding opportunities for researchers (professors, postdocs, and grad students) ]: \url{http://www.oceanleadership.org/about-ocean-leadership/ocean-of-opportunities/}
	\item HBCU Fellowship: Ocean Leadership/IODP-USIO for Students of Historically Black Colleges and Universities, \url{http://www.oceanleadership.org/education/diversity/hbcu-fellowship/}
	\item HBCU Educator at Sea: \url{http://www.oceanleadership.org/education/diversity/hbcu-educator/}
	\item MS PHD's Professional Development Program: The Minorities Striving and Pursuing Higher Degrees of Success in the Earth System Sciences (MS PHD'S) Professional Development Program, \url{http://www.oceanleadership.org/education/diversity/ms-phds-professional-development-program/}
	\item Schlanger Ocean Drilling Fellowship Program (merit-based awards for outstanding graduate students to conduct research related to the Integrated Ocean Drilling Program): \url{http://www.oceanleadership.org/programs-and-partnerships/usssp/schlanger-fellowship/}
	\end{enumerate}
\item American Geological Institute Foundation: \vspace{-0.3cm}
	\begin{enumerate} \itemsep -2pt
	\item William L. Fisher Congressional Geoscience Fellowship (for young geoscientists to get engaged in {\bf public policy}): \url{http://www.agifoundation.org/govtaffairs.html} and \url{http://www.agifoundation.org/endowments.html}
	\item AGI Minority Participation Program: Minority Participation Program Geoscience Student Scholarships for ``underrepresented ethnic-minority (undergraduate or graduate) students in the geosciences'', \url{http://www.agiweb.org/mpp/index.html}
	\end{enumerate}
\item Lady Davis Institute/Jewish General Hospital: \vspace{-0.3cm}
	\begin{enumerate} \itemsep -2pt
	\item Awards for ``graduate students (in biomedical science) and post-doctoral fellows/clinical fellows'': \url{http://www.ladydavis.ca/en/awards}
	\end{enumerate}
\item Adolph C. and Mary Sprague Miller Institute for Basic Research in Science: \vspace{-0.3cm}
	\begin{enumerate} \itemsep -2pt
	\item Miller Fellowships (for outstanding recent Ph.D.s / postdoctoral fellowship): \url{http://millerinstitute.berkeley.edu/topage.php?nav=11&to=1} or \url{http://millerinstitute.berkeley.edu/page.php?nav=11}
	\item Visiting Miller Research Professorships (for professors and research scientists): \url{http://millerinstitute.berkeley.edu/topage.php?nav=24&to=1} or \url{http://millerinstitute.berkeley.edu/page.php?nav=24}
	\item Miller Research Professorships (for professors in the UC system): \url{http://millerinstitute.berkeley.edu/topage.php?nav=15&to=1} or \url{http://millerinstitute.berkeley.edu/page.php?nav=15}
	\item Miller Senior Fellowships (Nominations are solicited by invitation only; Senior Fellow appointments are made to tenured UC Berkeley faculty for five years, possibly renewable for a subsequent five years, but no longer.): \url{http://millerinstitute.berkeley.edu/topage.php?nav=126&to=1}
	\end{enumerate}
\item Funda{\c{c}}{\~{a}}o para a Ci{\^{e}}ncia e a Tecnologia (FCT); Minist{\'{e}}rio da Ci{\^{e}}ncia, Technologia e Ensino Superior (MCTES): International Prize Fernando Gil in Philosophy of Science, \url{http://alfa.fct.mctes.pt/apoios/premios/fernando_gil/index.phtml.pt}
\item Wellcome Trust: \vspace{-0.3cm}
	\begin{enumerate} \itemsep -2pt
	\item Wellcome Trust Sanger Institute: \vspace{-0.2cm}
		\begin{enumerate} \itemsep -2pt
		\item \url{http://www.sanger.ac.uk/workstudy/}
		\item Postdoctoral fellows (for research in genomics): \url{http://www.sanger.ac.uk/workstudy/career/postdocs/}
		\item Graduate program (for research in genomics): \url{http://www.sanger.ac.uk/workstudy/phd/}
		\item Student placements and work experience (for research in genomics): \url{http://www.sanger.ac.uk/workstudy/placements/}
		\end{enumerate}
	\end{enumerate}
\item Paul B. Beeson Career Development Awards in Aging Research Program (formerly the Beeson Physician Faculty Scholars Program): \vspace{-0.3cm}
	\begin{enumerate} \itemsep -2pt
	\item \url{http://www.beeson.org/}
	\item ``Today, the Beeson program continues to foster the independent research careers of clinically trained investigators -- a growing cadre of talented physician-scientists -- whose research and leadership are enhancing the health and quality of life of Americans, particularly older people.''
	\item About the Program: \url{http://www.beeson.org/program_hx.cfm}
	\end{enumerate}
\item American Mathematical Society: \vspace{-0.3cm}
	\begin{enumerate} \itemsep -2pt
	\item AMS Fellowships and Scholarships: \vspace{-0.2cm}
		\begin{enumerate} \itemsep -2pt
		\item \url{http://e-math.ams.org/programs/ams-fellowships/ams-fellowships}
		\item AMS Centennial Research Fellowship Program: \url{http://e-math.ams.org/programs/ams-fellowships/centennial-fellow/emp-centflyer}
		\item Waldemar J. Trijitzinsky Memorial Awards: \url{http://e-math.ams.org/programs/ams-fellowships/trjitzinsky/trjitzinsky-award}
		\item Other Sources of Funding: \url{http://e-math.ams.org/programs/funding/funding}
		\end{enumerate}
	\end{enumerate}
\item --- --- --- --- --- --- --- --- --- --- --- --- --- --- --- --- --- --- --- --- --- --- --- --- --- --- --- --- --- --- ---
\item \colorbox{blue}{\bf Scholarships and Fellowships in Medicine}
% Scholarships and Fellowships in Medicine
\item Sarnoff Medical Student Research Fellowship Program (for US medical students interested in cardiovascular research): \url{http://www.sarnoffendowment.org/}
\item Mayo Clinic: \vspace{-0.3cm}
	\begin{enumerate} \itemsep -2pt
	\item Postbaccalaureate Research Education Program (PREP): \url{http://www.mayo.edu/mgs/postbac-program.html}
	\end{enumerate}
\item Paul B. Beeson Career Development Awards in Aging Research Program (formerly the Beeson Physician Faculty Scholars Program): \vspace{-0.3cm}
	\begin{enumerate} \itemsep -2pt
	\item \url{http://www.beeson.org/}
	\item ``Today, the Beeson program continues to foster the independent research careers of clinically trained investigators -- a growing cadre of talented physician-scientists -- whose research and leadership are enhancing the health and quality of life of Americans, particularly older people.''
	\item About the Program: \url{http://www.beeson.org/program_hx.cfm}
	\end{enumerate}
\item --- --- --- --- --- --- --- --- --- --- --- --- --- --- --- --- --- --- --- --- --- --- --- --- --- --- --- --- --- --- ---
\item \colorbox{blue}{\bf Scholarships and Fellowships in Science and Engineering}
% Scholarships and Fellowships in Science and Engineering
\item National Academies: \vspace{-0.3cm}
	\begin{enumerate} \itemsep -2pt
	\item Research Associateship Programs (graduate, postdoctoral, and senior level research opportunities): \url{http://sites.nationalacademies.org/pga/rap/}
	\item Ford Foundation Fellowship Programs (predoctoral, dissertation or postdoctoral fellowships for individuals seeking academic careers in science and engineering): \url{http://sites.nationalacademies.org/PGA/FordFellowships/index.htm}
	\item \url{http://nationalacademies.org/grantprograms.html}
	\item \url{http://sites.nationalacademies.org/pga/fellowships/}
	\item List of Fellowship, Scholarship, and Grant Databases: \url{http://sites.nationalacademies.org/PGA/Fellowships/PGA_046300}
	\item List of Outside Fellowships, Scholarships, and Grants Websites: \url{http://sites.nationalacademies.org/PGA/Fellowships/PGA_046301}
	\item Awards for junior and mid-career researchers: \url{http://www.nasonline.org/site/PageServer?pagename=AWARDS_main}
	\item National Academy of Engineering, NAE: \vspace{-0.2cm}
		\begin{enumerate} \itemsep -2pt
		\item NAE Grand Challenges Scholars Program: \url{http://www.grandchallengescholars.org/}
		\end{enumerate}
	\item National Science Foundation: \vspace{-0.2cm}
		\begin{enumerate} \itemsep -2pt
		\item International Research Fellowship Program (IRFP) for junior scientists and engineers: \url{http://www.nsf.gov/funding/pgm_summ.jsp?pims_id=5179}
		\item Integrative Graduate Education and Research Traineeship Program (IGERT) for undergraduates and graduate students in STEM: \url{http://www.nsf.gov/funding/pgm_summ.jsp?pims_id=12759}
		\item National Science Foundation's Graduate Research Fellowship Program (GRFP) for students seeking research degrees in STEM: \url{http://www.nsfgrfp.org/}
		\item NSF Alliances for Graduate Education and the Professoriate (AGEP) program (to help underrepresented minorities obtain graduate degrees in STEM and prepare them for faculty positions in academia): \url{http://www.nsfagep.org/}
		\item National Science Foundation's (NSF) East Asia and Pacific Summer Institutes (EAPSI) program: \vspace{-0.1cm}
			\begin{enumerate} \itemsep -1pt
			\item \url{http://www.nsf.gov/funding/pgm_summ.jsp?pims_id=5284}
			\item The East Asia and Pacific Summer Institutes (EAPSI) provide U.S. graduate students in science and engineering: \vspace{-0.1cm}
				\begin{itemize} \itemsep -1pt
				\item first-hand research experiences in Australia, China, Japan, Korea, New Zealand, Singapore or Taiwan
				\item an introduction to the science, science policy, and scientific infrastructure of the respective location
				\item an orientation to the society, culture and language.
				\end{itemize}
			\item ``The primary goals of EAPSI are to introduce students to East Asia and Pacific science and engineering in the context of a research setting, and to help students initiate scientific relationships that will better enable future collaboration with foreign counterparts.''
			\item ``All institutes, except Japan, last approximately eight weeks from June to August. Japan lasts approximately ten weeks from June to August (specific dates are available and updated at \url{http://www.nsfsi.org/}).''
			\item Example of Ph.D. student, Jakub Szefer, from Prof. Ruby Lee's lab at Princeton University, who interned with Prof. Cheng Chen-Mou from National Taiwan University: \url{http://www.nsf.gov/discoveries/disc_summ.jsp?cntn_id=118116&org=NSF}
			\end{enumerate}
		\end{enumerate}
	\end{enumerate}
\item United States Department of Defense (DoD): \vspace{-0.3cm}
	\begin{enumerate} \itemsep -2pt
	\item National Defense Education Program; Defense Advanced Research Projects Agency (DARPA): \vspace{-0.2cm}
		\begin{enumerate} \itemsep -2pt
		\item Science, Mathematics, and Research for Transformation (SMART) scholarship program: \vspace{-0.1cm}
			\begin{itemize} \itemsep -1pt
			\item \url{http://smart.asee.org/}
			\item Co-organized by the American Society for Engineering Education
			\end{itemize}
		\item National Security Science and Engineering Faculty Fellowships (NSSEFF): \url{http://www.ndep.us/ProgNSSEFF.aspx}
		\end{enumerate}
	\end{enumerate}
\item National Society of Professional Engineers: \vspace{-0.3cm}
	\begin{enumerate} \itemsep -2pt
	\item Scholarships for undergraduates and graduate students: \url{http://www.nspe.org/Students/Scholarships/index.html}
	\item NSPE-PEC George B. Hightower, P.E. Fellowship (for an outstanding engineering graduate student): \url{http://www.nspe.org/InterestGroups/PEC/Resources/Awards/hightower_fellowship.html}
	\item PEG Management Fellowship: \vspace{-0.2cm}
		\begin{enumerate} \itemsep -2pt
		\item \url{http://www.nspe.org/InterestGroups/PEG/Resources/AwardsAndScholarships/peg_fellowship.html}
		\item ``This scholarship is designed for graduate students who are pursuing an MBA, a master's degree in engineering management, or a master's degree in public administration.''
		\end{enumerate}
	\end{enumerate}
\item Technion -- Israel Institute of Technology: \vspace{-0.3cm}
	\begin{enumerate} \itemsep -2pt
	\item Department of Mathematics: Anna and Paul Erdos postdoctoral Fellowship, \url{http://www.math.technion.ac.il/Site/people/positions.html}
	\item Lady Davis Postdoctoral Fellowship
	\item Department of Electrical Engineering: \vspace{-0.2cm}
		\begin{enumerate} \itemsep -2pt
		\item The Andrew and Erna Finci Viterbi Fellowship Program (for graduate and post-doctoral fellows), \url{http://webee.technion.ac.il/Research/Fellowship-Programs}
		\item Lady Davis Fellowship Trust: Technion Fellowships (for visiting professors, post-doctoral researchers, as well as Masters and Ph.D. students), \url{http://ldft.huji.ac.il/upload/info/}
		\item \url{http://webee.technion.ac.il/Research/Fellowship-Programs}
		\end{enumerate}
	\end{enumerate}
\item Hebrew University: \vspace{-0.3cm}
	\begin{enumerate} \itemsep -2pt
	\item Lady Davis Fellowship Trust: Technion Fellowships (for visiting professors, post-doctoral researchers, as well as Masters and Ph.D. students), \url{http://ldft.huji.ac.il/upload/info/infoHUa.html}
	\end{enumerate}
\item Hertz Foundation: \vspace{-0.3cm}
	\begin{enumerate} \itemsep -2pt
	\item The Graduate Fellowship Award: \url{http://www.hertzfoundation.org/dx/Fellowships/award.aspx}
	\item Thesis Prize: \url{http://www.hertzfoundation.org/dx/Fellowships/thesis_winners.aspx}
	\end{enumerate}
\item Krell Institute, Inc.: \vspace{-0.3cm}
	\begin{enumerate} \itemsep -2pt
	\item DOE Computational Science Graduate Fellowship: \url{http://www.krellinst.org/csgf/index.shtml}
	\end{enumerate}
\item The Winston Churchill Foundation of the United States: \vspace{-0.3cm}
	\begin{enumerate} \itemsep -2pt
	\item The Churchill Scholarship: \url{http://winstonchurchillfoundation.org/index.php?hide=1&section=eligibility}
	\end{enumerate}
\item American Society for Engineering Education: \vspace{-0.3cm}
	\begin{enumerate} \itemsep -2pt
	\item \url{http://blogs.asee.org/fellowships/}
	\item Fellowship programs: \url{http://www.asee.org/fellowship-programs}
	\item Awards: \url{http://www.asee.org/member-resources/awards/full-list-of-awards}
	\item DuPont Minorities in Engineering Award: \vspace{-0.2cm}
		\begin{enumerate} \itemsep -2pt
		\item \url{http://www.asee.org/member-resources/awards/full-list-of-awards/national-awards/special#DuPont_Minorities_in_Engineering_Award}
		\item {\bf \color{blue} ``The DuPont Minorities in Engineering Award is conferred for outstanding achievements by an engineering or engineering technology educator in increasing student diversity within engineering and engineering technology programs.''}
		\end{enumerate}
	\end{enumerate}
\item Alexander von Humboldt-Stiftung/Foundation: \vspace{-0.3cm}
	\begin{enumerate} \itemsep -2pt
	\item Feodor Lynen Research Fellowship for Postdoctoral Researchers (junior postdocs): \url{http://www.humboldt-foundation.de/web/feodor-lynen-fellowship-postdoc.html}
	\item Friedrich Wilhelm Bessel Research Award (mid-career researchers): \url{http://www.humboldt-foundation.de/web/bessel-award.html}
	\item Georg Forster Research Fellowship for Postdoctoral Researchers (for non-German junior postdocs ``with above average qualifications''): \url{http://www.humboldt-foundation.de/web/georg-forster-fellowship-postdoc.html}
	\item Humboldt Research Fellowship for Postdoctoral Researchers (junior postdocs): \url{http://www.humboldt-foundation.de/web/771.html}
	\item Sofja Kovalevskaja Award (junior postdocs): \url{http://www.humboldt-foundation.de/web/kovalevskaja-award.html}
	\item Fraunhofer-Bessel Research Award: \url{http://www.humboldt-foundation.de/web/fraunhofer-bessel-award.html}
	\item \url{http://www.humboldt-foundation.de/web/home.html}
	\end{enumerate}
\item Santa Fe Institute: Omidyar Postdoctoral Fellowship; see \url{http://www.santafe.edu/education/fellowships/omidyar-postdoctoral/}
\item Applied Materials: Applied Materials Graduate Fellowship
\item American Society of Naval Engineers (ASNE): \vspace{-0.3cm}
	\begin{enumerate} \itemsep -2pt
	\item (Undergraduate and Graduate) Scholarships: \url{http://www.navalengineers.org/awards/scholarships/Pages/ASNELandingPage.aspx}
	\end{enumerate}
\item Lindau Meeting of Nobel Laureates and Students in Lindau (Oak Ridge Associated Universities, ORAU): \vspace{-0.3cm}
	\begin{enumerate} \itemsep -2pt
	\item Graduate Student Award program: \vspace{-0.2cm}
		\begin{enumerate} \itemsep -2pt
		\item \url{http://www.orau.org/lindau/}
		\item A student nominated to participate in this program must: \vspace{-0.1cm}
			\begin{enumerate} \itemsep -1pt
			\item Be a U.S. citizen
			\item Be currently enrolled as a full-time graduate student
			\item Be currently sponsored by, or working on, and supported by projects sponsored by, the agency to which the nomination is made, such as the U.S. Department of Energy Office of Science, the National Institutes of Health or other federal agency
			\item Have completed by June 2011 two years (but not more than four years) of study toward a doctoral degree in medicine or physiology, or in a related discipline, including the basic biomedical (or life) sciences
			\end{enumerate}
		\end{enumerate}
	\end{enumerate}
\item Research Councils UK (RCUK): \vspace{-0.3cm}
	\begin{enumerate} \itemsep -2pt
	\item RCUK Academic Fellowships: \vspace{-0.2cm}
		\begin{enumerate} \itemsep -2pt
		\item \url{http://www.rcuk.ac.uk/ResearchCareers/fellowships/Pages/home.aspx}
		\item \url{http://www.rcuk.ac.uk/ResearchCareers/fellowships/Pages/about.aspx}
		\item Dorothy Hodgkin Postgraduate Awards: \vspace{-0.1cm}
			\begin{enumerate} \itemsep -1pt
			\item \url{http://www.rcuk.ac.uk/ResearchCareers/dhpa/Pages/home.aspx}
			\item ``Dorothy Hodgkin Postgraduate Awards (DHPA) is a UK scheme to bring outstanding students from India, China, Hong Kong, South Africa, Brazil, Russia and the developing world to come and study for PhDs in top rated UK research facilities.''
			\end{enumerate}
		\end{enumerate}
	\item International Funding Opportunities: \vspace{-0.2cm}
		\begin{enumerate} \itemsep -2pt
		\item \url{http://www.rcuk.ac.uk/international/funding/FundingOpps/Pages/home.aspx}
		\item Early Career Researchers: \url{http://www.rcuk.ac.uk/international/funding/FundingOpps/Pages/EarlyCareer.aspx}
		\end{enumerate}
	\item Engineering and Physical Sciences Research Council: \vspace{-0.2cm}
		\begin{enumerate} \itemsep -2pt
		\item Programs: \vspace{-0.1cm}
			\begin{enumerate} \itemsep -1pt
			\item Physical sciences: \vspace{-0.1cm}
				\begin{itemize} \itemsep -1pt
				\item Organic synthetic chemistry studentships: \url{http://www.epsrc.ac.uk/about/progs/physsci/Pages/organicstudentships.aspx}
				\item Analytical science studentships: \url{http://www.epsrc.ac.uk/about/progs/physsci/Pages/analyticalstudentships.aspx}
				\end{itemize}
			\item Mathematical sciences: \vspace{-0.1cm}
				\begin{itemize} \itemsep -1pt
				\item Fellowships (for postdoctoral research): \url{http://www.epsrc.ac.uk/about/progs/maths/Pages/fellowships.aspx}
				\end{itemize}
			\end{enumerate}
		\item Funding: \vspace{-0.1cm}
			\begin{enumerate} \itemsep -1pt
			\item \url{http://www.epsrc.ac.uk/funding/Pages/default.aspx}
			\item Grants available [has funds for (new/junior) professors and to support international collaboration]: \url{http://www.epsrc.ac.uk/funding/grants/Pages/default.aspx}
			\item Calls for proposals (open/current funding calls for applications and future/proposed calls): \url{http://www.epsrc.ac.uk/funding/calls/Pages/default.aspx}
			\item Studentships (training grants for Ph.D. and Masters students, including international students): \url{http://www.epsrc.ac.uk/funding/students/Pages/default.aspx}
			\item Fellowships (from junior scientists and engineers engaged in postdoctoral research to senior researchers): \url{http://www.epsrc.ac.uk/funding/fellows/Pages/default.aspx}
			\end{enumerate}
		\end{enumerate}
	\item Biotechnology and Biological Sciences Research Council (BBSRC): \vspace{-0.2cm}
		\begin{enumerate} \itemsep -2pt
		\item ``The UK's leading funding agency for academic research and training in the non-clinical life sciences''
		\item Funding research: \vspace{-0.1cm}
			\begin{enumerate} \itemsep -1pt
			\item \url{http://www.bbsrc.ac.uk/funding/funding-index.aspx}
			\item Fellowships (for early career scientists, for supporting individuals seeking a change in research directions or scientists who are returning to research, and senior researchers): \url{http://www.bbsrc.ac.uk/funding/fellowships/fellowships-index.aspx}
			\item Studentships (Doctoral training grants, Masters training grants, postgraduate awards, and undergraduate research grants): \url{http://www.bbsrc.ac.uk/funding/studentships/studentships-index.aspx}
			\item Special opportunities (current calls for funding): \url{http://www.bbsrc.ac.uk/funding/opportunities/opportunities-index.aspx}
			\item Apply for funding (information about the process of applying for research funds): \url{http://www.bbsrc.ac.uk/funding/apply/apply-index.aspx}
			\end{enumerate}
		\end{enumerate}
	\item Science and Technology Facilities Council: \vspace{-0.2cm}
		\begin{enumerate} \itemsep -2pt
		\item STFC Grants and Awards: \vspace{-0.1cm}
			\begin{enumerate} \itemsep -1pt
			\item \url{http://www.stfc.ac.uk/Funding+and+Grants/501.aspx}
			\item ``The Science and Technology Facilities Council offers grants and support in Particle Physics, Astronomy, Nuclear Physics and Facility Development. It also provides support for research infrastructure, training, knowledge exchange and public engagement activities through a variety of funding schemes and activities.''
			\item STFC Funding Opportunities: \url{http://www.stfc.ac.uk/Funding%20and%20Grants/598.aspx}
			\item Postgraduate Studentships: \url{http://www.stfc.ac.uk/Funding+and+Grants/637.aspx} or \url{http://www.stfc.ac.uk/Funding%20and%20Grants/636.aspx}
			\end{enumerate}
		\item Fellowship opportunities: \vspace{-0.1cm}
			\begin{enumerate} \itemsep -1pt
			\item \url{http://www.stfc.ac.uk/Funding%20and%20Grants/508.aspx}
			\item ``Fellowship opportunities in Astronomy, Solar and Planetary Science, Particle Physics, Particle Astrophysics, Nuclear Physics, Development of STFC Neutron, Laser and Synchrotron Facilities within the UK.''
			\item There are postdoctoral and advanced research fellowships.
			\end{enumerate}
		\item Innovations Partnership Schemes (IPS and mini-IPS): \url{http://www.stfc.ac.uk/19213.aspx}
		\item IPS Fellowships: \vspace{-0.1cm}
			\begin{enumerate} \itemsep -1pt
			\item \url{http://www.stfc.ac.uk/19226.aspx}
			\item The IPS fellowship is a scheme designed to support a role to develop the commercial exploitation of technologies. This is not a research orientated fellowship.
			\end{enumerate}
		\item Follow-on-Funding: \vspace{-0.1cm}
			\begin{enumerate} \itemsep -1pt
			\item \url{http://www.stfc.ac.uk/19207.aspx}
			\item ``Follow on Funding is intended to provide financial support at the very early or pre-seed stage of turning research outputs into a commercial proposition. Unlike the other research councils, in STFC, industry partners are not allowed. If you have an industry partner, please use the mini-IPS or IPS scheme.''
			\item ``STFC staff, grant funded academics and researchers at CERN and ESO are eligible to apply for follow-on-funds (see the research grants handbook for CERN and ESO eligibility). STFC staff should first investigate whether they can be funded through proof of concept funding.''
			\end{enumerate}
		\end{enumerate}
	\item Natural Environment Research Council: \vspace{-0.2cm}
		\begin{enumerate} \itemsep -2pt
		\item Grants and studentships on the web: \vspace{-0.1cm}
			\begin{enumerate} \itemsep -1pt
			\item \url{http://www.nerc.ac.uk/research/gotw.asp}
			\item Grants on the web: \url{http://gotw.nerc.ac.uk/goti.asp?c=1}
			\end{enumerate}
		\item Funding: \vspace{-0.1cm}
			\begin{enumerate} \itemsep -1pt
			\item \url{http://www.nerc.ac.uk/funding/}
			\item Postgraduate training: \vspace{-0.1cm}
				\begin{itemize} \itemsep -1pt
				\item Postgraduate eligibility (requires UK/EU citizenship): \url{http://www.nerc.ac.uk/funding/available/postgrad/eligibility.asp}
				\end{itemize}
			\item Research Fellowship Scheme [for all nationalities]: \url{http://www.nerc.ac.uk/funding/available/fellowships/}
			\item Research Experience Placements (REP) scheme [for undergraduates]: \url{http://www.nerc.ac.uk/funding/available/rep.asp}
			\item Research Grants: \vspace{-0.1cm}
				\begin{itemize} \itemsep -1pt
				\item Eligibility: \url{http://www.nerc.ac.uk/funding/available/researchgrants/eligibility.asp}
				\end{itemize}
			\end{enumerate}
		\item {\bf Other potential sources of funding}: \vspace{-0.1cm}
			\begin{enumerate} \itemsep -1pt
			\item \url{http://www.nerc.ac.uk/funding/otherfunding.asp}
			\item Look at the ``Higher Education Funding Councils'' for each country (England, Wales, Northern Ireland, and Scotland)
			\end{enumerate}
		\end{enumerate}
	\end{enumerate}
\item Nuffield Foundation: \vspace{-0.3cm}
	\begin{enumerate} \itemsep -2pt
	\item Undergraduate research bursaries in science: \url{http://www.nuffieldfoundation.org/undergraduate-research-bursaries-0}
	\item Funding for social policy projects in the UK: \vspace{-0.2cm}
		\begin{enumerate} \itemsep -2pt
		\item \url{http://www.nuffieldfoundation.org/social-policy}
		\item \url{http://www.nuffieldfoundation.org/children-and-families-law-society-education-and-open-door}
		\end{enumerate}
	\item Apply for funding: \url{http://www.nuffieldfoundation.org/apply-for-funding}
	\item Africa program: \url{http://www.nuffieldfoundation.org/africa-programme-0}
	\item Nuffield Farming Scholarships Trust: \vspace{-0.2cm}
		\begin{enumerate} \itemsep -2pt
		\item Nuffield Farming Scholarships: \url{http://www.nuffieldscholar.org/}
		\end{enumerate}
	\item The Nuffield Trust (or, The Nuffield Trust for Research and Policy Studies in Health Services): \vspace{-0.2cm}
		\begin{enumerate} \itemsep -2pt
		\item Fellowships: \vspace{-0.1cm}
			\begin{enumerate} \itemsep -1pt
			\item \url{http://www.nuffieldtrust.org.uk/fellowships/index.aspx?id=43}
			\item Rock Carling fellowship (for senior researchers in public health): \url{http://www.nuffieldtrust.org.uk/fellowships/index.aspx?id=112}
			\item John Fry Fellowship (for senior researchers in public health): \url{http://www.nuffieldtrust.org.uk/fellowships/index.aspx?id=109}
			\item Harkness Fellowships in Health Care Policy: \vspace{-0.1cm}
				\begin{itemize} \itemsep -1pt
				\item ``Since September 2009 The Nuffield Trust have been the proud co-sponsors of the prestigious Harkness Fellowships programme with The Commonwealth Fund.''
				\item ``These offer an unparalleled opportunity for the health policy analysts of the future to conduct original research and learn about healthcare in North America.''
				\item ``Mid-career health policy researchers and practitioners (including doctors, health services managers, journalists and government officials) are supported to spend 9 to 12 months in the United States conducting a policy-oriented research project and working with leading U.S. health policy experts.''
				\end{itemize}
			\end{enumerate}
		\end{enumerate}
	\end{enumerate}
\item U.S. Department of Homeland Security (DHS): \vspace{-0.3cm}
	\begin{enumerate} \itemsep -2pt
	\item DHS Scholarship and Fellowship Program: \url{http://www.orau.gov/dhsed/}
	\end{enumerate}
\item ACT, Inc.: \vspace{-0.3cm}
	\begin{enumerate} \itemsep -2pt
	\item Barry M. Goldwater Scholarship and Excellence in Education Program (for US residents who will be college upperclassmen in STEM fields in the following academic year): \url{http://www.act.org/goldwater/}
	\end{enumerate}
\item Massachusetts Institute of Technology: \vspace{-0.3cm}
	\begin{enumerate} \itemsep -2pt
	\item MIT School of Engineering: \vspace{-0.2cm}
		\begin{enumerate} \itemsep -2pt
		\item Lemelson-MIT Program: \vspace{-0.1cm}
			\begin{enumerate} \itemsep -1pt
			\item \url{http://web.mit.edu/invent/}
			\item Lemelson-MIT Awards for Invention and Innovation: \url{http://web.mit.edu/invent/a-main.html}
			\end{enumerate}
		\end{enumerate}
	\end{enumerate}
\item --- --- --- --- --- --- --- --- --- --- --- --- --- --- --- --- --- --- --- --- --- --- --- --- --- --- --- --- --- --- ---
\item \colorbox{blue}{\bf Scholarships and Fellowships in Various Fields (Including Creative Arts, Teaching, and Sports)}
% Scholarships and Fellowships in Various Fields (Including Creative Arts, Teaching, and Sports)
\item U.S. Department of Education: \vspace{-0.3cm}
	\begin{enumerate} \itemsep -2pt
	\item Robert C. Byrd Honors Scholarship Program: \vspace{-0.2cm}
		\begin{enumerate} \itemsep -2pt
		\item High school graduates who have been accepted for enrollment at institutions of higher education (IHEs), have demonstrated outstanding academic achievement, and show promise of continued academic excellence may apply to states in which they are residents.
		\item \url{http://www2.ed.gov/programs/iduesbyrd/index.html}
		\end{enumerate}
	\item \colorbox{yellow}{\bf Jacob K. Javits Fellowships Program}: \vspace{-0.1cm}
		\begin{enumerate} \itemsep -1pt
		\item This program provides fellowships to students of superior academic ability -- selected on the basis of demonstrated achievement, financial need, and exceptional promise -- to undertake study at the doctoral and Master of Fine Arts level in selected fields of arts, humanities, and social sciences.
		\item \url{http://www2.ed.gov/programs/jacobjavits/index.html}
		\end{enumerate}
	\item Close Up Fellowship Program: \vspace{-0.2cm}
		\begin{enumerate} \itemsep -2pt
		\item This program provides financial aid to enable low-income students, their teachers, and recent immigrants to come to Washington, D.C., to study the operations of the three branches of the federal government.
		\item \url{http://www2.ed.gov/programs/closeup/index.html}
		\end{enumerate}
	\item {\bf \color{blue} B.J. Stupak Olympic Scholarships}: \vspace{-0.2cm}
		\begin{enumerate} \itemsep -2pt
		\item This program provides financial assistance to athletes who are training at the U.S. Olympic Education Center or one of the U.S. Olympic training centers and who are pursuing a postsecondary education at institutions of higher education (IHEs).
		\item \url{http://www2.ed.gov/programs/olympic/index.html}
		\end{enumerate}
	\item {\bf \color{blue} Teacher Education Assistance for College and Higher Education (TEACH) Grant Program}: \vspace{-0.2cm}
		\begin{enumerate} \itemsep -2pt
		\item Through the College Cost Reduction and Access Act of 2007, Congress created the Teacher Education Assistance for College and Higher Education (TEACH) Grant Program that provides grants of up to \$4,000 per year to students who intend to teach in a public or private elementary or secondary school that serves students from low-income families.
		\item \url{http://studentaid.ed.gov/PORTALSWebApp/students/english/TEACH.jsp}
		\end{enumerate}
	\item Scholarship search engine: \url{https://studentaid2.ed.gov/getmoney/scholarship/}
	\item Financial Aid: \vspace{-0.2cm}
		\begin{enumerate} \itemsep -2pt
		\item \url{http://www2.ed.gov/finaid/landing.jhtml?src=rt}
		\item \url{http://studentaid.ed.gov/PORTALSWebApp/students/english/funding.jsp}
		\item Paying for college: \url{http://www.college.gov}
		\item Student Aid (has information for students at all levels and parents): \url{http://studentaid.ed.gov/}
		\item Student Aid Eligibility: \url{http://studentaid.ed.gov/PORTALSWebApp/students/english/aideligibility.jsp?tab=funding}
		\item Federal Student Aid: \url{http://federalstudentaid.ed.gov/}
		\item Academic Competitiveness Grant: The Academic Competitiveness Grant provides up to \$750 for the first year of undergraduate study and up to \$1,300 for the second year of undergraduate study. See \url{http://studentaid.ed.gov/PORTALSWebApp/students/english/NewPrograms.jsp}.
		\end{enumerate}
	\item Free Application for Federal Student Aid (FAFSA): \vspace{-0.2cm}
		\begin{enumerate} \itemsep -2pt
		\item Financial Aid Estimator Tool (FAFSA4caster): \url{http://www.fafsa4caster.ed.gov/F4CApp/index/index.jsf}
		\item \url{http://www.fafsa.ed.gov/}
		\end{enumerate}
	\item Federal Pell Grant Program: \url{http://www2.ed.gov/programs/fpg/index.html}
	\end{enumerate}
\item European Commission: \vspace{-0.3cm}
	\begin{enumerate} \itemsep -2pt
	\item Erasmus Programme (for Europeans): \url{http://ec.europa.eu/education/lifelong-learning-programme/doc80_en.htm}
	\item Erasmus Mundus (for non-Europeans): \url{http://ec.europa.eu/education/external-relation-programmes/doc72_en.htm}
	\end{enumerate}
\item Woodrow Wilson Foundation: \vspace{-0.3cm}
	\begin{enumerate} \itemsep -2pt
	\item {\bf \color{blue} The Woodrow Wilson-Rockefeller Brothers Fund Fellowships for Aspiring Teachers of Color (for underrepresented minorities seeking a career as a K-12 public school teacher in the US)}: \url{http://www.woodrow.org/teaching-fellowships/wwrbf/index.php}
	\item {\bf \color{blue} Woodrow Wilson Teaching Fellowship (for a MS program in teacher education, who would teach at high-need urban and rural schools or $\ge$ 3 years)}: \url{http://www.wwteachingfellowship.org/}
	\item {\bf \color{blue} Leonore Annenberg Teaching Fellowship (for a MS program in teacher education, who would teach at high-need urban and rural schools or $\ge$ 3 years)}: \url{http://www.woodrow.org/teaching-fellowships/annenberg/index.php}
	\item MMUF Travel \& Research Grants (for graduate students who participated in the Mellon Mays Undergraduate Fellowship Program): \url{http://www.woodrow.org/higher-education-fellowships/opportunity/research/index.php}
	\item MMUF Dissertation Grants (for graduate students who participated in the Mellon Mays Undergraduate Fellowship Program): \url{http://www.woodrow.org/higher-education-fellowships/opportunity/dissertation/index.php}
	\item Charlotte W. Newcombe Doctoral Dissertation Fellowship (for Ph.D. students writing their theses on ethical or religious values in all fields of the humanities and social sciences): \url{http://www.woodrow.org/higher-education-fellowships/religion_ethics/index.php}
	\item {\bf \color{blue} Woodrow Wilson Dissertation Fellowship in Women�s Studies}: \url{http://www.woodrow.org/higher-education-fellowships/women_gender/index.php}
	\item Doris Duke Conservation Fellowship program (Masters students seeking careers as practicing conservationists): \url{http://www.woodrow.org/higher-education-fellowships/conservation/index.php}
	\item Thomas R. Pickering Graduate Foreign Affairs Fellowship: \vspace{-0.2cm}
		\begin{enumerate} \itemsep -2pt
		\item Prior to joining the United States Department of State Foreign Service, this fellowship supports students entering a Masters program in the following fields: \vspace{-0.1cm}
			\begin{enumerate} \itemsep -1pt
			\item {\bf public policy}
			\item international affairs
			\item public administration
			\item academic fields such as: \vspace{-0.1cm}
				\begin{itemize} \itemsep -1pt
				\item business
				\item economics
				\item political science
				\item sociology
				\item foreign languages
				\end{itemize}
			\end{enumerate}
		\item \url{http://www.woodrow.org/higher-education-fellowships/foreign_affairs/pickering_grad/index.php}
		\end{enumerate}
	\item Thomas R. Pickering Undergraduate Foreign Affairs Fellowship (for undergraduates seeking to join the United States Department of State Foreign Service): \url{http://www.woodrow.org/higher-education-fellowships/foreign_affairs/pickering_undergrad/index.php}
	\end{enumerate}
\item Burroughs Wellcome Fund: \vspace{-0.3cm}
	\begin{enumerate} \itemsep -2pt
	\item Career Awards for Medical Scientists (post-Ph.D.): \url{http://www.bwfund.org/pages/188/Career-Awards-for-Medical-Scientists/}
	\item {\bf \color{blue} Career Award for Science and Mathematics Teachers (science or mathematics K-12 teachers in North Carolina public schools)}: \url{http://www.bwfund.org/pages/379/Career-Awards-for-Science-and-Mathematics-Teachers/}
	\end{enumerate}
\item Susan G. Komen for the Cure\textregistered: The Komen College Scholarship Program, \url{http://ww5.komen.org/ResearchGrants/CollegeScholarshipAward.html}
\item University of Kansas Madison \& Lila Self Graduate Fellowship (Ph.D. fellowships for business, economics, and STEM): \url{http://www2.ku.edu/~selfpro/}
\item Nationally Coveted College Scholarships, Graduate School Fellowships \& Postdoctoral Awards: \url{http://scholarships.fatomei.com/}
\item The Andrew W. Mellon Foundation: \vspace{-0.3cm}
	\begin{enumerate} \itemsep -2pt
	\item Fellowships \& Scholarships for undergraduates: \url{http://www.mmuf.org/undergraduates/explore-your-opportunities/fellowships-scholorships}
	\end{enumerate}
\item Siebel Scholars Foundation: \vspace{-0.3cm}
	\begin{enumerate} \itemsep -2pt
	\item For students in selected business, bioengineering, and computer science graduate programs
	\item Only available for students at selected universities.
	\item \url{http://www.siebelscholars.com/scholars}
	\item \url{http://www.siebelscholars.com/}
	\end{enumerate}
\item Aspen Institute (for leaders, e.g. in business, education, community service, and politics): \vspace{-0.3cm}
	\begin{enumerate} \itemsep -2pt
	\item Catto Fellowship Program: \url{http://www.aspeninstitute.org/leadership-programs/catto-fellowship-program}
	\item Rodel Fellowship Program: \url{http://www.aspeninstitute.org/leadership-programs/aspen-institute-rodel-fellowships-public-le-/about-rodel-fellowship-program}
	\item Henry Crown Fellowship Program: \url{http://www.aspeninstitute.org/leadership-programs/henry-crown-fellowship-program}
	\end{enumerate}
\item Smithsonian Institution: \vspace{-0.3cm}
	\begin{enumerate} \itemsep -2pt
	\item Postdoctoral Fellowships, Predoctoral Fellowships, and Graduate Student Fellowships: \vspace{-0.2cm}
		\begin{enumerate} \itemsep -2pt
		\item \url{http://www.si.edu/ofg/infotoapply.htm}
		\item \url{http://www.si.edu/ofg/fell.htm}
		\item \url{http://www.si.edu/ofg/ofgapp.htm}
		\item fields of research and study: \vspace{-0.1cm}
			\begin{enumerate} \itemsep -1pt
			\item {\bf \color{blue} American History, American Material and Folk Culture, and the History of Music and Musical Instruments}
			\item History of Science and Technology
			\item {\bf \color{blue} History of Art, Design, Crafts, and the Decorative Arts}
			\item Anthropology, Archaeology, Linguistics, and Ethnic Studies
			\item Evolutionary, Systematic, Behavioral, Environmental, and Conservation Biology
			\item Earth, Mineral, and Planetary Science
			\item Materials Characterization and Conservation
			\end{enumerate}
		\end{enumerate}
	\item Internship opportunities: \url{http://www.si.edu/ofg/internopp.htm}
	\item Research centers: \url{http://www.si.edu/research/}. [ It also has lots of information for K-12 teachers. It has resources, funding, and internship opportunities for undergraduates and graduate students pursing research in various aspects of humanities, social science, and natural science. ]
	\item Freer Gallery of Art / Arthur M. Sackler Gallery: \vspace{-0.2cm}
		\begin{enumerate} \itemsep -2pt
		\item Fellowships: \url{http://www.asia.si.edu/research/fellowships.asp}
		\end{enumerate}
	\item National Museum of American History: \vspace{-0.2cm}
		\begin{enumerate} \itemsep -2pt
		\item Jerome and Dorothy Lemelson Center for the Study of Invention and Innovation: \vspace{-0.1cm}
			\begin{enumerate} \itemsep -1pt
			\item The Lemelson Center Fellows Program (for Ph.D. students and postdocs): \url{http://invention.smithsonian.org/resources/research_fellowships.aspx}
			\end{enumerate}
		\end{enumerate}
	\end{enumerate}
\item Intercollegiate Studies Institute (ISI): \vspace{-0.3cm}
	\begin{enumerate} \itemsep -2pt
	\item William E. Simon Fellowship for Noble Purpose (for American undergraduates who are planning to use the fellowship grant for serving humanity -- in their own ways): \url{http://www.isi.org/programs/fellowships/simon.html}
	\item {\bf \color{blue} Richard M. Weaver Fellowship (for Americans who are attending a graduate program and are intending to pursue a career in academia/teaching)}: \url{http://www.isi.org/programs/fellowships/richard_weaver.html}
	\item Western Civilization Fellowships (for Americans who are attending a graduate program about Western culture/civilization): \url{http://www.isi.org/programs/fellowships/western_civilization.html}
	\item Salvatori Fellowship (for Americans who are attending a graduate program about early American history): \url{http://www.isi.org/programs/fellowships/salvatori.html}
	\item Bache Renshaw Fellowship for Doctoral Study in Education (for Americans who plan to attend doctoral programs in education): \url{http://www.isi.org/programs/fellowships/bache_renshaw.html}
	\item \url{http://www.isi.org/programs/fellowships/fellowships.html}
	\end{enumerate}
\item Le Fonds qu{\'{e}}b{\'{e}}cois de la recherche sur la nature et les technologies (The Quebec Research Fund on nature and technology): \vspace{-0.3cm}
	\begin{enumerate} \itemsep -2pt
	\item Scholarships: \url{http://www.fqrnt.gouv.qc.ca/en/bourses/index.htm}
	\end{enumerate}
\item Horatio Alger Association of Distinguished Americans, Inc.: \vspace{-0.3cm}
	\begin{enumerate} \itemsep -2pt
	\item Scholarship Programs (for US high school seniors who have faced and overcome great obstacles in their young lives): \url{https://www.horatioalger.org/scholarships/sp.cfm}
	\item Awards: \vspace{-0.2cm}
		\begin{enumerate} \itemsep -2pt
		\item \url{http://www.horatioalger.org/aboutus.cfm}
		\item Horatio Alger Award: ``dedicated community leaders who demonstrate individual initiative and a commitment to excellence; as exemplified by remarkable achievements accomplished through honesty, hard work, self-reliance and perseverance over adversity''
		\item International Horatio Alger Award: ``recipients of this award must have overcome humble beginnings and/or adversity to achieve success. They serve as outstanding role models to the international community and are committed to the Association's mission of encouraging and educating today's young people.''
		\item Norman Vincent Peale Award: ``a Member who has made exceptional humanitarian contributions to society, who has been an active participant in the Association, and who continues to exhibit courage, tenacity and integrity in the face of great challenges. ''
		\end{enumerate}
	\end{enumerate}
\item The W. Garfield Weston Foundation: \vspace{-0.3cm}
	\begin{enumerate} \itemsep -2pt
	\item Entrance Awards \& Upper Year Garfield Weston Awards (for students pursuing college or CEGEP studies in Canada): \url{http://www.garfieldwestonawards.ca/en/about}
	\end{enumerate}
\item Canadian Merit Scholarship Foundation (\url{http://www.cmsf.ca/}): Loran Award (undergraduate funding for Canadian citizens and permanent residents), \url{http://www.loranaward.ca/}
\item StartingBloc: \vspace{-0.3cm}
	\begin{enumerate} \itemsep -2pt
	\item StartingBloc Fellowship: \vspace{-0.2cm}
		\begin{enumerate} \itemsep -2pt
		\item \url{http://www.startingbloc.org/fellowship}
		\item For people who believe that economic value creation and social value creation are complementary... For people who believe in making money and doing good, and creating social and economic impact... 
		\item The Institute for Social Innovation is a ``conference'' to learn about global issues, ``corporate social responsibility, social entrepreneurship, cross sector partnerships and sustainability. Sessions are led by top academics, corporate innovators, social entrepreneurs, activists and government officials.'' 
		\end{enumerate}
	\end{enumerate}
\item The John D. and Catherine T. MacArthur Foundation: \vspace{-0.3cm}
	\begin{enumerate} \itemsep -2pt
	\item Applying for Grants: \url{http://www.macfound.org/site/c.lkLXJ8MQKrH/b.913959/k.E1BE/Applying_for_Grants.htm}
	\item Financial \& Grant Information: \url{http://www.macfound.org/site/c.lkLXJ8MQKrH/b.938093/k.9E4C/Financial__Grant_Information.htm}
	\item MacArthur Fellows Program: \url{http://www.macfound.org/site/c.lkLXJ8MQKrH/b.959463/k.9D7D/Fellows_Program.htm}
	\end{enumerate}
\item Wenner-Gren Foundations (The Wenner-Gren Center Foundation for Scientific Research, The Axel Wenner-Gren Foundation for International Exchange of Scientists and The Foundation Wenner-Grenska Samfundet): Fellowships (for Swedish postdocs), \url{http://www.swgc.org/stipendier.aspx}
\item {\'{E}}gide: \vspace{-0.3cm}
	\begin{enumerate} \itemsep -2pt
	\item EGIDE Latitudes: \url{http://www.egidelatitudes.fr/jahia/Jahia/site/egidelatitudes}
	\item Call for applications to scholarship opportunities (including a scholarship for French citizens to study abroad): \url{http://www.egide.asso.fr/jahia/Jahia/accueil/appels}
	\item Eiffel excellence scholarship programme (organized by the French Ministry of Foreign and European Affairs): \vspace{-0.2cm}
		\begin{enumerate} \itemsep -2pt
		\item \url{http://www.egide.asso.fr/jahia/Jahia/appels/eiffel}
		\item For non-French citizens pursuing advanced degrees.
		\end{enumerate}
	\end{enumerate}
\item Gottlieb Daimler and Karl Benz Foundation: \vspace{-0.3cm}
	\begin{enumerate} \itemsep -2pt
	\item {\bf \color{blue} Ph.D. fellowship for international students to study in Germany}; see \url{http://www.daimler-benz-stiftung.de/home/fellowship/en/start.html}
	\end{enumerate}
\item The San Diego Foundation: \vspace{-0.3cm}
	\begin{enumerate} \itemsep -2pt
	\item San Diego Foundation Community Scholarship Program: \vspace{-0.2cm}
		\begin{enumerate} \itemsep -2pt
		\item \url{http://www.sdfoundation.org/GrantsScholarships/Scholarships.aspx}
		\item Available scholarships: \url{http://www.sdfoundation.org/GrantsScholarships/Scholarships/ForStudents/AvailableScholarships.aspx}. Also, see \url{http://www.sdfoundation.org/GrantsScholarships/Scholarships/ForStudents/AvailableScholarships/CommonApplicationScholarships.aspx#twomey}
		\item It has scholarships for: \vspace{-0.1cm}
			\begin{enumerate} \itemsep -1pt
			\item graduating high school seniors
			\item current undergraduates
			\item non-traditional college students: \vspace{-0.1cm}
				\begin{itemize} \itemsep -1pt
				\item mature-age students
				\item mature student
				\item adult learner
				\item adult student
				\item adults who are returning to college
				\end{itemize}
			\item people pursuing teaching certificates
			\item students attending grad school
			\item students attending trade/vocational school
			\item foster youth
			\item students in various ethnic groups
			\item students in different geographical locations
			\item {\bf \color{blue} students pursuing education in certain fields, such as engineering, nursing, music, and arts and humanities}
			\end{enumerate}
		\item Separate Scholarships: \url{http://www.sdfoundation.org/GrantsScholarships/Scholarships/ForStudents/AvailableScholarships/SeparateScholarships.aspx}
		\item Other Scholarships and Financial Aid Resources: \url{http://www.sdfoundation.org/GrantsScholarships/Scholarships/ForStudents/AvailableScholarships/OtherScholarshipsandFinancialAidResources.aspx}
		\item Financial Aid Information: \url{http://www.sdfoundation.org/GrantsScholarships/Scholarships/ForStudents/Resources/FinancialAidInformation.aspx}
		\end{enumerate}
	\item Grant Opportunities (for non-profit organizations): \url{http://www.sdfoundation.org/GrantsScholarships/ForNonprofits/GrantOpportunities.aspx}
	\end{enumerate}
\item Ewing Marion Kauffman Foundation: \vspace{-0.3cm}
	\begin{enumerate} \itemsep -2pt
	\item Kauffman Dissertation Fellowship Program (for ``Ph.D., D.B.A., or other doctoral students at accredited U.S. universities to support dissertations in the area of entrepreneurship''): \url{http://www.kauffman.org/research-and-policy/kauffman-dissertation-fellowship-program.aspx}
	\item Kauffman Junior Faculty Fellowship in Entrepreneurship Research: \vspace{-0.2cm}
		\begin{enumerate} \itemsep -2pt
		\item \url{http://www.kauffman.org/research-and-policy/kauffman-junior-faculty-fellowship-in-entrepreneurship.aspx}
		\item ``to recognize tenured or tenure-track junior faculty members at accredited U.S. universities who are beginning to establish a record of scholarship and exhibit the potential to make significant contributions to the body of research in the field of entrepreneurship''
		\end{enumerate}
	\item Ewing Marion Kauffman Prize Medal for Distinguished Research in Entrepreneurship (for promising young scholars in the field of entrepreneurship): \url{http://www.kauffman.org/research-and-policy/kauffman-prize-medal-for-entrepreneurship-research.aspx}
	\item Kauffman Legal Fellowship Program (for post-J.D. research fellowship): \url{http://www.kauffman.org/research-and-policy/kauffman-legal-fellowship-program.aspx}
	\item Kauffman Global Scholars Program (for non-American top young entrepreneurs): \url{http://www.kauffman.org/entrepreneurship/kauffman-global-scholars-program.aspx}
	\item Entrepreneur Fellows program (for M.D.s and Ph.D.s who want to become high-tech start-up entrepreneurs): \url{http://www.kauffman.org/entrepreneurship/entrepreneur-fellows-program.aspx}
	\item Entrepreneur Postdoctoral Fellows program (for postdocs who want to become high-tech start-up entrepreneurs): \url{http://www.kauffman.org/entrepreneurship/entrepreneur-postdoctoral-fellows-program.aspx}
	\item Kauffman Fellows Program (``to educate and train future venture capitalists and future leaders of high-growth companies''): \url{http://www.kauffman.org/entrepreneurship/kauffman-fellows.aspx}
	\item Kauffman Foundation Outstanding Postdoctoral Entrepreneur Award: \url{http://www.kauffman.org/entrepreneurship/outstanding-postdoctoral-entrepreneur-award.aspx}
	\end{enumerate}
\item Killam Fellowships Program: \vspace{-0.3cm}
	\begin{enumerate} \itemsep -2pt
	\item \url{http://www.killamfellowships.com/}
	\item The Killam Fellowships Program allows undergraduate students from Canada and the United States to participate in a program of binational residential exchange.
	\item Killam Fellows spend either one semester or a full academic year as an exchange student in the host country.
	\end{enumerate}
\item Canada Council for the Arts: \vspace{-0.3cm}
	\begin{enumerate} \itemsep -2pt
	\item Killam Research Fellowship: \vspace{-0.2cm}
		\begin{enumerate} \itemsep -2pt
		\item \url{http://killam.canadacouncil.ca/welcome.asp}
		\item For researchers in the following fields, and interdisciplinary fields between these fields: \vspace{-0.1cm}
			\begin{enumerate} \itemsep -1pt
			\item humanities
			\item social sciences
			\item natural sciences
			\item health sciences
			\item engineering
			\end{enumerate}
		\item For outstanding researchers who are Canadian citizens or permanent residents
		\end{enumerate}
	\item Killam Prizes (and Killam Research Fellowships): \url{http://www.canadacouncil.ca/prizes/killam}
	\end{enumerate}
\item Killam Trusts: \vspace{-0.3cm}
	\begin{enumerate} \itemsep -2pt
	\item Killam Scholarship and Prize Programs (multiple fields in selected Canadian universities): \url{http://www.killamtrusts.ca/index.asp}
	\item Killam Award winners: \url{http://www.killamtrusts.ca/awardwinners.asp}
	\item Killam Scholarship and Prize Programs at various institutions (including universities): \url{http://www.killamtrusts.ca/uofAlberta.asp}
	\end{enumerate}
\item U.S. Department of State: \vspace{-0.3cm}
	\begin{enumerate} \itemsep -2pt
	\item Bureau of Educational and Cultural Affairs: \vspace{-0.2cm}
		\begin{enumerate} \itemsep -2pt
		\item Institute of International Education (administrator of program): \vspace{-0.1cm}
			\begin{enumerate} \itemsep -1pt
			\item Council for International Exchange of Scholars: \vspace{-0.1cm}
				\begin{itemize} \itemsep -1pt
				\item Fulbright Programs (for U.S. and non-U.S. Scholars): \url{http://www.cies.org/Fulbright_programs.htm}; \url{http://www.cies.org/about_fulb.htm}; \url{http://us.fulbrightonline.org/about.html}; \url{http://foreign.fulbrightonline.org/}; \url{http://exchanges.state.gov/academicexchanges/index/fulbright-program.html}; and \url{http://fulbright.state.gov/}
				\item Hubert H. Humphrey Fellowship Program: \vspace{-0.1cm}
					\begin{itemize} \itemsep -1pt
					\item For mid-career professionals in the following fields: economic development/finance and banking, agricultural and rural development, natural resources, environmental policy, and climate change, human resource management, communications/journalism, teaching of English as a foreign language, educational administration, planning, and policy, substance abuse education, treatment, and prevention, HIV/AIDS policy and prevention, public health policy and management, {\bf public policy} analysis and public administration, law and human rights, urban and regional planning, trafficking in persons - policy and prevention, technology policy and management, and higher education administration
					\item \url{http://www.humphreyfellowship.org/}
					\item \url{http://exchanges.state.gov/globalexchanges/humphrey-fellowship.html}
					\end{itemize}
				\end{itemize}
			\item International programs for scholars (search under each continent): \url{http://www.iie.org/en/Our-Global-Reach}
			\end{enumerate}
		\item International Documentary Filmmakers Fellowship: \vspace{-0.1cm}
			\begin{enumerate} \itemsep -1pt
			\item \url{http://exchanges.state.gov/cultural/docfilmmakers.html}
			\item \url{http://smpa.gwu.edu/doccenter/fellowship.php}
			\item For ``emerging or mid-career documentary filmmakers''
			\item Intensive six-week program at the Documentary Center, The George Washington University
			\end{enumerate}
		\item Office of English Language Programs: \vspace{-0.1cm}
			\begin{enumerate} \itemsep -1pt
			\item English Language Fellow Program (for ``highly qualified U.S. educators in the field of Teaching English to Speakers of Other Languages, TESOL''): \url{http://exchanges.state.gov/englishteaching/el-fellow.html}
			\item English Language Specialist Program: \vspace{-0.1cm}
				\begin{itemize} \itemsep -1pt
				\item \url{http://exchanges.state.gov/englishteaching/el-specialist.html}
				\item U.S. academics in the fields of Teaching English as a Foreign Language (TEFL) / Teaching English as a Second Language (TESL) and Applied Linguistics
				\end{itemize}
			\item E-Teacher Scholarship Program (for English teaching professionals living outside of the United States): \url{http://exchanges.state.gov/englishteaching/eteacher.html}
			\item English Access Microscholarship Program (Access): \vspace{-0.1cm}
				\begin{itemize} \itemsep -1pt
				\item \url{http://exchanges.state.gov/englishteaching/eam.html}
				\item The English Access Microscholarship Program (Access) provides a foundation of English language skills to non-elite, 14 - 18 year old students through afterschool classes and intensive summer learning activities.
				\end{itemize}
			\item \url{http://exchanges.state.gov/englishteaching/index.html}
			\end{enumerate}
		\item Office of Global Educational Programs: \vspace{-0.1cm}
			\begin{enumerate} \itemsep -1pt
			\item Community College Initiative: \vspace{-0.1cm}
				\begin{itemize} \itemsep -1pt
				\item For ``individuals from Brazil, Egypt, Ghana, Indonesia, Pakistan, South Africa, Turkey, and selected countries in Central America to spend one year studying at community colleges in the United States and earn a vocational certificate.''
				\item ``The program provides academic instruction in selected fields including agriculture, applied engineering, business management and administration, health professions, information technology, media, and tourism and hospitality management, while also immersing participants in U.S. society and cultural life.''
				\item ``Participants are recruited from historically underserved populations and may not have had opportunities for formal job training or higher education. Most participants are in their early- to mid-twenties and many already have work experience.''
				\item \url{http://exchanges.state.gov/globalexchanges/community-colleges-initiative.html}
				\end{itemize}
			\item {\bf \color{blue} Benjamin A. Gilman International Scholarship Program}: \vspace{-0.1cm}
				\begin{itemize} \itemsep -1pt
				\item ``The Benjamin A. Gilman International Scholarship Program provides scholarships to U.S. undergraduates with financial need for study abroad, including students from diverse backgrounds and students going to non-traditional study abroad destinations.''
				\item ``The applicant must be receiving a Federal Pell Grant or provide proof that he/she will be receiving a Pell Grant at the time of application or during the term of his/her study abroad.''
				\item \url{http://exchanges.state.gov/globalexchanges/gilman-scholarship-program.html}
				\end{itemize}
			\item Global Undergraduate Exchange Program (Global UGRAD Program): \vspace{-0.1cm}
				\begin{itemize} \itemsep -1pt
				\item \url{http://exchanges.state.gov/academicexchanges/guep.html}
				\item The Global Undergraduate Exchange Program (also known as the Global UGRAD Program) provides one semester and academic year scholarships to outstanding undergraduate students from underrepresented sectors in East Asia, Eurasia and Central Asia, the Near East and South Asia and the Western Hemisphere for non-degree full-time study combined with community service, internships and cultural enrichment.
				\end{itemize}
			\item Professors and Research Scholars: \url{http://exchanges.state.gov/jexchanges/programs/professor.html}
			\item Short-Term Scholar: \url{http://exchanges.state.gov/jexchanges/programs/shortterm.html}
			\item Student, College/University: \vspace{-0.1cm}
				\begin{itemize} \itemsep -1pt
				\item \url{http://exchanges.state.gov/jexchanges/programs/ucstudent.html}
				\item The College/University Student Program gives foreign students the opportunity to study at an American degree-granting post-secondary accredited educational institution, including colleges and universities. Students may participate in degree and non-degree programs. They must pursue a full-time course of study and maintain satisfactory advancement toward the completion of their academic program.
				\end{itemize}
			\item Study of the United States Institutes for Scholars: \vspace{-0.1cm}
				\begin{itemize} \itemsep -1pt
				\item Study of the United States Institutes for Scholars  are designed to strengthen curricula and improve the quality of teaching about the United States in academic institutions overseas.
				\item Foreign university faculty, secondary educators and other scholars spend approximately four weeks at host universities where they take part in a series of lectures, seminar discussions and site visits related to each institute's theme.
				\item They learn about American educational philosophies, explore new teaching methods and pursue related research interests.
				\item Interests of these institutes: \vspace{-0.1cm}
					\begin{itemize} \itemsep -1pt
					\item American Politics and Political Thought
					\item Contemporary American Literature
					\item Journalism and Media
					\item Religious Pluralism in the United States
					\item Secondary School Educators
					\item U.S. Culture and Society
					\item U.S. Foreign Policy
					\item U.S. National Security
					\end{itemize}
				\item \url{http://exchanges.state.gov/academicexchanges/scholars.html}
				\end{itemize}
			\item Study of the United States Institutes for Student Leaders: \vspace{-0.1cm}
				\begin{itemize} \itemsep -1pt
				\item Study of the United States Institutes for Student Leaders are five-to-six-week academic programs for foreign undergraduate leaders.
				\item Hosted by U.S. academic institutions throughout the United States, the Student Leader Institutes include an intensive academic component, an educational tour of other regions of the country, local community service activities and a unique opportunity for participants to get to know their American peers.
				\item \url{http://exchanges.state.gov/academicexchanges/students.html}
				\item Interests of the institutes: \vspace{-0.1cm}
					\begin{itemize} \itemsep -1pt
					\item Comparative {\bf Public Policy} for Pakistani Student Leaders
					\item Energy and the Environment
					\item Global Environmental Issues
					\item New Media
					\item Religious Pluralism in the U.S.
					\item Social Entrepreneurship
					\item U.S. Foreign Policy for East Asian Student Leaders
					\item Western Hemisphere Student Leaders 
					\item Women's Leadership
					\end{itemize}
				\end{itemize}
			\item Edmund S. Muskie Graduate Fellowship: \vspace{-0.1cm}
				\begin{itemize} \itemsep -1pt
				\item \url{http://exchanges.state.gov/academicexchanges/muskie.html}
				\item The Edmund S. Muskie Graduate Fellowship Program (Muskie) confers fellowships for Master's degree-level study in the U.S. in the fields of business administration, economics, education, environmental policy and management, international affairs, journalism/mass communications, law, library and information science, public administration, public health and {\bf public policy} for students and professionals from Eurasia.
				\item Candidates are recruited through a merit-based competition administered by the International Research \& Exchanges Board (IREX).
				\item U.S. host campuses are also selected through a competition process and generally provide tuition waivers of fifty percent.
				\item Approximately 145 fellowships are awarded each academic year.
				\end{itemize}
			\item Critical Language Scholarship Program: \vspace{-0.1cm}
				\begin{itemize} \itemsep -1pt
				\item \url{http://exchanges.state.gov/academicexchanges/sli2.html}
				\item The Critical Language Scholarship (CLS) Program provides overseas foreign language instruction and cultural enrichment experiences in 13 critical need languages for U.S. students in higher education.
				\item The CLS Program is part of a U.S. government effort to expand dramatically the number of Americans studying and mastering critical need foreign languages.
				\item Undergraduate, master's and doctoral-level students of diverse disciplines and majors are encouraged to apply for the seven-to-10-week-long programs.
				\item Participants are expected to continue their language study beyond the scholarship period, and later apply their critical language skills in their future professional careers.
				\end{itemize}
			\item Critical Language Enhancement Award (CLEA): \vspace{-0.1cm}
				\begin{itemize} \itemsep -1pt
				\item \url{http://exchanges.state.gov/academicexchanges/clea2.html}
				\item The Critical Language Enhancement Award (CLEA) provides funding to eligible Fulbright U.S. Student Program Grantees who intend to use one of the following languages for their Fulbright project: \vspace{-0.1cm}
					\begin{itemize} \itemsep -1pt
					\item Arabic (all dialiects)
					\item Azeri
					\item Bangla/Bengali
					\item Bhasa Indonesia
					\item Chinese (Mandarin Only)
					\item Farsi
					\item Gujarati
					\item Hindi
					\item Korean
					\item Marathi
					\item Pashto
					\item Punjabi
					\item Russian
					\item Turkish
					\item Urdu
					\end{itemize}
				\end{itemize}
			\end{enumerate}
		\item Office of International Visitors: \vspace{-0.1cm}
			\begin{enumerate} \itemsep -1pt
			\item International Visitor Leadership Program (IVLP): \vspace{-0.1cm}
				\begin{itemize} \itemsep -1pt
				\item \url{http://exchanges.state.gov/ivlp/index.html}
				\item \url{http://exchanges.state.gov/ivlp/ivlp.html}
				\item The Office of International Visitors manages and funds the International Visitor Leadership Program (IVLP).
				\item Launched in 1940, the IVLP is a professional exchange program that seeks to build mutual understanding between the U.S. and other nations through carefully designed short-term visits to the U.S. for current and emerging foreign leaders.
				\item These visits reflect the International Visitors' professional interests and support the foreign policy goals of the United States.
				\end{itemize}
			\end{enumerate}
		\item Program Search (find international exchange programs sponsored by the Bureau of Educational and Cultural Affairs): \url{http://exchanges.state.gov/index/search.html}
		\end{enumerate}
	\end{enumerate}
\item Mexican American Legal Defense and Educational Fund (MALDEF): \vspace{-0.3cm}
	\begin{enumerate} \itemsep -2pt
	\item Scholarship Resources: \url{http://maldef.org/leadership/scholarships/}
	\item MALDEF Law School Scholarship Program: \vspace{-0.2cm}
		\begin{enumerate} \itemsep -2pt
		\item MALDEF's Law School Scholarship Program provides several scholarships in varying amounts to deserving law students with a commitment to advancing the civil rights of Latinos.
		\item MALDEF's Law School Scholarship Program is open to all law students who will be enrolled full-time in an American-accredited law school in 2010-2011.
		\item Scholarships are awarded to students based on their commitment to serve the Latino community through law; their past achievement and potential for achievement; and their financial need.
		\item \url{http://maldef.org/leadership/scholarships/law_school_scholarship_program/index.html}
		\end{enumerate}
	\item Undergraduate Scholarship Resource Guide: \url{http://maldef.org/leadership/scholarships/resources/index.html}
	\end{enumerate}
\item Ashoka: \vspace{-0.3cm}
	\begin{enumerate} \itemsep -2pt
	\item Ashoka Fellows (to promote and support social entrepreneurship): \url{http://www.ashoka.org/fellows}
	\end{enumerate}
\item Heinz Family Foundation: \vspace{-0.3cm}
	\begin{enumerate} \itemsep -2pt
	\item Heinz Award Criteria: \vspace{-0.2cm}
		\begin{enumerate} \itemsep -2pt
		\item \url{http://heinzawards.net/awards/criteria}
		\item The Heinz Endowments
		\item Attributes and qualities of awardees: \vspace{-0.1cm}
			\begin{enumerate} \itemsep -1pt
			\item an enormous capacity to love
			\item smile
			\item take risks
			\item question
			\item work hard
			\item believe in the power of the individual to improve the lives of others
			\end{enumerate}
		\item ``Candidates [should] possess a remarkable mix of vision, optimism, creativity and hard work which, when combined, produce tangible achievements of lasting good.''
		\item Nominees must exhibit the following personal characteristics: \vspace{-0.1cm}
			\begin{enumerate} \itemsep -1pt
			\item A passion for excellence that goes beyond intellectual curiosity;
			\item A concern for humanity rooted in a deep sensitivity for the well-being of others; 
			\item A knowledge of self which acknowledges weaknesses but relies on individual strengths;
			\item A gritty determination that will see a job through to completion despite the inevitable setbacks;
			\item A broad vision which extends far beyond the particular and embraces something universal.
			\end{enumerate}
		\item Work of the candidates for a Heinz Award must meet the following criteria: \vspace{-0.1cm}
			\begin{enumerate} \itemsep -1pt
			\item Be significant and not a ``quick fix.''
			\item Have an enduring and meaningful impact.
			\item Be creative and innovative, and
			\item Be sufficiently tangible to serve as a model for replication elsewhere.
			\end{enumerate}
		\item ``In addition, candidates should be actively working in the field in which they are nominated with the hope that, in receiving this award, their potential for future societal contribution will be enhanced.''
		\end{enumerate}
	\item Categories: \vspace{-0.2cm}
		\begin{enumerate} \itemsep -2pt
		\item Arts \& Humanities
		\item Environment
		\item Human Condition
		\item {\bf Public Policy}
		\item Technology, Economy, + Employment
		\end{enumerate}
	\end{enumerate}
\item Echoing Green: \vspace{-0.3cm}
	\begin{enumerate} \itemsep -2pt
	\item Echoing Green Fellowship: \vspace{-0.2cm}
		\begin{enumerate} \itemsep -2pt
		\item \url{http://www.echoinggreen.org/fellowship}
		\item Has information on eligibility, the benefits of the fellowship, and application cycle and dates.
		\end{enumerate}
	\item Echoing Green Fellows: \url{http://www.echoinggreen.org/fellows}
	\end{enumerate}
\item Ben Franklin Technology Partners (BFTP): \vspace{-0.3cm}
	\begin{enumerate} \itemsep -2pt
	\item Innovation Works (IW): \vspace{-0.2cm}
		\begin{enumerate} \itemsep -2pt
		\item AlphaLab: \vspace{-0.1cm}
			\begin{enumerate} \itemsep -1pt
			\item ``An immersive environment where entrepreneurs can tap IW's onsite experts for business and market advice and exchange ideas with other entrepreneurs launching in similar markets''
			\end{enumerate}
		\end{enumerate}
	\end{enumerate}
\item Carnegie Corporation of New York: \vspace{-0.3cm}
	\begin{enumerate} \itemsep -2pt
	\item Carnegie Scholars Program (not available in 2010): \url{http://carnegie.org/programs/carnegie-scholars/}
	\end{enumerate}
\item New York Women's Foundation: \vspace{-0.3cm}
	\begin{enumerate} \itemsep -2pt
	\item Finch Scholar Program (with the Finch College Alumnae Association): \vspace{-0.2cm}
		\begin{enumerate} \itemsep -2pt
		\item \url{http://www.nywf.org/internship.html} and \url{http://www.finchcollege.org/}
		\item ``Our partnership with the Finch Scholar Program allows us to provide practical community service experience to an outstanding local student enrolled in college. The internship affords the Finch Scholar opportunities to work in meaningful ways in a nonprofit organization with exposure to social change philanthropy, participatory grantmaking, advocacy and {\bf public policy}. Generally, we offer one scholarship per year with a stipend.''
		\item \url{http://www.finchcollege.org/newFinchScholarPrgm.html}
		\item \url{http://www.finchcollege.org/newscholarships.html}
		\end{enumerate}
	\end{enumerate}
\item The Rockefeller Foundation: \vspace{-0.3cm}
	\begin{enumerate} \itemsep -2pt
	\item The Bellagio Center: \vspace{-0.2cm}
		\begin{enumerate} \itemsep -2pt
		\item \url{http://www.rockefellerfoundation.org/bellagio-center}
		\item Residency Programs: \vspace{-0.1cm}
			\begin{enumerate} \itemsep -1pt
			\item \url{http://www.rockefellerfoundation.org/bellagio-center/residency-programs}
			\item ``The Bellagio Residency program offers scholars, artists, thought leaders, policymakers and practitioners a serene setting conducive to focused, goal-oriented work, and the unparalleled opportunity to establish new connections with fellow residents, across a stimulating array of disciplines and geographies.  The Bellagio Center community generates new knowledge to solve some of the most complex problems facing our world and creates art that inspires reflection, understanding, and imagination.''
			\item Scholarly Residencies: \vspace{-0.1cm}
				\begin{itemize} \itemsep -1pt
				\item ``Researchers in the humanities, natural sciences, social sciences and other academic disciplines''
				\item ``The Center typically offers one-month residencies for no more than 12 scholars and scientists at a time. Individuals in any discipline and from any part of the world are welcome to apply. The Center maintains a core focus on projects consistent with the Foundation's mission to expand opportunities for poor or vulnerable people and to help see that the benefits of globalization are shared more widely. It also seeks to include beyond that core a wide variety of projects from all academic disciplines.''
				\item \url{http://www.rockefellerfoundation.org/bellagio-center/residency-programs/scholarly-residencies}
				\end{itemize}
			\item Creative Artist Residencies: \vspace{-0.1cm}
				\begin{itemize} \itemsep -1pt
				\item ``Artists, composers, writers''
				\item ``Bellagio creative artist residencies for composers, novelists, playwrights, poets, video/filmmakers and visual artists provide time for disciplined work, individual reflection, and collegial engagement, uninterrupted by the usual professional and personal demands. The Center typically offers one-month stays for no more than three to five creative artists at a time. Artists of significant achievement from any country are welcome to apply.''
				\item \url{http://www.rockefellerfoundation.org/bellagio-center/residency-programs/creative-artist-residencies}
				\end{itemize}
			\item Practitioner Residencies: \vspace{-0.1cm}
				\begin{itemize} \itemsep -1pt
				\item ``Policymakers, nonprofit leaders, journalists and public advocates''
				\item ``The Center offers residencies to professionals in fields relevant to the Rockefeller Foundation's issue areas. The Center maintains a core focus on projects consistent with our mission, to expand opportunities for poor or vulnerable people and to help see that the benefits of globalization are shared more widely.   We seek practitioner applicants with demonstrated leadership qualities and the capacity to contribute to the intellectual life at the Center.''
				\item \url{http://www.rockefellerfoundation.org/bellagio-center/residency-programs/practitioner-residencies}
				\end{itemize}
			\end{enumerate}
		\item {\bf \color{blue} Creative Arts Fellowships}: \vspace{-0.1cm}
			\begin{enumerate} \itemsep -1pt
			\item ``This high-profile program hosts visual artists at the Bellagio Center for three-month residencies that inspire creativity and promote interaction between the arts and other fields. Creative Arts Fellows, like other participants in Bellagio residency programs, have the time and space to work independently during the day. They also enjoy and benefit from a lively community of scholars, writers, policymakers and other artists who gather in the evening for dinner and occasional presentations.  The combination of private work space, an extended stay, a generous stipend and a unique group of fellow residents makes a Creative Arts Fellowship at the Bellagio Center a remarkable opportunity.''
			\item \url{http://www.rockefellerfoundation.org/bellagio-center/creative-arts-fellowships}
			\end{enumerate}
		\end{enumerate}
	\end{enumerate}
\item Wellcome Trust: \vspace{-0.3cm}
	\begin{enumerate} \itemsep -2pt
	\item Wellcome Trust Book Prize: \vspace{-0.2cm}
		\begin{enumerate} \itemsep -2pt
		\item \url{http://www.wellcomebookprize.org/About-the-prize/index.htm}
		\item ``The Wellcome Trust Book Prize celebrates the best of medicine in literature by awarding 25 000 each year for the finest fiction or non-fiction book centered around medicine.''
		\end{enumerate}
	\end{enumerate}
\item The Kennedy Memorial Trust: \vspace{-0.3cm}
	\begin{enumerate} \itemsep -2pt
	\item \url{http://www.kennedytrust.org.uk/}
	\item Kennedy Scholarship: \url{http://www.kennedytrust.org.uk/display.aspx?Id=1165&pid=0}
	\item Frank Knox Fellowships: \url{http://www.kennedytrust.org.uk/display.aspx?Id=1175&pid=0}
	\end{enumerate}
\item Foreign \& Commonwealth Office / United Kingdom: \vspace{-0.3cm}
	\begin{enumerate} \itemsep -2pt
	\item Chevening scholarships: \vspace{-0.2cm}
		\begin{enumerate} \itemsep -2pt
		\item \url{http://www.fco.gov.uk/en/about-us/what-we-do/scholarships/}
		\item ``The Chevening programme, has, over 26 years, provided more than 30,000 Scholarships at Higher Education Institutions (HEIs) in the UK for postgraduate students or researchers from countries across the world.''
		\end{enumerate}
	\item {\bf Marshall Scholarships} finance young Americans of high ability to study for a graduate degree in the United Kingdom: \url{http://www.marshallscholarship.org/}
	\end{enumerate}
\item Ministry of Education, Culture, Sports, Science and Technology (MEXT) / Japan: \vspace{-0.3cm}
	\begin{enumerate} \itemsep -2pt
	\item \url{http://www.mext.go.jp/english/}
	\item Monbukagakusho Scholarship: \vspace{-0.2cm}
		\begin{enumerate} \itemsep -2pt
		\item \url{http://en.wikipedia.org/wiki/Monbukagakusho_Scholarship}
		\item \url{http://project.monbusho.org/old/} and \url{http://www.monbusho.org/}
		\end{enumerate}
	\end{enumerate}
\item Institute of International Education (IIE): \vspace{-0.3cm}
	\begin{enumerate} \itemsep -2pt
	\item GE Foundation Scholar-Leaders Program: \vspace{-0.2cm}
		\begin{enumerate} \itemsep -2pt
		\item \url{http://www.iie.org/en/Programs/GE-Foundation-Scholar-Leaders-Program}
		\item ``The GE Foundation Scholar-Leaders Program began in 1987 in Mexico and now supports outstanding students in higher education in fourteen countries around the world. The program initially provided traditional financial support for university education, but has developed into an exciting Leadership Development Program to complement the student's academic curriculum.''
		\item Eligibility: ``Students in their first year of study in engineering, technology, business, finance, management, or economics attending a participating university. GE Foundation Scholar-Leaders qualification requirements vary by region.''
		\end{enumerate}
	\end{enumerate}
\item British Council: \vspace{-0.3cm}
	\begin{enumerate} \itemsep -2pt
	\item Shine! 2011: International Student Awards: \vspace{-0.2cm}
		\begin{enumerate} \itemsep -2pt
		\item \url{http://www.educationuk.org/shine}
		\item For international students in the United Kingdom
		\end{enumerate}
	\item Funding your studies: \vspace{-0.2cm}
		\begin{enumerate} \itemsep -2pt
		\item \url{http://www.britishcouncil.org/learning-funding-your-studies.htm}
		\item Education UK: \url{http://www.educationuk.org/pls/hot_bc/page_pls_user_advice?x=&y=&a=0&d=4460}
		\item 9/11 Scholarship Fund: \vspace{-0.1cm}
			\begin{enumerate} \itemsep -1pt
			\item \url{http://www.britishcouncil.org/911scholarships.htm}
			\item ``The 9/11 Scholarship Fund supports international students who were directly affected by the 2001 terrorist events in the US. Find out more how each scholarship offers the opportunity to study at a UK college or university every year.''
			\end{enumerate}
		\end{enumerate}
	\item {\it Youth in Action} European program: \url{http://www.britishcouncil.org/youthinaction}
	\item British Council Arts Group: \vspace{-0.2cm}
		\begin{enumerate} \itemsep -2pt
		\item Support and funding overview: \url{http://www.britishcouncil.org/arts-support-and-funding-overview.htm}
		\item Visual arts support and funding: \url{http://www.britishcouncil.org/arts-visual-arts-funding.htm}
		\item Drama and dance support and funding: \url{http://www.britishcouncil.org/arts-performing-arts-funding.htm}
		\item Literature support and funding: \url{http://www.britishcouncil.org/arts-literature-support-and-funding.htm}
		\item Film support and funding: \url{http://www.britishcouncil.org/arts-film-funding.htm}
		\item Music support and funding: \url{http://www.britishcouncil.org/arts-music-funding.htm}
		\item Architecture, design, fashion support and funding: \url{http://www.britishcouncil.org/arts-adf-funding.htm}
		\item International Short Film Festival Support Scheme: \url{http://www.britishcouncil.org/arts-film-short-films-scheme.htm}
		\end{enumerate}
	\end{enumerate}
\item Alfred P. Sloan Foundation: \vspace{-0.3cm}
	\begin{enumerate} \itemsep -2pt
	\item Sloan Research Fellowships: \vspace{-0.2cm}
		\begin{enumerate} \itemsep -2pt
		\item \url{http://www.sloan.org/fellowships}
		\item Hold a Ph.D. (or equivalent) in chemistry, physics, mathematics, computer science, economics, neuroscience or computational and evolutionary molecular biology, or in a related interdisciplinary field;
		\item Be members of the regular faculty (i.e., tenure track) of a degree-granting college or university in the United States or Canada; and
		\item Normally, be no more than six years from completion of the most recent Ph.D. or equivalent as of the year of their nomination.
		\end{enumerate}
	\end{enumerate}
\item --- --- --- --- --- --- --- --- --- --- --- --- --- --- --- --- --- --- --- --- --- --- --- --- --- --- --- --- --- --- ---
\item \colorbox{blue}{\bf Scholarships and Fellowships in Business (including Finance, Entrepreneurship, and Accounting)}
% Scholarships and Fellowships in Business (including Finance, Entrepreneurship, and Accounting)
\item IREX: \vspace{-0.3cm}
	\begin{enumerate} \itemsep -2pt
	\item Opportunities ``for individuals, organizations, universities, and alumni'': \url{http://www.irex.org/apply}
	\item Edmund S. Muskie Graduate Fellowship Program: \vspace{-0.2cm}
		\begin{enumerate} \itemsep -2pt
		\item : \url{http://www.irex.org/application/edmund-s-muskie-graduate-fellowship-program-application}
		\item ``The Muskie Program is open to graduate students and professionals from Armenia, Azerbaijan, Belarus, Georgia, Kazakhstan, Kyrgyzstan, Moldova, Russia, Tajikistan, Turkmenistan, Ukraine and Uzbekistan for one-year non-degree, one-year degree, or two-year degree study in the United States.''
		\item ``Eligible fields of study for the Muskie Program are: business administration, economics, education, environmental management, international affairs, journalism and mass communication, law, library and information science, public administration, public health, and {\bf public policy}.''
		\end{enumerate}
	\end{enumerate}
\item Sponsors for Educational Opportunity (SEO): \vspace{-0.3cm}
	\begin{enumerate} \itemsep -2pt
	\item Alternative Investment Fellowship Program: \vspace{-0.2cm}
		\begin{enumerate} \itemsep -2pt
		\item \url{http://www.seo-usa.org/Fellowship}
		\item Eligibility: \vspace{-0.1cm}
			\begin{enumerate} \itemsep -1pt
			\item \url{http://www.seo-usa.org/FellowshipEligibility}
			\item The program is open to professionals traditionally underrepresented in alternative investments who are in the first year (or second year with a third-year offer) of an analyst program at an investment bank.
			\item Corporate finance, M\&A, leveraged finance and structured finance analysts are preferred.
			\item Management consultants will also be considered.
			\end{enumerate}
		\end{enumerate}
	\item The SEO Scholars Program: \vspace{-0.2cm}
		\begin{enumerate} \itemsep -2pt
		\item \url{http://www.seo-usa.org/Scholars}
		\item The SEO Scholars Program is a rigorous out-of-school academic enrichment program that prepares motivated New York City public high school students of color to gain admission to and succeed at competitive colleges and universities throughout the country.  Numerous studies confirm that rigorous academics are the single most important factor for low-income and minority students in gaining college admission and earning a degree.  However, U.S. Department of Education research shows that ``A'' work in low-income schools equals ``C'' work in affluent schools.
		\item Admissions: \url{http://www.seo-usa.org/ScholarsAdmissions}
		\item Roadmap To Success: \url{http://www.seo-usa.org/ScholarsRoadmapToSuccess}
		\item Enrichment Programs: \url{http://www.seo-usa.org/ScholarsEnrichmentPrograms}
		\item Volunteering: \url{http://www.seo-usa.org/ScholarsVolunteering}
		\item Andrew Golkin Fund: \vspace{-0.1cm}
			\begin{enumerate} \itemsep -1pt
			\item \url{http://www.seo-usa.org/ScholarsAndrewGolkinFund}
			\item \url{http://www.seo-usa.org/andrewgolkinfund/index.html}
			\end{enumerate}
		\item Franklin H. and Shirley B. Williams Scholarship Fund: \url{http://www.seo-usa.org/ScholarsFHSBW}
		\item The Advantages of Attending a Competitive College: \url{http://www.seo-usa.org/ScholarsAdvantages}
		\end{enumerate}
	\item Career program: \vspace{-0.2cm}
		\begin{enumerate} \itemsep -2pt
		\item \url{http://www.seo-usa.org/Career}
		\item The SEO Career Program places students of color interested in finance, philanthropy, business and corporate law in internships with competitive pay, rigorous training, support through mentors, and broad access to industry professionals. 
		\item Sponsors for Educational Opportunity (SEO) is the nation's premiere summer internship program for talented underrepresented students of color that can lead to full-time job offers.
		\item SEO offers internship opportunities in the following areas: \vspace{-0.1cm}
			\begin{enumerate} \itemsep -1pt
			\item Corporate Financial Leadership: \url{http://www.seo-usa.org/Career/Corporate_Financial_Leadership}
			\item Banking/Asset Management Areas: \vspace{-0.1cm}
				\begin{itemize} \itemsep -1pt
				\item Investment Banking: \url{http://www.seo-usa.org/Career/Investment_Banking}
				\item Sales \& Trading: \url{http://www.seo-usa.org/Career/Sales_&_Trading}
				\item Investment Research: \url{http://www.seo-usa.org/Career/Investment_Research}
				\item Transaction Services: \url{http://www.seo-usa.org/Career/Transaction_Services}
				\item Asset Management: \url{http://www.seo-usa.org/Career/Asset_Management}
				\item Accounting/Finance: \url{http://www.seo-usa.org/Career/Accounting/Finance}
				\item Information Technology: \url{http://www.seo-usa.org/Career/Information_Technology}
				\end{itemize}
			\item Corporate Law: \url{http://www.seo-usa.org/Career/Corporate_Law}
			\item Nonprofit: \url{http://www.seo-usa.org/Career/Nonprofit}
			\item SEO-U: Freshmen and Sophomore Training: \url{http://www.seo-usa.org/Career/SEO-U:Freshmen_&_Sophomore_Training}
			\end{enumerate}
		\item Application Deadlines: \url{http://www.seo-usa.org/CareerApplicationDeadlines}
		\item Eligibility Information: \url{http://www.seo-usa.org/CareerEligibilityInfo}
		\item Application Tips: \url{http://www.seo-usa.org/CareerApplicationTips}
		\item Interview Tips: \url{http://www.seo-usa.org/CareerInterviewTips}
		\end{enumerate}
	\end{enumerate}
\item --- --- --- --- --- --- --- --- --- --- --- --- --- --- --- --- --- --- --- --- --- --- --- --- --- --- --- --- --- --- ---
\item \colorbox{blue}{\bf Scholarships for Studying Abroad}
% Scholarships for Studying Abroad
\item U.S. Department of State: \vspace{-0.3cm}
	\begin{enumerate} \itemsep -2pt
	\item Bureau of Educational and Cultural Affairs: \vspace{-0.2cm}
		\begin{enumerate} \itemsep -2pt
		\item Benjamin A. Gilman International Scholarship: \vspace{-0.1cm}
			\begin{enumerate} \itemsep -1pt
			\item \url{http://exchanges.state.gov/globalexchanges/gilman-scholarship-program.html}
			\item ``The Benjamin A. Gilman International Scholarship Program provides scholarships to U.S. undergraduates with financial need for study abroad, including students from diverse backgrounds and students going to non-traditional study abroad destinations.  Established under the International Academic Opportunity Act of 2000, Gilman Scholarships provide up to \$5,000 for American students to pursue overseas study for college credit.''
			\item Critical Need Languages: Students studying critical need languages are eligible for up to \$3,000 in additional funding as part of the Gilman Critical Need Language Supplement program. Those critical need languages include: \vspace{-0.1cm}
				\begin{itemize} \itemsep -1pt
				\item Arabic
				\item Chinese
				\item Korean
				\item Russian
				\item Turkic (Azerbaijani, Kazakh, Kyrgyz, Turkish, Turkmen, Uzbek)
				\item Persian (Farsi, Dari, Kurdish, Pashto, Tajiki)
				\item Indic (Hindi, Urdu, Nepali, Sinhala, Bengali, Punjabi, Marathi, Gujurati, Sindhi)
				\end{itemize}
			\item \url{http://www.iie.org/en/Programs/Gilman-Scholarship-Program}
			\item \url{http://www.iie.org/en/Programs/Gilman-Scholarship-Program/About-the-Program}
			\end{enumerate}
		\end{enumerate}
	\end{enumerate}
\item Council on International Educational Exchange (CIEE): \vspace{-0.3cm}
	\begin{enumerate} \itemsep -2pt
	\item CIEE Scholarships: \url{http://www.ciee.org/study/scholarships/index.aspx}
	\end{enumerate}
\item IES Abroad (formerly Institute of European Studies / Institute for the International Education of Students): \vspace{-0.3cm}
	\begin{enumerate} \itemsep -2pt
	\item Scholarships and Financial Aid: \url{https://www.iesabroad.org/IES/Scholarships_and_Aid/financialAid.html}
	\item IES Abroad Need-Based Financial Aid: \url{https://www.iesabroad.org/IES/Scholarships_and_Aid/Need-Based/needBasedFinancialAid.html}
	\item IES Abroad Merit-Based Scholarships: \url{https://www.iesabroad.org/IES/Scholarships_and_Aid/Merit_Based/meritBasedFinancialAid.html}
	\item IES Abroad Public University Grants: \url{https://www.iesabroad.org/IES/Scholarships_and_Aid/publicScholarship.html}
	\end{enumerate}
\item American Institute For Foreign Study (AIFS): \vspace{-0.3cm}
	\begin{enumerate} \itemsep -2pt
	\item AIFS Study Abroad Programs: \vspace{-0.2cm}
		\begin{enumerate} \itemsep -2pt
		\item \url{http://www.aifsabroad.com/programs.asp}
		\item AIFS Study Abroad Scholarships: \url{http://www.aifsabroad.com/scholarships.asp}
		\end{enumerate}
	\end{enumerate}
\item --- --- --- --- --- --- --- --- --- --- --- --- --- --- --- --- --- --- --- --- --- --- --- --- --- --- --- --- --- --- ---
\item \colorbox{blue}{\bf Scholarships and Fellowships in Public Policy and Public Health}
% Scholarships and Fellowships in Public Policy and Public Health
\item The Commonwealth Fund: \vspace{-0.3cm}
	\begin{enumerate} \itemsep -2pt
	\item Commonwealth Fund fellowship programs: \vspace{-0.2cm}
		\begin{enumerate} \itemsep -2pt
		\item \url{http://www.commonwealthfund.org/Fellowships.aspx}
		\item ``Commonwealth Fund fellowship programs are designed to give promising young researchers the opportunity for in-depth study of various health care policy topics, working with investigators, policy analysts, government officials, and others in a number of U.S. and international settings.''
		\item The Commonwealth Fund/Harvard University Fellowship in Minority Health Policy: \url{http://www.commonwealthfund.org/Fellowships/Minority-Health-Policy-Fellowship.aspx}
		\item Harkness Fellowships in Health Care Policy and Practice: \url{http://www.commonwealthfund.org/Fellowships/Harkness-Fellowships.aspx}
		\item Australian-American Health Policy Fellowship: \url{http://www.commonwealthfund.org/Fellowships/Australian-American-Health-Policy-Fellowships.aspx}
		\item Ian Axford (New Zealand) Fellowships in Public Policy: \url{http://www.commonwealthfund.org/Fellowships/Ian-Axford-Fellowships.aspx}
		\end{enumerate}
	\end{enumerate}
\item American Institute of Aeronautics and Astronautics (AIAA): \vspace{-0.3cm}
	\begin{enumerate} \itemsep -2pt
	\item Federal Government Fellows Program: \vspace{-0.2cm}
		\begin{enumerate} \itemsep -2pt
		\item \url{http://www.aiaa.org/content.cfm?pageid=731}
		\item Shaping U.S. {\bf public policy} concerning aerospace research and the aerospace industry
		\end{enumerate}
	\end{enumerate}
\item IEEE-USA: \vspace{-0.3cm}
	\begin{enumerate} \itemsep -2pt
	\item Congressional Fellowship
	\item Engineering \& Diplomacy (State Department) Fellowship
	\item For IEEE-USA members to support the creation and modification of technology-related public policies
	\item \url{http://ieeeusa.org/policy/govfel/default.asp}
	\end{enumerate}
\item American Mathematical Society: \vspace{-0.3cm}
	\begin{enumerate} \itemsep -2pt
	\item Fellowships and Awards (Policy and Advocacy: Government Relations \& Programs): \vspace{-0.2cm}
		\begin{enumerate} \itemsep -2pt
		\item \url{http://e-math.ams.org/policy/government/fellow-awards/fellow-awards}
		\item Mass Media Fellowships: \url{http://e-math.ams.org/programs/ams-fellowships/media-fellow/massmediafellow}
		\item AMS-AAAS Congressional Fellowship: \url{http://e-math.ams.org/programs/ams-fellowships/ams-aaas/ams-aaas-congressional-fellowship}
		\end{enumerate}
	\end{enumerate}
\item American Association for the Advancement of Science: \vspace{-0.3cm}
	\begin{enumerate} \itemsep -2pt
	\item AAAS Science \& Technology Policy Fellowships: \url{http://fellowships.aaas.org/index.shtml}
	\end{enumerate}
\item --- --- --- --- --- --- --- --- --- --- --- --- --- --- --- --- --- --- --- --- --- --- --- --- --- --- --- --- --- --- ---
\item \colorbox{blue}{\bf Scholarships and Fellowships in Social Science and Humanities}
% Scholarships and Fellowships in Social Science and Humanities
\item United States Institute of Peace (USIP): \vspace{-0.3cm}
	\begin{enumerate} \itemsep -2pt
	\item Jennings Randolph Peace Scholarship Dissertation Program (for Ph.D. students working on topics related to peace, conflict, and international security): \url{http://www.usip.org/grants-fellowships/jennings-randolph-peace-scholarship-dissertation-program}
	\end{enumerate}
\item Library of Congress: \vspace{-0.3cm}
	\begin{enumerate} \itemsep -2pt
	\item Kluge Fellowships: \vspace{-0.2cm}
		\begin{enumerate} \itemsep -2pt
		\item Research in the humanities and social sciences, especially interdisciplinary, cross-cultural or multilingual
		\item Open to scholars worldwide with a Ph.D. or other terminal advanced degree conferred within seven years of the July 15 deadline
		\item \url{http://www.loc.gov/loc/kluge/fellowships/kluge.html}
		\end{enumerate}
	\item J. Franklin Jameson Fellowship Research in American History (junior postdocs): \url{http://www.loc.gov/loc/kluge/fellowships/jameson.html}
	\item Kislak Short Term Fellowship Opportunities in American Studies (students, postdocs, and faculty): \url{http://www.loc.gov/loc/kluge/fellowships/kislakshort.html}
	\item Kislak Fellowship in American Studies (Ph.D. requirement): \url{http://www.loc.gov/loc/kluge/fellowships/kislak.html}
	\end{enumerate}
\item American Historical Association (AHA): \vspace{-0.3cm}
	\begin{enumerate} \itemsep -2pt
	\item AHA Research Grants: \url{http://www.historians.org/prizes/Grants.htm}
	\item Fellowships: \url{http://www.historians.org/prizes/Fellowships.htm}
	\end{enumerate}
\item American Sociological Association: \vspace{-0.3cm}
	\begin{enumerate} \itemsep -2pt
	\item ASA Dissertation Award: \url{http://www.asanet.org/about/awards/dissertation.cfm}
	\end{enumerate}
\item American Psychological Association: \vspace{-0.3cm}
	\begin{enumerate} \itemsep -2pt
	\item Scholarships, Grants, and Awards: \url{http://www.apa.org/about/awards/index.aspx}
	\end{enumerate}
\item American Anthropological Association (AAA): \vspace{-0.3cm}
	\begin{enumerate} \itemsep -2pt
	\item AAA Minority Dissertation Fellowship Program (for minority Ph.D. candidates in anthropology): \url{http://www.aaanet.org/cmtes/minority/Minfellow.cfm}
	\item Margaret Mead Award (for young scholars in anthropology): \url{http://www.aaanet.org/about/Prizes-Awards/AAA-Margaret-Mead-Award.cfm}
	\item COSWA Award: \vspace{-0.2cm}
		\begin{enumerate} \itemsep -2pt
		\item The COSWA Award (formerly the Squeaky Wheel Award), sponsored by the Committee on the Status of Women in Anthropology (COSWA), recognizes individuals who have demonstrated the courage to bring to light and investigate practices in anthropology that are potentially discriminatory to women, or have acted to improve the status of women in anthropology through activities that raise awareness of women's contribution to anthropology or identify barriers to full participation by women in anthropology.
		\item \url{http://www.aaanet.org/about/Prizes-Awards/COSWA-Award.cfm}
		\end{enumerate}
	\item David M. Schneider Award (for Ph.D. students in anthropology): \url{http://www.aaanet.org/about/Prizes-Awards/David-Schneider-Award.cfm}
	\item Links to ``Section Prizes \& Awards'': \url{http://www.aaanet.org/about/Prizes-Awards/section_awards.cfm}
	\item List of national (US) and international ``Grants and Fellowships'': \url{http://www.aaanet.org/profdev/fellowships/}
	\item \url{http://www.aaanet.org/}
	\end{enumerate}
\item National Academy of Social Insurance: \vspace{-0.3cm}
	\begin{enumerate} \itemsep -2pt
	\item John Heinz Dissertation Award (Ph.D. students writing their thesis on the planning and implementation of social insurance): \url{http://www.nasi.org/studentopps/heinz}
	\end{enumerate}
\item National Endowment for the Humanities's Division of Research Programs, grants and fellowship opportunities: \url{http://www.neh.gov/grants/}
\item {\it The Henry Luce Foundation}'s Luce Scholars Program to help US graduates learn more about Asia and Asian culture(s): \url{http://www.hluce.org/lsprogram.aspx}
\item Institute for Humane Studies at George Mason University: \vspace{-0.3cm}
	\begin{enumerate} \itemsep -2pt
	\item Humane Studies Fellowships: \vspace{-0.2cm}
		\begin{enumerate} \itemsep -2pt
		\item \url{http://www.theihs.org/programs/humane-studies-fellowships}
		\item Humane Studies Fellowships are awarded to graduate students and outstanding undergraduates planning academic careers with liberty-advancing research interests.
		\item The fellowships are open to students in a range of fields, such as economics, philosophy, law, political science, anthropology, and literature.
		\end{enumerate}
	\end{enumerate}
\item The Gilder Lehrman Institute of American History: Gilder Lehrman History Scholars \& Gilder Lehrman One-Week Scholars (for sophomores or juniors majoring in American history or American Studies), \url{http://www.gilderlehrman.org/education/hs_program_details.php}
\item Myra Sadker Foundation: \vspace{-0.3cm}
	\begin{enumerate} \itemsep -2pt
	\item \url{http://www.sadker.org/awards.html}
	\item Teacher Award: Designed to promote and support teacher projects (K-12) that help students learn about and respect group differences, promote fairness, and in other ways build upon the values and contributions of Myra Sadker's work. Each project should have a gender dimension.
	\item Student Award: Designed to encourage student ideas, activities and projects (K-12) that promote respect for group differences, fairness, and in other ways build upon the values and contributions of Myra Sadker's work. Each project should have a gender dimension. 
	\item Doctoral Dissertation Award: Designed to promote and support graduate students engaged in educational equity research. Doctoral level dissertations that explore or promote educational equity and fairness based on gender, race, ethnicity, religion, class, sexual orientation, or other such variables will be considered for support. Each dissertation should have a gender dimension.
	\end{enumerate}
\item IREX: \vspace{-0.3cm}
	\begin{enumerate} \itemsep -2pt
	\item Opportunities ``for individuals, organizations, universities, and alumni'': \url{http://www.irex.org/apply}
	\item Edmund S. Muskie Graduate Fellowship Program: \vspace{-0.2cm}
		\begin{enumerate} \itemsep -2pt
		\item : \url{http://www.irex.org/application/edmund-s-muskie-graduate-fellowship-program-application}
		\item ``The Muskie Program is open to graduate students and professionals from Armenia, Azerbaijan, Belarus, Georgia, Kazakhstan, Kyrgyzstan, Moldova, Russia, Tajikistan, Turkmenistan, Ukraine and Uzbekistan for one-year non-degree, one-year degree, or two-year degree study in the United States.''
		\item ``Eligible fields of study for the Muskie Program are: business administration, economics, education, environmental management, international affairs, journalism and mass communication, law, library and information science, public administration, public health, and {\bf public policy}.''
		\end{enumerate}
	\item Legal Education and Development (LEAD) Fellowship: \vspace{-0.2cm}
		\begin{enumerate} \itemsep -2pt
		\item \url{http://www.irex.org/application/legal-education-and-development-lead-fellowship-application}
		\item Legal Education and Development Fellowship Program (LEAD) in Tajikistan
		\item Eligibility: \vspace{-0.1cm}
			\begin{enumerate} \itemsep -1pt
			\item Is a citizen, national, or permanent resident qualified to hold a valid passport issued by Tajikistan;
			\item Is the recipient of an undergraduate degree in law (four- or five-year study) by the time of the application;
			\item Is able to begin the academic exchange program in the United States in the summer of 2011;
			\item Is able to receive and maintain a United States J-1 visa.
			\end{enumerate}
		\end{enumerate}
	\item Community Solutions Program: \vspace{-0.2cm}
		\begin{enumerate} \itemsep -2pt
		\item \url{http://www.irex.org/application/community-solutions-information-applicants}
		\item ``a professional development program for the best and brightest global community leaders working in Transparency \& Accountability, Tolerance/Conflict Resolution, Environmental Issues, and Women's Issues''
		\item ``Competition for the Community Solutions Program is merit-based and open to community leaders, ages 25-38 at the time of application''
		\end{enumerate}
	\item Crimea Undergraduate Exchange Program (Crimea UGRAD) Application: \vspace{-0.2cm}
		\begin{enumerate} \itemsep -2pt
		\item \url{http://www.irex.org/application/crimea-undergraduate-exchange-program-crimea-ugrad-application}
		\item ``The Crimea UGRAD Program is open to undergraduate students from the Autonomous Republic of Crimea for one academic year of non-degree study in a US university or community college.''
		\end{enumerate}
	\end{enumerate}
\item {\it Demos}: \vspace{-0.3cm}
	\begin{enumerate} \itemsep -2pt
	\item The Ed Baker Fellowship in Democratic Values: \vspace{-0.2cm}
		\begin{enumerate} \itemsep -2pt
		\item \url{http://www.demos.org/edbakerfellowship.cfm}
		\item ``Based in our New York offices, Ed Baker Fellows will give voice to strong democratic values within a wide range of potential issues, including voting rights, citizen engagement, immigration policy and civic inclusion, campaign finance reform and money in politics, and media reform, among others.''
		\end{enumerate}
	\item Fellows Program: \vspace{-0.2cm}
		\begin{enumerate} \itemsep -2pt
		\item \url{http://www.demos.org/fellowsapp.cfm}
		\item \url{http://www.demos.org/program.cfm?currentprogramID=5A196E48-3FF4-6C82-50CBCA5825B661BA}
		\item ``The Fellows Program at Demos provides support and community for writers and thinkers with well-defined projects that aim to advance the values at the core of Demos' programs and mission: a robust and inclusive democracy; shared prosperity; strong \& effective public governance; and global interdependence.''
		\end{enumerate}
	\end{enumerate}
\item Research Councils UK (RCUK): \vspace{-0.3cm}
	\begin{enumerate} \itemsep -2pt
	\item Economic and Social Research Council (ESRC): \vspace{-0.2cm}
		\begin{enumerate} \itemsep -2pt
		\item Academic (funding opportunities for students, postdocs, and professors): \url{http://www.esrcsocietytoday.ac.uk/ESRCInfoCentre/index_academic.aspx}
		\item Professorial Fellowships (for leading senior social scientists): \url{http://www.esrcsocietytoday.ac.uk/ESRCInfoCentre/opportunities/professorial/}
		\item Funding opportunities: \vspace{-0.1cm}
			\begin{enumerate} \itemsep -1pt
			\item \url{http://www.esrcsocietytoday.ac.uk/ESRCInfoCentre/index_government.aspx}
			\item \url{http://www.esrcsocietytoday.ac.uk/ESRCInfoCentre/opportunities/}
			\item ESRC Research Funding Guide / ESRC's Funding Rules: \url{http://www.esrcsocietytoday.ac.uk/ESRCInfoCentre/opportunities/research_funding}
			\item Eligibility for Research Council Funding: \url{http://www.esrcsocietytoday.ac.uk/ESRCInfoCentre/opportunities/eligibility}
			\item Current Funding Opportunities: \url{http://www.esrcsocietytoday.ac.uk/ESRCInfoCentre/opportunities/current_funding_opportunities/}
			\item Forthcoming funding opportunities: \url{http://www.esrcsocietytoday.ac.uk/ESRCInfoCentre/opportunities/forthcoming_opportunities/}
			\item Placement Fellows Scheme: \url{http://www.esrcsocietytoday.ac.uk/ESRCInfoCentre/opportunities/placement/}
			\item Professorial Fellowships: \url{http://www.esrcsocietytoday.ac.uk/ESRCInfoCentre/opportunities/professorial/}
			\item Early Career Researchers (including Postdoctoral Fellowships, International Training, and Networking Opportunities): \url{http://www.esrcsocietytoday.ac.uk/ESRCInfoCentre/opportunities/earlycareer/}
			\item Postgraduate and Career Development Opportunities: \url{http://www.esrcsocietytoday.ac.uk/ESRCInfoCentre/opportunities/postgraduate/}
			\item International Funding Opportunities: \url{http://www.esrcsocietytoday.ac.uk/ESRCInfoCentre/opportunities/international/}
			\item Joint Funding Opportunities: \url{http://www.esrcsocietytoday.ac.uk/ESRCInfoCentre/opportunities/jointfunding/}
			\item Annual competitions: \url{http://www.esrcsocietytoday.ac.uk/ESRCInfoCentre/opportunities/annual/index.aspx#3}
			\end{enumerate}
		\end{enumerate}
	\item Arts and Humanities Research Council (AHRC): \vspace{-0.2cm}
		\begin{enumerate} \itemsep -2pt
		\item Funding Opportunities: \vspace{-0.1cm}
			\begin{enumerate} \itemsep -1pt
			\item \url{http://www.ahrc.ac.uk/FundingOpportunities/Pages/default.aspx}
			\item Fellowships: \url{http://www.ahrc.ac.uk/FundingOpportunities/Pages/Fellowships.aspx}
			\item Fellowships - route for early career researchers: \url{http://www.ahrc.ac.uk/FundingOpportunities/Pages/Fellowshipserc.aspx}
			\item Placement Fellowship based in the Department for Culture, Media and Sport (DCMS) - Climate Change: \url{http://www.ahrc.ac.uk/FundingOpportunities/Pages/PlacementFellowshipDCMS-Climatechange.aspx}
			\item Placement Fellowship based in the Department for Culture, Media and Sport (DCMS) - Health and Wellbeing: \url{http://www.ahrc.ac.uk/FundingOpportunities/Pages/PlacementFellowshipDCMShealthandwellbeing.aspx}
			\item Research Grants - route for early career researchers: \url{http://www.ahrc.ac.uk/FundingOpportunities/Pages/RG-EarlyCareers.aspx}
			\item Research Grants - Speculative Research: \url{http://www.ahrc.ac.uk/FundingOpportunities/Pages/RG-SpeculativeResearch.aspx}
			\item Research Grants - Standard Route: \url{http://www.ahrc.ac.uk/FundingOpportunities/Pages/RG-StandardRoute.aspx}
			\item Postgraduate Funding (for Masters and Ph.D. students): \url{http://www.ahrc.ac.uk/FundingOpportunities/Pages/summaryinformationforprospectivepostgraduatestudents.aspx}
			\item Browse Funding Opportunities: \url{http://www.ahrc.ac.uk/FundingOpportunities/Pages/BrowseOpportunities.aspx}
			\end{enumerate}
		\end{enumerate}
	\end{enumerate}
\item World Bank Institute (WBI): \vspace{-0.3cm}
	\begin{enumerate} \itemsep -2pt
	\item Or The World Bank Group
	\item Scholarships: \url{http://wbi.worldbank.org/wbi/scholarships} or \url{http://www.worldbank.org/wbi/scholarships/home.html}
	\end{enumerate}
\item --- --- --- --- --- --- --- --- --- --- --- --- --- --- --- --- --- --- --- --- --- --- --- --- --- --- --- --- --- --- ---
\item \colorbox{blue}{\bf Fellowships in Art and Music}
% Fellowships in Art and Music
\item The Kresge Foundation: \vspace{-0.3cm}
	\begin{enumerate} \itemsep -2pt
	\item \url{http://www.kresge.org/index.php/what/detroit_program/kresge_arts_in_detroit/}
	\item Kresge Artist Fellowships: \vspace{-0.2cm}
		\begin{enumerate} \itemsep -2pt
		\item ``Kresge Artist Fellowships seek to advance the art forms and professional careers of artists from the visual, performing and literary arts as well as elevate the profile of the artistic community and encourage creative expression in the region. Each year, Kresge will provide funding for 18 fellowships of \$25,000 each, which are awarded to artists living and working in metropolitan Detroit.''
		\item ``The fellowships recognize creative vision and commitment to excellence within a wide range of artistic disciplines, including artists who have been classically and academically trained, self taught artists and artists whose art forms have been passed down through cultural and traditional heritage.''
		\item ``Kresge Arts in Detroit is committed to supporting artists from diverse cultural backgrounds at all stages of their professional careers.''
		\item \url{http://kresge.collegeforcreativestudies.edu/}
		\item \url{http://kresge.collegeforcreativestudies.edu/kaf_guidelines.html}
		\item Information Sessions: \url{http://kresge.collegeforcreativestudies.edu/kaf_sessions.html}
		\end{enumerate}
	\item Kresge Eminent Artist Award: \vspace{-0.2cm}
		\begin{enumerate} \itemsep -2pt
		\item ``Kresge Eminent Artist Award recognizes an exceptional artist for his or her professional achievements and contributions to the cultural community, and encourages that individual's pursuit of a chosen art form as well as an ongoing commitment to metropolitan Detroit. Each year, one highly accomplished individual will be presented with the award which includes a \$50,000 prize.''
		\item \url{http://kresge.collegeforcreativestudies.edu/eminent-artist-award.html}
		\end{enumerate}
	\end{enumerate}
\item Guggenheim Fellowships from the {\it John Simon Guggenheim Memorial Foundation}: \url{http://www.gf.org/applicants}
\item The John F. Kennedy Center for the Performing Arts: \vspace{-0.3cm}
	\begin{enumerate} \itemsep -2pt
	\item DeVos Institute of Arts Management at the Kennedy Center: \vspace{-0.2cm}
		\begin{enumerate} \itemsep -2pt
		\item DeVos Institute Programs: \vspace{-0.1cm}
			\begin{enumerate} \itemsep -1pt
			\item Kennedy Center Fellowship Program: \vspace{-0.1cm}
				\begin{itemize} \itemsep -1pt
				\item \url{http://www.kennedy-center.org/education/artsmanagement/fellowships.cfm}
				\item \url{http://www.kennedy-center.org/education/artsmanagement/fellowships/home.html}
				\item ``The Kennedy Center Fellowship Program began in 2001, and provides comprehensive study to 10 arts managers at the Kennedy Center with coursework in strategic planning, marketing, and development; three practical work rotations in Center departments; and a series of professional development seminars. The paid fellowships are full-time and last nine months from September through May.''
				\end{itemize}
			\item DeVos Institute Summer International Fellowship Program at the Kennedy Center: \vspace{-0.1cm}
				\begin{itemize} \itemsep -1pt
				\item \url{http://www.kennedy-center.org/education/artsmanagement/fellowships.cfm}
				\item \url{http://www.kennedy-center.org/education/artsmanagement/international_faq.cfm}
				\item ``The Summer International Fellowship Program provides practical experience to 15 mid-to-high level arts leaders currently working in international nonprofit performing arts organizations. This full-time, four-week intensive program takes place at the Kennedy Center each July; Fellows attend each summer for three consecutive years. While at the Center, the fellows take classes and refine strategic plans for their home organizations.''
				\end{itemize}
			\item U.S. Department of State International Exchange Programs: \vspace{-0.1cm}
				\begin{itemize} \itemsep -1pt
				\item \url{http://www.kennedy-center.org/education/state/}
				\item ``The U.S. Department of State and The Kennedy Center have teamed to produce international exchange opportunities through the Performing Artists Cultural Visitors Program and International Cultural Fellows Mentoring Program.''
				\item Performing Artists Cultural Visitors Program: \url{http://www.kennedy-center.org/education/state/cultural/}
				\item International Cultural Fellows Mentoring Program: \url{http://www.kennedy-center.org/education/state/fellows/}
				\item ``Visitors, comprised of modern and hip-hop dancers, theater technicians/designers/actors, as well as classical and jazz musicians, engage with American colleagues in the creation and performance of their discipline in Washington, D.C. and in another American city.''
				\item ``The Fellows, comprised of arts managers and presenters from outside the United States, attend arts management seminars led by Kennedy Center staff, travel to another American city to study with a mentor organization, and visit New York City to meet with experts in their field.''
				\end{itemize}
			\end{enumerate}
		\end{enumerate}
	\item The National Symphony Orchestra (NSO): \vspace{-0.2cm}
		\begin{enumerate} \itemsep -2pt
		\item National Symphony Orchestra Youth Fellowship Program: \vspace{-0.1cm}
			\begin{itemize} \itemsep -1pt
			\item \url{http://www.kennedy-center.org/nso/nsoed/youthfellowship.cfm}
			\item \url{http://www.kennedy-center.org/explorer/artists/?entity_id=10811&source_type=B}
			\item ``Now in its 30th season, the National Symphony Orchestra Youth Fellowship Program is an orchestral training project for high school musicians.''
			\item ``From its inception in 1980-81 to the present, the program provides Washington metropolitan area high school students with scholarships to study privately with NSO members, as well as opportunities to observe NSO rehearsals; attend concerts; and to participate in seminars, discussions, and master classes with musicians, conductors, and NSO and Kennedy Center management.''
			\item ``There are 20 students in the NSO Youth Fellowship Program for 2009-10.''
			\item ``Participation by ethnic minorities is encouraged.''
			\item ``Priority is given to students entering 10th grade in order to provide as sustained a training as possible.''
			\end{itemize}
		\end{enumerate}
	\end{enumerate}
\item League of American Orchestras: \vspace{-0.3cm}
	\begin{enumerate} \itemsep -2pt
	\item Fellowships: \vspace{-0.2cm}
		\begin{enumerate} \itemsep -2pt
		\item \url{http://www.americanorchestras.org/learning_and_leadership/fellowships.html}
		\item Orchestra Management Fellowship Program: \vspace{-0.1cm}
			\begin{enumerate} \itemsep -1pt
			\item \url{http://www.americanorchestras.org/learning_and_leadership/omfp.html}
			\item ``This year-long, highly competitive program is designed to launch executive careers in orchestra management.''
			\item ``Along with an intense course of study, fellows undertake a series of residencies with orchestras of various sizes across the U.S. receiving invaluable work experience and the support of host orchestra staff, in particular that of the orchestra�s executive director.''
			\item ``Fellows also participate in other League leadership seminars throughout the year and receive a comprehensive overview of the classical music industry.''
			\end{enumerate}
		\item ``The League's Fellowship programs identify and prepare the future leaders of tomorrow, today.''
		\item ``Long-term curricula, developed for conductors, executive directors, and managers looking to advance, provide intensive education, hands-on learning, and valuable networking opportunities.''
		\end{enumerate}
	\end{enumerate}
\item Americans for the Arts: \vspace{-0.3cm}
	\begin{enumerate} \itemsep -2pt
	\item Event scholarships (scholarships to attend events): \url{http://www.artsusa.org/events/scholarships.asp}
	\item \url{http://www.artsusa.org/news/annual_awards/default.asp}
	\item Alene Valkanas State Arts Advocacy Award\url{http://www.artsusa.org/news/annual_awards/alene_valkanas/default.asp}
	\item Arts Education Award (awarded to institutions): \url{http://www.artsusa.org/news/annual_awards/arts_education/default.asp}
	\item Emerging Leader Award: \url{http://www.artsusa.org/news/annual_awards/emerging_leader/default.asp}
	\item Michael Newton Award for United Arts Funds Leadership (management and fundraising): \url{http://www.artsusa.org/news/annual_awards/michael_newton/default.asp}
	\item Selina Roberts Ottum Award (contributions to the field of the arts): \url{http://www.artsusa.org/news/annual_awards/selina_roberts_ottum/default.asp}
	\item United States Urban Arts Federation (USUAF): \vspace{-0.2cm}
		\begin{enumerate} \itemsep -2pt
		\item Ray Hanley Innovation Award: \url{http://www.artsusa.org/networks/usuaf/hanley.asp}
		\end{enumerate}
	\end{enumerate}
\item NEA National Heritage Fellowship (for master folk and traditional artists): \url{http://www.nea.gov/honors/heritage/index.html}
\item NEA Jazz Masters Fellowship (jazz artists): \url{http://www.arts.gov/honors/jazz/index.html}
\item Fellowships for Creative Writers [or NEA Literature Fellowships: Creative Writing]: \url{http://www.nea.gov/grants/apply/Lit/index.html} or \url{http://www.arts.gov/grants/apply/Lit/index.html}
\item Carnegie Investment Bank: Carnegie Art Award (for distinguished artists born or living in the Nordic countries), \url{http://www.carnegie.se/sv/ArtAward/About-Carnegie-Art-Award/}, \url{http://www.carnegie.se/artaward/}, and \url{http://www.carnegie.se/en/about/Operations/Carnegie-Art-Award/}
\item Robert McCann Foundation: \vspace{-0.3cm}
	\begin{enumerate} \itemsep -2pt
	\item Funding for artists and designers ``from all Scottish colleges and art schools'' to: \vspace{-0.2cm}
		\begin{enumerate} \itemsep -2pt
		\item extend their training in an area of specialization; OR
		\item finance a project ``in the craft industries associated with film and television''
		\end{enumerate}
	\item \url{http://robertmccannfoundation.com/how.html}
	\end{enumerate}
\item Alexander von Humboldt-Stiftung/Foundation: \vspace{-0.3cm}
	\begin{enumerate} \itemsep -2pt
	\item Hezekiah Wardwell Fellowship (for musicians or musicologists from Spain): \url{http://www.humboldt-foundation.de/web/wardwell-en.html}
	\end{enumerate}
\item Canada Council for the Arts: \vspace{-0.3cm}
	\begin{enumerate} \itemsep -2pt
	\item Endowments and Prizes: \vspace{-0.2cm}
		\begin{enumerate} \itemsep -2pt
		\item \url{http://www.canadacouncil.ca/prizes/}
		\item Prizes and fellowships for Canadian artists and scholars to recognize their contributions to the arts, humanities, and sciences
		\item Categories of prizes and fellowships: \vspace{-0.1cm}
			\begin{enumerate} \itemsep -1pt
			\item dance
			\item inter-arts
			\item media arts
			\item music
			\item theatre
			\item visual arts
			\item writing and publishing
			\end{enumerate}
		\end{enumerate}
	\item Grant Programs: \url{http://www.canadacouncil.ca/grants/}
	\end{enumerate}
\item Institute for Humane Studies at George Mason University: \vspace{-0.3cm}
	\begin{enumerate} \itemsep -2pt
	\item Film \& Fiction Scholarships: \vspace{-0.2cm}
		\begin{enumerate} \itemsep -2pt
		\item Students pursuing MFAs in a variety of areas are eligible: film directing, production, screenwriting, playwriting, fiction, and literary-nonfiction writing
		\item \url{http://www.theihs.org/node/448}
		\end{enumerate}
	\end{enumerate}
\item --- --- --- --- --- --- --- --- --- --- --- --- --- --- --- --- --- --- --- --- --- --- --- --- --- --- --- --- --- --- ---
\item \colorbox{blue}{\bf Scholarships and Fellowships for Underrepresented Minorities}
% Scholarships and Fellowships for Underrepresented Minorities
\item Lists of scholarships and fellowships for underrepresented minorities: \vspace{-0.3cm}
	\begin{enumerate} \itemsep -2pt
	\item Chris Enstrom, ``Cashing in on Diversity Grants and Scholarships,'' in Graduating Engineer \& Computer Careers. Available at: \url{http://www.graduatingengineer.com/higher-education/20061129/Cashing-in-on-Diversity-Grants-and-Scholarships-}; last accessed on August 25, 2010.
	\end{enumerate}
\item Gates Millennium Scholars (GMS) scholarship (for underrepresented minorities in the US): \url{http://www.gmsp.org/}
\item Society of Women Engineers (SWE): SWE Scholarships and other scholarships, \url{http://societyofwomenengineers.swe.org/index.php?option=com_content&task=view&id=222&Itemid=111}
\item Coalition to Diversify Computing: \url{http://www.cdc-computing.org/scholarships/}
\item IES Abroad (formerly Institute of European Studies / Institute for the International Education of Students): \vspace{-0.3cm}
	\begin{enumerate} \itemsep -2pt
	\item Diversity Abroad: \vspace{-0.2cm}
		\begin{enumerate} \itemsep -2pt
		\item \url{https://www.iesabroad.org/IES/Diversity/diversity.html}
		\item Programs to improve student diversity in study abroad programs
		\item IES Abroad Diversity Scholarships: \vspace{-0.1cm}
			\begin{enumerate} \itemsep -1pt
			\item IES Abroad Merit-Based Scholarship for Under-represented Students: \url{https://www.iesabroad.org/IES/Scholarships_and_Aid/Diversity_Scholarships/diversityScholarship.html}
			\item IES Abroad Merit-Based David Porter Diversity Scholarship (Up to \$5,000!): \url{https://www.iesabroad.org/IES/Scholarships_and_Aid/Merit_Based/davidPorterScholarship.html}
			\item HBCU Scholarships: \url{https://www.iesabroad.org/IES/Scholarships_and_Aid/Diversity_Scholarships/hbcuScholarship.html}
			\item HACU-IES Abroad Merit/Need-Based Scholarship: \url{https://www.iesabroad.org/IES/Scholarships_and_Aid/Diversity_Scholarships/HACUScholarship.html}
			\end{enumerate}
		\end{enumerate}
	\end{enumerate}
\item MassMutual Scholars Program: \vspace{-0.3cm}
	\begin{enumerate} \itemsep -2pt
	\item Applicants must be undergraduates of African American/Black, Asian/Pacific Islander or Hispanic decent in the US.
	\item Reside or plan to attend an institution in one of the following metropolitan areas: \vspace{-0.2cm}
		\begin{enumerate} \itemsep -2pt
		\item Atlanta, GA
		\item Chicago, IL
		\item Central New Jersey
		\item Denver, CO
		\item Houston, TX
		\item Miami, FL
		\item Los Angeles, CA
		\item San Antonio, TX
		\item San Francisco, CA
		\end{enumerate}
	\item Be majoring in business, economics, finance, financial planning, management, marketing or sales.
	\item \url{http://www.hsf.net/massmutual.aspx}
	\item \url{http://www.apiasf.org/scholarship_apiasf_massmutual.html}
	\end{enumerate}
\item {\it NASA}'s Minority University Research and Education Program (MUREP): \vspace{-0.3cm}
	\begin{enumerate} \itemsep -2pt
	\item \url{http://www.nasa.gov/offices/education/programs/national/murep/home/index.html}
	\item \url{http://www.nasa.gov/offices/education/about/murep_overview.html}
	\item Jenkins Pre-doctoral Fellowship Project, JPFP: \url{http://www.nasa.gov/offices/education/programs/descriptions/Jenkins_Predoctoral_Fellowship_Project.html}
	\end{enumerate}
\item UNCF: \vspace{-0.3cm}
	\begin{enumerate} \itemsep -2pt
	\item UNCF Special Programs Corporation: \vspace{-0.2cm}
		\begin{enumerate} \itemsep -2pt
		\item Harriett G. Jenkins Pre-doctoral Fellowship Program (JPFP) for underrepresented minorities pursuing graduate degrees in STEM: \url{http://www.uncfsp.org/spknowledge/default.aspx?page=program.view&areaid=1&contentid=177&typeid=jpfp}
		\item NASA Science and Technology Institute (NSTI) Summer Scholars Program (10-week summer research scholarship): \url{http://www.uncfsp.org/spknowledge/default.aspx?page=program.view&areaid=1&contentid=172&typeid=nstiinternship}
		\item Motivating Undergraduates in Science and Technology (MUST) Program for undergraduates in STEM: \url{http://www.uncfsp.org/spknowledge/default.aspx?page=program.view&areaid=1&contentid=346&typeid=must}
		\item Institute for International {\bf Public Policy} Fellows Program: \url{http://www.uncfsp.org/IIPP}
		\item \url{http://www.uncfsp.org/spknowledge/default.aspx?page=home.default}
		\end{enumerate}
	\item UNCF scholarship resources: \url{http://www.uncf.org/forstudents/scholarship.asp}
	\item UNCF $\cdot$ Merck Science Initiative: scholarships and fellowships: \url{http://umsi.uncf.org/ScholarshipsInternshipsFellowships/tabid/151/Default.aspx}
	\end{enumerate}
\item Hispanic College Fund: \vspace{-0.3cm}
	\begin{enumerate} \itemsep -2pt
	\item Scholarships: \url{http://www.hispanicfund.org/scholarships/} and \url{http://scholarships.hispanicfund.org/applications/}
	\item NASA MUST Scholarship Program: \url{http://www.hispanicfund.org/nasa-must/}
	\item Hispanic Youth Symposium (scholarships are awarded to winners of the art competition, talent competition, and speech competition): \url{http://www.hispanicyouth.org/about-the-program}
	\item \url{http://www.hispanicfund.org/}
	\end{enumerate}
\item Hispanic Heritage Foundation (HHF): \vspace{-0.3cm}
	\begin{enumerate} \itemsep -2pt
	\item Scholarships and Resources: \url{http://www.hispanicheritage.org/youth_int.php?sec=80}
	\item \url{http://www.hispanicheritage.org/}
	\end{enumerate}
\item Hispanic Scholarship Fund (HSF): \vspace{-0.3cm}
	\begin{enumerate} \itemsep -2pt
	\item Scholarship programs for: \vspace{-0.2cm}
		\begin{enumerate} \itemsep -2pt
		\item college students
		\item community college transfer students
		\item high school students
		\item Gates Millennium Scholars
		\item See \url{http://www.hsf.net/innercontent.aspx?id=34}
		\end{enumerate}
	\item \url{http://www.hsf.net/}
	\end{enumerate}
\item League of United Latin American Citizens (LULAC): \vspace{-0.3cm}
	\begin{enumerate} \itemsep -2pt
	\item LULAC National Educational Service Centers, Inc: \vspace{-0.2cm}
		\begin{enumerate} \itemsep -2pt
		\item \url{http://www.lnesc.org/}
		\item LULAC National Scholarship Fund (LNSF): \vspace{-0.1cm}
			\begin{enumerate} \itemsep -1pt
			\item \url{http://www.lulac.org/programs/education/scholarships/}
			\item \url{http://lnesc.org/index.asp?Type=B_BASIC&SEC={3AEDB506-F425-4E58-B9F6-44867E2FD943}}
%http://lnesc.org/index.asp?Type=B_BASIC&SEC={3AEDB506-F425-4E58-B9F6-44867E2FD943}
			\item Applicants must meet the following criteria to be considered for a scholarship: \vspace{-0.1cm}
				\begin{itemize} \itemsep -1pt
				\item Must be a U.S. citizen or legal resident
				\item Must have applied to or be enrolled in a   college, university, or graduate school, including 2-year colleges, or vocational schools that lead to an associate�s degree
				\item A student will not be eligible for a scholarship if he/she is related to a scholarship committee member, the Council President, or an individual contributor to the local funds of the Council
				\end{itemize}
			\item National Scholastic Achievement Awards (for high school seniors entering college, university, or vocational school)
			\item Honors Awards (for high school seniors entering college, university, or vocational school)
			\item General Awards (Need, community involvement, and leadership activities will also be considered)
			\item General Electric Foundation/ LULAC Scholarship program: for underrepresented minorities (US freshmen) entering their sophomore year as majors in Business or Engineering with a cumulative college G.P.A. $\leq$ 3.25/4.0; these students must be enrolled in a 4-year undergraduate program.
			\end{enumerate}
		\end{enumerate}
	\end{enumerate}
\item Hispanic Association of Colleges and Universities (HACU): \vspace{-0.3cm}
	\begin{enumerate} \itemsep -2pt
	\item HACU Student Programs Overview: \vspace{-0.2cm}
		\begin{enumerate} \itemsep -2pt
		\item \url{http://www.hacu.net/hacu/HACU_Student_Programs_EN.asp?SnID=1942709283}
		\item HACU Scholarship Programs: \vspace{-0.1cm}
			\begin{enumerate} \itemsep -1pt
			\item \url{http://www.hacu.net/hacu/Scholarships_EN.asp?SnID=1942709283}
			\item Includes scholarships for students in: \vspace{-0.1cm}
				\begin{itemize} \itemsep -1pt
				\item Accounting
				\item Behavioral Health
				\item Business
				\item Clinical Psychology
				\item Computer Engineering
				\item Computer Science
				\item Dental Technician
				\item Electrical Engineering
				\item Engineering
				\item Food Merchandising
				\item Information Technology
				\item International Business
				\item Management
				\item Marketing
				\item Mass Media
				\item Mental Health
				\item Merchandising
				\item Nursing
				\item Physician Assistant
				\item (Pre) Optometry
				\item (Pre) Dental
				\item (Pre) Medicine
				\item (Pre) Pharmacy
				\item Public Health
				\item Public Relations
				\item Retail Management
				\item Sports Marketing
				\item Technology
				\end{itemize}
			\end{enumerate}
		\item ``D{\'{a}}ndole Alas a Tu {\'{E}}xito/Giving Flight to Your Success'' travel award program (Southwest Airlines' Travel Award Program): \vspace{-0.1cm}
			\begin{enumerate} \itemsep -1pt
			\item For students with financial need who have to across the United States to participate in their undergraduate or graduate degree programs
			\item \url{http://www.hacu.net/hacu/Lanzate_EN.asp?SnID=1942709283}
			\item \url{http://www.hacu.net/hacu/Lanzate1_EN.asp?SnID=1808826658}
			\end{enumerate}
		\item HACU Study Abroad Scholarship Programs: \vspace{-0.1cm}
			\begin{enumerate} \itemsep -1pt
			\item \url{http://www.hacu.net/hacu/Study_Abroad_EN.asp?SnID=1808826658}
			\item HACU-Global Learning Semesters (GLS) Program: Hispanic Study Abroad Scholars: \url{http://www.studyabroadscholars.org/index.html}
			\item HACU-American Institute for Foreign Study (AIFS) Scholarship Program: \url{http://www.aifsabroad.com/scholarships.asp#hacu}
			\item HACU-Institute for the International Education of Students (IES) Scholarship Program: \url{https://www.iesabroad.org/IES/home.html}
			\item Hispanic Study Abroad Scholars program: \url{http://www.studyabroadscholars.org/index.html}
			\end{enumerate}
		\item Scholarship Resource List: \url{http://www.hacu.net/hacu/Scholarship_Resource_List_EN.asp?SnID=1109551622}
%		\item Scholarship Resource List: \url{http://www.hacu.net/hacu/Scholarship_Resource_List_EN.asp?SnID=1942709283}		-- Redundant
		\end{enumerate}
	\end{enumerate}
\item Congressional Hispanic Caucus Institute (CHCI): \vspace{-0.3cm}
	\begin{enumerate} \itemsep -2pt
	\item CHCI Scholarship: \vspace{-0.2cm}
		\begin{enumerate} \itemsep -2pt
		\item \url{http://www.chci.org/scholarships/}
		\item CHCI's scholarship opportunities are afforded to Latino students in the United States who have a history of performing public service-oriented activities in their communities and who demonstrate a desire to continue their civic engagement in the future. There is no GPA or academic major requirement. Students with excellent leadership potential are encouraged to apply.
		\item Scholarship awards are intended to provide assistance with tuition, room and board, textbooks, and other educational expenses associated with college enrollment.
		\item Students continue to receive annual disbursements as long as they maintain good academic standing.
		\item CHCI scholarships provide recipients with a one time scholarship of: \vspace{-0.1cm}
			\begin{enumerate} \itemsep -1pt
			\item \$1,000 community college or AA/AS granting institution
			\item \$2,500 4-year academic institution
			\item \$5,000 graduate-level institution
			\end{enumerate}
		\item Eligibility Criteria: \vspace{-0.1cm}
			\begin{enumerate} \itemsep -1pt
			\item Full-time enrollment in a United States Department of Education accredited community college, four-year university, or graduate/professional program during the period for which scholarship is requested
			\item Demonstrated financial need
			\item Consistent, active participation in public and/or community service activities
			\item Strong writing skills
			\item U.S. citizenship or legal permanent residency
			\end{enumerate}
		\end{enumerate}
	\item CHCI Fellowships: \vspace{-0.2cm}
		\begin{enumerate} \itemsep -2pt
		\item \url{http://www.chci.org/fellowships/}
		\item CHCI {\bf Public Policy} Fellowship: \vspace{-0.1cm}
			\begin{enumerate} \itemsep -1pt
			\item This is a paid Fellowship Program that offers talented Latinos, who have earned a bachelor's degree within two years of the program start date, the opportunity to gain hands-on experience at the national level in public policy.
			\item Fellows have the opportunity to work in congressional offices and federal agencies, depending on their area of interest.  Some past focus areas have included international affairs, economic development, health and education policy, housing, or local government.
			\item Program Dates: August to May (10-month internship)
			\item \url{http://www.chci.org/fellowships/page/chci-public-policy-fellowship}
			\end{enumerate}
		\item CHCI Graduate Fellowship Program: \vspace{-0.1cm}
			\begin{enumerate} \itemsep -1pt
			\item The CHCI Graduate Fellowship Program seeks to enhance participants' leadership abilities, strengthen professional skills and ultimately produce more competent and competitive Latino professionals in underserved {\bf public policy} issue areas.
			\item This paid Fellowship Program offers exceptional Latinos who have earned a graduate degree or higher related to a chosen policy issue area within three years of program start date unparalleled exposure to hands-on experience in public policy.
			\item This program focuses specifically on the areas of: \vspace{-0.1cm}
				\begin{itemize} \itemsep -1pt
				\item Higher Education: CHCI Graduate Higher Education Fellowship, \url{http://www.chci.org/fellowships/page/chci-graduate-higher-education-fellowship}
				\item Secondary Education: CHCI Graduate Secondary Education Fellowship, \url{http://www.chci.org/fellowships/page/chci-graduate-secondary-education-fellowship}
				\item Health: CHCI Graduate Health Fellowship, \url{http://www.chci.org/fellowships/page/chci-graduate-health-fellowship}
				\item Housing: CHCI Graduate Housing Fellowship, \url{http://www.chci.org/fellowships/page/chci-graduate-housing-fellowship}
				\item International Affairs (includes last three months abroad in Mexico): CHCI Graduate International Affairs Fellowship, \url{http://www.chci.org/fellowships/page/chci-graduate-international-affairs-fellowship}
				\item Law: CHCI Graduate Law Fellowship, \url{http://www.chci.org/fellowships/page/chci-graduate-law-fellowship}
				\item STEM (Science, Technology, Engineering and Math): CHCI Graduate STEM Fellowship, \url{http://www.chci.org/fellowships/page/chci-graduate-stem-fellowship}
				\end{itemize}
			\item Program Dates: August to May (10-month internship)
			\item \url{http://www.chci.org/fellowships/page/chci-graduate-fellowship-program}
			\end{enumerate}
		\end{enumerate}
	\end{enumerate}
\item American Indian Graduate Center (AIGC): \vspace{-0.3cm}
	\begin{enumerate} \itemsep -2pt
	\item AIGC scholarships and fellowships: \vspace{-0.2cm}
		\begin{enumerate} \itemsep -2pt
		\item for advanced degree students in art, music, environmental studies, journalism, communications, medicine, dentistry, public health, nursing, or other health-related fields
		\item for members of Wisconsin, New Mexico or Arizona tribes.
		\item \url{http://www.aigc.com/02scholarships/scholarships.htm}
		\item AIGC Fellowship (Graduate) for Native Americans and their descendants seeking advanced degrees: \url{http://www.aigc.com/02scholarships/aigc/fellowship.htm}
		\item Rainer Scholarship (for grad students): \url{http://www.aigc.com/02scholarships/rainer.htm}
		\end{enumerate}
	\item List of resources about scholarships and fellowships: \vspace{-0.2cm}
		\begin{enumerate} \itemsep -2pt
		\item \url{http://www.aigc.com/08otherscholarship/otherscholarships.html}
		\item Scholarships: \url{http://www.aigc.com/08otherscholarship/scholarships.htm}
		\item Fellowships: \url{http://www.aigc.com/08otherscholarship/fellowships.htm}
		\end{enumerate}
	\item Gates Millennium Scholar Program (for individuals seeking basic and advanced degrees): \url{http://www.aigc.com/03gms/gms.htm}
	\end{enumerate}
\item Asian \& Pacific Islander American Scholarship Fund (APIASF) scholarship resources: \url{http://www.apiasf.org/scholarships.html}
\item American Association of University Women: \vspace{-0.3cm}
	\begin{enumerate} \itemsep -2pt
	\item \url{http://www.aauw.org/learn/fellowships_grants/index.cfm}
	\end{enumerate}
\item Sigma Delta Epsilon-Graduate Women in Science (GWIS): \url{http://www.gwis.org/programs.html}
\item Society of Hispanic Professional Engineers (SHPE): \vspace{-0.3cm}
	\begin{enumerate} \itemsep -2pt
	\item Advancing Hispanic Excellence in Technology, Engineering, Math and Science (AHETEMS) Foundation: \url{http://www.ahetems.org/}
	\item AHETEMS Scholarship Program: \url{http://www.ahetems.org/scholarships/}
	\item Graduate \& Young Professional Fellowship Program (encourage young professionals to engage in {\bf public policy}): \url{http://www.ahetems.org/graduate/graduate-young-professional-fellowship-program/}
	\item SHPE/GEM Fellowship (for graduate students in STEM at GEM Member Universities): \url{http://www.ahetems.org/graduate/shpe-gem-graduate-award/}. See \url{http://www.gemfellowship.org/gem-universities/university-members} for a list of GEM member universities.
	\end{enumerate}
\item National Society of Black Engineers (NSBE): \vspace{-0.3cm}
	\begin{enumerate} \itemsep -2pt
	\item Scholarships: \url{http://www.nsbe.org/Programs/Scholarships.aspx}
	\end{enumerate}
\item The Society of Mexican American Engineers and Scientists (MAES): \vspace{-0.3cm}
	\begin{enumerate} \itemsep -2pt
	\item Scholarships \& Awards: \url{http://www.maes-natl.org/index.php?meid=328}
	\item MAES Scholarship Program: \url{http://www.maes-natl.org/index.php?module=ContentExpress&func=display&ceid=518&meid=241}
	\end{enumerate}
\item SACNAS (Society for Advancement of Chicanos and Native Americans in Science): \vspace{-0.3cm}
	\begin{enumerate} \itemsep -2pt
	\item Scholarships: \url{http://www.sacnas.org/webadindex.cfm?webadcategory_id=7}
	\item Fellowships: \url{http://www.sacnas.org/webadIndex.cfm?webadcategory_id=5}
	\end{enumerate}
\item {\it Center for the Advancement of Hispanics in Science and Engineering Education} (CAHSEE): \vspace{-0.3cm}
	\begin{enumerate} \itemsep -2pt
	\item Scholarships: \url{http://www.cahsee.org/6resources/scholarships.asp.htm}
	\end{enumerate}
\item National Consortium for Graduate Degrees for Minorities in Engineering and Science, Inc.: \vspace{-0.3cm}
	\begin{enumerate} \itemsep -2pt
	\item National GEM Consortium: GEM Fellowship, \url{http://www.gemfellowship.org/gem-fellowship/application-requirements}
	\end{enumerate}
\item National Physical Science Consortium (NPSC): \vspace{-0.3cm}
	\begin{enumerate} \itemsep -2pt
	\item NPSC Graduate Fellowship: \url{http://www.npsc.org/}
	\end{enumerate}
\item Finch College Alumnae Association: \vspace{-0.3cm}
	\begin{enumerate} \itemsep -2pt
	\item The Finch College Alumnae Foundation Education Grant: \vspace{-0.2cm}
		\begin{enumerate} \itemsep -2pt
		\item \url{http://www.finchcollege.org/newscholarships.html}
		\item \url{http://www.finchcollege.org/newFinchGrantQandA.html}
		\item ``THE FINCH GRANT, an annual program where four community college women entering a four year college are awarded a grant of \$1500 which can be used toward any needs to completing college.  The selection is determined by a panel of college professors.''
		\end{enumerate}
	\end{enumerate}
\item : \url{}
\item : \url{}
\item : \url{}
\item : \url{}
\item : \url{}
\item \S\ref{phdandpostdocfellowships} has more information concerning scholarships and fellowships in the following areas: \vspace{-0.3cm}
	\begin{enumerate} \itemsep -2pt
	\item electronic design automation (EDA), and related areas of design automation: \vspace{-0.2cm}
		\begin{enumerate} \itemsep -2pt
		\item bio design automation (BDA)
		\item Lab-on-chip design (LoC) automation
		\item MEMS/NEMS design automation
		\end{enumerate}
	\item digital VLSI design
	\item analog and mixed-signal (AMS) VLSI design
	\item computer architecture
	\item parallel computing
	\item concurrent programming
	\item data mining
	\item theoretical computer science
	\end{enumerate}
\item Ph.D. dissertation awards: \vspace{-0.3cm}
	\begin{enumerate} \itemsep -2pt
	\item --- --- --- --- --- --- --- --- --- --- --- --- --- --- --- --- --- --- --- --- --- --- --- --- --- --- --- --- --- --- ---
	\item \colorbox{blue}{\bf Ph.D. Dissertation Awards for Computer Science}
	% Ph.D. Dissertation Awards for Computer Science
	\item ACM Doctoral Dissertation Award: \url{http://awards.acm.org/doctoral_dissertation/}
	\item ACM Outstanding Ph.D. Dissertation Award in Electronic Design Automation: \url{http://www.sigda.org/opda.html}
	\item EDAA Outstanding Dissertation Award (European Design and Automation Association, EDAA): \url{http://www.edaa.com/dissertation_award.html} and \url{http://www.esat.kuleuven.be/micas/EDAA-Award/index.php}
	\item EuroSys Roger Needham PhD Award (in the systems area): \vspace{-0.2cm}
		\begin{enumerate} \itemsep -2pt
		\item Areas in systems include: \vspace{-0.1cm}
			\begin{enumerate} \itemsep -1pt
			\item operating systems
			\item distributed systems
			\item real-time systems
			\item systems aspects of databases
			\item language runtimes
			\item \colorbox{yellow}{\bf embedded systems}
			\item computer networks
			\end{enumerate}
		\item \url{http://www.eurosys.org/phdprize/index.php}
		\end{enumerate}
	\item ACM SIGPLAN Outstanding Doctoral Dissertation Award: \url{http://www.sigplan.org/award-dissertation.htm}
	\item ACM SIGKDD Doctoral Disseration Award (in data mining and knowledge discovery): \url{http://www.sigkdd.org/awards_dissertation.php}
	\item ACM SIGMOD Jim Gray Doctoral Dissertation Award (in the database field): \url{http://www.sigmod.org/sigmod-awards/doctoral-dissertation-award}
	\item Special Interest Group of the ACM on Management Information Systems (SIGMIS): \vspace{-0.2cm}
		\begin{enumerate} \itemsep -2pt
		\item ACM SIGMIS Doctoral Dissertation Award Competition (at the International Conference on Information Systems, ICIS): \url{http://ai.arizona.edu/icis2009/program/dissertation.html} and \url{http://icis2010.aisnet.org/dissertation_award.htm}
		\end{enumerate}
	\item Association for Symbolic Logic: \vspace{-0.2cm}
		\begin{enumerate} \itemsep -2pt
		\item ``The Sacks Prize is awarded for the most outstanding doctoral dissertation in mathematical logic''.
		\item \url{http://www.aslonline.org/Sacks_nominations.html} and \url{http://www.aslonline.org/info-prizes.html}
		\end{enumerate}
	\item European Association for Computer Science Logic (EACSL): \vspace{-0.2cm}
		\begin{enumerate} \itemsep -2pt
		\item Ackermann Award (for outstanding dissertations in Logic in Computer Science): \url{http://www.eacsl.org/} and \url{http://www.eacsl.org/award.html}
		\end{enumerate}
	\item European Coordinating Committee for Artificial Intelligence (ECCAI): \vspace{-0.2cm}
		\begin{enumerate} \itemsep -2pt
		\item 201X Artificial Intelligence Dissertation Award: \url{http://www.eccai.org/diss-award/current.shtml}
		\end{enumerate}
	\item European Conference on Wireless Sensor Networks (EWSN 201X, \url{http://www.nes.uni-due.de/ewsn2011}) and CONET, the Cooperating Objects Network of Excellence: Ph.D. Thesis Award Competition, \url{http://www.cooperating-objects.eu/}
	\item --- --- --- --- --- --- --- --- --- --- --- --- --- --- --- --- --- --- --- --- --- --- --- --- --- --- --- --- --- --- ---
	\item \colorbox{blue}{\bf Ph.D. Dissertation Awards for Mathematics}
	% Ph.D. Dissertation Awards for Mathematics
	\item International Center for Scientific Research (CIRS): \vspace{-0.2cm}
		\begin{enumerate} \itemsep -2pt
		\item E. W. Beth Dissertation Prize (for outstanding dissertations in the fields of Logic, Language and Information): \url{http://www.cirs.net/prix/awards.php?id=481}
		\end{enumerate}
	\item The Association for Operations Management, APICS (Advancing Productivity, Innovation, and Competitive Success): \vspace{-0.2cm}
		\begin{enumerate} \itemsep -2pt
		\item Plossl Doctoral Dissertation Competition: The APICS Educational and Research Foundation, will annually grant one award of \$2,500 for a doctoral dissertation dealing with any topic in operations management. Sample topics include operations strategy, operations planning and control systems, supply chain management, quality management, Six Sigma, facility location, forecasting, just-in-time/lean production systems, and project management. Entrants must be candidates for the doctorate in operations management. The dissertation must be approved by the primary thesis advisor and not more than 50\% completed at time of submission. See \url{http://www.apics.org/Education/ERFoundation/Competitions/plossl.htm}.
		\end{enumerate}
	\item SIAM Richard C. DiPrima Prize: \vspace{-0.2cm}
		\begin{enumerate} \itemsep -2pt
		\item The Richard C. DiPrima Prize is awarded every two years to a junior scientist, based on an outstanding doctoral dissertation in applied mathematics.
		\item \url{http://www.siam.org/prizes/nominations/nom_diprima.php}
		\item \url{http://www.siam.org/prizes/sponsored/diprima.php}
		\end{enumerate}
	\item MOS A.W. Tucker Prize: \vspace{-0.2cm}
		\begin{enumerate} \itemsep -2pt
		\item It is awarded for an outstanding doctoral thesis in any aspect of mathematical optimization.
		\item \url{http://www.mathprog.org/?nav=tucker}
		\end{enumerate}
	\item --- --- --- --- --- --- --- --- --- --- --- --- --- --- --- --- --- --- --- --- --- --- --- --- --- --- --- --- --- --- ---
	\item \colorbox{blue}{\bf Other Ph.D. Dissertation Awards}
	% Other Ph.D. Dissertation Awards
	\item Institute for Operations Research and the Management Sciences (INFORMS): \vspace{-0.2cm}
		\begin{enumerate} \itemsep -2pt
		\item INFORMS George B. Dantzig Dissertation Award: \url{http://www.informs.org/Recognize-Excellence/INFORMS-Prizes-Awards/George-B.-Dantzig-Dissertation-Award}
		\item Best Dissertation Award (Technology Management Section, for Ph.D. theses in technology management): \url{http://www.informs.org/Recognize-Excellence/INFORMS-Community-Prizes-and-Awards2/Technology-Management-Section/Best-Dissertation-Award}
		\item TSL Dissertation Prize (Transportation Science and Logistics Section, for doctoral dissertations in the transportation science and logistics area): \url{http://www.informs.org/Recognize-Excellence/INFORMS-Community-Prizes-and-Awards2/Transportation-Science-and-Logistics-Section/TSL-Dissertation-Prize}
		\item Best Dissertation Award (Telecommunications Section, for Ph.D. theses in telecommunications): \url{http://www.informs.org/Recognize-Excellence/INFORMS-Community-Prizes-and-Awards2/Telecommunications-Section/Best-Dissertation-Award}
		\item Frank M. Bass Dissertation Paper Award (Society for Marketing Science, for the best marketing paper derived from a Ph.D. thesis published in an INFORMS-sponsored journal): \url{http://www.informs.org/Recognize-Excellence/INFORMS-Community-Prizes-and-Awards2/Society-for-Marketing-Science/Frank-M.-Bass-Dissertation-Paper-Award}
		\item SOLA - Air Products Bi-Annual Dissertation Award (Section on Location Analysis, for Ph.D. theses on location related research): \url{http://www.informs.org/Recognize-Excellence/INFORMS-Community-Prizes-and-Awards2/Section-on-Location-Analysis/SOLA-Air-Products-Bi-Annual-Dissertation-Award}
		\end{enumerate}
	\item EURO Doctoral Dissertation Award (EDDA) (in operations research): \url{http://www.euro-online.org/display.php?page=edda1}
	\end{enumerate}
\item Other awards: \vspace{-0.3cm}
	\begin{itemize} \itemsep -2pt
	\item --- --- --- --- --- --- --- --- --- --- --- --- --- --- --- --- --- --- --- --- --- --- --- --- --- --- --- --- --- --- ---
	\item \colorbox{blue}{\bf Awards for Computer Science}
	% Awards for Computer Science
	\item ACM SIGMOD Undergraduate Award: \url{http://www.sigmod.org/sigmod-awards/sigmod-awards#undergraduate}
	\item European Association of Theoretical Computer Science (EATCS): Presburger Award (for young researchers in theoretical computer science), \url{http://www.eatcs.org/index.php/presburger}.
	\item Computer Research Association: \vspace{-0.2cm}
		\begin{enumerate} \itemsep -2pt
		\item Committee on the Status of Women in Computing Research (CRA-W): \vspace{-0.1cm}
			\begin{enumerate} \itemsep -1pt
			\item Borg Early Career Award (BECA): \url{http://www.cra-w.org/borg}
			\end{enumerate}
		\end{enumerate}
	\item European Conference on Wireless Sensor Networks (EWSN 201X, \url{http://www.nes.uni-due.de/ewsn2011}) and CONET, the Cooperating Objects Network of Excellence: Ph.D. Thesis Award Competition, \url{http://www.cooperating-objects.eu/}. ``Cooperating Objects combine the strong functional aspects of embedded systems, pervasive computing and wireless sensor networks. Cooperating objects entities federate themselves into dynamic and loose networks in order to reach a common goal. This common goal will typically be related to sensing or control.''
	\item --- --- --- --- --- --- --- --- --- --- --- --- --- --- --- --- --- --- --- --- --- --- --- --- --- --- --- --- --- --- ---
	\item \colorbox{blue}{\bf Awards for Biomedical Engineering}
	% Awards for Biomedical Engineering
	\item Biomedical Engineering Society (BMES): \vspace{-0.2cm}
		\begin{enumerate} \itemsep -2pt
		\item Rita Schaffer Young Investigator Award (for junior researchers in biomedical engineering): \url{http://www.bmes.org/aws/BMES/pt/sp/awards_investigator}
		\item Graduate and Undergraduate Student Awards: \url{http://www.bmes.org/aws/BMES/pt/sp/awards_student}
		\end{enumerate}
	\item --- --- --- --- --- --- --- --- --- --- --- --- --- --- --- --- --- --- --- --- --- --- --- --- --- --- --- --- --- --- ---
	\item \colorbox{blue}{\bf Awards for Mechanical Engineering}
	% Awards for Mechanical Engineering
	\item American Society of Mechanical Engineers (ASME): \vspace{-0.2cm}
		\begin{enumerate} \itemsep -2pt
		\item Henry Hess Award (authors of research papers who are below 31 years old): \url{http://www.asme.org/Governance/Honors/SocietyAwards/Henry_Hess_Award.cfm}
		\item Pi Tau Sigma Gold Medal (outstanding junior engineers): \url{http://www.asme.org/Governance/Honors/SocietyAwards/Pi_Tau_Sigma_Gold_Medal.cfm}
		\item Marshall B. Peterson Award (researchers in tribology who are below 30 years old): \url{http://www.asme.org/Governance/Honors/SocietyAwards/Marshall_B_Peterson_Award.cfm}
		\item Y.C. Fung Young Investigator Award (for young researchers in bioengineering): \url{http://www.asme.org/Governance/Honors/SocietyAwards/YC_Fung_Young_Investigator.cfm}
		\end{enumerate}
	\item --- --- --- --- --- --- --- --- --- --- --- --- --- --- --- --- --- --- --- --- --- --- --- --- --- --- --- --- --- --- ---
	\item \colorbox{blue}{\bf Awards for Civil Engineering}
	% Awards for Civil Engineering
	\item American Society of Civil Engineers (ASCE): \vspace{-0.3cm}
		\begin{enumerate} \itemsep -2pt
		\item Edmund Friedman Young Engineer Award for Professional Achievement (for junior engineers under the age of 36): \url{http://www.asce.org/AwardsContent.aspx?id=16776}
		\item Committee on Younger Members (CYM) Awards (for junior engineers): \url{http://www.asce.org/Content.aspx?id=11311}
		\item Collingwood Prize (for civil engineering researchers under the age of 35): \url{http://www.asce.org/AwardsContent.aspx?id=15352}
		\end{enumerate}
	\item --- --- --- --- --- --- --- --- --- --- --- --- --- --- --- --- --- --- --- --- --- --- --- --- --- --- --- --- --- --- ---
	\item \colorbox{blue}{\bf Awards for Chemical Engineering}
	% Awards for Chemical Engineering
	\item American Institute of Chemical Engineers (AIChE) awards: \url{http://www.aiche.org/Students/Awards/index.aspx}
	\item --- --- --- --- --- --- --- --- --- --- --- --- --- --- --- --- --- --- --- --- --- --- --- --- --- --- --- --- --- --- ---
	\item \colorbox{blue}{\bf Awards for Systems Engineering}
	% Awards for Systems Engineering
	\item International Council on Systems Engineering (INCOSE) Stevens Doctoral Award (for Promising Research in Systems Engineering and Integration; A.B.D.s / Ph.D. candidates): \url{http://www.incose.org/about/foundation/doctoralaward.aspx}
	\item --- --- --- --- --- --- --- --- --- --- --- --- --- --- --- --- --- --- --- --- --- --- --- --- --- --- --- --- --- --- ---
	\item \colorbox{blue}{\bf Awards for Mathematics, Operations Research, \& Management Sciences}
	% Awards for Mathematics, Operations Research, and Management Sciences
	\item Institute for Operations Research and the Management Sciences (INFORMS): \vspace{-0.2cm}
		\begin{enumerate} \itemsep -2pt
		\item INFORMS Undergraduate Operations Research Prize: \url{http://www.informs.org/Recognize-Excellence/INFORMS-Prizes-Awards/INFORMS-Undergraduate-Operations-Research-Prize}
		\item Optimization Prize for Young Researchers: \url{http://www.informs.org/Recognize-Excellence/INFORMS-Community-Prizes-and-Awards2/Optimization-Society/Optimization-Prize-for-Young-Researchers}
		\item Underrepresented Minorities and Women Honoraria: \url{http://www.informs.org/Recognize-Excellence/INFORMS-Community-Prizes-and-Awards2/Simulation-Society/Underrepresented-Minorities-and-Women-Honoraria}
		\item Best Dissertation Proposal Competition (College on Organization Science, for Ph.D. proposals in organizational science): \url{http://www.informs.org/Recognize-Excellence/INFORMS-Community-Prizes-and-Awards2/College-on-Organization-Science/Best-Dissertation-Proposal-Competition}
		\item ISMS Doctoral Dissertation Proposal Competition (Society for Marketing Science, for Ph.D. proposals in marketing): \url{http://www.informs.org/Recognize-Excellence/INFORMS-Community-Prizes-and-Awards2/Society-for-Marketing-Science/ISMS-Doctoral-Dissertation-Proposal-Competition}
		\end{enumerate}
	\item Alice T. Schafer Mathematics Prize For Excellence in Mathematics by an Undergraduate Woman: \url{http://www.awm-math.org/schaferprize.html}
	\item European Prize in Combinatorics: \vspace{-0.2cm}
		\begin{enumerate} \itemsep -2pt
		\item The prize is established to recognize excellent contributions in Combinatorics by young European researchers (eligibility of EU) not older than 35. 
		\item \url{http://www.math.tu-berlin.de/EuroComb05/prize.html}
		\end{enumerate}
	\item The AMS-MAA-SIAM Frank and Brennie Morgan Prize for Outstanding Research in Mathematics by an Undergraduate Student: \url{http://www.maa.org/awards/morgan.html}; \url{http://www.ams.org/profession/prizes-awards/ams-prizes/morgan-prize}; and \url{http://www.siam.org/prizes/sponsored/morgan.php}
	\item --- --- --- --- --- --- --- --- --- --- --- --- --- --- --- --- --- --- --- --- --- --- --- --- --- --- --- --- --- --- ---
	% Lists of awards
	\item \colorbox{blue}{\bf Lists of awards}: \vspace{-0.2cm}
		\begin{enumerate} \itemsep -2pt
		\item Association for Women in Science: \url{http://www.awis.org/displaycommon.cfm?an=1&subarticlenbr=69}
		\item International Center for Scientific Research (CIRS): \url{http://www.cirs.net/indexenglish.htm}
		\end{enumerate}
	\end{itemize}
\end{enumerate}














%%%%%%%%%%%%%%%%%%%%%%%%%%%%%%%%%%%%%%%%%%%
\section{Funding Nonprofit Organizations}
\label{fundingnonprofitorg}

Funding nonprofit organizations (including colleges and universities): \vspace{-0.3cm}
\begin{enumerate} \itemsep -4pt
\item Alfred P. Sloan Foundation: \vspace{-0.3cm}
	\begin{enumerate} \itemsep -2pt
	\item Major Program Areas: \url{http://www.sloan.org/program/1}
	\item Apply for Grants: \url{http://www.sloan.org/apply}
	\end{enumerate}
\item The Commonwealth Fund: \vspace{-0.3cm}
	\begin{enumerate} \itemsep -2pt
	\item Grants \& Programs: \vspace{-0.2cm}
		\begin{enumerate} \itemsep -2pt
		\item \url{http://www.commonwealthfund.org/Grants-and-Programs.aspx}
		\item ``The Fund supports independent research on health and social issues and makes grants to improve health care practice and policy. We are dedicated to helping people become more informed about their health care and improving care for vulnerable populations such as children, the elderly, low-income families, minorities, and the uninsured.''
		\end{enumerate}
	\end{enumerate}
\item The Heinz Endowments (Howard Heinz Endowment \& Vira I. Heinz Endowment): \vspace{-0.3cm}
	\begin{enumerate} \itemsep -2pt
	\item \url{http://www.heinz.org/grants.aspx}
	\item grant-making programs (for non-profit organizations): \vspace{-0.2cm}
		\begin{enumerate} \itemsep -2pt
		\item Arts \& Culture
		\item Children, Youth \& Families
		\item Education
		\item Environment
		\item Innovation Economy
		\end{enumerate}
	\end{enumerate}
\item Ford Foundation: \vspace{-0.3cm}
	\begin{enumerate} \itemsep -2pt
	\item Grants: \vspace{-0.2cm}
		\begin{enumerate} \itemsep -2pt
		\item \url{http://www.fordfoundation.org/grants/}
		\item Individuals Seeking Fellowships: \vspace{-0.1cm}
			\begin{enumerate} \itemsep -1pt
			\item \url{http://www.fordfoundation.org/grants/individuals-seeking-fellowships}
			\item Ford Foundation Fellowship Programs: \url{http://sites.nationalacademies.org/PGA/FordFellowships/index.htm}
			\item Ford Foundation International Fellowships Program: \url{http://www.fordifp.net/}
			\end{enumerate}
		\item Organizations Seeking Grants: \url{http://www.fordfoundation.org/grants/organizations-seeking-grants}
		\item Other Philanthropic Resources: \url{http://www.fordfoundation.org/grants/other-philanthropic-resources}
		\item Grant Search Results (list of grants): \url{http://www.fordfoundation.org/grants/search}
		\end{enumerate}
	\end{enumerate}
\item The Rockefeller Foundation: \vspace{-0.3cm}
	\begin{enumerate} \itemsep -2pt
	\item Grants \& Grantees: \vspace{-0.2cm}
		\begin{enumerate} \itemsep -2pt
		\item \url{http://www.rockefellerfoundation.org/grants}
		\item What We Fund: \url{http://www.rockefellerfoundation.org/grants/what-we-fund}
		\item Resources for Grantseekers: Links to other Philanthropic Sources, \url{http://www.rockefellerfoundation.org/grants/resources-grantseekers}
		\end{enumerate}
	\end{enumerate}
\item Carnegie Corporation of New York: \vspace{-0.3cm}
	\begin{enumerate} \itemsep -2pt
	\item Grantseekers: \vspace{-0.2cm}
		\begin{enumerate} \itemsep -2pt
		\item \url{http://carnegie.org/grants/grantseekers/}
		\item What we fund: \url{http://carnegie.org/grants/grantseekers/what-we-fund/}
		\item What we don't fund: \url{http://carnegie.org/grants/grantseekers/what-we-dont-fund/}
		\end{enumerate}
		\item Grants database: \url{http://carnegie.org/grants/grants-database/} and \url{http://carnegie.org/grants/}
		\item (Past) individual foundation grants: \url{http://carnegie.org/publications/carnegie-reporter/single/view/article/item/221/}
	\end{enumerate}
\item The Kresge Foundation: \vspace{-0.3cm}
	\begin{enumerate} \itemsep -2pt
	\item fields of interest: \vspace{-0.2cm}
		\begin{enumerate} \itemsep -2pt
		\item health,
		\item the environment,
		\item community development,
		\item arts and culture,
		\item education, and
		\item human services
		\end{enumerate}
	\item Values Criteria (for grantmaking): \url{http://www.kresge.org/index.php/who/our_values_criteria/}
	\item funding methods: \vspace{-0.2cm}
		\begin{enumerate} \itemsep -2pt
		\item \url{http://www.kresge.org/index.php/how/index/}
		\item \url{http://www.kresge.org/index.php/our_funding_methods/index/}
		\end{enumerate}
	\item Challenge Grant: \vspace{-0.2cm}
		\begin{enumerate} \itemsep -2pt
		\item \url{http://www.kresge.org/index.php/our_funding_methods/challenge_grant_program/}
		\item ``The Kresge Foundation awards facilities capital as a challenge grant to help nonprofit organizations build their base of private financial support as they conduct capital campaigns to build or renovate their facilities.''
		\item ``Facilities capital challenge grants are awarded to organizations that cater specifically to the needs of poor, disadvantaged and disenfranchised in six program areas: Health Program, the Environment Program, Arts and Culture Program, Education Program, Human Services Program, and Community Development / Detroit Program.''
		\item ``Most challenge grant awards are made to U.S.-based organizations. On rare occasions, we award challenge grants to international organizations undertaking exceptional projects that align with the strategic objectives of a given program and advance Kresge's values.''
		\end{enumerate}
	\item Detroit Program: \vspace{-0.2cm}
		\begin{enumerate} \itemsep -2pt
		\item Kresge Arts Support: \url{http://www.kresge.org/index.php/what/detroit_program/kresge_arts_support/}
		\item Kresge Arts in Detroit: \url{http://www.kresge.org/index.php/what/detroit_program/kresge_arts_in_detroit/}
		\end{enumerate}
	\item Our Grants: \vspace{-0.2cm}
		\begin{enumerate} \itemsep -2pt
		\item \url{http://www.kresge.org/index.php/our_grants/index/}
		\item grants database: \url{http://maps.foundationcenter.org/grantmakers/index.php?gmkey=KRES002}
		\item Arts and Community Building: \vspace{-0.1cm}
			\begin{enumerate} \itemsep -1pt
			\item \url{http://www.kresge.org/index.php/what/arts_and_culture/arts_and_community_building#Community Arts}
			\item ``Cultural institutions and artists animate our communities, bring disparate people together to share common experiences, and help us imagine a better future. As the demographics of our communities become more diverse, artists and cultural institutions help us bridge differences and build cross-cultural understanding. As our economy struggles, creative enterprises and creative sector leaders offer hope for community renewal and new job development.''
			\item two pilot initiatives: College/Arts initiative, and the Community Arts initiative
			\item ``The pilot cities [for the Community Arts initiative] include Baltimore, Maryland; Birmingham, Alabama; Detroit, Michigan; St. Louis, Missouri; and Tucson, Arizona.''
			\item ``Grants for Arts and Community Building are by invitation only.''
			\end{enumerate}
		\end{enumerate}
	\end{enumerate}
\item New York Women's Foundation: \vspace{-0.3cm}
	\begin{enumerate} \itemsep -2pt
	\item Grant Information and Application: \vspace{-0.2cm}
		\begin{enumerate} \itemsep -2pt
		\item \url{http://www.nywf.org/grant.html}
		\item focus areas: \vspace{-0.1cm}
			\begin{enumerate} \itemsep -1pt
			\item Anti-Violence and Safety
			\item Economic Security
			\item Health, Sexual Rights and Reproductive Justice
			\end{enumerate}
		\item ``Grants usually range from \$50,000 to a maximum of \$70,000 [that last for a year, and can be renewed up to 5 years].''
		\end{enumerate}
	\end{enumerate}
\item The Foundation Center: \vspace{-0.3cm}
	\begin{enumerate} \itemsep -2pt
	\item Grantseekers: \url{http://foundationcenter.org/getstarted/}
	\item Find funders: \url{http://foundationcenter.org/findfunders/}
	\item GrantSpace$^{\rm SM}$: \vspace{-0.2cm}
		\begin{enumerate} \itemsep -2pt
		\item \url{http://grantspace.org/}
		\item ``GrantSpace$^{\rm SM}$ will help you gain the knowledge and skills you need to get grants, manage your nonprofit, and improve your community.''
		\item ``Established in 1956 and today supported by close to 550 foundations, the Foundation Center is a national nonprofit service organization recognized as the nation�s leading authority on organized philanthropy, connecting nonprofits and the grantmakers supporting them to tools they can use and information they can trust. Its audiences include grantseekers, grantmakers, researchers, policymakers, the media, and the general public. The Center maintains the most comprehensive database on U.S. grantmakers and their grants; issues a wide variety of print, electronic, and online information resources; conducts and publishes research on trends in foundation growth, giving, and practice; and offers an array of free and affordable educational programs.''
		\item Resources for Non-U.S. Grantseekers: \url{http://grantspace.org/Tools/Knowledge-Base/Resources-for-Non-U.S.-Grantseekers}
		\item Resources for Individual Grantseekers: \vspace{-0.1cm}
			\begin{enumerate} \itemsep -1pt
			\item \url{http://grantspace.org/Tools/Knowledge-Base/Individual-Grantseekers}
			\item \url{http://gtionline.foundationcenter.org/}
			\item General: \url{http://grantspace.org/Tools/Knowledge-Base/Individual-Grantseekers/General}
			\item Artists: \url{http://grantspace.org/Tools/Knowledge-Base/Individual-Grantseekers/Artists}
			\item Students: \url{http://grantspace.org/Tools/Knowledge-Base/Individual-Grantseekers/Students}
			\item Fiscal Sponsorship: \url{http://grantspace.org/Tools/Knowledge-Base/Individual-Grantseekers/Fiscal-Sponsorship}
			\item For-Profit Enterprises: \url{http://grantspace.org/Tools/Knowledge-Base/Individual-Grantseekers/For-Profit-Enterprises}
			\end{enumerate}
		\end{enumerate}
	\end{enumerate}
\item The Lemelson Foundation: \vspace{-0.3cm}
	\begin{enumerate} \itemsep -2pt
	\item \url{http://web.mit.edu/invent/w-foundation.html}
	\item Programs \& Grants: \url{http://www.lemelson.org/programs-grants}
	\item Grantmaking: \url{http://www.lemelson.org/grantmaking}
	\end{enumerate}
\item Partnership for Higher Education in Africa (PHEA): \vspace{-0.3cm}
	\begin{enumerate} \itemsep -2pt
	\item \url{http://www.foundation-partnership.org/} and \url{http://www.foundation-partnership.org/index.php?id=1}
	\item Grants Database: \url{http://www.foundation-partnership.org/index.php?id=2}
	\item Partnership Publications: \url{http://www.foundation-partnership.org/index.php?id=3}
	\end{enumerate}
\item Smithsonian Institution: \vspace{-0.3cm}
	\begin{enumerate} \itemsep -2pt
	\item Smithsonian Institution Traveling Exhibition Service (SITES): \vspace{-0.2cm}
		\begin{enumerate} \itemsep -2pt
		\item Smithsonian Community Grant program (supported by MetLife Foundation): \vspace{-0.1cm}
			\begin{enumerate} \itemsep -1pt
			\item \url{http://www.sites.si.edu/funding/grant2.htm}
			\item ``This program seeks to deepen connections between SITES' host venues and their communities by encouraging exhibitors to engage their local audiences in new and exciting ways while creating broader access to our exhibitions.''
			\item ``Under this new program, eligible SITES exhibitors may apply for up to \$5,000 for expenses related to public, educational programming produced in conjunction with a SITES exhibit. Exhibitors may choose to enhance current program offerings or to create a new program especially suited to the topic of the exhibition.''
			\end{enumerate}
		\end{enumerate}
	\end{enumerate}
\end{enumerate}
















%%%%%%%%%%%%%%%%%%%%%%%%%%%%%%%%%%%%%%%%%%%
\section{Technology-Related Public Policy}
\label{techpublicpolicy}

Resources for engagement in creating technology-related public policy: \vspace{-0.3cm}
\begin{enumerate} \itemsep -4pt
\item Yale Journal of Law \& Technology (YJOLT): \vspace{-0.3cm}
	\begin{enumerate} \itemsep -2pt
	\item \url{http://www.yjolt.org/}
	\item \url{http://wingenroth.org/}
	\end{enumerate}
\item ACM Public Policy Office: \vspace{-0.3cm}
	\begin{enumerate} \itemsep -2pt
	\item It represents ACM and its US Public Policy Council (USACM) on information technology policy issues that impact the computing field.
	\item It seeks to educate policymakers and the public about policies that will that foster innovations in computing and related disciplines in ways that benefit society.
	\item It also informs ACM's members and the public about policy developments through its weblog, Washington Update newsletter and articles in ACM publications.
	\item ACM US Public Policy Council (USACM): \url{http://usacm.acm.org/}
	\item ACM Committee on Computers and Public Policy (CCPP): \url{http://www.acm.org/public-policy/acm-committee-on-computers-and-public-policy}
	\item \url{http://www.acm.org/public-policy}
	\end{enumerate}
\item IEEE: \vspace{-0.3cm}
	\begin{enumerate} \itemsep -2pt
	\item IEEE-USA: \url{http://www.ieeeusa.org/policy/default.asp}
	\item Smart Grids: \url{http://smartgrid.ieee.org/public-policy}
	\end{enumerate}
\item Computing Community Consortium (CCC): \url{http://www.cra.org/ccc/}
\item Computing Research Association (CRA): \vspace{-0.3cm}
	\begin{enumerate} \itemsep -2pt
	\item \url{http://www.cra.org/}
	\item CRA Government Affairs: \url{http://www.cra.org/govaffairs/index.php}
	\end{enumerate}
\item EngineeringPolicy.org: \url{http://www.engineeringpolicy.org/}
\item Congressional Bi-Partisan Robotics Caucus: \url{http://www.roboticscaucus.org/}
\item Advisory Committee for the Congressional Research and Development $[$R\&D$]$ Caucus: \url{http://www.researchcaucus.org/}
\item {\it National Academies Press} (NAP), from the (US) {\it National Academies}: \url{http://www.nap.edu/}
\item {\it Coalition to Diversify Computing}: \url{http://www.cdc-computing.org/}
\item American Institute of Aeronautics and Astronautics (AIAA): \vspace{-0.3cm}
	\begin{enumerate} \itemsep -2pt
	\item \url{http://www.aiaa.org/content.cfm?pageid=7}
	\end{enumerate}
\item : \url{}
\item : \url{}
\item : \url{}
\item : \url{}
\item : \url{}
\item : \url{}
\item : \url{}
\item : \url{}
\end{enumerate}






%%%%%%%%%%%%%%%%%%%%%%%%%%%%%%%%%%%%%%%%%%%
\section{Feminist Outreach}
\label{feministoutreach}

Feminist outreach: \vspace{-0.3cm}
\begin{enumerate} \itemsep -4pt
\item Myra Sadker Foundation: \vspace{-0.3cm}
	\begin{enumerate} \itemsep -2pt
	\item $100+$ Ideas to Promote Gender Equity in Schools and Beyond: \url{http://www.sadker.org/100ideas.html}
	\item Gender Equity Activities: \url{http://www.sadker.org/WhatYouCanDo.html}
	\item Gender Equity Activities for Concerned Citizens: \url{http://www.sadker.org/GenderEquity-citizens.html}
	\item Gender Equity Activities for Families: \url{http://www.sadker.org/GenderEquity-family.html}
	\item Gender Equity Activities for Teachers: \vspace{-0.2cm}
		\begin{enumerate} \itemsep -2pt
		\item Early Childhood: \url{http://www.sadker.org/GenderEquity-teacher1.html}
		\item Primary Grades: \url{http://www.sadker.org/GenderEquity-teacher2.html}
		\item Upper Elementary: \url{http://www.sadker.org/GenderEquity-teacher3.html}
		\item Middle and High School: \url{http://www.sadker.org/GenderEquity-teacher4.html}
		\end{enumerate}
	\item Resources for feminism and links to web pages of feminist organizations: \url{http://www.sadker.org/ReadsLinks.html}
	\end{enumerate}
\item Feminist student organizations at colleges and universities: \vspace{-0.3cm}
	\begin{enumerate} \itemsep -2pt
	\item For example, at the University of Southern California, the organizations associated with feminist causes are: \vspace{-0.2cm}
		\begin{enumerate} \itemsep -2pt
		\item {\it USC Center for Women \& Men}: \url{http://www.usc.edu/student-affairs/cwm/links.html}
		\item {\it USC Women's Student Assembly}: \url{http://www-scf.usc.edu/~wsausc/Welcome.html}
		\end{enumerate}
	\end{enumerate}
\item International Women's Day: \url{http://www.internationalwomensday.com/}
\item Gender Across Borders: \vspace{-0.3cm}
	\begin{enumerate} \itemsep -2pt
	\item Feminism Resources: \url{http://www.genderacrossborders.com/feminist-resources/}
	\end{enumerate}
\item {\it V-Day}: \vspace{-0.3cm}
	\begin{enumerate} \itemsep -2pt
	\item \url{http://www.vday.org/}
	\item Organization that helps women plan and organize events to bring awareness about sexual assault, and what we can do to reduce sexual assault.
	\end{enumerate}
\item {\it Take Back The Night}: \vspace{-0.3cm}
	\begin{enumerate} \itemsep -2pt
	\item \url{http://www.takebackthenight.org/}
	\item Organization that helps women plan and organize events to bring awareness about sexual assault, and what we can do to reduce sexual assault. It also encourages sexual assault survivors to speak out about their sexual assaults, so that they would shame their perpetrators and let other women (and men) know that they is nothing to be ashamed of as sexual assault survivors. This is because the faults lie 100\% with the perpetrators, and not with the survivors.
	\end{enumerate}
\item {\it United Nations Development Fund for Women} (UNIFEM): \vspace{-0.3cm}
	\begin{enumerate} \itemsep -2pt
	\item \url{http://www.unifem.org/}
	\item Organization that addresses many challenges faced by girls and women.
	\end{enumerate}
\item {\it National Organization for Women}: \vspace{-0.3cm}
	\begin{enumerate} \itemsep -2pt
	\item \url{http://www.now.org/}
	\item Feminist organization in the US.
	\end{enumerate}
\item {\it A Woman's Nation}: \vspace{-0.3cm}
	\begin{enumerate} \itemsep -2pt
	\item \url{http://www.shriverreport.com/awn/}
	\item \url{http://awomansnation.com} or \url{http://www.shriverreport.com/}
	\end{enumerate}
\item {\it Peace Over Violence} is a non-profit, feminist, multicultural, volunteer organization dedicated to a building healthy relationships, families and communities free from sexual, domestic and interpersonal violence: \url{http://peaceoverviolence.org/}
\item SoulSpeakOut: \url{http://www.soulspeakout.org/resources/}
\item {\it Haven Hills}: \url{http://havenhills.org/}
%\item MaleSurvivor: \url{http://www.malesurvivor.org/}
\end{enumerate}












%%%%%%%%%%%%%%%%%%%%%%%%%%%%%%%%%%%%%%%%%%%
\section{Outreach: Professional Organizations}
\label{outreachproorgs}

Professional organizations: \vspace{-0.3cm}
\begin{enumerate} \itemsep -4pt
\item --- --- --- --- --- --- --- --- --- --- --- --- --- --- --- --- --- --- --- --- --- --- --- --- --- --- --- --- --- --- ---
\item \colorbox{blue}{\bf Professional Organizations for the Performance, Literary, and Visual Arts}
% Professional Organizations for the Performance, Literary, and Visual Arts
\item Americans for the Arts: \vspace{-0.3cm}
	\begin{enumerate} \itemsep -2pt
	\item \url{http://www.americansforthearts.org/get_involved/membership/default.asp}
	\item \url{http://www.artsusa.org/get_involved/membership/default.asp}
	\item Provides membership for organizations and individuals
	\item Individual membership are available for: \vspace{-0.2cm}
		\begin{enumerate} \itemsep -2pt
		\item Students
		\item Entrepreneurs (e.g., people in art management)
		\item Innovators
		\item Colleagues (artists)
		\end{enumerate}
	\item Americans for the Arts {\bf Emerging Leader Program}: \vspace{-0.2cm}
		\begin{enumerate} \itemsep -2pt
		\item \url{http://www.artsusa.org/networks/emerging_leaders/resources/default.asp}
		\item Has various resources for professional development, including mentoring
		\end{enumerate}
	\item Advocacy ({\bf public policy}): \url{http://www.artsusa.org/get_involved/advocate.asp}
	\end{enumerate}
\item --- --- --- --- --- --- --- --- --- --- --- --- --- --- --- --- --- --- --- --- --- --- --- --- --- --- --- --- --- --- ---
\item \colorbox{blue}{\bf Professional Organizations for the Musical Artists}
% Professional Organizations for the Musical Artists
\item The Recording Academy: \url{http://www.grammy365.com/join/membership-types}
\end{enumerate}













%%%%%%%%%%%%%%%%%%%%%%%%%%%%%%%%%%%%%%%%%%%
\section{Other Outreach}
\label{otheroutreach}

Other outreach: \vspace{-0.3cm}
\begin{enumerate} \itemsep -4pt
\item The Joy McCann Foundation: \vspace{-0.3cm}
	\begin{enumerate} \itemsep -2pt
	\item The Joy McCann Professorships in Law: \url{http://www.mccannfoundation.org/law.htm}
	\end{enumerate}
\item National Academy of Sciences: \vspace{-0.3cm}
	\begin{enumerate} \itemsep -2pt
	\item {\it Science \& Entertainment Exchange} program: \vspace{-0.2cm}
		\begin{enumerate} \itemsep -2pt
		\item \url{http://www.scienceandentertainmentexchange.org/}
		\item Provide science and engineering knowledge to help professionals in the entertainment industry create engaging storylines involving science and technology.
		\end{enumerate}
	\end{enumerate}
\item U.S. Department of State: \vspace{-0.3cm}
	\begin{enumerate} \itemsep -2pt
	\item Bureau of Educational and Cultural Affairs: \vspace{-0.2cm}
		\begin{enumerate} \itemsep -2pt
		\item Programs: \url{http://exchanges.state.gov/jexchanges/programs.html}
		\item Fulbright Classroom Teacher Exchange Program: \vspace{-0.1cm}
			\begin{enumerate} \itemsep -1pt
			\item \url{http://exchanges.state.gov/globalexchanges/fulbright-teacher-exchange-program.html}
			\item ``The Fulbright Classroom Teacher Exchange provides opportunities for primary and secondary teachers to exchange positions with colleagues in other countries. The participants contribute to mutual understanding by bringing international knowledge and perspectives to the U.S. and foreign classrooms, schools and communities. Full-time U.S. teachers can take part in either a year-long or semester-long direct exchange with a counterpart in another country.''
			\end{enumerate}
		\item FORTUNE/U.S. State Department Global Women's Mentoring Partnership: \vspace{-0.1cm}
			\begin{enumerate} \itemsep -1pt
			\item \url{http://exchanges.state.gov/citizens/professionals/fortunepartnership.html}
			\item ``This public-private partnership places talented, emerging women leaders from all over the world in mentoring programs with FORTUNE's Most Powerful Women Leaders.''
			\end{enumerate}
		\item Edward R. Murrow Program for Journalists: \vspace{-0.1cm}
			\begin{enumerate} \itemsep -1pt
			\item \url{http://exchanges.state.gov/ivlp/murrow.html}
			\item ``The Edward R. Murrow Program for Journalists invites rising international journalists to travel to the United States and examine journalistic principles and practices.''
			\end{enumerate}
		\item International Visitor Leadership Program: \vspace{-0.1cm}
			\begin{enumerate} \itemsep -1pt
			\item \url{http://exchanges.state.gov/ivlp/ivlp.html}
			\item ``These visits reflect the International Visitors' professional interests and support the foreign policy goals of the United States.''
			\item ``International Visitors are current or emerging leaders in government, politics, the media, education, the arts, business and other key fields.''
			\item ``International Visitors travel to the U.S. for carefully designed programs that reflect their professional interests and U.S. foreign policy goals. They travel in a variety of thematic programs, either individually or in groups, for up to three weeks. While in the U.S., International Visitors typically visit Washington, DC and three additional towns or cities that highlight the tremendous diversity of the U.S. They attend professional appointments with their American counterparts, learn about the U.S. system of government at the national, state and local levels, visit American schools, and experience American culture and social life.''
			\item ``There is no application for this program. International Visitors are selected and nominated annually by American Foreign Service Officers at U.S. Embassies around the world.''
			\end{enumerate}
		\item Au Pair: \vspace{-0.1cm}
			\begin{enumerate} \itemsep -1pt
			\item \url{http://exchanges.state.gov/jexchanges/programs/aupair.html}
			\item ``Through the Au Pair program, foreign nationals between 18 and 26 years of age participate in the home life of a host family. Au pairs provide limited childcare services for up to 12 months. An extension of 6, 9, or 12 months may be granted in certain cases.''
			\end{enumerate}
		\item Summer Work Travel: \vspace{-0.1cm}
			\begin{enumerate} \itemsep -1pt
			\item \url{http://exchanges.state.gov/jexchanges/programs/swt.html}
			\item ``In the summer work travel program, post-secondary students may enter the United States to work and travel during their summer vacation. Participants can be admitted to the program more than once. The maximum length of the program is four months.''
			\end{enumerate}
		\item Internship: \vspace{-0.1cm}
			\begin{enumerate} \itemsep -1pt
			\item \url{http://exchanges.state.gov/jexchanges/programs/intern.html}
			\item ``Internship programs are designed to allow foreign professionals to come to the United States to gain exposure to U.S. culture and to receive training in U.S. business practices in their chosen occupational field.  The maximum duration of an internship in any occupational field is 12 months. Upon completion of their exchange programs, participants are expected to return to their home countries.''
			\end{enumerate}
		\item Professional Exchanges Division: \vspace{-0.1cm}
			\begin{enumerate} \itemsep -1pt
			\item \url{http://exchanges.state.gov/citizens/profs.html}
			\item ``The Professional Exchanges division provides grants to U.S. nonprofit organizations to carry out exchange programs that support the professional development of foreign participants. The purpose of each exchange program is to engage with foreign leaders in critical professions, to demonstrate respect for foreign cultures, and to promote mutual understanding between the people of the United States and other countries.''
			\item ``Professional exchanges typically last several years and include internships, study tours or workshops in the United States and in the host country. Participants come from a variety of professions including education administrators, public servants, journalists, labor union officials, entrepreneurs, environmental leaders, jurists, lawyers, and civic leaders.''
			\end{enumerate}
		\end{enumerate}
	\end{enumerate}
\item Teach For All: \url{http://teachforallnetwork.org/}
\item --- --- --- --- --- --- --- --- --- --- --- --- --- --- --- --- --- --- --- --- --- --- --- --- --- --- --- --- --- --- ---
\item \colorbox{blue}{\bf Resources for Artists and Musicians}
% Resources for Artists and Musicians
\item League of American Orchestras and the Association of Performing Arts Presenters: \vspace{-0.3cm}
	\begin{enumerate} \itemsep -2pt
	\item {\it ArtistsfromAbroad.org}: \vspace{-0.2cm}
		\begin{enumerate} \itemsep -2pt
		\item \url{http://www.artistsfromabroad.org/}
		\item ``{\it ArtistsfromAbroad.org} features complete and up-to-date guidance on the visa process and tax treatment for foreign guest artists.''
		\end{enumerate}
	\end{enumerate}
\item Young Concert Artists, Inc. \vspace{-0.3cm}
	\begin{enumerate} \itemsep -2pt
	\item Composer Program (for American composers between 20 and 26 years of age): \url{http://www.yca.org/auditions/}
	\end{enumerate}
\item The John F. Kennedy Center for the Performing Arts: \vspace{-0.3cm}
	\begin{enumerate} \itemsep -2pt
	\item Mary Lou Williams Women in Jazz Emerging Artist Workshop: \vspace{-0.2cm}
		\begin{enumerate} \itemsep -2pt
		\item \url{http://www.kennedy-center.org/programs/jazz/womeninjazz/competition.html}
		\item ``The workshop provides female jazz artists ages 18 to 35 with an opportunity to explore and develop their artistry under the guidance of leading jazz artists and instructors. Each year, the workshop will focus on a specific instrument.''
		\item ``The 2011 Mary Lou Williams Women in Jazz Emerging Artist Workshop is open to advanced female jazz pianists who plan to pursue jazz performance as a career. Eligibility is exclusive to pianists who will be 18-35 years old on May 18, 2011 and have never recorded or been contracted to record as a leader or co-leader on a major label at the time of application. All applicants must be proficient in English.''
		\end{enumerate}
	\end{enumerate}
\item Grantmakers in the Arts (GIA): \vspace{-0.3cm}
	\begin{enumerate} \itemsep -2pt
	\item ``The mission of Grantmakers in the Arts (GIA) is to provide leadership and service to advance the use of philanthropic resources on behalf of arts and culture.''
	\item Arts Funding Topics: \url{http://www.giarts.org/arts-funding-topics}
	\end{enumerate}
\item The Dana Foundation: \vspace{-0.3cm}
	\begin{enumerate} \itemsep -2pt
	\item Arts Education program: \vspace{-0.2cm}
		\begin{enumerate} \itemsep -2pt
		\item Arts Education Grants: \vspace{-0.1cm}
			\begin{enumerate} \itemsep -1pt
			\item \url{http://www.dana.org/grants/BrowseArtsGrants.aspx}
			\item ``In 2001, The Dana Foundation created the Arts Education program with a sole focus of providing grants to support professional development for teaching artists and in-school arts specialists. The first several years of grants were to  programs in New York City, Washington, DC, Los Angeles and to organizations with a 50 mile radius of the three.''
			\item ``The Rural Initiative launched in 2006 with 6 grants awarded to organizations providing professional development in rural areas of the United States.''
			\end{enumerate}
		\end{enumerate}
	\end{enumerate}
\item writing/poetry contests: \vspace{-0.3cm}
	\begin{enumerate} \itemsep -2pt
	\item International 3-Day Novel Contest: \url{http://www.3daynovel.com/about/?contest}
	\end{enumerate}
\end{enumerate}




%%%%%%%%%%%%%%%%%%%%%%%%%%%%%%%%%%%%%%%%%%%
\section{Christian Colleges and Universities}
\label{christianunis}

Christian colleges and universities: \vspace{-0.3cm}
\begin{enumerate} \itemsep -4pt
\item List of Christian colleges and universities: \vspace{-0.3cm}
	\begin{enumerate} \itemsep -2pt
	\item Council for Christian Colleges and Universities (CCCU): \vspace{-0.2cm}
		\begin{enumerate} \itemsep -2pt
		\item \url{http://en.wikipedia.org/wiki/Council_for_Christian_Colleges_and_Universities}
		\item \url{http://www.cccu.org/}
		\end{enumerate}
	\item Christian College Consortium: \vspace{-0.2cm}
		\begin{enumerate} \itemsep -2pt
		\item \url{http://en.wikipedia.org/wiki/Christian_College_Consortium}
		\item \url{http://www.ccconsortium.org/}
		\end{enumerate}
	\end{enumerate}
\item California Baptist University, Riverside
\item Messiah College (Grantham, PA): \vspace{-0.3cm}
	\begin{enumerate} \itemsep -2pt
	\item Department of Engineering: \vspace{-0.2cm}
		\begin{enumerate} \itemsep -2pt
		\item \url{http://www.messiah.edu/departments/engineering/}
		\item B.S. programs in: \vspace{-0.1cm}
			\begin{enumerate} \itemsep -1pt
			\item Biomedical Engineering
			\item Computer Engineering
			\item Electrical Engineering
			\end{enumerate}
		\end{enumerate}
	\item Department of Information and Mathematical Sciences: \vspace{-0.2cm}
		\begin{enumerate} \itemsep -2pt
		\item \url{http://www.messiah.edu/departments/mathsci/index.html}
		\item Offers a B.A. Computer Science program
		\end{enumerate}
	\end{enumerate}
\end{enumerate}
























%%%%%%%%%%%%%%%%%%%%%%%%%%%%%%%%%%%%%%%%%
% Thoughts and Resources for Specific Areas and Topics
%	% This is written by Zhiyang Ong for his management of information and tasks.
%
% It includes information on professional development, including membership of professional organizations and networking societies.





%%%%%%%%%%%%%%%%%%%%%%%%%%%%%%%%%%%%%%%%%%%
\section{Heuristic for Locating Outreach Resources}
\label{heuristiclocateoutreach}

\proc{Find}$(\varphi, \tau)$ is a heuristic for locating resources for outreach activities, which includes finding information about the following: \vspace{-0.3cm}
\begin{enumerate} \itemsep -4pt
\item awards
\item career resources (including material for career guidance)
\item competitions and contests
\item educational material (e.g., suggested activities and curricular) for specific areas, such as marine sciences and electrical/computer engineering
\item fellowships
\item internships
\item scholarships
\item summer camps
\item summer programs (or summer schools); here, summer schools refer to short educational programs that last from days (e.g., a weekend for the ACM SIGDA Design Automation Summer School) to about a month (e.g., Santa Fe Institute's Complex Systems Summer Schools)
\end{enumerate}
\ \\

Its input $\tau$ is the deadline by which this search process must terminate. For example, if I have to apply for internships by next week, I would use the date of a week from now as the deadline $\tau$. In line \ref{find-pt-professional-org}, an example of a professional organization is the Institute of Electrical and Electronics Engineers (IEEE). The term ``good'' that is used in line \ref{find-pt-gd-uni} is an arbitrary measure of quality determined by the reader/user. \\

A reading group (in line \ref{find-pt-reading-grp}) is a small group of (graduate) students, which may possibly include professors and postdocs, that meet regularly (e.g., once/twice a week) to discuss papers that they have read since the previous meeting/discussion. Each individual in the reading group can be assigned a paper to read and present at the next meeting. The aim of a reading group is to improve the coverage of papers in our research area that each member has read. This is important for interdisciplinary research, since grad students working in interdisciplinary research areas have so much ground to cover. \\

Line \ref{find-pt-athletics} uses the term ``athletics department'' to refer to an administrative department at an American college or university that is in charge of managing varsity/NCAA sports teams. An example of a profession-specific networking organization (line \ref{find-pt-netwk-org}) is DVClub. In line \ref{find-pt-domain-specific-www}, a domain-specific web page is {\it SAT Live!}. An example of a corporate research laboratory (line \ref{find-pt-corporate-research-labs}) is ``Cadence Research Laboratories'' (\url{http://www.cadence.com/cadence/cadence_labs/pages/default.aspx}), and an example of a research institute (line \ref{find-pt-research-institute}) is Santa Fe Institute.




\begin{codebox}
\Procname{$\proc{Find}(\varphi, \tau)$}
\zi	\Comment {\it Input $\varphi \gets $ Item to find out about}
\zi	\Comment {\it Input $\tau \gets $ Deadline for the search process}
\zi	\Comment {\it Output $\kappa \gets $ List of resources about $\varphi$}
\zi
\li \While ( [ resources about $\varphi$ are inadequate ] AND [ $\tau$ has not yet passed ] )
	\Do
\li	Find out the professional organizations for the field of $\varphi$	\label{find-pt-professional-org}
\li	\For each professional organization in the field
		\Do
\li		Check if it has information about $\varphi$ in its web pages, publications, or mailing list archive
\li		\If (it has information about $\varphi$)
			\Then
\li			Add that information to $\kappa$
			\End
		\End
\zi
\li	\For each good (college OR university)	\label{find-pt-gd-uni}
		\Do
\li		\If ($\varphi == $ summer programs )
			\Then
\li			Search for summer programs in the web pages of departments \& schools/colleges
\li		\ElseIf ($\varphi == $ summer camps )
			\Then
\li			Search for summer camps in the web pages of departments \& schools/colleges
\li			Search for summer camps in the web pages of administrative/athletics departments	\label{find-pt-athletics}
\li		\ElseNoIf
\li			Search for $\varphi$ in the web pages of the department(s), including its news section/archive
\li			Search for $\varphi$ in the web pages of professors, postdoctoral researchers, \& students
\li			Search for $\varphi$ in the web pages of reading groups		\label{find-pt-reading-grp}
\li			Search for $\varphi$ in the web pages of student organizations
\li			Search for $\varphi$ in the mailing list archive of classes \& the department
\li			Search for $\varphi$ in the mailing list archive of research groups/labs and projects
\li			Search for $\varphi$ in the mailing list archive of reading groups
\li			Search for $\varphi$ in the mailing list archive of student organizations
			\End
\zi
\li		\If (it has information about $\varphi$)
			\Then
\li			Add that information to $\kappa$
			\End
		\End
\zi
\li	Search for $\varphi$ in the mailing list archive of open-source projects
\li	Search for $\varphi$ in the mailing list archive of profession-specific networking organizations	\label{find-pt-netwk-org}
\li	Search for $\varphi$ in the web pages of domain-specific web pages	\label{find-pt-domain-specific-www}
\li	Search for $\varphi$ in the web pages of research scientists in corporate research labs	\label{find-pt-corporate-research-labs}
\li	Search for $\varphi$ in the web pages of research scientists in research institutes	\label{find-pt-research-institute}
\li	\If ( [ mailing list archive OR web page ] has information about $\varphi$)
		\Then
\li		Add that information to $\kappa$
		\End
	\End	
\li \Return $\kappa$
\end{codebox}




%%%%%%%%%%%%%%%%%%%%%%%%%%%%%%%%%%%%%%%%%%%
\section{General Outreach Resources}
\label{generaloutreachresources}

General outreach resources: \vspace{-0.3cm}
\begin{enumerate} \itemsep -4pt
\item volunteering opportunities: \vspace{-0.3cm}
	\begin{enumerate} \itemsep -2pt
	\item Engineers Without Borders: \url{http://www.ewb-international.org/}
	\item Australian Volunteers International: \url{http://www.australianvolunteers.com/}
	\item Youth Challenge Australia: \url{http://www.youthchallenge.com.au/}
	\item Go Volunteer: \url{http://www.govolunteer.com.au/}
	\item Volunteer Search: \url{http://www.volunteersearch.gov.au/}
	\item Conservation Volunteers: \url{http://www.conservationvolunteers.com.au/volunteer}
	\item Volunteering Australia: \url{http://www.volunteeringaustralia.org/html/s01_home/home.asp}
	\item Sponsors for Educational Opportunity (SEO): \vspace{-0.2cm}
		\begin{enumerate} \itemsep -2pt
		\item Philanthropy \& Volunteerism Resources, \url{http://www.seo-usa.org/AlumniResources}
		\item Volunteer Leadership Opportunities: \url{http://www.seo-usa.org/Alumni_Volunteer}
		\end{enumerate}
	\item : \url{}
	\end{enumerate}
\item public health and preventive medicine: \vspace{-0.3cm}
	\begin{enumerate} \itemsep -2pt
	\item U.S. Department of Health \& Human Services: \vspace{-0.2cm}
		\begin{enumerate} \itemsep -2pt
		\item Agency for Healthcare Research and Quality (AHRQ): \vspace{-0.1cm}
			\begin{enumerate} \itemsep -1pt
			\item Prevention \& Care Management: Resources and Materials, \url{http://www.ahrq.gov/clinic/ppipix.htm}
			\end{enumerate}
		\end{enumerate}
	\end{enumerate}
\item career resources: \vspace{-0.3cm}
	\begin{enumerate} \itemsep -2pt
	\item CRAC: The Career Development Organisation: \vspace{-0.2cm}
		\begin{enumerate} \itemsep -2pt
		\item {\it icould}: \vspace{-0.1cm}
			\begin{enumerate} \itemsep -1pt
			\item \url{http://icould.com/about/}
			\item Resource for students, people who are commencing their careers or are making changes in their careers, career counselors, parents, educators, human resource staff, and employers.
			\item icould, {\it Stories by Life Theme}, in icould: Watch Career Stories. Available online at: \url{http://icould.com/watch-career-stories/by-life-theme/}; last accessed on December 25, 2010. [ Has articles briefly describing how people pursued their career goals or their career paths as they went through different experiences in life. This includes people who ``blossomed after school,'' changed careers or became entrepreneurs, had no plans, took risks, encountered turning points, faced adversity, have disabilities, went through financial hardship, or got laid off. It also has stories of people who volunteered, took a gap year, or pursued internships. ]
			\item icould, {\it Stories by Job Type}, in icould: Watch Career Stories. Available online at: \url{http://icould.com/watch-career-stories/by-job-type/}; last accessed on December 25, 2010. [ Includes stories of people in automotive retail, customer services, engineering, education, and many other job types. ]
			\end{enumerate}
		\end{enumerate}
	\item Jobs for the Future: \vspace{-0.2cm}
		\begin{enumerate} \itemsep -2pt
		\item \url{http://www.jff.org/}
		\item Current Projects: \url{http://www.jff.org/projects/current}
		\item Publications: \url{http://www.jff.org/publications}
		\item Policy: \url{http://www.jff.org/policy}
		\item Funders (funding agencies/organizations): \url{http://www.jff.org/funders}
		\item Programs: \url{http://www.jff.org/index.php?select=work}
		\end{enumerate}
	\item SkillsUSA: \vspace{-0.2cm}
		\begin{enumerate} \itemsep -2pt
		\item ``SkillsUSA is a partnership of students, teachers and industry working together to ensure America has a skilled work force. SkillsUSA helps each student excel.''
		\item Educators: \vspace{-0.1cm}
			\begin{enumerate} \itemsep -1pt
			\item \url{http://www.skillsusa.org/educators/index.shtml}
			\item Programs and Curricula: \url{http://www.skillsusa.org/educators/programs.shtml}
			\end{enumerate}
		\item Students: \vspace{-0.1cm}
			\begin{enumerate} \itemsep -1pt
			\item \url{http://www.skillsusa.org/students/index.shtml}
			\item Scholarships \& Financial Aid--SkillsUSA-related Scholarships: \url{http://www.skillsusa.org/students/scholarships.shtml}
			\end{enumerate}
		\item SkillsUSA competitions: \url{http://www.skillsusa.org/compete/index.shtml}
		\end{enumerate}
	\item others: \vspace{-0.2cm}
		\begin{enumerate} \itemsep -2pt
		\item public speaking and leadership: \vspace{-0.1cm}
			\begin{enumerate} \itemsep -1pt
			\item {\it Toastmasters International} is a non-profit educational organization that teaches public speaking and leadership skills through a worldwide network of meeting locations. Available online at: \url{http://www.toastmasters.org/}; last accessed on January 7, 2010.
			\end{enumerate}
		\end{enumerate}
	\end{enumerate}
\end{enumerate}




%%%%%%%%%%%%%%%%%%%%%%%%%%%%%%%%%%%%%%%%%%%
\section{Youth Outreach}
\label{youthoutreach}

Resources for youth outreach: \vspace{-0.3cm}
\begin{enumerate} \itemsep -4pt
%%%%%%%%%%%%%%%%%%%%%%%
\item educational (computer) games: \vspace{-0.3cm}
	\begin{enumerate} \itemsep -2pt
	\item Chevron Corporation: \vspace{-0.2cm}
		\begin{enumerate} \itemsep -2pt
		\item Energyville (about issues concerning energy and the environment): \url{http://www.willyoujoinus.com/energyville/}
		\end{enumerate}
	\item {\it Lego Digital Designer (LDD)}: \vspace{-0.2cm}
		\begin{enumerate} \itemsep -2pt
		\item CAD software for building Lego toys on Windows and Mac OS X platforms
		\item Free software, as in free beer
		\item \url{http://designbyme.lego.com/en-us/Default.aspx} and \url{http://ldd.lego.com/}
		\end{enumerate}
	\item Robocode: \vspace{-0.2cm}
		\begin{enumerate} \itemsep -2pt
		\item \url{http://en.wikipedia.org/wiki/Robocode} and \url{http://robocode.sourceforge.net/}
		\item Learn how to develop computer programs that will control a robot
		\end{enumerate}
	\item {\it Skill-Life}: \vspace{-0.2cm}
		\begin{enumerate} \itemsep -2pt
		\item \url{http://skill-life.com/}
		\item Use online games to teach youth life skills concerning financial literacy, nutrition, and citizenship.
		\end{enumerate}
	\item PowerUp (IBM with TryScience/New York Hall of Science): \vspace{-0.2cm}
		\begin{enumerate} \itemsep -2pt
		\item \url{http://www.powerupthegame.org/}
		\item Computer game to teach youths about energy conservation, global warming, renewable energy, and sustainable engineering
		\end{enumerate}
	\item EnergyNet: \vspace{-0.2cm}
		\begin{enumerate} \itemsep -2pt
		\item \url{http://www.energynet.net/games/}
		\item Computer game to teach youths about energy efficiency, and other topics related to energy
		\end{enumerate}
	\end{enumerate}
%%%%%%%%%%%%%%%%%%%%%%%
\item summer camps: \vspace{-0.3cm}
	\begin{enumerate} \itemsep -2pt
	\item United States Naval Academy: \vspace{-0.2cm}
		\begin{enumerate} \itemsep -2pt
		\item Naval Academy Athletic Association: \vspace{-0.1cm}
			\begin{enumerate} \itemsep -1pt
			\item Sports camps: \url{http://www.navysports.com/camps/navy-camps.html}
			\end{enumerate}
		\end{enumerate}
	\end{enumerate}
%%%%%%%%%%%%%%%%%%%%%%%
\item competitions for youths: \vspace{-0.3cm}
	\begin{enumerate} \itemsep -2pt
	\item International Geography Olympiad (for high school students): \url{http://www.geoolympiad.org/}
	\item International Linguistic Olympiad (for high school students): \url{http://en.wikipedia.org/wiki/International_Linguistics_Olympiad}
	\item International Philosophy Olympiad (for high school students): \url{http://www.philosophy-olympiad.org/}
	\item JA Worldwide: Responsible People Business Competition (for students in North and South America, and Europe), \url{http://www.responsible-business.org/}
	\item The Choral Arts Society of Washington: \vspace{-0.2cm}
		\begin{enumerate} \itemsep -2pt
		\item \url{http://www.choralarts.org/MLK-Celebration-Community-Initiative/Writing-Competition.aspx}
		\item ``As part of our MLK Celebration Community Initiative and in celebration of Black History Month, The Choral Arts Society of Washington hosts an annual writing competition for students in grades K-12.''
		\item ``Each year, students are presented with a different writing prompt and are asked to respond in poetic form.''
		\item ``Students are encouraged to be creative in their writing and to use their knowledge of Martin Luther King, Jr.'s life, the Civil Rights Movement, and current events as inspiration for their writing.''
		\end{enumerate}
	\item Vocal Arts DC (or Vocal Arts Society): \vspace{-0.2cm}
		\begin{enumerate} \itemsep -2pt
		\item Young Artists Competition: \vspace{-0.1cm}
			\begin{enumerate} \itemsep -1pt
			\item \url{http://vocalartsdc.org/youngartists.shtml}
			\item ``Each year, Vocal Arts DC holds a vocal competition open to all singers who are residents of the greater DC area, including Baltimore and Annapolis.''
			\item ``Singers are asked to submit a CD for review along with a sample recital program that the singer is prepared to sing in recital. The CDs will be reviewed in a blind audition and finalist will be selected for live auditions.''
			\item ``Two winners are selected from the finalists and are presented in the Art Song Discovery Series in four different venues across the greater DC area.''
			\end{enumerate}
		\end{enumerate}
	\item The John F. Kennedy Center for the Performing Arts: \vspace{-0.2cm}
		\begin{enumerate} \itemsep -2pt
		\item The National Symphony Orchestra (NSO): \vspace{-0.1cm}
			\begin{enumerate} \itemsep -1pt
			\item Young Soloists' Competition (High School Division; Washington metropolitan area): \url{http://www.kennedy-center.org/nso/nsoed/youngsoloists.cfm#concerts}
			\end{enumerate}
		\end{enumerate}
	\item Center for Interactive Learning and Collaboration (CILC): \vspace{-0.2cm}
		\begin{enumerate} \itemsep -2pt
		\item Kids Creating Community Content KC$^{3}$ International Contest (for students in Middle and High School): \vspace{-0.1cm}
			\begin{enumerate} \itemsep -1pt
			\item \url{http://kc3.cilc.org/} and \url{http://kc3.cilc.org/guidelines.htm}
			\item Make a short film to educate others about the uniqueness of your community, geographical region, natural/agricultural resources, local/national treasures, culture/heritage, or country.
			\end{enumerate}
		\end{enumerate}
	\end{enumerate}
%%%%%%%%%%%%%%%%%%%%%%%
\item educational resources: \vspace{-0.3cm}
	\begin{itemize} \itemsep -2pt
	\item Xcel Energy Foundation: \vspace{-0.2cm}
		\begin{enumerate} \itemsep -2pt
		\item Focus Area Grants: \vspace{-0.1cm}
			\begin{enumerate} \itemsep -1pt
			\item \url{http://www.xcelenergy.com/Minnesota/Company/Community/Xcel%20Energy%20Foundation/Pages/Focus_Area_Grants.aspx}
			\item Scope of eligible funding, and details on the grant application process
			\end{enumerate}
		\item Education Initiatives: \vspace{-0.1cm}
			\begin{enumerate} \itemsep -1pt
			\item \url{http://www.xcelenergy.com/Minnesota/Company/Community/Education%20Initiatives/Pages/Education_Initiatives.aspx}
			\item Energy Safety Calendar Program, K-6: \vspace{-0.1cm}
				\begin{itemize} \itemsep -1pt
				\item \url{http://www.xcelenergy.com/New%20Mexico/Company/Community/Education%20Initiatives/Pages/Energy_Safety_Calendar_ProgramK-6.aspx}
				\item ``The Energy Safety Calendar Program offers K-6 students in our service territory a great opportunity to learn about electricity and natural gas safety.''
				\end{itemize}
			\end{enumerate}
		\item Safety World: \vspace{-0.1cm}
			\begin{enumerate} \itemsep -1pt
			\item \url{http://www.xcelenergy.com/New%20Mexico/Company/Community/Education%20Initiatives/Pages/Safety_World.aspx}
			\item e-SMART kid: \vspace{-0.1cm}
				\begin{itemize} \itemsep -1pt
				\item \url{http://www.e-smartonline.net/xcelenergy/}
				\item Help children and youth learn about ``electricity and natural gas and how to use them safely''
				\end{itemize}
			\end{enumerate}
		\item Energy Classroom: \vspace{-0.1cm}
			\begin{enumerate} \itemsep -1pt
			\item \url{http://www.energyclassroom.com/}
			\item \url{http://www.xcelenergy.com/Minnesota/Company/Community/Pages/Energy_Classroom.aspx}
			\item Educational material for students about energy sources, energy conservation, and environmental protection
			\item For Teachers (educational material and suggested class activities): \url{http://www.energyclassroom.com/index.php?id=34&page=For_Teachers}
			\end{enumerate}
		\item Power Plant Tour Information: \url{http://www.xcelenergy.com/New%20Mexico/Company/About_Energy_and_Rates/Power%20Generation/Pages/Power_Plant_Tour_Information.aspx}
		\end{enumerate}
	\item HowStuffWorks, Inc.: \url{http://www.howstuffworks.com/}
	\item Chevron Corporation: \vspace{-0.2cm}
		\begin{enumerate} \itemsep -2pt
		\item {\it Will you join us}: \vspace{-0.1cm}
			\begin{enumerate} \itemsep -1pt
			\item Energy issues: \url{http://www.willyoujoinus.com/energyissues/}
			\item Tools and resources: \vspace{-0.1cm}
				\begin{itemize} \itemsep -1pt
				\item \url{http://www.willyoujoinus.com/toolsresources/}
				\item Helpful links (includes K-12 educational material): \url{http://www.willyoujoinus.com/toolsresources/helpfullinks/}
				\end{itemize}
			\item MPG Optimizer: \url{http://www.willyoujoinus.com/usingenergywisely/mpgoptimizer/}
			\item Energy generator: \url{http://www.willyoujoinus.com/usingenergywisely/energygenerator/}
			\end{enumerate}
		\end{enumerate}
	\item National Energy Foundation: \vspace{-0.2cm}
		\begin{enumerate} \itemsep -2pt
		\item \url{http://www.nef.org.uk/} and \url{http://www.nef1.org/}
		\item Students: \url{http://www.nef1.org/students.html}
		\item Educators: \url{http://www.nef1.org/educators.html}
		\item Schools: \vspace{-0.1cm}
			\begin{enumerate} \itemsep -1pt
			\item \url{http://www.nef.org.uk/communities/schools/index.html}
			\item Helpful links: \url{http://www.nef.org.uk/communities/schools/energylinks.html}
			\item School Resources: \url{http://www.nef.org.uk/communities/schools/resources/index.html}
			\item {\it LogiCity} is a fun interactive computer game with a difference. It's a game set in a 3D virtual city with five main activities where you are set the task of reducing the carbon footprint of an average resident. See \url{http://www.nef.org.uk/communities/schools/logicity.html}.
			\end{enumerate}
		\item Resources: \url{http://www.nef.org.uk/actonCO2/index.asp}
		\item Igniting Creative Energy - A National Student Challenge: \vspace{-0.1cm}
			\begin{enumerate} \itemsep -1pt
			\item \url{http://www.ignitingcreativeenergy.org/}
			\item Students: \url{http://www.ignitingcreativeenergy.org/students.html}
			\end{enumerate}
		\end{enumerate}
	\item StartSpot Mediaworks: \vspace{-0.2cm}
		\begin{enumerate} \itemsep -2pt
		\item StartSpot Network: \vspace{-0.1cm}
			\begin{enumerate} \itemsep -1pt
			\item HomeworkSpot: \vspace{-0.1cm}
				\begin{itemize} \itemsep -1pt
				\item \url{http://www.homeworkspot.com/}
				\item Science Fair Project Center: \url{http://www.homeworkspot.com/sciencefair/}
				\end{itemize}
			\end{enumerate}
		\end{enumerate}
	\item Super Science Fair Projects: \url{http://www.super-science-fair-projects.com/}
	\item All Science Fair Projects: Science Fair Projects with Complete Instructions, \url{http://www.all-science-fair-projects.com/}
	\item The Science Club: \vspace{-0.2cm}
		\begin{enumerate} \itemsep -2pt
		\item \url{http://scienceclub.org/}
		\item Science Fair Idea Exchange: \url{http://scienceclub.org/scifair.html}
		\end{enumerate}
	\item Oracle Education Foundation: \vspace{-0.2cm}
		\begin{enumerate} \itemsep -2pt
		\item \url{http://www.oraclefoundation.org/}
		\item ThinkQuest: \vspace{-0.1cm}
			\begin{enumerate} \itemsep -1pt
			\item \url{http://www.thinkquest.org/en/}
			\item ThinkQuest International Competition: \url{http://www.thinkquest.org/competition/}
			\item Projects: \url{http://thinkquest.org/en/projects/index.html}
			\item Library: \url{http://thinkquest.org/pls/html/think.library}
			\item Example of a computer game developed by students: Crisis! - The Game, \url{http://library.thinkquest.org/20331/game/}
			\end{enumerate}
		\end{enumerate}
	\item University of Minnesota: \vspace{-0.2cm}
		\begin{enumerate} \itemsep -2pt
		\item Institute on Community Integration; College of Education and Human Development: \vspace{-0.1cm}
			\begin{enumerate} \itemsep -1pt
			\item National Center on Secondary Education and Transition (NCSET): \vspace{-0.1cm}
				\begin{itemize} \itemsep -1pt
				\item \url{http://www.ncset.org/}
				\item NCSET Topics: \url{http://www.ncset.org/topics/default.asp}
				\item Web Sites: \url{http://www.ncset.org/websites/default.asp}
				\item The Youthhood!: \url{http://www.youthhood.org/}
				\end{itemize}
			\end{enumerate}
		\end{enumerate}
	\item Jobs for America's Graduates: \vspace{-0.2cm}
		\begin{enumerate} \itemsep -2pt
		\item \url{http://www.jag.org/}
		\item JAG Model program applications: \vspace{-0.1cm}
			\begin{enumerate} \itemsep -1pt
			\item \url{http://www.jag.org/model.htm}
			\item Programs are available for students in middle school and high school, high school dropouts, high school seniors, students in alternative education programs, and college underclassmen
			\end{enumerate}
		\item JAG Career Corner: \url{http://www.jag.org/jag_career_corner.htm}
		\item Students: \url{http://www.jag.org/students.htm}
		\item Resource library: \url{http://www.jag.org/library.htm}
		\item Performance outcomes: \url{http://www.jag.org/outcomes.htm}
		\item Funding: \url{http://www.jag.org/funding.htm}
		\end{enumerate}
	\item Alliance to Save Energy: \vspace{-0.2cm}
		\begin{enumerate} \itemsep -2pt
		\item Energy Hog campaign: \vspace{-0.1cm}
			\begin{enumerate} \itemsep -1pt
			\item \url{http://www.energyhog.org/}
			\item Adults: \url{http://www.energyhog.org/adult/adults.htm}
			\item Children: \url{http://www.energyhog.org/childrens.htm}
			\end{enumerate}
		\end{enumerate}
	\item Learning First Alliance: \vspace{-0.2cm}
		\begin{enumerate} \itemsep -2pt
		\item \url{http://www.learningfirst.org/}
		\item Issues and publications: \url{http://www.learningfirst.org/issues}
		\item Resources: \url{http://www.learningfirst.org/resources}
		\end{enumerate}
	\item NaMaYa: \url{http://www.namaya.com/}
	\item NIXTY: \url{http://nixty.com/}
	\item K12 Open Ed: \url{http://www.k12opened.com/wiki/index.php/Main_Page}
	\item Learning Is For Everyone: \url{http://www.learningis4everyone.org/}
	\item The Smithsonian Commons Prototype: \url{http://www.si.edu/commons/prototype/}
	\item Futurelab: Resources for educators and parents, \url{http://www.futurelab.org.uk/resources}
	\item Innosight Institute: Resources for education, \url{http://www.innosightinstitute.org/practices/education/}
	\item WGBH Educational Foundation: \url{http://www.wgbh.org/}
	\item Discovery Education: \vspace{-0.2cm}
		\begin{enumerate} \itemsep -2pt
		\item Classroom resources: \url{http://school.discoveryeducation.com/}
		\item Home resources: \url{http://school.discoveryeducation.com/homeworkhelp/homework_help_home.html}
		\end{enumerate}
	\item The Gilder Lehrman Institute of American History: \vspace{-0.2cm}
		\begin{enumerate} \itemsep -2pt
		\item \url{http://www.gilderlehrman.org/}
		\item Resources for teachers and schools: \url{http://www.gilderlehrman.org/teachers/}
		\item Civil War Essay Contest (for students in selected K-12 schools): \url{http://www.gilderlehrman.org/affiliate/civil_war.php}
		\end{enumerate}
	\item The GRAMMY Museum: \vspace{-0.2cm}
		\begin{enumerate} \itemsep -2pt
		\item Teacher curriculum and resources. Available online at: \url{http://www.grammymuseum.org/interior.php?section=education&page=teachercurriculum}; last accessed on November 15, 2010.
		\end{enumerate}
	\item Purdue University: \vspace{-0.2cm}
		\begin{enumerate} \itemsep -2pt
		\item Department of Entomology: \vspace{-0.1cm}
			\begin{enumerate} \itemsep -1pt
			\item Genomics Analogy Model for Educators (G.A.M.E.): \url{http://www.entm.purdue.edu/extensiongenomics/GAME/default.html}
			\end{enumerate}
		\end{enumerate}
	\item Verizon Thinkfinity: \url{http://www.thinkfinity.org/about-us}
	\item Oregon Virtual School District (ORVSD): \vspace{-0.2cm}
		\begin{enumerate} \itemsep -2pt
		\item \url{http://orvsd.org/}
		\item ``Oregon Virtual School District (ORVSD) helps integrate technology into Oregon public school classrooms by giving teachers access to free tech tools and resources online.''
		\item ``The Oregon Virtual School District is a program led by the Oregon Department of Education that, in cooperation with a consortium of virtual learning providers throughout the state, seeks to increase access and availability of online learning and teaching resources free of charge to public school teachers of Oregon. Oregon State University is providing hosting and development resources through a partnership with the OSU Open Source Lab and the OSU Business Solutions Group.''
		\end{enumerate}
	\item The Association of Educational Publishers (AEP): \vspace{-0.2cm}
		\begin{enumerate} \itemsep -2pt
		\item The AEP Awards: \vspace{-0.1cm}
			\begin{enumerate} \itemsep -1pt
			\item \url{http://www.aepweb.org/awards/index.htm}
			\item Look at the winners of previous AEP awards to determine some of the good educational resources that are available
			\end{enumerate}
		\end{enumerate}
	\item Educational Dividends: \vspace{-0.2cm}
		\begin{enumerate} \itemsep -2pt
		\item \url{http://www.educationaldividends.com/}
		\item Teachers: \vspace{-0.1cm}
			\begin{enumerate} \itemsep -1pt
			\item \url{http://www.educationaldividends.com/index.asp?menu=Teachers}
			\item Teaching Tools: \url{http://www.educationaldividends.com/teachers/tools.asp}
			\item Reference Desk: \vspace{-0.1cm}
				\begin{itemize} \itemsep -1pt
				\item \url{http://www.educationaldividends.com/teachers/reference.asp}
				\item Standards Reference Desk (resources for education standards in the US at the national, state, and local levels): \url{http://www.educationaldividends.com/teachers/standards_desk.asp}
				\item How We Learn: Learning Styles, \url{http://www.educationaldividends.com/teachers/learning_styles.asp}
				\item How We Learn: Multiple Intelligences, \url{http://www.educationaldividends.com/teachers/multiple_intelligences.asp}
				\item Statistics Desk (statistical information about education in the US): \url{http://www.educationaldividends.com/teachers/statistics_desk.asp}
				\end{itemize}
			\item Information about the teaching profession: \vspace{-0.1cm}
				\begin{itemize} \itemsep -1pt
				\item \url{http://www.educationaldividends.com/teachers/welcome.asp}
				\item Office of Occupational Statistics and Employment Projections, ``Educational Services,'' in {\it Career Guide to Industries}, 2010-11 Edition, U.S. Bureau of Labor Statistics, U.S. Department of Labor, Washington, DC, December 17, 2009. Available online at: \url{http://stats.bls.gov/oco/cg/cgs034.htm}; last accessed on December 8, 2010. [ Suggested citation: Bureau of Labor Statistics, U.S. Department of Labor, {\it Career Guide to Industries, 2010-11 Edition}, Educational Services , on the Internet at \url{http://www.bls.gov/oco/cg/cgs034.htm} (visited December 07, 2010). ]
				\item Experience Teaching: \url{http://www.educationaldividends.com/teachers/experience.asp}
				\item Continuous Improvement: \url{http://www.educationaldividends.com/teachers/toolkit.asp}
				\end{itemize}
			\end{enumerate}
		\item Personality and Career Tests: \url{http://www.educationaldividends.com/teachers/tests.asp}
		\end{enumerate}
	\item Smithsonian Institution: \vspace{-0.2cm}
		\begin{enumerate} \itemsep -2pt
		\item Educators: \url{http://www.si.edu/Educators}
		\item Smithsonian Institution Traveling Exhibition Service (SITES): \vspace{-0.1cm}
			\begin{enumerate} \itemsep -1pt
			\item For Teachers: \url{http://www.sites.si.edu/education/teachers_res2.htm}
			\end{enumerate}
		\item Smithsonian Folkways Recordings (or simply, Smithsonian Folkways): \vspace{-0.1cm}
			\begin{enumerate} \itemsep -1pt
			\item Tools for Teaching: \url{http://www.folkways.si.edu/tools_for_teaching/introduction.aspx}
			\end{enumerate}
		\item Freer Gallery of Art / Arthur M. Sackler Gallery: \vspace{-0.1cm}
			\begin{enumerate} \itemsep -1pt
			\item Resources for Educators: \url{http://www.asia.si.edu/explore/teacherResources.asp}
			\item Explore + Learn: Browse Online Resources by Area: \vspace{-0.1cm}
				\begin{itemize} \itemsep -1pt
				\item \url{http://www.asia.si.edu/explore/default.asp}
				\item Has resources for art in: \vspace{-0.1cm}
					\begin{itemize} \itemsep -1pt
					\item The Americas
					\item Ancient Egypt
					\item Ancient Near East
					\item Islamic world
					\item China
					\item Japan
					\item Korea
					\item South Asia
					\item Himalayas
					\item Southeast Asian
					\item It also has biblical manuscripts and contemporary art
					\end{itemize}
				\end{itemize}
			\item Online Exhibition Features: \url{http://www.asia.si.edu/exhibitions/online.asp}
			\item Collections: \url{http://www.asia.si.edu/collections/default.asp}
			\end{enumerate}
		\item National Museum of American History: \vspace{-0.1cm}
			\begin{enumerate} \itemsep -1pt
			\item Jerome and Dorothy Lemelson Center for the Study of Invention and Innovation: \vspace{-0.1cm}
				\begin{itemize} \itemsep -1pt
				\item Resources: \vspace{-0.1cm}
					\begin{itemize} \itemsep -1pt
					\item \url{http://invention.smithsonian.org/resources/}
					\item \url{http://invention.smithsonian.org/resources/default_sites_weblinks.aspx}
					\item Invention stories - archives, articles, audio, and video: \url{http://invention.smithsonian.org/resources/default_index.aspx}
					\end{itemize}
				\item Educational Materials: \vspace{-0.1cm}
					\begin{itemize} \itemsep -1pt
					\item \url{http://invention.smithsonian.org/resources/menu_edu_materials.aspx}
					\item Experiments: \url{http://invention.smithsonian.org/resources/menu_edu_materials.aspx?MaterialTypeID=3&MaterialTypeDesc=Experiments}
					\item Educational Materials: \url{http://invention.smithsonian.org/resources/menu_edu_materials_f.aspx?MaterialTypeDesc=Features}
					\end{itemize}
				\item Centerpieces: \vspace{-0.1cm}
					\begin{itemize} \itemsep -1pt
					\item \url{http://invention.smithsonian.org/centerpieces/}
					\item \url{http://invention.smithsonian.org/centerpieces/iap-info.aspx}
					\item Electric guitar: \url{http://invention.smithsonian.org/centerpieces/electricguitar/index.htm}
					\item Innovative Lives: \url{http://invention.smithsonian.org/centerpieces/ilives/}
					\item ``Exploring the History of Women Inventors'' by J.E. Bedi (in {\it Innovative Lives}): \url{http://invention.smithsonian.org/centerpieces/ilives/womeninventors.html}
					\item Whole Cloth: \url{http://invention.smithsonian.org/centerpieces/whole_cloth/index.html}
					\item The Quartz Watch: \url{http://invention.smithsonian.org/centerpieces/quartz/index.html}
					\item Edison Invents!: All about Thomas Edison and his invention, \url{http://invention.smithsonian.org/centerpieces/edison/default.asp}
					\end{itemize}
				\item Modern Inventors Documentation Program (MIND): \url{http://invention.smithsonian.org/resources/mind_resources.aspx}
				\item Invention at Play: \vspace{-0.1cm}
					\begin{itemize} \itemsep -1pt
					\item \url{http://inventionatplay.org/}
					\item Resources: \url{http://inventionatplay.org/resources.html}
					\item Invention Playhouse: \url{http://inventionatplay.org/playhouse_main.html}
					\item Inventors' Stories: \url{http://inventionatplay.org/inventors_main.html}
					\item Does play matter? (using play to help children learn and think): \url{http://inventionatplay.org/matter_main.html}
					\end{itemize}
				\item Spark!Lab: \vspace{-0.1cm}
					\begin{itemize} \itemsep -1pt
					\item \url{http://sparklab.si.edu/}
					\item About Spark!Lab (introduce children to the process of innovation via play and fun activities): \url{http://sparklab.si.edu/spark-about.html}
					\item Activities \& Experiments: \url{http://sparklab.si.edu/spark-experiments.html}
					\item Inventor Profiles: \url{http://sparklab.si.edu/spark-inventors.html}
					\item Resources: \url{http://sparklab.si.edu/spark-resources.html}
					\end{itemize}
				\end{itemize}
			\end{enumerate}
		\end{enumerate}
	\item Economic and Social Research Council (ESRC): \vspace{-0.2cm}
		\begin{enumerate} \itemsep -2pt
		\item {\it Social Science for Schools}; Science in Society Team: \vspace{-0.1cm}
			\begin{enumerate} \itemsep -1pt
			\item \url{http://www.esrcsocietytoday.ac.uk/ESRCInfoCentre/ssfs/}
			\item Social science resources: \url{http://www.esrcsocietytoday.ac.uk/ESRCInfoCentre/ssfs/resources/}
			\item Career guides for different disciplines in social science and economics: \url{http://www.esrcsocietytoday.ac.uk/ESRCInfoCentre/ssfs/careers/}
			\item Related online resources: \url{http://www.esrcsocietytoday.ac.uk/ESRCInfoCentre/ssfs/links/}
			\end{enumerate}
		\end{enumerate}
	\end{itemize}
\item National Council for Accreditation of Teacher Education (NCATE): \vspace{-0.3cm}
	\begin{enumerate} \itemsep -2pt
	\item \url{http://www.ncate.org/}
	\item Has resources about degree programs in education and their accreditation, as well as how to become a teacher
	\item State-specific Recognized Programs by NCATE and Specialized Professional Associations (SPAs): \vspace{-0.2cm}
		\begin{enumerate} \itemsep -2pt
		\item \url{http://www.ncate.org/tabid/165/Default.aspx}
		\item Find out about educational programs in: \vspace{-0.1cm}
			\begin{enumerate} \itemsep -1pt
			\item special education
			\item early childhood education
			\item educational leadership
			\item educational technology specialist
			\item elementary education
			\item English
			\item health education
			\item foreign languages
			\item gifted education
			\item mathematics
			\item physical education
			\item science education
			\item school psychology
			\item secondary computer science education
			\item social studies
			\item Teachers of English to Speakers of Other Languages (TESOL)
			\item technology and engineering educators
			\end{enumerate}
		\end{enumerate}
	\item Financial Aid Resources for Teacher Education Students: \url{http://www.ncate.org/Public/CurrentFutureTeachers/FinancialAidResources/tabid/351/Default.aspx}
	\end{enumerate}
%%%%%%%%%%%%%%%%%%%%%%%
\item scholarships: \vspace{-0.3cm}
	\begin{enumerate} \itemsep -2pt
	\item U.S. Department of State: \vspace{-0.2cm}
		\begin{enumerate} \itemsep -2pt
		\item Bureau of Educational and Cultural Affairs: \vspace{-0.1cm}
			\begin{enumerate} \itemsep -1pt
			\item National Security Language Initiative for Youth (NSLI-Y): \vspace{-0.1cm}
				\begin{itemize} \itemsep -1pt
				\item \url{http://exchanges.state.gov/youth/programs/nsli.html}
				\item ``The State Department�s National Security Language Initiative for Youth (NSLI-Y) provides merit-based scholarships to U.S. high school students and recent graduates interested in learning less-commonly studied foreign languages.''
				\end{itemize}
			\end{enumerate}
		\end{enumerate}
	\end{enumerate}
%%%%%%%%%%%%%%%%%%%%%%%
\item underrepresented minorities: \vspace{-0.3cm}
	\begin{enumerate} \itemsep -2pt
	\item The University of North Carolina at Chapel Hill: \vspace{-0.2cm}
		\begin{enumerate} \itemsep -2pt
		\item Gary Bishop, {\it Research}, Department of Computer Science, The University of North Carolina at Chapel Hill. Available at: \url{http://wwwx.cs.unc.edu/~gb/wp/research/}; last accessed on September 3, 2010. [ Has plenty of information and resources (including learning aids and material) to help people who are visually or mobility impaired learn. ]
		\end{enumerate}
	\item Myra Sadker Foundation: \vspace{-0.2cm}
		\begin{enumerate} \itemsep -2pt
		\item $100+$ Ideas to Promote Gender Equity in Schools and Beyond: \url{http://www.sadker.org/100ideas.html}
		\item Gender Equity Activities: \url{http://www.sadker.org/WhatYouCanDo.html}
		\item Gender Equity Activities for Concerned Citizens: \url{http://www.sadker.org/GenderEquity-citizens.html}
		\item Gender Equity Activities for Families: \url{http://www.sadker.org/GenderEquity-family.html}
		\item Gender Equity Activities for Teachers: \vspace{-0.1cm}
			\begin{enumerate} \itemsep -1pt
			\item Early Childhood: \url{http://www.sadker.org/GenderEquity-teacher1.html}
			\item Primary Grades: \url{http://www.sadker.org/GenderEquity-teacher2.html}
			\item Upper Elementary: \url{http://www.sadker.org/GenderEquity-teacher3.html}
			\item Middle and High School: \url{http://www.sadker.org/GenderEquity-teacher4.html}
			\end{enumerate}
		\item Resources for feminism and links to web pages of feminist organizations: \url{http://www.sadker.org/ReadsLinks.html}
		\end{enumerate}
	\item League of United Latin American Citizens (LULAC): \vspace{-0.3cm}
		\begin{enumerate} \itemsep -2pt
		\item LULAC National Educational Service Centers, Inc: \vspace{-0.2cm}
			\begin{enumerate} \itemsep -2pt
			\item \url{http://www.lnesc.org/}
			\item Programs: \vspace{-0.1cm}
				\begin{itemize} \itemsep -1pt
				\item Improving literacy among Latino/Latina youth
				\item Encouraging Latino/Latina youth to pursue careers in science and engineering
				\item Helping Latino/Latina youth acquire leadership skills
				\item Improving college access for Latino/Latina youth by mentoring and summer programs (e.g., Gear-Up, Upward Bound, and the PALMS Initiative)
				\item Helping Latino/Latina families acquire financial success, so that Latino/Latina youth can pursue higher education
				\item Scholarships for Latino/Latina youth
				\item \url{http://lnesc.org/index.asp?Type=B_BASIC&SEC={808B6D04-913C-483F-8A05-5BD44B03ED62}}
				\end{itemize}
			\end{enumerate}
		\end{enumerate}
	\item ASPIRA: \vspace{-0.2cm}
		\begin{enumerate} \itemsep -2pt
		\item ASPIRA Programs for Latino/Latina youth: \url{http://aspira.org/manuals/aspira-programs}
		\end{enumerate}
	\end{enumerate}
%%%%%%%%%%%%%%%%%%%%%%%
\item places to visit: \vspace{-0.3cm}
	\begin{enumerate} \itemsep -2pt
	\item Exploratorium @ The Palace of Fine Arts (San Francisco, CA): \url{http://www.exploratorium.edu/}
	\item Educational Dividends: \vspace{-0.2cm}
		\begin{enumerate} \itemsep -2pt
		\item \url{http://www.educationaldividends.com/}
		\item Suggestions for organizing field trips to explore your interests: \url{http://www.educationaldividends.com/students/student_issues.asp}
		\item Career exploration: \url{http://www.educationaldividends.com/students/career_choices.asp}
		\item Computer skills: \url{http://www.educationaldividends.com/students/technology.asp}
		\item Quizzes to help you find out what is your preferred learning style and to discover more about your personality: \url{http://www.educationaldividends.com/students/learning_quiz.asp}
		\item Resources to help you learn about various topics in science, mathematics, social science, and humanities: \url{http://www.educationaldividends.com/students/resources.asp}
		\end{enumerate}
	\end{enumerate}
%%%%%%%%%%%%%%%%%%%%%%%
\item resources for at-risk youths: \vspace{-0.3cm}
	\begin{enumerate} \itemsep -2pt
	\item At-Risk Youth: \url{http://www.at-risk.org/}
	\item Peace First: \vspace{-0.2cm}
		\begin{enumerate} \itemsep -2pt
		\item \url{http://www.peacefirst.org/site/}
		\item To help youths become ``problem-solvers, rather than witnesses, or victims of their surrounding''
		\item To reduce youth homicide rates
		\item Teach children ``critical conflict resolution skills''
		\item Help teachers improve their ``conflict resolution and classroom management skills''
		\item To encourage youths to help each other, and get them to break up fights
		\item ``The Peace First curriculum is tailored to meet the developmental needs of students in Pre-K through eighth grade. Once a week, young adult volunteers and classroom teachers work together to teach students about friendship, communication, and conflict resolution through the use of experiential activities. First graders learn about communicating their feelings, third graders work on being peacemakers in their classroom, and fifth graders explore how to resolve and deescalate conflicts.''
		\item Has programs for students/youths, teachers, principals, and volunteers.
		\end{enumerate}
	\item Americans for the Arts: \vspace{-0.2cm}
		\begin{enumerate} \itemsep -2pt
		\item YouthARTS: \vspace{-0.1cm}
			\begin{enumerate} \itemsep -1pt
			\item \url{http://www.artsusa.org/youtharts/index.asp}
			\item ``The YouthARTS site is designed to give arts agencies, juvenile justice agencies, social service organizations, and other community-based organizations detailed information about how to plan, run, provide training, and evaluate arts programs for at-risk youth.''
			\end{enumerate}
		\end{enumerate}
	\end{enumerate}
%%%%%%%%%%%%%%%%%%%%%%%
\item general music and arts education: \vspace{-0.3cm}
	\begin{enumerate} \itemsep -2pt
	\item Americans for the Arts: \vspace{-0.2cm}
		\begin{enumerate} \itemsep -2pt
		\item Americans for the Arts, ``Ten Simple Ways Parents Can Get More Art in Their Kids' Lives.'' Available online at: \url{http://www.americansforthearts.org/public_awareness/get_involved/001.asp}; last accessed on November 30, 2010.
		\item YouthARTS: \vspace{-0.1cm}
			\begin{enumerate} \itemsep -1pt
			\item \url{http://www.artsusa.org/youtharts/index.asp}
			\item ``The YouthARTS site is designed to give arts agencies, juvenile justice agencies, social service organizations, and other community-based organizations detailed information about how to plan, run, provide training, and evaluate arts programs for at-risk youth.''
			\end{enumerate}
		\end{enumerate}
	\item The John F. Kennedy Center for the Performing Arts: \vspace{-0.2cm}
		\begin{enumerate} \itemsep -2pt
		\item Kennedy Center Institute for Arts Management: \url{http://artsmanagerfba.artsmanager.org/common/Pages/About.aspx}
		\item {\sc ArtsEdge}: \vspace{-0.1cm}
			\begin{enumerate} \itemsep -1pt
			\item The National Standards for Arts Education for Grades K-4, 5-8, and 9-12: \url{http://artsedge.kennedy-center.org/educators/standards.aspx}
			\item Tips and guides for educators: \url{http://artsedge.kennedy-center.org/educators/how-to.aspx}
			\item Lesson plans for educators: \url{http://artsedge.kennedy-center.org/educators/lessons.aspx}
			\item Information for parents, guardians, foster parents, baby-sitters, and grandparents: \url{http://artsedge.kennedy-center.org/families.aspx}
			\item Information for students: \url{http://artsedge.kennedy-center.org/students.aspx}
			\item Themes for artistic, cultural, academic, and intellectual exploration: \url{http://artsedge.kennedy-center.org/themes.aspx}
			\item Multimedia: \url{http://artsedge.kennedy-center.org/multimedia.aspx}
			\end{enumerate}
		\end{enumerate}
	\end{enumerate}
\item music education: \vspace{-0.3cm}
	\begin{enumerate} \itemsep -2pt
	\item Washington Performing Arts Society (WPAS): \vspace{-0.2cm}
		\begin{enumerate} \itemsep -2pt
		\item WPAS Education \& Community -- Connections through the Arts Education Programs for All Ages: \vspace{-0.1cm}
			\begin{enumerate} \itemsep -1pt
			\item The Capitol Jazz Project: \vspace{-0.1cm}
				\begin{itemize} \itemsep -1pt
				\item \url{http://www.wpas.org/educcomm/programsforyoungpeople/capitoljazzproject.aspx}
				\item ``Washington Performing Arts Society (WPAS) and the D.C. Public Schools, in collaboration with Jazz at Lincoln Center, has launched The Capitol Jazz Project, an important step in supporting music education for all students in the District of Colombia.''
				\item ``Through the Capitol Jazz Project, students hone their listening, performing, improvising, composing, arranging, music reading, and notation skills.''
				\item ``The Capitol Jazz Project is being implemented in 6 D.C. middle schools with a total enrollment of more than 500 music students.''
				\item ``A true collaboration, The Capitol Jazz Project brings the combined resources and expertise of WPAS, Jazz at Lincoln Center, and the D.C. Public Schools to create a model music education program.''
				\end{itemize}
			\item Joseph and Goldie Feder Memorial String Competition: \vspace{-0.1cm}
				\begin{itemize} \itemsep -1pt
				\item \url{http://www.wpas.org/educcomm/programsforyoungpeople/josephandgoldiefedermemorialstringcompetition.aspx}
				\item ``The Feder String Competition inspires and nurtures D.C. area youth in grades 6 through 12 who study violin, viola, cello, and double bass.''
				\item ``Each year, 80 students compete for 30 awards and scholarships.''
				\item ``Held each spring, WPAS awards cash prizes toward private lessons, scholarships for summer study programs, and tickets for top winners and their family members to attend a WPAS concert.''
				\item ``Winners of the competition are also given special performance opportunities such as on the Kennedy Center's Millennium Stage and The Shakespeare Theatre Company's Happenings at the Harman series.''
				\end{itemize}
			\item WPAS Summer Performing Arts Academy summer programs: \vspace{-0.1cm}
				\begin{itemize} \itemsep -1pt
				\item \url{http://www.wpas.org/educcomm/programsforyoungpeople/wpassummerperformingartsacademy.aspx}
				\end{itemize}
			\end{enumerate}
		\end{enumerate}
	\item Young Concert Artists, Inc. \vspace{-0.2cm}
		\begin{enumerate} \itemsep -2pt
		\item Annaliese Soros Educational Residency Program: \url{http://www.yca.org/auditions/}
		\end{enumerate}
	\item The Choral Arts Society of Washington: \vspace{-0.2cm}
		\begin{enumerate} \itemsep -2pt
		\item Classroom Resources: \url{http://www.choralarts.org/Education/Classroom-Resources.aspx}
		\end{enumerate}
	\item League of American Orchestras: \vspace{-0.2cm}
		\begin{enumerate} \itemsep -2pt
		\item Career planning: \vspace{-0.1cm}
			\begin{enumerate} \itemsep -1pt
			\item Resources for pre-college students, college students, and graduate students: \url{http://www.americanorchestras.org/career_center/career_planning.html}
			\item Arts Administration programs: \url{http://www.americanorchestras.org/career_center/arts_admin_programs.html}
			\item Non-profit management, {\bf public policy} and leadership programs: \url{http://www.americanorchestras.org/career_center/resources_non_prof_and.html}
			\end{enumerate}
		\end{enumerate}
	\item The John F. Kennedy Center for the Performing Arts: \vspace{-0.2cm}
		\begin{enumerate} \itemsep -2pt
		\item Betty Carter's Jazz Ahead: \vspace{-0.1cm}
			\begin{enumerate} \itemsep -1pt
			\item \url{http://www.kennedy-center.org/programs/jazz/jazzahead/}
			\item ``Music residency program for young people''
			\item ``The Jazz Ahead program identifies outstanding, emerging jazz artists in their mid-teens to age thirty, and brings them together under the tutelage of experienced artist-instructors who coach and counsel them, helping to polish their performance, composing and arranging skills.''
			\item ``The two week-long residency program includes daily workshops and rehearsals with established jazz artists, and culminate in three concerts on the Kennedy Center Millennium Stage, which will be broadcast live over the internet.''
			\end{enumerate}
		\item The National Symphony Orchestra (NSO): \vspace{-0.1cm}
			\begin{enumerate} \itemsep -1pt
			\item The National Symphony Orchestra's Summer Music Institute (SMI): \vspace{-0.1cm}
				\begin{itemize} \itemsep -1pt
				\item \url{http://www.kennedy-center.org/nso/nsoed/smi/home.cfm}
				\item ``Every summer, approximately 70 students (ages 15-20) from all over the nation meet in Washington, D.C., to attend the National Symphony Orchestra's Summer Music Institute (SMI).''
				\item ``The Institute offers four weeks of private lessons, rehearsals, coaching by National Symphony Orchestra members, classes, and lectures to prepare aspiring musicians for their futures in music.''
				\end{itemize}
			\item Young Associates' Program: \vspace{-0.1cm}
				\begin{itemize} \itemsep -1pt
				\item \url{http://www.kennedy-center.org/nso/nsoed/youngassociates.html}
				\item ``The National Symphony Orchestra (NSO) is sponsoring its Young Associates' Program for high school students in grades 11 and 12 in the Washington, DC, metropolitan area who are interested in pursuing a musical career.''
				\item ``Twenty outstanding instrumentalists (pianists are not included) will be selected to attend rehearsals of the National Symphony Orchestra and take part in seminars with conductors, artists, NSO musicians, and representatives of the arts management field.''
				\item ``Through this program, the Young Associates will acquire an appreciation of the wide range of skills, knowledge, and abilities--managerial as well as musical--that are required to put together a performance by a major symphony orchestra. Selection process is by application.''
				\item ``The core of the program involves attendance at rehearsals of the National Symphony Orchestra at the Kennedy Center and observation of various guest artists. In addition to attending NSO rehearsals, students participate in workshops to explore careers in management, music education, publicity, music library, and other professions that are essential to the life of every successful orchestra.''
				\item ``Students do not play their instruments as part of the program. Students learn through listening, observation, and asking questions of professionals.''
				\end{itemize}
			\end{enumerate}
		\end{enumerate}
	\end{enumerate}
\item dance education: \vspace{-0.3cm}
	\begin{enumerate} \itemsep -2pt
	\item The Washington Ballet: \vspace{-0.2cm}
		\begin{enumerate} \itemsep -2pt
		\item The Washington School of Ballet (TWSB): \vspace{-0.1cm}
			\begin{enumerate} \itemsep -1pt
			\item Summer Intensive program (requires an audition): \url{http://www.washingtonballet.org/the-school/summer-intensive/}
			\end{enumerate}
		\item TWB's EXCEL! scholarship program (for DanceDC students): \vspace{-0.1cm}
			\begin{enumerate} \itemsep -1pt
			\item \url{http://www.washingtonballet.org/community-engagement/default.htm}
			\item \url{http://www.washingtonballet.org/community-engagement/other-programs/}
			\item Also, has need-based scholarships
			\end{enumerate}
		\end{enumerate}
	\item The John F. Kennedy Center for the Performing Arts: \vspace{-0.2cm}
		\begin{enumerate} \itemsep -2pt
		\item Exploring Ballet With Suzanne Farrell: A Three-Week Summer Ballet Intensive for Young Dancers: \vspace{-0.1cm}
			\begin{enumerate} \itemsep -1pt
			\item \url{http://www.kennedy-center.org/education/farrell/}
			\item ``In July and August, students from across the United States and around the world will participate in the eighteenth annual session of the Kennedy Center's ballet training program Exploring Ballet with Suzanne Farrell. The three-week residency for dancers ages 14 to 18 with at least five years of ballet training will be held at the Kennedy Center from August 1 - August 20, 2011.''
			\item ``During the three-week period, students take two ballet technique classes a day, six days a week, with Ms. Farrell. Students also participate in a number of cultural activities to enhance their experience in Washington, D.C., including museum visits, trips to historical landmarks, and attending performances.''
			\end{enumerate}
		\item Dance Theatre of Harlem Residency program: \vspace{-0.1cm}
			\begin{enumerate} \itemsep -1pt
			\item \url{http://www.kennedy-center.org/education/community/programs.html#artistic}
			\item ``Since 1993, the Kennedy Center's Dance Theatre of Harlem Residency program has provided ballet training for male and female students age 8-18 with identified promise in ballet taught by Dance Theatre of Harlem (DTH) instructors or former principal dancers.''
			\item ``Students are selected by audition for a twenty-class series, culminating with a public demonstration and performance on a Kennedy Center main stage.''
			\item ``Classical ballet training is taught in four class levels, from novice to advance.''
			\item ``Students must have at least one year of ballet training to qualify for the program.''
			\end{enumerate}
		\end{enumerate}
	\end{enumerate}
%%%%%%%%%%%%%%%%%%%%%%%
\item JA Worldwide (Junior Achievement): \vspace{-0.3cm}
	\begin{enumerate} \itemsep -2pt
	\item \url{http://www.ja.org/}
	\item Resources for educators: \url{http://www.ja.org/involved/involved_educat.shtml}
	\item Resources for parents: \url{http://www.ja.org/involved/involved_parents.shtml}
	\item Resources for students: \url{http://www.ja.org/involved/involved_students.shtml}
	\end{enumerate}
\item U.S. Department of State: \vspace{-0.3cm}
	\begin{enumerate} \itemsep -2pt
	\item Programs for Americans and non-Americans.
	\item Summer Work Travel - In the summer work travel program: \url{http://exchanges.state.gov/}
	\item Cultural Programs Division: \url{http://exchanges.state.gov/cultural/index.html}
	\item Youth Programs Division: \url{http://exchanges.state.gov/youth/index.html}
	\item EducationUSA: \url{http://educationusa.state.gov/}
	\item International Visitor Leadership Program: \url{http://exchanges.state.gov/ivlp/ivlp.html}
	\item Programs for non-U.S. Citizens: \url{http://exchanges.state.gov/prog-non-us.html}
	\item Programs for U.S. Citizens: \url{http://exchanges.state.gov/prog-us.html}
	\item Resources for Students: \url{http://exchanges.state.gov/student.html}
	\item Bureau of Educational and Cultural Affairs: \vspace{-0.2cm}
		\begin{enumerate} \itemsep -2pt
		\item Future Leaders Exchange (FLEX) Program: \vspace{-0.1cm}
			\begin{enumerate} \itemsep -1pt
			\item \url{http://exchanges.state.gov/youth/programs/flex.html}
			\item ``The Future Leaders Exchange (FLEX) Program gives students (ages 15-17) the chance to live with a host family and attend a U.S. high school for a year.''
			\end{enumerate}
		\item Office of Citizen Exchanges: \vspace{-0.1cm}
			\begin{enumerate} \itemsep -1pt
			\item Youth Programs Division: \vspace{-0.1cm}
				\begin{itemize} \itemsep -1pt
				\item \url{http://exchanges.state.gov/youth/index.html}
				\item Has programs for youths in various parts of the world
				\item ``The Youth Programs Division is committed to empowering the next generation and establishing long-lasting ties between the United States and other countries through exchange programs and institutional partnerships. Programs focus primarily on secondary schools and promote mutual understanding, leadership development, educational transformation and democratic ideals.''
				\end{itemize}
			\item SportsUnited: \vspace{-0.1cm}
				\begin{itemize} \itemsep -1pt
				\item \url{http://exchanges.state.gov/sports/index.html}
				\item SportsUnited is an international sports programming initiative designed to help start a dialogue at the grassroots level with non-elite boys and girls ages 7-17.
				\item The programs aid youth in discovering how success in athletics can be translated into the development of life skills and achievement in the classroom.
				\item Foreign participants are given an opportunity to establish links with U.S. sports professionals and exposure to American life and culture.
				\item Americans learn about foreign cultures and the challenges young people from overseas face today.
				\item The U.S. Department of State has programmed initiatives in: baseball, basketball, football, track and field, soccer, volleyball, wrestling, archery, boxing, swimming, fencing, table tennis, ice skating, weightlifting, water polo and managing sports community centers.
				\item Countries covered by this program are listed on the web page.
				\item Sports Envoy Program: \vspace{-0.1cm}
					\begin{itemize} \itemsep -1pt
					\item \url{http://exchanges.state.gov/sports/envoy1.html}
					\item Working with the national sports leagues and the U.S. Olympic Committee, athletes and coaches in various sports are chosen to serve as envoys or ambassadors of sport in overseas programs that include conducting clinics, visiting schools and speaking to youth.
					\item The American athletes and coaches conduct drills and team building activities, as well as engage the youth in a dialogue on the importance of an education, positive health practices and respect for diversity.
					\end{itemize}
				\item Sports Grant Competition: \vspace{-0.1cm}
					\begin{itemize} \itemsep -1pt
					\item The Bureau of Educational and Cultural Affairs (ECA) has an annual open competition under its International Sports Programming Initiative.
					\item Public and private non-profit organizations, 501(c)(3), may submit proposals to discuss approaches designed to enhance and improve the infrastructure of youth sports programs.
					\item The focus of all programs must be reaching out to non-elite youth ages 7-17 and/or their coaches/administrators.
					\item There are four themes that a proposal can address; Youth Sports Management, Training Sports Coaches, Sport and Disability, and Sport and Health.
					\item The list of eligible countries changes each year.
					\item \url{http://exchanges.state.gov/sports/index/sports-grant-competition.html}
					\end{itemize}
				\item Sports Visitor Program: \vspace{-0.1cm}
					\begin{itemize} \itemsep -1pt
					\item Nominated by our U.S. embassies overseas, selected athletes, managers and coaches are brought to the U.S. for technical sports training, sports management, conflict resolution training and exposure to valuable U.S. sports contacts and then are encouraged to return to conduct in-country clinics for youth with their newly learned skills.
					\item \url{http://exchanges.state.gov/sports/visitors.html}
					\end{itemize}
				\end{itemize}
			\end{enumerate}
		\end{enumerate}
	\end{enumerate}
\item U.S. Department of Labor: \vspace{-0.3cm}
	\begin{enumerate} \itemsep -2pt
	\item Wage and Hour Division: \vspace{-0.2cm}
		\begin{enumerate} \itemsep -2pt
		\item YouthRules!: \vspace{-0.1cm}
			\begin{enumerate} \itemsep -1pt
			\item \url{http://youthrules.dol.gov/}
			\item Has information for youths, parents, educators, and employers on how to let youth work part-time safely
			\item Teens: \url{http://youthrules.dol.gov/teens/default.htm}
			\item Parents: \url{http://youthrules.dol.gov/parents/default.htm}
			\item Educators: \url{http://youthrules.dol.gov/educators/default.htm}
			\item Employers: \url{http://youthrules.dol.gov/employers/default.htm}
			\item Resources: \url{http://youthrules.dol.gov/resources.htm}
			\item Compliance Assistance: \url{http://youthrules.dol.gov/ca.htm}
			\end{enumerate}
		\end{enumerate}
	\end{enumerate}
\item ASCL Educational Services, Inc. (Marc McCulloch): \vspace{-0.3cm}
	\begin{enumerate} \itemsep -2pt
	\item Transitions: Life Skills for Personal Success!: \vspace{-0.2cm}
		\begin{enumerate} \itemsep -2pt
		\item Curriculum \& Materials: \url{http://transitions.ascl.info/infomaterials}
		\item Soft Skills: \url{http://transitions.ascl.info/infoskills}
		\end{enumerate}
	\end{enumerate}
\item Partnership for 21st Century Skills: \vspace{-0.3cm}
	\begin{enumerate} \itemsep -2pt
	\item \url{http://www.p21.org/}
	\item Framework for 21st Century Learning: \url{http://www.p21.org/index.php?option=com_content&task=view&id=254&Itemid=119}
	\item Tools and Resources: \url{http://www.p21.org/index.php?option=com_content&task=view&id=273&Itemid=139}
	\end{enumerate}
\item National Career and Technical Education Foundation (NCTEF): \vspace{-0.3cm}
	\begin{enumerate} \itemsep -2pt
	\item States' Career Clusters Initiative (SCCI): \vspace{-0.2cm}
		\begin{enumerate} \itemsep -2pt
		\item \url{http://www.careerclusters.org/}
		\item The 16 Career Clusters: \url{http://www.careerclusters.org/16clusters.cfm}
		\item Plans of Study: \url{http://www.careerclusters.org/resources/web/pos.cfm}
		\item Knowledge and Skills Charts: \url{http://www.careerclusters.org/resources/web/ks.php}
		\item Crosswalks: \url{http://www.careerclusters.org/crosswalks.cfm}
		\item Publications: \url{http://www.careerclusters.org/publications.php}
		\item Sixteen Career Clusters and Their Pathways: \url{http://www.careerclusters.org/list16clusters.php}
		\item Career Clusters Models: \url{http://www.careerclusters.org/resources/web/16ccall.php?action=models}
		\item Career Clusters Brochure Previews: \url{http://www.careerclusters.org/resources/web/16ccall.php?action=brochures}
		\item Career Clusters Interest Survey: \url{http://www.careerclusters.org/ccinterestsurvey.php}
		\item Related Websites: \url{http://www.careerclusters.org/related.php}
		\end{enumerate}
	\end{enumerate}
\item U. S. Department of Labor: \vspace{-0.3cm}
	\begin{enumerate} \itemsep -2pt
	\item Employment and Training Administration: \vspace{-0.2cm}
		\begin{enumerate} \itemsep -2pt
		\item CareerOneStop: \vspace{-0.1cm}
			\begin{enumerate} \itemsep -1pt
			\item \url{http://www.careeronestop.org/}
			\item Students, parents, and career advisors: \url{http://www.careeronestop.org/studentsandcareeradvisors/studentsandcareeradvisors.aspx}
			\end{enumerate}
		\end{enumerate}
	\end{enumerate}
\item U. S. Department of Defense: \vspace{-0.3cm}
	\begin{enumerate} \itemsep -2pt
	\item ASVAB Career Exploration Program: \vspace{-0.2cm}
		\begin{enumerate} \itemsep -2pt
		\item \url{http://www.asvabprogram.com/}
		\item Learn about yourself: \url{http://www.asvabprogram.com/index.cfm?fuseaction=learn.main}
		\item Explore careers: \url{http://www.asvabprogram.com/index.cfm?fuseaction=explore.main}
		\item Plan for your future: \url{http://www.asvabprogram.com/index.cfm?fuseaction=plan.main}
		\item Information for educators and career counselors: \url{http://www.asvabprogram.com/index.cfm?fuseaction=edu.main}
		\item Information for parents: \url{http://www.asvabprogram.com/index.cfm?fuseaction=parents.main}
		\end{enumerate}
	\end{enumerate}
\end{enumerate}



%%%%%%%%%%%%%%%%%%%%%%%%%%%%%%%%%%%%%%%%%%%
\section{Internship Opportunities}
\label{Internship Opportunities}

Internship opportunities: \vspace{-0.3cm}
\begin{enumerate} \itemsep -4pt
\item Canada: \vspace{-0.3cm}
	\begin{enumerate} \itemsep -2pt
	\item SWAP: \vspace{-0.2cm}
		\begin{enumerate} \itemsep -2pt
		\item \url{http://www.swap.ca/}
		\item For Canadians who want to work abroad: \url{http://www.swap.ca/out_eng/index.aspx}
		\item For citizens of selected countries who want to work in Canada: \url{http://www.swap.ca/in_eng/partner_organizations.aspx}
		\end{enumerate}
	\end{enumerate}
\item Singapore: \vspace{-0.3cm}
	\begin{enumerate} \itemsep -2pt
	\item Speedwing Training (Asia) Pte Ltd: \vspace{-0.2cm}
		\begin{enumerate} \itemsep -2pt
		\item \url{http://www.speedwing.org/}
		\item For Singaporeans who want to work in the United States, Canada, Germany, and New Zealand
		\item For citizens of selected countries who want to work in Singapore
		\end{enumerate}
	\end{enumerate}
\end{enumerate}

%%%%%%%%%%%%%%%%%%%%%%%%%%%%%%%%%%%%%%%%%%%
\subsection{Internship Opportunities in Australia}
\label{internshipaus}

Internship Opportunities in Australia: \vspace{-0.3cm}
\begin{enumerate} \itemsep -4pt
\item The Association of Professional Engineers, Scientists and Managers, Australia: \url{http://www.apesma.asn.au/index.asp} --- Ask for guide to internships in your region/major; free student membership
\item Engineers Australia: \url{http://www.engineersaustralia.org.au/} --- Ask for guide to internships in your region/major; free student membership
\item CPA Australia: \url{http://www.cpaaustralia.com.au/cps/rde/xchg/cpa/hs.xsl/index.html} and \url{http://www.cpaaustralia.com.au/cps/rde/xchg/careers/site/index_ENA_HTML.htm/cps/rde/xchg/SID-3F57FECB-EEFEF50E/careers/site/204_ENA_HTML.htm}
\item Institute of Chartered Accountants in Australia: \url{http://www.charteredaccountants.com.au/}
\item 
\end{enumerate}


%%%%%%%%%%%%%%%%%%%%%%%%%%%%%%%%%%%%%%%%%%%
\subsection{Internship Opportunities in Europe}
\label{internshipeu}

Internship Opportunities in Portugal: \vspace{-0.3cm}
\begin{enumerate} \itemsep -4pt
\item Portugal: \vspace{-0.3cm}
	\begin{enumerate} \itemsep -2pt
	\item IAESTE Portugal (The International Association for the Exchange of Students for Technical Experience): \url{http://www.iaeste.pt/en/foreign-trainees/why-portugal/}
	\end{enumerate}
\item United Kingdom: \vspace{-0.3cm}
	\begin{enumerate} \itemsep -2pt
	\item Graduate Talent Pool: \url{http://graduatetalentpool.direct.gov.uk/}
	\end{enumerate}
\end{enumerate}




%%%%%%%%%%%%%%%%%%%%%%%%%%%%%%%%%%%%%%%%%%%
\subsection{Internship Opportunities in the United States}
\label{internshipsus}

Internship Opportunities in the United States: \vspace{-0.3cm}
\begin{enumerate} \itemsep -4pt
\item Use the Procedure \proc{Find}$(\varphi, \tau)$ in \S\ref{heuristiclocateoutreach} to look up internship opportunities and lists of internship opportunities.

Look at government organizations (e.g., the White House), nonprofit organizations (e.g., Engineers Without Borders), professional organizations (e.g., IEEE and ACM), colleges and universities, and companies (e.g., Intel, Google, and start-ups).

You can start your search by looking at the organizations that provide resources for underrepresented minorities as well as resources for scholarships and fellowships. These information can be found in other sections of this document.

If you do not know where to start, speak to a professor or staff member at the career center of your college/university. Alternatively, you can ask your awesome resident advisors (RAs).

My personal advice is to start your search based on your interests and skill set. You can always narrow the search space based on factors, such as geographical location, later on.

Competitive internships, especially research internships in electrical and computer engineering or computer science, weed out many students from applying via demanding job requirements. For example, if you want to apply for research internships with electronic design automation (EDA) companies and corporate research labs, you would need to have significant experience designing integrated circuits and developing EDA software. The stringent job requirements also mean that students need to plan in advance (say, about a year) about the internships that they would like to seek, and plan to acquire the necessary skill set and experiences before the application deadlines (which can be several months before the start of your internship).

Taking as many challenging classes as you can possibly cope, especially in electrical and computer engineering or computer science, would provide you with a skill set that allows you to apply for competitive internships in many fields. Apart from taking challenging classes as well as engaging in research and/or open source projects, you can try to acquire additional skills and experience in your free time to boost the competitiveness of your internship application. Certain skills and experiences, such as compiler design, are hard to acquire in your free time, so it would be ``easier'' to take classes that would help you acquire those skills and experiences.

Note that you may want to look into creating your own entrepreneurial venture, say an EDA start-up or organization in social entrepreneurship, rather than to seek an internship. Also, seeking an internship abroad is always a good addition to your resume/CV.
\item National Science Foundation: \vspace{-0.3cm}
	\begin{enumerate} \itemsep -2pt
	\item Research Experiences for Undergraduates (REU): \vspace{-0.2cm}
		\begin{enumerate} \itemsep -2pt
		\item \url{http://www.nsf.gov/crssprgm/reu/reu_search.cfm}
		\item Academic fields: \vspace{-0.1cm}
			\begin{enumerate} \itemsep -1pt
			\item Astronomical Sciences
			\item Atmospheric and Geospace Sciences
			\item Biological Sciences
			\item Chemistry
			\item Computer and Information Science and Engineering
			\item Cyberinfrastructure
			\item Department of Defense (DoD)
			\item Earth Sciences
			\item Education and Human Resources
			\item Engineering
			\item Ethics and Values Studies
			\item International Science and Engineering
			\item Materials Research
			\item Mathematical Sciences
			\item Ocean Sciences
			\item Physics
			\item Polar Programs
			\item Social, Behavioral, and Economic Sciences
			\end{enumerate}
		\end{enumerate}
	\end{enumerate}
\item Society for Industrial and Applied Mathematics: \vspace{-0.3cm}
	\begin{enumerate} \itemsep -2pt
	\item Internship and Career Information in Industry, Research Institutions, and Government Labs: \url{http://www.siam.org/careers/internships.php}
	\end{enumerate}
\item American Institute of Physics (AIP): \vspace{-0.3cm}
	\begin{enumerate} \itemsep -2pt
	\item Society of Physics Students (SPS): \vspace{-0.2cm}
		\begin{enumerate} \itemsep -2pt
		\item SPS Internships: \url{http://www.spsnational.org/programs/internships/}
		\item Research Opportunities: \url{http://www.spsnational.org/programs/research/}
		\end{enumerate}
	\end{enumerate}
%%%%%%%%%%%%%%%%%%%%%%%%%%%%%%%%%%%%%%
%%%%%%%%%%%%%%%%%%%%%%%%%%%%%%%%%%%%%%
%%%%%%%%%%%%%%%%%%%%%%%%%%%%%%%%%%%%%%
\item United States Office of Personnel Management: \vspace{-0.3cm}
	\begin{enumerate} \itemsep -2pt
	\item USAJOBS: \vspace{-0.2cm}
		\begin{enumerate} \itemsep -2pt
		\item Student Jobs: \url{http://www.usajobs.gov/studentjobs/}
		\end{enumerate}
	\end{enumerate}
%%%%%%%%%%%%%%%%%%%%%%%%%%%%%%%%%%%%%%
%%%%%%%%%%%%%%%%%%%%%%%%%%%%%%%%%%%%%%
%%%%%%%%%%%%%%%%%%%%%%%%%%%%%%%%%%%%%%
\item Americans for the Arts: \vspace{-0.3cm}
	\begin{enumerate} \itemsep -2pt
	\item Internship Program: \url{http://www.americansforthearts.org/about_us/internships.asp}
	\end{enumerate}
\item New York Women's Foundation: \vspace{-0.3cm}
	\begin{enumerate} \itemsep -2pt
	\item Internship Opportunities: \url{http://www.nywf.org/internship.html}
	\item Volunteer Opportunities: \url{http://www.nywf.org/volunteer.html}
	\end{enumerate}
\item Council on International Educational Exchange (CIEE): \url{http://www.ciee.org/hire/index.aspx}
\item The John F. Kennedy Center for the Performing Arts: \vspace{-0.3cm}
	\begin{enumerate} \itemsep -2pt
	\item Kennedy Center Arts Management Internships: \url{http://www.kennedy-center.org/education/artsmanagement/internships/}
	\end{enumerate}
\item Washington Performing Arts Society (WPAS): \vspace{-0.3cm}
	\begin{enumerate} \itemsep -2pt
	\item Internships with WPAS: \vspace{-0.2cm}
		\begin{enumerate} \itemsep -2pt
		\item \url{http://www.wpas.org/aboutwpas/opportunities/intern.aspx}
		\item ``WPAS offers internships throughout the year. Applicants should be highly motivated, creative and hard-working individuals with an interest in all aspects of arts management. It is required that applicants have previous office experience.''
		\item In addition, applicants should possess: \vspace{-0.1cm}
			\begin{enumerate} \itemsep -1pt
			\item Interest/background in music, dance or performance art
			\item Strong organizational skills
			\item Effective writing and communication skills
			\item Ability to learn quickly, handle multiple tasks, take initiative, and work independently with little supervision
			\item High energy level and ability to work well in deadline and/or pressure situations
			\item Computer literacy
			\end{enumerate}
		\item ``WPAS interns leave our offices with a better understanding of arts management, knowledge of artists in a variety of fields (classical music, world music, dance and performance art), contacts in theaters throughout the D.C. metro area, practical experience and a portfolio of work. The internship is unpaid, however stipends are occasionally granted during the performance year (September - May). Interns are also invited to attend many WPAS performances on a complimentary basis.''
		\item Types of internships: \vspace{-0.1cm}
			\begin{enumerate} \itemsep -1pt
			\item Accounting Internship
			\item Development Internship
			\item Education Internship
			\item Marketing/Public Relations Internship
			\item Office Administration Internship
			\item Programming Internship
			\end{enumerate}
		\end{enumerate}
	\end{enumerate}
\item The Washington Ballet: Internships, \url{http://www.washingtonballet.org/about-twb/auditions-employment/#internships}
\item The Choral Arts Society of Washington: \vspace{-0.3cm}
	\begin{enumerate} \itemsep -2pt
	\item Internship and Apprenticeship Program: \url{http://www.choralarts.org/About-Us/Internships-and-Apprenticeships.aspx}
	\end{enumerate}
\item League of American Orchestras: Internships, \url{http://www.americanorchestras.org/career_center/internships.html}
%%%%%%%%%%%%%%%%%%%%%%%%%%%%%%%%%%%%%%
%%%%%%%%%%%%%%%%%%%%%%%%%%%%%%%%%%%%%%
%%%%%%%%%%%%%%%%%%%%%%%%%%%%%%%%%%%%%%
\item Congressional Hispanic Caucus Institute (CHCI): \vspace{-0.3cm}
	\begin{enumerate} \itemsep -2pt
	\item CHCI United Health Foundation Scholars: \vspace{-0.2cm}
		\begin{enumerate} \itemsep -2pt
		\item \url{http://www.chci.org/scholarships/page/chci-united-health-foundation-scholars-}
		\item In addition to providing scholarship opportunities for Latino youth, the United Health Foundation decided to partner with CHCI to create a six-month internship program for students interested in the medical field.
		\item Seventeen participants enrolled in either a full-time undergraduate or graduate course of study at an accredited two- or four-year college, university, vocational or technical school were selected.
		\end{enumerate}
	\item CHCI Congressional Internship: \vspace{-0.2cm}
		\begin{enumerate} \itemsep -2pt
		\item The purpose of the Congressional Internship Program (CIP) is to expose young Latinos to the legislative process and to strengthen their professional and leadership skills, ultimately promoting the presence of Latinos on Capitol Hill.
		\item The Congressional Internship Program provides college students with a paid Congressional work placement on Capitol Hill for a period of twelve weeks (Spring/Fall) or eight weeks (Summer). This unmatched experience allows students to learn first hand about our nation's legislative process.
		\end{enumerate}
	\end{enumerate}
\item Mexican American Legal Defense and Educational Fund (MALDEF): Law Clerk Summer Internship program, \url{http://maldef.org/about/jobs/index.html}
\item Hispanic Association of Colleges and Universities (HACU): \vspace{-0.3cm}
	\begin{enumerate} \itemsep -2pt
	\item HACU National Internship Program (HNIP): \url{http://www.hacu.net/hacu/HNIP_EN.asp}
	\end{enumerate}
%%%%%%%%%%%%%%%%%%%%%%%%%%%%
\item Smithsonian Institution: \vspace{-0.3cm}
	\begin{enumerate} \itemsep -2pt
	\item Smithsonian Institution Traveling Exhibition Service (SITES): \vspace{-0.2cm}
		\begin{enumerate} \itemsep -2pt
		\item Internship programs: \url{http://www.sites.si.edu/interns/internships.htm}
		\item ``The Smithsonian Institution Traveling Exhibition Service internship programs allows people with diverse interests, strengths, and goals to experience an educational environment where they can work and learn from professionals in the museum field.''
		\item ``SITES offers internship opportunities in a variety of different areas: public relations, development (fund raising), research, and project design.''
		\end{enumerate}
	\item Smithsonian Folkways Recordings (or simply, Smithsonian Folkways): \vspace{-0.2cm}
		\begin{enumerate} \itemsep -2pt
		\item Internships: \url{http://www.folkways.si.edu/about_us/jobs.aspx}
		\end{enumerate}
	\item Freer Gallery of Art / Arthur M. Sackler Gallery: \vspace{-0.2cm}
		\begin{enumerate} \itemsep -2pt
		\item Internships: \url{http://www.asia.si.edu/research/internships.asp}
		\end{enumerate}
	\item National Museum of American History: \vspace{-0.2cm}
		\begin{enumerate} \itemsep -2pt
		\item Jerome and Dorothy Lemelson Center for the Study of Invention and Innovation: \vspace{-0.1cm}
			\begin{enumerate} \itemsep -1pt
			\item Archival Internships: \url{http://invention.smithsonian.org/resources/research_interns.aspx}
			\end{enumerate}
		\end{enumerate}
	\end{enumerate}
%%%%%%%%%%%%%%%%%%%%%%%%%%%%
\item Council on International Educational Exchange (CIEE): \vspace{-0.3cm}
	\begin{enumerate} \itemsep -2pt
	\item CIEE's Trainee Program: \vspace{-0.2cm}
		\begin{enumerate} \itemsep -2pt
		\item part of the J-1 visa category of the US government�s Exchange Visitor Program
		\item \url{http://www.ciee.org/trainee/}
		\end{enumerate}
	\item CIEE Work \& Travel USA; and Internship USA: \vspace{-0.2cm}
		\begin{enumerate} \itemsep -2pt
		\item \url{http://www.ciee.org/hire/}
		\item \url{http://www.ciee.org/wat/}
		\end{enumerate}
	\end{enumerate}
\item American Institute For Foreign Study (AIFS): \vspace{-0.3cm}
	\begin{enumerate} \itemsep -2pt
	\item Camp America Counselors and Summer Staff: \url{http://www.aifs.com/work_travel.asp}
	\item Au Pair Placement: \url{http://www.aifs.com/au_pair.asp}
	\end{enumerate}
\item U.S. Department of State: \vspace{-0.3cm}
	\begin{enumerate} \itemsep -2pt
	\item Bureau of Educational and Cultural Affairs: \vspace{-0.2cm}
		\begin{enumerate} \itemsep -2pt
		\item International cultural programs: \url{http://exchanges.state.gov/cultural/related-cultural-programs.html}
		\item Office of Global Educational Programs: \vspace{-0.1cm}
			\begin{enumerate} \itemsep -1pt
			\item Camp Counselor: \vspace{-0.1cm}
				\begin{itemize} \itemsep -1pt
				\item \url{http://exchanges.state.gov/jexchanges/programs/camp.html}
				\item Camp counselors interact with groups of American youth by overseeing their camp activities during the U.S. summer.
				\item Through the Camp Counselor program, American campers have the chance to gain knowledge of foreign cultures, while foreign participants increase their knowledge of American culture.
				\item Participants must be at least 18 years of age and may work as counselors in U.S. summer camps for up to four months. Extensions are not allowed. They receive a combination a pay and benefits equal to Americans who work in the same position.
				\end{itemize}
			\end{enumerate}
		\item Private Sector Exchange office: \vspace{-0.1cm}
			\begin{enumerate} \itemsep -1pt
			\item \url{http://exchanges.state.gov/jexchanges/index.html}
			\item The Private Sector Exchange office designates, monitors and partners with U.S. organizations, including government agencies, academic institutions, educational and cultural organizations, and corporations, that administer the Exchange Visitor Program.
			\item Au Pair program: \vspace{-0.1cm}
				\begin{itemize} \itemsep -1pt
				\item Through the Au Pair program, foreign nationals between 18 and 26 years of age participate in the home life of a host family. Au pairs provide limited childcare services for up to 12 months. An extension of 6, 9, or 12 months may be granted in certain cases.
				\item \url{http://exchanges.state.gov/jexchanges/programs/aupair.html}
				\end{itemize}
			\item Internships: \vspace{-0.1cm}
				\begin{itemize} \itemsep -1pt
				\item \url{http://exchanges.state.gov/jexchanges/programs/intern.html}
				\item Internship programs are designed to allow foreign professionals to come to the United States to gain exposure to U.S. culture and to receive training in U.S. business practices in their chosen occupational field.
				\item The maximum duration of an internship in any occupational field is 12 months.
				\item Upon completion of their exchange programs, participants are expected to return to their home countries.
				\item The State Department allows internships in the following occupational categories: \vspace{-0.1cm}
					\begin{itemize} \itemsep -1pt
					\item Agriculture, Forestry, and Fishing
					\item Arts and Culture
					\item Construction and Building Trades
					\item Education, Social Sciences, Library Science, Counseling and Social Services
					\item Health Related Occupations
					\item Hospitality and Tourism
					\item Information Media and Communications
					\item Management, Business, Commerce and Finance
					\item Public Administration and Law
					\item The Sciences, Engineering, Architecture, Mathematics, and Industrial Occupations.
					\end{itemize}
				\item An Intern must be a foreign national: \vspace{-0.1cm}
					\begin{itemize} \itemsep -1pt
					\item Who is currently enrolled in and pursuing studies at a foreign degree- or certificate-granting post-secondary academic institution outside the United States, or
					\item Who has graduated from such an institution no more than 12 months prior to his or her exchange visitor program start date.
					\end{itemize}
				\item Interns cannot work in unskilled or casual labor positions, in positions that require or involve child care or elder care, or in any kind of position that involves medical patient care or contact. Nor can interns work in positions that require more than 20 per cent clerical or office support work.
				\end{itemize}
			\item The Summer Work Travel Program: \vspace{-0.1cm}
				\begin{itemize} \itemsep -1pt
				\item \url{http://exchanges.state.gov/jexchanges/programs/swt.html}
				\item In the summer work travel program, post-secondary students may enter the United States to work and travel during their summer vacation.
				\item Participants can be admitted to the program more than once.
				\item The maximum length of the program is four months.
				\item Most of the time, participants work in unskilled service positions at resorts, hotels, restaurants, and amusement parks. However, they may also work in other types of organizations.
				\item For example, they could work in architectural firms, scientific research organizations, graphic art/publishing and other media communication businesses, advertising agencies, computer software and electronics firms, legal offices, etc.
				\item The program may not exceed four-months and must be finished during the student's summer vacation.
				\item Participants receive pay and benefits equal to an American working in the same or similar position.
				\end{itemize}
			\item Training programs: \vspace{-0.1cm}
				\begin{itemize} \itemsep -1pt
				\item \url{http://exchanges.state.gov/jexchanges/programs/trainee.html}
				\item Training programs are designed to allow foreign professionals to come to the United States to gain exposure to U.S. culture and to receive training in U.S. business practices in their chosen occupational field.
				\item Foreign nationals have had the opportunity to train with some of the finest employers in the U.S., gaining real time experience in their chosen career fields.
				\item Upon completion of their exchange programs, participants are expected to return to their home countries to utilize their newly learned skills and knowledge to advance their careers and share their experiences with their communities.
				\item The State Department allows training programs in the following occupational categories: \vspace{-0.1cm}
					\begin{itemize} \itemsep -1pt
					\item Agriculture, Forestry, and Fishing
					\item Arts and Culture
					\item Construction and Building Trades
					\item Education, Social Sciences, Library Science, Counseling and Social Services
					\item Health Related Occupations
					\item Hospitality and Tourism
					\item Information Media and Communications
					\item Management, Business, Commerce and Finance
					\item Public Administration and Law
					\item The Sciences, Engineering, Architecture, Mathematics, and Industrial Occupations.
					\end{itemize}
				\item A trainee must be a foreign national who has: \vspace{-0.1cm}
					\begin{itemize} \itemsep -1pt
					\item A degree or professional certificate from a foreign post-secondary academic institution and at least one year of prior related work experience in his or her occupational field outside the United States, or
					\item Five years of work experience outside the United States in the occupational field in which they are seeking training.
					\end{itemize}
				\end{itemize}
			\item Specialists: \vspace{-0.1cm}
				\begin{itemize} \itemsep -1pt
				\item \url{http://exchanges.state.gov/jexchanges/programs/specialist.html}
				\item This category is for a participant who is an expert in a field of specialized knowledge or skill who will demonstrate such skills in the United States. Such exchanges are to provide opportunities to increase the exchange knowledge and ideas between American and foreign specialists. The maximum duration of this program is one year.
				\item This category is for foreign nationals who are experts in a field of specialized knowledge or skill, coming to the United States for observing, consulting, or demonstrating their special skills, except: Professors and Research Scholars, Short-Term Scholars, and Alien Physicians.
				\item Individuals participating in the specialist program are: \vspace{-0.1cm}
					\begin{itemize} \itemsep -1pt
					\item Experts in a field of specialized knowledge or skill;
					\item Seeks to travel to the United States for the purpose of observing, consulting, or demonstrating their special knowledge or skills;
					\item Does not fill a permanent or long-term position of employment while in the U.S.
					\end{itemize}
				\end{itemize}
			\item International Visitor: \vspace{-0.1cm}
				\begin{itemize} \itemsep -1pt
				\item \url{http://exchanges.state.gov/jexchanges/programs/intl_visitor.html}
				\item The international visitor category enables visitors to better understand American culture and enhanced American knowledge of foreign cultures.
				\item This category is for individuals who are recognized as potential leaders in their own country, selected by the Department of State to participate in observation tours, discussions, consultation, professional meetings, conferences, workshops and travel.
				\item The maximum duration of the program is one year.
				\end{itemize}
			\item Alien Physician: \vspace{-0.1cm}
				\begin{itemize} \itemsep -1pt
				\item \url{http://exchanges.state.gov/jexchanges/programs/physician.html}
				\item The Alien Physician program is for foreign national physicians seeking entry into U.S. graduate medical education programs or training at accredited U.S. schools of medicine or other U.S. institutions.
				\item There are generally two types of exchange programs in which a foreign national physician (also referred to as a foreign/international medical graduate) participates: \vspace{-0.1cm}
					\begin{itemize} \itemsep -1pt
					\item Clinical training in the �alien physician� category
					\item Non-Clinical training in the �research scholar� category
					\end{itemize}
				\end{itemize}
			\item FORTUNE/U.S. State Department Global Women's Mentoring Partnership: \vspace{-0.1cm}
				\begin{itemize} \itemsep -1pt
				\item \url{http://exchanges.state.gov/citizens/professionals/fortunepartnership.html}
				\item This public-private partnership places talented, emerging women leaders from all over the world in mentoring programs with FORTUNE's Most Powerful Women Leaders.
				\item For three weeks, American and international participants work together in mentoring relationships to share the skills and experiences necessary for strengthening women�s leadership.
				\end{itemize}
			\item American Council of Young Political Leaders (ACYPL): \vspace{-0.1cm}
				\begin{itemize} \itemsep -1pt
				\item \url{http://exchanges.state.gov/citizens/profs/acypl.html}
				\item \url{http://www.acypl.org/}
				\item For 44 years, the American Council of Young Political Leaders (ACYPL) has designed, organized and managed unique international educational exchanges for young political leaders (ages 25-40) worldwide.
				\item ACYPL programs are designed to promote mutual understanding, respect, and friendship and to cultivate long-lasting relationships among young people who are poised to become tomorrow's global leaders and policy makers.
				\item American participants are nominated by members of Congress, governors, political party leaders, and ACYPL alumni, while international delegates are selected from countries where ACYPL is currently conducting programs by international program partners with the U.S. Embassy input.
				\end{itemize}
			\item Edward R. Murrow Program for Journalists: \vspace{-0.1cm}
				\begin{itemize} \itemsep -1pt
				\item \url{http://exchanges.state.gov/ivlp/murrow.html}
				\item The Edward R. Murrow Program for Journalists invites rising international journalists to travel to the United States and examine journalistic principles and practices.
				\end{itemize}
			\end{enumerate}
		\item Office of Citizen Exchanges: \vspace{-0.1cm}
			\begin{enumerate} \itemsep -1pt
			\item Youth Programs Division: \vspace{-0.1cm}
				\begin{itemize} \itemsep -1pt
				\item \url{http://exchanges.state.gov/youth/index.html}
				\item The Youth Programs Division is committed to empowering the successor generation and establishing long-lasting ties between the United States and other countries through exchange programs and institutional partnerships.
				\item Programs focus primarily on secondary schools and promote mutual understanding, leadership development, educational transformation, and democratic ideals.
				\item Year-Long Programs, Short Term Programs, and Virtual Partnerships: \url{http://exchanges.state.gov/youth/programs-by-type.html}
				\item Programs for Young Americans, and Programs for International Students and Teachers: \url{http://exchanges.state.gov/youth/programs-by-participants.html}
				\item Opportunities for American Hosts: Families and Schools, \url{http://exchanges.state.gov/youth/opps-for-am-hosts.html}
				\item Programs for High School Students: \url{http://exchanges.state.gov/youth/programs.html}
				\end{itemize}
			\item Professional Exchanges Division: \vspace{-0.1cm}
				\begin{itemize} \itemsep -1pt
				\item \url{http://exchanges.state.gov/citizens/profs.html}
				\item The Professional Exchanges division provides grants to U.S. nonprofit organizations to carry out exchange programs that support the professional development of foreign participants. The purpose of each exchange program is to engage with foreign leaders in critical professions, to demonstrate respect for foreign cultures, and to promote mutual understanding between the people of the United States and other countries.
				\item Professional exchanges typically last several years and include internships, study tours or workshops in the United States and in the host country. Participants come from a variety of professions including education administrators, public servants, journalists, labor union officials, entrepreneurs, environmental leaders, jurists, lawyers, and civic leaders.
				\item ECA grant opportunities: \vspace{-0.1cm}
					\begin{itemize} \itemsep -1pt
					\item Open Funding Opportunities: Requests For Grant Proposals (RFGPs), \url{http://exchanges.state.gov/grants/open2.html}
					\item Grants.gov: \url{http://www.grants.gov/}
					\end{itemize}
				\item Grants by Region: \vspace{-0.1cm}
					\begin{itemize} \itemsep -1pt
					\item \url{http://exchanges.state.gov/citizens/professionals/grant-region.html}
					\item Africa 
					\item East Asia and the Pacific 
					\item Europe and Eurasia 
					\item North Africa and the Middle East 
					\item South and Central Asia 
					\item Western Hemisphere 
					\item Multi-regional
					\end{itemize}
				\end{itemize}
			\end{enumerate}
		\end{enumerate}
	\end{enumerate}
\end{enumerate}






%%%%%%%%%%%%%%%%%%%%%%%%%%%%%%%%%%%%%%%%%%%
\section{Resources on Studying Abroad}
\label{resourcesonstudyingabroad}

Resources on studying abroad: \vspace{-0.3cm}
\begin{enumerate} \itemsep -4pt
\item Council on International Educational Exchange (CIEE): \vspace{-0.3cm}
	\begin{enumerate} \itemsep -2pt
	\item Study abroad programs for high school students from the United States: \vspace{-0.2cm}
		\begin{enumerate} \itemsep -2pt
		\item \url{http://www.ciee.org/hsabroad/index.html}
		\item \url{http://www.ciee.org/hsabroad/high-school-study-abroad/index.html}
		\item These programs include:: \vspace{-0.1cm}
			\begin{enumerate} \itemsep -1pt
			\item High School Abroad programs (for U.S. high school students)
			\item Summer High School Abroad programs (for U.S. high school students)
			\item Gap Year Abroad programs (for recent U.S. high school graduates)
			\end{enumerate}
		\end{enumerate}
	\end{enumerate}
\item U.S. Department of State: \vspace{-0.3cm}
	\begin{enumerate} \itemsep -2pt
	\item Bureau of Educational and Cultural Affairs: \vspace{-0.2cm}
		\begin{enumerate} \itemsep -2pt
		\item Office of Global Educational Programs: \vspace{-0.1cm}
			\begin{enumerate} \itemsep -1pt
			\item EducationUSA: \vspace{-0.1cm}
				\begin{itemize} \itemsep -1pt
				\item EducationUSA is a network of more than 400 student advising centers, which offer accurate, comprehensive, objective and timely information about educational opportunities in the United States and guidance to qualified individuals on how best to access those opportunities. This includes information about application procedures, standardized test requirements, student visas, financial aid, and the full range of accredited U.S. higher education institutions.
				\item \url{http://exchanges.state.gov/globalexchanges/index/educationusa.html}
				\item \url{http://www.educationusa.state.gov/} and \url{http://www.educationusa.info/centers.php}
				\end{itemize}
			\item Open Doors: \vspace{-0.1cm}
				\begin{itemize} \itemsep -1pt
				\item The Educational Information and Resources Branch funds Open Doors, a census of foreign students and scholars in the U.S. and of U.S. students studying abroad published annually by the Institute for International Education.
				\item Open Doors data is used by U.S. embassies, the Departments of State, Commerce, and Education, and U.S. colleges and universities to inform policy decisions about educational exchanges, trade in educational services, and study abroad activity.
				\item \url{http://exchanges.state.gov/globalexchanges/index/open_doors.html}
				\item \url{http://www.opendoors.iienetwork.org/}
				\end{itemize}
			\end{enumerate}
		\item EducationUSA: \vspace{-0.1cm}
			\begin{enumerate} \itemsep -1pt
			\item \url{http://educationusa.state.gov/}
			\item For U.S. (college) students who want to study/work abroad: \url{http://www.educationusa.info/pages/students/forus.php}
			\end{enumerate}
		\end{enumerate}
	\end{enumerate}
\item IES Abroad (formerly Institute of European Studies / Institute for the International Education of Students): \vspace{-0.3cm}
	\begin{enumerate} \itemsep -2pt
	\item \url{https://www.iesabroad.org/} and \url{https://www.iesabroad.org/IES/home.html}
	\end{enumerate}
\item Global Learning Semesters, Inc.: \vspace{-0.3cm}
	\begin{enumerate} \itemsep -2pt
	\item Summer in the Mediterranean: \vspace{-0.2cm}
		\begin{enumerate} \itemsep -2pt
		\item \url{http://www.globalsemesters.com/Mediterranean.html}
		\item Has programs in the following areas: \vspace{-0.1cm}
			\begin{enumerate} \itemsep -1pt
			\item Art \& Photography
			\item Early Christianity
			\item Greek Heritage
			\item International Marketing
			\item Music
			\end{enumerate}
		\end{enumerate}
	\end{enumerate}
\item American Institute For Foreign Study (AIFS): \vspace{-0.3cm}
	\begin{enumerate} \itemsep -2pt
	\item \url{http://www.aifs.com/}
	\item College Study Abroad: \url{http://www.aifsabroad.com/}
	\item For high school students: \vspace{-0.2cm}
		\begin{enumerate} \itemsep -2pt
		\item Gifted Education: \url{http://www.aifs.com/gifted_education.asp}
		\item High School Study and Travel: \url{http://www.aifs.com/highschool_study_travel.asp}
		\item Academic Year in America (AYA): \url{http://www.academicyear.org/?source=AIFS}
		\end{enumerate}
	\end{enumerate}
\end{enumerate}





%%%%%%%%%%%%%%%%%%%%%%%%%%%%%%%%%%%%%%%%%%%
\section{College Preparation}
\label{collegepreparation}

College preparation: \vspace{-0.3cm}
\begin{enumerate} \itemsep -4pt
\item {\it Guide to Online Schools} [or {\it GuideToOnlineSchools.com}], {\it The Top 53 College Preparation Resources for Students}. Available at: \url{http://www.guidetoonlineschools.com/tips-and-tools/college-prep-resources}; last accessed on August 25, 2010.
\item U.S. Department of Education's resources for parents to help their children learn: \url{http://www2.ed.gov/parents/academic/help/hyc.html} and \url{http://www2.ed.gov/parents/academic/help/homework/index.html}
\item The College Board: \vspace{-0.3cm}
	\begin{enumerate} \itemsep -2pt
	\item Information about SATs, college preparation, and financial aid
	\item {\it Trends in Higher Education} series 201X: \url{http://trends.collegeboard.org/}
	\item \url{http://www.collegeboard.com/}
	\end{enumerate}
\item {\it Accreditation.org}: \vspace{-0.3cm}
	\begin{enumerate} \itemsep -2pt
	\item Information about the accreditation of engineering degree programs around the world
	\item \url{http://www.accreditation.org/}
	\end{enumerate}
\item {\it New York Times}: \vspace{-0.3cm}
	\begin{enumerate} \itemsep -2pt
	\item The Learning Network: \url{http://learning.blogs.nytimes.com/category/test-yourself/}
	\item New York Times Magazine: \vspace{-0.2cm}
		\begin{enumerate} \itemsep -2pt
		\item The Sep 20, 2010 issue has many articles covering how technology can be used to improve education in K-12 programs. Available online at: \url{http://www.nytimes.com/indexes/2010/09/19/magazine/index.html?ref=magazine}; last accessed on September 20, 2010.
		\item ``New York Times Magazine Features Technology in Education,'' in {\it CCC Blog}, Computing Community Consortium (CCC), Computing Research Association (CRA), Sep 20, 2010. Available online at: \url{http://www.cccblog.org/2010/09/20/new-york-times-magazine-features-technology-in-education/}; last accessed on September 20, 2010.
		\item Articles in this issue discuss: \vspace{-0.1cm}
			\begin{enumerate} \itemsep -1pt
			\item How journalists can make use of technology to automate certain tasks, and improve their productivity and effectiveness in covering news stories
			\item How children can create computer games that introduces them to careers in computing and helps them to develop skills in computational thinking
			\item How to learn things without a lot of rote learning, to have fun while learning, and to use technology to make learning more fun
			\end{enumerate}
		\end{enumerate}
	\end{enumerate}
\item University of Southern California, USC: \vspace{-0.3cm}
	\begin{enumerate} \itemsep -2pt
	\item USC Office of Continuing Education and Summer Programs: \vspace{-0.2cm}
		\begin{enumerate} \itemsep -2pt
		\item \url{http://cesp.usc.edu/}
		\item These programs allow students in K-12 to earn credit at USC, and exposes them to different majors/professions, like medicine, engineering, creative writing, or film making.
		\item Students can benefit from these programs, and learn about different academic disciplines before applying to college. This would help them in their college applications.
		\item Underrepresented minority students can get scholarships to attend these programs. So, if parents have financial difficulty paying for the programs, they can seek financial aid for this.
		\item Also, current undergraduates can also sign up for programs to learn about marketing, finance, and entrepreneurship. They can also do summer research with USC researchers.
		\end{enumerate}
	\item Summer sports programs for youths: \vspace{-0.2cm}
		\begin{enumerate} \itemsep -2pt
		\item SC Futbol Academy (USC Soccer Camps): \url{http://www.usctrojans.com/sports/w-soccer/spec-rel/021610aaa.html}
		\item Mick Haley's USC Girls Volleyball Camp: \url{http://www.usctrojans.com/sports/w-volley/spec-rel/volley-camp.html}
		\item Salo Swim Camp: \url{http://www.saloswimcamp.com/on-line/default.asp}
		\item USC NYSP Trojan KidSCamp: \url{http://sait.usc.edu/recsports/site_content/youth_sports/nysptk.html}
		\item After School Sports Connection, ASSC (operates in fall, spring, and summer): \url{http://sait.usc.edu/recsports/site_content/youth_sports/assc.html}
		\end{enumerate}
	\end{enumerate}
\item Telluride Association: \vspace{-0.3cm}
	\begin{enumerate} \itemsep -2pt
	\item Telluride Association Summer Program (TASP) [ for high school students ]: \url{http://www.tellurideassociation.org/programs/high_school_students/tasp/tasp_general_info.html}
	\item Telluride Association Sophomore Seminar (TASS) [ for high school students ]: \url{http://www.tellurideassociation.org/programs/high_school_students/tass/tass_general_info.html}
	\item Resources for high school educators to nominate summer program applicants: \url{http://www.tellurideassociation.org/programs/high_school_students/hs_resources/hs_resources_general_information.html}
	\end{enumerate}
\item MathNerds: \vspace{-0.3cm}
	\begin{enumerate} \itemsep -2pt
	\item \url{http://www.mathnerds.com/}
	\item ``Provides free, discovery-based, mathematical guidance via an international, volunteer network of mathematicians.''
	\item If you have a mathematical problem to solve, you can ask mathematicans at {\it MathNerds} for help.
	\item They would require you to discuss your attempted approaches/solutions.
	\item If you have not made attempts to solve the problem, they will not give you much guidance.
	\item In addition, they cannot solve problems for you.
	\item They provide guidance for mathematical problems from K-12 material through undergraduate mathematics and statistics classes.
	\item They also provide help for selected topics in advanced mathematics classes (for graduate students).
	\item Other resources: \url{http://www.mathnerds.com/links/links.aspx}
	\end{enumerate}
\item Hobsons: \vspace{-0.3cm}
	\begin{enumerate} \itemsep -2pt
	\item CollegeView (Hobsons' college recruiting services): \url{http://www.collegeview.com/index.jsp}
	\end{enumerate}
\item Sponsors for Educational Opportunity (SEO): \vspace{-0.3cm}
	\begin{enumerate} \itemsep -2pt
	\item Resources: \url{http://www.seo-usa.org/ScholarsResources}
	\end{enumerate}
\item U.S. Department of Education: \vspace{-0.3cm}
	\begin{enumerate} \itemsep -2pt
	\item Students.gov: \url{http://www.students.gov/STUGOVWebApp/index.jsp}
	\item college.gov: \url{http://www.college.gov/wps/portal}
	\end{enumerate}
\item U.S. Department of State: \vspace{-0.3cm}
	\begin{enumerate} \itemsep -2pt
	\item Bureau of Educational and Cultural Affairs: \vspace{-0.2cm}
		\begin{enumerate} \itemsep -2pt
		\item EducationUSA: \vspace{-0.1cm}
			\begin{enumerate} \itemsep -1pt
			\item Information for international students: \url{http://www.educationusa.info/students.php}
			\end{enumerate}
		\end{enumerate}
	\end{enumerate}
\item Congressional Hispanic Caucus Institute (CHCI): \vspace{-0.3cm}
	\begin{enumerate} \itemsep -2pt
	\item CHCI Education Center: \vspace{-0.2cm}
		\begin{enumerate} \itemsep -2pt
		\item \url{http://www.chci.org/education_center/}
		\item Has resources on college planning, financial aid, scholarships, college internships, and housing.
		\item For Parents: \url{http://www.chci.org/education_center/page/for-parents}
		\item For Students: \url{http://www.chci.org/education_center/page/for-students}
		\end{enumerate}
	\end{enumerate}
\item My College Options: \vspace{-0.3cm}
	\begin{enumerate} \itemsep -2pt
	\item \url{http://www.mycollegeoptions.org/}
	\item ``My College Options is a FREE college planning service, offering assistance to students, parents, high schools, counselors, and teachers nationwide.''
	\item ``It is designed to assist high school students in exploring a wide range of post-secondary opportunities, with special emphasis on the college search process.''
	\end{enumerate}
\end{enumerate}

Resources for financial aid: \vspace{-0.3cm}
\begin{enumerate} \itemsep -4pt
\item {\it Guide to Online Schools} [or {\it GuideToOnlineSchools.com}], {\it Financial Aid}. Available at: \url{http://www.guidetoonlineschools.com/financial-aid}; last accessed on August 25, 2010.
\item The Institute for College Access \& Success, {\it Links} [ Resources that provide information about student loans and student debt ]. Available at: \url{http://projectonstudentdebt.org/links.vp.html}; last accessed on September 4, 2010. [ Also, see \url{http://projectonstudentdebt.org/advice.vp.html} and \url{http://ticas.org/about.vp.html}. ]
\end{enumerate}


Information about colleges and universities: \vspace{-0.3cm}
\begin{enumerate} \itemsep -4pt
\item The Institute for College Access \& Success, {\it College InSight}. Available at: \url{http://college-insight.org/}; last accessed on September 4, 2010.
\item 
\end{enumerate}



%%%%%%%%%%%%%%%%%%%%%%%%%%%%%%%%%%%%%%%%%%%
\section{Outreach for Students in Colleges and Universities}
\label{outreachcollege}

Resources to reach out to students in colleges and universities: \vspace{-0.3cm}
\begin{enumerate} \itemsep -4pt
%%%%%%%%%%%%%%%%%%%%%%%%%%%%%
\item Film contests: \vspace{-0.3cm}
	\begin{enumerate} \itemsep -2pt
	\item Ed Wood Film Festival [@ USC]: \vspace{-0.2cm}
		\begin{enumerate} \itemsep -2pt
		\item Celebrating independent filmmaking at USC and named for the famous director, the Ed Wood Film Festival is put on by a committee of Residential Education staff members at New Residential College, chaired by the Cinema Floor RA's.
		\item Teams of students come together to obtain the year's secret theme in which to write, shoot, and edit their very own short film within 24 hours. A week later, the films are shown at USC's Norris Cinema and a panel of judges selects the Festival winners in a variety of categories.
		\item \url{http://sait.usc.edu/resed/Programs.aspx}
		\end{enumerate}
	\item Reel LA: Parkside International Film Festival [or USC Reel LA Film Festival at USC]; see \url{http://www-scf.usc.edu/~pirc/areagov/government.php}
	\end{enumerate}
%%%%%%%%%%%%%%%%%%%%%%%%%%%%%
\item residential education: \vspace{-0.3cm}
	\begin{enumerate} \itemsep -2pt
	\item Telluride Association: \vspace{-0.2cm}
		\begin{enumerate} \itemsep -2pt
		\item Information about how to reside at the Cornell Branch (also known as Telluride House or CBTA) and the Michigan Branch of Telluride Association, which are ``residential colleges'': \url{http://www.tellurideassociation.org/programs/university_students.html}
		\item Awards for residents at the Cornell or Michigan Branch: \url{http://www.tellurideassociation.org/programs/university_students/us_awards.html}
		\end{enumerate}
	\end{enumerate}
%%%%%%%%%%%%%%%%%%%%%%%%%%%%%
\item MathNerds: \vspace{-0.3cm}
	\begin{enumerate} \itemsep -2pt
	\item \url{http://www.mathnerds.com/}
	\item ``Provides free, discovery-based, mathematical guidance via an international, volunteer network of mathematicians.''
	\item If you have a mathematical problem to solve, you can ask mathematicans at {\it MathNerds} for help.
	\item They would require you to discuss your attempted approaches/solutions.
	\item If you have not made attempts to solve the problem, they will not give you much guidance.
	\item In addition, they cannot solve problems for you.
	\item They provide guidance for mathematical problems from K-12 material through undergraduate mathematics and statistics classes.
	\item They also provide help for selected topics in advanced mathematics classes (for graduate students).
	\end{enumerate}
%%%%%%%%%%%%%%%%%%%%%%%%%%%%%
\item Invent Now: \vspace{-0.3cm}
	\begin{enumerate} \itemsep -2pt
	\item 
	\end{enumerate}
\item Journal of Young Investigators (JYI): \vspace{-0.3cm}
	\begin{enumerate} \itemsep -2pt
	\item \url{http://www.jyi.org/}
	\item ``peer-reviewed journal for undergraduates''
	\item ``JYI's web journal (which is also called JYI) is dedicated to the presentation of undergraduate research in science, mathematics, and engineering. It publishes the best submissions from undergraduates, with an emphasis on both the quality of research and the manner in which it is communicated. The journal, JYI, also allows students to experience the other side of the scientific publication process: the review process. Students working with their faculty advisors review the work of their peers and determine whether that work is acceptable for publication in JYI.''
	\end{enumerate}
\item The Recording Academy: \vspace{-0.3cm}
	\begin{enumerate} \itemsep -2pt
	\item GRAMMY U: \vspace{-0.2cm}
		\begin{enumerate} \itemsep -2pt
		\item \url{http://www.grammy365.com/grammy-u}
		\item GRAMMY U is a unique and fast-growing community of full-time college students, primarily between the ages of 17 and 25,  who are pursuing a career in the recording industry.
		\item The Recording Academy created GRAMMY U to help prepare college students for their careers in the music industry through networking, educational programs and performance opportunities.
		\item GRAMMY U is designed to enhance students' current academic curriculum with access to recording industry professionals to give an ``out of classroom'' perspective on the recording industry.
		\end{enumerate}
	\end{enumerate}
%%%%%%%%%%%%%%%%%%%%%%%%%%%%%
\item --- --- --- --- --- --- --- --- --- --- --- --- --- --- --- --- --- --- --- --- --- --- --- --- --- --- --- --- --- --- ---
\item \colorbox{blue}{\bf Help for Underrepresented Minorities}
% Help for Underrepresented Minorities
\item INROADS, Inc.: \vspace{-0.3cm}
	\begin{enumerate} \itemsep -2pt
	\item Internships: \url{http://www.inroads.org/interns/internWhatItTakes.jsp}
	\end{enumerate}
\item The PhD Project: \vspace{-0.3cm}
	\begin{enumerate} \itemsep -2pt
	\item \url{http://www.phdproject.org/index.html}
	\item Program and informational network to encourage ``African-Americans, Hispanic-Americans and Native Americans'' to pursue Ph.D. programs in business and seek careers in academia.
	\item Annual PhD Project Conference: \vspace{-0.2cm}
		\begin{enumerate} \itemsep -2pt
		\item Conference: \vspace{-0.1cm}
			\begin{enumerate} \itemsep -1pt
			\item \url{http://www.phdproject.org/conference.html}
			\item \url{http://www.phdproject.org/conference_application.html}
			\item For prospective Ph.D. students in business to learn more about Ph.D. programs in business, the Ph.D. application process, and life in graduate school.
			\item Registration Policy: \vspace{-0.1cm}
				\begin{itemize} \itemsep -1pt
				\item If you are selected to attend the conference you will be required to pay a \$200 registration fee which can be processed via credit card during the registration process. All travel and conferences expenses will paid by The PhD Project (total conference expenses for hotel, meals, materials, and transportation are valued at approximately \$1,500 per invited attendee.) Your investment of the \$200 registration fee will be refunded if you enter a full-time, AACSB accredited business doctoral program within 3 years of attending the conference. 
				\item If you previously attended a PhD Project Conference, you may submit an application to be reviewed, however if you are selected to attend, The PhD Project will only cover hotel costs (shared room with another participant). You will be required to pay the registration and travel costs
				\end{itemize}
			\end{enumerate}
		\item Resources for Potential/Current Doctoral Students: \vspace{-0.1cm}
			\begin{enumerate} \itemsep -1pt
			\item \url{http://www.phdproject.org/resources.html}
			\item Information about good business schools that offer Ph.D. programs, preparation for the GMAT, and the life in graduate school as a Ph.D. student.
			\item Suggested Reading: \vspace{-0.1cm}
				\begin{itemize} \itemsep -1pt
				\item \url{http://www.phdproject.org/reading.html}
				\item Has information life in graduate school as a Ph.D. student, racial diversity/issues in higher education, job searching in academia, and work-life balance for female Ph.D. students.
				\end{itemize}
			\end{enumerate}
		\item The PhD Project Doctoral Student Association (DSA): \vspace{-0.1cm}
			\begin{enumerate} \itemsep -1pt
			\item The PhD Project network: \vspace{-0.1cm}
				\begin{itemize} \itemsep -1pt
				\item \url{http://www.myphdnetwork.org/}
				\item ``There are 5 discipline specific associations covering the major areas of business education: Accounting, Finance, Information Systems, Management, Marketing.''
				\end{itemize}
			\end{enumerate}
		\end{enumerate}
	\end{enumerate}
\item MS-to-Ph.D. program for underrepresented minorities at Fisk and Vanderbilt in certain areas of
science (including astronomy, material science, and physics)
\item Outreach programs for underrepresented minorities to help them get into medical (and/or graduate) schools. Search for ``PREP (Post-baccalaureate Research Education Programs),'' which have stipends. E.g., Georgetown University School of Medicine, and George Washington University's medical school
\item New York University: \vspace{-0.3cm}
	\begin{enumerate} \itemsep -2pt
	\item Leonard N. Stern School of Business: \vspace{-0.2cm}
		\begin{enumerate} \itemsep -2pt
		\item Stern Pre-Doctoral program: \url{http://www.stern.nyu.edu/AcademicPrograms/PhD/Pre-Doctoral/index.htm}
		\end{enumerate}
	\end{enumerate}
\end{enumerate}



%%%%%%%%%%%%%%%%%%%%%%%%%%%%%%%%%%%%%%%%%%%
\section{Science \& Engineering Outreach}
\label{stemoutreach}

%%%%%%%%%%%%%%%%%%%%%%%%%%%%%%%%%%%%%%%%%%%
\subsection{Precollege Science \& Engineering Outreach}
\label{stemoutreachk12}

Science and engineering outreach to high-school (and middle-school) students, and their parents, teachers, and career counselors: \vspace{-0.3cm}
\begin{enumerate} \itemsep -4pt
\item {\it MentorNet}: \vspace{-0.3cm}
	\begin{enumerate} \itemsep -2pt
	\item \url{http://www.mentornet.net/}
	\item Enables people to network with scientists, engineers, and professors in Science, Technology, Engineering, and Mathematics (STEM)
	\item Is very supportive of minorities, so that more minorities (particularly underrepresented minorities) can be attracted to STEM careers
	\end{enumerate}
\item International Science Olympiad (for high school students): \vspace{-0.3cm}
	\begin{enumerate} \itemsep -2pt
	\item International Olympiad in Informatics: \url{http://en.wikipedia.org/wiki/International_Olympiad_in_Informatics} and \url{http://www.ioinformatics.org/index.shtml}
	\item International Mathematical Olympiad: \url{http://www.imo-official.org/}
	\item International Physics Olympiad: \url{http://www.jyu.fi/tdk/kastdk/olympiads/}
	\item International Chemistry Olympiad: \url{http://www.icho.sk/}
	\item International Biology Olympiad: \url{http://www.ibo-info.org/}
	\item \url{http://scienceolympiads.org/}
	\end{enumerate}
\item International Astronomy Olympiad: \url{http://www.issp.ac.ru/iao/}
\item International Earth Science Olympiad: \url{http://en.wikipedia.org/wiki/International_Earth_Science_Olympiad}
\item International Junior Science Olympiad (for students younger than 15 years old): \url{http://www.ijso-official.org/home}
\item Teen Leadership Institute Science, Technology, Engineering, and Math (STEM) programs @ YWCA Greater Pittsburgh; see \url{http://www.ywcapgh.org/STEM_Programs.asp}
\item For Inspiration and Recognition of Science and Technology (FIRST): \url{http://www.usfirst.org/} (including resources and guides to mentoring); scholarships @ \url{http://www.usfirst.org/aboutus/content.aspx?id=508}; and robotics programs @ \url{http://www.usfirst.org/roboticsprograms/frc/default.aspx?=966}
\item Mac Hyman, ``Good Choices for Great Careers in the Mathematical Sciences,'' talk given at 2008 SIAM Annual Meeting. Available at: \url{http://client.blueskybroadcast.com/siam08/hyman/index.html}; last accessed on August 25, 2010. Also, see \url{http://meetings.siam.org/program.cfm?CONFCODE=AN08}, \url{http://www.siam.org/meetings/an08/program.php}, and \url{http://www.siam.org/meetings/an08/}.
\item {\it RoboCup}\texttrademark\ competitions: \vspace{-0.2cm}
	\begin{enumerate} \itemsep -2pt
	\item Junior category for K-12 students involves contests the these areas of challenges: \vspace{-0.1cm}
		\begin{enumerate} \itemsep -1pt
		\item soccer
		\item dance
		\item rescue operations
		\end{enumerate}
	\item \url{http://www.robocup.org/}
	\end{enumerate}
\item {\it Curriki}, which is an online educational resource for teachers, students, and parents in K-12: \url{http://www.curriki.org/xwiki/bin/view/Main/About}
%%%%%%%%%%%%%%%%%%%%%%%%%%%%%%%%%%%%%%%%
%%%%%%%%%%%%%%%%%%%%%%%%%%%%%%%%%%%%%%%%
\item Electrical and computer engineering and/or computer science: \vspace{-0.2cm}
	\begin{enumerate} \itemsep -2pt
	\item {\it TopCoder} coding and design contests: \vspace{-0.2cm}
		\begin{enumerate} \itemsep -2pt
		\item High School category
		\item \url{http://www.topcoder.com/}
		\end{enumerate}
	\item Student Cluster Competition (SCC): \vspace{-0.2cm}
		\begin{enumerate} \itemsep -2pt
		\item SCC is held at each (annual) SC conference, which is the International Conference for High Performance Computing, Networking, Storage, and Analysis. IEEE Computer Society and the Association for Computing Machinery are the sponsors for this conference.
		\item During SC10, teams consisting of six students, undergraduate and/or high school, will showcase the amazing power of clusters and the ability to utilize open source software to solve interesting and important problems. They will compete in real-time on the exhibit floor to run a workload of real-world applications on clusters of their own design while never exceeding the dictated power limit.
		\item During SC10 in New Orleans, teams will assemble, test and tune their machines and run the HPCC benchmarks until the starting bell rings on Monday night at the Exhibit Opening Gala where they will be given the competition data sets. In full view of conference attendees, teams will execute the prescribed workload while showing progress and science visualization output on large high-resolution displays in their areas. Teams race to correctly complete the greatest number of application runs during the competition period until the close of the exhibit floor on Wednesday evening.
		\item \url{http://sc10.supercomputing.org/?pg=studentcluster.html}
		\end{enumerate}
	\item Institute of Electrical and Electronics Engineers, IEEE: \vspace{-0.3cm}
		\begin{enumerate} \itemsep -2pt
		\item {\it IEEE Educational Activities} recommended resources: \url{http://www.ieee.org/education_careers/education/preuniversity/resources/index.html}
		\item Engineering Projects in Community Service (EPICS) in IEEE: \vspace{-0.2cm}
			\begin{enumerate} \itemsep -2pt
			\item High school students collaborate with college students in engineering projects to benefit the community
			\item \url{http://www.ieee.org/education_careers/education/preuniversity/epics_high.html}
			\end{enumerate}
		\item Talk given by John Cohn at the IEEE International Symposium on Circuits and Systems (ISCAS), May 18-21, 2008. The talk is titled, ``Kids these days. How we can inspire the next generation of Engineers and Scientists?'' See \url{http://ewh.ieee.org/soc/icss/IEEE-ISCAS-08-Tue-Keynote-JC/IEEE-ISCAS-08-Tue-Keynote-JC.HTML}. [ Alternatively, go to: IEEE Circuits and Systems Society, \url{http://www.ieee-cas.org/}: Select the ``Resources'' tab in the menu bar, and select the ``ISCAS Keynote Videos'' option. Click on the video link with the appropriate title. ]
		\end{enumerate}
	\item Association for Computing Machinery (ACM): \vspace{-0.2cm}
		\begin{enumerate} \itemsep -2pt
		\item Sanjeev Arora, Boaz Barak, and Luca Trevisan, ``Survey Papers and Essays,'' in {\it Theory Matters Wiki: Theoretical Computer Science (TCS) Advocacy Wiki}, SIGACT Committee for the Advancement of Theoretical Computer Science, ACM Special Interest Group on Algorithms and Computation Theory (SIGACT), Association for Computing Machinery, February 25, 2010. Available at: \url{http://theorymatters.org/pmwiki/pmwiki.php?n=Main.SurveyCollection}; last accessed on September 14, 2010.
		\end{enumerate}
	\item WGBH Educational Foundation: \vspace{-0.2cm}
		\begin{enumerate} \itemsep -2pt
		\item Dot Diva / New Image for Computing (NIC) initiative: \vspace{-0.1cm}
			\begin{enumerate} \itemsep -1pt
			\item \url{http://dotdiva.org/}
			\item Resource for parents and teachers: \url{http://dotdiva.org/parents.html}
			\end{enumerate}
		\end{enumerate}
	\item Silicon Valley StRUT: \vspace{-0.2cm}
		\begin{enumerate} \itemsep -2pt
		\item Students Recycling Used Technology, StRUT, Competition; StRUT Competition consists of: \vspace{-0.1cm}
			\begin{enumerate} \itemsep -1pt
			\item Disassemble and Reassemble A Computer 
			\item Create and Present a Powerpoint Presentation 
			\item Computer Parts Identification and Challenge Test  
			\item Team Quiz Bowl on Computer Technology and Related Subjects
			\item \url{http://www.svstrut.org/cms/content/section/1/5/}
			\item Teacher Resources: \url{http://www.svstrut.org/cms/component/option,com_weblinks/catid,11/Itemid,10/}
			\item [ Resources to Support ] Curriculum for Engineering and Computer Technology Education: \url{http://www.svstrut.org/cms/content/view/8/18/}
			\end{enumerate}
		\item \url{http://www.svstrut.org/cms/}
		\end{enumerate}
	\item Google Code Jam (programming contest): \url{http://code.google.com/codejam/} and \url{http://en.wikipedia.org/wiki/Google_Code_Jam}
	\item University of Illinois at Urbana-Champaign (UIUC): \vspace{-0.2cm}
		\begin{enumerate} \itemsep -2pt
		\item College of Engineering; Department of Computer Science: \vspace{-0.1cm}
			\begin{enumerate} \itemsep -1pt
			\item Outreach \& Diversity: \url{http://cs.illinois.edu/outreach}
			\item ChicTech: \url{http://cs.illinois.edu/outreach/chictech}
			\item Technical Ambassadors: \url{http://cs.illinois.edu/outreach/tac}
			\item Games4Girls: \url{http://cs.illinois.edu/outreach/games4girls}
			\item Workshops \& Camps: \url{http://cs.illinois.edu/outreach/k12}
			\item \url{http://cs.illinois.edu/outreach}
			\end{enumerate}
		\end{enumerate}
	\item Carnegie Mellon University: \vspace{-0.2cm}
		\begin{enumerate} \itemsep -2pt
		\item women@SCS School of Computer Science, Carnegie Mellon University: \vspace{-0.1cm}
			\begin{enumerate} \itemsep -1pt
			\item Papers: \url{http://women.cs.cmu.edu/Resources/Papers/}
			\item Alumnae Interviews / Profiles: \url{http://women.cs.cmu.edu/Who/Alumnae/alumInterviews.php}
			\item Job and Research Opportunities: \url{http://www.women.cs.cmu.edu/Resources/JobsResearch/}
			\item Career Advice: \url{http://women.cs.cmu.edu/Resources/JobsResearch/careeradvice.php}
			\item Other Sites: \url{http://www.women.cs.cmu.edu/Miscellaneous/Other/}
			\end{enumerate}
		\end{enumerate}
	\item {\it Quora}: \vspace{-0.2cm}
		\begin{enumerate} \itemsep -2pt
		\item ``If a 10-year-old wanted to start programming today, what language path would be the most valuable moving forward?'' Available online at: \url{http://www.quora.com/If-a-10-year-old-wanted-to-start-programming-today-what-language-path-would-be-the-most-valuable-moving-forward}; last accessed on November 23, 2010.
		\end{enumerate}
	\end{enumerate}
%%%%%%%%%%%%%%%%%%%%%%%%%%%%%%%%%%%%%%%%
%%%%%%%%%%%%%%%%%%%%%%%%%%%%%%%%%%%%%%%%
\item Engineering Education Service Center (EESC): \vspace{-0.3cm}
	\begin{enumerate} \itemsep -2pt
	\item Has lists of: \vspace{-0.2cm}
		\begin{enumerate} \itemsep -2pt
		\item Educational material: \vspace{-0.1cm}
			\begin{enumerate} \itemsep -1pt
			\item books
			\item DVDs
			\item resource kits for teachers
			\end{enumerate}
		\item engineering camps (for the summer in the United States): \url{http://www.engineeringedu.com/camps/}
		\item {\it Women in Engineering} programs at US engineering schools: \url{http://www.engineeringedu.com/wie.html}
		\item US engineering schools: \url{http://www.engineeringedu.com/engrschools.htm}
		\item competitions for youths, including high school students: \url{http://www.engineeringedu.com/competitions.html}
		\item online resources
		\item list of professional organizations in engineering (or engineering societies): \url{http://www.engineeringedu.com/soc1.html}
		\item scholarships: \url{http://www.engineeringedu.com/scholars.html}
		\end{enumerate}
	\item It has resources for K-12 students, and their teachers and parents. It also has information for girls who are seeking careers in engineering; in addition, it provides their parents and teachers with information to guide the girls.
	\item It runs a workshop (in the US) for mother-daughter pairs to encourage girls to pursue careers in engineering.
	\item \url{http://www.engineeringedu.com/}
	\end{enumerate}
\item TryNano.org: \vspace{-0.3cm}
	\begin{enumerate} \itemsep -2pt
	\item Information about educational opportunities and careers in nanotechnology and nanoscience
	\item \url{TryNano.org}
	\end{enumerate}
\item {\it Mathematical Association of America} (MAA): \vspace{-0.3cm}
	\begin{enumerate} \itemsep -2pt
	\item Middle/High School Students: \url{http://www.maa.org/students/middle_high/}
	\item Parents: \url{http://www.maa.org/students/Parents.html}
	\item MAA American Mathematics Competitions: \vspace{-0.2cm}
		\begin{enumerate} \itemsep -2pt
		\item {\it Students} [resources]. Available at: \url{http://amc.maa.org/a-activities/a4-for-students/s-index.shtml}; last accessed on September 2, 2010.
		\item It includes tips to help students do well in math contests and Olympiads, a reading list for students interested in mathematics, problems from past math contests and Olympiads, and other resources from the World Wide Web.
		\end{enumerate}
	\item {\it Fun Math Sites}. Available at: \url{http://www.maa.org/students/funsites.html}; last accessed on September 2, 2010.
	\item Special Interest Group on Mathematics and the Arts (SIGMAA-ARTS): Resources, see \url{http://myweb.cwpost.liu.edu/aburns/sigmaa-arts/resources.html}.
	\item Special Interest Group of the MAA on Quantitative Literacy (SIGMAA QL): \url{http://sigmaa.maa.org/ql/}
	\end{enumerate}
\item eGFI (Engineering, Go For It!): \vspace{-0.3cm}
	\begin{enumerate} \itemsep -2pt
	\item Provides information for students, parents, and teachers about educational pathways and careers in engineering
	\item \url{http://egfi-k12.org/}
	\end{enumerate}
\item {\it Sloan Career Cornerstone Center}: \vspace{-0.3cm}
	\begin{enumerate} \itemsep -2pt
	\item Career exploration resources in STEM (science, technology, engineering, mathematics, computing, and healthcare)
	\item \url{http://www.careercornerstone.org/}
	\end{enumerate}
\item {\it TryEngineering}: \vspace{-0.3cm}
	\begin{enumerate} \itemsep -2pt
	\item Career exploration resources for engineering
	\item \url{http://www.tryengineering.org/}
	\end{enumerate}
\item {\it Junior Engineering Technical Society, JETS}: \vspace{-0.3cm}
	\begin{enumerate} \itemsep -2pt
	\item Career exploration resources for engineering
	\item \url{http://www.jets.org/}
	\end{enumerate}
\item {\it American Society of Mechanical Engineers, ASME}: \vspace{-0.3cm}
	\begin{enumerate} \itemsep -2pt
	\item K-12 Student Resources: \url{http://www.asme.org/Communities/Students/K12/} and \url{http://www.asme.org/Education/PreCollege/EngineeringResources/}
	\item Engineering Camps: \url{http://www.asme.org/Communities/Students/K12/Camps.cfm}
	\end{enumerate}
\item BESTRobotics, Inc.: \vspace{-0.3cm}
	\begin{enumerate} \itemsep -2pt
	\item BEST (Boosting Engineering, Science, and Technology) competition: \vspace{-0.2cm}
		\begin{enumerate} \itemsep -2pt
		\item \url{http://best.eng.auburn.edu/}
		\item Hosted at Auburn University's Samuel Ginn College of Engineering
		\item BEST World Championship: \url{http://best.eng.auburn.edu/world-championship/}
		\end{enumerate}
	\end{enumerate}
\item {\it American Society of Civil Engineers, ASCE}: \vspace{-0.3cm}
	\begin{enumerate} \itemsep -2pt
	\item Outreach resource for K-12 students, and their parents and teachers
	\item \url{http://content.asce.org/asceville/index.html}
	\end{enumerate}
\item {\it Science.gov} (USA.gov for Science): Internship and Fellowship Opportunities in Science (for high school students); see \url{http://www.science.gov/internships/k-12.html}
\item {\it iTunes U}: \vspace{-0.3cm}
	\begin{enumerate} \itemsep -2pt
	\item {\it iTunes} is required to listen to or watch these lectures, talks, and presentations.
	\item Access {\it iTunes U} at: \url{http://deimos3.apple.com/indigo/main/main.html?v0=WWW-AMUS-ITUNESU070521-N48LX}
	\item WGBH's Teachers' Domain -- Boston's PBS Station: Video presentation on ``Engineering for the Red Planet''; see \url{http://deimos3.apple.com/WebObjects/Core.woa/Browse/wgbh.org.1416254059.01416254061.1416793683?i=1951581658}. Also, check out its video clip on ``Carbon Fiber Car of the Future''.
	\item {\it iTunes U} is a set of webcast and podcasts, where you can easily find audio and video clips for lectures, seminars, announcements, virtual tours, and so on. For example, some professors from schools like MIT or Berkeley will provide audio/video clips of their lectures on {\it iTunes U}.
	\item This can help in exploring different majors during the college application process, or before a college student declares her/his major(s). If a student is not sure if she/he wants to double major in deaf studies and linguistics, this student can check out some linguistics lectures from her/his (preferred) college/university, if it uses {\it iTunes U}, or those from other universities.
	\end{enumerate}
\item High School Ace's College Prep Guide: \url{http://highschoolace.com/ace/colleges.cfm}
\item {\it Dr. Sally Ride} (America�s first woman in space): \vspace{-0.3cm}
	\begin{enumerate} \itemsep -2pt
	\item {\it Sally Ride Science}'s resources for educators: \url{https://www.sallyridescience.com/for_educators}
	\item Sally Ride Science Educator Institutes (to educate K-12 teachers about science): \url{https://www.sallyridescience.com/for_educators/institutes}
	\item {\it Sally Ride Science Academy} helps teachers to increase their students' interest in science: \url{https://www.sallyridescience.com/academy}
	\item {\it Sally Ride Science}'s resources for teachers: \url{https://www.sallyridescience.com/resources}
	\item {\it Sally Ride Science Festivals} are events for girls from the $5^{th}$ grade to the $8^{th}$ grade: \url{https://www.sallyridescience.com/festivals}
	\item {\it Sally Ride Science Camps} are summer camps for girls from the $4^{th}$ grade to the $9^{th}$ grade: \url{http://www.sallyridecamps.com/}
	\item GRAIL MoonKAM: \vspace{-0.2cm}
		\begin{enumerate} \itemsep -2pt
		\item ``GRAIL MoonKAM (Moon Knowledge Acquired by Middle school students) is GRAIL's signature education and public outreach program.''
		\item ``GRAIL MoonKAM will engage middle schools across the country in the GRAIL mission and lunar exploration.''
		\item \url{https://www.grailmoonkam.com/}
		\end{enumerate}
	\item EarthKAM: \vspace{-0.2cm}
		\begin{enumerate} \itemsep -2pt
		\item EarthKAM (Earth Knowledge Acquired by Middle school students) is a NASA educational outreach program enabling students, teachers and the public to learn about Earth from the unique perspective of space.
		\item \url{https://earthkam.ucsd.edu/}
		\end{enumerate}
	\end{enumerate}
\item Andrew Rader Studios: \vspace{-0.3cm}
	\begin{enumerate} \itemsep -2pt
	\item Chem4Kids.com: \url{http://www.chem4kids.com/}
	\end{enumerate}
\item {\it American Association for the Advancement of Science, AAAS}: \vspace{-0.3cm}
	\begin{enumerate} \itemsep -2pt
	\item ENTRY POINT! for Students With Disabilities (in STEM): \url{http://www.aaas.org/careercenter/fellowships/} and \url{http://ehrweb.aaas.org/entrypoint/}
	\item AAAS Mass Media Science \& Engineering Fellows Program (for STEM grad students to intern in mass media companies): \url{http://www.aaas.org/programs/education/MassMedia/}
	\item Diversity Issues: \url{http://sciencecareers.sciencemag.org/career_magazine/diversity_issues/}
	\item Internships involving science and journalism, human rights, scientific freedom, responsibility, or law: \url{http://www.aaas.org/careercenter/} and \url{http://www.aaas.org/careercenter/internships/scienceminority.shtml} (AAAS Minority Science Writers Internship)
	\item Kinetic City: \url{http://www.kineticcity.com/}
	\end{enumerate}
\item {\it NASA} resources for students: \url{http://www.nasa.gov/audience/forstudents/index.html} and \url{http://www.nasa.gov/offices/education/programs/national/summer/education_resources/index.html} (NASA Summer of Innovation)
\item National Academy of Engineering, NAE: \vspace{-0.3cm}
	\begin{enumerate} \itemsep -2pt
	\item NAE Grand Challenges: \vspace{-0.2cm}
		\begin{enumerate} \itemsep -2pt
		\item Includes a list of NAE Grand Challenges, which indicate some of the challenges faced by people on a global scale that can be partially solved by engineers. This can be used to get children and youths to be excited about engineering.
		\item NAE Grand Challenges: \vspace{-0.1cm}
			\begin{enumerate} \itemsep -1pt
			\item Make solar energy economical
			\item Provide energy from fusion
			\item Develop carbon sequestration methods
			\item Manage the nitrogen cycle
			\item Provide access to clean water
			\item Restore and improve urban infrastructure
			\item Advance health informatics
			\item Engineer better medicines
			\item Reverse-engineer the brain
			\item Prevent nuclear terror
			\item Secure cyberspace
			\item Enhance virtual reality
			\item Advance personalized learning
			\item Engineer the tools of scientific discovery
			\end{enumerate}
		\item \url{http://www.engineeringchallenges.org/}
		\item NAE Grand Challenge K12 Partners Program: \vspace{-0.1cm}
			\begin{enumerate} \itemsep -1pt
			\item \url{http://www.grandchallengek12.org/about}
			\item 5-Part Make it Happen Plan: \url{http://www.grandchallengek12.org/5-part-plan}
			\end{enumerate}
		\end{enumerate}
	\item {\it National Academy of Engineering}'s technological literacy program for people (students, parents, and educators) to learn more about technology: \url{http://www.nae.edu/nae/techlithome.nsf}
	\item Greatest Engineering Achievements: \url{http://www.greatachievements.org/}
	\end{enumerate}
\item National Science Foundation: \vspace{-0.3cm}
	\begin{enumerate} \itemsep -2pt
	\item Broadening Participation in Computing (BPC): \vspace{-0.2cm}
		\begin{enumerate} \itemsep -2pt
		\item \url{http://www.bpcportal.org/}
		\item \url{http://www.bpcportal.org/bpc/shared/home.jhtml;jsessionid=0MIUYDR5U4ARXABAVRSSFEQ?_requestid=9445}
		\item \url{http://www.nsf.gov/funding/pgm_summ.jsp?pims_id=13510}
		\item \url{http://www.nsf.gov/funding/pgm_summ.jsp?pims_id=13510&org=NSF&sel_org=NSF&from=fund}
		\item ``Broadening Participation in Computing (BPC) is a NSF sponsored program with the goal of significantly increasing the number of underrepresented graduates in the computing disciplines, with an emphasis on women, persons with disabilities, and minorities (African Americans, Hispanics, American Indians, Alaska Natives, Native Hawaiians, and Pacific Islanders).''
		\item Broadening Participation in Computing Digital Library: \vspace{-0.1cm}
			\begin{enumerate} \itemsep -1pt
			\item \url{http://www.bpcportal.org/bpc/interdiscipline/dl_index.jhtml;jsessionid=ROYEHJV1UQYWNABAVRSSFEQ?comm=BPC}
			\item Includes resources for different target populations: \vspace{-0.1cm}
				\begin{itemize} \itemsep -1pt
				\item Women
				\item African Americans
				\item Hispanic Americans, or Latinas and Latinos
				\item People with disabilities
				\item Native Americans
				\end{itemize}
			\item It also includes resources for different topics, such as mentoring, recruitment, retention, and work-life balance.
			\end{enumerate}
		\item Alliances (other professional organizations): \url{http://www.bpcportal.org/bpc/comm/projects.jhtml}
		\end{enumerate}
	\item The National Science Digital Library (NSDL): \vspace{-0.2cm}
		\begin{enumerate} \itemsep -2pt
		\item \url{http://www.nsdl.org/} and \url{http://www.nsdl.org/browse/}
		\item ``The National Science Digital Library is a national network dedicated to advancing STEM teaching and learning for all learners, in both formal and informal settings, and the locus of activity for the National Science Foundation's National STEM Distributed Learning program.''
		\item Outreach materials: \vspace{-0.1cm}
			\begin{enumerate} \itemsep -1pt
			\item \url{http://www.nsdl.org/pd/?pager=materials}
			\item Has outreach materials for educators in K-12 and higher educational institutions.
			\end{enumerate}
		\item Resources for K-12 Teachers: \url{http://nsdl.org/resources_for/k12_teachers/}
		\item Resources for Librarians: \url{http://nsdl.org/resources_for/librarians/}
		\item Billingual Resources: \url{http://www.nsdlnetwork.org/collections/billingual-resources}
		\item NSDL on {\it iTunes U}: \url{http://www.nsdl.org/iTunesU/}
		\item Collections: \url{http://www.nsdl.org/browse/?subject=All}
		\item NSDL Pathways: \vspace{-0.1cm}
			\begin{enumerate} \itemsep -1pt
			\item \url{http://nsdl.org/about/?pager=pathways}
			\item ``Pathways are large projects that are aggregators and stewards of resources and services to broad categories of users---either discipline-focused (e.g. chemistry), or audience-focused (e.g. middle school educators), or resources of a specific type or format (e.g. multimedia content).''
			\item ``They are digital library portals developed and managed in partnership with organizations and institutions that have a history and expertise in serving their portal's target audiences.''
			\item ``They contribute metadata (descriptive information) about their resources to NSDL to make their resources searchable and discoverable via the NSDL.org portal, in addition to their own portals.''
			\end{enumerate}
		\item {\bf NSDL Science Literacy Maps}: \vspace{-0.1cm}
			\begin{enumerate} \itemsep -1pt
			\item \url{http://strandmaps.nsdl.org/}
			\item ``{\it NSDL Science Literacy Maps} are a tool for teachers and students to find resources that relate to specific science and math concepts. The maps illustrate connections between concepts as well as how concepts build upon one another across grade levels.''
			\end{enumerate}
		\item NSDL Professional Development: \url{http://www.nsdl.org/pd/}
		\item NSDL Technical Network Services: \vspace{-0.1cm}
			\begin{enumerate} \itemsep -1pt
			\item \url{http://www.nsdl.org/about/?pager=tns}
			\item \url{http://nsdlnetwork.org/}
			\item \url{http://nsdlnetwork.org/content/book/page/953/about-nsdl-technical-network-services}
			\end{enumerate}
		\item NSDL Resource Center: \url{http://nsdlnetwork.org/content/book/951/page/954/about-nsdl-resource-center}
		\end{enumerate}
	\end{enumerate}
\item {\it American Chemical Society} Science for Kids program (for students in K-12): \url{http://portal.acs.org/portal/acs/corg/content?_nfpb=true&_pageLabel=PP_TRANSITIONMAIN&node_id=878&use_sec=false&sec_url_var=region1&__uuid=984d4ee7-4214-4d35-9899-bc2f91dee58b}
\item {\it California Digital Educator Consortium}, ``Digital Educator,'' Digital Learning Center: \url{http://www.digitaleducator.com/}
\item Kenny Felder, ``Selected Other Educational Sites on the Web''. Available at: \url{http://www4.ncsu.edu/unity/lockers/users/f/felder/public/kenny/edulinks.html}; last accessed on August 28, 2010.
\item FHSST (Free High School Science Texts); free textbooks for grades 10-12 in Physics, Chemistry, and Mathematics. Available at: \url{http://www.fhsst.org/}; last accessed on August 28, 2010.
\item John Baez, {\it Usenet Physics FAQ}, Department of Mathematics, University of California, Riverside, September 2009. Available at: \url{http://math.ucr.edu/home/baez/physics/}; last accessed on August 28, 2010.
\item {\it American Society for Engineering Education}: \vspace{-0.3cm}
	\begin{enumerate} \itemsep -2pt
	\item Science and Engineering Apprenticeship Program (SEAP): \vspace{-0.2cm}
		\begin{enumerate} \itemsep -2pt
		\item ``The Science and Engineering Apprenticeship Program (SEAP) provides an opportunity for students to participate in research at a Department of Navy (DoN) laboratory during the summer.''
		\item ``The goals of SEAP are to encourage participating students to pursue science and engineering careers, to further their education via mentoring by laboratory personnel and their participation in research, and to make them aware of DoN Research and technology efforts, which can lead to employment within the DoN.''
		\item ``High school students who have completed at least Grade 9. A graduating senior is eligible to apply.''
		\item ``Must be 16 years of age for most laboratories. Some laboratories may accept a 15 year old applicant. Please check individual lab description for more details.''
		\item ``Applicants must be US citizens and participation by Permanent Resident Aliens is limited. Please check individual lab descriptions for participation of Permanent Resident Aliens.''
		\item \url{http://seap.asee.org/}
		\end{enumerate}
	\end{enumerate}
\item robots.net, {\it Robot Competitions} (list of robot competitions and contests) : \url{http://robots.net/rcfaq.html}
\item International Council on Systems Engineering (INCOSE): \vspace{-0.3cm}
	\begin{enumerate} \itemsep -2pt
	\item Careers in Systems Engineering: \url{http://www.incose.org/educationcareers/careersinsystemseng.aspx}
	\item Frequently Asked Questions for Students [about Systems Engineering]: \url{http://www.incose.org/educationcareers/faqsforstudents.aspx}
	\item What is Systems Engineering?: \url{http://www.incose.org/practice/whatissystemseng.aspx}
	\end{enumerate}
\item {\it National Society of Professional Engineers}: \vspace{-0.3cm}
	\begin{enumerate} \itemsep -2pt
	\item A Sightseer's Guide to Engineering: \url{http://www.engineeringsights.org/}
	\end{enumerate}
\item {\it Engineers Dedicated to a Better Tomorrow (a.k.a., DedicatedEngineers)}: \vspace{-0.3cm}
	\begin{enumerate} \itemsep -2pt
	\item The ``K-12 Crowd'' (Students, Teachers, Guidance Counselors and Parents): \url{http://www.dedicatedengineers.org/intro_for_K-12.htm}
	\item \url{http://www.dedicatedengineers.org/}
	\end{enumerate}
\item National Engineers Week Foundation: \vspace{-0.3cm}
	\begin{enumerate} \itemsep -2pt
	\item Discover Engineering: \url{http://www.discoverengineering.org/}
	\item Introduce A Girl to Engineering: \url{http://www.eweek.org/EngineersWeek/IntroduceAGirl.aspx}
	\item All About Engineering: \url{http://www.eweek.org/AboutEngineering/AboutEngineering.aspx}
	\end{enumerate}
\item University of California: \vspace{-0.3cm}
	\begin{enumerate} \itemsep -2pt
	\item The Coalition For Science After School: \vspace{-0.2cm}
		\begin{enumerate} \itemsep -2pt
		\item \url{http://afterschoolscience.org/}
		\item ``Promoting high-quality afterschool science'' ... ``The Coalition for Science After School envisions the day when young people from all backgrounds have access to high-quality science, technology, engineering and mathematics (STEM) learning beyond the classroom.''
		\item Tools for advocates--Championing afterschool science: \url{http://afterschoolscience.org/tools/}
		\item Program resources--Enhancing the quality of afterschool opportunities: \url{http://afterschoolscience.org/resources/}
		\item The National After School Science Directory: \vspace{-0.1cm}
			\begin{enumerate} \itemsep -1pt
			\item \url{http://afterschoolscience.org/directory/}
			\item ``The National After School Science Directory is a searchable database designed to increase access to high-quality science, technology, engineering and math (STEM) education beyond the classroom for youth and families across the nation. The Directory houses thousands of STEM opportunities, submitted by science centers, museums, schools and other youth-serving organizations. Search our Directory to view opportunities to connect the America's youth to high-quality STEM learning experiences.''
			\end{enumerate}
		\item Become an advocate: \url{http://afterschoolscience.org/tools/advocate.php}
		\item Funders (funding organizations/agencies): \url{http://afterschoolscience.org/tools/funders.php}
		\end{enumerate}
	\end{enumerate}
\item Harvey Mudd College: \vspace{-0.3cm}
	\begin{enumerate} \itemsep -2pt
	\item Francis Edward Su, {\it Math Fun Facts!}, Department of Mathematics, Harvey Mudd College: \url{http://www.math.hmc.edu/funfacts/}
	\end{enumerate}
\item Clay Mathematics Institute: \vspace{-0.3cm}
	\begin{enumerate} \itemsep -2pt
	\item Program in Mathematics for Young Scientists, PROMYS: \vspace{-0.2cm}
		\begin{enumerate} \itemsep -2pt
		\item \url{http://www.claymath.org/programs/outreach/PROMYS/}
		\item \url{http://math.bu.edu/people/promys/}
		\item \url{http://www.promys.org/}
		\end{enumerate}
	\item Ross Program (for pre-college students): \vspace{-0.2cm}
		\begin{enumerate} \itemsep -2pt
		\item \url{http://www.claymath.org/programs/outreach/ross/}
		\item \url{http://www.math.ohio-state.edu/ross/}
		\end{enumerate}
	\item CMI Summer Schools: \url{http://www.claymath.org/programs/summer_school/}
	\end{enumerate}
\item Consortium for Ocean Leadership: \vspace{-0.3cm}
	\begin{enumerate} \itemsep -2pt
	\item Oceans of Opportunity (for African American students in K-12, and colleges and universities -- includes undergraduates and grad students): \url{http://www.oceanleadership.org/education/diversity/oceans-of-opportunity/}
	\item The JOIDES Resolution (The JR) scientific research vessel [ Deep Earth Academy ]: \vspace{-0.2cm}
		\begin{enumerate} \itemsep -2pt
		\item Fun \& Games: \url{http://joidesresolution.org/node/53}
		\item Discovery Center: \url{http://joidesresolution.org/node/44}
		\item Just for Kids Blog: \url{http://joidesresolution.org/node/366}
		\end{enumerate}
	\item National Ocean Sciences Bowl (high school academic competition that provides a forum for talented students to test their knowledge of the marine sciences including biology, chemistry, physics, and geology): \vspace{-0.2cm}
		\begin{enumerate} \itemsep -2pt
		\item \url{http://www.nosb.org/}
		\item Career Resources: \url{http://www.nosb.org/ocean-careers/career-resources/}
		\end{enumerate}
	\item Integrated Ocean Drilling Program (IODP), IODP United States Implementing Organization (IODP-USIO): \vspace{-0.2cm}
		\begin{enumerate} \itemsep -2pt
		\item U.S.-sponsored Teacher at Sea Program (for US teachers to participate in seagoing research experiences aboard the JOIDES Resolution): \url{http://www.iodp-usio.org/Education/TAS.html}
		\end{enumerate}
	\item Careers: \url{http://www.oceanleadership.org/education/deep-earth-academy/students/careers/}
	\end{enumerate}
\item The Oceanography Society: \vspace{-0.3cm}
	\begin{enumerate} \itemsep -2pt
	\item Careers in Oceanography: Profiles, \url{http://www.tos.org/resources/career_profiles.html}
	\item Links [includes links to educational material for students in K-12]: \url{http://www.tos.org/resources/links.html}
	\end{enumerate}
\item American Geophysical Union: \vspace{-0.3cm}
	\begin{enumerate} \itemsep -2pt
	\item Bright Students Training as Research Scientists (Bright STaRS): \vspace{-0.2cm}
		\begin{enumerate} \itemsep -2pt
		\item \url{http://www.agu.org/education/diversity_programs/bstars.shtml}
		\item ``High school students participating in after-school and summer research experiences in the Earth and space sciences are invited to participate in the AGU Bright STaRS program. The Bright STaRS program provides a dedicated forum for $\sim$50 students to present their own research results to the scientific community and learn about exciting research, education, and career opportunities in the geosciences.''
		\end{enumerate}
	\end{enumerate}
\item American Geological Institute, AGI: \vspace{-0.3cm}
	\begin{enumerate} \itemsep -2pt
	\item AGI Education Department: \url{http://www.agiweb.org/geoeducation.html}
	\end{enumerate}
\item Society for Science \& the Public (SSP): \vspace{-0.3cm}
	\begin{enumerate} \itemsep -2pt
	\item Intel International Science \& Engineering Fair (Intel ISEF), which is a pre-college science competition: \url{http://www.societyforscience.org/isef/}
	\item Broadcom MASTERS\texttrademark\ competition (which stands for Broadcom Math, Applied Science, Technology and Engineering for Rising Stars): \vspace{-0.2cm}
		\begin{enumerate} \itemsep -2pt
		\item Is a U.S. ``national science, technology, engineering, and math competition for America's $6^{th}$, $7^{th}$, and $8^{th}$ graders.''
		\item \url{http://www.societyforscience.org/masters} or \url{http://www.broadcomfoundation.org/masters/}
		\end{enumerate} 
	\item Science resources: \url{http://www.societyforscience.org/resources}
	\item Science News: \url{http://www.sciencenews.org/}
	\item Science News for Kids (for ``children of ages 9-14, their teachers and their parents''): \url{http://www.societyforscience.org/sciencenewsforkids} and \url{http://www.sciencenewsforkids.org/}
	\end{enumerate}
\item Institute for Operations Research and the Management Sciences (INFORMS): \vspace{-0.3cm}
	\begin{enumerate} \itemsep -2pt
	\item Operations Research: The Science of Better, \url{http://www.scienceofbetter.org/}
	\end{enumerate}
\item Technion - Israel Institute of Technology: \vspace{-0.3cm}
	\begin{enumerate} \itemsep -2pt
	\item SciTech - the summer camp for talented students ($11^{th}$ and $12^{th}$ graders from all over the world): \url{http://www.scitech.technion.ac.il/}
	\end{enumerate}
\item USA Science \& Engineering Festival: \url{http://www.usasciencefestival.org/}
\item Girl Scouts: \vspace{-0.3cm}
	\begin{enumerate} \itemsep -2pt
	\item Girl Scouts of Western New York: \vspace{-0.2cm}
		\begin{enumerate} \itemsep -2pt
		\item STEM Resource Guide: \url{http://www.gswny.org/Data/Documents/STEM%2520Resource%2520Guide%25202010-Oct-11.pdf}
		\item Also, see \url{http://www.gswny.org/Programs/Awards/Gold/}; scroll to the bottom of the page and look under the subsection heading, ``Tell Us About Your Gold Award Project''
		\end{enumerate}
	\item Science, Technology, Engineering and Math (STEM): \url{http://www.girlscouts.org/program/program_opportunities/science/}
	\end{enumerate}
\item American Museum of Science and Energy (AMSE): \vspace{-0.3cm}
	\begin{enumerate} \itemsep -2pt
	\item \url{http://www.amse.org/}
	\item Owned by the US Department of Energy, and managed under Oak Ridge National Laboratory
	\item Educators: \url{http://www.amse.org/content.aspx?article=1140&parent=30}
	\item Educational Programs: \url{http://www.amse.org/content.aspx?article=1139&parent=30}
	\item Home school programs: \url{http://www.amse.org/content.aspx?article=1169&parent=30}
	\item Online resources: \url{http://www.amse.org/content.aspx?article=1170&parent=30}
	\end{enumerate}
\item Center for Energy Workforce Development (CEWD): \vspace{-0.3cm}
	\begin{enumerate} \itemsep -2pt
	\item Teachers and guidance counselors: \vspace{-0.2cm}
		\begin{enumerate} \itemsep -2pt
		\item \url{http://www.cewd.org/educators_index.asp}
		\item Lesson plans for teachers: \url{http://www.cewd.org/educators_lessonplans.asp}
		\end{enumerate}
	\item Parents: \url{http://www.cewd.org/parents_index.asp}
	\end{enumerate}
\item TryScience: \url{http://tryscience.net/tryscinetmain.nsf/Welcome?OpenPage}
\item The Dana Foundation: \vspace{-0.3cm}
	\begin{enumerate} \itemsep -2pt
	\item Brainy Kids: \vspace{-0.2cm}
		\begin{enumerate} \itemsep -2pt
		\item \url{http://www.dana.org/resources/brainykids/}
		\item Fun: \vspace{-0.1cm}
			\begin{enumerate} \itemsep -1pt
			\item \url{http://dana.org/resources/brainykids/detail.aspx?folder_id=104}
			\item Has interactive online games, activities, and fun quizzes on: \vspace{-0.1cm}
				\begin{itemize} \itemsep -1pt
				\item biology
				\item health
				\item neuroscience
				\item astronomy
				\item chemistry
				\item ecology
				\end{itemize}
			\end{enumerate}
		\item The Lab: \vspace{-0.1cm}
			\begin{enumerate} \itemsep -1pt
			\item \url{http://dana.org/resources/brainykids/detail.aspx?folder_id=106}
			\item Has maps of the brain, virtual dissections, resources for science fairs, and virtual microscopes
			\end{enumerate}
		\item Lesson Plans: \vspace{-0.1cm}
			\begin{enumerate} \itemsep -1pt
			\item \url{http://dana.org/resources/brainykids/detail.aspx?folder_id=108}
			\item Includes resources that cover the history of science and technology, lesson plans for K-12 science teachers, and science news for youths.
			\end{enumerate}
		\item The Mindboggling Workbook: \vspace{-0.1cm}
			\begin{enumerate} \itemsep -1pt
			\item \url{http://www.dana.org/uploadedFiles/The_Dana_Alliances/mindboggling_workbook.pdf}
			\item ``A fun-filled activity book about the brain for children in grades K-3 (ages 5-9). Provides an introduction to how the brain works, what the brain does, its importance, and how to take care of it.''
			\end{enumerate}
		\end{enumerate}
	\end{enumerate}
\item University of New Mexico: \vspace{-0.3cm}
	\begin{enumerate} \itemsep -2pt
	\item Department of Mathematics and Statistics: \vspace{-0.2cm}
		\begin{enumerate} \itemsep -2pt
		\item UNM - PNM Statewide Mathematics Contest (sponsored by the PNM Foundation): \url{http://mathcontest.unm.edu/}
		\end{enumerate}
	\end{enumerate}
\item Center for Energy Workforce (CEWD): \vspace{-0.3cm}
	\begin{enumerate} \itemsep -2pt
	\item Get Into Energy: \vspace{-0.2cm}
		\begin{enumerate} \itemsep -2pt
		\item \url{http://www.getintoenergy.com/index.asp} and \url{http://www.getintoenergy.com/careers.asp}
		\item Fun educational resources for students: \url{http://www.getintoenergy.com/students.asp}
		\item Career Quiz: \vspace{-0.1cm}
			\begin{enumerate} \itemsep -1pt
			\item \url{http://www.getintoenergy.com/search/careerquizj.asp}
			\item Help you find out more about career options in the energy field
			\end{enumerate}
		\item Career Resources: \vspace{-0.1cm}
			\begin{enumerate} \itemsep -1pt
			\item \url{http://www.getintoenergy.com/careerresources.asp}
			\item Has information on: \vspace{-0.1cm}
				\begin{itemize} \itemsep -1pt
				\item Training Programs (technical schools and colleges)
				\item Work-based Programs (apprenticeships and internships)
				\item Featured Employers
				\end{itemize}
			\end{enumerate}
		\item Skills Needed in the Energy Field: \vspace{-0.1cm}
			\begin{enumerate} \itemsep -1pt
			\item \url{http://www.getintoenergy.com/skills.asp}
			\item List skills for different kinds of jobs in the energy field
			\end{enumerate}
		\item Information for parents: \url{http://www.getintoenergy.com/Parents.asp}
		\item Information for teachers and guidance counselors: \url{http://www.getintoenergy.com/Educators.asp}
		\end{enumerate}
	\end{enumerate}
\item University of Utah: \vspace{-0.3cm}
	\begin{enumerate} \itemsep -2pt
	\item Department of Electrical and Computer Engineering: \vspace{-0.2cm}
		\begin{enumerate} \itemsep -2pt
		\item Prof. Cynthia Furse: \vspace{-0.1cm}
			\begin{enumerate} \itemsep -1pt
			\item Cynthia Furse, {\it K-12 Engineering Outreach}, August 2007. Available online at: \url{http://www.ece.utah.edu/~cfurse/K12.html}; last accessed on December 10, 2010.
			\item Cynthia Furse, {\it U Dream. U Design. U Create.}, Department of Electrical and Computer Engineering, University of Utah. Available online at: \url{http://www.ece.utah.edu/~cfurse/NSF/}; last accessed on December 10, 2010.
			\end{enumerate}
		\end{enumerate}
	\end{enumerate}
\item Society for Industrial and Applied Mathematics: \vspace{-0.3cm}
	\begin{enumerate} \itemsep -2pt
	\item Public Awareness: \vspace{-0.2cm}
		\begin{enumerate} \itemsep -2pt
		\item Math Competitions, \url{http://www.siam.org/publicawareness/competitions.php}
		\item Moody's Mega Math Challenge (M3 Challenge) is an applied mathematics competition for high school students. Available online at: \url{http://m3challenge.siam.org/}; last accessed on December 13, 2010.
		\item {\it Math Matters, Apply It!}: \url{http://www.siam.org/careers/matters.php}
		\item Nuggets: \url{http://www.siam.org/publicawareness/nuggets.php}
		\end{enumerate}
	\item Society for Industrial and Applied Mathematics, ``Unveiling Why Do Math,'' May 27, 2010. Available online at: \url{http://www.siam.org/about/news-siam.php?id=1741}; last accessed on December 13, 2010.
	\end{enumerate}
\item International Federation of Operational Research Societies (IFORS): \vspace{-0.3cm}
	\begin{enumerate} \itemsep -2pt
	\item Association of European Operational Research Societies (EURO): \vspace{-0.2cm}
		\begin{enumerate} \itemsep -2pt
		\item {\it What is Operational Research?}: \url{http://www.euro-online.org/display.php?pageid=197&}
		\item Applications of OR in music, literature, and aesthetics: \url{http://www.euro-online.org/display.php?pageid=211&}
		\item 24 Hours Operations Research: \url{http://www.24hor.org/}
		\item Branding OR: \url{http://www.euro-online.org/display.php?pageid=198&}
		\end{enumerate}
	\end{enumerate}
\item American Institute of Aeronautics and Astronautics (AIAA): \vspace{-0.3cm}
	\begin{enumerate} \itemsep -2pt
	\item Students \& Educators: \url{http://www.aiaa.org/content.cfm?pageid=5}
	\item Ask An Engineer: \url{http://www.aiaa.org/content.cfm?pageid=214}
	\item Kid's Place: \vspace{-0.2cm}
		\begin{enumerate} \itemsep -2pt
		\item \url{http://www.aiaa.org/content.cfm?pageid=473}
		\item Enjoy games, puzzles, fun experiments, teen-recommended books and movies, and more.
		\end{enumerate}
	\item History of Flight Timeline: \url{http://www.aiaa.org/content.cfm?pageid=260}
	\item Ask Polaris: \vspace{-0.2cm}
		\begin{enumerate} \itemsep -2pt
		\item \url{http://www.askpolaris.org/}
		\item Resource for career exploration in aerospace engineering and related fields
		\end{enumerate}
	\end{enumerate}
\item Massachusetts Institute of Technology: \vspace{-0.3cm}
	\begin{enumerate} \itemsep -2pt
	\item MIT School of Engineering: \vspace{-0.2cm}
		\begin{enumerate} \itemsep -2pt
		\item Lemelson-MIT Program: \vspace{-0.1cm}
			\begin{enumerate} \itemsep -1pt
			\item \url{http://web.mit.edu/invent/}
			\item Inventor's Handbook: \url{http://web.mit.edu/invent/h-main.html}
			\item Games \& Trivia; \url{http://web.mit.edu/invent/g-main.html}
			\item Links \& Resources: \url{http://web.mit.edu/invent/r-main.html}
			\end{enumerate}
		\end{enumerate}
	\end{enumerate}
\item BT Group plc: \vspace{-0.3cm}
	\begin{enumerate} \itemsep -2pt
	\item British Telecommunications plc (BT): \vspace{-0.2cm}
		\begin{enumerate} \itemsep -2pt
		\item BT Young Scientist \& Technology Exhibition: \vspace{-0.1cm}
			\begin{enumerate} \itemsep -1pt
			\item \url{http://www.btyoungscientist.com/}
			\item \url{http://www.btyoungscientist.com/all-you-need-to-know/}
			\item Science and technology fair for high/secondary school students in Ireland
			\end{enumerate}
		\end{enumerate}
	\end{enumerate}
\item NHS Medical Careers: \vspace{-0.3cm}
	\begin{enumerate} \itemsep -2pt
	\item \url{http://www.medicalcareers.nhs.uk/Default.aspx}
	\item Provides information about careers in medicine for prospective medical students, medical students, medical school graduates (or young medical professionals), (medical speciality) trainers, and medical specialists.
	\end{enumerate}
\item British Science Association: \vspace{-0.3cm}
	\begin{enumerate} \itemsep -2pt
	\item British Science Festival: \vspace{-0.2cm}
		\begin{enumerate} \itemsep -2pt
		\item \url{http://www.britishscienceassociation.org/web/BritishScienceFestival/AboutFestival/index.htm}
		\item Festival Student Bursaries: \url{http://www.britishscienceassociation.org/web/BritishScienceFestival/StudentBursaries/index.htm}
		\end{enumerate}
	\item National Science \& Engineering Week: \url{http://www.britishscienceassociation.org/web/NSEW/index.htm}
	\item Clubs, CREST Awards and Fairs (programs and activities for children and youth, 5-19 years of age): \url{http://www.britishscienceassociation.org/web/ccaf/index.htm}
	\item National Science \& Engineering Competition: \url{http://www.britishscienceassociation.org/web/NSEC/index.htm} and \url{http://www.thebigbangfair.co.uk/nsec/}
	\end{enumerate}
\item Research Councils UK (RCUK): \vspace{-0.3cm}
	\begin{enumerate} \itemsep -2pt
	\item \url{http://www.rcuk.ac.uk/per/Pages/Schools.aspx}
	\item Schoolscience: \vspace{-0.2cm}
		\begin{enumerate} \itemsep -2pt
		\item \url{http://www.schoolscience.co.uk/}
		\item For students and educators in K-12 to enrich the learning experiences of science topics, and help students connect classroom material to the real world.
		\item Teacher Zone - professional resources for teachers: \url{http://www.schoolscience.co.uk/teacher_zone.cfm}
		\item Interactive Learning Resources: \url{http://www.schoolscience.co.uk/interactives.cfm}
		\item Free Resources: \url{http://www.schoolscience.co.uk/freebies.cfm}
		\item Competitions: \url{http://www.schoolscience.co.uk/competitions.cfm}
		\item Research focus: \url{http://www.schoolscience.co.uk/research_focus.cfm}
		\item Resources on the World Wide Web: \url{http://www.schoolscience.co.uk/sciencelink.cfm}
		\end{enumerate}
	\item Researchers in Residence (RinR): \vspace{-0.2cm}
		\begin{enumerate} \itemsep -2pt
		\item \url{http://www.researchersinresidence.ac.uk/cms/schools-colleges/}
		\item For students in middle and high schools to job shadow (observe first-hand) a Ph.D. student or postdoctoral researcher in her/his research activities for up to a week, so that students can learn what doing research in her/his research area is like. In addition, the researcher would explain in laypeople's terms what her/his research is about. It can be considered as an externship program.
		\end{enumerate}
	\item Nuffield Bursaries: \vspace{-0.2cm}
		\begin{enumerate} \itemsep -2pt
		\item \url{http://www.nuffieldfoundation.org/capacity-building}
		\item \url{http://www.nuffieldfoundation.org/science-bursaries-schools-and-colleges}
		\item For high school juniors/seniors to pursue a research internship in science and engineering.
		\end{enumerate}
	\item CREST (Creativity in Science and Technology): \vspace{-0.2cm}
		\begin{enumerate} \itemsep -2pt
		\item \url{http://www.britishscienceassociation.org/web/ccaf/CREST/index.htm}
		\item Program to help students get engaged in a science or engineering project, where they learn how to solve real problems in science or engineering.
		\end{enumerate}
	\end{enumerate}
\item Nuffield Foundation: \vspace{-0.3cm}
	\begin{enumerate} \itemsep -2pt
	\item Science bursaries for schools and colleges: \url{http://www.nuffieldfoundation.org/science-bursaries-schools-and-colleges}
	\item Students: \url{http://www.nuffieldfoundation.org/students}
	\item Twenty First Century Science: \vspace{-0.2cm}
		\begin{enumerate} \itemsep -2pt
		\item \url{http://www.21stcenturyscience.org/}
		\item ``Twenty First Century Science is a set of GCSE science courses giving all 14-16-year-olds a worthwhile and inspiring experience of science. The strength of the programme is that it meets the needs, through flexible options, of those who will go on to be professional scientists and of those who will not.''
		\item The Courses: \url{http://www.21stcenturyscience.org/the-courses/}
		\item Assessment overview: \url{http://www.21stcenturyscience.org/assess/}
		\item Teaching resources: \url{http://www.21stcenturyscience.org/resources/}
		\end{enumerate}
	\item Science in Society: \vspace{-0.2cm}
		\begin{enumerate} \itemsep -2pt
		\item \url{http://www.scienceinsocietyadvanced.org/}
		\item ``Science in Society is an interesting and topical GCE advanced level course. It aims to develop the knowledge and skills that are needed for students to understand how science works, analyse contemporary issues involving science and technology and communicate their scientific appreciation and understanding to others.''
		\end{enumerate}
	\item Parents: \url{http://www.nuffieldfoundation.org/parents}
	\item Education: \url{http://www.nuffieldfoundation.org/education}
	\item Teachers (has excellent resources for science and mathematics): \url{http://www.nuffieldfoundation.org/teachers}
	\item Capacity building: \url{http://www.nuffieldfoundation.org/capacity-building}
	\end{enumerate}
\item The Story of Stuff Project (by Annie Leonard): \vspace{-0.3cm}
	\begin{enumerate} \itemsep -2pt
	\item \url{http://www.storyofstuff.com/}
	\item ``The Story of Stuff Project was created by Annie Leonard to leverage and extend the film's impact. We amplify public discourse on a series of environmental, social and economic concerns and facilitate the growing Story of Stuff community's involvement in strategic efforts to build a more sustainable and just world.''
	\item Resources: \vspace{-0.2cm}
		\begin{enumerate} \itemsep -2pt
		\item \url{http://www.storyofstuff.com/resources.php}
		\item The Story of Stuff Project PDFs: \url{http://www.storyofstuff.com/dl-pdfs.php}
		\item Teaching Tools: \url{http://www.storyofstuff.com/teach.php}
		\item More About Stuff: \url{http://www.storyofstuff.com/aboutstuff.php}
		\item Recommended Reading \& Bibliography: \url{http://www.storyofstuff.com/reading.php}
		\item Get Involved: \url{http://www.storyofstuff.com/getinvolved.php}
		\item Curricula: \url{http://storyofstuff.org/curricula.php}
		\end{enumerate}
	\end{enumerate}
\item Facing the Future: \vspace{-0.3cm}
	\begin{enumerate} \itemsep -2pt
	\item \url{http://www.facingthefuture.org/}
	\item ``{\it Facing the Future} engages students in learning by making academics relevant to their lives. We empower students to think critically, develop a global perspective, and participate in positive solutions for a sustainable future.''
	\item Curriculum Alignment with Education Standards: \url{http://www.facingthefuture.org/Curriculum/AlignmentwithEducationStandards/tabid/116/Default.aspx}
	\item Global Sustainability Curriculum Finder: \url{http://www.facingthefuture.org/Curriculum/FindCurriculumthatisRightforYou/tabid/68/Default.aspx}
	\item Download FREE Global Issues and Sustainability Curriculum: \url{http://www.facingthefuture.org/Curriculum/DownloadFreeCurriculum/tabid/114/Default.aspx}
	\item Classroom Examples: How Engaging Curriculum Can Help Address Classroom Challenges, \url{http://www.facingthefuture.org/ForEducators/ClassroomExamples/tabid/213/Default.aspx}
	\item Our Impact on Student Achievement: \url{http://www.facingthefuture.org/ForEducators/OurImpactonStudentAchievement/tabid/73/Default.aspx}
	\item Action Project Database: \url{http://www.facingthefuture.org/ServiceLearning/ActionProjectDatabase/tabid/94/Default.aspx}
	\item Service Learning Examples: \url{http://www.facingthefuture.org/ServiceLearning/ExamplesofStudentsTakingAction/tabid/147/Default.aspx}
	\item Curriculum: \url{http://www.facingthefuture.org/Curriculum/CurriculumHome/tabid/113/Default.aspx}
	\end{enumerate}
\item U.S. Department of Energy: \vspace{-0.3cm}
	\begin{enumerate} \itemsep -2pt
	\item Office of Science: \vspace{-0.2cm}
		\begin{enumerate} \itemsep -2pt
		\item U.S. Department of Energy (DOE) National Science Bowl\textregistered: \vspace{-0.1cm}
			\begin{enumerate} \itemsep -1pt
			\item \url{http://www.scied.science.doe.gov/nsb/default.htm}
			\item ``The U.S. Department of Energy (DOE) National Science Bowl\textregistered\ is a nationwide academic competition that tests students' knowledge in all areas of science. High school and middle school students are quizzed in a fast paced question-and-answer format similar to Jeopardy. Competing teams from diverse backgrounds are comprised of four students, one alternate, and a teacher who serves as an advisor and coach.''
			\end{enumerate}
		\item Argonne National Laboratory: \vspace{-0.1cm}
			\begin{enumerate} \itemsep -1pt
			\item Division of Educational Programs: \vspace{-0.1cm}
				\begin{itemize} \itemsep -1pt
				\item Newton BBS Ask A Scientist: \url{http://www.newton.dep.anl.gov/aas.htm}
				\end{itemize}
			\end{enumerate}
		\end{enumerate}
	\item Office of Energy Efficiency and Renewable Energy (EERE): \vspace{-0.2cm}
		\begin{enumerate} \itemsep -2pt
		\item Kids Saving Energy: \vspace{-0.1cm}
			\begin{enumerate} \itemsep -1pt
			\item \url{http://www.eere.energy.gov/kids/index.html}
			\item K-12 Lesson Plans \& Activities: \url{http://www1.eere.energy.gov/education/lessonplans/}
			\item Energy Savers: \url{http://www.energysavers.gov/}
			\item Games and activities: \url{http://www.eere.energy.gov/kids/games.html}
			\item Smart home: \url{http://www.eere.energy.gov/kids/smart_home.html}
			\item About renewable energy: \url{http://www.eere.energy.gov/kids/renergy.html}
			\end{enumerate}
		\end{enumerate}
	\item Contest \& Competitions: \url{http://www.energy.gov/contests&competitions.htm}
	\end{enumerate}
\item United States Department of Defense (DoD): \vspace{-0.3cm}
	\begin{enumerate} \itemsep -2pt
	\item National Defense Education Program; Defense Advanced Research Projects Agency (DARPA): \vspace{-0.2cm}
		\begin{enumerate} \itemsep -2pt
		\item Resource for Students: \url{http://www.ndep.us/GetInvoStu.aspx}
		\item Resource for Educators: \url{http://www.ndep.us/GetInvoTea.aspx}
		\end{enumerate}
	\end{enumerate}
\item Project Lead The Way: \vspace{-0.3cm}
	\begin{enumerate} \itemsep -2pt
	\item \url{http://www.pltw.org/}
	\item Getting started: \url{http://www.pltw.org/getting-started/getting-started}
	\item Program support: \url{http://www.pltw.org/program-support/program-support}
	\item Grants available to schools and teachers: \url{http://www.pltw.org/pltw-in-the-news/grants-available-schools-teachers-and-classrooms}
	\item Students: \url{http://www.pltw.org/students/students}
	\item Educators and Administrators: \url{http://www.pltw.org/educators-administrators/educators-administrators-overview}
	\item Parents: \url{http://www.pltw.org/parents/parents}
	\end{enumerate}
\item National Science Teachers Association: \vspace{-0.3cm}
	\begin{enumerate} \itemsep -2pt
	\item \url{http://www.exploravision.org/}
	\item Science competition for K-12 students
	\end{enumerate}
\item American Mathematical Society: \vspace{-0.3cm}
	\begin{enumerate} \itemsep -2pt
	\item Some career resources for mathematics: \url{http://e-math.ams.org/samplings/samplings}
	\end{enumerate}
\item American Institute of Physics (AIP): \vspace{-0.3cm}
	\begin{enumerate} \itemsep -2pt
	\item Physics Success Stories: \url{http://www.aip.org/success/}
	\item Physics is for you; Career Services Division: \vspace{-0.2cm}
		\begin{enumerate} \itemsep -2pt
		\item \url{http://www.aip.org/careersvc/pify/}
		\item Physicists at work: \url{http://www.aip.org/careersvc/pify/yellow.html}
		\end{enumerate}
	\item Society of Physics Students (SPS): \vspace{-0.2cm}
		\begin{enumerate} \itemsep -2pt
		\item Careers Using Physics (CUP): \vspace{-0.1cm}
			\begin{enumerate} \itemsep -1pt
			\item \url{http://www.spsnational.org/cup/}
			\item Advice: \url{http://www.spsnational.org/cup/advice/index.html}
			\item Resources: \url{http://www.spsnational.org/cup/resources.html}
			\item Preparing to Teach: \url{http://www.spsnational.org/cup/teach/index.html}
			\end{enumerate}
		\end{enumerate}
	\item ComPADRE Digital Library: \vspace{-0.2cm}
		\begin{enumerate} \itemsep -2pt
		\item \url{http://www.compadre.org/}
		\item The Physics Career Resource: \url{http://www.compadre.org/careers/}
		\end{enumerate}
	\item Career guidance for high school and undergraduate students: \url{http://www.aip.org/statistics/trends/career.html}
	\item Gayle A. Buck, Jack G. Hehn, and Diandra L. Leslie-Pelecky (Editors), ``The Role of Physics Departments in Preparing K-12 Teachers,'' American Institute of Physics. Available online at: \url{http://www.aip.org/education/teacherprep/}; last accessed on January 9, 2010.
	\item American Geophysical Union: \vspace{-0.2cm}
		\begin{enumerate} \itemsep -2pt
		\item Students \& Teachers: \url{http://www.agu.org/education/students_teachers.shtml}
		\item Diversity Programs: \url{http://www.agu.org/education/diversity_programs/}
		\end{enumerate}
	\end{enumerate}
\item Institute for Operations Research and the Management Sciences (INFORMS): \vspace{-0.3cm}
	\begin{enumerate} \itemsep -2pt
	\item Career FAQ's: \url{http://www.informs.org/Build-Your-Career/INFORMS-Student-Union/Career-Center/Career-FAQ-s}
	\end{enumerate}
\item American Institute of Mathematics: \vspace{-0.3cm}
	\begin{enumerate} \itemsep -2pt
	\item Math Teachers' Circle Network: \vspace{-0.2cm}
		\begin{enumerate} \itemsep -2pt
		\item Classroom Materials: \url{http://www.mathteacherscircle.org/resources/classroommaterials.html}
		\item Helpful Resources: \url{http://www.mathteacherscircle.org/resources/general.html}
		\end{enumerate}
	\item Resources for the Math Community: \vspace{-0.2cm}
		\begin{enumerate} \itemsep -2pt
		\item \url{http://www.aimath.org/mathcommunity/}
		\item David W. Farmer, ``The AIM REU: individual projects with a common theme,'' in the {\it Proceedings of the Conference on Promoting Undergraduate Research in Mathematics}, American Mathematical Society, 2006. Available online at: \url{http://www.aimath.org/mathcommunity/farmerREU.pdf}; last accessed on January 9, 2010. [ ``AIM Research Experience for Undergraduates (REU)'' ]
		\item Sally Koutsoliotas and David W. Farmer, ``Preparing students to give talks,'' American Institute of Mathematics. Available online at: \url{http://www.aimath.org/mathcommunity/studenttalks.pdf}; last accessed on January 9, 2010. [ ``Preparing students to give talks'' ]
		\end{enumerate}
	\end{enumerate}
\item Invent Now: \vspace{-0.3cm}
	\begin{enumerate} \itemsep -2pt
	\item Camp Invention: \vspace{-0.2cm}
		\begin{enumerate} \itemsep -2pt
		\item ``Summer enrichment program for children entering grades one through six.''
		\item ``The Camp Invention program instills vital 21st century life skills such as problem-solving and teamwork through hands-on fun!''
		\item Parents: \url{http://www.invent.org/camp/parents.aspx}
		\item Teachers: \url{http://www.invent.org/camp/teachers.aspx}
		\end{enumerate}
	\end{enumerate}
\item Massachusetts Institute of Technology: \vspace{-0.3cm}
	\begin{enumerate} \itemsep -2pt
	\item MIT School of Engineering: \vspace{-0.2cm}
		\begin{enumerate} \itemsep -2pt
		\item Lemelson-MIT Program: \vspace{-0.1cm}
			\begin{enumerate} \itemsep -1pt
			\item \url{http://web.mit.edu/invent/}
			\item Invention Dimension (for children): \url{http://web.mit.edu/invent/invent-main.html}
			\end{enumerate}
		\end{enumerate}
	\end{enumerate}
\item The Lemelson Foundation: \vspace{-0.3cm}
	\begin{enumerate} \itemsep -2pt
	\item \url{http://web.mit.edu/invent/w-foundation.html}
	\item Programs \& Grants: \url{http://www.lemelson.org/programs-grants}
	\item Grantmaking: \url{http://www.lemelson.org/grantmaking}
	\end{enumerate}
\item Smithsonian Institution: \vspace{-0.3cm}
	\begin{enumerate} \itemsep -2pt
	\item Smithsonian Kids: \url{http://www.si.edu/Kids}
	\item National Museum of American History: \vspace{-0.2cm}
		\begin{enumerate} \itemsep -2pt
		\item Lemelson Center for the Study of Invention and Innovation: \vspace{-0.1cm}
			\begin{enumerate} \itemsep -1pt
			\item \url{http://inventionatplay.org/index.html}
			\item Resources: \url{http://inventionatplay.org/resources.html}
			\end{enumerate}
		\end{enumerate}
	\end{enumerate}
%%%%%%%%%%%%%%%%%%%%%%%%%%%%%%%%%%%%%%%%
%%%%%%%%%%%%%%%%%%%%%%%%%%%%%%%%%%%%%%%%
\item Scholarships: \vspace{-0.3cm}
	\begin{enumerate} \itemsep -2pt
	\item IEEE Presidents' Scholarship: \url{http://www.ieee.org/education_careers/education/preuniversity/scholarship.html}
	\item ACM/SIGDA {\it P. O. Pistilli scholarship}: \vspace{-0.1cm}
		\begin{enumerate} \itemsep -1pt
		\item Supported by the Design Automation Conference which ACM/SIGDA sponsors, the objective of the P. O. Pistilli Scholarship is to increase the pool of professionals in Electrical Engineering and Computer Science from underrepresented groups (Women, African American, Hispanic, American Indian, and Disabled).
		\item Scholarships of \$4000 per year, renewable for up to 5 years, are awarded annually to 2-7 high school seniors from the above mentioned under represented groups who have a 3.00 GPA or better (on a 4.00 scale), have demonstrated high achievement in math and science courses, have expressed a strong desire to pursue careers in electrical engineering, computer engineering, or computer science, and who have demonstrated substantial financial need.
		\item U.S. citizenship is not required, but applicants must be U.S. residents when they apply and must plan to attend an accredited US college or university.
		\item \url{http://www.sigda.org/pistilli.html}
		\end{enumerate}
	\item Engineering Education Service Center (EESC): \url{http://www.engineeringedu.com/scholars.html}
	\item ASME-ASME Auxiliary FIRST Clarke Scholarships: \url{http://www.asme.org/Education/College/FinancialAid/High_School_Seniors.cfm} and \url{http://www.asme.org/Education/College/FinancialAid/Auxiliary_FIRST_Clarke.cfm}
	\item International Petroleum Institute�s High School Scholarships (for individuals entering a college program in engineering): \url{http://www.asme-ipti.org/public/pagscholarshipprograms.aspx}
	\item American Institute of Chemical Engineers (AIChE): \vspace{-0.2cm}
		\begin{enumerate} \itemsep -2pt
		\item Fuels and Petrochemicals Division Scholarship (for high school students entering undergraduate programs in engineering or science that are related to fuels and petrochemicals): \url{http://www.aiche.org/Students/Awards/F_PDScholarship.aspx}
		\item Minority Scholarship Awards for Incoming College Freshmen (for underrepresented minorities entering an undergraduate chemical engineering program): \url{http://www.aiche.org/Students/Awards/MinorityScholarshipAwardsIncomingFreshmen.aspx}
		\end{enumerate}
	\item Sallie Mae Fund: \vspace{-0.3cm}
		\begin{enumerate} \itemsep -2pt
		\item \url{http://www.thesalliemaefund.org/smfnew/index.html}
		\item List of scholarship resources: \url{http://www.thesalliemaefund.org/smfnew/sections/search.html}
		\item Top 10 Tips for Planning and Paying for College: \url{http://www.thesalliemaefund.org/smfnew/fin_aid/index.html}
		\item Scholarships: \url{http://www.thesalliemaefund.org/smfnew/scholarship/index.html} and \url{http://www.thesalliemaefund.org/smfnew/sections/apply.html}
		\item Important information for parents about saving for college and getting financial aid: \vspace{-0.2cm}
			\begin{enumerate} \itemsep -2pt
			\item \url{http://www.thesalliemaefund.org/smfnew/sections/download.html}
			\item This information is also available in Spanish. Summaries are also available in other languages such as: \vspace{-0.1cm}
				\begin{itemize} \itemsep -1pt
				\item French
				\item German
				\item Italian
				\item Korean
				\item Russian
				\item Simplified and Traditional Chinese
				\item Tagalog
				\item Vietnamese
				\end{itemize}
			\item Top 10 Tips for Planning and Paying for College: \url{http://www.thesalliemaefund.org/smfnew/fin_aid/index.html}
			\end{enumerate}
		\item Kids2College program: \url{http://www.thesalliemaefund.org/smfnew/initiatives/kidscollege.html}
		\item For African-American individuals entering college: \vspace{-0.2cm}
			\begin{enumerate} \itemsep -2pt
			\item Black College Dollars: \url{http://www.thesalliemaefund.org/smfnew/scholarship_directory/index.html}
			\item \url{http://www.thesalliemaefund.org/smfnew/initiatives/aa.html}
			\end{enumerate}
		\item For Hispanic Americans, or Latinos/Latinas: \vspace{-0.2cm}
			\begin{enumerate} \itemsep -2pt
			\item \url{http://www.thesalliemaefund.org/smfnew/pdf/Scholarship_Directory.pdf}
			\item Latino College Dollars: \url{http://www.latinocollegedollars.org/}
			\end{enumerate}
		\end{enumerate}
	\item {\it American Chemical Society}: \vspace{-0.3cm}
		\begin{enumerate} \itemsep -2pt
		\item ACS Scholars Program (for underrepresented minorities in, or entering, an undergraduate program in chemistry, biochemistry, or chemical engineering): \url{http://portal.acs.org/portal/acs/corg/content?_nfpb=true&_pageLabel=PP_SUPERARTICLE&node_id=1650&use_sec=false&sec_url_var=region1&__uuid=b3b583cf-18ae-4fb0-9375-33f75a0ccf49}
		\item Project SEED Scholarships (for high school seniors who have worked at least one summer at a science institute under the Project SEED program): \url{http://portal.acs.org/portal/acs/corg/content?_nfpb=true&_pageLabel=PP_SUPERARTICLE&node_id=2031&use_sec=false&sec_url_var=region1&__uuid=99bc6a62-3e78-4b2a-be3f-50b28f7ff265}
		\end{enumerate}
	\item The Posse Foundation: \url{http://www.possefoundation.org/}
	\item Hispanic Scholarship Fund (HSF) scholarship programs for high school students: \url{http://www.hsf.net/innerContent.aspx?id=426}
	\item Asian \& Pacific Islander American Scholarship Fund (APIASF): scholarships for individuals entering college as freshmen; see \url{http://www.apiasf.org/scholarship_apiasf.html}
	\item Nationally Coveted College Scholarships, Graduate School Fellowships \& Postdoctoral Awards: \url{http://scholarships.fatomei.com/}
	\item {\it SPIE} Scholarship Program (for high school students entering college to study optics, photonics, imaging, optoelectronics, or related program): \url{http://spie.org//x1733.xml?WT.svl=mddm14}
	\item Susan G. Komen for the Cure\textregistered: The Komen College Scholarship Program, \url{http://ww5.komen.org/ResearchGrants/CollegeScholarshipAward.html}
	\item National Society of Professional Engineers's list of scholarships for high school students: \url{http://www.nspe.org/Students/Scholarships/index.html}
	\item AWM Essay Contest: Biographies of Contemporary Women in Mathematics; see \url{http://www.awm-math.org/biographies/contest.html}
	\item National Engineers Week Future City Competition (students from $6^{th}$--$8^{th}$ grades): \url{http://www.futurecity.org/}
	\item National Ocean Sciences Bowl: \vspace{-0.2cm}
		\begin{enumerate} \itemsep -2pt
		\item \url{http://www.nosb.org/ocean-careers/}
		\item National Ocean Scholar Program (for high school seniors who are current/past participants of the Bowl, and are seeking a career in the ocean sciences or a marine-related field): \url{http://www.nosb.org/ocean-careers/national-ocean-scholar-program/}
		\end{enumerate}
	\item National Center for Women \& Information Technology (NCWIT): \vspace{-0.2cm}
		\begin{enumerate} \itemsep -2pt
		\item NCWIT Award for Aspirations in Computing (for young women in high school): \url{http://www.ncwit.org/work.awards.aspiration.html}
		\end{enumerate}
	\end{enumerate}
%%%%%%%%%%%%%%%%%%%%%%%%%%%%%%%%%%%%%%%%
%%%%%%%%%%%%%%%%%%%%%%%%%%%%%%%%%%%%%%%%
\item Resources for teachers/educators: \vspace{-0.3cm}
	\begin{enumerate} \itemsep -2pt
	\item Google: \vspace{-0.2cm}
		\begin{enumerate} \itemsep -2pt
		\item Google Teacher Academy (for teachers to learn how to use Google technologies to facilitate teaching): \url{http://www.google.com/educators/gta.html}
		\item Classroom activities (suggestions): \url{http://www.google.com/educators/activities.html}
		\end{enumerate}
	\item IEEE Teacher In-Service Program (TISP): \vspace{-0.2cm}
		\begin{enumerate} \itemsep -2pt
		\item \url{http://www.ieee.org/education_careers/education/preuniversity/tispt/index.html}
		\item Lesson Plans for Pre-university Instructors: \url{http://www.ieee.org/education_careers/education/preuniversity/resources/index.html}
		\end{enumerate}
	\item Global Challenge Award: \url{http://www.globalchallengeaward.org/display/public/Home}
	\item Teachers' Domain (to teach students about science, engineering, and the arts): \url{http://www.teachersdomain.org/}
	\item {\it TeachEngineering} digital library: \vspace{-0.2cm}
		\begin{enumerate} \itemsep -2pt
		\item The {\it TeachEngineering} digital library provides teacher-tested, standards-based engineering content for K-12 teachers engineering content for K12 teachers to use in science and math classrooms. Engineering lessons connect real-world experiences with curricular content already taught in K-12 classrooms. Mapped to educational content standards, {\it TeachEngineering}'s comprehensive curricula are hands-on, free, and relevant to children's daily lives.
		\item \url{http://www.teachengineering.com/index.php}
		\end{enumerate}
	\item Engineering Pathway: \url{http://www.engineeringpathway.com/ep/index.jhtml}
	\item {\it American Society of Mechanical Engineers, ASME}: \url{http://www.asme.org/Education/PreCollege/TeacherResources/}
	\item {\it National Science Foundation} resources for the K-12 classroom: \url{http://nsf.gov/news/classroom/engineering.jsp}
	\item {\it NASA}: \url{http://www.nasa.gov/audience/foreducators/index.html}
	\item The Mathematical Association of America: \vspace{-0.2cm}
		\begin{enumerate} \itemsep -2pt
		\item Pre-College Programs: \url{http://www.maa.org/funding/pre_college.html}. Also, see \url{http://www.maa.org/funding/undergraduate.html}.
		\item Special Interest Group of the Mathematical Association of America on the use of the World-Wide Web in Undergraduate Mathematics Instruction (Web SIGMAA). Available at: \url{http://math.chapman.edu/websigmaa/index.php/Main_Page}; last accessed on September 2, 2010.
		\item SIGMAA TAHSM (Teaching Advanced High School Mathematics). Available at: \url{http://sigmaa.maa.org/tahsm/}; last accessed on September 2, 2010.
		\item Special Interest Group on Statistics Education: \url{http://sigmaa.maa.org/stat-ed/}
		\end{enumerate}
	\item Math for America: \vspace{-0.2cm}
		\begin{enumerate} \itemsep -2pt
		\item M$f$A Master Teacher Fellowship program: \vspace{-0.1cm}
			\begin{enumerate} \itemsep -1pt
			\item The Math for America Master Teacher Fellowship program rewards exceptional public secondary school math teachers with a four-year Fellowship.
			\item M$f$A Master Teacher Fellowships are currently available in Berkeley, Boston and New York City.
			\item \url{http://www.mathforamerica.org/web/guest/master-teachers}
			\end{enumerate}
		\item M$f$A Early Career Fellows: \vspace{-0.1cm}
			\begin{enumerate} \itemsep -1pt
			\item The Math for America Early Career Fellowship is awarded to public secondary school math teachers early in their careers.
			\item The M$f$A Early Career Fellowship requires a commitment of four years and is available in New York City. 
			\item \url{http://www.mathforamerica.org/early-career-fellows}
			\end{enumerate}
		\item M$f$A Fellows: \vspace{-0.1cm}
			\begin{enumerate} \itemsep -1pt
			\item \url{http://www.mathforamerica.org/web/guest/mfa-fellows}
			\end{enumerate}
		\item Teachers resources: \url{http://www.mathforamerica.org/web/guest/teacher-resources} and \url{http://www.mathforamerica.org/teacher-resources/classroom} (classroom resources)
		\item Resources for professional development (teachers): \url{http://www.mathforamerica.org/teacher-resources/professional}
		\item \url{http://www.mathforamerica.org/home}
		\end{enumerate}
	\item Association for Symbolic Logic (ASL): \vspace{-0.2cm}
		\begin{enumerate} \itemsep -2pt
		\item Guidelines on Logic Education: \url{http://www.ucalgary.ca/aslcle/guidelines}
		\item Educational Logic Software: \url{http://www.ucalgary.ca/aslcle/logic-courseware}
		\end{enumerate}
	\item Consortium for Ocean Leadership: \vspace{-0.2cm}
		\begin{enumerate} \itemsep -2pt
		\item Educational Resources: \url{http://www.oceanleadership.org/gulf-oil-spill/educational-resources/}
		\item The JOIDES Resolution (The JR) scientific research vessel [ Deep Earth Academy ]: \vspace{-0.1cm}
			\begin{enumerate} \itemsep -1pt
			\item Teacher Resources (to teach students about geology and physical geography): \url{http://joidesresolution.org/node/46}
			\item Teachers at Sea/On-board Education Officer (for teachers to go on scientific expeditions on board): \url{http://joidesresolution.org/node/453}
			\end{enumerate}
		\item Integrated Ocean Drilling Program (IODP) -- IODP United States Implementing Organization (IODP-USIO): \vspace{-0.1cm}
			\begin{enumerate} \itemsep -1pt
			\item Teaching Materials: \url{http://www.iodp-usio.org/Education/educ.html}
			\end{enumerate}
		\item Deep Earth Academy (includes suggested ``curriculum and classroom activities for kindergarten through college level''): \vspace{-0.1cm}
			\begin{enumerate} \itemsep -1pt
			\item \url{http://www.oceanleadership.org/education/deep-earth-academy/}
			\item For Educators: \url{http://www.oceanleadership.org/education/deep-earth-academy/educators/}
			\end{enumerate}
		\end{enumerate}
	\item Virginia Institute of Marine Science (College of William and Mary): \vspace{-0.2cm}
		\begin{enumerate} \itemsep -2pt
		\item Bridge Ocean Education Teacher Resource Center: \url{http://web.vims.edu/bridge/?svr=www#}
		\end{enumerate}
	\item American Geological Institute: \vspace{-0.2cm}
		\begin{enumerate} \itemsep -2pt
		\item Awards for teachers: \url{http://www.agiweb.org/education/awards/index.html}
		\item Edward C. Roy, Jr. Award For Excellence in K-8 Earth Science Teaching (for middle school teachers in the US who are teaching earth science): \url{http://www.agiweb.org/education/awards/ed-roy/}
		\item Presidential Awards for Excellence in Mathematics \& Science Teaching, PAEMST (for kindergarten and K-12 teachers in the US who are teaching students about STEM fields): \url{http://www.agiweb.org/education/awards/paemst.html}
		\item National Association of Geoscience Teachers (NAGT) Outstanding Earth Science Teacher Award: \url{http://www.agiweb.org/education/awards/nagt.html}
		\item American Association of Petroleum Geologists' (AAPG) National Earth Science Teacher of the Year Award: \url{http://www.agiweb.org/education/awards/aapg.html}
		\item Curriculum Materials and Activities: \url{http://www.agiweb.org/education/curriculum/index.html}
		\item K-12 Professional Development Programs: \url{http://www.agiweb.org/education/pd/index.html}
		\item Educational Resources: \url{http://www.agiweb.org/education/resource/index.html}
		\end{enumerate}
	\item Institute for Broadening Participation: \vspace{-0.2cm}
		\begin{enumerate} \itemsep -2pt
		\item PathwaysToScience.org: \vspace{-0.1cm}
			\begin{enumerate} \itemsep -1pt
			\item For K-12 teachers (resources to encourage students to be interested in STEM): \url{http://www.pathwaystoscience.org/Teachers.asp}
			\end{enumerate}
		\end{enumerate}
	\item National Science Foundation: \vspace{-0.2cm}
		\begin{enumerate} \itemsep -2pt
		\item The National Science Digital Library (NSDL): \vspace{-0.1cm}
			\begin{enumerate} \itemsep -1pt
			\item Resources for K-12 Teachers: \url{http://nsdl.org/resources_for/k12_teachers/}
			\end{enumerate}
		\end{enumerate}
	\item National Academy of Engineering, NAE: \vspace{-0.2cm}
		\begin{enumerate} \itemsep -2pt
		\item NAE Grand Challenges: \vspace{-0.1cm}
			\begin{enumerate} \itemsep -1pt
			\item Includes a list of NAE Grand Challenges, which indicate some of the challenges faced by people on a global scale that can be partially solved by engineers. This can be used to get children and youths to be excited about engineering. 
			\item NAE Grand Challenges: \vspace{-0.1cm}
				\begin{itemize} \itemsep -1pt
				\item Make solar energy economical
				\item Provide energy from fusion
				\item Develop carbon sequestration methods
				\item Manage the nitrogen cycle
				\item Provide access to clean water
				\item Restore and improve urban infrastructure
				\item Advance health informatics
				\item Engineer better medicines
				\item Reverse-engineer the brain
				\item Prevent nuclear terror
				\item Secure cyberspace
				\item Enhance virtual reality
				\item Advance personalized learning
				\item Engineer the tools of scientific discovery
				\end{itemize}
			\item \url{http://www.engineeringchallenges.org/}
			\end{enumerate}
		\item NAE Grand Challenge K12 Partners Program: \vspace{-0.1cm}
			\begin{enumerate} \itemsep -1pt
			\item Can be used by schools/teachers to raise awareness of global challenges among students and to encourage students to plan career paths to tackle these challenges
			\item 5-Part Make it Happen Plan (includes suggested activities for students in elementary school to learn about basic science and engineering concepts that are relevant to solve the NAE grand challenges): \url{http://www.grandchallengek12.org/5-part-plan}
			\item \url{http://www.grandchallengek12.org/about}
			\end{enumerate}
		\item {\it National Academy of Engineering}'s technological literacy program for people (students, parents, and educators) to learn more about technology: \url{http://www.nae.edu/nae/techlithome.nsf}
		\end{enumerate}
	\item Women in Technology (WIT): \vspace{-0.2cm}
		\begin{enumerate} \itemsep -2pt
		\item Girls In Technology (GIT): \vspace{-0.1cm}
			\begin{enumerate} \itemsep -1pt
			\item Get Involved: \vspace{-0.1cm}
				\begin{itemize} \itemsep -1pt
				\item \url{http://www.girlsintechnology.org/getinvolved.cfm}
				\item Teacher: teach girls about IT as an after-school activity or in a summer camp session
				\item Assistant Teacher: Assist instructors in GIT sessions, after-school activities, or summer camp sessions
				\item Develop Curriculum: Develop a curriculum for a supported GIT educational program
				\item Mentor: Mentor a girl in one of [GIT's] supported programs
				\item Job Shadow: ``Let a girl shadow you at work''
				\item Guest Speaker: ``Speak to a group of girls on a topic both you and they enjoy, such as computers, technology, education, how to take apart computers, how to build a web site, etc.''
				\end{itemize}
			\end{enumerate}
		\end{enumerate}
	\item Organization for Economic Co-operation and Development (OECD): \vspace{-0.2cm}
		\begin{enumerate} \itemsep -2pt
		\item Programme for International Student Assessment (PISA): \vspace{-0.1cm}
			\begin{enumerate} \itemsep -1pt
			\item {\it PISA 2009 Results}. Available online at: \url{http://www.oecd.org/document/61/0,3343,en_32252351_32235731_46567613_1_1_1_1,00.html}; last accessed on December 10, 2010. [ Includes suggestions to improve learning outcomes, as well as education policies and practices. ]
			\end{enumerate}
		\end{enumerate}
	\item American Institute of Aeronautics and Astronautics (AIAA): \vspace{-0.2cm}
		\begin{enumerate} \itemsep -2pt
		\item K-12 Educators: \url{http://www.aiaa.org/content.cfm?pageid=208}
		\end{enumerate}
	\item Research Councils UK (RCUK): \vspace{-0.2cm}
		\begin{enumerate} \itemsep -2pt
		\item Biotechnology and Biological Sciences Research Council (BBSRC): \vspace{-0.1cm}
			\begin{enumerate} \itemsep -1pt
			\item Resources for schools and young people: \url{http://www.bbsrc.ac.uk/society/schools/schools-index.aspx}
			\item Teaching resources: publications and web-based activities: \vspace{-0.1cm}
				\begin{itemize} \itemsep -1pt
				\item Primary (ages 5-12) resources: \url{http://www.bbsrc.ac.uk/society/schools/primary/primary-index.aspx}
				\item Secondary (ages 12-16) and post-16 resources: \url{http://www.bbsrc.ac.uk/society/schools/secondary/secondary-index.aspx}
				\end{itemize}
			\end{enumerate}
		\end{enumerate}
	\item Nuffield Foundation: \vspace{-0.2cm}
		\begin{enumerate} \itemsep -2pt
		\item Education: \url{http://www.nuffieldfoundation.org/education}
		\item Teachers: \vspace{-0.1cm}
			\begin{enumerate} \itemsep -1pt
			\item (Excellent) resources in science and mathematics: \url{http://www.nuffieldfoundation.org/teachers}
			\item \url{http://www.nuffieldfoundation.org/teachers-0}
			\end{enumerate}
		\end{enumerate}
	\item Wellcome Trust: \vspace{-0.2cm}
		\begin{enumerate} \itemsep -2pt
		\item Education resources: \url{http://www.wellcome.ac.uk/Education-resources/index.htm}
		\item {\it yourgenome.org}: \vspace{-0.1cm}
			\begin{enumerate} \itemsep -1pt
			\item \url{http://www.yourgenome.org/}
			\item Resources for teachers about genomics: \url{http://www.yourgenome.org/landing_teachers.shtml}
			\end{enumerate}
		\item Network of Science Learning Centers (Science Learning Centers): \vspace{-0.1cm}
			\begin{enumerate} \itemsep -1pt
			\item \url{https://www.sciencelearningcentres.org.uk/}
			\item Awards and Bursaries: \vspace{-0.1cm}
				\begin{itemize} \itemsep -1pt
				\item \url{https://www.sciencelearningcentres.org.uk/centres/national/awards-and-bursaries}
				\item \url{https://www.sciencelearningcentres.org.uk/about/impact-awards}
				\end{itemize}
			\item Resource collections: \url{https://www.sciencelearningcentres.org.uk/resources}
			\item Curriculum resources for primary, secondary, and tertiary education: \url{https://www.sciencelearningcentres.org.uk/curriculum}
			\end{enumerate}
		\end{enumerate}
	\end{enumerate}
%%%%%%%%%%%%%%%%%%%%%%%%%%%%%%%%%%%%%%%%
%%%%%%%%%%%%%%%%%%%%%%%%%%%%%%%%%%%%%%%%
\item Underrepresented minorities: \vspace{-0.3cm}
	\begin{enumerate} \itemsep -2pt
	\item University of Washington: \vspace{-0.2cm}
		\begin{enumerate} \itemsep -2pt
		\item Department of Computer Science and Engineering: \vspace{-0.1cm}
			\begin{enumerate} \itemsep -1pt
			\item {\it AccessComputing}: \vspace{-0.1cm}
				\begin{itemize} \itemsep -1pt
				\item \url{http://www.washington.edu/accesscomputing/}
				\item Has resources to help students with disabilities to pursue ``undergraduate and graduate degrees and careers in computing fields''.
				\item It runs the ``Summer Academy for Advancing Deaf \& Hard of Hearing in Computing'' for youths who are hearing impaired: \url{http://www.washington.edu/accesscomputing/dhh/academy/index.html}
				\end{itemize}
			\end{enumerate}
		\end{enumerate}
	%%%%%%%%%%%%%%%%%%%%%%%%%
	\item Engineer Girl: \vspace{-0.2cm}
		\begin{enumerate} \itemsep -2pt
		\item Resources for students, parents, and teachers to encourage girls to explore careers and educational opportunities in engineering
		\item Created by the National Academy of Sciences and The National Academy of Engineering
		\item Contests for K-12 students: \url{http://www.engineergirl.org/?id=4436}
		\item \url{http://www.engineergirl.org/}
		\end{enumerate}
	\item Engineering Your Life: \url{http://www.engineeryourlife.org/}
	\item GirlGeeks: \url{http://www.girlgeeks.org/}
	\item {\it Women in Science, Technology, Engineering, and Mathematics ON THE AIR!}: \vspace{-0.2cm}
		\begin{enumerate} \itemsep -2pt
		\item Audio resources that describe stories about women in science, technology, engineering, and mathematics (STEM) fields
		\item \url{http://www.womeninscience.org/}
		\end{enumerate}
	\item {\it Women Scientists in History}: \url{http://www.hypatiamaze.org/}
	\item Association for Women in Mathematics (AWM): \vspace{-0.2cm}
		\begin{enumerate} \itemsep -2pt
		\item \url{http://www.awm-math.org/}
		\item Education: \vspace{-0.1cm}
			\begin{enumerate} \itemsep -1pt
			\item \url{http://sites.google.com/site/awmmath/awm-resources/education}
			\item Includes information for students in middle school, high school, college and university (including graduate students). It also includes information for parents and teachers/educators.
			\end{enumerate}
		\item Women in Math, Science, and Society: \url{http://sites.google.com/site/awmmath/women-in-math-science-and-society}
		\item Essay contest on biographies of contemporary women in mathematics: \url{http://sites.google.com/site/awmmath/programs/essay-contest}
		\end{enumerate}
	\item Women in Technology (WIT): \vspace{-0.2cm}
		\begin{enumerate} \itemsep -2pt
		\item Girls in Technology: \vspace{-0.1cm}
			\begin{enumerate} \itemsep -1pt
			\item \url{http://www.girlsintechnology.org/}
			\item WIT Education Foundation: provides educational programs for girls in technology
			\item TeamBusiness Fundraiser: ``A combined fundraiser and program for girls in Grades 9-12 across the Metro DC area. Each year, up to forty girls participate with mentors and WIT volunteers in a full-day business simulation workshop conducted by TeamBusiness USA. The teams competed as companies, learning how to run a technology company in a fun and exciting simulation environment.''
			\item Hispanic Youth Foundation: ``In 2005, GIT established a partnership with the Hispanic Youth Foundation (HYF) and provided a grant to fund HYF�s innovative Laptops for Learning Dollars program, providing laptops and Internet connections for elementary and middle school students and their families in Arlington County and the City of Manassas.''
			\item Empower Girls -- CLCP Clubs: ``Empower Girls after-school programs were held at Hybla Valley Elementary School and Sacramento Community Center. GIT/WITEF provided funding to run these programs in conjunction with the Fairfax County Computer Learning Center Partnership (CLCP). The selected centers serve economically challenged communities in Fairfax County.''
			\end{enumerate}
		\end{enumerate}
	%%%%%%%%%%%%%%%%%%%%%%%%%
	\item National Society of Black Engineers (NSBE) competitions for high school/K-12 students: \url{http://www.nsbe.org/Programs/Competitions/NSBE-Jr-.aspx}
	\item The Society of Mexican American Engineers and Scientists (MAES): MAES PreCollege Outreach Programs, \url{http://www.maes-natl.org/index.php?module=ContentExpress&func=display&ceid=16&meid=236}
	\item {\it Center for the Advancement of Hispanics in Science and Engineering Education} (CAHSEE): \vspace{-0.2cm}
		\begin{enumerate} \itemsep -2pt
		\item STEM - The Science, Technology, Engineering \& Mathematics Institute (for students from grades 5 through 11): \url{http://www.cahsee.org/2programs/stem.asp.htm}
		\item YEP - Young Educators Program (fellows would learn how to train students in the aforementioned STEM Institute): \url{http://www.cahsee.org/2programs/yep.asp.htm}
		\item CAYSA - Central American Young Scholar Awards: \url{http://www.cahsee.org/2programs/caysa.asp.htm}. ``The CAYSA ceremonies honor more than 60 Washington, D.C. area high school seniors of Central American descent who have demonstrated remarkable success throughout all four years of high school. Students must be of Central American descent and have at least a 3.0 gpa.''
		\item Scholarships: \url{http://www.cahsee.org/6resources/scholarships.asp.htm}
		\item \url{http://www.cahsee.org/about/about.asp.htm}
		\end{enumerate}
	%%%%%%%%%%%%%%%%%%%%%%%%%
	\item International Computer Science Institute (UC Berkeley): \vspace{-0.2cm}
		\begin{enumerate} \itemsep -2pt
		\item Berkeley Foundation for Opportunities in Information Technology, BFOIT: \vspace{-0.1cm}
			\begin{enumerate} \itemsep -1pt
			\item BFOIT Programs for women and underrepresented minorities (African Americans and Chicanos/Latinos) in middle/high school who are interested in electrical/computer engineering and computer science careers: \url{http://www.bfoit.org/programs.html}
			\end{enumerate}
		\end{enumerate}
	\item Institute for Broadening Participation: \vspace{-0.2cm}
		\begin{enumerate} \itemsep -2pt
		\item PathwaysToScience.org: \vspace{-0.1cm}
			\begin{enumerate} \itemsep -1pt
			\item PathwaysToScience.org is a portal website supporting pathways to the STEM fields: science, technology, engineering, and mathematics.
			\item Particular emphasis is placed on connecting traditionally underrepresented groups with STEM programs and resources, including funding and mentoring opportunities. 
			\item For K-12 students: \url{http://www.pathwaystoscience.org/K12.asp}
			\item STEM Resources by Institution (colleges, universities, and US national research laboratories): \url{http://www.pathwaystoscience.org/Institution.asp}
			\item profiles of people and programs in STEM: \vspace{-0.3cm}
				\begin{itemize} \itemsep -2pt
				\item \url{http://www.pathwaystoscience.org/Profiles.asp}
				\item Find out about the career paths of underrepresented minorities in STEM
				\item Find out about programs that are offered by institutions for underrepresented minorities in STEM
				\end{itemize}
			\item Directory of partners (organizations that cooperate with or support the Institute for Broadening Participation): \url{http://www.pathwaystoscience.org/Partners.asp}
			\item Additional resources: \url{http://www.pathwaystoscience.org/Ideaexchange.asp}
			\end{enumerate}
		\item Maine Pathways to STEM (Science, Technology, Engineering \& Mathematics): \vspace{-0.1cm}
			\begin{enumerate} \itemsep -1pt
			\item \url{http://www.mainestem.org/}
			\item K-12 Teachers \& University Faculty: \url{http://www.mainestem.org/METeachersFaculty.asp}
			\item K-12 STEM Resources: \url{http://www.mainestem.org/MEK12.asp}
			\end{enumerate}
		\end{enumerate}
	\item Building Engineering and Science Talent, BEST: \vspace{-0.2cm}
		\begin{enumerate} \itemsep -2pt
		\item \url{http://www.bestworkforce.org/}
		\item Publications: \url{http://www.bestworkforce.org/publications.htm}
		\item List of programs to help underrepresented minority students in K-12 schools explore careers in STEM: \url{http://www.bestworkforce.org/links.htm}
		\end{enumerate}
	\item American Indian Science and Engineering Society (AISES): \vspace{-0.2cm}
		\begin{enumerate} \itemsep -2pt
		\item Pre-college programs: \vspace{-0.1cm}
			\begin{enumerate} \itemsep -1pt
			\item \url{http://www.aises.org/Programs}
			\item Resources: \url{http://www.aises.org/Programs/Resources}
			\end{enumerate}
		\end{enumerate}
	\end{enumerate}
\end{enumerate}







%%%%%%%%%%%%%%%%%%%%%%%%%%%%%%%%%%%%%%%%%%%
\subsection{Science \& Engineering Outreach for Undergraduates, Grad Students, \& Postdocs}
\label{stemoutreachcollegegradsch}


Science, mathematics, and engineering outreach to undergraduates, graduate students, and postdocs: \vspace{-0.3cm}
\begin{enumerate} \itemsep -4pt
\item Mac Hyman, ``Good Choices for Great Careers in the Mathematical Sciences,'' talk given at 2008 SIAM Annual Meeting. Available at: \url{http://client.blueskybroadcast.com/siam08/hyman/index.html}; last accessed on August 25, 2010. Also, see \url{http://meetings.siam.org/program.cfm?CONFCODE=AN08}, \url{http://www.siam.org/meetings/an08/program.php}, and \url{http://www.siam.org/meetings/an08/}.
\item {\it Accreditation.org}: \vspace{-0.3cm}
	\begin{enumerate} \itemsep -2pt
	\item Information about the accreditation of engineering degree programs around the world
	\item \url{http://www.accreditation.org/}
	\end{enumerate}
\item John Baez, ``How to Learn Math and Physics,'' Department of Mathematics, University of California, Riverside, December 24, 2007. Available at: \url{http://math.ucr.edu/home/baez/books.html}; last accessed on August 28, 2010.
\item {\it MentorNet}: \vspace{-0.3cm}
	\begin{enumerate} \itemsep -2pt
	\item \url{http://www.mentornet.net/}
	\item Enables people to network with scientists, engineers, and professors in Science, Technology, Engineering, and Mathematics (STEM)
	\item Is very supportive of minorities, so that more minorities (particularly underrepresented minorities) can be attracted to STEM careers
	\end{enumerate}
\item {\it The Indus Entrepreneurs (TiE)} for networking among high-tech entrepreneurs, start-up co-founders, venture capitalists, and angel investors: \url{http://www.tie.org/}
\item National Academy of Engineering, NAE: \vspace{-0.3cm}
	\begin{enumerate} \itemsep -2pt
	\item Includes a list of NAE Grand Challenges, which can provide some suggestions for research trajectories
	\item Summit Series on the Grand Challenges: Includes the National Grand Challenges Summits
	\item \url{http://www.engineeringchallenges.org/}
	\end{enumerate}
\item {\it National Society of Professional Engineers}: \vspace{-0.3cm}
	\begin{enumerate} \itemsep -2pt
	\item Student Resources: \vspace{-0.2cm}
		\begin{enumerate} \itemsep -2pt
		\item \url{http://www.nspe.org/Students/Resources/index.html}
		\item An Employment Guidelines Checklist for the Engineer Job Applicant: \url{http://www.nspe.org/Students/Resources/checklist.html}
		\end{enumerate}
	\item Career Center: \url{http://www.nspe.org/CareerCenter/index.html}
	\item A Sightseer's Guide to Engineering: \url{http://www.engineeringsights.org/}
	\end{enumerate}
\item {\it JustGarciaHill} ``Study Skills for Budding Scientists'': \url{http://www.justgarciahill.org/index.php/science-study-skills.html}
\item {\it NASA} resources for students: \vspace{-0.3cm}
	\begin{enumerate} \itemsep -2pt
	\item \url{http://www.nasa.gov/audience/forstudents/index.html}
	\item NASA University Student Launch Initiative, or USLI: \url{http://www.nasa.gov/offices/education/programs/descriptions/University_Student_Launch_Initiative.html}
	\end{enumerate}
\item {\it iTunes U}: \vspace{-0.3cm}
	\begin{enumerate} \itemsep -2pt
	\item {\it iTunes} is required to listen to or watch these lectures, talks, and presentations.
	\item Access {\it iTunes U} at: \url{http://www.apple.com/education/itunes-u/} or \url{http://deimos3.apple.com/indigo/main/main.html?v0=WWW-AMUS-ITUNESU070521-N48LX}
	\item {\it iTunes U} is a set of webcast and podcasts, where you can easily find audio and video clips for lectures, seminars, announcements, virtual tours, and so on. For example, some professors from schools like MIT or Berkeley will provide audio/video clips of their lectures on {\it iTunes U}.
	\item This can help in exploring different majors before a college student declares her/his major(s). If a student is not sure if she/he wants to double major in deaf studies and linguistics, this student can check out some linguistics lectures from her/his (preferred) college/university, if it uses {\it iTunes U}, or those from other universities.
	\end{enumerate}
\item Harvey Mudd College: \vspace{-0.3cm}
	\begin{enumerate} \itemsep -2pt
	\item Francis Edward Su, {\it Math Fun Facts!}, Department of Mathematics, Harvey Mudd College: \url{http://www.math.hmc.edu/funfacts/}
	\end{enumerate}
\item Engineering Pathway: \url{http://www.engineeringpathway.com/ep/index.jhtml}
\item Rochester Institute of Technology, ``Biology \& Biotechnology Paid Co-op/Internships for 2011,'' Department of Biological Sciences, Rochester Institute of Technology: \url{http://people.rit.edu/gtfsbi/Symp/summer.htm}
\item {\it Mathematical Association of America (MAA)} information on educational pathways and career opportunities: \vspace{-0.3cm}
	\begin{enumerate} \itemsep -2pt
	\item Undergraduate Students: \url{http://www.maa.org/students/undergrad/}
	\item Graduate Students: \url{http://www.maa.org/students/grad/}
	\item Underrepresented Groups: \url{http://www.maa.org/programs/underrep.html}
	\item Mathematical Association of America (MAA) MathFest (for students in mathematics): \url{http://www.maa.org/mathfest/}
	\item MAA Online Columns: \url{http://www.maa.org/news/columns.html}
	\end{enumerate}
\item New Zealand Institute of Mathematics and its Applications (NZIMA): \vspace{-0.3cm}
	\begin{enumerate} \itemsep -2pt
	\item {\it MathsReach}: Careers (information about careers based on a higher education in mathematics or related field): \url{http://www.mathsreach.org/Careers}
	\end{enumerate}
\item {\it Engineers Dedicated to a Better Tomorrow (a.k.a., DedicatedEngineers)}: \vspace{-0.3cm}
	\begin{enumerate} \itemsep -2pt
	\item [Resources for] College Students and Faculty/Staff Members: \url{http://www.dedicatedengineers.org/intro_for_college.htm}
	\item \url{http://www.dedicatedengineers.org/}
	\end{enumerate}
\item American Institute of Physics: \vspace{-0.3cm}
	\begin{enumerate} \itemsep -2pt
	\item GradschoolShopper.com: \vspace{-0.2cm}
		\begin{enumerate} \itemsep -2pt
		\item \url{http://www.gradschoolshopper.com/}
		\item ``Find information on graduate programs in physics, astronomy, and other physical sciences''
		\end{enumerate}
	\item Career guidance for high school and undergraduate students: \url{http://www.aip.org/statistics/trends/career.html}
	\item American Geophysical Union: \vspace{-0.2cm}
		\begin{enumerate} \itemsep -2pt
		\item Diversity Programs: \url{http://www.agu.org/education/diversity_programs/}
		\end{enumerate}
	\end{enumerate}
\item {\it icademic.org} resources for the life sciences and engineering: \url{http://www.icademic.org/}
\item The Oceanography Society: \vspace{-0.3cm}
	\begin{enumerate} \itemsep -2pt
	\item Hands-On Oceanography: peer-reviewed activities appropriate for undergraduate and/or graduate classes in oceanography, \url{http://www.tos.org/hands-on/index.html}
	\end{enumerate}
%%%%%%%%%%%%%%%%%%%%%%%%%%%%%%%%%%%%%%%
%%%%%%%%%%%%%%%%%%%%%%%%%%%%%%%%%%%%%%%
\item outreach activities (including mentoring) to students in K-12: \vspace{-0.3cm}
	\begin{enumerate} \itemsep -2pt
	\item Research Councils UK (RCUK): \vspace{-0.2cm}
		\begin{enumerate} \itemsep -2pt
		\item Researchers in Residence (RinR): \vspace{-0.1cm}
			\begin{enumerate} \itemsep -1pt
			\item \url{http://www.researchersinresidence.ac.uk/cms/}
			\item \url{http://www.researchersinresidence.ac.uk/cms/researchers/}
			\item Mentor middle and high school students who are job shadowing (observing you first-hand) in your research activities for up to a week, so that they can learn what doing research in your research area is like. You should explain in laypeople's terms what your research is about. That is, be a mentor for the externships of middle and high school students.
			\end{enumerate}
		\end{enumerate}
	\end{enumerate}
%%%%%%%%%%%%%%%%%%%%%%%%%%%%%%%%%%%%%%%
%%%%%%%%%%%%%%%%%%%%%%%%%%%%%%%%%%%%%%%
\item competitions: \vspace{-0.3cm}
	\begin{enumerate} \itemsep -2pt
	\item Invent Now, Inc.: \vspace{-0.2cm}
		\begin{enumerate} \itemsep -2pt
		\item Collegiate Inventors Competition: \url{http://www.invent.org/collegiate/} [ Resources for {\color{blue} Patent Search Strategy} are available. \colorbox{blue}{\bf This is the ultimate competition for US students in science and engineering.} ]
		\end{enumerate}
	\item INFORMS Doing Good with Good OR - Student Competition: \vspace{-0.2cm}
		\begin{enumerate} \itemsep -2pt
		\item Doing Good with Good OR-Student Competition is held each year to identify and honor outstanding projects in the field of operations research and the management sciences conducted by a student or student group that have a significant societal impact.
		\item \url{http://www.informs.org/Recognize-Excellence/INFORMS-Prizes-Awards/Doing-Good-with-Good-OR}
		\end{enumerate}
	\item AWM Essay Contest: Biographies of Contemporary Women in Mathematics; see \url{http://www.awm-math.org/biographies/contest.html}
	\item American Society of Mechanical Engineers (ASME): \vspace{-0.2cm}
		\begin{enumerate} \itemsep -2pt
		\item Student Design Competition: \url{http://www.asme.org/Events/Contests/DesignContest/Student_Design_Competition.cfm}
		\item ASME Student Mechanism and Robot Design Competition: \url{http://www.asme.org/Events/Contests/Student_Mechanism_Robot_2.cfm}
		\end{enumerate}
	\item American Institute of Chemical Engineers (AIChE) competitions: \url{http://www.aiche.org/Students/Awards/index.aspx}
	\item Association for Unmanned Vehicle Systems International (AUVSI): \vspace{-0.2cm}
		\begin{enumerate} \itemsep -2pt
		\item AUVSI Student Competitions: \vspace{-0.1cm}
			\begin{enumerate} \itemsep -1pt
			\item \url{http://www.auvsi.org/AUVSI/AUVSI/Home/Default.aspx}, or \url{http://www.auvsi.org/}
			\item Annual Intelligent Ground Vehicle Competition (IGVC): \url{http://www.igvc.org/}
			\item Annual Student Unmanned Air System (SUAS) Competition: \url{http://65.210.16.57/studentcomp2010/default.html}
			\item International Aerial Robotics Competition (IARC): \url{http://iarc.angel-strike.com/}
			\item AUVSI and ONR's International Autonomous Surface Vehicle (ASV) Competition [ASVC]
			\item AUVSI Foundation and ONR's (U.S. Office of Naval Research) 4th International RoboBoats Competition: \url{http://www.auvsifoundation.org/AUVSI/FOUNDATION/Competitions/ASVCompetition/Default.aspx?C=00000000-0000-0000-0000-000000000000}
			\item AUVSI Foundation and ONR's (U.S. Office of Naval Research) International RoboSub Competition (or AUVSI and ONR's International Autonomous Underwater Vehicle Competition): \url{http://www.auvsifoundation.org/AUVSI/FOUNDATION/Competitions/AUVCompetition/Default.aspx}
			\item ONR: U.S. Office of Naval Research
			\end{enumerate}
		\end{enumerate}
	\item American Institute of Aeronautics and Astronautics (AIAA): \vspace{-0.2cm}
		\begin{enumerate} \itemsep -2pt
		\item Design Competitions: \url{http://www.aiaa.org/content.cfm?pageid=210}
		\end{enumerate}
	\item National Aeronautics and Space Administration: \vspace{-0.2cm}
		\begin{enumerate} \itemsep -2pt
		\item NASA's Langley Research Center: \vspace{-0.1cm}
			\begin{enumerate} \itemsep -1pt
			\item SpaceTech Engineering Design Challenge: \url{http://spacetech.larc.nasa.gov}
			\end{enumerate}
		\end{enumerate}
	\item American Concrete Institute (ACI): \vspace{-0.2cm}
		\begin{enumerate} \itemsep -2pt
		\item Competitions: \url{http://www.concrete.org/STUDENTS/st_competitions.htm}
		\end{enumerate}
	\end{enumerate}
%%%%%%%%%%%%%%%%%%%%%%%%%%%%%%%%%%%%%%%
%%%%%%%%%%%%%%%%%%%%%%%%%%%%%%%%%%%%%%%
\item underrepresented minorities: \vspace{-0.3cm}
	\begin{enumerate} \itemsep -2pt
	\item The Society of Women Engineers: \url{http://societyofwomenengineers.swe.org/}
	\item Association for Women in Science (AWIS): \url{http://www.awis.org/} and \url{http://www.awis.affiniscape.com/displaycommon.cfm?an=1&subarticlenbr=19}
	\item Association for Women in Mathematics (AWM): \vspace{-0.2cm}
		\begin{enumerate} \itemsep -2pt
		\item \url{http://www.awm-math.org/}
		\item Education: \vspace{-0.1cm}
			\begin{enumerate} \itemsep -1pt
			\item \url{http://sites.google.com/site/awmmath/awm-resources/education}
			\item Includes information for students in middle school, high school, college and university (including graduate students). It also includes information for parents and teachers/educators.
			\end{enumerate}
		\item Career advice and opportunities: \url{http://sites.google.com/site/awmmath/awm-resources/career}
		\item Women in Math, Science, and Society: \url{http://sites.google.com/site/awmmath/women-in-math-science-and-society}
		\item Essay contest on biographies of contemporary women in mathematics: \url{http://sites.google.com/site/awmmath/programs/essay-contest}
		\end{enumerate}
	\item Sigma Delta Epsilon-Graduate Women in Science (GWIS): \url{http://www.gwis.org/}
	\item Society of Hispanic Professional Engineers (SHPE): \vspace{-0.2cm}
		\begin{enumerate} \itemsep -2pt
		\item Advancing Hispanic Excellence in Technology, Engineering, Math and Science (AHETEMS) Foundation: \url{http://www.ahetems.org/}
		\item AHETEMS Scholarship Program: \url{http://www.ahetems.org/scholarships/}
		\item Graduate \& Young Professional Fellowship Program (encourage young professionals to engage in {\bf public policy}): \url{http://www.ahetems.org/graduate/graduate-young-professional-fellowship-program/}
		\item SHPE/GEM Fellowship (for graduate students in STEM at GEM Member Universities): \url{http://www.ahetems.org/graduate/shpe-gem-graduate-award/}. See \url{http://www.gemfellowship.org/gem-universities/university-members} for a list of GEM member universities.
		\item Internship opportunities: \url{http://www.ahetems.org/scholar-internships/}
		\item \url{http://oneshpe.shpe.org/wps/portal/national}
		\end{enumerate}
	\item National Society of Black Engineers (NSBE): \vspace{-0.2cm}
		\begin{enumerate} \itemsep -2pt
		\item Scholarships: \url{http://www.nsbe.org/Programs/Scholarships.aspx}
		\item Competitions for undergraduates and graduate students: \url{http://www.nsbe.org/Programs/Competitions/Collegiate.aspx}
		\item \url{http://www.nsbe.org/}
		\end{enumerate}
	\item The Society of Mexican American Engineers and Scientists (MAES): \vspace{-0.2cm}
		\begin{enumerate} \itemsep -2pt
		\item MAES Undergraduate and Graduate Outreach Programs (including ``GRE/Graduate Application Fee Waivers''): \url{http://www.maes-natl.org/index.php?module=ContentExpress&func=display&ceid=90&meid=238}
		\item Scholarships \& Awards: \url{http://www.maes-natl.org/index.php?meid=328}
		\item MAES Scholarship Program: \url{http://www.maes-natl.org/index.php?module=ContentExpress&func=display&ceid=518&meid=241}
		\end{enumerate}
	\item SACNAS (Society for Advancement of Chicanos and Native Americans in Science): \vspace{-0.2cm}
		\begin{enumerate} \itemsep -2pt
		\item Scholarships: \url{http://www.sacnas.org/webadindex.cfm?webadcategory_id=7}
		\item Fellowships: \url{http://www.sacnas.org/webadIndex.cfm?webadcategory_id=5}
		\end{enumerate}
	\item {\it Center for the Advancement of Hispanics in Science and Engineering Education} (CAHSEE): \vspace{-0.2cm}
		\begin{enumerate} \itemsep -2pt
		\item YESP - Young Engineers \& Scientists Program: \url{http://www.cahsee.org/2programs/yesp.asp.htm}. ``This program places talented Hispanic college students in the research labs of government agencies.''
		\item Scholarships: \url{http://www.cahsee.org/6resources/scholarships.asp.htm}
		\end{enumerate}
	\item American Geophysical Union: \vspace{-0.2cm}
		\begin{enumerate} \itemsep -2pt
		\item Has a list of organizations for specific underrepresented ethnic-minority groups in the geosciences and physics: \vspace{-0.1cm}
			\begin{enumerate} \itemsep -1pt
			\item \url{http://www.agu.org/education/diversity_programs/}
			\item These organizations may have information about scholarships, fellowships, and educational material for K-12 and college students.
			\end{enumerate}
		\end{enumerate}
	\item Institute for Broadening Participation: \vspace{-0.2cm}
		\begin{enumerate} \itemsep -2pt
		\item Minorities Striving and Pursuing Higher Degrees of Success in Earth System Science (MS PHD'S\textregistered) initiative: \vspace{-0.1cm}
			\begin{enumerate} \itemsep -1pt
			\item \url{http://www.msphds.org/}
			\item Prospective Students/Mentees: \url{http://www.msphds.org/prospective.asp}
			\item For MS PHD'S Students: \url{http://www.msphds.org/students.asp}
			\end{enumerate}
		\item PathwaysToScience.org: \vspace{-0.1cm}
			\begin{enumerate} \itemsep -1pt
			\item Resources for undergraduate students: \url{http://www.pathwaystoscience.org/Undergrads.asp}
			\item Resources for graduate students: \url{http://www.pathwaystoscience.org/Grad.asp}
			\item Resources for postdocs: \url{http://www.pathwaystoscience.org/Postdocs_portal.asp}
			\item STEM Resources by Institution (colleges, universities, and US national research laboratories): \url{http://www.pathwaystoscience.org/Institution.asp}
			\item Additional resources: \url{http://www.pathwaystoscience.org/Ideaexchange.asp}
			\end{enumerate}
		\item National Alliance for Doctoral Studies in the Mathematical Sciences: \vspace{-0.1cm}
			\begin{enumerate} \itemsep -1pt
			\item \url{http://www.mathalliance.org/}
			\item Student/Alliance Scholars: \url{http://www.mathalliance.org/scholars.asp}
			\item Alliance Mentors / Alliance Undergraduate Mentors: \url{http://www.mathalliance.org/mentors.asp}
			\item Alliance Programs: \url{http://www.mathalliance.org/programs.asp}
			\end{enumerate}
		\item Alliances for Graduate Education and the Professoriate (AGEP): \vspace{-0.1cm}
			\begin{enumerate} \itemsep -1pt
			\item \url{http://www.agep.us/}
			\end{enumerate}
		\item Maine Pathways to STEM (Science, Technology, Engineering \& Mathematics): \vspace{-0.1cm}
			\begin{enumerate} \itemsep -1pt
			\item \url{http://www.mainestem.org/}
			\item K-12 Teachers \& University Faculty: \url{http://www.mainestem.org/METeachersFaculty.asp}
			\item Graduate \& Undergraduate Students: \url{http://www.mainestem.org/MEUndergradGrad.asp}
			\end{enumerate}
		\end{enumerate}
	\item ARTSI (Advancing Robotics Technology for Societal Impact) Alliance: \vspace{-0.2cm}
		\begin{enumerate} \itemsep -2pt
		\item \url{http://artsialliance.org/}
		\item ``The ARTSI (Advancing Robotics Technology for Societal Impact) Alliance is a collaborative education and research project centered around robotics for healthcare, the arts, and entrepreneurship.  Spelman College, a historically black college (HBCU) for women is leading the alliance in partnership with several other HBCUs and Research I (R1) institutions.''
		\item Summer REU (Research Experience for Undergraduates) program: \url{http://artsialliance.org/Summer-REU-Program}
		\end{enumerate}
	\item Women in Technology (WIT): \vspace{-0.2cm}
		\begin{enumerate} \itemsep -2pt
		\item \url{http://www.womenintechnology.org/index.asp}
		\item WIT Mentor-Prot{\'{e}}g{\'{e}} Program: \url{http://www.womenintechnology.org/content.asp?contentid=59}
		\item {\bf \color{blue} WIT Career Transition Resource Guide}: \url{http://www.womenintechnology.org/content.asp?contentid=146}
		\item Girls In Technology (GIT): \vspace{-0.1cm}
			\begin{enumerate} \itemsep -1pt
			\item Get Involved: \vspace{-0.1cm}
				\begin{itemize} \itemsep -1pt
				\item \url{http://www.girlsintechnology.org/getinvolved.cfm}
				\item Teacher: teach girls about IT as an after-school activity or in a summer camp session
				\item Assistant Teacher: Assist instructors in GIT sessions, after-school activities, or summer camp sessions
				\item Develop Curriculum: Develop a curriculum for a supported GIT educational program
				\item Mentor: Mentor a girl in one of [GIT's] supported programs
				\item Job Shadow: ``Let a girl shadow you at work''
				\item Guest Speaker: ``Speak to a group of girls on a topic both you and they enjoy, such as computers, technology, education, how to take apart computers, how to build a web site, etc.''
				\end{itemize}
			\end{enumerate}
		\end{enumerate}
	\item Arizona State University: \vspace{-0.2cm}
		\begin{enumerate} \itemsep -2pt
		\item {\it Career}WISE: \vspace{-0.1cm}
			\begin{enumerate} \itemsep -1pt
			\item \url{http://careerwise.asu.edu/}
			\item Helpful resources for female graduate/Ph.D. students in science and engineering.
			\end{enumerate}
		\end{enumerate}
	\item American Indian Science and Engineering Society (AISES): \vspace{-0.2cm}
		\begin{enumerate} \itemsep -2pt
		\item Programs for undergraduates and grad students (including scholarships and internships): \vspace{-0.1cm}
			\begin{enumerate} \itemsep -1pt
			\item \url{http://www.aises.org/Programs}
			\item Resources: \url{http://www.aises.org/Programs/Resources}
			\end{enumerate}
		\end{enumerate}
	\end{enumerate}
\end{enumerate}




%%%%%%%%%%%%%%%%%%%%%%%%%%%%%%%%%%%%%%%%%%%
\subsection{Other Science and Engineering Outreach}
\label{otherstemoutreach}

Other Science and Engineering Outreach: \vspace{-0.3cm}
\begin{enumerate} \itemsep -4pt
\item Frontiers of Engineering (networking event for mid-career engineers): \url{http://www.naefrontiers.org/}
\item Consortium for Ocean Leadership: \vspace{-0.3cm}
	\begin{enumerate} \itemsep -2pt
	\item Resources for scientists in the marine sciences to use in outreach activities: \url{http://www.oceanleadership.org/education/deep-earth-academy/scientists/}
	\end{enumerate}
\item The Oceanography Society: \vspace{-0.3cm}
	\begin{enumerate} \itemsep -2pt
	\item Education and Public Outreach (EPO): A Guide for Scientists [material that scientists and professors can use for outreach activities], \url{http://www.tos.org/epo_guide/index.html}
	\end{enumerate}
\item The Joy McCann Foundation: \vspace{-0.3cm}
	\begin{enumerate} \itemsep -2pt
	\item McCann Scholar (for professors in medicine, science, and nursing): \url{http://www.mccannfoundation.org/scholars.htm}
	\item The Joy McCann Professorship for Women in Medicine: \url{http://www.mccannfoundation.org/medicine.htm}
	\end{enumerate}
\item U.S. National Academies: \vspace{-0.3cm}
	\begin{enumerate} \itemsep -2pt
	\item International Activities of the U.S. National Academies -- Science, Engineering \& Medicine: Working toward a better world: \vspace{-0.2cm}
		\begin{enumerate} \itemsep -2pt
		\item \url{http://sites.nationalacademies.org/International/}
		\item Solving the grand challenges: \vspace{-0.1cm}
			\begin{enumerate} \itemsep -1pt
			\item Energy and the Environment
			\item Global Health
			\item Water Resources
			\item Agriculture and Food Security
			\item International Security
			\item Population
			\end{enumerate}
		\item Help other countries build/improve their capacities: \vspace{-0.1cm}
			\begin{enumerate} \itemsep -1pt
			\item Cooperative Program with Pakistan 
			\item African Science Academies 
			\item Visiting Math Lecturer Program in Cambodia 
			\item Humanitarian Relief Efforts
			\item Improved Road Safety
			\item Science-based Decision Making for Sustainability
			\item Science Academies' Input to G8 Summits
			\end{enumerate}
		\item Scientific Cooperation: \vspace{-0.1cm}
			\begin{enumerate} \itemsep -1pt
			\item Building Bridges in the Middle East
			\item Cooperation with Iran
			\item Human Rights
			\item Frontiers of Science and Engineering Symposia
			\item Travel Grants
			\item International Conference on Women's Issues in Transportation
			\end{enumerate}
		\item Advising the U.S. Government: \vspace{-0.1cm}
			\begin{enumerate} \itemsep -1pt
			\item Science \& Technology in Foreign Policy
			\item Health 
			\item Science and Security
			\end{enumerate}
		\end{enumerate}
	\end{enumerate}
\item National Academy of Engineering: \vspace{-0.3cm}
	\begin{enumerate} \itemsep -2pt
	\item The Charles Stark Draper Prize (``to recognize innovative engineering achievements and their reduction to practice in ways that have led to important benefits and significant improvement in the well being and freedom of humanity''): \url{http://www.draperprize.org/}
	\item NAE Grand Challenge Scholars Program: \url{http://www.grandchallengescholars.org/}
	\end{enumerate}
\item United States Department of Defense (DoD): \vspace{-0.3cm}
	\begin{enumerate} \itemsep -2pt
	\item National Defense Education Program; Defense Advanced Research Projects Agency (DARPA): \vspace{-0.2cm}
		\begin{enumerate} \itemsep -2pt
		\item Resource for scientists and engineers to mentor youths, so that they would look into pursuing careers in science and engineering: \url{http://www.ndep.us/GetInvoSci.aspx}
		\item STEM Learning Modules (SLM): \vspace{-0.1cm}
			\begin{enumerate} \itemsep -1pt
			\item \url{http://www.ndep.us/ProgSLM.aspx}
			\item Help educators develop programs in science and engineering in K-12 institutions, so that youths would be encouraged to explore careers in science and engineering
			\end{enumerate}
		\end{enumerate}
	\end{enumerate}
\item Hewlett-Packard Development Company: \vspace{-0.3cm}
	\begin{enumerate} \itemsep -2pt
	\item HP Catalyst Initiative (grants for STEM education in colleges and universities): \url{http://www.hp.com/hpinfo/socialinnovation/catalyst.html}
	\item HP EdTech Innovators Award (for higher educational institutions that integrate IT into the curricular): \url{http://www.hp.com/hpinfo/socialinnovation/edtech.html}
	\end{enumerate}
\item The William and Flora Hewlett Foundation (Hewlett Foundation): \vspace{-0.3cm}
	\begin{enumerate} \itemsep -2pt
	\item Funding Programs: \url{http://www.hewlett.org/programs}
	\item Grantseekers: \url{http://www.hewlett.org/grants/grantseekers}
	\end{enumerate}
\item The Sloan Consortium (Sloan-C): \vspace{-0.3cm}
	\begin{enumerate} \itemsep -2pt
	\item Sloan-C Awards (for recognizing outstanding work in the field of online education) and Sloan-C Fellows: \url{http://sloanconsortium.org/aboutus/awards}
	\item Mayadas Leadership Award in Online Education: \url{http://sloanconsortium.org/mayadas_award}
	\end{enumerate}
\item W.K. Kellogg Foundation: \vspace{-0.3cm}
	\begin{enumerate} \itemsep -2pt
	\item Grant database: \url{http://www.wkkf.org/grants/grants-database.aspx}
	\end{enumerate}
\item Hewlett-Packard Company: \vspace{-0.3cm}
	\begin{enumerate} \itemsep -2pt
	\item HP community investment for education, economic development, and the environment: \url{http://www.hp.com/hpinfo/socialinnovation/focus.html}
	\item Entrepreneurship education: \vspace{-0.2cm}
		\begin{enumerate} \itemsep -2pt
		\item \url{http://www.hp.com/hpinfo/globalcitizenship/society/social/entrepreneurship.html}
		\item HP Graduate Entrepreneurship Training through IT (GET-IT)
		\item HP Entrepreneurship Learning Program (HELP)
		\end{enumerate}
	\item HP Innovations in Education grants: \url{http://www.hp.com/hpinfo/globalcitizenship/society/social/innovations.html}
	\end{enumerate}
\item General Electric Company: \vspace{-0.3cm}
	\begin{enumerate} \itemsep -2pt
	\item GE Foundation: \vspace{-0.2cm}
		\begin{enumerate} \itemsep -2pt
		\item Developing Futures\texttrademark\ in Education program (which encompasses the GE College Bound Program): \url{http://www.ge.com/foundation/developing_futures_in_education/index.jsp}
		\item Environment, health and safety, and health industry training programs (outside the US): \url{http://www.ge.com/foundation/international_programs/training.jsp}
		\item Student, education and scholarship initiatives: \url{http://www.ge.com/foundation/international_programs/education_initiatives.jsp}
		\end{enumerate}
	\end{enumerate}
\item The GRAMMY Foundation: \vspace{-0.3cm}
	\begin{enumerate} \itemsep -2pt
	\item GRAMMY Foundation Grants: \vspace{-0.2cm}
		\begin{enumerate} \itemsep -2pt
		\item \url{http://www2.grammy.com/GRAMMY_Foundation/Grants/}
		\item It funds {\bf Scientific Research Projects} as well as {\it Archiving And Preservation Projects}.
		\item Concerning scientific research projects: ``The GRAMMY Foundation Grant Program awards grants to organizations and individuals to support research on the impact of music on the human condition. Examples might include the study of the effects of music on mood, cognition and healing, as well as the medical and occupational well-being of music professionals and the creative process underlying music.'' [ E.g., look at music therapy as a possible research topic/area. ]
		\end{enumerate}
	\end{enumerate}
\item The Dana Foundation: \vspace{-0.3cm}
	\begin{enumerate} \itemsep -2pt
	\item \url{http://www.dana.org/grants/}
	\item Has grants for: \vspace{-0.2cm}
		\begin{enumerate} \itemsep -2pt
		\item Brain and Immuno-Imaging
		\item Clinical Neuroscience
		\item Human Immunology
		\item Neuroimmunology of Brain Infections and Cancers
		\end{enumerate}
		\item Deadlines and Requests for Proposals (RFP): \url{http://www.dana.org/grants/deadlines.aspx}
	\end{enumerate}
%%%%%%%%%%%%%%%%%%%%%%%%%%%%%%%%%%%%%%%
% underrepresented minorities
\item Institute for Broadening Participation: \vspace{-0.3cm}
	\begin{enumerate} \itemsep -2pt
	\item PathwaysToScience.org: \vspace{-0.2cm}
		\begin{enumerate} \itemsep -2pt
		\item Resources for faculty and administrators (to facilitate STEM outreach activities as well as the recruitment of underrepresented minorities to the student body and faculty): \url{http://www.pathwaystoscience.org/Faculty.asp}
		\end{enumerate}
	\end{enumerate}
\item National Center for Women \& Information Technology (NCWIT): \vspace{-0.3cm}
	\begin{enumerate} \itemsep -2pt
	\item NCWIT Academic Alliance Seed Fund (for developing and implementing initiatives in colleges and universities to recruit and retain women in computing and information technology): \url{http://www.ncwit.org/work.awards.seed.html}
	\item NCWIT Symons Innovator Award (for outstanding women who have successfully built and funded an IT business): \url{http://www.ncwit.org/work.awards.innovator.html}
	\end{enumerate}
\item Women in Technology (WIT): \vspace{-0.3cm}
	\begin{enumerate} \itemsep -2pt
	\item Girls In Technology (GIT): \vspace{-0.2cm}
		\begin{enumerate} \itemsep -2pt
		\item Get Involved: \vspace{-0.1cm}
			\begin{itemize} \itemsep -1pt
			\item \url{http://www.girlsintechnology.org/getinvolved.cfm}
			\item Teacher: teach girls about IT as an after-school activity or in a summer camp session
			\item Assistant Teacher: Assist instructors in GIT sessions, after-school activities, or summer camp sessions
			\item Develop Curriculum: Develop a curriculum for a supported GIT educational program
			\item Mentor: Mentor a girl in one of [GIT's] supported programs
			\item Job Shadow: ``Let a girl shadow you at work''
			\item Guest Speaker: ``Speak to a group of girls on a topic both you and they enjoy, such as computers, technology, education, how to take apart computers, how to build a web site, etc.''
			\end{itemize}
		\end{enumerate}
	\end{enumerate}
\item European Platform of Women Scientists (EPWS): \vspace{-0.3cm}
	\begin{enumerate} \itemsep -2pt
	\item \url{http://www.epws.org/}
	\item Members: \url{http://www.epws.org/index.php?option=com_content&task=blogcategory&id=134&Itemid=4652}
	\end{enumerate}
\end{enumerate}





Commercializing academic research into products and services via start-ups: \vspace{-0.3cm}
\begin{enumerate} \itemsep -4pt
\item Ben Franklin Technology Partners (BFTP): \vspace{-0.3cm}
	\begin{enumerate} \itemsep -2pt
	\item Innovation Works (IW): \vspace{-0.2cm}
		\begin{enumerate} \itemsep -2pt
		\item For universities in the Pittsburgh metropolitan area
		\item University Innovation Grants (UIGs) / University Grants: \vspace{-0.1cm}
			\begin{enumerate} \itemsep -1pt
			\item For technology validation, market research, prototype development, and intellectual property evaluation
			\item Available online at: \url{http://www.innovationworks.org/OurPrograms/UniversityGrants/tabid/115/Default.aspx}; last accessed on November 14, 2010.
			\end{enumerate}
		\end{enumerate}
	\end{enumerate}
\end{enumerate}









%%%%%%%%%%%%%%%%%%%%%%%%%%%%%%%%%%%%%%%%%%%
\subsection{Electrical and Computer Engineering \& Computer Science Outreach}
\label{ececsoutreach}

Electrical and computer engineering, and computer science outreach: \vspace{-0.3cm}
\begin{enumerate} \itemsep -4pt
\item IEEE: \vspace{-0.3cm}
	\begin{enumerate} \itemsep -2pt
	\item {\it IEEE-USA Salary Service} provides a survey of jobs in electrical and computer engineering: \url{http://www.ieeeusa.org/careers/salary/}
	\item {\it IEEE Santa Clara Valley Section PACE}: Professional Activities Committee for Engineers (PACE); see \url{http://www.ewh.ieee.org/r6/scv/PACE/}
	\item {\it IEEE Santa Clara Valley Section}: \url{http://ewh.ieee.org/r6/scv/} and \url{http://www.ieee.org/scv}
	\item 
	\end{enumerate}
\item Association for Computing Machinery, ACM: \vspace{-0.3cm}
	\begin{enumerate} \itemsep -2pt
	\item Sanjeev Arora, Boaz Barak, and Luca Trevisan, ``Survey Papers and Essays,'' in {\it Theory Matters Wiki: Theoretical Computer Science (TCS) Advocacy Wiki}, SIGACT Committee for the Advancement of Theoretical Computer Science, ACM Special Interest Group on Algorithms and Computation Theory (SIGACT), Association for Computing Machinery, February 25, 2010. Available at: \url{http://theorymatters.org/pmwiki/pmwiki.php?n=Main.SurveyCollection}; last accessed on September 14, 2010.
	\item Online Resources for Graduating Students: \url{http://www.acm.org/membership/student/resources-for-grads}
	\end{enumerate}
\item VLSI design and verification: \vspace{-0.3cm}
	\begin{enumerate} \itemsep -2pt
	\item {\it DVClub} for individuals interested in VLSI verification: \url{http://www.dvclub.org/}
	\item {\it DeepChip.com}: \url{http://www.deepchip.com}
	\end{enumerate}
%%%%%%%%%%%%%%%%%%%%%%%%%%%%%%%
\item undergraduates: \vspace{-0.3cm}
	\begin{enumerate} \itemsep -2pt
	\item {\it Humanitarian FOSS Project}: \vspace{-0.2cm}
		\begin{enumerate} \itemsep -2pt
		\item Where FOSS refers to Free and Open Source Software
		\item For computer science and engineering students
		\item \url{http://www.hfoss.org/}
		\end{enumerate}
	\item {\it SIGDA Design Automation Summer School}: \vspace{-0.2cm}
		\begin{enumerate} \itemsep -2pt
		\item {\it NSF�SRC�SIGDA�DAC Design Automation Summer School}
		\item \url{http://www.sigda.org/dass.html}
		\item Travel grants are provided to defray travel and accommodation expenses
		\end{enumerate}
	\item {\it Young Student Support Program at DAC}: \vspace{-0.2cm}
		\begin{enumerate} \itemsep -2pt
		\item Also known as {\it DAC Young Student Support Program}
		\item \url{http://www.sigda.org/youngstudent.html}
		\item Travel grants are provided to defray travel and accommodation expenses
		\end{enumerate}
	\item {\it ACM Student Research Competition at Design Automation Conference}: \vspace{-0.2cm}
		\begin{enumerate} \itemsep -2pt
		\item Sponsored by {\it Microsoft Research}
		\item \url{http://www.sigda.org/studentcomp.html}
		\item Also, see {\it ACM Student Research Competition} @ \url{http://src.acm.org/}.
		\end{enumerate}
	\item Job database for positions in the Video Game, Animation, VFX, and Software/Technology industries: \url{http://www.creativeheads.net/}
	\end{enumerate}
%%%%%%%%%%%%%%%%%%%%%%%%%%%%%%
\item graduate students: \vspace{-0.3cm}
	\begin{enumerate} \itemsep -2pt
	\item {\it SIGDA Design Automation Summer School}: \vspace{-0.2cm}
		\begin{enumerate} \itemsep -2pt
		\item {\it NSF�SRC�SIGDA�DAC Design Automation Summer School}
		\item \url{http://www.sigda.org/dass.html}
		\item Travel grants are provided to defray travel and accommodation expenses
		\end{enumerate}
	\item {\it Young Student Support Program at DAC}: \vspace{-0.2cm}
		\begin{enumerate} \itemsep -2pt
		\item Also known as {\it DAC Young Student Support Program}
		\item \url{http://www.sigda.org/youngstudent.html}
		\item Travel grants are provided to defray travel and accommodation expenses
		\end{enumerate}
	\item {\it ACM Student Research Competition at Design Automation Conference}: \vspace{-0.2cm}
		\begin{enumerate} \itemsep -2pt
		\item Sponsored by {\it Microsoft Research}
		\item \url{http://www.sigda.org/studentcomp.html}
		\item Also, see {\it ACM Student Research Competition} @ \url{http://src.acm.org/}.
		\end{enumerate}
	\item {\it SIGDA University Booth at DAC}: \vspace{-0.2cm}
		\begin{enumerate} \itemsep -2pt
		\item Or, {\it SIGDA/DAC University Booth}
		\item \url{http://www.sigda.org/ubooth.html}
		\end{enumerate}
	\item {\it SIGDA Ph.D. Forum at DAC}: \vspace{-0.2cm}
		\begin{enumerate} \itemsep -2pt
		\item \url{http://www.sigda.org/phdforum.html}
		\item \url{http://www.sigda.org/daforum/}
		\end{enumerate}
	\item {\it DAC Graduate Scholarship}: \vspace{-0.2cm}
		\begin{enumerate} \itemsep -2pt
		\item {\it A. Richard Newton Graduate Scholarships} to Support Graduate Research and Study
		\item \url{http://www.sigda.org/gradscholarship.html}
		\end{enumerate}
	\end{enumerate}
%%%%%%%%%%%%%%%%%%%%%%%%%%%%%%
\item competitions, and programming contests and challenges: \vspace{-0.3cm}
	\begin{itemize} \itemsep -2pt
	\item {\it SIGDA CADathlon at ICCAD}: \vspace{-0.2cm}
		\begin{enumerate} \itemsep -2pt
		\item \url{http://www.sigda.org/programs/cadathlon/}
		\item \url{http://www.sigda.org/cadathlon.html}
		\item Travel grants are provided to defray travel and accommodation expenses
		\end{enumerate}
	\item ISPD Programming Contest: \url{http://www.ispd.cc/contests/}
	\item ACM International Workshop on Timing Issues in the Specification and Synthesis of Digital Systems (TAU Workshop): \vspace{-0.2cm}
		\begin{enumerate} \itemsep -2pt
		\item Power Grid Simulation Contest: \url{http://www.tauworkshop.com/PREVIOUS/contest_2011.html}
		\end{enumerate}
	\item IEEE Computer Society Simulator Design competition: \url{http://www.computer.org/portal/web/competition}
	\item {\it DAC/ISSCC Student Design Contest}: \vspace{-0.2cm}
		\begin{enumerate} \itemsep -2pt
		\item \url{http://www.dac.com}
		\end{enumerate}
	\item {\it ACM/IEEE International Conference on Formal Methods and Models for Codesign -- Design Contest}: \vspace{-0.2cm}
		\begin{enumerate} \itemsep -2pt
		\item MEMOCODE Hardware/Software Co-Design Contest (MEMOCODE HW/SW co-design contest)
		\item \url{http://www-memocode2010.imag.fr/}
		\item \url{http://memocode2010.csail.mit.edu/redmine/wiki/memocode2010/Results}
		\end{enumerate}
	\item {\it International Low Power Design Contest}: \vspace{-0.2cm}
		\begin{enumerate} \itemsep -2pt
		\item ACM/IEEE International Symposium on Low Power Electronics and Design (ISLPED) -- Design Contest
		\item The International Symposium on Low Power Electronics and Design is holding the International Low Power Design Contest to provide a forum for universities and research organizations to showcase original ``power-aware'' designs and to highlight the innovations and design choices targeted at low power.
		\item The goal is to encourage and highlight design-oriented approaches to power reduction.
		\item \url{http://www.islped.org/2010/index.html}
		\end{enumerate}
	\item {\it University LSI Design Contest @ ASP-DAC}: \vspace{-0.2cm}
		\begin{enumerate} \itemsep -2pt
		\item Application areas or types of circuits of the original LSI circuit designs include (but are not limited to): \vspace{-0.1cm}
			\begin{enumerate} \itemsep -1pt
			\item Analog, RF and Mixed-Signal Circuits
			\item Digital Signal Processing
			\item Microprocessors
			\item Custom ASIC
			\end{enumerate} 
		\item Methods or technology used for implementation include: \vspace{-0.1cm}
			\begin{enumerate} \itemsep -1pt
			\item Full Custom and Cell-Based LSIs
			\item Gate Arrays
			\item FPGA/PLDs.
			\end{enumerate}
		\item \url{http://www.aspdac.com/aspdac2011/cfd/}
		\end{enumerate}
	\item IEEE Programming Challenge at IWLS: \url{http://www.iwls.org/challenge/}
	\item IEEE Asian Solid-State Circuits Conference (A-SSCC) Student Design Contest: \url{http://a-sscc2010.a-sscc.org/contest.html}
	\item {\it VLSI Conference 2011 - Design Contest}: \vspace{-0.2cm}
		\begin{enumerate} \itemsep -2pt
		\item Design/project fields include (but not limited to): \vspace{-0.1cm}
			\begin{enumerate} \itemsep -1pt
			\item Digital Integrated Circuits
			\item Analog Integrated Circuits
			\item FPGA based designs
			\item Computer Architectures/ Processors
			\item Reconfigurable Computing Systems
			\item SoC / Platform-based designs
			\item Embedded Systems
			\item MEMS/Optics/Bio-Chips
			\item Innovative Design Methodologies and Verification Techniques.
			\end{enumerate}
		\item \url{http://vlsiconference.com/vlsi2011/submissions_design_contest.html}
		\end{enumerate}
	\item {\it Satisfiability Modulo Theories Competition} (SMT-COMP): \vspace{-0.2cm}
		\begin{enumerate} \itemsep -2pt
		\item Competition for SMT solvers
		\item \url{http://www.smtcomp.org/2010/}
		\end{enumerate}
	\item {\it SAT Competition 201X}, where $X > 0$ \& $X {\it mod} 2 = 1$: \vspace{-0.2cm}
		\begin{enumerate} \itemsep -2pt
		\item The purpose of the competition is to identify new challenging benchmarks and to promote new solvers for the propositional satisfiability problem (SAT) as well as to compare them with state-of-the-art solvers.
		\item \url{http://www.satcompetition.org/}
		\end{enumerate}
	\item {\it SAT-Race 201X}, where $X > 0$ \& $X {\it mod} 2 = 0$: \vspace{-0.2cm}
		\begin{enumerate} \itemsep -2pt
		\item SAT-Race 201X is a competitive event for solvers of the Boolean Satisfiability (SAT) problem. 
		\item In contrast to the SAT Competitions, the focus of SAT-Race is on application benchmarks only.
		\item \url{http://baldur.iti.uka.de/sat-race-2010/}
		\end{enumerate}
	\item Hardware Model Checking Competition (HWMCC): \url{http://fmv.jku.at/hwmcc10/}
	\item {\it CADE ATP System Competition} (CASC): \vspace{-0.2cm}
		\begin{enumerate} \itemsep -2pt
		\item It is a yearly competition of fully automated theorem provers for classical first order logic.
		\item \url{http://www.cs.miami.edu/~tptp/CASC/}
		\end{enumerate}
	\item Apple Design Awards: \url{http://developer.apple.com/wwdc/ada/index.html}
	\item {\it International Constraint Solver Competition}: \vspace{-0.2cm}
		\begin{enumerate} \itemsep -2pt
		\item Also known as: \vspace{-0.2cm}
			\begin{enumerate} \itemsep -2pt
			\item International Constraint Solver Competition (CSP, Max-CSP and Weighted-CSP competition)
			\item International CSP Solver Competition (CSP, Max-CSP and Weighted-CSP competition)
			\end{enumerate}
		\item The Fourth International Constraint Solver Competition (CSC'2009) is organized to improve our knowledge of what is behind the efficiency of constraint satisfaction algorithms, heuristics, solving strategies, and constraint systems.
		\item \url{http://cpai.ucc.ie/}
		\end{enumerate}
	\item International Conference on Field-Programmable Technology (FPT 201X): \vspace{-0.2cm}
		\begin{enumerate} \itemsep -2pt
		\item FPT Design Competition: \url{http://cas.ee.ic.ac.uk/people/as999/FPTDesignComp/}
		\end{enumerate}
	\item International Microwave Symposium: Student Design Competitions -- Jan (includes AMS circuit simulation, and AMS/RF EDA); \url{http://ims2011.org/Technical_Program/Student_Design_Competitions.html}
	\item {\it QBFEVAL'1X}: \vspace{-0.2cm}
		\begin{enumerate} \itemsep -2pt
		\item QBF Solver competition for solvers to determine Quantified Boolean Formula (QBF) satisfiability.
		\item QBFLIB is a collection of instances, solvers, and tools related to Quantified Boolean Formula (QBF) satisfiability. See \url{http://www.qbflib.org/}.
		\item \url{http://www.qbflib.org/index_eval.php}
		\end{enumerate}
	\item {\it Pseudo-Boolean Competition 201X}: \vspace{-0.2cm}
		\begin{enumerate} \itemsep -2pt
		\item Competition for pseudo-Boolean solvers.
		\item \url{http://www.cril.univ-artois.fr/PB10/}
		\end{enumerate}
	\item {\it Answer Set Programming System Competition}: \vspace{-0.2cm}
		\begin{enumerate} \itemsep -2pt
		\item \url{http://dtai.cs.kuleuven.be/events/ASP-competition/}
		\end{enumerate}
	\item {\it Max-SAT Evaluation, Max-SAT 201X}: \vspace{-0.2cm}
		\begin{enumerate} \itemsep -2pt
		\item Competition for Max-SAT solvers
		\item \url{http://www.maxsat.udl.cat/}
		\item \url{http://www.maxsat.udl.cat/09/}
		\end{enumerate}
	\item {\it IEEEXtreme 24 Hour Programming Challenge}: \vspace{-0.2cm}
		\begin{enumerate} \itemsep -2pt
		\item Programming contest for college students
		\item \url{http://portal.ieee.org/web/membership/students/scholarshipsawardscontests/ieeextreme.html}
		\end{enumerate}
	\item {\it ACM International Collegiate Programming Contest} (ACM-ICPC or ICPC): \vspace{-0.2cm}
		\begin{enumerate} \itemsep -2pt
		\item Programming contest for college students
		\item Official web page: \url{http://cm.baylor.edu/welcome.icpc}
		\item Other web resources: \vspace{-0.1cm}
			\begin{enumerate} \itemsep -1pt
			\item {\it Wikipedia}: \url{http://en.wikipedia.org/wiki/ACM_International_Collegiate_Programming_Contest}
			\item {\it }: \url{}
			\item {\it }: \url{}
			\item {\it Valladolid Online Judge Site}: \url{http://acm.uva.es/}
			\item {\it ACMSolver :: Art of Programming Contest, Tips and Tricks for C, C++, Java}: \url{http://www.acmsolver.org/}
			\end{enumerate}
		\item 
		\end{enumerate}
	\item {\it TopCoder} coding and design contests: \vspace{-0.2cm}
		\begin{enumerate} \itemsep -2pt
		\item The contests cover various fields, such as: \vspace{-0.1cm}
			\begin{enumerate} \itemsep -1pt
			\item Algorithm
			\item Conceptualization
			\item Specification
			\item Architecture
			\item Component Design
			\item Component Development
			\item Assembly
			\item Test Scenarios
			\item Test Suites
			\item UI Prototype
			\item Rich Internet Application (RIA) Build
			\item Bug Race
			\item Marathon Match
			\item High School (for high school students)
			\item Copilot Opportunities
			\end{enumerate}
		\item \url{http://www.topcoder.com/}
		\end{enumerate}
	\item IEEE Presidents' Change the World competition: \vspace{-0.2cm}
		\begin{enumerate} \itemsep -2pt
		\item The IEEE Presidents� Change the World Competition recognizes students who develop unique solutions to real-world problems using engineering, science, computing and leadership skills to benefit their community, the world at large, or both. 
		\item \url{http://www.ieeechangetheworld.org/}
		\end{enumerate}
	\item Google Code Jam (programming contest): \url{http://code.google.com/codejam/} and \url{http://en.wikipedia.org/wiki/Google_Code_Jam}
	\item {\it RoboCup}\texttrademark\ competitions: \vspace{-0.2cm}
		\begin{enumerate} \itemsep -2pt
		\item Has different categories, including soccer, rescue operations, and home applications.
		\item \url{http://www.robocup.org/}
		\end{enumerate}
	\item ICFP Programming Contest (ICFP refers to International Conference on Functional Programming): \url{http://icfpcontest.org/}
	\item Student Cluster Competition (SCC): \vspace{-0.2cm}
		\begin{enumerate} \itemsep -2pt
		\item SCC is held at each (annual) SC conference, which is the International Conference for High Performance Computing, Networking, Storage, and Analysis. IEEE Computer Society and the Association for Computing Machinery are the sponsors for this conference.
		\item During SC10, teams consisting of six students, undergraduate and/or high school, will showcase the amazing power of clusters and the ability to utilize open source software to solve interesting and important problems. They will compete in real-time on the exhibit floor to run a workload of real-world applications on clusters of their own design while never exceeding the dictated power limit.
		\item During SC10 in New Orleans, teams will assemble, test and tune their machines and run the HPCC benchmarks until the starting bell rings on Monday night at the Exhibit Opening Gala where they will be given the competition data sets. In full view of conference attendees, teams will execute the prescribed workload while showing progress and science visualization output on large high-resolution displays in their areas. Teams race to correctly complete the greatest number of application runs during the competition period until the close of the exhibit floor on Wednesday evening.
		\item \url{http://sc10.supercomputing.org/?pg=studentcluster.html}
		\end{enumerate}
	\item Cypress Semiconductor Corporation: \vspace{-0.2cm}
		\begin{enumerate} \itemsep -2pt
		\item ARM Cortex-M3 PSoC\textregistered\ 5 Design Challenge: \url{http://www.cypress.com/?id=3271}
		\end{enumerate}
	\item Mentor Graphics: \vspace{-0.2cm}
		\begin{enumerate} \itemsep -2pt
		\item PCB Technology Leadership Awards (PCB design contest): \url{http://www.mentor.com/products/pcb-system-design/tla/index.cfm?v=mentorgraphics&p=handout:tla&a=print_card&g=sdd&s=1x1&c=ocid_2203&cmpid=3911}, or \url{http://www.mentor.com/go/tla}
		\end{enumerate}
	\item INFORMS Data Mining Contest: \vspace{-0.2cm}
		\begin{enumerate} \itemsep -2pt
		\item \url{http://ifors.org/web/call-for-participation-informs-data-mining-contest-2010/}
		\item \url{http://kaggle.com/informs2010}
		\end{enumerate}
	\item INFORMS Doing Good with Good OR - Student Competition: \vspace{-0.2cm}
		\begin{enumerate} \itemsep -2pt
		\item Doing Good with Good OR-Student Competition is held each year to identify and honor outstanding projects in the field of operations research and the management sciences conducted by a student or student group that have a significant societal impact.
		\item \url{http://www.informs.org/Recognize-Excellence/INFORMS-Prizes-Awards/Doing-Good-with-Good-OR}
		\end{enumerate}
	\item HPC Challenge Award Competition: \url{http://www.hpcchallenge.org/}
	\item Sphere Online Judge, SPOJ (programming contest): \url{http://www.spoj.pl/}
	\item High Performance and Scientific Computing Contest (Argonne National Laboratory, U.S. Department of Energy, DOE): \url{https://wiki.alcf.anl.gov/index.php/HPSC_Contest_Information}
	\item Argonne National Laboratory, ANL; Mathematics and Computer Science Division: \vspace{-0.2cm}
		\begin{enumerate} \itemsep -2pt
		\item J. H. Wilkinson Prize for Numerical Software (for developers of numerical software): \url{http://www.mcs.anl.gov/research/opportunities/wilkinsonprize/index.php}
		\end{enumerate}
	\item Society for Industrial and Applied Mathematics, SIAM: \vspace{-0.2cm}
		\begin{enumerate} \itemsep -2pt
		\item SIAM/ACM Prize in Computational Science and Engineering: \url{http://www.siam.org/prizes/sponsored/cse.php}. [ For developers of mathematical and computational tools and methods for the solution of science and engineering. Or, for developers of computational science and engineering software. ]
		\end{enumerate}
	\end{itemize}
	\item Sun HPC Software Programming Challenge (Oracle Corporation): \url{http://wikis.sun.com/display/HPCContest/Home}
%%%%%%%%%%%%%%%%%%%%%%%%%%%%%%
\item News media: \vspace{-0.3cm}
	\begin{itemize} \itemsep -2pt
	\item --- --- --- --- --- --- --- --- --- --- --- --- --- --- --- --- --- --- --- --- --- --- --- --- --- --- --- --- --- --- ---
	\item \colorbox{blue}{\bf News media for Electronic Design Automation}
	% News media for Electronic Design Automation
	\item {\it EDACafe}: \url{http://www.edacafe.com/}
	\item {\it SIGDA E-Newsletter} (SIGDA Electronic Newsletter): \url{http://www.sigda.org/newsletter/}
	\item {\it DeepChip.com}: \url{http://www.deepchip.com}
	\item --- --- --- --- --- --- --- --- --- --- --- --- --- --- --- --- --- --- --- --- --- --- --- --- --- --- --- --- --- --- ---
	\item \colorbox{blue}{\bf News media for Electrical and Computer Engineering}
	% News media for Electrical and Computer Engineering
	\item {\it EE Times} (Electronic Engineering Times): \url{http://www.eetimes.com/}
	\item {\it EDN} (Electrical Design News): \url{http://www.edn.com/}
	\item {\it IEEE Spectrum}: \url{http://spectrum.ieee.org/}
	\item {\it The Institute} (from IEEE): \url{http://www.theinstitute.ieee.org}
	\item {\it IEEE-USA Today's Engineer}: \url{http://www.todaysengineer.org/}
	\item {\it DeepChip.com}: \url{http://www.deepchip.com}
	\item --- --- --- --- --- --- --- --- --- --- --- --- --- --- --- --- --- --- --- --- --- --- --- --- --- --- --- --- --- --- ---
	\item \colorbox{blue}{\bf News media for Computer Science and Engineering, Information Systems, and IT}
	% News media for Computer Science and Engineering, Information Systems, and IT
	\item {\it ACM TechNews}: \url{http://technews.acm.org/}
	\item {\it TechCareers}: \url{http://www.techcareers.com/}
	\item {\it }: \url{}
	\item {\it }: \url{}
	\item {\it }: \url{}
	\item {\it }: \url{}
	\item {\it }: \url{}
	\item {\it }: \url{}
	\item {\it }: \url{}
	\item --- --- --- --- --- --- --- --- --- --- --- --- --- --- --- --- --- --- --- --- --- --- --- --- --- --- --- --- --- --- ---
	\item \colorbox{blue}{\bf Other News Media}
	% Other News Media
	\item {\it iTunes U}
	\item {\it YouTube EDU}
	\end{itemize}
%%%%%%%%%%%%%%%%%%%%%%%%%%%%%%
\item underrepresented minorities: \vspace{-0.3cm}
	\begin{enumerate} \itemsep -2pt
	\item women: \vspace{-0.2cm}
		\begin{enumerate} \itemsep -2pt
		\item IEEE Women in Engineering (WIE): \url{http://www.ieee.org/membership_services/membership/women/index.html?WT.mc_id=WIE_nav1}
		\item ACM-W: \url{http://women.acm.org/}
		\item Computer Research Association's Committee on the Status of Women in Computing Research (CRA-W): \vspace{-0.1cm}
			\begin{enumerate} \itemsep -1pt
			\item \url{http://www.cra-w.org/}
			\item Computing Research Association's Committee on the Status of Women (CRA-W) and the Coalition to Diversify Computing (CDC), {\it CompArch Summer School on Parallel Programming and Architectures}. Available at: \url{http://www.princeton.edu/~archss/}; last accessed on September 3, 2010.
			\end{enumerate}
		\item National Center for Women \& Information Technology: \url{http://www.ncwit.org/}
		\item African-American Women in Technology organization (AAWIT): \url{http://www.aawit.net/09/index.cfm}
		\item Grace Hopper Celebration of Women in Computing (conference for female IT students, professors, and professionals): \url{http://gracehopper.org/} or \url{http://gracehopper.org/2010/}
		\item Anita Borg Institute for Women and Technology: \vspace{-0.1cm}
			\begin{enumerate} \itemsep -1pt
			\item Has many programs for female students and professionals: \url{http://anitaborg.org/}
			\end{enumerate}
		\end{enumerate}
	\end{enumerate}
\end{enumerate}







%%%%%%%%%%%%%%%%%%%%%%%%%%%%%%%%%%%%%%%%%%%
\section{Scholarships, Fellowships, Awards, and Financial Aid}
\label{scholarshipsfinaidawards}

Resources for scholarships, fellowships, and financial aid: \vspace{-0.3cm}
\begin{enumerate} \itemsep -4pt
\item --- --- --- --- --- --- --- --- --- --- --- --- --- --- --- --- --- --- --- --- --- --- --- --- --- --- --- --- --- --- ---
\item \colorbox{blue}{\bf Lists of Scholarships and Fellowships}
% Lists of Scholarships and Fellowships
\item List of scholarships: \vspace{-0.3cm}
	\begin{enumerate} \itemsep -2pt
	\item Engineering Education Service Center, EESC (Engineering): \url{http://www.engineeringedu.com/scholars.html}
	\item High Performance and Embedded Architecture and Compilation, HiPEAC (Computer Science and Engineering): \url{http://www.hipeac.net/all_jobs_op}
	\item Office of Doctoral Programs at USC Viterbi School of Engineering, {\bf University of Southern California}. External Fellowships and other support: \url{http://viterbi.usc.edu/students/phd/fellowships-and-other-support/external-fellowships.htm}. USC Fellowships: \url{http://viterbi.usc.edu/students/phd/fellowships-and-other-support/usc-fellowships.htm}
	\item Columbia College, {\bf Columbia University} in the City of New York: \url{http://www.college.columbia.edu/students/fellowships/catalog}
	\item {\bf New York University} School of Law: \url{http://www.law.nyu.edu/financialaid/supplementalaid/fellowships/index.htm}
	\item Swedish Institute: \vspace{-0.2cm}
		\begin{enumerate} \itemsep -2pt
		\item The Swedish Institute, a government agency, administers over 500 scholarships each year for students and researchers coming to Sweden to pursue their objectives at a Swedish university.
		\item Study in Sweden: scholarships, \url{http://www.studyinsweden.se/Scholarships/}
		\item Swedish Institute (SI): \url{http://www.si.se/English/Navigation/Scholarships-and-exchanges/} [ Has special programs for Pakistanis and Turkish citizens ]
		\end{enumerate}
	\item The Swedish Foundation for International Cooperation in Research and Higher Education (STINT): \vspace{-0.2cm}
		\begin{enumerate} \itemsep -2pt
		\item \url{http://www.stint.se/en}
		\item Scholarships and grants: \url{http://www.stint.se/en/scholarships_and_grants}
		\end{enumerate}
	\item Center for the Advancement of Hispanics in Science and Engineering Education (CAHSEE): \url{http://www.cahsee.org/6resources/scholarships.asp.htm}
	\item University of Wisconsin-Madison: \vspace{-0.2cm}
		\begin{enumerate} \itemsep -2pt
		\item Grants Information Collection: A Cooperating Collection of the Foundation Center Library Network, \url{http://grants.library.wisc.edu/}
		\end{enumerate}
	\item {\it Find A PhD}: \url{http://www.findaphd.com/}
	\item QS World Grad School Tour Scholarships (QS Quacquarelli Symonds Limited): \url{http://graduateschool.topuniversities.com/world-grad-school-tour/scholarships}
	\item GlobalGrant (requires paid access to the list of scholarships and fellowships): \url{http://www.globalgrant.com/en/stipendier.html} and \url{http://www.globalgrant.com/}
	\item Stockholm University: \vspace{-0.2cm}
		\begin{enumerate} \itemsep -2pt
		\item \url{http://www.su.se/pub/jsp/polopoly.jsp?d=777&a=1770}
		\item \url{http://www.su.se/pub/jsp/polopoly.jsp?d=797}
		\item \url{http://www.su.se/pub/jsp/polopoly.jsp?d=788}
		\item \url{http://www.su.se/pub/jsp/polopoly.jsp?d=777&a=1769}
		\end{enumerate}
	\item NordForsk (in Norwegian): \url{http://www.nordforsk.org/index.cfm}
	\item Wallenberg Scholars (in Swedish): \url{http://www.wallenberg.com/default.aspx} or \url{http://www.wallenberg.com/in-english.aspx}
	\item Royal Institute of Technology (in Swedish): \url{http://www.kth.se/aktuellt/stipendier/stipendier-och-anslag-1.2024}
	\item European Commission: \vspace{-0.2cm}
		\begin{enumerate} \itemsep -2pt
		\item Marie Curie Fellowships: \vspace{-0.1cm}
			\begin{enumerate} \itemsep -1pt
			\item \url{http://cordis.europa.eu/fp7/people/home_en.html}
			\item \url{http://ec.europa.eu/research/mariecurieactions/}
			\item \url{http://ec.europa.eu/research/fp6/mariecurie-actions/action/fellow_en.html}
			\item \url{http://www.mariecurie.org/}
			\end{enumerate}
		\item Euraxess: \url{http://ec.europa.eu/euraxess/}
		\item \url{http://ec.europa.eu/index_en.htm}
		\end{enumerate}
	\item Science Please (for research positions in life sciences in The Netherlands and Belgium, including Ph.D. and postdoc positions): \url{http://www.scienceplease.com/} or \url{http://www.scienceplease.com/about-us}
	\item University of Gothenburg: \vspace{-0.2cm}
		\begin{enumerate} \itemsep -2pt
		\item ResearchResearch: \url{http://www.researchresearch.com/} or \url{http://www.gu.se/english/research/scholarships/ResearchResearch/}
		\item Scholarship links: \url{http://www.gu.se/english/research/scholarships/scholarship_links/}
		\item Scholarships at University of Gothenburg: \url{http://www.gu.se/english/research/scholarships/gu/}
		\end{enumerate}
	\item Princeton University; The Graduate School: \url{http://gradschool.princeton.edu/financial/}
	\item National Association for Bilingual Education: \vspace{-0.2cm}
		\begin{enumerate} \itemsep -2pt
		\item List of Scholarships: \url{http://www.nabe.org/scholarship.html}
		\end{enumerate}
	\item {\bf Pennsylvania State University}: \vspace{-0.2cm}
		\begin{enumerate} \itemsep -2pt
		\item Office of Engineering Diversity; Penn State College of Engineering: \vspace{-0.1cm}
			\begin{enumerate} \itemsep -1pt
			\item Undergraduate Student Scholarships: \url{http://www.engr.psu.edu/oed/UnderScholarships.html}
			\item Graduate Student Scholarships: \url{http://www.engr.psu.edu/oed/GradScholarships.html}
			\item High School Student Scholarships: \url{http://www.engr.psu.edu/oed/HighSchoolScholarships.html}
			\item Disabled Student Scholarships: \url{http://www.engr.psu.edu/oed/DisabScholarships.html}
			\item Corporate Office of Engineering Diversity (OED) Scholarships: \url{http://www.engr.psu.edu/oed/OEDScholarships.html}
			\end{enumerate}
		\item University Fellowships Office: \vspace{-0.1cm}
			\begin{enumerate} \itemsep -1pt
			\item \url{http://sites.google.com/site/psuufo/}
			\item Prestigious Scholarships: \url{http://sites.google.com/site/psuufo/prestigious}
			\item Penn State Scholarships: \url{http://sites.google.com/site/psuufo/internal-scholarships}
			\item Other resources: \url{http://sites.google.com/site/psuufo/resources}
			\end{enumerate}
		\end{enumerate}
	\item {\bf Peterson's} college search: \vspace{-0.2cm}
		\begin{enumerate} \itemsep -2pt
		\item {\it College Scholarship Search}: \url{http://www.petersons.com/college-search/scholarship-search.aspx}
		\end{enumerate}
	\item Society for Industrial and Applied Mathematics (SIAM): \vspace{-0.2cm}
		\begin{enumerate} \itemsep -2pt
		\item Fellowship \& Research Opportunities: \url{http://www.siam.org/students/resources/fellowship.php}
		\end{enumerate}
	\item Institute of International Education (IIE): \vspace{-0.2cm}
		\begin{enumerate} \itemsep -2pt
		\item {\it Funding for US Study Online}: \vspace{-0.1cm}
			\begin{enumerate} \itemsep -1pt
			\item \url{http://www.fundingusstudy.org/}
			\end{enumerate}
		\end{enumerate}
	\end{enumerate}
\item --- --- --- --- --- --- --- --- --- --- --- --- --- --- --- --- --- --- --- --- --- --- --- --- --- --- --- --- --- --- ---
\item \colorbox{blue}{\bf Scholarships and Fellowships in Electrical and Computer Engineering}
% Scholarships and Fellowships in Electrical and Computer Engineering
\item IEEE: \vspace{-0.3cm}
	\begin{enumerate} \itemsep -2pt
	\item IEEE Awards, Competitions, and Scholarships: \url{http://www.ieee.org/membership_services/membership/students/awards/index.html}
	\item IEEE Circuits and Systems Society Pre-Doctoral Scholarships: Announced via email from IEEE Circuits and Systems Society
	\item IEEE Power \& Energy Society: \vspace{-0.2cm}
		\begin{enumerate} \itemsep -2pt
		\item G. Ray Ekenstam Memorial Scholarship: \vspace{-0.1cm}
			\begin{enumerate} \itemsep -1pt
			\item \url{http://www.ieee-pes.org/g-ray-ekenstam-memorial-scholarship}
			\item ``The Scholarship Fund awards, on an annual basis, a scholarship to a qualified undergraduate student who seeks an electrical engineering degree in the field of power or a related discipline, from an accredited US university or college.''
			\end{enumerate}
		\end{enumerate}
	\item IEEE Reliability Society: \vspace{-0.2cm}
		\begin{enumerate} \itemsep -2pt
		\item IEEE Reliability Society Scholarship: \url{http://www.ieee.org/portal/cms_docs_relsoc/relsoc/newsflipper/RS_Scholarship_Application.pdf} [ Look under the tab/option on ``Useful Information'' in the panel on the left. ]
		\end{enumerate}
	\end{enumerate}
\item The George Michael Memorial HPC Fellowship Program: \vspace{-0.3cm}
	\begin{enumerate} \itemsep -2pt
	\item The Association of Computing Machinery (ACM), IEEE Computer Society and SC Conference series have established the High Performance Computing (HPC) Ph.D. Fellowship Program. The SC conference is the International Conference for High Performance Computing, Networking, Storage, and Analysis. IEEE Computer Society and the Association for Computing Machinery are the sponsors for this conference.
	\item Every year, up to three fellowship recipients will each receive a stipend of at least \$5,000 (U.S.) for one academic year, plus travel support to attend the SC conference.
	\item See \url{http://sc10.supercomputing.org/?searchterm=fellowship&pg=GeorgeMichaelMemorial.html}
	\end{enumerate}
\item Intel: \vspace{-0.3cm}
	\begin{enumerate} \itemsep -2pt
	\item Intel Foundation Fellowship: \vspace{-0.2cm}
		\begin{enumerate} \itemsep -2pt
		\item Intel Foundation Ph.D. Fellowship % \url{http://intelscholarships.intel.com/}
		\item \url{http://www.intel.com/education/highered/studentprograms/fellowship.htm}
		\item This awards two-year fellowships to Ph.D. candidates pursuing leading-edge work in fields related to Intel's business and research interests.
		\item Fellowships are available at select U.S. universities, by invitation only, and focus on Ph.D. students who have completed at least one year of study.
		\item The fellowship includes a cash award (tuition/fees/stipend), an Intel mentor, and the opportunity to participate in an internship at Intel.
		\end{enumerate}
	\end{enumerate}
\item IBM: \vspace{-0.3cm}
	\begin{enumerate} \itemsep -2pt
	\item \url{http://www-304.ibm.com/jct01005c/university/scholars/phdfellowship}
	\item IBM Ph.D. Fellowship Award
	\item The IBM Ph.D. Fellowship Awards is an intensely competitive program which honors exceptional Ph.D. students in many academic disciplines and areas of study, for example: computer science and engineering, electrical and mechanical engineering , physical sciences (including chemistry, material sciences, and physics), mathematical sciences (including optimization), business sciences (including financial services, communication, and learning/knowledge), and service sciences, management, and engineering.
	\item IBM Herman Goldstine Postdoctoral Fellowship in Mathematical Sciences: \url{http://domino.research.ibm.com/comm/research_projects.nsf/pages/goldstine.index.html}
	\item Josef Raviv Memorial Postdoctoral Fellowship; see \url{http://domino.research.ibm.com/comm/research.nsf/pages/d.compsci.josef.raviv.general.info.html}, \url{http://domino.research.ibm.com/comm/research.nsf/pages/d.compsci.raviv.winner.html}, and \url{http://domino.research.ibm.com/comm/research.nsf/pages/d.compsci.raviv.winner2008.html}
	\end{enumerate}
\item AMD: Ph.D. fellowship, \url{http://developer.amd.com/programs/fellowship/Pages/default.aspx}
\item Qualcomm, {\it Qualcomm Innovation Fellowship} for Ph.D. students in Electrical Engineering and Computer Science at Stanford, UC Berkeley, UCLA, UCSD, and USC: \url{http://www.qualcomm.com/innovation/research/university_relations/innovation_fellowship/qinf10.html}
\item NVIDIA: \vspace{-0.3cm}
	\begin{enumerate} \itemsep -2pt
	\item NVIDIA Fellowship Program; see \url{http://www.nvidia.com/page/fellowship_programs.html}
	\end{enumerate}
\item Automatic RF Techniques Group (ARFTG): \vspace{-0.3cm}
	\begin{enumerate} \itemsep -2pt
	\item Microwave Measurement Student Fellowship (for ``graduate students who show promise and interest in pursuing research related to improvement of radio frequency and microwave measurement techniques''): \url{http://www.arftg.org/student_fellowship.html}
	\end{enumerate}
\item Gallium Arsenide Applications Symposium (GAAS) Association: \vspace{-0.3cm}
	\begin{enumerate} \itemsep -2pt
	\item GAAS PhD Student Fellowship (for Ph.D. students who have accepted papers at the European Microwave Integrated Circuits Conference, EuMIC): \url{http://www.gaas-symposium.org/english/awards_fellowship.htm} and \url{http://www.eumweek.com/2010/EuMIC.asp?id=c}
	\end{enumerate}
\item The Institution of Engineering and Technology, IET: \vspace{-0.3cm}
	\begin{enumerate} \itemsep -2pt
	\item Hudswell International Research Scholarship: \url{http://www.theiet.org/about/scholarships-awards/ambition/postgraduate1/hudswell-what.cfm}
	\item IET Postgraduate Scholarship: \url{http://www.theiet.org/about/scholarships-awards/ambition/postgraduate1/postgrad-what.cfm}
	\end{enumerate}
\item --- --- --- --- --- --- --- --- --- --- --- --- --- --- --- --- --- --- --- --- --- --- --- --- --- --- --- --- --- --- ---
\item \colorbox{blue}{\bf Scholarships and Fellowships in Computer Science}
% Scholarships and Fellowships in Computer Science
\item ACM Special Interest Group on Symbolic and Algebraic Manipulation (SIGSAM): List of Ph.D. positions in computer algebra and symbolic computation, as listed by SIGSAM; see \url{http://www.sigsam.org/opportunities.phtml?searchterm=fellowship}
\item Carnegie Mellon University: \vspace{-0.3cm}
	\begin{enumerate} \itemsep -2pt
	\item women@SCS School of Computer Science: \vspace{-0.2cm}
		\begin{enumerate} \itemsep -2pt
		\item Individuals, Corporations \& Organizations: \url{http://women.cs.cmu.edu/Resources/Funding/}
		\end{enumerate}
	\end{enumerate}
\item IBM: \vspace{-0.3cm}
	\begin{enumerate} \itemsep -2pt
	\item \url{http://www-304.ibm.com/jct01005c/university/scholars/phdfellowship}
	\item IBM Ph.D. Fellowship Award
	\item The IBM Ph.D. Fellowship Awards is an intensely competitive program which honors exceptional Ph.D. students in many academic disciplines and areas of study, for example: computer science and engineering, electrical and mechanical engineering , physical sciences (including chemistry, material sciences, and physics), mathematical sciences (including optimization), business sciences (including financial services, communication, and learning/knowledge), and service sciences, management, and engineering.
	\item IBM Herman Goldstine Postdoctoral Fellowship in Mathematical Sciences: \url{http://domino.research.ibm.com/comm/research_projects.nsf/pages/goldstine.index.html}
	\item Josef Raviv Memorial Postdoctoral Fellowship; see \url{http://domino.research.ibm.com/comm/research.nsf/pages/d.compsci.josef.raviv.general.info.html}, \url{http://domino.research.ibm.com/comm/research.nsf/pages/d.compsci.raviv.winner.html}, and \url{http://domino.research.ibm.com/comm/research.nsf/pages/d.compsci.raviv.winner2008.html}
	\end{enumerate}
\item Computing Innovation Fellows (CIFellows); post my profile on \url{http://cifellows.org/profiles/}; also see \url{http://www.cifellows.org/}
\item Microsoft: \vspace{-0.3cm}
	\begin{enumerate} \itemsep -2pt
	\item Microsoft Research Graduate Women's Scholarship: \url{http://research.microsoft.com/en-us/collaboration/awards/fellows-women.aspx}
	\item Microsoft Research PhD Fellowship: \url{http://research.microsoft.com/en-us/collaboration/awards/apply-us.aspx}
	\end{enumerate}
\item Google: \vspace{-0.3cm}
	\begin{enumerate} \itemsep -2pt
	\item Google Fellowship Program; see \url{http://googleblog.blogspot.com/2009/05/best-and-brightest.html}
	\end{enumerate}
\item NVIDIA: \vspace{-0.3cm}
	\begin{enumerate} \itemsep -2pt
	\item NVIDIA Fellowship Program; see \url{http://www.nvidia.com/page/fellowship_programs.html}
	\end{enumerate}
\item Facebook Ph.D. Fellowship: \url{http://www.facebook.com/careers/fellowship.php}
\item Yahoo! Labs: Yahoo! Key Scientific Challenges Program, \url{http://labs.yahoo.com/ksc}
\item Qualcomm, {\it Qualcomm Innovation Fellowship} for Ph.D. students in Electrical Engineering and Computer Science at Stanford, UC Berkeley, UCLA, UCSD, and USC: \url{http://www.qualcomm.com/innovation/research/university_relations/innovation_fellowship/qinf10.html} and \url{http://www.qualcomm.com/innovation/research/university_relations/innovation_fellowship/}
\item Computing Research Association (CRA): Outstanding Undergraduate Researchers, \url{http://www.cra.org/awards/undergrad-current/}
\item {\color{blue} European Research Consortium for Informatics and Mathematics (ERCIM)}: \vspace{-0.3cm}
	\begin{enumerate} \itemsep -2pt
	\item ERCIM Alain Bensoussan Fellowship Programme (for Ph.D. degree holders in selected research areas): \url{http://fellowship.ercim.eu/} and \url{http://www.ercim.eu/news/283-fellowship-programme}; research areas are listed at: \url{http://fellowship.ercim.eu/home/topic}. Deadlines are on April 30 and September 30 annually.
	\end{enumerate}
\item {\it Theory Matters Wiki}; Theoretical Computer Science (TCS) Advocacy Wiki: \vspace{-0.3cm}
	\begin{enumerate} \itemsep -2pt
	\item Funding Opportunities and Tips: \url{http://theorymatters.org/pmwiki/pmwiki.php?n=Main.FundingOpportunities}
	\end{enumerate}
\item Kurt G{\"{o}}del Research Prize Fellowship: \vspace{-0.3cm}
	\begin{enumerate} \itemsep -2pt
	\item 2 Ph.D. (pre-doctoral) fellowships
	\item 2 post-doctoral fellowships
	\item 1 unrestricted fellowship
	\item $[$Scope of the$]$ original fellowship proposals [includes] the areas of: \vspace{-0.2cm}
		\begin{enumerate} \itemsep -2pt
		\item set theory
		\item recursion theory
		\item proof theory/intuitionism
		\item model theory
		\item computer assisted reasoning
		\item philosophy of mathematics 
		\end{enumerate}
	\item All fellowship proposals, regardless of subject area, will be judged according to: \vspace{-0.2cm}
		\begin{enumerate} \itemsep -2pt
		\item the relevance and resemblance of the research (finished and proposed) to the great insights and originality of Kurt G{\"{o}}del
		\item its general interest and clarity of motivation
		\item its rigorous scientific quality and depth. 
		\end{enumerate}
	\item \url{http://fellowship.logic.at/}
	\end{enumerate}
\item Hewlett-Packard Company: \vspace{-0.3cm}
	\begin{enumerate} \itemsep -2pt
	\item Hewlett-Packard Labs India (Bengaluru / Bangalore): \vspace{-0.2cm}
		\begin{enumerate} \itemsep -2pt
		\item {\it BITS - HP Labs India Ph.D. Fellowship} for Research related to Information Technologies: \vspace{-0.1cm}
			\begin{enumerate} \itemsep -1pt
			\item \url{http://www.hpl.hp.com/india/bits-hplindia_phd/index.html} or \url{http://www.hpl.hp.com/india/bits-hplindia_phd/}
			\item \url{http://www.hpl.hp.com/india/bits-hplindia_phd/iiitbphd.html}
			\item BITS, Pilani and HP Labs India jointly offer a unique PhD fellowship for research in Information and Communication Technologies (ICT) relevant to fast-growing markets like India.
			\item HP Labs India currently has ongoing Ph.D. Fellowships with BITS Pilani and IIIT, Bangalore: \url{http://www.hpl.hp.com/india/bits-hplindia_phd/university.html}
			\end{enumerate}
		\item Open Innovation Office: \vspace{-0.1cm}
			\begin{enumerate} \itemsep -1pt
			\item \url{http://www.hpl.hp.com/open_innovation/}
			\item HP Labs Innovation Research Program (IRP): \url{http://www.hpl.hp.com/open_innovation/irp/index.html}
			\end{enumerate}
		\end{enumerate}
	\end{enumerate}
\item Code for America (CfA): \vspace{-0.3cm}
	\begin{enumerate} \itemsep -2pt
	\item CfA Fellowship (develop web applications for local governments in the US): \url{http://codeforamerica.org/fellows/}
	\end{enumerate}
\item University of Minnesota, Twin Cities: \vspace{-0.3cm}
	\begin{enumerate} \itemsep -2pt
	\item College of Science and Engineering: \vspace{-0.2cm}
		\begin{enumerate} \itemsep -2pt
		\item Charles Babbage Institute: \vspace{-0.1cm}
			\begin{enumerate} \itemsep -1pt
			\item Adelle and Erwin Tomash Graduate Fellowship (for Ph.D. candidates doing research in the history of IT/computing - all but dissertation Ph.D. students only): \url{http://www.cbi.umn.edu/research/tfellowship.html}
			\item Arthur L. Norberg Travel Fund (short-term grants-in-aid to help scholars with travel expenses to use archival collections at the Charles Babbage Institute): \url{http://www.cbi.umn.edu/research/ntravelfund.html}
			\end{enumerate}
		\end{enumerate}
	\end{enumerate}
\item --- --- --- --- --- --- --- --- --- --- --- --- --- --- --- --- --- --- --- --- --- --- --- --- --- --- --- --- --- --- ---
\item \colorbox{blue}{\bf Scholarships and Fellowships in Biomedical Engineering}
% Scholarships and Fellowships in Biomedical Engineering
\item Whitaker International Fellows and Scholars Program: \vspace{-0.3cm}
	\begin{enumerate} \itemsep -2pt
	\item For graduate/Ph.D. students and postdocs in biomedical engineering
	\item \url{http://www.whitaker.org/home}
	\end{enumerate}
\item --- --- --- --- --- --- --- --- --- --- --- --- --- --- --- --- --- --- --- --- --- --- --- --- --- --- --- --- --- --- ---
\item \colorbox{blue}{\bf Scholarships and Fellowships in Optical Engineering}
% Scholarships and Fellowships in Optical Engineering
\item {\it SPIE} -- The International Society for Optical Engineering: \vspace{-0.3cm}
	\begin{enumerate} \itemsep -2pt
	\item ``SPIE Scholarship Program'' for undergraduates or graduate students studying optics, photonics, imaging, or optoelectronics program or related discipline (e.g., physics, electrical engineering): \url{http://spie.org//x1733.xml?WT.svl=mddm14}
	\item Other scholarships (including scholarships for students doing research in nanolithography techniques and lasers): \url{http://spie.org/x1736.xml}
	\end{enumerate}
\item {\it Kidger Optics Associates} Michael Kidger Memorial Scholarship (to a college freshman, or sophomore of optical design): \url{http://www.kidger.com/mkms_requirements.html}
\item --- --- --- --- --- --- --- --- --- --- --- --- --- --- --- --- --- --- --- --- --- --- --- --- --- --- --- --- --- --- ---
\item \colorbox{blue}{\bf Scholarships and Fellowships in Mechanical Engineering}
% Scholarships and Fellowships in Mechanical Engineering
\item American Society of Mechanical Engineers (ASME): \vspace{-0.3cm}
	\begin{enumerate} \itemsep -2pt
	\item Graduate Teaching Fellowships (for Ph.D. students in mechanical engineering): \url{http://www.asme.org/Education/College/FinancialAid/Graduate_Teaching_Fellowships.cfm}
	\item ASME Scholarships: \vspace{-0.2cm}
		\begin{enumerate} \itemsep -2pt
		\item \url{http://www.asme.org/Education/College/FinancialAid/Scholarships.cfm}
		\item US Undergraduates: \url{http://www.asme.org/Education/College/FinancialAid/US_Undergraduates.cfm}
		\item Graduate Students: \url{http://www.asme.org/Education/College/FinancialAid/Graduate_Students.cfm}
		\item International Students: \url{http://www.asme.org/Education/College/FinancialAid/International_Undergraduates.cfm}
		\end{enumerate}
	\item Auxiliary Scholarships: \vspace{-0.2cm}
		\begin{enumerate} \itemsep -2pt
		\item \url{http://www.asme.org/Education/College/FinancialAid/Auxiliary_Scholarships.cfm}
		\item Undergraduate Scholarships: \url{http://www.asme.org/Education/College/FinancialAid/Undergraduate_Scholarships.cfm}
		\item Graduate Scholarships: \url{http://www.asme.org/Education/College/FinancialAid/Graduate_Scholarships.cfm}
		\item Rice-Cullimore Scholarship (for international graduate students in the US): \url{http://www.asme.org/Education/College/FinancialAid/RiceCullimore_Scholarship.cfm}
		\end{enumerate}
	\item International Petroleum Institute�s College Scholarships (for undergraduates): \url{http://www.asme-ipti.org/public/pagscholarshipprograms.aspx}
	\item International Petroleum Institute�s Graduate Fellowship (for individuals entering a graduate program in mechanical engineering, and has an interest in the petroleum industry): \url{http://www.asme-ipti.org/public/pagscholarshipprograms.aspx} and \url{http://www.asme.org/Communities/Students/Grad/Fellowships.cfm}
	\end{enumerate}
\item --- --- --- --- --- --- --- --- --- --- --- --- --- --- --- --- --- --- --- --- --- --- --- --- --- --- --- --- --- --- ---
\item \colorbox{blue}{\bf Scholarships and Fellowships in Civil Engineering}
% Scholarships and Fellowships in Civil Engineering
\item American Society of Civil Engineers (ASCE): \vspace{-0.3cm}
	\begin{enumerate} \itemsep -2pt
	\item Jack E. Leisch Memorial National Graduate Fellowship (for graduate students in transportation/traffic engineering): \url{http://www.asce.org/Content.aspx?id=25021}
	\item Scholarships \& Fellowships (for undergraduates and graduate students): \url{http://www.asce.org/Content.aspx?id=18337}
	\end{enumerate}
\item American Concrete Institute (ACI): \vspace{-0.3cm}
	\begin{enumerate} \itemsep -2pt
	\item ACI Foundation Fellowships \& Scholarships: \url{http://www.concrete.org/STUDENTS/ST_SCHOLARSHIPS.HTM}
	\end{enumerate}
\item Institute of Transportation Engineers: \vspace{-0.3cm}
	\begin{enumerate} \itemsep -2pt
	\item Burton W. Marsh Fellowship for Graduate Study in Traffic and Transportation Engineering: \url{http://www.ite.org/education/Burton_W_MarshFellowship.asp}
	\end{enumerate}
\item --- --- --- --- --- --- --- --- --- --- --- --- --- --- --- --- --- --- --- --- --- --- --- --- --- --- --- --- --- --- ---
\item \colorbox{blue}{\bf Scholarships and Fellowships in Chemical Engineering}
% Scholarships and Fellowships in Chemical Engineering
\item American Institute of Chemical Engineers (AIChE) scholarships (includes scholarships for underrepresented minorities): \url{http://www.aiche.org/Students/Scholarships/index.aspx}
\item --- --- --- --- --- --- --- --- --- --- --- --- --- --- --- --- --- --- --- --- --- --- --- --- --- --- --- --- --- --- ---
\item \colorbox{blue}{\bf Scholarships and Fellowships in Aerospace Engineering}
% Scholarships and Fellowships in Aerospace Engineering
\item American Institute of Aeronautics and Astronautics (AIAA): \vspace{-0.3cm}
	\begin{enumerate} \itemsep -2pt
	\item AIAA Foundation Scholarships: \vspace{-0.2cm}
		\begin{enumerate} \itemsep -2pt
		\item \url{http://www.aiaa.org/content.cfm?pageid=211}
		\item For undergraduates and graduate students
		\item Named scholarships for undergraduates are: \vspace{-0.1cm}
			\begin{enumerate} \itemsep -1pt
			\item \url{http://www.aiaa.org/content.cfm?pageid=226}
			\item A. Thomas Young Scholarship
			\item L. S. ``Skip'' Fletcher Scholarship 
			\item Sam F. Iacobellis Scholarship
			\item Robert L. Crippen Scholarship
			\item E. C. ``Pete'' Aldridge Scholarship
			\item Liquid Propulsion Technical Committee Scholarship
			\item Space Transportation Technical Committee Scholarship
			\item Digital Avionics Technical Committee Scholarship (4)
			\item Next Century of Flight Scholarship (2)
			\item Leatrice Gregory Pendray Scholarship
			\end{enumerate}
		\item Awards for graduate students: \vspace{-0.1cm}
			\begin{enumerate} \itemsep -1pt
			\item \url{http://www.aiaa.org/content.cfm?pageid=227}
			\item Martin Summerfield Propellants and Combustion Graduate Award
			\item Guidance, Navigation, And Control Graduate Award
			\item Gordon C. Oates Air Breathing Propulsion Graduate Award
			\item William T. Piper, Sr. General Aviation Systems Graduate Award
			\item Orville and Wilbur Wright Graduate Award
			\item John Leland Atwood Graduate Award
			\item Open Topic Graduate Award
			\end{enumerate}
		\end{enumerate}
	\item Student Design Competition Award: \url{http://www.aiaa.org/content.cfm?pageid=401}
	\end{enumerate}
\item --- --- --- --- --- --- --- --- --- --- --- --- --- --- --- --- --- --- --- --- --- --- --- --- --- --- --- --- --- --- ---
\item \colorbox{blue}{\bf Scholarships and Fellowships in Mathematics}
% Scholarships and Fellowships in Mathematics
\item Association for Women in Mathematics (AWM): \vspace{-0.3cm}
	\begin{enumerate} \itemsep -2pt
	\item Travel grants: \url{http://sites.google.com/site/awmmath/programs/travel-grants}
	\item Alice T. Schafer Mathematics Prize for excellence in mathematics by an undergraduate woman: \url{http://sites.google.com/site/awmmath/programs/schafer-prize}
	\item The {\it Ruth I. Michler Memorial Prize} of the AWM is awarded annually to a woman recently promoted to Associate Professor or an equivalent position in the mathematical sciences: \url{http://sites.google.com/site/awmmath/programs/michler-prize}
	\end{enumerate}
\item Seth Bonder Scholarship for Applied Operations Research in Health Services: \url{http://www.informs.org/Recognize-Excellence/INFORMS-Community-Prizes-and-Awards/Seth-Bonder-Scholarship-for-Applied-Operations-Research-in-Health-Services}
\item Oberwolfach Foundation: \vspace{-0.3cm}
	\begin{enumerate} \itemsep -2pt
	\item Oberwolfach Prize (for young European mathematicians): \url{http://www.mfo.de/programme/prize/}
	\item John Todd Fellowship (or John Todd Award) [for young excellent mathematicians working in numerical analysis]: \url{http://www.mfo.de/programme/todd/}
	\end{enumerate}
\item Clay Mathematics Institute: Clay Research Award, \url{http://www.claymath.org/research_award/}
\item --- --- --- --- --- --- --- --- --- --- --- --- --- --- --- --- --- --- --- --- --- --- --- --- --- --- --- --- --- --- ---
\item \colorbox{blue}{\bf Scholarships and Fellowships in Science}
% Scholarships and Fellowships in Science
\item {\it Science.gov} (USA.gov for Science): \vspace{-0.3cm}
	\begin{enumerate} \itemsep -2pt
	\item Internship and Fellowship Opportunities in Science for Undergraduate Students: \url{http://www.science.gov/internships/undergrad.html}
	\item Graduate Students/Postdoctoral Fellowships: \url{http://www.science.gov/internships/graduate.html}
	\end{enumerate}
\item Heinz Family Philanthropies: \vspace{-0.3cm}
	\begin{enumerate} \itemsep -2pt
	\item Teresa Heinz Scholars for Environmental Research program (for Ph.D./MS students working on their thesis in environmental science/engineering) at selected universities: \url{http://www.heinzfamily.org/programs/environmentalscholars.html}
	\item \url{http://www.heinzfamily.org/}
	\end{enumerate}
\item Mayo Clinic: \vspace{-0.3cm}
	\begin{enumerate} \itemsep -2pt
	\item Postbaccalaureate Research Education Program (PREP): \url{http://www.mayo.edu/mgs/postbac-program.html}
	\end{enumerate}
\item {\it American Chemical Society (ACS)}: \vspace{-0.3cm}
	\begin{enumerate} \itemsep -2pt
	\item ACS-Hach Land Grant Undergraduate Scholarship (for chemistry undergraduates at a partner institution of ACS, and who plan to become chemistry teachers in US high schools): \url{http://portal.acs.org/portal/acs/corg/content?_nfpb=true&_pageLabel=PP_SUPERARTICLE&node_id=2243&use_sec=false&sec_url_var=region1&__uuid=eb054647-53e0-4594-81e8-8ef49159f3f4}
	\item ACS-Hach Second Career Teacher Scholarship (for graduates in chemistry or related areas who are entering an education masters program or teacher certification program): \url{http://portal.acs.org/portal/acs/corg/content?_nfpb=true&_pageLabel=PP_SUPERARTICLE&node_id=2244&use_sec=false&sec_url_var=region1&__uuid=4c27333f-4aad-481e-aaa4-f1db045d4eb4}
	\item ACS Scholars Program (for undergraduate underrepresented minorities majoring in chemistry, biochemistry, or chemical engineering): \url{http://portal.acs.org/portal/acs/corg/content?_nfpb=true&_pageLabel=PP_SUPERARTICLE&node_id=1650&use_sec=false&sec_url_var=region1&__uuid=b3b583cf-18ae-4fb0-9375-33f75a0ccf49}
	\item Scholarships: \url{http://portal.acs.org/portal/acs/corg/content?_nfpb=true&_pageLabel=PP_TRANSITIONMAIN&node_id=630&use_sec=false&sec_url_var=region1&__uuid=98e85c05-be75-4283-a97c-7a63ab4c3178}
	\end{enumerate}
\item European Molecular Biology Organization: \vspace{-0.3cm}
	\begin{enumerate} \itemsep -2pt
	\item EMBO Short-Term Fellowships (for junior researchers, including Ph.D. students): \url{http://www.embo.org/programmes/fellowships/short-term.html}
	\item EMBO Long-Term Fellowships (for junior researchers/postdocs): \url{http://www.embo.org/programmes/fellowships/long-term.html}
	\end{enumerate}
\item L'OR{\'{E}}AL: \vspace{-0.3cm}
	\begin{enumerate} \itemsep -2pt
	\item ``For Women in Science'' program: \url{http://www.lorealusa.com/forwomeninscience} or \url{http://www.lorealusa.com/_en/_us/index.aspx?direct1=00008&direct2=00008/00001}
	\item Alternatively, go to \url{http://www.lorealusa.com/_en/_us/} and select the ``For Women in Science'' tab.
	\item Check out the ``L'Or{\'{e}}al USA Fellowships for Women in Science'' (US postdocs), ``UNESCO-L'Or{\'{e}}al Fellowships for Women in Science'' (for female Ph.D. students and postdocs in the life sciences), and the ``L'Or{\'{e}}al-UNESCO Awardss for Women in Science'' (for distinguished female scientists)
	\end{enumerate}
\item American Institute of Physics (AIP): \vspace{-0.3cm}
	\begin{enumerate} \itemsep -2pt
	\item AIP and Member Society Government Science Fellowships: \vspace{-0.2cm}
		\begin{enumerate} \itemsep -2pt
		\item \url{http://www.aip.org/gov/fellowships.html}
		\item American Institute of Physics State Department Science Fellowship: \url{http://www.aip.org/gov/fellowships/sdf.html}
		\item American Institute of Physics Congressional Science Fellowship: \url{http://www.aip.org/gov/fellowships/cf.html}
		\item American Physical Society Congressional Science Fellowship: \url{http://www.aps.org/policy/fellowships/congressional.cfm}
		\item American Geophysical Union Congressional Science Fellowship: \url{http://www.agu.org/sci_pol/cong_fellowship/}
		\item Optical Society of America Congressional Science Fellowships: \url{http://www.osa.org/about_osa/public_policy/congressional_fellowships/default.aspx}
		\item For US citizens with good track records in research
		\end{enumerate}
	\item American Geophysical Union: \vspace{-0.2cm}
		\begin{enumerate} \itemsep -2pt
		\item Research Grants and Awards: \url{http://www.agu.org/about/honors/research_grants/}
		\item Student Travel Grants: \url{http://www.agu.org/education/grants/travel.shtml}
		\item Research Grants \& Awards: \url{http://www.agu.org/education/grants/research.shtml}
		\item Mass Media Fellowship: \url{http://www.agu.org/news/mass_media_fellowship/}
		\end{enumerate}
	\item Society of Physics Students (SPS): \vspace{-0.2cm}
		\begin{enumerate} \itemsep -2pt
		\item SPS Scholarships: \url{http://www.spsnational.org/programs/scholarships/}
		\item SPS Awards: \url{http://www.spsnational.org/programs/awards/}
		\end{enumerate}
	\end{enumerate}
\item Consortium for Ocean Leadership: \vspace{-0.3cm}
	\begin{enumerate} \itemsep -2pt
	\item Employment, Internships, and Opportunities [ includes funding opportunities for researchers (professors, postdocs, and grad students) ]: \url{http://www.oceanleadership.org/about-ocean-leadership/ocean-of-opportunities/}
	\item HBCU Fellowship: Ocean Leadership/IODP-USIO for Students of Historically Black Colleges and Universities, \url{http://www.oceanleadership.org/education/diversity/hbcu-fellowship/}
	\item HBCU Educator at Sea: \url{http://www.oceanleadership.org/education/diversity/hbcu-educator/}
	\item MS PHD's Professional Development Program: The Minorities Striving and Pursuing Higher Degrees of Success in the Earth System Sciences (MS PHD'S) Professional Development Program, \url{http://www.oceanleadership.org/education/diversity/ms-phds-professional-development-program/}
	\item Schlanger Ocean Drilling Fellowship Program (merit-based awards for outstanding graduate students to conduct research related to the Integrated Ocean Drilling Program): \url{http://www.oceanleadership.org/programs-and-partnerships/usssp/schlanger-fellowship/}
	\end{enumerate}
\item American Geological Institute Foundation: \vspace{-0.3cm}
	\begin{enumerate} \itemsep -2pt
	\item William L. Fisher Congressional Geoscience Fellowship (for young geoscientists to get engaged in {\bf public policy}): \url{http://www.agifoundation.org/govtaffairs.html} and \url{http://www.agifoundation.org/endowments.html}
	\item AGI Minority Participation Program: Minority Participation Program Geoscience Student Scholarships for ``underrepresented ethnic-minority (undergraduate or graduate) students in the geosciences'', \url{http://www.agiweb.org/mpp/index.html}
	\end{enumerate}
\item Lady Davis Institute/Jewish General Hospital: \vspace{-0.3cm}
	\begin{enumerate} \itemsep -2pt
	\item Awards for ``graduate students (in biomedical science) and post-doctoral fellows/clinical fellows'': \url{http://www.ladydavis.ca/en/awards}
	\end{enumerate}
\item Adolph C. and Mary Sprague Miller Institute for Basic Research in Science: \vspace{-0.3cm}
	\begin{enumerate} \itemsep -2pt
	\item Miller Fellowships (for outstanding recent Ph.D.s / postdoctoral fellowship): \url{http://millerinstitute.berkeley.edu/topage.php?nav=11&to=1} or \url{http://millerinstitute.berkeley.edu/page.php?nav=11}
	\item Visiting Miller Research Professorships (for professors and research scientists): \url{http://millerinstitute.berkeley.edu/topage.php?nav=24&to=1} or \url{http://millerinstitute.berkeley.edu/page.php?nav=24}
	\item Miller Research Professorships (for professors in the UC system): \url{http://millerinstitute.berkeley.edu/topage.php?nav=15&to=1} or \url{http://millerinstitute.berkeley.edu/page.php?nav=15}
	\item Miller Senior Fellowships (Nominations are solicited by invitation only; Senior Fellow appointments are made to tenured UC Berkeley faculty for five years, possibly renewable for a subsequent five years, but no longer.): \url{http://millerinstitute.berkeley.edu/topage.php?nav=126&to=1}
	\end{enumerate}
\item Funda{\c{c}}{\~{a}}o para a Ci{\^{e}}ncia e a Tecnologia (FCT); Minist{\'{e}}rio da Ci{\^{e}}ncia, Technologia e Ensino Superior (MCTES): International Prize Fernando Gil in Philosophy of Science, \url{http://alfa.fct.mctes.pt/apoios/premios/fernando_gil/index.phtml.pt}
\item Wellcome Trust: \vspace{-0.3cm}
	\begin{enumerate} \itemsep -2pt
	\item Wellcome Trust Sanger Institute: \vspace{-0.2cm}
		\begin{enumerate} \itemsep -2pt
		\item \url{http://www.sanger.ac.uk/workstudy/}
		\item Postdoctoral fellows (for research in genomics): \url{http://www.sanger.ac.uk/workstudy/career/postdocs/}
		\item Graduate program (for research in genomics): \url{http://www.sanger.ac.uk/workstudy/phd/}
		\item Student placements and work experience (for research in genomics): \url{http://www.sanger.ac.uk/workstudy/placements/}
		\end{enumerate}
	\end{enumerate}
\item Paul B. Beeson Career Development Awards in Aging Research Program (formerly the Beeson Physician Faculty Scholars Program): \vspace{-0.3cm}
	\begin{enumerate} \itemsep -2pt
	\item \url{http://www.beeson.org/}
	\item ``Today, the Beeson program continues to foster the independent research careers of clinically trained investigators -- a growing cadre of talented physician-scientists -- whose research and leadership are enhancing the health and quality of life of Americans, particularly older people.''
	\item About the Program: \url{http://www.beeson.org/program_hx.cfm}
	\end{enumerate}
\item American Mathematical Society: \vspace{-0.3cm}
	\begin{enumerate} \itemsep -2pt
	\item AMS Fellowships and Scholarships: \vspace{-0.2cm}
		\begin{enumerate} \itemsep -2pt
		\item \url{http://e-math.ams.org/programs/ams-fellowships/ams-fellowships}
		\item AMS Centennial Research Fellowship Program: \url{http://e-math.ams.org/programs/ams-fellowships/centennial-fellow/emp-centflyer}
		\item Waldemar J. Trijitzinsky Memorial Awards: \url{http://e-math.ams.org/programs/ams-fellowships/trjitzinsky/trjitzinsky-award}
		\item Other Sources of Funding: \url{http://e-math.ams.org/programs/funding/funding}
		\end{enumerate}
	\end{enumerate}
\item --- --- --- --- --- --- --- --- --- --- --- --- --- --- --- --- --- --- --- --- --- --- --- --- --- --- --- --- --- --- ---
\item \colorbox{blue}{\bf Scholarships and Fellowships in Medicine}
% Scholarships and Fellowships in Medicine
\item Sarnoff Medical Student Research Fellowship Program (for US medical students interested in cardiovascular research): \url{http://www.sarnoffendowment.org/}
\item Mayo Clinic: \vspace{-0.3cm}
	\begin{enumerate} \itemsep -2pt
	\item Postbaccalaureate Research Education Program (PREP): \url{http://www.mayo.edu/mgs/postbac-program.html}
	\end{enumerate}
\item Paul B. Beeson Career Development Awards in Aging Research Program (formerly the Beeson Physician Faculty Scholars Program): \vspace{-0.3cm}
	\begin{enumerate} \itemsep -2pt
	\item \url{http://www.beeson.org/}
	\item ``Today, the Beeson program continues to foster the independent research careers of clinically trained investigators -- a growing cadre of talented physician-scientists -- whose research and leadership are enhancing the health and quality of life of Americans, particularly older people.''
	\item About the Program: \url{http://www.beeson.org/program_hx.cfm}
	\end{enumerate}
\item --- --- --- --- --- --- --- --- --- --- --- --- --- --- --- --- --- --- --- --- --- --- --- --- --- --- --- --- --- --- ---
\item \colorbox{blue}{\bf Scholarships and Fellowships in Science and Engineering}
% Scholarships and Fellowships in Science and Engineering
\item National Academies: \vspace{-0.3cm}
	\begin{enumerate} \itemsep -2pt
	\item Research Associateship Programs (graduate, postdoctoral, and senior level research opportunities): \url{http://sites.nationalacademies.org/pga/rap/}
	\item Ford Foundation Fellowship Programs (predoctoral, dissertation or postdoctoral fellowships for individuals seeking academic careers in science and engineering): \url{http://sites.nationalacademies.org/PGA/FordFellowships/index.htm}
	\item \url{http://nationalacademies.org/grantprograms.html}
	\item \url{http://sites.nationalacademies.org/pga/fellowships/}
	\item List of Fellowship, Scholarship, and Grant Databases: \url{http://sites.nationalacademies.org/PGA/Fellowships/PGA_046300}
	\item List of Outside Fellowships, Scholarships, and Grants Websites: \url{http://sites.nationalacademies.org/PGA/Fellowships/PGA_046301}
	\item Awards for junior and mid-career researchers: \url{http://www.nasonline.org/site/PageServer?pagename=AWARDS_main}
	\item National Academy of Engineering, NAE: \vspace{-0.2cm}
		\begin{enumerate} \itemsep -2pt
		\item NAE Grand Challenges Scholars Program: \url{http://www.grandchallengescholars.org/}
		\end{enumerate}
	\item National Science Foundation: \vspace{-0.2cm}
		\begin{enumerate} \itemsep -2pt
		\item International Research Fellowship Program (IRFP) for junior scientists and engineers: \url{http://www.nsf.gov/funding/pgm_summ.jsp?pims_id=5179}
		\item Integrative Graduate Education and Research Traineeship Program (IGERT) for undergraduates and graduate students in STEM: \url{http://www.nsf.gov/funding/pgm_summ.jsp?pims_id=12759}
		\item National Science Foundation's Graduate Research Fellowship Program (GRFP) for students seeking research degrees in STEM: \url{http://www.nsfgrfp.org/}
		\item NSF Alliances for Graduate Education and the Professoriate (AGEP) program (to help underrepresented minorities obtain graduate degrees in STEM and prepare them for faculty positions in academia): \url{http://www.nsfagep.org/}
		\item National Science Foundation's (NSF) East Asia and Pacific Summer Institutes (EAPSI) program: \vspace{-0.1cm}
			\begin{enumerate} \itemsep -1pt
			\item \url{http://www.nsf.gov/funding/pgm_summ.jsp?pims_id=5284}
			\item The East Asia and Pacific Summer Institutes (EAPSI) provide U.S. graduate students in science and engineering: \vspace{-0.1cm}
				\begin{itemize} \itemsep -1pt
				\item first-hand research experiences in Australia, China, Japan, Korea, New Zealand, Singapore or Taiwan
				\item an introduction to the science, science policy, and scientific infrastructure of the respective location
				\item an orientation to the society, culture and language.
				\end{itemize}
			\item ``The primary goals of EAPSI are to introduce students to East Asia and Pacific science and engineering in the context of a research setting, and to help students initiate scientific relationships that will better enable future collaboration with foreign counterparts.''
			\item ``All institutes, except Japan, last approximately eight weeks from June to August. Japan lasts approximately ten weeks from June to August (specific dates are available and updated at \url{http://www.nsfsi.org/}).''
			\item Example of Ph.D. student, Jakub Szefer, from Prof. Ruby Lee's lab at Princeton University, who interned with Prof. Cheng Chen-Mou from National Taiwan University: \url{http://www.nsf.gov/discoveries/disc_summ.jsp?cntn_id=118116&org=NSF}
			\end{enumerate}
		\end{enumerate}
	\end{enumerate}
\item United States Department of Defense (DoD): \vspace{-0.3cm}
	\begin{enumerate} \itemsep -2pt
	\item National Defense Education Program; Defense Advanced Research Projects Agency (DARPA): \vspace{-0.2cm}
		\begin{enumerate} \itemsep -2pt
		\item Science, Mathematics, and Research for Transformation (SMART) scholarship program: \vspace{-0.1cm}
			\begin{itemize} \itemsep -1pt
			\item \url{http://smart.asee.org/}
			\item Co-organized by the American Society for Engineering Education
			\end{itemize}
		\item National Security Science and Engineering Faculty Fellowships (NSSEFF): \url{http://www.ndep.us/ProgNSSEFF.aspx}
		\end{enumerate}
	\end{enumerate}
\item National Society of Professional Engineers: \vspace{-0.3cm}
	\begin{enumerate} \itemsep -2pt
	\item Scholarships for undergraduates and graduate students: \url{http://www.nspe.org/Students/Scholarships/index.html}
	\item NSPE-PEC George B. Hightower, P.E. Fellowship (for an outstanding engineering graduate student): \url{http://www.nspe.org/InterestGroups/PEC/Resources/Awards/hightower_fellowship.html}
	\item PEG Management Fellowship: \vspace{-0.2cm}
		\begin{enumerate} \itemsep -2pt
		\item \url{http://www.nspe.org/InterestGroups/PEG/Resources/AwardsAndScholarships/peg_fellowship.html}
		\item ``This scholarship is designed for graduate students who are pursuing an MBA, a master's degree in engineering management, or a master's degree in public administration.''
		\end{enumerate}
	\end{enumerate}
\item Technion -- Israel Institute of Technology: \vspace{-0.3cm}
	\begin{enumerate} \itemsep -2pt
	\item Department of Mathematics: Anna and Paul Erdos postdoctoral Fellowship, \url{http://www.math.technion.ac.il/Site/people/positions.html}
	\item Lady Davis Postdoctoral Fellowship
	\item Department of Electrical Engineering: \vspace{-0.2cm}
		\begin{enumerate} \itemsep -2pt
		\item The Andrew and Erna Finci Viterbi Fellowship Program (for graduate and post-doctoral fellows), \url{http://webee.technion.ac.il/Research/Fellowship-Programs}
		\item Lady Davis Fellowship Trust: Technion Fellowships (for visiting professors, post-doctoral researchers, as well as Masters and Ph.D. students), \url{http://ldft.huji.ac.il/upload/info/}
		\item \url{http://webee.technion.ac.il/Research/Fellowship-Programs}
		\end{enumerate}
	\end{enumerate}
\item Hebrew University: \vspace{-0.3cm}
	\begin{enumerate} \itemsep -2pt
	\item Lady Davis Fellowship Trust: Technion Fellowships (for visiting professors, post-doctoral researchers, as well as Masters and Ph.D. students), \url{http://ldft.huji.ac.il/upload/info/infoHUa.html}
	\end{enumerate}
\item Hertz Foundation: \vspace{-0.3cm}
	\begin{enumerate} \itemsep -2pt
	\item The Graduate Fellowship Award: \url{http://www.hertzfoundation.org/dx/Fellowships/award.aspx}
	\item Thesis Prize: \url{http://www.hertzfoundation.org/dx/Fellowships/thesis_winners.aspx}
	\end{enumerate}
\item Krell Institute, Inc.: \vspace{-0.3cm}
	\begin{enumerate} \itemsep -2pt
	\item DOE Computational Science Graduate Fellowship: \url{http://www.krellinst.org/csgf/index.shtml}
	\end{enumerate}
\item The Winston Churchill Foundation of the United States: \vspace{-0.3cm}
	\begin{enumerate} \itemsep -2pt
	\item The Churchill Scholarship: \url{http://winstonchurchillfoundation.org/index.php?hide=1&section=eligibility}
	\end{enumerate}
\item American Society for Engineering Education: \vspace{-0.3cm}
	\begin{enumerate} \itemsep -2pt
	\item \url{http://blogs.asee.org/fellowships/}
	\item Fellowship programs: \url{http://www.asee.org/fellowship-programs}
	\item Awards: \url{http://www.asee.org/member-resources/awards/full-list-of-awards}
	\item DuPont Minorities in Engineering Award: \vspace{-0.2cm}
		\begin{enumerate} \itemsep -2pt
		\item \url{http://www.asee.org/member-resources/awards/full-list-of-awards/national-awards/special#DuPont_Minorities_in_Engineering_Award}
		\item {\bf \color{blue} ``The DuPont Minorities in Engineering Award is conferred for outstanding achievements by an engineering or engineering technology educator in increasing student diversity within engineering and engineering technology programs.''}
		\end{enumerate}
	\end{enumerate}
\item Alexander von Humboldt-Stiftung/Foundation: \vspace{-0.3cm}
	\begin{enumerate} \itemsep -2pt
	\item Feodor Lynen Research Fellowship for Postdoctoral Researchers (junior postdocs): \url{http://www.humboldt-foundation.de/web/feodor-lynen-fellowship-postdoc.html}
	\item Friedrich Wilhelm Bessel Research Award (mid-career researchers): \url{http://www.humboldt-foundation.de/web/bessel-award.html}
	\item Georg Forster Research Fellowship for Postdoctoral Researchers (for non-German junior postdocs ``with above average qualifications''): \url{http://www.humboldt-foundation.de/web/georg-forster-fellowship-postdoc.html}
	\item Humboldt Research Fellowship for Postdoctoral Researchers (junior postdocs): \url{http://www.humboldt-foundation.de/web/771.html}
	\item Sofja Kovalevskaja Award (junior postdocs): \url{http://www.humboldt-foundation.de/web/kovalevskaja-award.html}
	\item Fraunhofer-Bessel Research Award: \url{http://www.humboldt-foundation.de/web/fraunhofer-bessel-award.html}
	\item \url{http://www.humboldt-foundation.de/web/home.html}
	\end{enumerate}
\item Santa Fe Institute: Omidyar Postdoctoral Fellowship; see \url{http://www.santafe.edu/education/fellowships/omidyar-postdoctoral/}
\item Applied Materials: Applied Materials Graduate Fellowship
\item American Society of Naval Engineers (ASNE): \vspace{-0.3cm}
	\begin{enumerate} \itemsep -2pt
	\item (Undergraduate and Graduate) Scholarships: \url{http://www.navalengineers.org/awards/scholarships/Pages/ASNELandingPage.aspx}
	\end{enumerate}
\item Lindau Meeting of Nobel Laureates and Students in Lindau (Oak Ridge Associated Universities, ORAU): \vspace{-0.3cm}
	\begin{enumerate} \itemsep -2pt
	\item Graduate Student Award program: \vspace{-0.2cm}
		\begin{enumerate} \itemsep -2pt
		\item \url{http://www.orau.org/lindau/}
		\item A student nominated to participate in this program must: \vspace{-0.1cm}
			\begin{enumerate} \itemsep -1pt
			\item Be a U.S. citizen
			\item Be currently enrolled as a full-time graduate student
			\item Be currently sponsored by, or working on, and supported by projects sponsored by, the agency to which the nomination is made, such as the U.S. Department of Energy Office of Science, the National Institutes of Health or other federal agency
			\item Have completed by June 2011 two years (but not more than four years) of study toward a doctoral degree in medicine or physiology, or in a related discipline, including the basic biomedical (or life) sciences
			\end{enumerate}
		\end{enumerate}
	\end{enumerate}
\item Research Councils UK (RCUK): \vspace{-0.3cm}
	\begin{enumerate} \itemsep -2pt
	\item RCUK Academic Fellowships: \vspace{-0.2cm}
		\begin{enumerate} \itemsep -2pt
		\item \url{http://www.rcuk.ac.uk/ResearchCareers/fellowships/Pages/home.aspx}
		\item \url{http://www.rcuk.ac.uk/ResearchCareers/fellowships/Pages/about.aspx}
		\item Dorothy Hodgkin Postgraduate Awards: \vspace{-0.1cm}
			\begin{enumerate} \itemsep -1pt
			\item \url{http://www.rcuk.ac.uk/ResearchCareers/dhpa/Pages/home.aspx}
			\item ``Dorothy Hodgkin Postgraduate Awards (DHPA) is a UK scheme to bring outstanding students from India, China, Hong Kong, South Africa, Brazil, Russia and the developing world to come and study for PhDs in top rated UK research facilities.''
			\end{enumerate}
		\end{enumerate}
	\item International Funding Opportunities: \vspace{-0.2cm}
		\begin{enumerate} \itemsep -2pt
		\item \url{http://www.rcuk.ac.uk/international/funding/FundingOpps/Pages/home.aspx}
		\item Early Career Researchers: \url{http://www.rcuk.ac.uk/international/funding/FundingOpps/Pages/EarlyCareer.aspx}
		\end{enumerate}
	\item Engineering and Physical Sciences Research Council: \vspace{-0.2cm}
		\begin{enumerate} \itemsep -2pt
		\item Programs: \vspace{-0.1cm}
			\begin{enumerate} \itemsep -1pt
			\item Physical sciences: \vspace{-0.1cm}
				\begin{itemize} \itemsep -1pt
				\item Organic synthetic chemistry studentships: \url{http://www.epsrc.ac.uk/about/progs/physsci/Pages/organicstudentships.aspx}
				\item Analytical science studentships: \url{http://www.epsrc.ac.uk/about/progs/physsci/Pages/analyticalstudentships.aspx}
				\end{itemize}
			\item Mathematical sciences: \vspace{-0.1cm}
				\begin{itemize} \itemsep -1pt
				\item Fellowships (for postdoctoral research): \url{http://www.epsrc.ac.uk/about/progs/maths/Pages/fellowships.aspx}
				\end{itemize}
			\end{enumerate}
		\item Funding: \vspace{-0.1cm}
			\begin{enumerate} \itemsep -1pt
			\item \url{http://www.epsrc.ac.uk/funding/Pages/default.aspx}
			\item Grants available [has funds for (new/junior) professors and to support international collaboration]: \url{http://www.epsrc.ac.uk/funding/grants/Pages/default.aspx}
			\item Calls for proposals (open/current funding calls for applications and future/proposed calls): \url{http://www.epsrc.ac.uk/funding/calls/Pages/default.aspx}
			\item Studentships (training grants for Ph.D. and Masters students, including international students): \url{http://www.epsrc.ac.uk/funding/students/Pages/default.aspx}
			\item Fellowships (from junior scientists and engineers engaged in postdoctoral research to senior researchers): \url{http://www.epsrc.ac.uk/funding/fellows/Pages/default.aspx}
			\end{enumerate}
		\end{enumerate}
	\item Biotechnology and Biological Sciences Research Council (BBSRC): \vspace{-0.2cm}
		\begin{enumerate} \itemsep -2pt
		\item ``The UK's leading funding agency for academic research and training in the non-clinical life sciences''
		\item Funding research: \vspace{-0.1cm}
			\begin{enumerate} \itemsep -1pt
			\item \url{http://www.bbsrc.ac.uk/funding/funding-index.aspx}
			\item Fellowships (for early career scientists, for supporting individuals seeking a change in research directions or scientists who are returning to research, and senior researchers): \url{http://www.bbsrc.ac.uk/funding/fellowships/fellowships-index.aspx}
			\item Studentships (Doctoral training grants, Masters training grants, postgraduate awards, and undergraduate research grants): \url{http://www.bbsrc.ac.uk/funding/studentships/studentships-index.aspx}
			\item Special opportunities (current calls for funding): \url{http://www.bbsrc.ac.uk/funding/opportunities/opportunities-index.aspx}
			\item Apply for funding (information about the process of applying for research funds): \url{http://www.bbsrc.ac.uk/funding/apply/apply-index.aspx}
			\end{enumerate}
		\end{enumerate}
	\item Science and Technology Facilities Council: \vspace{-0.2cm}
		\begin{enumerate} \itemsep -2pt
		\item STFC Grants and Awards: \vspace{-0.1cm}
			\begin{enumerate} \itemsep -1pt
			\item \url{http://www.stfc.ac.uk/Funding+and+Grants/501.aspx}
			\item ``The Science and Technology Facilities Council offers grants and support in Particle Physics, Astronomy, Nuclear Physics and Facility Development. It also provides support for research infrastructure, training, knowledge exchange and public engagement activities through a variety of funding schemes and activities.''
			\item STFC Funding Opportunities: \url{http://www.stfc.ac.uk/Funding%20and%20Grants/598.aspx}
			\item Postgraduate Studentships: \url{http://www.stfc.ac.uk/Funding+and+Grants/637.aspx} or \url{http://www.stfc.ac.uk/Funding%20and%20Grants/636.aspx}
			\end{enumerate}
		\item Fellowship opportunities: \vspace{-0.1cm}
			\begin{enumerate} \itemsep -1pt
			\item \url{http://www.stfc.ac.uk/Funding%20and%20Grants/508.aspx}
			\item ``Fellowship opportunities in Astronomy, Solar and Planetary Science, Particle Physics, Particle Astrophysics, Nuclear Physics, Development of STFC Neutron, Laser and Synchrotron Facilities within the UK.''
			\item There are postdoctoral and advanced research fellowships.
			\end{enumerate}
		\item Innovations Partnership Schemes (IPS and mini-IPS): \url{http://www.stfc.ac.uk/19213.aspx}
		\item IPS Fellowships: \vspace{-0.1cm}
			\begin{enumerate} \itemsep -1pt
			\item \url{http://www.stfc.ac.uk/19226.aspx}
			\item The IPS fellowship is a scheme designed to support a role to develop the commercial exploitation of technologies. This is not a research orientated fellowship.
			\end{enumerate}
		\item Follow-on-Funding: \vspace{-0.1cm}
			\begin{enumerate} \itemsep -1pt
			\item \url{http://www.stfc.ac.uk/19207.aspx}
			\item ``Follow on Funding is intended to provide financial support at the very early or pre-seed stage of turning research outputs into a commercial proposition. Unlike the other research councils, in STFC, industry partners are not allowed. If you have an industry partner, please use the mini-IPS or IPS scheme.''
			\item ``STFC staff, grant funded academics and researchers at CERN and ESO are eligible to apply for follow-on-funds (see the research grants handbook for CERN and ESO eligibility). STFC staff should first investigate whether they can be funded through proof of concept funding.''
			\end{enumerate}
		\end{enumerate}
	\item Natural Environment Research Council: \vspace{-0.2cm}
		\begin{enumerate} \itemsep -2pt
		\item Grants and studentships on the web: \vspace{-0.1cm}
			\begin{enumerate} \itemsep -1pt
			\item \url{http://www.nerc.ac.uk/research/gotw.asp}
			\item Grants on the web: \url{http://gotw.nerc.ac.uk/goti.asp?c=1}
			\end{enumerate}
		\item Funding: \vspace{-0.1cm}
			\begin{enumerate} \itemsep -1pt
			\item \url{http://www.nerc.ac.uk/funding/}
			\item Postgraduate training: \vspace{-0.1cm}
				\begin{itemize} \itemsep -1pt
				\item Postgraduate eligibility (requires UK/EU citizenship): \url{http://www.nerc.ac.uk/funding/available/postgrad/eligibility.asp}
				\end{itemize}
			\item Research Fellowship Scheme [for all nationalities]: \url{http://www.nerc.ac.uk/funding/available/fellowships/}
			\item Research Experience Placements (REP) scheme [for undergraduates]: \url{http://www.nerc.ac.uk/funding/available/rep.asp}
			\item Research Grants: \vspace{-0.1cm}
				\begin{itemize} \itemsep -1pt
				\item Eligibility: \url{http://www.nerc.ac.uk/funding/available/researchgrants/eligibility.asp}
				\end{itemize}
			\end{enumerate}
		\item {\bf Other potential sources of funding}: \vspace{-0.1cm}
			\begin{enumerate} \itemsep -1pt
			\item \url{http://www.nerc.ac.uk/funding/otherfunding.asp}
			\item Look at the ``Higher Education Funding Councils'' for each country (England, Wales, Northern Ireland, and Scotland)
			\end{enumerate}
		\end{enumerate}
	\end{enumerate}
\item Nuffield Foundation: \vspace{-0.3cm}
	\begin{enumerate} \itemsep -2pt
	\item Undergraduate research bursaries in science: \url{http://www.nuffieldfoundation.org/undergraduate-research-bursaries-0}
	\item Funding for social policy projects in the UK: \vspace{-0.2cm}
		\begin{enumerate} \itemsep -2pt
		\item \url{http://www.nuffieldfoundation.org/social-policy}
		\item \url{http://www.nuffieldfoundation.org/children-and-families-law-society-education-and-open-door}
		\end{enumerate}
	\item Apply for funding: \url{http://www.nuffieldfoundation.org/apply-for-funding}
	\item Africa program: \url{http://www.nuffieldfoundation.org/africa-programme-0}
	\item Nuffield Farming Scholarships Trust: \vspace{-0.2cm}
		\begin{enumerate} \itemsep -2pt
		\item Nuffield Farming Scholarships: \url{http://www.nuffieldscholar.org/}
		\end{enumerate}
	\item The Nuffield Trust (or, The Nuffield Trust for Research and Policy Studies in Health Services): \vspace{-0.2cm}
		\begin{enumerate} \itemsep -2pt
		\item Fellowships: \vspace{-0.1cm}
			\begin{enumerate} \itemsep -1pt
			\item \url{http://www.nuffieldtrust.org.uk/fellowships/index.aspx?id=43}
			\item Rock Carling fellowship (for senior researchers in public health): \url{http://www.nuffieldtrust.org.uk/fellowships/index.aspx?id=112}
			\item John Fry Fellowship (for senior researchers in public health): \url{http://www.nuffieldtrust.org.uk/fellowships/index.aspx?id=109}
			\item Harkness Fellowships in Health Care Policy: \vspace{-0.1cm}
				\begin{itemize} \itemsep -1pt
				\item ``Since September 2009 The Nuffield Trust have been the proud co-sponsors of the prestigious Harkness Fellowships programme with The Commonwealth Fund.''
				\item ``These offer an unparalleled opportunity for the health policy analysts of the future to conduct original research and learn about healthcare in North America.''
				\item ``Mid-career health policy researchers and practitioners (including doctors, health services managers, journalists and government officials) are supported to spend 9 to 12 months in the United States conducting a policy-oriented research project and working with leading U.S. health policy experts.''
				\end{itemize}
			\end{enumerate}
		\end{enumerate}
	\end{enumerate}
\item U.S. Department of Homeland Security (DHS): \vspace{-0.3cm}
	\begin{enumerate} \itemsep -2pt
	\item DHS Scholarship and Fellowship Program: \url{http://www.orau.gov/dhsed/}
	\end{enumerate}
\item ACT, Inc.: \vspace{-0.3cm}
	\begin{enumerate} \itemsep -2pt
	\item Barry M. Goldwater Scholarship and Excellence in Education Program (for US residents who will be college upperclassmen in STEM fields in the following academic year): \url{http://www.act.org/goldwater/}
	\end{enumerate}
\item Massachusetts Institute of Technology: \vspace{-0.3cm}
	\begin{enumerate} \itemsep -2pt
	\item MIT School of Engineering: \vspace{-0.2cm}
		\begin{enumerate} \itemsep -2pt
		\item Lemelson-MIT Program: \vspace{-0.1cm}
			\begin{enumerate} \itemsep -1pt
			\item \url{http://web.mit.edu/invent/}
			\item Lemelson-MIT Awards for Invention and Innovation: \url{http://web.mit.edu/invent/a-main.html}
			\end{enumerate}
		\end{enumerate}
	\end{enumerate}
\item --- --- --- --- --- --- --- --- --- --- --- --- --- --- --- --- --- --- --- --- --- --- --- --- --- --- --- --- --- --- ---
\item \colorbox{blue}{\bf Scholarships and Fellowships in Various Fields (Including Creative Arts, Teaching, and Sports)}
% Scholarships and Fellowships in Various Fields (Including Creative Arts, Teaching, and Sports)
\item U.S. Department of Education: \vspace{-0.3cm}
	\begin{enumerate} \itemsep -2pt
	\item Robert C. Byrd Honors Scholarship Program: \vspace{-0.2cm}
		\begin{enumerate} \itemsep -2pt
		\item High school graduates who have been accepted for enrollment at institutions of higher education (IHEs), have demonstrated outstanding academic achievement, and show promise of continued academic excellence may apply to states in which they are residents.
		\item \url{http://www2.ed.gov/programs/iduesbyrd/index.html}
		\end{enumerate}
	\item \colorbox{yellow}{\bf Jacob K. Javits Fellowships Program}: \vspace{-0.1cm}
		\begin{enumerate} \itemsep -1pt
		\item This program provides fellowships to students of superior academic ability -- selected on the basis of demonstrated achievement, financial need, and exceptional promise -- to undertake study at the doctoral and Master of Fine Arts level in selected fields of arts, humanities, and social sciences.
		\item \url{http://www2.ed.gov/programs/jacobjavits/index.html}
		\end{enumerate}
	\item Close Up Fellowship Program: \vspace{-0.2cm}
		\begin{enumerate} \itemsep -2pt
		\item This program provides financial aid to enable low-income students, their teachers, and recent immigrants to come to Washington, D.C., to study the operations of the three branches of the federal government.
		\item \url{http://www2.ed.gov/programs/closeup/index.html}
		\end{enumerate}
	\item {\bf \color{blue} B.J. Stupak Olympic Scholarships}: \vspace{-0.2cm}
		\begin{enumerate} \itemsep -2pt
		\item This program provides financial assistance to athletes who are training at the U.S. Olympic Education Center or one of the U.S. Olympic training centers and who are pursuing a postsecondary education at institutions of higher education (IHEs).
		\item \url{http://www2.ed.gov/programs/olympic/index.html}
		\end{enumerate}
	\item {\bf \color{blue} Teacher Education Assistance for College and Higher Education (TEACH) Grant Program}: \vspace{-0.2cm}
		\begin{enumerate} \itemsep -2pt
		\item Through the College Cost Reduction and Access Act of 2007, Congress created the Teacher Education Assistance for College and Higher Education (TEACH) Grant Program that provides grants of up to \$4,000 per year to students who intend to teach in a public or private elementary or secondary school that serves students from low-income families.
		\item \url{http://studentaid.ed.gov/PORTALSWebApp/students/english/TEACH.jsp}
		\end{enumerate}
	\item Scholarship search engine: \url{https://studentaid2.ed.gov/getmoney/scholarship/}
	\item Financial Aid: \vspace{-0.2cm}
		\begin{enumerate} \itemsep -2pt
		\item \url{http://www2.ed.gov/finaid/landing.jhtml?src=rt}
		\item \url{http://studentaid.ed.gov/PORTALSWebApp/students/english/funding.jsp}
		\item Paying for college: \url{http://www.college.gov}
		\item Student Aid (has information for students at all levels and parents): \url{http://studentaid.ed.gov/}
		\item Student Aid Eligibility: \url{http://studentaid.ed.gov/PORTALSWebApp/students/english/aideligibility.jsp?tab=funding}
		\item Federal Student Aid: \url{http://federalstudentaid.ed.gov/}
		\item Academic Competitiveness Grant: The Academic Competitiveness Grant provides up to \$750 for the first year of undergraduate study and up to \$1,300 for the second year of undergraduate study. See \url{http://studentaid.ed.gov/PORTALSWebApp/students/english/NewPrograms.jsp}.
		\end{enumerate}
	\item Free Application for Federal Student Aid (FAFSA): \vspace{-0.2cm}
		\begin{enumerate} \itemsep -2pt
		\item Financial Aid Estimator Tool (FAFSA4caster): \url{http://www.fafsa4caster.ed.gov/F4CApp/index/index.jsf}
		\item \url{http://www.fafsa.ed.gov/}
		\end{enumerate}
	\item Federal Pell Grant Program: \url{http://www2.ed.gov/programs/fpg/index.html}
	\end{enumerate}
\item European Commission: \vspace{-0.3cm}
	\begin{enumerate} \itemsep -2pt
	\item Erasmus Programme (for Europeans): \url{http://ec.europa.eu/education/lifelong-learning-programme/doc80_en.htm}
	\item Erasmus Mundus (for non-Europeans): \url{http://ec.europa.eu/education/external-relation-programmes/doc72_en.htm}
	\end{enumerate}
\item Woodrow Wilson Foundation: \vspace{-0.3cm}
	\begin{enumerate} \itemsep -2pt
	\item {\bf \color{blue} The Woodrow Wilson-Rockefeller Brothers Fund Fellowships for Aspiring Teachers of Color (for underrepresented minorities seeking a career as a K-12 public school teacher in the US)}: \url{http://www.woodrow.org/teaching-fellowships/wwrbf/index.php}
	\item {\bf \color{blue} Woodrow Wilson Teaching Fellowship (for a MS program in teacher education, who would teach at high-need urban and rural schools or $\ge$ 3 years)}: \url{http://www.wwteachingfellowship.org/}
	\item {\bf \color{blue} Leonore Annenberg Teaching Fellowship (for a MS program in teacher education, who would teach at high-need urban and rural schools or $\ge$ 3 years)}: \url{http://www.woodrow.org/teaching-fellowships/annenberg/index.php}
	\item MMUF Travel \& Research Grants (for graduate students who participated in the Mellon Mays Undergraduate Fellowship Program): \url{http://www.woodrow.org/higher-education-fellowships/opportunity/research/index.php}
	\item MMUF Dissertation Grants (for graduate students who participated in the Mellon Mays Undergraduate Fellowship Program): \url{http://www.woodrow.org/higher-education-fellowships/opportunity/dissertation/index.php}
	\item Charlotte W. Newcombe Doctoral Dissertation Fellowship (for Ph.D. students writing their theses on ethical or religious values in all fields of the humanities and social sciences): \url{http://www.woodrow.org/higher-education-fellowships/religion_ethics/index.php}
	\item {\bf \color{blue} Woodrow Wilson Dissertation Fellowship in Women�s Studies}: \url{http://www.woodrow.org/higher-education-fellowships/women_gender/index.php}
	\item Doris Duke Conservation Fellowship program (Masters students seeking careers as practicing conservationists): \url{http://www.woodrow.org/higher-education-fellowships/conservation/index.php}
	\item Thomas R. Pickering Graduate Foreign Affairs Fellowship: \vspace{-0.2cm}
		\begin{enumerate} \itemsep -2pt
		\item Prior to joining the United States Department of State Foreign Service, this fellowship supports students entering a Masters program in the following fields: \vspace{-0.1cm}
			\begin{enumerate} \itemsep -1pt
			\item {\bf public policy}
			\item international affairs
			\item public administration
			\item academic fields such as: \vspace{-0.1cm}
				\begin{itemize} \itemsep -1pt
				\item business
				\item economics
				\item political science
				\item sociology
				\item foreign languages
				\end{itemize}
			\end{enumerate}
		\item \url{http://www.woodrow.org/higher-education-fellowships/foreign_affairs/pickering_grad/index.php}
		\end{enumerate}
	\item Thomas R. Pickering Undergraduate Foreign Affairs Fellowship (for undergraduates seeking to join the United States Department of State Foreign Service): \url{http://www.woodrow.org/higher-education-fellowships/foreign_affairs/pickering_undergrad/index.php}
	\end{enumerate}
\item Burroughs Wellcome Fund: \vspace{-0.3cm}
	\begin{enumerate} \itemsep -2pt
	\item Career Awards for Medical Scientists (post-Ph.D.): \url{http://www.bwfund.org/pages/188/Career-Awards-for-Medical-Scientists/}
	\item {\bf \color{blue} Career Award for Science and Mathematics Teachers (science or mathematics K-12 teachers in North Carolina public schools)}: \url{http://www.bwfund.org/pages/379/Career-Awards-for-Science-and-Mathematics-Teachers/}
	\end{enumerate}
\item Susan G. Komen for the Cure\textregistered: The Komen College Scholarship Program, \url{http://ww5.komen.org/ResearchGrants/CollegeScholarshipAward.html}
\item University of Kansas Madison \& Lila Self Graduate Fellowship (Ph.D. fellowships for business, economics, and STEM): \url{http://www2.ku.edu/~selfpro/}
\item Nationally Coveted College Scholarships, Graduate School Fellowships \& Postdoctoral Awards: \url{http://scholarships.fatomei.com/}
\item The Andrew W. Mellon Foundation: \vspace{-0.3cm}
	\begin{enumerate} \itemsep -2pt
	\item Fellowships \& Scholarships for undergraduates: \url{http://www.mmuf.org/undergraduates/explore-your-opportunities/fellowships-scholorships}
	\end{enumerate}
\item Siebel Scholars Foundation: \vspace{-0.3cm}
	\begin{enumerate} \itemsep -2pt
	\item For students in selected business, bioengineering, and computer science graduate programs
	\item Only available for students at selected universities.
	\item \url{http://www.siebelscholars.com/scholars}
	\item \url{http://www.siebelscholars.com/}
	\end{enumerate}
\item Aspen Institute (for leaders, e.g. in business, education, community service, and politics): \vspace{-0.3cm}
	\begin{enumerate} \itemsep -2pt
	\item Catto Fellowship Program: \url{http://www.aspeninstitute.org/leadership-programs/catto-fellowship-program}
	\item Rodel Fellowship Program: \url{http://www.aspeninstitute.org/leadership-programs/aspen-institute-rodel-fellowships-public-le-/about-rodel-fellowship-program}
	\item Henry Crown Fellowship Program: \url{http://www.aspeninstitute.org/leadership-programs/henry-crown-fellowship-program}
	\end{enumerate}
\item Smithsonian Institution: \vspace{-0.3cm}
	\begin{enumerate} \itemsep -2pt
	\item Postdoctoral Fellowships, Predoctoral Fellowships, and Graduate Student Fellowships: \vspace{-0.2cm}
		\begin{enumerate} \itemsep -2pt
		\item \url{http://www.si.edu/ofg/infotoapply.htm}
		\item \url{http://www.si.edu/ofg/fell.htm}
		\item \url{http://www.si.edu/ofg/ofgapp.htm}
		\item fields of research and study: \vspace{-0.1cm}
			\begin{enumerate} \itemsep -1pt
			\item {\bf \color{blue} American History, American Material and Folk Culture, and the History of Music and Musical Instruments}
			\item History of Science and Technology
			\item {\bf \color{blue} History of Art, Design, Crafts, and the Decorative Arts}
			\item Anthropology, Archaeology, Linguistics, and Ethnic Studies
			\item Evolutionary, Systematic, Behavioral, Environmental, and Conservation Biology
			\item Earth, Mineral, and Planetary Science
			\item Materials Characterization and Conservation
			\end{enumerate}
		\end{enumerate}
	\item Internship opportunities: \url{http://www.si.edu/ofg/internopp.htm}
	\item Research centers: \url{http://www.si.edu/research/}. [ It also has lots of information for K-12 teachers. It has resources, funding, and internship opportunities for undergraduates and graduate students pursing research in various aspects of humanities, social science, and natural science. ]
	\item Freer Gallery of Art / Arthur M. Sackler Gallery: \vspace{-0.2cm}
		\begin{enumerate} \itemsep -2pt
		\item Fellowships: \url{http://www.asia.si.edu/research/fellowships.asp}
		\end{enumerate}
	\item National Museum of American History: \vspace{-0.2cm}
		\begin{enumerate} \itemsep -2pt
		\item Jerome and Dorothy Lemelson Center for the Study of Invention and Innovation: \vspace{-0.1cm}
			\begin{enumerate} \itemsep -1pt
			\item The Lemelson Center Fellows Program (for Ph.D. students and postdocs): \url{http://invention.smithsonian.org/resources/research_fellowships.aspx}
			\end{enumerate}
		\end{enumerate}
	\end{enumerate}
\item Intercollegiate Studies Institute (ISI): \vspace{-0.3cm}
	\begin{enumerate} \itemsep -2pt
	\item William E. Simon Fellowship for Noble Purpose (for American undergraduates who are planning to use the fellowship grant for serving humanity -- in their own ways): \url{http://www.isi.org/programs/fellowships/simon.html}
	\item {\bf \color{blue} Richard M. Weaver Fellowship (for Americans who are attending a graduate program and are intending to pursue a career in academia/teaching)}: \url{http://www.isi.org/programs/fellowships/richard_weaver.html}
	\item Western Civilization Fellowships (for Americans who are attending a graduate program about Western culture/civilization): \url{http://www.isi.org/programs/fellowships/western_civilization.html}
	\item Salvatori Fellowship (for Americans who are attending a graduate program about early American history): \url{http://www.isi.org/programs/fellowships/salvatori.html}
	\item Bache Renshaw Fellowship for Doctoral Study in Education (for Americans who plan to attend doctoral programs in education): \url{http://www.isi.org/programs/fellowships/bache_renshaw.html}
	\item \url{http://www.isi.org/programs/fellowships/fellowships.html}
	\end{enumerate}
\item Le Fonds qu{\'{e}}b{\'{e}}cois de la recherche sur la nature et les technologies (The Quebec Research Fund on nature and technology): \vspace{-0.3cm}
	\begin{enumerate} \itemsep -2pt
	\item Scholarships: \url{http://www.fqrnt.gouv.qc.ca/en/bourses/index.htm}
	\end{enumerate}
\item Horatio Alger Association of Distinguished Americans, Inc.: \vspace{-0.3cm}
	\begin{enumerate} \itemsep -2pt
	\item Scholarship Programs (for US high school seniors who have faced and overcome great obstacles in their young lives): \url{https://www.horatioalger.org/scholarships/sp.cfm}
	\item Awards: \vspace{-0.2cm}
		\begin{enumerate} \itemsep -2pt
		\item \url{http://www.horatioalger.org/aboutus.cfm}
		\item Horatio Alger Award: ``dedicated community leaders who demonstrate individual initiative and a commitment to excellence; as exemplified by remarkable achievements accomplished through honesty, hard work, self-reliance and perseverance over adversity''
		\item International Horatio Alger Award: ``recipients of this award must have overcome humble beginnings and/or adversity to achieve success. They serve as outstanding role models to the international community and are committed to the Association's mission of encouraging and educating today's young people.''
		\item Norman Vincent Peale Award: ``a Member who has made exceptional humanitarian contributions to society, who has been an active participant in the Association, and who continues to exhibit courage, tenacity and integrity in the face of great challenges. ''
		\end{enumerate}
	\end{enumerate}
\item The W. Garfield Weston Foundation: \vspace{-0.3cm}
	\begin{enumerate} \itemsep -2pt
	\item Entrance Awards \& Upper Year Garfield Weston Awards (for students pursuing college or CEGEP studies in Canada): \url{http://www.garfieldwestonawards.ca/en/about}
	\end{enumerate}
\item Canadian Merit Scholarship Foundation (\url{http://www.cmsf.ca/}): Loran Award (undergraduate funding for Canadian citizens and permanent residents), \url{http://www.loranaward.ca/}
\item StartingBloc: \vspace{-0.3cm}
	\begin{enumerate} \itemsep -2pt
	\item StartingBloc Fellowship: \vspace{-0.2cm}
		\begin{enumerate} \itemsep -2pt
		\item \url{http://www.startingbloc.org/fellowship}
		\item For people who believe that economic value creation and social value creation are complementary... For people who believe in making money and doing good, and creating social and economic impact... 
		\item The Institute for Social Innovation is a ``conference'' to learn about global issues, ``corporate social responsibility, social entrepreneurship, cross sector partnerships and sustainability. Sessions are led by top academics, corporate innovators, social entrepreneurs, activists and government officials.'' 
		\end{enumerate}
	\end{enumerate}
\item The John D. and Catherine T. MacArthur Foundation: \vspace{-0.3cm}
	\begin{enumerate} \itemsep -2pt
	\item Applying for Grants: \url{http://www.macfound.org/site/c.lkLXJ8MQKrH/b.913959/k.E1BE/Applying_for_Grants.htm}
	\item Financial \& Grant Information: \url{http://www.macfound.org/site/c.lkLXJ8MQKrH/b.938093/k.9E4C/Financial__Grant_Information.htm}
	\item MacArthur Fellows Program: \url{http://www.macfound.org/site/c.lkLXJ8MQKrH/b.959463/k.9D7D/Fellows_Program.htm}
	\end{enumerate}
\item Wenner-Gren Foundations (The Wenner-Gren Center Foundation for Scientific Research, The Axel Wenner-Gren Foundation for International Exchange of Scientists and The Foundation Wenner-Grenska Samfundet): Fellowships (for Swedish postdocs), \url{http://www.swgc.org/stipendier.aspx}
\item {\'{E}}gide: \vspace{-0.3cm}
	\begin{enumerate} \itemsep -2pt
	\item EGIDE Latitudes: \url{http://www.egidelatitudes.fr/jahia/Jahia/site/egidelatitudes}
	\item Call for applications to scholarship opportunities (including a scholarship for French citizens to study abroad): \url{http://www.egide.asso.fr/jahia/Jahia/accueil/appels}
	\item Eiffel excellence scholarship programme (organized by the French Ministry of Foreign and European Affairs): \vspace{-0.2cm}
		\begin{enumerate} \itemsep -2pt
		\item \url{http://www.egide.asso.fr/jahia/Jahia/appels/eiffel}
		\item For non-French citizens pursuing advanced degrees.
		\end{enumerate}
	\end{enumerate}
\item Gottlieb Daimler and Karl Benz Foundation: \vspace{-0.3cm}
	\begin{enumerate} \itemsep -2pt
	\item {\bf \color{blue} Ph.D. fellowship for international students to study in Germany}; see \url{http://www.daimler-benz-stiftung.de/home/fellowship/en/start.html}
	\end{enumerate}
\item The San Diego Foundation: \vspace{-0.3cm}
	\begin{enumerate} \itemsep -2pt
	\item San Diego Foundation Community Scholarship Program: \vspace{-0.2cm}
		\begin{enumerate} \itemsep -2pt
		\item \url{http://www.sdfoundation.org/GrantsScholarships/Scholarships.aspx}
		\item Available scholarships: \url{http://www.sdfoundation.org/GrantsScholarships/Scholarships/ForStudents/AvailableScholarships.aspx}. Also, see \url{http://www.sdfoundation.org/GrantsScholarships/Scholarships/ForStudents/AvailableScholarships/CommonApplicationScholarships.aspx#twomey}
		\item It has scholarships for: \vspace{-0.1cm}
			\begin{enumerate} \itemsep -1pt
			\item graduating high school seniors
			\item current undergraduates
			\item non-traditional college students: \vspace{-0.1cm}
				\begin{itemize} \itemsep -1pt
				\item mature-age students
				\item mature student
				\item adult learner
				\item adult student
				\item adults who are returning to college
				\end{itemize}
			\item people pursuing teaching certificates
			\item students attending grad school
			\item students attending trade/vocational school
			\item foster youth
			\item students in various ethnic groups
			\item students in different geographical locations
			\item {\bf \color{blue} students pursuing education in certain fields, such as engineering, nursing, music, and arts and humanities}
			\end{enumerate}
		\item Separate Scholarships: \url{http://www.sdfoundation.org/GrantsScholarships/Scholarships/ForStudents/AvailableScholarships/SeparateScholarships.aspx}
		\item Other Scholarships and Financial Aid Resources: \url{http://www.sdfoundation.org/GrantsScholarships/Scholarships/ForStudents/AvailableScholarships/OtherScholarshipsandFinancialAidResources.aspx}
		\item Financial Aid Information: \url{http://www.sdfoundation.org/GrantsScholarships/Scholarships/ForStudents/Resources/FinancialAidInformation.aspx}
		\end{enumerate}
	\item Grant Opportunities (for non-profit organizations): \url{http://www.sdfoundation.org/GrantsScholarships/ForNonprofits/GrantOpportunities.aspx}
	\end{enumerate}
\item Ewing Marion Kauffman Foundation: \vspace{-0.3cm}
	\begin{enumerate} \itemsep -2pt
	\item Kauffman Dissertation Fellowship Program (for ``Ph.D., D.B.A., or other doctoral students at accredited U.S. universities to support dissertations in the area of entrepreneurship''): \url{http://www.kauffman.org/research-and-policy/kauffman-dissertation-fellowship-program.aspx}
	\item Kauffman Junior Faculty Fellowship in Entrepreneurship Research: \vspace{-0.2cm}
		\begin{enumerate} \itemsep -2pt
		\item \url{http://www.kauffman.org/research-and-policy/kauffman-junior-faculty-fellowship-in-entrepreneurship.aspx}
		\item ``to recognize tenured or tenure-track junior faculty members at accredited U.S. universities who are beginning to establish a record of scholarship and exhibit the potential to make significant contributions to the body of research in the field of entrepreneurship''
		\end{enumerate}
	\item Ewing Marion Kauffman Prize Medal for Distinguished Research in Entrepreneurship (for promising young scholars in the field of entrepreneurship): \url{http://www.kauffman.org/research-and-policy/kauffman-prize-medal-for-entrepreneurship-research.aspx}
	\item Kauffman Legal Fellowship Program (for post-J.D. research fellowship): \url{http://www.kauffman.org/research-and-policy/kauffman-legal-fellowship-program.aspx}
	\item Kauffman Global Scholars Program (for non-American top young entrepreneurs): \url{http://www.kauffman.org/entrepreneurship/kauffman-global-scholars-program.aspx}
	\item Entrepreneur Fellows program (for M.D.s and Ph.D.s who want to become high-tech start-up entrepreneurs): \url{http://www.kauffman.org/entrepreneurship/entrepreneur-fellows-program.aspx}
	\item Entrepreneur Postdoctoral Fellows program (for postdocs who want to become high-tech start-up entrepreneurs): \url{http://www.kauffman.org/entrepreneurship/entrepreneur-postdoctoral-fellows-program.aspx}
	\item Kauffman Fellows Program (``to educate and train future venture capitalists and future leaders of high-growth companies''): \url{http://www.kauffman.org/entrepreneurship/kauffman-fellows.aspx}
	\item Kauffman Foundation Outstanding Postdoctoral Entrepreneur Award: \url{http://www.kauffman.org/entrepreneurship/outstanding-postdoctoral-entrepreneur-award.aspx}
	\end{enumerate}
\item Killam Fellowships Program: \vspace{-0.3cm}
	\begin{enumerate} \itemsep -2pt
	\item \url{http://www.killamfellowships.com/}
	\item The Killam Fellowships Program allows undergraduate students from Canada and the United States to participate in a program of binational residential exchange.
	\item Killam Fellows spend either one semester or a full academic year as an exchange student in the host country.
	\end{enumerate}
\item Canada Council for the Arts: \vspace{-0.3cm}
	\begin{enumerate} \itemsep -2pt
	\item Killam Research Fellowship: \vspace{-0.2cm}
		\begin{enumerate} \itemsep -2pt
		\item \url{http://killam.canadacouncil.ca/welcome.asp}
		\item For researchers in the following fields, and interdisciplinary fields between these fields: \vspace{-0.1cm}
			\begin{enumerate} \itemsep -1pt
			\item humanities
			\item social sciences
			\item natural sciences
			\item health sciences
			\item engineering
			\end{enumerate}
		\item For outstanding researchers who are Canadian citizens or permanent residents
		\end{enumerate}
	\item Killam Prizes (and Killam Research Fellowships): \url{http://www.canadacouncil.ca/prizes/killam}
	\end{enumerate}
\item Killam Trusts: \vspace{-0.3cm}
	\begin{enumerate} \itemsep -2pt
	\item Killam Scholarship and Prize Programs (multiple fields in selected Canadian universities): \url{http://www.killamtrusts.ca/index.asp}
	\item Killam Award winners: \url{http://www.killamtrusts.ca/awardwinners.asp}
	\item Killam Scholarship and Prize Programs at various institutions (including universities): \url{http://www.killamtrusts.ca/uofAlberta.asp}
	\end{enumerate}
\item U.S. Department of State: \vspace{-0.3cm}
	\begin{enumerate} \itemsep -2pt
	\item Bureau of Educational and Cultural Affairs: \vspace{-0.2cm}
		\begin{enumerate} \itemsep -2pt
		\item Institute of International Education (administrator of program): \vspace{-0.1cm}
			\begin{enumerate} \itemsep -1pt
			\item Council for International Exchange of Scholars: \vspace{-0.1cm}
				\begin{itemize} \itemsep -1pt
				\item Fulbright Programs (for U.S. and non-U.S. Scholars): \url{http://www.cies.org/Fulbright_programs.htm}; \url{http://www.cies.org/about_fulb.htm}; \url{http://us.fulbrightonline.org/about.html}; \url{http://foreign.fulbrightonline.org/}; \url{http://exchanges.state.gov/academicexchanges/index/fulbright-program.html}; and \url{http://fulbright.state.gov/}
				\item Hubert H. Humphrey Fellowship Program: \vspace{-0.1cm}
					\begin{itemize} \itemsep -1pt
					\item For mid-career professionals in the following fields: economic development/finance and banking, agricultural and rural development, natural resources, environmental policy, and climate change, human resource management, communications/journalism, teaching of English as a foreign language, educational administration, planning, and policy, substance abuse education, treatment, and prevention, HIV/AIDS policy and prevention, public health policy and management, {\bf public policy} analysis and public administration, law and human rights, urban and regional planning, trafficking in persons - policy and prevention, technology policy and management, and higher education administration
					\item \url{http://www.humphreyfellowship.org/}
					\item \url{http://exchanges.state.gov/globalexchanges/humphrey-fellowship.html}
					\end{itemize}
				\end{itemize}
			\item International programs for scholars (search under each continent): \url{http://www.iie.org/en/Our-Global-Reach}
			\end{enumerate}
		\item International Documentary Filmmakers Fellowship: \vspace{-0.1cm}
			\begin{enumerate} \itemsep -1pt
			\item \url{http://exchanges.state.gov/cultural/docfilmmakers.html}
			\item \url{http://smpa.gwu.edu/doccenter/fellowship.php}
			\item For ``emerging or mid-career documentary filmmakers''
			\item Intensive six-week program at the Documentary Center, The George Washington University
			\end{enumerate}
		\item Office of English Language Programs: \vspace{-0.1cm}
			\begin{enumerate} \itemsep -1pt
			\item English Language Fellow Program (for ``highly qualified U.S. educators in the field of Teaching English to Speakers of Other Languages, TESOL''): \url{http://exchanges.state.gov/englishteaching/el-fellow.html}
			\item English Language Specialist Program: \vspace{-0.1cm}
				\begin{itemize} \itemsep -1pt
				\item \url{http://exchanges.state.gov/englishteaching/el-specialist.html}
				\item U.S. academics in the fields of Teaching English as a Foreign Language (TEFL) / Teaching English as a Second Language (TESL) and Applied Linguistics
				\end{itemize}
			\item E-Teacher Scholarship Program (for English teaching professionals living outside of the United States): \url{http://exchanges.state.gov/englishteaching/eteacher.html}
			\item English Access Microscholarship Program (Access): \vspace{-0.1cm}
				\begin{itemize} \itemsep -1pt
				\item \url{http://exchanges.state.gov/englishteaching/eam.html}
				\item The English Access Microscholarship Program (Access) provides a foundation of English language skills to non-elite, 14 - 18 year old students through afterschool classes and intensive summer learning activities.
				\end{itemize}
			\item \url{http://exchanges.state.gov/englishteaching/index.html}
			\end{enumerate}
		\item Office of Global Educational Programs: \vspace{-0.1cm}
			\begin{enumerate} \itemsep -1pt
			\item Community College Initiative: \vspace{-0.1cm}
				\begin{itemize} \itemsep -1pt
				\item For ``individuals from Brazil, Egypt, Ghana, Indonesia, Pakistan, South Africa, Turkey, and selected countries in Central America to spend one year studying at community colleges in the United States and earn a vocational certificate.''
				\item ``The program provides academic instruction in selected fields including agriculture, applied engineering, business management and administration, health professions, information technology, media, and tourism and hospitality management, while also immersing participants in U.S. society and cultural life.''
				\item ``Participants are recruited from historically underserved populations and may not have had opportunities for formal job training or higher education. Most participants are in their early- to mid-twenties and many already have work experience.''
				\item \url{http://exchanges.state.gov/globalexchanges/community-colleges-initiative.html}
				\end{itemize}
			\item {\bf \color{blue} Benjamin A. Gilman International Scholarship Program}: \vspace{-0.1cm}
				\begin{itemize} \itemsep -1pt
				\item ``The Benjamin A. Gilman International Scholarship Program provides scholarships to U.S. undergraduates with financial need for study abroad, including students from diverse backgrounds and students going to non-traditional study abroad destinations.''
				\item ``The applicant must be receiving a Federal Pell Grant or provide proof that he/she will be receiving a Pell Grant at the time of application or during the term of his/her study abroad.''
				\item \url{http://exchanges.state.gov/globalexchanges/gilman-scholarship-program.html}
				\end{itemize}
			\item Global Undergraduate Exchange Program (Global UGRAD Program): \vspace{-0.1cm}
				\begin{itemize} \itemsep -1pt
				\item \url{http://exchanges.state.gov/academicexchanges/guep.html}
				\item The Global Undergraduate Exchange Program (also known as the Global UGRAD Program) provides one semester and academic year scholarships to outstanding undergraduate students from underrepresented sectors in East Asia, Eurasia and Central Asia, the Near East and South Asia and the Western Hemisphere for non-degree full-time study combined with community service, internships and cultural enrichment.
				\end{itemize}
			\item Professors and Research Scholars: \url{http://exchanges.state.gov/jexchanges/programs/professor.html}
			\item Short-Term Scholar: \url{http://exchanges.state.gov/jexchanges/programs/shortterm.html}
			\item Student, College/University: \vspace{-0.1cm}
				\begin{itemize} \itemsep -1pt
				\item \url{http://exchanges.state.gov/jexchanges/programs/ucstudent.html}
				\item The College/University Student Program gives foreign students the opportunity to study at an American degree-granting post-secondary accredited educational institution, including colleges and universities. Students may participate in degree and non-degree programs. They must pursue a full-time course of study and maintain satisfactory advancement toward the completion of their academic program.
				\end{itemize}
			\item Study of the United States Institutes for Scholars: \vspace{-0.1cm}
				\begin{itemize} \itemsep -1pt
				\item Study of the United States Institutes for Scholars  are designed to strengthen curricula and improve the quality of teaching about the United States in academic institutions overseas.
				\item Foreign university faculty, secondary educators and other scholars spend approximately four weeks at host universities where they take part in a series of lectures, seminar discussions and site visits related to each institute's theme.
				\item They learn about American educational philosophies, explore new teaching methods and pursue related research interests.
				\item Interests of these institutes: \vspace{-0.1cm}
					\begin{itemize} \itemsep -1pt
					\item American Politics and Political Thought
					\item Contemporary American Literature
					\item Journalism and Media
					\item Religious Pluralism in the United States
					\item Secondary School Educators
					\item U.S. Culture and Society
					\item U.S. Foreign Policy
					\item U.S. National Security
					\end{itemize}
				\item \url{http://exchanges.state.gov/academicexchanges/scholars.html}
				\end{itemize}
			\item Study of the United States Institutes for Student Leaders: \vspace{-0.1cm}
				\begin{itemize} \itemsep -1pt
				\item Study of the United States Institutes for Student Leaders are five-to-six-week academic programs for foreign undergraduate leaders.
				\item Hosted by U.S. academic institutions throughout the United States, the Student Leader Institutes include an intensive academic component, an educational tour of other regions of the country, local community service activities and a unique opportunity for participants to get to know their American peers.
				\item \url{http://exchanges.state.gov/academicexchanges/students.html}
				\item Interests of the institutes: \vspace{-0.1cm}
					\begin{itemize} \itemsep -1pt
					\item Comparative {\bf Public Policy} for Pakistani Student Leaders
					\item Energy and the Environment
					\item Global Environmental Issues
					\item New Media
					\item Religious Pluralism in the U.S.
					\item Social Entrepreneurship
					\item U.S. Foreign Policy for East Asian Student Leaders
					\item Western Hemisphere Student Leaders 
					\item Women's Leadership
					\end{itemize}
				\end{itemize}
			\item Edmund S. Muskie Graduate Fellowship: \vspace{-0.1cm}
				\begin{itemize} \itemsep -1pt
				\item \url{http://exchanges.state.gov/academicexchanges/muskie.html}
				\item The Edmund S. Muskie Graduate Fellowship Program (Muskie) confers fellowships for Master's degree-level study in the U.S. in the fields of business administration, economics, education, environmental policy and management, international affairs, journalism/mass communications, law, library and information science, public administration, public health and {\bf public policy} for students and professionals from Eurasia.
				\item Candidates are recruited through a merit-based competition administered by the International Research \& Exchanges Board (IREX).
				\item U.S. host campuses are also selected through a competition process and generally provide tuition waivers of fifty percent.
				\item Approximately 145 fellowships are awarded each academic year.
				\end{itemize}
			\item Critical Language Scholarship Program: \vspace{-0.1cm}
				\begin{itemize} \itemsep -1pt
				\item \url{http://exchanges.state.gov/academicexchanges/sli2.html}
				\item The Critical Language Scholarship (CLS) Program provides overseas foreign language instruction and cultural enrichment experiences in 13 critical need languages for U.S. students in higher education.
				\item The CLS Program is part of a U.S. government effort to expand dramatically the number of Americans studying and mastering critical need foreign languages.
				\item Undergraduate, master's and doctoral-level students of diverse disciplines and majors are encouraged to apply for the seven-to-10-week-long programs.
				\item Participants are expected to continue their language study beyond the scholarship period, and later apply their critical language skills in their future professional careers.
				\end{itemize}
			\item Critical Language Enhancement Award (CLEA): \vspace{-0.1cm}
				\begin{itemize} \itemsep -1pt
				\item \url{http://exchanges.state.gov/academicexchanges/clea2.html}
				\item The Critical Language Enhancement Award (CLEA) provides funding to eligible Fulbright U.S. Student Program Grantees who intend to use one of the following languages for their Fulbright project: \vspace{-0.1cm}
					\begin{itemize} \itemsep -1pt
					\item Arabic (all dialiects)
					\item Azeri
					\item Bangla/Bengali
					\item Bhasa Indonesia
					\item Chinese (Mandarin Only)
					\item Farsi
					\item Gujarati
					\item Hindi
					\item Korean
					\item Marathi
					\item Pashto
					\item Punjabi
					\item Russian
					\item Turkish
					\item Urdu
					\end{itemize}
				\end{itemize}
			\end{enumerate}
		\item Office of International Visitors: \vspace{-0.1cm}
			\begin{enumerate} \itemsep -1pt
			\item International Visitor Leadership Program (IVLP): \vspace{-0.1cm}
				\begin{itemize} \itemsep -1pt
				\item \url{http://exchanges.state.gov/ivlp/index.html}
				\item \url{http://exchanges.state.gov/ivlp/ivlp.html}
				\item The Office of International Visitors manages and funds the International Visitor Leadership Program (IVLP).
				\item Launched in 1940, the IVLP is a professional exchange program that seeks to build mutual understanding between the U.S. and other nations through carefully designed short-term visits to the U.S. for current and emerging foreign leaders.
				\item These visits reflect the International Visitors' professional interests and support the foreign policy goals of the United States.
				\end{itemize}
			\end{enumerate}
		\item Program Search (find international exchange programs sponsored by the Bureau of Educational and Cultural Affairs): \url{http://exchanges.state.gov/index/search.html}
		\end{enumerate}
	\end{enumerate}
\item Mexican American Legal Defense and Educational Fund (MALDEF): \vspace{-0.3cm}
	\begin{enumerate} \itemsep -2pt
	\item Scholarship Resources: \url{http://maldef.org/leadership/scholarships/}
	\item MALDEF Law School Scholarship Program: \vspace{-0.2cm}
		\begin{enumerate} \itemsep -2pt
		\item MALDEF's Law School Scholarship Program provides several scholarships in varying amounts to deserving law students with a commitment to advancing the civil rights of Latinos.
		\item MALDEF's Law School Scholarship Program is open to all law students who will be enrolled full-time in an American-accredited law school in 2010-2011.
		\item Scholarships are awarded to students based on their commitment to serve the Latino community through law; their past achievement and potential for achievement; and their financial need.
		\item \url{http://maldef.org/leadership/scholarships/law_school_scholarship_program/index.html}
		\end{enumerate}
	\item Undergraduate Scholarship Resource Guide: \url{http://maldef.org/leadership/scholarships/resources/index.html}
	\end{enumerate}
\item Ashoka: \vspace{-0.3cm}
	\begin{enumerate} \itemsep -2pt
	\item Ashoka Fellows (to promote and support social entrepreneurship): \url{http://www.ashoka.org/fellows}
	\end{enumerate}
\item Heinz Family Foundation: \vspace{-0.3cm}
	\begin{enumerate} \itemsep -2pt
	\item Heinz Award Criteria: \vspace{-0.2cm}
		\begin{enumerate} \itemsep -2pt
		\item \url{http://heinzawards.net/awards/criteria}
		\item The Heinz Endowments
		\item Attributes and qualities of awardees: \vspace{-0.1cm}
			\begin{enumerate} \itemsep -1pt
			\item an enormous capacity to love
			\item smile
			\item take risks
			\item question
			\item work hard
			\item believe in the power of the individual to improve the lives of others
			\end{enumerate}
		\item ``Candidates [should] possess a remarkable mix of vision, optimism, creativity and hard work which, when combined, produce tangible achievements of lasting good.''
		\item Nominees must exhibit the following personal characteristics: \vspace{-0.1cm}
			\begin{enumerate} \itemsep -1pt
			\item A passion for excellence that goes beyond intellectual curiosity;
			\item A concern for humanity rooted in a deep sensitivity for the well-being of others; 
			\item A knowledge of self which acknowledges weaknesses but relies on individual strengths;
			\item A gritty determination that will see a job through to completion despite the inevitable setbacks;
			\item A broad vision which extends far beyond the particular and embraces something universal.
			\end{enumerate}
		\item Work of the candidates for a Heinz Award must meet the following criteria: \vspace{-0.1cm}
			\begin{enumerate} \itemsep -1pt
			\item Be significant and not a ``quick fix.''
			\item Have an enduring and meaningful impact.
			\item Be creative and innovative, and
			\item Be sufficiently tangible to serve as a model for replication elsewhere.
			\end{enumerate}
		\item ``In addition, candidates should be actively working in the field in which they are nominated with the hope that, in receiving this award, their potential for future societal contribution will be enhanced.''
		\end{enumerate}
	\item Categories: \vspace{-0.2cm}
		\begin{enumerate} \itemsep -2pt
		\item Arts \& Humanities
		\item Environment
		\item Human Condition
		\item {\bf Public Policy}
		\item Technology, Economy, + Employment
		\end{enumerate}
	\end{enumerate}
\item Echoing Green: \vspace{-0.3cm}
	\begin{enumerate} \itemsep -2pt
	\item Echoing Green Fellowship: \vspace{-0.2cm}
		\begin{enumerate} \itemsep -2pt
		\item \url{http://www.echoinggreen.org/fellowship}
		\item Has information on eligibility, the benefits of the fellowship, and application cycle and dates.
		\end{enumerate}
	\item Echoing Green Fellows: \url{http://www.echoinggreen.org/fellows}
	\end{enumerate}
\item Ben Franklin Technology Partners (BFTP): \vspace{-0.3cm}
	\begin{enumerate} \itemsep -2pt
	\item Innovation Works (IW): \vspace{-0.2cm}
		\begin{enumerate} \itemsep -2pt
		\item AlphaLab: \vspace{-0.1cm}
			\begin{enumerate} \itemsep -1pt
			\item ``An immersive environment where entrepreneurs can tap IW's onsite experts for business and market advice and exchange ideas with other entrepreneurs launching in similar markets''
			\end{enumerate}
		\end{enumerate}
	\end{enumerate}
\item Carnegie Corporation of New York: \vspace{-0.3cm}
	\begin{enumerate} \itemsep -2pt
	\item Carnegie Scholars Program (not available in 2010): \url{http://carnegie.org/programs/carnegie-scholars/}
	\end{enumerate}
\item New York Women's Foundation: \vspace{-0.3cm}
	\begin{enumerate} \itemsep -2pt
	\item Finch Scholar Program (with the Finch College Alumnae Association): \vspace{-0.2cm}
		\begin{enumerate} \itemsep -2pt
		\item \url{http://www.nywf.org/internship.html} and \url{http://www.finchcollege.org/}
		\item ``Our partnership with the Finch Scholar Program allows us to provide practical community service experience to an outstanding local student enrolled in college. The internship affords the Finch Scholar opportunities to work in meaningful ways in a nonprofit organization with exposure to social change philanthropy, participatory grantmaking, advocacy and {\bf public policy}. Generally, we offer one scholarship per year with a stipend.''
		\item \url{http://www.finchcollege.org/newFinchScholarPrgm.html}
		\item \url{http://www.finchcollege.org/newscholarships.html}
		\end{enumerate}
	\end{enumerate}
\item The Rockefeller Foundation: \vspace{-0.3cm}
	\begin{enumerate} \itemsep -2pt
	\item The Bellagio Center: \vspace{-0.2cm}
		\begin{enumerate} \itemsep -2pt
		\item \url{http://www.rockefellerfoundation.org/bellagio-center}
		\item Residency Programs: \vspace{-0.1cm}
			\begin{enumerate} \itemsep -1pt
			\item \url{http://www.rockefellerfoundation.org/bellagio-center/residency-programs}
			\item ``The Bellagio Residency program offers scholars, artists, thought leaders, policymakers and practitioners a serene setting conducive to focused, goal-oriented work, and the unparalleled opportunity to establish new connections with fellow residents, across a stimulating array of disciplines and geographies.  The Bellagio Center community generates new knowledge to solve some of the most complex problems facing our world and creates art that inspires reflection, understanding, and imagination.''
			\item Scholarly Residencies: \vspace{-0.1cm}
				\begin{itemize} \itemsep -1pt
				\item ``Researchers in the humanities, natural sciences, social sciences and other academic disciplines''
				\item ``The Center typically offers one-month residencies for no more than 12 scholars and scientists at a time. Individuals in any discipline and from any part of the world are welcome to apply. The Center maintains a core focus on projects consistent with the Foundation's mission to expand opportunities for poor or vulnerable people and to help see that the benefits of globalization are shared more widely. It also seeks to include beyond that core a wide variety of projects from all academic disciplines.''
				\item \url{http://www.rockefellerfoundation.org/bellagio-center/residency-programs/scholarly-residencies}
				\end{itemize}
			\item Creative Artist Residencies: \vspace{-0.1cm}
				\begin{itemize} \itemsep -1pt
				\item ``Artists, composers, writers''
				\item ``Bellagio creative artist residencies for composers, novelists, playwrights, poets, video/filmmakers and visual artists provide time for disciplined work, individual reflection, and collegial engagement, uninterrupted by the usual professional and personal demands. The Center typically offers one-month stays for no more than three to five creative artists at a time. Artists of significant achievement from any country are welcome to apply.''
				\item \url{http://www.rockefellerfoundation.org/bellagio-center/residency-programs/creative-artist-residencies}
				\end{itemize}
			\item Practitioner Residencies: \vspace{-0.1cm}
				\begin{itemize} \itemsep -1pt
				\item ``Policymakers, nonprofit leaders, journalists and public advocates''
				\item ``The Center offers residencies to professionals in fields relevant to the Rockefeller Foundation's issue areas. The Center maintains a core focus on projects consistent with our mission, to expand opportunities for poor or vulnerable people and to help see that the benefits of globalization are shared more widely.   We seek practitioner applicants with demonstrated leadership qualities and the capacity to contribute to the intellectual life at the Center.''
				\item \url{http://www.rockefellerfoundation.org/bellagio-center/residency-programs/practitioner-residencies}
				\end{itemize}
			\end{enumerate}
		\item {\bf \color{blue} Creative Arts Fellowships}: \vspace{-0.1cm}
			\begin{enumerate} \itemsep -1pt
			\item ``This high-profile program hosts visual artists at the Bellagio Center for three-month residencies that inspire creativity and promote interaction between the arts and other fields. Creative Arts Fellows, like other participants in Bellagio residency programs, have the time and space to work independently during the day. They also enjoy and benefit from a lively community of scholars, writers, policymakers and other artists who gather in the evening for dinner and occasional presentations.  The combination of private work space, an extended stay, a generous stipend and a unique group of fellow residents makes a Creative Arts Fellowship at the Bellagio Center a remarkable opportunity.''
			\item \url{http://www.rockefellerfoundation.org/bellagio-center/creative-arts-fellowships}
			\end{enumerate}
		\end{enumerate}
	\end{enumerate}
\item Wellcome Trust: \vspace{-0.3cm}
	\begin{enumerate} \itemsep -2pt
	\item Wellcome Trust Book Prize: \vspace{-0.2cm}
		\begin{enumerate} \itemsep -2pt
		\item \url{http://www.wellcomebookprize.org/About-the-prize/index.htm}
		\item ``The Wellcome Trust Book Prize celebrates the best of medicine in literature by awarding 25 000 each year for the finest fiction or non-fiction book centered around medicine.''
		\end{enumerate}
	\end{enumerate}
\item The Kennedy Memorial Trust: \vspace{-0.3cm}
	\begin{enumerate} \itemsep -2pt
	\item \url{http://www.kennedytrust.org.uk/}
	\item Kennedy Scholarship: \url{http://www.kennedytrust.org.uk/display.aspx?Id=1165&pid=0}
	\item Frank Knox Fellowships: \url{http://www.kennedytrust.org.uk/display.aspx?Id=1175&pid=0}
	\end{enumerate}
\item Foreign \& Commonwealth Office / United Kingdom: \vspace{-0.3cm}
	\begin{enumerate} \itemsep -2pt
	\item Chevening scholarships: \vspace{-0.2cm}
		\begin{enumerate} \itemsep -2pt
		\item \url{http://www.fco.gov.uk/en/about-us/what-we-do/scholarships/}
		\item ``The Chevening programme, has, over 26 years, provided more than 30,000 Scholarships at Higher Education Institutions (HEIs) in the UK for postgraduate students or researchers from countries across the world.''
		\end{enumerate}
	\item {\bf Marshall Scholarships} finance young Americans of high ability to study for a graduate degree in the United Kingdom: \url{http://www.marshallscholarship.org/}
	\end{enumerate}
\item Ministry of Education, Culture, Sports, Science and Technology (MEXT) / Japan: \vspace{-0.3cm}
	\begin{enumerate} \itemsep -2pt
	\item \url{http://www.mext.go.jp/english/}
	\item Monbukagakusho Scholarship: \vspace{-0.2cm}
		\begin{enumerate} \itemsep -2pt
		\item \url{http://en.wikipedia.org/wiki/Monbukagakusho_Scholarship}
		\item \url{http://project.monbusho.org/old/} and \url{http://www.monbusho.org/}
		\end{enumerate}
	\end{enumerate}
\item Institute of International Education (IIE): \vspace{-0.3cm}
	\begin{enumerate} \itemsep -2pt
	\item GE Foundation Scholar-Leaders Program: \vspace{-0.2cm}
		\begin{enumerate} \itemsep -2pt
		\item \url{http://www.iie.org/en/Programs/GE-Foundation-Scholar-Leaders-Program}
		\item ``The GE Foundation Scholar-Leaders Program began in 1987 in Mexico and now supports outstanding students in higher education in fourteen countries around the world. The program initially provided traditional financial support for university education, but has developed into an exciting Leadership Development Program to complement the student's academic curriculum.''
		\item Eligibility: ``Students in their first year of study in engineering, technology, business, finance, management, or economics attending a participating university. GE Foundation Scholar-Leaders qualification requirements vary by region.''
		\end{enumerate}
	\end{enumerate}
\item British Council: \vspace{-0.3cm}
	\begin{enumerate} \itemsep -2pt
	\item Shine! 2011: International Student Awards: \vspace{-0.2cm}
		\begin{enumerate} \itemsep -2pt
		\item \url{http://www.educationuk.org/shine}
		\item For international students in the United Kingdom
		\end{enumerate}
	\item Funding your studies: \vspace{-0.2cm}
		\begin{enumerate} \itemsep -2pt
		\item \url{http://www.britishcouncil.org/learning-funding-your-studies.htm}
		\item Education UK: \url{http://www.educationuk.org/pls/hot_bc/page_pls_user_advice?x=&y=&a=0&d=4460}
		\item 9/11 Scholarship Fund: \vspace{-0.1cm}
			\begin{enumerate} \itemsep -1pt
			\item \url{http://www.britishcouncil.org/911scholarships.htm}
			\item ``The 9/11 Scholarship Fund supports international students who were directly affected by the 2001 terrorist events in the US. Find out more how each scholarship offers the opportunity to study at a UK college or university every year.''
			\end{enumerate}
		\end{enumerate}
	\item {\it Youth in Action} European program: \url{http://www.britishcouncil.org/youthinaction}
	\item British Council Arts Group: \vspace{-0.2cm}
		\begin{enumerate} \itemsep -2pt
		\item Support and funding overview: \url{http://www.britishcouncil.org/arts-support-and-funding-overview.htm}
		\item Visual arts support and funding: \url{http://www.britishcouncil.org/arts-visual-arts-funding.htm}
		\item Drama and dance support and funding: \url{http://www.britishcouncil.org/arts-performing-arts-funding.htm}
		\item Literature support and funding: \url{http://www.britishcouncil.org/arts-literature-support-and-funding.htm}
		\item Film support and funding: \url{http://www.britishcouncil.org/arts-film-funding.htm}
		\item Music support and funding: \url{http://www.britishcouncil.org/arts-music-funding.htm}
		\item Architecture, design, fashion support and funding: \url{http://www.britishcouncil.org/arts-adf-funding.htm}
		\item International Short Film Festival Support Scheme: \url{http://www.britishcouncil.org/arts-film-short-films-scheme.htm}
		\end{enumerate}
	\end{enumerate}
\item Alfred P. Sloan Foundation: \vspace{-0.3cm}
	\begin{enumerate} \itemsep -2pt
	\item Sloan Research Fellowships: \vspace{-0.2cm}
		\begin{enumerate} \itemsep -2pt
		\item \url{http://www.sloan.org/fellowships}
		\item Hold a Ph.D. (or equivalent) in chemistry, physics, mathematics, computer science, economics, neuroscience or computational and evolutionary molecular biology, or in a related interdisciplinary field;
		\item Be members of the regular faculty (i.e., tenure track) of a degree-granting college or university in the United States or Canada; and
		\item Normally, be no more than six years from completion of the most recent Ph.D. or equivalent as of the year of their nomination.
		\end{enumerate}
	\end{enumerate}
\item --- --- --- --- --- --- --- --- --- --- --- --- --- --- --- --- --- --- --- --- --- --- --- --- --- --- --- --- --- --- ---
\item \colorbox{blue}{\bf Scholarships and Fellowships in Business (including Finance, Entrepreneurship, and Accounting)}
% Scholarships and Fellowships in Business (including Finance, Entrepreneurship, and Accounting)
\item IREX: \vspace{-0.3cm}
	\begin{enumerate} \itemsep -2pt
	\item Opportunities ``for individuals, organizations, universities, and alumni'': \url{http://www.irex.org/apply}
	\item Edmund S. Muskie Graduate Fellowship Program: \vspace{-0.2cm}
		\begin{enumerate} \itemsep -2pt
		\item : \url{http://www.irex.org/application/edmund-s-muskie-graduate-fellowship-program-application}
		\item ``The Muskie Program is open to graduate students and professionals from Armenia, Azerbaijan, Belarus, Georgia, Kazakhstan, Kyrgyzstan, Moldova, Russia, Tajikistan, Turkmenistan, Ukraine and Uzbekistan for one-year non-degree, one-year degree, or two-year degree study in the United States.''
		\item ``Eligible fields of study for the Muskie Program are: business administration, economics, education, environmental management, international affairs, journalism and mass communication, law, library and information science, public administration, public health, and {\bf public policy}.''
		\end{enumerate}
	\end{enumerate}
\item Sponsors for Educational Opportunity (SEO): \vspace{-0.3cm}
	\begin{enumerate} \itemsep -2pt
	\item Alternative Investment Fellowship Program: \vspace{-0.2cm}
		\begin{enumerate} \itemsep -2pt
		\item \url{http://www.seo-usa.org/Fellowship}
		\item Eligibility: \vspace{-0.1cm}
			\begin{enumerate} \itemsep -1pt
			\item \url{http://www.seo-usa.org/FellowshipEligibility}
			\item The program is open to professionals traditionally underrepresented in alternative investments who are in the first year (or second year with a third-year offer) of an analyst program at an investment bank.
			\item Corporate finance, M\&A, leveraged finance and structured finance analysts are preferred.
			\item Management consultants will also be considered.
			\end{enumerate}
		\end{enumerate}
	\item The SEO Scholars Program: \vspace{-0.2cm}
		\begin{enumerate} \itemsep -2pt
		\item \url{http://www.seo-usa.org/Scholars}
		\item The SEO Scholars Program is a rigorous out-of-school academic enrichment program that prepares motivated New York City public high school students of color to gain admission to and succeed at competitive colleges and universities throughout the country.  Numerous studies confirm that rigorous academics are the single most important factor for low-income and minority students in gaining college admission and earning a degree.  However, U.S. Department of Education research shows that ``A'' work in low-income schools equals ``C'' work in affluent schools.
		\item Admissions: \url{http://www.seo-usa.org/ScholarsAdmissions}
		\item Roadmap To Success: \url{http://www.seo-usa.org/ScholarsRoadmapToSuccess}
		\item Enrichment Programs: \url{http://www.seo-usa.org/ScholarsEnrichmentPrograms}
		\item Volunteering: \url{http://www.seo-usa.org/ScholarsVolunteering}
		\item Andrew Golkin Fund: \vspace{-0.1cm}
			\begin{enumerate} \itemsep -1pt
			\item \url{http://www.seo-usa.org/ScholarsAndrewGolkinFund}
			\item \url{http://www.seo-usa.org/andrewgolkinfund/index.html}
			\end{enumerate}
		\item Franklin H. and Shirley B. Williams Scholarship Fund: \url{http://www.seo-usa.org/ScholarsFHSBW}
		\item The Advantages of Attending a Competitive College: \url{http://www.seo-usa.org/ScholarsAdvantages}
		\end{enumerate}
	\item Career program: \vspace{-0.2cm}
		\begin{enumerate} \itemsep -2pt
		\item \url{http://www.seo-usa.org/Career}
		\item The SEO Career Program places students of color interested in finance, philanthropy, business and corporate law in internships with competitive pay, rigorous training, support through mentors, and broad access to industry professionals. 
		\item Sponsors for Educational Opportunity (SEO) is the nation's premiere summer internship program for talented underrepresented students of color that can lead to full-time job offers.
		\item SEO offers internship opportunities in the following areas: \vspace{-0.1cm}
			\begin{enumerate} \itemsep -1pt
			\item Corporate Financial Leadership: \url{http://www.seo-usa.org/Career/Corporate_Financial_Leadership}
			\item Banking/Asset Management Areas: \vspace{-0.1cm}
				\begin{itemize} \itemsep -1pt
				\item Investment Banking: \url{http://www.seo-usa.org/Career/Investment_Banking}
				\item Sales \& Trading: \url{http://www.seo-usa.org/Career/Sales_&_Trading}
				\item Investment Research: \url{http://www.seo-usa.org/Career/Investment_Research}
				\item Transaction Services: \url{http://www.seo-usa.org/Career/Transaction_Services}
				\item Asset Management: \url{http://www.seo-usa.org/Career/Asset_Management}
				\item Accounting/Finance: \url{http://www.seo-usa.org/Career/Accounting/Finance}
				\item Information Technology: \url{http://www.seo-usa.org/Career/Information_Technology}
				\end{itemize}
			\item Corporate Law: \url{http://www.seo-usa.org/Career/Corporate_Law}
			\item Nonprofit: \url{http://www.seo-usa.org/Career/Nonprofit}
			\item SEO-U: Freshmen and Sophomore Training: \url{http://www.seo-usa.org/Career/SEO-U:Freshmen_&_Sophomore_Training}
			\end{enumerate}
		\item Application Deadlines: \url{http://www.seo-usa.org/CareerApplicationDeadlines}
		\item Eligibility Information: \url{http://www.seo-usa.org/CareerEligibilityInfo}
		\item Application Tips: \url{http://www.seo-usa.org/CareerApplicationTips}
		\item Interview Tips: \url{http://www.seo-usa.org/CareerInterviewTips}
		\end{enumerate}
	\end{enumerate}
\item --- --- --- --- --- --- --- --- --- --- --- --- --- --- --- --- --- --- --- --- --- --- --- --- --- --- --- --- --- --- ---
\item \colorbox{blue}{\bf Scholarships for Studying Abroad}
% Scholarships for Studying Abroad
\item U.S. Department of State: \vspace{-0.3cm}
	\begin{enumerate} \itemsep -2pt
	\item Bureau of Educational and Cultural Affairs: \vspace{-0.2cm}
		\begin{enumerate} \itemsep -2pt
		\item Benjamin A. Gilman International Scholarship: \vspace{-0.1cm}
			\begin{enumerate} \itemsep -1pt
			\item \url{http://exchanges.state.gov/globalexchanges/gilman-scholarship-program.html}
			\item ``The Benjamin A. Gilman International Scholarship Program provides scholarships to U.S. undergraduates with financial need for study abroad, including students from diverse backgrounds and students going to non-traditional study abroad destinations.  Established under the International Academic Opportunity Act of 2000, Gilman Scholarships provide up to \$5,000 for American students to pursue overseas study for college credit.''
			\item Critical Need Languages: Students studying critical need languages are eligible for up to \$3,000 in additional funding as part of the Gilman Critical Need Language Supplement program. Those critical need languages include: \vspace{-0.1cm}
				\begin{itemize} \itemsep -1pt
				\item Arabic
				\item Chinese
				\item Korean
				\item Russian
				\item Turkic (Azerbaijani, Kazakh, Kyrgyz, Turkish, Turkmen, Uzbek)
				\item Persian (Farsi, Dari, Kurdish, Pashto, Tajiki)
				\item Indic (Hindi, Urdu, Nepali, Sinhala, Bengali, Punjabi, Marathi, Gujurati, Sindhi)
				\end{itemize}
			\item \url{http://www.iie.org/en/Programs/Gilman-Scholarship-Program}
			\item \url{http://www.iie.org/en/Programs/Gilman-Scholarship-Program/About-the-Program}
			\end{enumerate}
		\end{enumerate}
	\end{enumerate}
\item Council on International Educational Exchange (CIEE): \vspace{-0.3cm}
	\begin{enumerate} \itemsep -2pt
	\item CIEE Scholarships: \url{http://www.ciee.org/study/scholarships/index.aspx}
	\end{enumerate}
\item IES Abroad (formerly Institute of European Studies / Institute for the International Education of Students): \vspace{-0.3cm}
	\begin{enumerate} \itemsep -2pt
	\item Scholarships and Financial Aid: \url{https://www.iesabroad.org/IES/Scholarships_and_Aid/financialAid.html}
	\item IES Abroad Need-Based Financial Aid: \url{https://www.iesabroad.org/IES/Scholarships_and_Aid/Need-Based/needBasedFinancialAid.html}
	\item IES Abroad Merit-Based Scholarships: \url{https://www.iesabroad.org/IES/Scholarships_and_Aid/Merit_Based/meritBasedFinancialAid.html}
	\item IES Abroad Public University Grants: \url{https://www.iesabroad.org/IES/Scholarships_and_Aid/publicScholarship.html}
	\end{enumerate}
\item American Institute For Foreign Study (AIFS): \vspace{-0.3cm}
	\begin{enumerate} \itemsep -2pt
	\item AIFS Study Abroad Programs: \vspace{-0.2cm}
		\begin{enumerate} \itemsep -2pt
		\item \url{http://www.aifsabroad.com/programs.asp}
		\item AIFS Study Abroad Scholarships: \url{http://www.aifsabroad.com/scholarships.asp}
		\end{enumerate}
	\end{enumerate}
\item --- --- --- --- --- --- --- --- --- --- --- --- --- --- --- --- --- --- --- --- --- --- --- --- --- --- --- --- --- --- ---
\item \colorbox{blue}{\bf Scholarships and Fellowships in Public Policy and Public Health}
% Scholarships and Fellowships in Public Policy and Public Health
\item The Commonwealth Fund: \vspace{-0.3cm}
	\begin{enumerate} \itemsep -2pt
	\item Commonwealth Fund fellowship programs: \vspace{-0.2cm}
		\begin{enumerate} \itemsep -2pt
		\item \url{http://www.commonwealthfund.org/Fellowships.aspx}
		\item ``Commonwealth Fund fellowship programs are designed to give promising young researchers the opportunity for in-depth study of various health care policy topics, working with investigators, policy analysts, government officials, and others in a number of U.S. and international settings.''
		\item The Commonwealth Fund/Harvard University Fellowship in Minority Health Policy: \url{http://www.commonwealthfund.org/Fellowships/Minority-Health-Policy-Fellowship.aspx}
		\item Harkness Fellowships in Health Care Policy and Practice: \url{http://www.commonwealthfund.org/Fellowships/Harkness-Fellowships.aspx}
		\item Australian-American Health Policy Fellowship: \url{http://www.commonwealthfund.org/Fellowships/Australian-American-Health-Policy-Fellowships.aspx}
		\item Ian Axford (New Zealand) Fellowships in Public Policy: \url{http://www.commonwealthfund.org/Fellowships/Ian-Axford-Fellowships.aspx}
		\end{enumerate}
	\end{enumerate}
\item American Institute of Aeronautics and Astronautics (AIAA): \vspace{-0.3cm}
	\begin{enumerate} \itemsep -2pt
	\item Federal Government Fellows Program: \vspace{-0.2cm}
		\begin{enumerate} \itemsep -2pt
		\item \url{http://www.aiaa.org/content.cfm?pageid=731}
		\item Shaping U.S. {\bf public policy} concerning aerospace research and the aerospace industry
		\end{enumerate}
	\end{enumerate}
\item IEEE-USA: \vspace{-0.3cm}
	\begin{enumerate} \itemsep -2pt
	\item Congressional Fellowship
	\item Engineering \& Diplomacy (State Department) Fellowship
	\item For IEEE-USA members to support the creation and modification of technology-related public policies
	\item \url{http://ieeeusa.org/policy/govfel/default.asp}
	\end{enumerate}
\item American Mathematical Society: \vspace{-0.3cm}
	\begin{enumerate} \itemsep -2pt
	\item Fellowships and Awards (Policy and Advocacy: Government Relations \& Programs): \vspace{-0.2cm}
		\begin{enumerate} \itemsep -2pt
		\item \url{http://e-math.ams.org/policy/government/fellow-awards/fellow-awards}
		\item Mass Media Fellowships: \url{http://e-math.ams.org/programs/ams-fellowships/media-fellow/massmediafellow}
		\item AMS-AAAS Congressional Fellowship: \url{http://e-math.ams.org/programs/ams-fellowships/ams-aaas/ams-aaas-congressional-fellowship}
		\end{enumerate}
	\end{enumerate}
\item American Association for the Advancement of Science: \vspace{-0.3cm}
	\begin{enumerate} \itemsep -2pt
	\item AAAS Science \& Technology Policy Fellowships: \url{http://fellowships.aaas.org/index.shtml}
	\end{enumerate}
\item --- --- --- --- --- --- --- --- --- --- --- --- --- --- --- --- --- --- --- --- --- --- --- --- --- --- --- --- --- --- ---
\item \colorbox{blue}{\bf Scholarships and Fellowships in Social Science and Humanities}
% Scholarships and Fellowships in Social Science and Humanities
\item United States Institute of Peace (USIP): \vspace{-0.3cm}
	\begin{enumerate} \itemsep -2pt
	\item Jennings Randolph Peace Scholarship Dissertation Program (for Ph.D. students working on topics related to peace, conflict, and international security): \url{http://www.usip.org/grants-fellowships/jennings-randolph-peace-scholarship-dissertation-program}
	\end{enumerate}
\item Library of Congress: \vspace{-0.3cm}
	\begin{enumerate} \itemsep -2pt
	\item Kluge Fellowships: \vspace{-0.2cm}
		\begin{enumerate} \itemsep -2pt
		\item Research in the humanities and social sciences, especially interdisciplinary, cross-cultural or multilingual
		\item Open to scholars worldwide with a Ph.D. or other terminal advanced degree conferred within seven years of the July 15 deadline
		\item \url{http://www.loc.gov/loc/kluge/fellowships/kluge.html}
		\end{enumerate}
	\item J. Franklin Jameson Fellowship Research in American History (junior postdocs): \url{http://www.loc.gov/loc/kluge/fellowships/jameson.html}
	\item Kislak Short Term Fellowship Opportunities in American Studies (students, postdocs, and faculty): \url{http://www.loc.gov/loc/kluge/fellowships/kislakshort.html}
	\item Kislak Fellowship in American Studies (Ph.D. requirement): \url{http://www.loc.gov/loc/kluge/fellowships/kislak.html}
	\end{enumerate}
\item American Historical Association (AHA): \vspace{-0.3cm}
	\begin{enumerate} \itemsep -2pt
	\item AHA Research Grants: \url{http://www.historians.org/prizes/Grants.htm}
	\item Fellowships: \url{http://www.historians.org/prizes/Fellowships.htm}
	\end{enumerate}
\item American Sociological Association: \vspace{-0.3cm}
	\begin{enumerate} \itemsep -2pt
	\item ASA Dissertation Award: \url{http://www.asanet.org/about/awards/dissertation.cfm}
	\end{enumerate}
\item American Psychological Association: \vspace{-0.3cm}
	\begin{enumerate} \itemsep -2pt
	\item Scholarships, Grants, and Awards: \url{http://www.apa.org/about/awards/index.aspx}
	\end{enumerate}
\item American Anthropological Association (AAA): \vspace{-0.3cm}
	\begin{enumerate} \itemsep -2pt
	\item AAA Minority Dissertation Fellowship Program (for minority Ph.D. candidates in anthropology): \url{http://www.aaanet.org/cmtes/minority/Minfellow.cfm}
	\item Margaret Mead Award (for young scholars in anthropology): \url{http://www.aaanet.org/about/Prizes-Awards/AAA-Margaret-Mead-Award.cfm}
	\item COSWA Award: \vspace{-0.2cm}
		\begin{enumerate} \itemsep -2pt
		\item The COSWA Award (formerly the Squeaky Wheel Award), sponsored by the Committee on the Status of Women in Anthropology (COSWA), recognizes individuals who have demonstrated the courage to bring to light and investigate practices in anthropology that are potentially discriminatory to women, or have acted to improve the status of women in anthropology through activities that raise awareness of women's contribution to anthropology or identify barriers to full participation by women in anthropology.
		\item \url{http://www.aaanet.org/about/Prizes-Awards/COSWA-Award.cfm}
		\end{enumerate}
	\item David M. Schneider Award (for Ph.D. students in anthropology): \url{http://www.aaanet.org/about/Prizes-Awards/David-Schneider-Award.cfm}
	\item Links to ``Section Prizes \& Awards'': \url{http://www.aaanet.org/about/Prizes-Awards/section_awards.cfm}
	\item List of national (US) and international ``Grants and Fellowships'': \url{http://www.aaanet.org/profdev/fellowships/}
	\item \url{http://www.aaanet.org/}
	\end{enumerate}
\item National Academy of Social Insurance: \vspace{-0.3cm}
	\begin{enumerate} \itemsep -2pt
	\item John Heinz Dissertation Award (Ph.D. students writing their thesis on the planning and implementation of social insurance): \url{http://www.nasi.org/studentopps/heinz}
	\end{enumerate}
\item National Endowment for the Humanities's Division of Research Programs, grants and fellowship opportunities: \url{http://www.neh.gov/grants/}
\item {\it The Henry Luce Foundation}'s Luce Scholars Program to help US graduates learn more about Asia and Asian culture(s): \url{http://www.hluce.org/lsprogram.aspx}
\item Institute for Humane Studies at George Mason University: \vspace{-0.3cm}
	\begin{enumerate} \itemsep -2pt
	\item Humane Studies Fellowships: \vspace{-0.2cm}
		\begin{enumerate} \itemsep -2pt
		\item \url{http://www.theihs.org/programs/humane-studies-fellowships}
		\item Humane Studies Fellowships are awarded to graduate students and outstanding undergraduates planning academic careers with liberty-advancing research interests.
		\item The fellowships are open to students in a range of fields, such as economics, philosophy, law, political science, anthropology, and literature.
		\end{enumerate}
	\end{enumerate}
\item The Gilder Lehrman Institute of American History: Gilder Lehrman History Scholars \& Gilder Lehrman One-Week Scholars (for sophomores or juniors majoring in American history or American Studies), \url{http://www.gilderlehrman.org/education/hs_program_details.php}
\item Myra Sadker Foundation: \vspace{-0.3cm}
	\begin{enumerate} \itemsep -2pt
	\item \url{http://www.sadker.org/awards.html}
	\item Teacher Award: Designed to promote and support teacher projects (K-12) that help students learn about and respect group differences, promote fairness, and in other ways build upon the values and contributions of Myra Sadker's work. Each project should have a gender dimension.
	\item Student Award: Designed to encourage student ideas, activities and projects (K-12) that promote respect for group differences, fairness, and in other ways build upon the values and contributions of Myra Sadker's work. Each project should have a gender dimension. 
	\item Doctoral Dissertation Award: Designed to promote and support graduate students engaged in educational equity research. Doctoral level dissertations that explore or promote educational equity and fairness based on gender, race, ethnicity, religion, class, sexual orientation, or other such variables will be considered for support. Each dissertation should have a gender dimension.
	\end{enumerate}
\item IREX: \vspace{-0.3cm}
	\begin{enumerate} \itemsep -2pt
	\item Opportunities ``for individuals, organizations, universities, and alumni'': \url{http://www.irex.org/apply}
	\item Edmund S. Muskie Graduate Fellowship Program: \vspace{-0.2cm}
		\begin{enumerate} \itemsep -2pt
		\item : \url{http://www.irex.org/application/edmund-s-muskie-graduate-fellowship-program-application}
		\item ``The Muskie Program is open to graduate students and professionals from Armenia, Azerbaijan, Belarus, Georgia, Kazakhstan, Kyrgyzstan, Moldova, Russia, Tajikistan, Turkmenistan, Ukraine and Uzbekistan for one-year non-degree, one-year degree, or two-year degree study in the United States.''
		\item ``Eligible fields of study for the Muskie Program are: business administration, economics, education, environmental management, international affairs, journalism and mass communication, law, library and information science, public administration, public health, and {\bf public policy}.''
		\end{enumerate}
	\item Legal Education and Development (LEAD) Fellowship: \vspace{-0.2cm}
		\begin{enumerate} \itemsep -2pt
		\item \url{http://www.irex.org/application/legal-education-and-development-lead-fellowship-application}
		\item Legal Education and Development Fellowship Program (LEAD) in Tajikistan
		\item Eligibility: \vspace{-0.1cm}
			\begin{enumerate} \itemsep -1pt
			\item Is a citizen, national, or permanent resident qualified to hold a valid passport issued by Tajikistan;
			\item Is the recipient of an undergraduate degree in law (four- or five-year study) by the time of the application;
			\item Is able to begin the academic exchange program in the United States in the summer of 2011;
			\item Is able to receive and maintain a United States J-1 visa.
			\end{enumerate}
		\end{enumerate}
	\item Community Solutions Program: \vspace{-0.2cm}
		\begin{enumerate} \itemsep -2pt
		\item \url{http://www.irex.org/application/community-solutions-information-applicants}
		\item ``a professional development program for the best and brightest global community leaders working in Transparency \& Accountability, Tolerance/Conflict Resolution, Environmental Issues, and Women's Issues''
		\item ``Competition for the Community Solutions Program is merit-based and open to community leaders, ages 25-38 at the time of application''
		\end{enumerate}
	\item Crimea Undergraduate Exchange Program (Crimea UGRAD) Application: \vspace{-0.2cm}
		\begin{enumerate} \itemsep -2pt
		\item \url{http://www.irex.org/application/crimea-undergraduate-exchange-program-crimea-ugrad-application}
		\item ``The Crimea UGRAD Program is open to undergraduate students from the Autonomous Republic of Crimea for one academic year of non-degree study in a US university or community college.''
		\end{enumerate}
	\end{enumerate}
\item {\it Demos}: \vspace{-0.3cm}
	\begin{enumerate} \itemsep -2pt
	\item The Ed Baker Fellowship in Democratic Values: \vspace{-0.2cm}
		\begin{enumerate} \itemsep -2pt
		\item \url{http://www.demos.org/edbakerfellowship.cfm}
		\item ``Based in our New York offices, Ed Baker Fellows will give voice to strong democratic values within a wide range of potential issues, including voting rights, citizen engagement, immigration policy and civic inclusion, campaign finance reform and money in politics, and media reform, among others.''
		\end{enumerate}
	\item Fellows Program: \vspace{-0.2cm}
		\begin{enumerate} \itemsep -2pt
		\item \url{http://www.demos.org/fellowsapp.cfm}
		\item \url{http://www.demos.org/program.cfm?currentprogramID=5A196E48-3FF4-6C82-50CBCA5825B661BA}
		\item ``The Fellows Program at Demos provides support and community for writers and thinkers with well-defined projects that aim to advance the values at the core of Demos' programs and mission: a robust and inclusive democracy; shared prosperity; strong \& effective public governance; and global interdependence.''
		\end{enumerate}
	\end{enumerate}
\item Research Councils UK (RCUK): \vspace{-0.3cm}
	\begin{enumerate} \itemsep -2pt
	\item Economic and Social Research Council (ESRC): \vspace{-0.2cm}
		\begin{enumerate} \itemsep -2pt
		\item Academic (funding opportunities for students, postdocs, and professors): \url{http://www.esrcsocietytoday.ac.uk/ESRCInfoCentre/index_academic.aspx}
		\item Professorial Fellowships (for leading senior social scientists): \url{http://www.esrcsocietytoday.ac.uk/ESRCInfoCentre/opportunities/professorial/}
		\item Funding opportunities: \vspace{-0.1cm}
			\begin{enumerate} \itemsep -1pt
			\item \url{http://www.esrcsocietytoday.ac.uk/ESRCInfoCentre/index_government.aspx}
			\item \url{http://www.esrcsocietytoday.ac.uk/ESRCInfoCentre/opportunities/}
			\item ESRC Research Funding Guide / ESRC's Funding Rules: \url{http://www.esrcsocietytoday.ac.uk/ESRCInfoCentre/opportunities/research_funding}
			\item Eligibility for Research Council Funding: \url{http://www.esrcsocietytoday.ac.uk/ESRCInfoCentre/opportunities/eligibility}
			\item Current Funding Opportunities: \url{http://www.esrcsocietytoday.ac.uk/ESRCInfoCentre/opportunities/current_funding_opportunities/}
			\item Forthcoming funding opportunities: \url{http://www.esrcsocietytoday.ac.uk/ESRCInfoCentre/opportunities/forthcoming_opportunities/}
			\item Placement Fellows Scheme: \url{http://www.esrcsocietytoday.ac.uk/ESRCInfoCentre/opportunities/placement/}
			\item Professorial Fellowships: \url{http://www.esrcsocietytoday.ac.uk/ESRCInfoCentre/opportunities/professorial/}
			\item Early Career Researchers (including Postdoctoral Fellowships, International Training, and Networking Opportunities): \url{http://www.esrcsocietytoday.ac.uk/ESRCInfoCentre/opportunities/earlycareer/}
			\item Postgraduate and Career Development Opportunities: \url{http://www.esrcsocietytoday.ac.uk/ESRCInfoCentre/opportunities/postgraduate/}
			\item International Funding Opportunities: \url{http://www.esrcsocietytoday.ac.uk/ESRCInfoCentre/opportunities/international/}
			\item Joint Funding Opportunities: \url{http://www.esrcsocietytoday.ac.uk/ESRCInfoCentre/opportunities/jointfunding/}
			\item Annual competitions: \url{http://www.esrcsocietytoday.ac.uk/ESRCInfoCentre/opportunities/annual/index.aspx#3}
			\end{enumerate}
		\end{enumerate}
	\item Arts and Humanities Research Council (AHRC): \vspace{-0.2cm}
		\begin{enumerate} \itemsep -2pt
		\item Funding Opportunities: \vspace{-0.1cm}
			\begin{enumerate} \itemsep -1pt
			\item \url{http://www.ahrc.ac.uk/FundingOpportunities/Pages/default.aspx}
			\item Fellowships: \url{http://www.ahrc.ac.uk/FundingOpportunities/Pages/Fellowships.aspx}
			\item Fellowships - route for early career researchers: \url{http://www.ahrc.ac.uk/FundingOpportunities/Pages/Fellowshipserc.aspx}
			\item Placement Fellowship based in the Department for Culture, Media and Sport (DCMS) - Climate Change: \url{http://www.ahrc.ac.uk/FundingOpportunities/Pages/PlacementFellowshipDCMS-Climatechange.aspx}
			\item Placement Fellowship based in the Department for Culture, Media and Sport (DCMS) - Health and Wellbeing: \url{http://www.ahrc.ac.uk/FundingOpportunities/Pages/PlacementFellowshipDCMShealthandwellbeing.aspx}
			\item Research Grants - route for early career researchers: \url{http://www.ahrc.ac.uk/FundingOpportunities/Pages/RG-EarlyCareers.aspx}
			\item Research Grants - Speculative Research: \url{http://www.ahrc.ac.uk/FundingOpportunities/Pages/RG-SpeculativeResearch.aspx}
			\item Research Grants - Standard Route: \url{http://www.ahrc.ac.uk/FundingOpportunities/Pages/RG-StandardRoute.aspx}
			\item Postgraduate Funding (for Masters and Ph.D. students): \url{http://www.ahrc.ac.uk/FundingOpportunities/Pages/summaryinformationforprospectivepostgraduatestudents.aspx}
			\item Browse Funding Opportunities: \url{http://www.ahrc.ac.uk/FundingOpportunities/Pages/BrowseOpportunities.aspx}
			\end{enumerate}
		\end{enumerate}
	\end{enumerate}
\item World Bank Institute (WBI): \vspace{-0.3cm}
	\begin{enumerate} \itemsep -2pt
	\item Or The World Bank Group
	\item Scholarships: \url{http://wbi.worldbank.org/wbi/scholarships} or \url{http://www.worldbank.org/wbi/scholarships/home.html}
	\end{enumerate}
\item --- --- --- --- --- --- --- --- --- --- --- --- --- --- --- --- --- --- --- --- --- --- --- --- --- --- --- --- --- --- ---
\item \colorbox{blue}{\bf Fellowships in Art and Music}
% Fellowships in Art and Music
\item The Kresge Foundation: \vspace{-0.3cm}
	\begin{enumerate} \itemsep -2pt
	\item \url{http://www.kresge.org/index.php/what/detroit_program/kresge_arts_in_detroit/}
	\item Kresge Artist Fellowships: \vspace{-0.2cm}
		\begin{enumerate} \itemsep -2pt
		\item ``Kresge Artist Fellowships seek to advance the art forms and professional careers of artists from the visual, performing and literary arts as well as elevate the profile of the artistic community and encourage creative expression in the region. Each year, Kresge will provide funding for 18 fellowships of \$25,000 each, which are awarded to artists living and working in metropolitan Detroit.''
		\item ``The fellowships recognize creative vision and commitment to excellence within a wide range of artistic disciplines, including artists who have been classically and academically trained, self taught artists and artists whose art forms have been passed down through cultural and traditional heritage.''
		\item ``Kresge Arts in Detroit is committed to supporting artists from diverse cultural backgrounds at all stages of their professional careers.''
		\item \url{http://kresge.collegeforcreativestudies.edu/}
		\item \url{http://kresge.collegeforcreativestudies.edu/kaf_guidelines.html}
		\item Information Sessions: \url{http://kresge.collegeforcreativestudies.edu/kaf_sessions.html}
		\end{enumerate}
	\item Kresge Eminent Artist Award: \vspace{-0.2cm}
		\begin{enumerate} \itemsep -2pt
		\item ``Kresge Eminent Artist Award recognizes an exceptional artist for his or her professional achievements and contributions to the cultural community, and encourages that individual's pursuit of a chosen art form as well as an ongoing commitment to metropolitan Detroit. Each year, one highly accomplished individual will be presented with the award which includes a \$50,000 prize.''
		\item \url{http://kresge.collegeforcreativestudies.edu/eminent-artist-award.html}
		\end{enumerate}
	\end{enumerate}
\item Guggenheim Fellowships from the {\it John Simon Guggenheim Memorial Foundation}: \url{http://www.gf.org/applicants}
\item The John F. Kennedy Center for the Performing Arts: \vspace{-0.3cm}
	\begin{enumerate} \itemsep -2pt
	\item DeVos Institute of Arts Management at the Kennedy Center: \vspace{-0.2cm}
		\begin{enumerate} \itemsep -2pt
		\item DeVos Institute Programs: \vspace{-0.1cm}
			\begin{enumerate} \itemsep -1pt
			\item Kennedy Center Fellowship Program: \vspace{-0.1cm}
				\begin{itemize} \itemsep -1pt
				\item \url{http://www.kennedy-center.org/education/artsmanagement/fellowships.cfm}
				\item \url{http://www.kennedy-center.org/education/artsmanagement/fellowships/home.html}
				\item ``The Kennedy Center Fellowship Program began in 2001, and provides comprehensive study to 10 arts managers at the Kennedy Center with coursework in strategic planning, marketing, and development; three practical work rotations in Center departments; and a series of professional development seminars. The paid fellowships are full-time and last nine months from September through May.''
				\end{itemize}
			\item DeVos Institute Summer International Fellowship Program at the Kennedy Center: \vspace{-0.1cm}
				\begin{itemize} \itemsep -1pt
				\item \url{http://www.kennedy-center.org/education/artsmanagement/fellowships.cfm}
				\item \url{http://www.kennedy-center.org/education/artsmanagement/international_faq.cfm}
				\item ``The Summer International Fellowship Program provides practical experience to 15 mid-to-high level arts leaders currently working in international nonprofit performing arts organizations. This full-time, four-week intensive program takes place at the Kennedy Center each July; Fellows attend each summer for three consecutive years. While at the Center, the fellows take classes and refine strategic plans for their home organizations.''
				\end{itemize}
			\item U.S. Department of State International Exchange Programs: \vspace{-0.1cm}
				\begin{itemize} \itemsep -1pt
				\item \url{http://www.kennedy-center.org/education/state/}
				\item ``The U.S. Department of State and The Kennedy Center have teamed to produce international exchange opportunities through the Performing Artists Cultural Visitors Program and International Cultural Fellows Mentoring Program.''
				\item Performing Artists Cultural Visitors Program: \url{http://www.kennedy-center.org/education/state/cultural/}
				\item International Cultural Fellows Mentoring Program: \url{http://www.kennedy-center.org/education/state/fellows/}
				\item ``Visitors, comprised of modern and hip-hop dancers, theater technicians/designers/actors, as well as classical and jazz musicians, engage with American colleagues in the creation and performance of their discipline in Washington, D.C. and in another American city.''
				\item ``The Fellows, comprised of arts managers and presenters from outside the United States, attend arts management seminars led by Kennedy Center staff, travel to another American city to study with a mentor organization, and visit New York City to meet with experts in their field.''
				\end{itemize}
			\end{enumerate}
		\end{enumerate}
	\item The National Symphony Orchestra (NSO): \vspace{-0.2cm}
		\begin{enumerate} \itemsep -2pt
		\item National Symphony Orchestra Youth Fellowship Program: \vspace{-0.1cm}
			\begin{itemize} \itemsep -1pt
			\item \url{http://www.kennedy-center.org/nso/nsoed/youthfellowship.cfm}
			\item \url{http://www.kennedy-center.org/explorer/artists/?entity_id=10811&source_type=B}
			\item ``Now in its 30th season, the National Symphony Orchestra Youth Fellowship Program is an orchestral training project for high school musicians.''
			\item ``From its inception in 1980-81 to the present, the program provides Washington metropolitan area high school students with scholarships to study privately with NSO members, as well as opportunities to observe NSO rehearsals; attend concerts; and to participate in seminars, discussions, and master classes with musicians, conductors, and NSO and Kennedy Center management.''
			\item ``There are 20 students in the NSO Youth Fellowship Program for 2009-10.''
			\item ``Participation by ethnic minorities is encouraged.''
			\item ``Priority is given to students entering 10th grade in order to provide as sustained a training as possible.''
			\end{itemize}
		\end{enumerate}
	\end{enumerate}
\item League of American Orchestras: \vspace{-0.3cm}
	\begin{enumerate} \itemsep -2pt
	\item Fellowships: \vspace{-0.2cm}
		\begin{enumerate} \itemsep -2pt
		\item \url{http://www.americanorchestras.org/learning_and_leadership/fellowships.html}
		\item Orchestra Management Fellowship Program: \vspace{-0.1cm}
			\begin{enumerate} \itemsep -1pt
			\item \url{http://www.americanorchestras.org/learning_and_leadership/omfp.html}
			\item ``This year-long, highly competitive program is designed to launch executive careers in orchestra management.''
			\item ``Along with an intense course of study, fellows undertake a series of residencies with orchestras of various sizes across the U.S. receiving invaluable work experience and the support of host orchestra staff, in particular that of the orchestra�s executive director.''
			\item ``Fellows also participate in other League leadership seminars throughout the year and receive a comprehensive overview of the classical music industry.''
			\end{enumerate}
		\item ``The League's Fellowship programs identify and prepare the future leaders of tomorrow, today.''
		\item ``Long-term curricula, developed for conductors, executive directors, and managers looking to advance, provide intensive education, hands-on learning, and valuable networking opportunities.''
		\end{enumerate}
	\end{enumerate}
\item Americans for the Arts: \vspace{-0.3cm}
	\begin{enumerate} \itemsep -2pt
	\item Event scholarships (scholarships to attend events): \url{http://www.artsusa.org/events/scholarships.asp}
	\item \url{http://www.artsusa.org/news/annual_awards/default.asp}
	\item Alene Valkanas State Arts Advocacy Award\url{http://www.artsusa.org/news/annual_awards/alene_valkanas/default.asp}
	\item Arts Education Award (awarded to institutions): \url{http://www.artsusa.org/news/annual_awards/arts_education/default.asp}
	\item Emerging Leader Award: \url{http://www.artsusa.org/news/annual_awards/emerging_leader/default.asp}
	\item Michael Newton Award for United Arts Funds Leadership (management and fundraising): \url{http://www.artsusa.org/news/annual_awards/michael_newton/default.asp}
	\item Selina Roberts Ottum Award (contributions to the field of the arts): \url{http://www.artsusa.org/news/annual_awards/selina_roberts_ottum/default.asp}
	\item United States Urban Arts Federation (USUAF): \vspace{-0.2cm}
		\begin{enumerate} \itemsep -2pt
		\item Ray Hanley Innovation Award: \url{http://www.artsusa.org/networks/usuaf/hanley.asp}
		\end{enumerate}
	\end{enumerate}
\item NEA National Heritage Fellowship (for master folk and traditional artists): \url{http://www.nea.gov/honors/heritage/index.html}
\item NEA Jazz Masters Fellowship (jazz artists): \url{http://www.arts.gov/honors/jazz/index.html}
\item Fellowships for Creative Writers [or NEA Literature Fellowships: Creative Writing]: \url{http://www.nea.gov/grants/apply/Lit/index.html} or \url{http://www.arts.gov/grants/apply/Lit/index.html}
\item Carnegie Investment Bank: Carnegie Art Award (for distinguished artists born or living in the Nordic countries), \url{http://www.carnegie.se/sv/ArtAward/About-Carnegie-Art-Award/}, \url{http://www.carnegie.se/artaward/}, and \url{http://www.carnegie.se/en/about/Operations/Carnegie-Art-Award/}
\item Robert McCann Foundation: \vspace{-0.3cm}
	\begin{enumerate} \itemsep -2pt
	\item Funding for artists and designers ``from all Scottish colleges and art schools'' to: \vspace{-0.2cm}
		\begin{enumerate} \itemsep -2pt
		\item extend their training in an area of specialization; OR
		\item finance a project ``in the craft industries associated with film and television''
		\end{enumerate}
	\item \url{http://robertmccannfoundation.com/how.html}
	\end{enumerate}
\item Alexander von Humboldt-Stiftung/Foundation: \vspace{-0.3cm}
	\begin{enumerate} \itemsep -2pt
	\item Hezekiah Wardwell Fellowship (for musicians or musicologists from Spain): \url{http://www.humboldt-foundation.de/web/wardwell-en.html}
	\end{enumerate}
\item Canada Council for the Arts: \vspace{-0.3cm}
	\begin{enumerate} \itemsep -2pt
	\item Endowments and Prizes: \vspace{-0.2cm}
		\begin{enumerate} \itemsep -2pt
		\item \url{http://www.canadacouncil.ca/prizes/}
		\item Prizes and fellowships for Canadian artists and scholars to recognize their contributions to the arts, humanities, and sciences
		\item Categories of prizes and fellowships: \vspace{-0.1cm}
			\begin{enumerate} \itemsep -1pt
			\item dance
			\item inter-arts
			\item media arts
			\item music
			\item theatre
			\item visual arts
			\item writing and publishing
			\end{enumerate}
		\end{enumerate}
	\item Grant Programs: \url{http://www.canadacouncil.ca/grants/}
	\end{enumerate}
\item Institute for Humane Studies at George Mason University: \vspace{-0.3cm}
	\begin{enumerate} \itemsep -2pt
	\item Film \& Fiction Scholarships: \vspace{-0.2cm}
		\begin{enumerate} \itemsep -2pt
		\item Students pursuing MFAs in a variety of areas are eligible: film directing, production, screenwriting, playwriting, fiction, and literary-nonfiction writing
		\item \url{http://www.theihs.org/node/448}
		\end{enumerate}
	\end{enumerate}
\item --- --- --- --- --- --- --- --- --- --- --- --- --- --- --- --- --- --- --- --- --- --- --- --- --- --- --- --- --- --- ---
\item \colorbox{blue}{\bf Scholarships and Fellowships for Underrepresented Minorities}
% Scholarships and Fellowships for Underrepresented Minorities
\item Lists of scholarships and fellowships for underrepresented minorities: \vspace{-0.3cm}
	\begin{enumerate} \itemsep -2pt
	\item Chris Enstrom, ``Cashing in on Diversity Grants and Scholarships,'' in Graduating Engineer \& Computer Careers. Available at: \url{http://www.graduatingengineer.com/higher-education/20061129/Cashing-in-on-Diversity-Grants-and-Scholarships-}; last accessed on August 25, 2010.
	\end{enumerate}
\item Gates Millennium Scholars (GMS) scholarship (for underrepresented minorities in the US): \url{http://www.gmsp.org/}
\item Society of Women Engineers (SWE): SWE Scholarships and other scholarships, \url{http://societyofwomenengineers.swe.org/index.php?option=com_content&task=view&id=222&Itemid=111}
\item Coalition to Diversify Computing: \url{http://www.cdc-computing.org/scholarships/}
\item IES Abroad (formerly Institute of European Studies / Institute for the International Education of Students): \vspace{-0.3cm}
	\begin{enumerate} \itemsep -2pt
	\item Diversity Abroad: \vspace{-0.2cm}
		\begin{enumerate} \itemsep -2pt
		\item \url{https://www.iesabroad.org/IES/Diversity/diversity.html}
		\item Programs to improve student diversity in study abroad programs
		\item IES Abroad Diversity Scholarships: \vspace{-0.1cm}
			\begin{enumerate} \itemsep -1pt
			\item IES Abroad Merit-Based Scholarship for Under-represented Students: \url{https://www.iesabroad.org/IES/Scholarships_and_Aid/Diversity_Scholarships/diversityScholarship.html}
			\item IES Abroad Merit-Based David Porter Diversity Scholarship (Up to \$5,000!): \url{https://www.iesabroad.org/IES/Scholarships_and_Aid/Merit_Based/davidPorterScholarship.html}
			\item HBCU Scholarships: \url{https://www.iesabroad.org/IES/Scholarships_and_Aid/Diversity_Scholarships/hbcuScholarship.html}
			\item HACU-IES Abroad Merit/Need-Based Scholarship: \url{https://www.iesabroad.org/IES/Scholarships_and_Aid/Diversity_Scholarships/HACUScholarship.html}
			\end{enumerate}
		\end{enumerate}
	\end{enumerate}
\item MassMutual Scholars Program: \vspace{-0.3cm}
	\begin{enumerate} \itemsep -2pt
	\item Applicants must be undergraduates of African American/Black, Asian/Pacific Islander or Hispanic decent in the US.
	\item Reside or plan to attend an institution in one of the following metropolitan areas: \vspace{-0.2cm}
		\begin{enumerate} \itemsep -2pt
		\item Atlanta, GA
		\item Chicago, IL
		\item Central New Jersey
		\item Denver, CO
		\item Houston, TX
		\item Miami, FL
		\item Los Angeles, CA
		\item San Antonio, TX
		\item San Francisco, CA
		\end{enumerate}
	\item Be majoring in business, economics, finance, financial planning, management, marketing or sales.
	\item \url{http://www.hsf.net/massmutual.aspx}
	\item \url{http://www.apiasf.org/scholarship_apiasf_massmutual.html}
	\end{enumerate}
\item {\it NASA}'s Minority University Research and Education Program (MUREP): \vspace{-0.3cm}
	\begin{enumerate} \itemsep -2pt
	\item \url{http://www.nasa.gov/offices/education/programs/national/murep/home/index.html}
	\item \url{http://www.nasa.gov/offices/education/about/murep_overview.html}
	\item Jenkins Pre-doctoral Fellowship Project, JPFP: \url{http://www.nasa.gov/offices/education/programs/descriptions/Jenkins_Predoctoral_Fellowship_Project.html}
	\end{enumerate}
\item UNCF: \vspace{-0.3cm}
	\begin{enumerate} \itemsep -2pt
	\item UNCF Special Programs Corporation: \vspace{-0.2cm}
		\begin{enumerate} \itemsep -2pt
		\item Harriett G. Jenkins Pre-doctoral Fellowship Program (JPFP) for underrepresented minorities pursuing graduate degrees in STEM: \url{http://www.uncfsp.org/spknowledge/default.aspx?page=program.view&areaid=1&contentid=177&typeid=jpfp}
		\item NASA Science and Technology Institute (NSTI) Summer Scholars Program (10-week summer research scholarship): \url{http://www.uncfsp.org/spknowledge/default.aspx?page=program.view&areaid=1&contentid=172&typeid=nstiinternship}
		\item Motivating Undergraduates in Science and Technology (MUST) Program for undergraduates in STEM: \url{http://www.uncfsp.org/spknowledge/default.aspx?page=program.view&areaid=1&contentid=346&typeid=must}
		\item Institute for International {\bf Public Policy} Fellows Program: \url{http://www.uncfsp.org/IIPP}
		\item \url{http://www.uncfsp.org/spknowledge/default.aspx?page=home.default}
		\end{enumerate}
	\item UNCF scholarship resources: \url{http://www.uncf.org/forstudents/scholarship.asp}
	\item UNCF $\cdot$ Merck Science Initiative: scholarships and fellowships: \url{http://umsi.uncf.org/ScholarshipsInternshipsFellowships/tabid/151/Default.aspx}
	\end{enumerate}
\item Hispanic College Fund: \vspace{-0.3cm}
	\begin{enumerate} \itemsep -2pt
	\item Scholarships: \url{http://www.hispanicfund.org/scholarships/} and \url{http://scholarships.hispanicfund.org/applications/}
	\item NASA MUST Scholarship Program: \url{http://www.hispanicfund.org/nasa-must/}
	\item Hispanic Youth Symposium (scholarships are awarded to winners of the art competition, talent competition, and speech competition): \url{http://www.hispanicyouth.org/about-the-program}
	\item \url{http://www.hispanicfund.org/}
	\end{enumerate}
\item Hispanic Heritage Foundation (HHF): \vspace{-0.3cm}
	\begin{enumerate} \itemsep -2pt
	\item Scholarships and Resources: \url{http://www.hispanicheritage.org/youth_int.php?sec=80}
	\item \url{http://www.hispanicheritage.org/}
	\end{enumerate}
\item Hispanic Scholarship Fund (HSF): \vspace{-0.3cm}
	\begin{enumerate} \itemsep -2pt
	\item Scholarship programs for: \vspace{-0.2cm}
		\begin{enumerate} \itemsep -2pt
		\item college students
		\item community college transfer students
		\item high school students
		\item Gates Millennium Scholars
		\item See \url{http://www.hsf.net/innercontent.aspx?id=34}
		\end{enumerate}
	\item \url{http://www.hsf.net/}
	\end{enumerate}
\item League of United Latin American Citizens (LULAC): \vspace{-0.3cm}
	\begin{enumerate} \itemsep -2pt
	\item LULAC National Educational Service Centers, Inc: \vspace{-0.2cm}
		\begin{enumerate} \itemsep -2pt
		\item \url{http://www.lnesc.org/}
		\item LULAC National Scholarship Fund (LNSF): \vspace{-0.1cm}
			\begin{enumerate} \itemsep -1pt
			\item \url{http://www.lulac.org/programs/education/scholarships/}
			\item \url{http://lnesc.org/index.asp?Type=B_BASIC&SEC={3AEDB506-F425-4E58-B9F6-44867E2FD943}}
%http://lnesc.org/index.asp?Type=B_BASIC&SEC={3AEDB506-F425-4E58-B9F6-44867E2FD943}
			\item Applicants must meet the following criteria to be considered for a scholarship: \vspace{-0.1cm}
				\begin{itemize} \itemsep -1pt
				\item Must be a U.S. citizen or legal resident
				\item Must have applied to or be enrolled in a   college, university, or graduate school, including 2-year colleges, or vocational schools that lead to an associate�s degree
				\item A student will not be eligible for a scholarship if he/she is related to a scholarship committee member, the Council President, or an individual contributor to the local funds of the Council
				\end{itemize}
			\item National Scholastic Achievement Awards (for high school seniors entering college, university, or vocational school)
			\item Honors Awards (for high school seniors entering college, university, or vocational school)
			\item General Awards (Need, community involvement, and leadership activities will also be considered)
			\item General Electric Foundation/ LULAC Scholarship program: for underrepresented minorities (US freshmen) entering their sophomore year as majors in Business or Engineering with a cumulative college G.P.A. $\leq$ 3.25/4.0; these students must be enrolled in a 4-year undergraduate program.
			\end{enumerate}
		\end{enumerate}
	\end{enumerate}
\item Hispanic Association of Colleges and Universities (HACU): \vspace{-0.3cm}
	\begin{enumerate} \itemsep -2pt
	\item HACU Student Programs Overview: \vspace{-0.2cm}
		\begin{enumerate} \itemsep -2pt
		\item \url{http://www.hacu.net/hacu/HACU_Student_Programs_EN.asp?SnID=1942709283}
		\item HACU Scholarship Programs: \vspace{-0.1cm}
			\begin{enumerate} \itemsep -1pt
			\item \url{http://www.hacu.net/hacu/Scholarships_EN.asp?SnID=1942709283}
			\item Includes scholarships for students in: \vspace{-0.1cm}
				\begin{itemize} \itemsep -1pt
				\item Accounting
				\item Behavioral Health
				\item Business
				\item Clinical Psychology
				\item Computer Engineering
				\item Computer Science
				\item Dental Technician
				\item Electrical Engineering
				\item Engineering
				\item Food Merchandising
				\item Information Technology
				\item International Business
				\item Management
				\item Marketing
				\item Mass Media
				\item Mental Health
				\item Merchandising
				\item Nursing
				\item Physician Assistant
				\item (Pre) Optometry
				\item (Pre) Dental
				\item (Pre) Medicine
				\item (Pre) Pharmacy
				\item Public Health
				\item Public Relations
				\item Retail Management
				\item Sports Marketing
				\item Technology
				\end{itemize}
			\end{enumerate}
		\item ``D{\'{a}}ndole Alas a Tu {\'{E}}xito/Giving Flight to Your Success'' travel award program (Southwest Airlines' Travel Award Program): \vspace{-0.1cm}
			\begin{enumerate} \itemsep -1pt
			\item For students with financial need who have to across the United States to participate in their undergraduate or graduate degree programs
			\item \url{http://www.hacu.net/hacu/Lanzate_EN.asp?SnID=1942709283}
			\item \url{http://www.hacu.net/hacu/Lanzate1_EN.asp?SnID=1808826658}
			\end{enumerate}
		\item HACU Study Abroad Scholarship Programs: \vspace{-0.1cm}
			\begin{enumerate} \itemsep -1pt
			\item \url{http://www.hacu.net/hacu/Study_Abroad_EN.asp?SnID=1808826658}
			\item HACU-Global Learning Semesters (GLS) Program: Hispanic Study Abroad Scholars: \url{http://www.studyabroadscholars.org/index.html}
			\item HACU-American Institute for Foreign Study (AIFS) Scholarship Program: \url{http://www.aifsabroad.com/scholarships.asp#hacu}
			\item HACU-Institute for the International Education of Students (IES) Scholarship Program: \url{https://www.iesabroad.org/IES/home.html}
			\item Hispanic Study Abroad Scholars program: \url{http://www.studyabroadscholars.org/index.html}
			\end{enumerate}
		\item Scholarship Resource List: \url{http://www.hacu.net/hacu/Scholarship_Resource_List_EN.asp?SnID=1109551622}
%		\item Scholarship Resource List: \url{http://www.hacu.net/hacu/Scholarship_Resource_List_EN.asp?SnID=1942709283}		-- Redundant
		\end{enumerate}
	\end{enumerate}
\item Congressional Hispanic Caucus Institute (CHCI): \vspace{-0.3cm}
	\begin{enumerate} \itemsep -2pt
	\item CHCI Scholarship: \vspace{-0.2cm}
		\begin{enumerate} \itemsep -2pt
		\item \url{http://www.chci.org/scholarships/}
		\item CHCI's scholarship opportunities are afforded to Latino students in the United States who have a history of performing public service-oriented activities in their communities and who demonstrate a desire to continue their civic engagement in the future. There is no GPA or academic major requirement. Students with excellent leadership potential are encouraged to apply.
		\item Scholarship awards are intended to provide assistance with tuition, room and board, textbooks, and other educational expenses associated with college enrollment.
		\item Students continue to receive annual disbursements as long as they maintain good academic standing.
		\item CHCI scholarships provide recipients with a one time scholarship of: \vspace{-0.1cm}
			\begin{enumerate} \itemsep -1pt
			\item \$1,000 community college or AA/AS granting institution
			\item \$2,500 4-year academic institution
			\item \$5,000 graduate-level institution
			\end{enumerate}
		\item Eligibility Criteria: \vspace{-0.1cm}
			\begin{enumerate} \itemsep -1pt
			\item Full-time enrollment in a United States Department of Education accredited community college, four-year university, or graduate/professional program during the period for which scholarship is requested
			\item Demonstrated financial need
			\item Consistent, active participation in public and/or community service activities
			\item Strong writing skills
			\item U.S. citizenship or legal permanent residency
			\end{enumerate}
		\end{enumerate}
	\item CHCI Fellowships: \vspace{-0.2cm}
		\begin{enumerate} \itemsep -2pt
		\item \url{http://www.chci.org/fellowships/}
		\item CHCI {\bf Public Policy} Fellowship: \vspace{-0.1cm}
			\begin{enumerate} \itemsep -1pt
			\item This is a paid Fellowship Program that offers talented Latinos, who have earned a bachelor's degree within two years of the program start date, the opportunity to gain hands-on experience at the national level in public policy.
			\item Fellows have the opportunity to work in congressional offices and federal agencies, depending on their area of interest.  Some past focus areas have included international affairs, economic development, health and education policy, housing, or local government.
			\item Program Dates: August to May (10-month internship)
			\item \url{http://www.chci.org/fellowships/page/chci-public-policy-fellowship}
			\end{enumerate}
		\item CHCI Graduate Fellowship Program: \vspace{-0.1cm}
			\begin{enumerate} \itemsep -1pt
			\item The CHCI Graduate Fellowship Program seeks to enhance participants' leadership abilities, strengthen professional skills and ultimately produce more competent and competitive Latino professionals in underserved {\bf public policy} issue areas.
			\item This paid Fellowship Program offers exceptional Latinos who have earned a graduate degree or higher related to a chosen policy issue area within three years of program start date unparalleled exposure to hands-on experience in public policy.
			\item This program focuses specifically on the areas of: \vspace{-0.1cm}
				\begin{itemize} \itemsep -1pt
				\item Higher Education: CHCI Graduate Higher Education Fellowship, \url{http://www.chci.org/fellowships/page/chci-graduate-higher-education-fellowship}
				\item Secondary Education: CHCI Graduate Secondary Education Fellowship, \url{http://www.chci.org/fellowships/page/chci-graduate-secondary-education-fellowship}
				\item Health: CHCI Graduate Health Fellowship, \url{http://www.chci.org/fellowships/page/chci-graduate-health-fellowship}
				\item Housing: CHCI Graduate Housing Fellowship, \url{http://www.chci.org/fellowships/page/chci-graduate-housing-fellowship}
				\item International Affairs (includes last three months abroad in Mexico): CHCI Graduate International Affairs Fellowship, \url{http://www.chci.org/fellowships/page/chci-graduate-international-affairs-fellowship}
				\item Law: CHCI Graduate Law Fellowship, \url{http://www.chci.org/fellowships/page/chci-graduate-law-fellowship}
				\item STEM (Science, Technology, Engineering and Math): CHCI Graduate STEM Fellowship, \url{http://www.chci.org/fellowships/page/chci-graduate-stem-fellowship}
				\end{itemize}
			\item Program Dates: August to May (10-month internship)
			\item \url{http://www.chci.org/fellowships/page/chci-graduate-fellowship-program}
			\end{enumerate}
		\end{enumerate}
	\end{enumerate}
\item American Indian Graduate Center (AIGC): \vspace{-0.3cm}
	\begin{enumerate} \itemsep -2pt
	\item AIGC scholarships and fellowships: \vspace{-0.2cm}
		\begin{enumerate} \itemsep -2pt
		\item for advanced degree students in art, music, environmental studies, journalism, communications, medicine, dentistry, public health, nursing, or other health-related fields
		\item for members of Wisconsin, New Mexico or Arizona tribes.
		\item \url{http://www.aigc.com/02scholarships/scholarships.htm}
		\item AIGC Fellowship (Graduate) for Native Americans and their descendants seeking advanced degrees: \url{http://www.aigc.com/02scholarships/aigc/fellowship.htm}
		\item Rainer Scholarship (for grad students): \url{http://www.aigc.com/02scholarships/rainer.htm}
		\end{enumerate}
	\item List of resources about scholarships and fellowships: \vspace{-0.2cm}
		\begin{enumerate} \itemsep -2pt
		\item \url{http://www.aigc.com/08otherscholarship/otherscholarships.html}
		\item Scholarships: \url{http://www.aigc.com/08otherscholarship/scholarships.htm}
		\item Fellowships: \url{http://www.aigc.com/08otherscholarship/fellowships.htm}
		\end{enumerate}
	\item Gates Millennium Scholar Program (for individuals seeking basic and advanced degrees): \url{http://www.aigc.com/03gms/gms.htm}
	\end{enumerate}
\item Asian \& Pacific Islander American Scholarship Fund (APIASF) scholarship resources: \url{http://www.apiasf.org/scholarships.html}
\item American Association of University Women: \vspace{-0.3cm}
	\begin{enumerate} \itemsep -2pt
	\item \url{http://www.aauw.org/learn/fellowships_grants/index.cfm}
	\end{enumerate}
\item Sigma Delta Epsilon-Graduate Women in Science (GWIS): \url{http://www.gwis.org/programs.html}
\item Society of Hispanic Professional Engineers (SHPE): \vspace{-0.3cm}
	\begin{enumerate} \itemsep -2pt
	\item Advancing Hispanic Excellence in Technology, Engineering, Math and Science (AHETEMS) Foundation: \url{http://www.ahetems.org/}
	\item AHETEMS Scholarship Program: \url{http://www.ahetems.org/scholarships/}
	\item Graduate \& Young Professional Fellowship Program (encourage young professionals to engage in {\bf public policy}): \url{http://www.ahetems.org/graduate/graduate-young-professional-fellowship-program/}
	\item SHPE/GEM Fellowship (for graduate students in STEM at GEM Member Universities): \url{http://www.ahetems.org/graduate/shpe-gem-graduate-award/}. See \url{http://www.gemfellowship.org/gem-universities/university-members} for a list of GEM member universities.
	\end{enumerate}
\item National Society of Black Engineers (NSBE): \vspace{-0.3cm}
	\begin{enumerate} \itemsep -2pt
	\item Scholarships: \url{http://www.nsbe.org/Programs/Scholarships.aspx}
	\end{enumerate}
\item The Society of Mexican American Engineers and Scientists (MAES): \vspace{-0.3cm}
	\begin{enumerate} \itemsep -2pt
	\item Scholarships \& Awards: \url{http://www.maes-natl.org/index.php?meid=328}
	\item MAES Scholarship Program: \url{http://www.maes-natl.org/index.php?module=ContentExpress&func=display&ceid=518&meid=241}
	\end{enumerate}
\item SACNAS (Society for Advancement of Chicanos and Native Americans in Science): \vspace{-0.3cm}
	\begin{enumerate} \itemsep -2pt
	\item Scholarships: \url{http://www.sacnas.org/webadindex.cfm?webadcategory_id=7}
	\item Fellowships: \url{http://www.sacnas.org/webadIndex.cfm?webadcategory_id=5}
	\end{enumerate}
\item {\it Center for the Advancement of Hispanics in Science and Engineering Education} (CAHSEE): \vspace{-0.3cm}
	\begin{enumerate} \itemsep -2pt
	\item Scholarships: \url{http://www.cahsee.org/6resources/scholarships.asp.htm}
	\end{enumerate}
\item National Consortium for Graduate Degrees for Minorities in Engineering and Science, Inc.: \vspace{-0.3cm}
	\begin{enumerate} \itemsep -2pt
	\item National GEM Consortium: GEM Fellowship, \url{http://www.gemfellowship.org/gem-fellowship/application-requirements}
	\end{enumerate}
\item National Physical Science Consortium (NPSC): \vspace{-0.3cm}
	\begin{enumerate} \itemsep -2pt
	\item NPSC Graduate Fellowship: \url{http://www.npsc.org/}
	\end{enumerate}
\item Finch College Alumnae Association: \vspace{-0.3cm}
	\begin{enumerate} \itemsep -2pt
	\item The Finch College Alumnae Foundation Education Grant: \vspace{-0.2cm}
		\begin{enumerate} \itemsep -2pt
		\item \url{http://www.finchcollege.org/newscholarships.html}
		\item \url{http://www.finchcollege.org/newFinchGrantQandA.html}
		\item ``THE FINCH GRANT, an annual program where four community college women entering a four year college are awarded a grant of \$1500 which can be used toward any needs to completing college.  The selection is determined by a panel of college professors.''
		\end{enumerate}
	\end{enumerate}
\item : \url{}
\item : \url{}
\item : \url{}
\item : \url{}
\item : \url{}
\item \S\ref{phdandpostdocfellowships} has more information concerning scholarships and fellowships in the following areas: \vspace{-0.3cm}
	\begin{enumerate} \itemsep -2pt
	\item electronic design automation (EDA), and related areas of design automation: \vspace{-0.2cm}
		\begin{enumerate} \itemsep -2pt
		\item bio design automation (BDA)
		\item Lab-on-chip design (LoC) automation
		\item MEMS/NEMS design automation
		\end{enumerate}
	\item digital VLSI design
	\item analog and mixed-signal (AMS) VLSI design
	\item computer architecture
	\item parallel computing
	\item concurrent programming
	\item data mining
	\item theoretical computer science
	\end{enumerate}
\item Ph.D. dissertation awards: \vspace{-0.3cm}
	\begin{enumerate} \itemsep -2pt
	\item --- --- --- --- --- --- --- --- --- --- --- --- --- --- --- --- --- --- --- --- --- --- --- --- --- --- --- --- --- --- ---
	\item \colorbox{blue}{\bf Ph.D. Dissertation Awards for Computer Science}
	% Ph.D. Dissertation Awards for Computer Science
	\item ACM Doctoral Dissertation Award: \url{http://awards.acm.org/doctoral_dissertation/}
	\item ACM Outstanding Ph.D. Dissertation Award in Electronic Design Automation: \url{http://www.sigda.org/opda.html}
	\item EDAA Outstanding Dissertation Award (European Design and Automation Association, EDAA): \url{http://www.edaa.com/dissertation_award.html} and \url{http://www.esat.kuleuven.be/micas/EDAA-Award/index.php}
	\item EuroSys Roger Needham PhD Award (in the systems area): \vspace{-0.2cm}
		\begin{enumerate} \itemsep -2pt
		\item Areas in systems include: \vspace{-0.1cm}
			\begin{enumerate} \itemsep -1pt
			\item operating systems
			\item distributed systems
			\item real-time systems
			\item systems aspects of databases
			\item language runtimes
			\item \colorbox{yellow}{\bf embedded systems}
			\item computer networks
			\end{enumerate}
		\item \url{http://www.eurosys.org/phdprize/index.php}
		\end{enumerate}
	\item ACM SIGPLAN Outstanding Doctoral Dissertation Award: \url{http://www.sigplan.org/award-dissertation.htm}
	\item ACM SIGKDD Doctoral Disseration Award (in data mining and knowledge discovery): \url{http://www.sigkdd.org/awards_dissertation.php}
	\item ACM SIGMOD Jim Gray Doctoral Dissertation Award (in the database field): \url{http://www.sigmod.org/sigmod-awards/doctoral-dissertation-award}
	\item Special Interest Group of the ACM on Management Information Systems (SIGMIS): \vspace{-0.2cm}
		\begin{enumerate} \itemsep -2pt
		\item ACM SIGMIS Doctoral Dissertation Award Competition (at the International Conference on Information Systems, ICIS): \url{http://ai.arizona.edu/icis2009/program/dissertation.html} and \url{http://icis2010.aisnet.org/dissertation_award.htm}
		\end{enumerate}
	\item Association for Symbolic Logic: \vspace{-0.2cm}
		\begin{enumerate} \itemsep -2pt
		\item ``The Sacks Prize is awarded for the most outstanding doctoral dissertation in mathematical logic''.
		\item \url{http://www.aslonline.org/Sacks_nominations.html} and \url{http://www.aslonline.org/info-prizes.html}
		\end{enumerate}
	\item European Association for Computer Science Logic (EACSL): \vspace{-0.2cm}
		\begin{enumerate} \itemsep -2pt
		\item Ackermann Award (for outstanding dissertations in Logic in Computer Science): \url{http://www.eacsl.org/} and \url{http://www.eacsl.org/award.html}
		\end{enumerate}
	\item European Coordinating Committee for Artificial Intelligence (ECCAI): \vspace{-0.2cm}
		\begin{enumerate} \itemsep -2pt
		\item 201X Artificial Intelligence Dissertation Award: \url{http://www.eccai.org/diss-award/current.shtml}
		\end{enumerate}
	\item European Conference on Wireless Sensor Networks (EWSN 201X, \url{http://www.nes.uni-due.de/ewsn2011}) and CONET, the Cooperating Objects Network of Excellence: Ph.D. Thesis Award Competition, \url{http://www.cooperating-objects.eu/}
	\item --- --- --- --- --- --- --- --- --- --- --- --- --- --- --- --- --- --- --- --- --- --- --- --- --- --- --- --- --- --- ---
	\item \colorbox{blue}{\bf Ph.D. Dissertation Awards for Mathematics}
	% Ph.D. Dissertation Awards for Mathematics
	\item International Center for Scientific Research (CIRS): \vspace{-0.2cm}
		\begin{enumerate} \itemsep -2pt
		\item E. W. Beth Dissertation Prize (for outstanding dissertations in the fields of Logic, Language and Information): \url{http://www.cirs.net/prix/awards.php?id=481}
		\end{enumerate}
	\item The Association for Operations Management, APICS (Advancing Productivity, Innovation, and Competitive Success): \vspace{-0.2cm}
		\begin{enumerate} \itemsep -2pt
		\item Plossl Doctoral Dissertation Competition: The APICS Educational and Research Foundation, will annually grant one award of \$2,500 for a doctoral dissertation dealing with any topic in operations management. Sample topics include operations strategy, operations planning and control systems, supply chain management, quality management, Six Sigma, facility location, forecasting, just-in-time/lean production systems, and project management. Entrants must be candidates for the doctorate in operations management. The dissertation must be approved by the primary thesis advisor and not more than 50\% completed at time of submission. See \url{http://www.apics.org/Education/ERFoundation/Competitions/plossl.htm}.
		\end{enumerate}
	\item SIAM Richard C. DiPrima Prize: \vspace{-0.2cm}
		\begin{enumerate} \itemsep -2pt
		\item The Richard C. DiPrima Prize is awarded every two years to a junior scientist, based on an outstanding doctoral dissertation in applied mathematics.
		\item \url{http://www.siam.org/prizes/nominations/nom_diprima.php}
		\item \url{http://www.siam.org/prizes/sponsored/diprima.php}
		\end{enumerate}
	\item MOS A.W. Tucker Prize: \vspace{-0.2cm}
		\begin{enumerate} \itemsep -2pt
		\item It is awarded for an outstanding doctoral thesis in any aspect of mathematical optimization.
		\item \url{http://www.mathprog.org/?nav=tucker}
		\end{enumerate}
	\item --- --- --- --- --- --- --- --- --- --- --- --- --- --- --- --- --- --- --- --- --- --- --- --- --- --- --- --- --- --- ---
	\item \colorbox{blue}{\bf Other Ph.D. Dissertation Awards}
	% Other Ph.D. Dissertation Awards
	\item Institute for Operations Research and the Management Sciences (INFORMS): \vspace{-0.2cm}
		\begin{enumerate} \itemsep -2pt
		\item INFORMS George B. Dantzig Dissertation Award: \url{http://www.informs.org/Recognize-Excellence/INFORMS-Prizes-Awards/George-B.-Dantzig-Dissertation-Award}
		\item Best Dissertation Award (Technology Management Section, for Ph.D. theses in technology management): \url{http://www.informs.org/Recognize-Excellence/INFORMS-Community-Prizes-and-Awards2/Technology-Management-Section/Best-Dissertation-Award}
		\item TSL Dissertation Prize (Transportation Science and Logistics Section, for doctoral dissertations in the transportation science and logistics area): \url{http://www.informs.org/Recognize-Excellence/INFORMS-Community-Prizes-and-Awards2/Transportation-Science-and-Logistics-Section/TSL-Dissertation-Prize}
		\item Best Dissertation Award (Telecommunications Section, for Ph.D. theses in telecommunications): \url{http://www.informs.org/Recognize-Excellence/INFORMS-Community-Prizes-and-Awards2/Telecommunications-Section/Best-Dissertation-Award}
		\item Frank M. Bass Dissertation Paper Award (Society for Marketing Science, for the best marketing paper derived from a Ph.D. thesis published in an INFORMS-sponsored journal): \url{http://www.informs.org/Recognize-Excellence/INFORMS-Community-Prizes-and-Awards2/Society-for-Marketing-Science/Frank-M.-Bass-Dissertation-Paper-Award}
		\item SOLA - Air Products Bi-Annual Dissertation Award (Section on Location Analysis, for Ph.D. theses on location related research): \url{http://www.informs.org/Recognize-Excellence/INFORMS-Community-Prizes-and-Awards2/Section-on-Location-Analysis/SOLA-Air-Products-Bi-Annual-Dissertation-Award}
		\end{enumerate}
	\item EURO Doctoral Dissertation Award (EDDA) (in operations research): \url{http://www.euro-online.org/display.php?page=edda1}
	\end{enumerate}
\item Other awards: \vspace{-0.3cm}
	\begin{itemize} \itemsep -2pt
	\item --- --- --- --- --- --- --- --- --- --- --- --- --- --- --- --- --- --- --- --- --- --- --- --- --- --- --- --- --- --- ---
	\item \colorbox{blue}{\bf Awards for Computer Science}
	% Awards for Computer Science
	\item ACM SIGMOD Undergraduate Award: \url{http://www.sigmod.org/sigmod-awards/sigmod-awards#undergraduate}
	\item European Association of Theoretical Computer Science (EATCS): Presburger Award (for young researchers in theoretical computer science), \url{http://www.eatcs.org/index.php/presburger}.
	\item Computer Research Association: \vspace{-0.2cm}
		\begin{enumerate} \itemsep -2pt
		\item Committee on the Status of Women in Computing Research (CRA-W): \vspace{-0.1cm}
			\begin{enumerate} \itemsep -1pt
			\item Borg Early Career Award (BECA): \url{http://www.cra-w.org/borg}
			\end{enumerate}
		\end{enumerate}
	\item European Conference on Wireless Sensor Networks (EWSN 201X, \url{http://www.nes.uni-due.de/ewsn2011}) and CONET, the Cooperating Objects Network of Excellence: Ph.D. Thesis Award Competition, \url{http://www.cooperating-objects.eu/}. ``Cooperating Objects combine the strong functional aspects of embedded systems, pervasive computing and wireless sensor networks. Cooperating objects entities federate themselves into dynamic and loose networks in order to reach a common goal. This common goal will typically be related to sensing or control.''
	\item --- --- --- --- --- --- --- --- --- --- --- --- --- --- --- --- --- --- --- --- --- --- --- --- --- --- --- --- --- --- ---
	\item \colorbox{blue}{\bf Awards for Biomedical Engineering}
	% Awards for Biomedical Engineering
	\item Biomedical Engineering Society (BMES): \vspace{-0.2cm}
		\begin{enumerate} \itemsep -2pt
		\item Rita Schaffer Young Investigator Award (for junior researchers in biomedical engineering): \url{http://www.bmes.org/aws/BMES/pt/sp/awards_investigator}
		\item Graduate and Undergraduate Student Awards: \url{http://www.bmes.org/aws/BMES/pt/sp/awards_student}
		\end{enumerate}
	\item --- --- --- --- --- --- --- --- --- --- --- --- --- --- --- --- --- --- --- --- --- --- --- --- --- --- --- --- --- --- ---
	\item \colorbox{blue}{\bf Awards for Mechanical Engineering}
	% Awards for Mechanical Engineering
	\item American Society of Mechanical Engineers (ASME): \vspace{-0.2cm}
		\begin{enumerate} \itemsep -2pt
		\item Henry Hess Award (authors of research papers who are below 31 years old): \url{http://www.asme.org/Governance/Honors/SocietyAwards/Henry_Hess_Award.cfm}
		\item Pi Tau Sigma Gold Medal (outstanding junior engineers): \url{http://www.asme.org/Governance/Honors/SocietyAwards/Pi_Tau_Sigma_Gold_Medal.cfm}
		\item Marshall B. Peterson Award (researchers in tribology who are below 30 years old): \url{http://www.asme.org/Governance/Honors/SocietyAwards/Marshall_B_Peterson_Award.cfm}
		\item Y.C. Fung Young Investigator Award (for young researchers in bioengineering): \url{http://www.asme.org/Governance/Honors/SocietyAwards/YC_Fung_Young_Investigator.cfm}
		\end{enumerate}
	\item --- --- --- --- --- --- --- --- --- --- --- --- --- --- --- --- --- --- --- --- --- --- --- --- --- --- --- --- --- --- ---
	\item \colorbox{blue}{\bf Awards for Civil Engineering}
	% Awards for Civil Engineering
	\item American Society of Civil Engineers (ASCE): \vspace{-0.3cm}
		\begin{enumerate} \itemsep -2pt
		\item Edmund Friedman Young Engineer Award for Professional Achievement (for junior engineers under the age of 36): \url{http://www.asce.org/AwardsContent.aspx?id=16776}
		\item Committee on Younger Members (CYM) Awards (for junior engineers): \url{http://www.asce.org/Content.aspx?id=11311}
		\item Collingwood Prize (for civil engineering researchers under the age of 35): \url{http://www.asce.org/AwardsContent.aspx?id=15352}
		\end{enumerate}
	\item --- --- --- --- --- --- --- --- --- --- --- --- --- --- --- --- --- --- --- --- --- --- --- --- --- --- --- --- --- --- ---
	\item \colorbox{blue}{\bf Awards for Chemical Engineering}
	% Awards for Chemical Engineering
	\item American Institute of Chemical Engineers (AIChE) awards: \url{http://www.aiche.org/Students/Awards/index.aspx}
	\item --- --- --- --- --- --- --- --- --- --- --- --- --- --- --- --- --- --- --- --- --- --- --- --- --- --- --- --- --- --- ---
	\item \colorbox{blue}{\bf Awards for Systems Engineering}
	% Awards for Systems Engineering
	\item International Council on Systems Engineering (INCOSE) Stevens Doctoral Award (for Promising Research in Systems Engineering and Integration; A.B.D.s / Ph.D. candidates): \url{http://www.incose.org/about/foundation/doctoralaward.aspx}
	\item --- --- --- --- --- --- --- --- --- --- --- --- --- --- --- --- --- --- --- --- --- --- --- --- --- --- --- --- --- --- ---
	\item \colorbox{blue}{\bf Awards for Mathematics, Operations Research, \& Management Sciences}
	% Awards for Mathematics, Operations Research, and Management Sciences
	\item Institute for Operations Research and the Management Sciences (INFORMS): \vspace{-0.2cm}
		\begin{enumerate} \itemsep -2pt
		\item INFORMS Undergraduate Operations Research Prize: \url{http://www.informs.org/Recognize-Excellence/INFORMS-Prizes-Awards/INFORMS-Undergraduate-Operations-Research-Prize}
		\item Optimization Prize for Young Researchers: \url{http://www.informs.org/Recognize-Excellence/INFORMS-Community-Prizes-and-Awards2/Optimization-Society/Optimization-Prize-for-Young-Researchers}
		\item Underrepresented Minorities and Women Honoraria: \url{http://www.informs.org/Recognize-Excellence/INFORMS-Community-Prizes-and-Awards2/Simulation-Society/Underrepresented-Minorities-and-Women-Honoraria}
		\item Best Dissertation Proposal Competition (College on Organization Science, for Ph.D. proposals in organizational science): \url{http://www.informs.org/Recognize-Excellence/INFORMS-Community-Prizes-and-Awards2/College-on-Organization-Science/Best-Dissertation-Proposal-Competition}
		\item ISMS Doctoral Dissertation Proposal Competition (Society for Marketing Science, for Ph.D. proposals in marketing): \url{http://www.informs.org/Recognize-Excellence/INFORMS-Community-Prizes-and-Awards2/Society-for-Marketing-Science/ISMS-Doctoral-Dissertation-Proposal-Competition}
		\end{enumerate}
	\item Alice T. Schafer Mathematics Prize For Excellence in Mathematics by an Undergraduate Woman: \url{http://www.awm-math.org/schaferprize.html}
	\item European Prize in Combinatorics: \vspace{-0.2cm}
		\begin{enumerate} \itemsep -2pt
		\item The prize is established to recognize excellent contributions in Combinatorics by young European researchers (eligibility of EU) not older than 35. 
		\item \url{http://www.math.tu-berlin.de/EuroComb05/prize.html}
		\end{enumerate}
	\item The AMS-MAA-SIAM Frank and Brennie Morgan Prize for Outstanding Research in Mathematics by an Undergraduate Student: \url{http://www.maa.org/awards/morgan.html}; \url{http://www.ams.org/profession/prizes-awards/ams-prizes/morgan-prize}; and \url{http://www.siam.org/prizes/sponsored/morgan.php}
	\item --- --- --- --- --- --- --- --- --- --- --- --- --- --- --- --- --- --- --- --- --- --- --- --- --- --- --- --- --- --- ---
	% Lists of awards
	\item \colorbox{blue}{\bf Lists of awards}: \vspace{-0.2cm}
		\begin{enumerate} \itemsep -2pt
		\item Association for Women in Science: \url{http://www.awis.org/displaycommon.cfm?an=1&subarticlenbr=69}
		\item International Center for Scientific Research (CIRS): \url{http://www.cirs.net/indexenglish.htm}
		\end{enumerate}
	\end{itemize}
\end{enumerate}














%%%%%%%%%%%%%%%%%%%%%%%%%%%%%%%%%%%%%%%%%%%
\section{Funding Nonprofit Organizations}
\label{fundingnonprofitorg}

Funding nonprofit organizations (including colleges and universities): \vspace{-0.3cm}
\begin{enumerate} \itemsep -4pt
\item Alfred P. Sloan Foundation: \vspace{-0.3cm}
	\begin{enumerate} \itemsep -2pt
	\item Major Program Areas: \url{http://www.sloan.org/program/1}
	\item Apply for Grants: \url{http://www.sloan.org/apply}
	\end{enumerate}
\item The Commonwealth Fund: \vspace{-0.3cm}
	\begin{enumerate} \itemsep -2pt
	\item Grants \& Programs: \vspace{-0.2cm}
		\begin{enumerate} \itemsep -2pt
		\item \url{http://www.commonwealthfund.org/Grants-and-Programs.aspx}
		\item ``The Fund supports independent research on health and social issues and makes grants to improve health care practice and policy. We are dedicated to helping people become more informed about their health care and improving care for vulnerable populations such as children, the elderly, low-income families, minorities, and the uninsured.''
		\end{enumerate}
	\end{enumerate}
\item The Heinz Endowments (Howard Heinz Endowment \& Vira I. Heinz Endowment): \vspace{-0.3cm}
	\begin{enumerate} \itemsep -2pt
	\item \url{http://www.heinz.org/grants.aspx}
	\item grant-making programs (for non-profit organizations): \vspace{-0.2cm}
		\begin{enumerate} \itemsep -2pt
		\item Arts \& Culture
		\item Children, Youth \& Families
		\item Education
		\item Environment
		\item Innovation Economy
		\end{enumerate}
	\end{enumerate}
\item Ford Foundation: \vspace{-0.3cm}
	\begin{enumerate} \itemsep -2pt
	\item Grants: \vspace{-0.2cm}
		\begin{enumerate} \itemsep -2pt
		\item \url{http://www.fordfoundation.org/grants/}
		\item Individuals Seeking Fellowships: \vspace{-0.1cm}
			\begin{enumerate} \itemsep -1pt
			\item \url{http://www.fordfoundation.org/grants/individuals-seeking-fellowships}
			\item Ford Foundation Fellowship Programs: \url{http://sites.nationalacademies.org/PGA/FordFellowships/index.htm}
			\item Ford Foundation International Fellowships Program: \url{http://www.fordifp.net/}
			\end{enumerate}
		\item Organizations Seeking Grants: \url{http://www.fordfoundation.org/grants/organizations-seeking-grants}
		\item Other Philanthropic Resources: \url{http://www.fordfoundation.org/grants/other-philanthropic-resources}
		\item Grant Search Results (list of grants): \url{http://www.fordfoundation.org/grants/search}
		\end{enumerate}
	\end{enumerate}
\item The Rockefeller Foundation: \vspace{-0.3cm}
	\begin{enumerate} \itemsep -2pt
	\item Grants \& Grantees: \vspace{-0.2cm}
		\begin{enumerate} \itemsep -2pt
		\item \url{http://www.rockefellerfoundation.org/grants}
		\item What We Fund: \url{http://www.rockefellerfoundation.org/grants/what-we-fund}
		\item Resources for Grantseekers: Links to other Philanthropic Sources, \url{http://www.rockefellerfoundation.org/grants/resources-grantseekers}
		\end{enumerate}
	\end{enumerate}
\item Carnegie Corporation of New York: \vspace{-0.3cm}
	\begin{enumerate} \itemsep -2pt
	\item Grantseekers: \vspace{-0.2cm}
		\begin{enumerate} \itemsep -2pt
		\item \url{http://carnegie.org/grants/grantseekers/}
		\item What we fund: \url{http://carnegie.org/grants/grantseekers/what-we-fund/}
		\item What we don't fund: \url{http://carnegie.org/grants/grantseekers/what-we-dont-fund/}
		\end{enumerate}
		\item Grants database: \url{http://carnegie.org/grants/grants-database/} and \url{http://carnegie.org/grants/}
		\item (Past) individual foundation grants: \url{http://carnegie.org/publications/carnegie-reporter/single/view/article/item/221/}
	\end{enumerate}
\item The Kresge Foundation: \vspace{-0.3cm}
	\begin{enumerate} \itemsep -2pt
	\item fields of interest: \vspace{-0.2cm}
		\begin{enumerate} \itemsep -2pt
		\item health,
		\item the environment,
		\item community development,
		\item arts and culture,
		\item education, and
		\item human services
		\end{enumerate}
	\item Values Criteria (for grantmaking): \url{http://www.kresge.org/index.php/who/our_values_criteria/}
	\item funding methods: \vspace{-0.2cm}
		\begin{enumerate} \itemsep -2pt
		\item \url{http://www.kresge.org/index.php/how/index/}
		\item \url{http://www.kresge.org/index.php/our_funding_methods/index/}
		\end{enumerate}
	\item Challenge Grant: \vspace{-0.2cm}
		\begin{enumerate} \itemsep -2pt
		\item \url{http://www.kresge.org/index.php/our_funding_methods/challenge_grant_program/}
		\item ``The Kresge Foundation awards facilities capital as a challenge grant to help nonprofit organizations build their base of private financial support as they conduct capital campaigns to build or renovate their facilities.''
		\item ``Facilities capital challenge grants are awarded to organizations that cater specifically to the needs of poor, disadvantaged and disenfranchised in six program areas: Health Program, the Environment Program, Arts and Culture Program, Education Program, Human Services Program, and Community Development / Detroit Program.''
		\item ``Most challenge grant awards are made to U.S.-based organizations. On rare occasions, we award challenge grants to international organizations undertaking exceptional projects that align with the strategic objectives of a given program and advance Kresge's values.''
		\end{enumerate}
	\item Detroit Program: \vspace{-0.2cm}
		\begin{enumerate} \itemsep -2pt
		\item Kresge Arts Support: \url{http://www.kresge.org/index.php/what/detroit_program/kresge_arts_support/}
		\item Kresge Arts in Detroit: \url{http://www.kresge.org/index.php/what/detroit_program/kresge_arts_in_detroit/}
		\end{enumerate}
	\item Our Grants: \vspace{-0.2cm}
		\begin{enumerate} \itemsep -2pt
		\item \url{http://www.kresge.org/index.php/our_grants/index/}
		\item grants database: \url{http://maps.foundationcenter.org/grantmakers/index.php?gmkey=KRES002}
		\item Arts and Community Building: \vspace{-0.1cm}
			\begin{enumerate} \itemsep -1pt
			\item \url{http://www.kresge.org/index.php/what/arts_and_culture/arts_and_community_building#Community Arts}
			\item ``Cultural institutions and artists animate our communities, bring disparate people together to share common experiences, and help us imagine a better future. As the demographics of our communities become more diverse, artists and cultural institutions help us bridge differences and build cross-cultural understanding. As our economy struggles, creative enterprises and creative sector leaders offer hope for community renewal and new job development.''
			\item two pilot initiatives: College/Arts initiative, and the Community Arts initiative
			\item ``The pilot cities [for the Community Arts initiative] include Baltimore, Maryland; Birmingham, Alabama; Detroit, Michigan; St. Louis, Missouri; and Tucson, Arizona.''
			\item ``Grants for Arts and Community Building are by invitation only.''
			\end{enumerate}
		\end{enumerate}
	\end{enumerate}
\item New York Women's Foundation: \vspace{-0.3cm}
	\begin{enumerate} \itemsep -2pt
	\item Grant Information and Application: \vspace{-0.2cm}
		\begin{enumerate} \itemsep -2pt
		\item \url{http://www.nywf.org/grant.html}
		\item focus areas: \vspace{-0.1cm}
			\begin{enumerate} \itemsep -1pt
			\item Anti-Violence and Safety
			\item Economic Security
			\item Health, Sexual Rights and Reproductive Justice
			\end{enumerate}
		\item ``Grants usually range from \$50,000 to a maximum of \$70,000 [that last for a year, and can be renewed up to 5 years].''
		\end{enumerate}
	\end{enumerate}
\item The Foundation Center: \vspace{-0.3cm}
	\begin{enumerate} \itemsep -2pt
	\item Grantseekers: \url{http://foundationcenter.org/getstarted/}
	\item Find funders: \url{http://foundationcenter.org/findfunders/}
	\item GrantSpace$^{\rm SM}$: \vspace{-0.2cm}
		\begin{enumerate} \itemsep -2pt
		\item \url{http://grantspace.org/}
		\item ``GrantSpace$^{\rm SM}$ will help you gain the knowledge and skills you need to get grants, manage your nonprofit, and improve your community.''
		\item ``Established in 1956 and today supported by close to 550 foundations, the Foundation Center is a national nonprofit service organization recognized as the nation�s leading authority on organized philanthropy, connecting nonprofits and the grantmakers supporting them to tools they can use and information they can trust. Its audiences include grantseekers, grantmakers, researchers, policymakers, the media, and the general public. The Center maintains the most comprehensive database on U.S. grantmakers and their grants; issues a wide variety of print, electronic, and online information resources; conducts and publishes research on trends in foundation growth, giving, and practice; and offers an array of free and affordable educational programs.''
		\item Resources for Non-U.S. Grantseekers: \url{http://grantspace.org/Tools/Knowledge-Base/Resources-for-Non-U.S.-Grantseekers}
		\item Resources for Individual Grantseekers: \vspace{-0.1cm}
			\begin{enumerate} \itemsep -1pt
			\item \url{http://grantspace.org/Tools/Knowledge-Base/Individual-Grantseekers}
			\item \url{http://gtionline.foundationcenter.org/}
			\item General: \url{http://grantspace.org/Tools/Knowledge-Base/Individual-Grantseekers/General}
			\item Artists: \url{http://grantspace.org/Tools/Knowledge-Base/Individual-Grantseekers/Artists}
			\item Students: \url{http://grantspace.org/Tools/Knowledge-Base/Individual-Grantseekers/Students}
			\item Fiscal Sponsorship: \url{http://grantspace.org/Tools/Knowledge-Base/Individual-Grantseekers/Fiscal-Sponsorship}
			\item For-Profit Enterprises: \url{http://grantspace.org/Tools/Knowledge-Base/Individual-Grantseekers/For-Profit-Enterprises}
			\end{enumerate}
		\end{enumerate}
	\end{enumerate}
\item The Lemelson Foundation: \vspace{-0.3cm}
	\begin{enumerate} \itemsep -2pt
	\item \url{http://web.mit.edu/invent/w-foundation.html}
	\item Programs \& Grants: \url{http://www.lemelson.org/programs-grants}
	\item Grantmaking: \url{http://www.lemelson.org/grantmaking}
	\end{enumerate}
\item Partnership for Higher Education in Africa (PHEA): \vspace{-0.3cm}
	\begin{enumerate} \itemsep -2pt
	\item \url{http://www.foundation-partnership.org/} and \url{http://www.foundation-partnership.org/index.php?id=1}
	\item Grants Database: \url{http://www.foundation-partnership.org/index.php?id=2}
	\item Partnership Publications: \url{http://www.foundation-partnership.org/index.php?id=3}
	\end{enumerate}
\item Smithsonian Institution: \vspace{-0.3cm}
	\begin{enumerate} \itemsep -2pt
	\item Smithsonian Institution Traveling Exhibition Service (SITES): \vspace{-0.2cm}
		\begin{enumerate} \itemsep -2pt
		\item Smithsonian Community Grant program (supported by MetLife Foundation): \vspace{-0.1cm}
			\begin{enumerate} \itemsep -1pt
			\item \url{http://www.sites.si.edu/funding/grant2.htm}
			\item ``This program seeks to deepen connections between SITES' host venues and their communities by encouraging exhibitors to engage their local audiences in new and exciting ways while creating broader access to our exhibitions.''
			\item ``Under this new program, eligible SITES exhibitors may apply for up to \$5,000 for expenses related to public, educational programming produced in conjunction with a SITES exhibit. Exhibitors may choose to enhance current program offerings or to create a new program especially suited to the topic of the exhibition.''
			\end{enumerate}
		\end{enumerate}
	\end{enumerate}
\end{enumerate}
















%%%%%%%%%%%%%%%%%%%%%%%%%%%%%%%%%%%%%%%%%%%
\section{Technology-Related Public Policy}
\label{techpublicpolicy}

Resources for engagement in creating technology-related public policy: \vspace{-0.3cm}
\begin{enumerate} \itemsep -4pt
\item Yale Journal of Law \& Technology (YJOLT): \vspace{-0.3cm}
	\begin{enumerate} \itemsep -2pt
	\item \url{http://www.yjolt.org/}
	\item \url{http://wingenroth.org/}
	\end{enumerate}
\item ACM Public Policy Office: \vspace{-0.3cm}
	\begin{enumerate} \itemsep -2pt
	\item It represents ACM and its US Public Policy Council (USACM) on information technology policy issues that impact the computing field.
	\item It seeks to educate policymakers and the public about policies that will that foster innovations in computing and related disciplines in ways that benefit society.
	\item It also informs ACM's members and the public about policy developments through its weblog, Washington Update newsletter and articles in ACM publications.
	\item ACM US Public Policy Council (USACM): \url{http://usacm.acm.org/}
	\item ACM Committee on Computers and Public Policy (CCPP): \url{http://www.acm.org/public-policy/acm-committee-on-computers-and-public-policy}
	\item \url{http://www.acm.org/public-policy}
	\end{enumerate}
\item IEEE: \vspace{-0.3cm}
	\begin{enumerate} \itemsep -2pt
	\item IEEE-USA: \url{http://www.ieeeusa.org/policy/default.asp}
	\item Smart Grids: \url{http://smartgrid.ieee.org/public-policy}
	\end{enumerate}
\item Computing Community Consortium (CCC): \url{http://www.cra.org/ccc/}
\item Computing Research Association (CRA): \vspace{-0.3cm}
	\begin{enumerate} \itemsep -2pt
	\item \url{http://www.cra.org/}
	\item CRA Government Affairs: \url{http://www.cra.org/govaffairs/index.php}
	\end{enumerate}
\item EngineeringPolicy.org: \url{http://www.engineeringpolicy.org/}
\item Congressional Bi-Partisan Robotics Caucus: \url{http://www.roboticscaucus.org/}
\item Advisory Committee for the Congressional Research and Development $[$R\&D$]$ Caucus: \url{http://www.researchcaucus.org/}
\item {\it National Academies Press} (NAP), from the (US) {\it National Academies}: \url{http://www.nap.edu/}
\item {\it Coalition to Diversify Computing}: \url{http://www.cdc-computing.org/}
\item American Institute of Aeronautics and Astronautics (AIAA): \vspace{-0.3cm}
	\begin{enumerate} \itemsep -2pt
	\item \url{http://www.aiaa.org/content.cfm?pageid=7}
	\end{enumerate}
\item : \url{}
\item : \url{}
\item : \url{}
\item : \url{}
\item : \url{}
\item : \url{}
\item : \url{}
\item : \url{}
\end{enumerate}






%%%%%%%%%%%%%%%%%%%%%%%%%%%%%%%%%%%%%%%%%%%
\section{Feminist Outreach}
\label{feministoutreach}

Feminist outreach: \vspace{-0.3cm}
\begin{enumerate} \itemsep -4pt
\item Myra Sadker Foundation: \vspace{-0.3cm}
	\begin{enumerate} \itemsep -2pt
	\item $100+$ Ideas to Promote Gender Equity in Schools and Beyond: \url{http://www.sadker.org/100ideas.html}
	\item Gender Equity Activities: \url{http://www.sadker.org/WhatYouCanDo.html}
	\item Gender Equity Activities for Concerned Citizens: \url{http://www.sadker.org/GenderEquity-citizens.html}
	\item Gender Equity Activities for Families: \url{http://www.sadker.org/GenderEquity-family.html}
	\item Gender Equity Activities for Teachers: \vspace{-0.2cm}
		\begin{enumerate} \itemsep -2pt
		\item Early Childhood: \url{http://www.sadker.org/GenderEquity-teacher1.html}
		\item Primary Grades: \url{http://www.sadker.org/GenderEquity-teacher2.html}
		\item Upper Elementary: \url{http://www.sadker.org/GenderEquity-teacher3.html}
		\item Middle and High School: \url{http://www.sadker.org/GenderEquity-teacher4.html}
		\end{enumerate}
	\item Resources for feminism and links to web pages of feminist organizations: \url{http://www.sadker.org/ReadsLinks.html}
	\end{enumerate}
\item Feminist student organizations at colleges and universities: \vspace{-0.3cm}
	\begin{enumerate} \itemsep -2pt
	\item For example, at the University of Southern California, the organizations associated with feminist causes are: \vspace{-0.2cm}
		\begin{enumerate} \itemsep -2pt
		\item {\it USC Center for Women \& Men}: \url{http://www.usc.edu/student-affairs/cwm/links.html}
		\item {\it USC Women's Student Assembly}: \url{http://www-scf.usc.edu/~wsausc/Welcome.html}
		\end{enumerate}
	\end{enumerate}
\item International Women's Day: \url{http://www.internationalwomensday.com/}
\item Gender Across Borders: \vspace{-0.3cm}
	\begin{enumerate} \itemsep -2pt
	\item Feminism Resources: \url{http://www.genderacrossborders.com/feminist-resources/}
	\end{enumerate}
\item {\it V-Day}: \vspace{-0.3cm}
	\begin{enumerate} \itemsep -2pt
	\item \url{http://www.vday.org/}
	\item Organization that helps women plan and organize events to bring awareness about sexual assault, and what we can do to reduce sexual assault.
	\end{enumerate}
\item {\it Take Back The Night}: \vspace{-0.3cm}
	\begin{enumerate} \itemsep -2pt
	\item \url{http://www.takebackthenight.org/}
	\item Organization that helps women plan and organize events to bring awareness about sexual assault, and what we can do to reduce sexual assault. It also encourages sexual assault survivors to speak out about their sexual assaults, so that they would shame their perpetrators and let other women (and men) know that they is nothing to be ashamed of as sexual assault survivors. This is because the faults lie 100\% with the perpetrators, and not with the survivors.
	\end{enumerate}
\item {\it United Nations Development Fund for Women} (UNIFEM): \vspace{-0.3cm}
	\begin{enumerate} \itemsep -2pt
	\item \url{http://www.unifem.org/}
	\item Organization that addresses many challenges faced by girls and women.
	\end{enumerate}
\item {\it National Organization for Women}: \vspace{-0.3cm}
	\begin{enumerate} \itemsep -2pt
	\item \url{http://www.now.org/}
	\item Feminist organization in the US.
	\end{enumerate}
\item {\it A Woman's Nation}: \vspace{-0.3cm}
	\begin{enumerate} \itemsep -2pt
	\item \url{http://www.shriverreport.com/awn/}
	\item \url{http://awomansnation.com} or \url{http://www.shriverreport.com/}
	\end{enumerate}
\item {\it Peace Over Violence} is a non-profit, feminist, multicultural, volunteer organization dedicated to a building healthy relationships, families and communities free from sexual, domestic and interpersonal violence: \url{http://peaceoverviolence.org/}
\item SoulSpeakOut: \url{http://www.soulspeakout.org/resources/}
\item {\it Haven Hills}: \url{http://havenhills.org/}
%\item MaleSurvivor: \url{http://www.malesurvivor.org/}
\end{enumerate}












%%%%%%%%%%%%%%%%%%%%%%%%%%%%%%%%%%%%%%%%%%%
\section{Outreach: Professional Organizations}
\label{outreachproorgs}

Professional organizations: \vspace{-0.3cm}
\begin{enumerate} \itemsep -4pt
\item --- --- --- --- --- --- --- --- --- --- --- --- --- --- --- --- --- --- --- --- --- --- --- --- --- --- --- --- --- --- ---
\item \colorbox{blue}{\bf Professional Organizations for the Performance, Literary, and Visual Arts}
% Professional Organizations for the Performance, Literary, and Visual Arts
\item Americans for the Arts: \vspace{-0.3cm}
	\begin{enumerate} \itemsep -2pt
	\item \url{http://www.americansforthearts.org/get_involved/membership/default.asp}
	\item \url{http://www.artsusa.org/get_involved/membership/default.asp}
	\item Provides membership for organizations and individuals
	\item Individual membership are available for: \vspace{-0.2cm}
		\begin{enumerate} \itemsep -2pt
		\item Students
		\item Entrepreneurs (e.g., people in art management)
		\item Innovators
		\item Colleagues (artists)
		\end{enumerate}
	\item Americans for the Arts {\bf Emerging Leader Program}: \vspace{-0.2cm}
		\begin{enumerate} \itemsep -2pt
		\item \url{http://www.artsusa.org/networks/emerging_leaders/resources/default.asp}
		\item Has various resources for professional development, including mentoring
		\end{enumerate}
	\item Advocacy ({\bf public policy}): \url{http://www.artsusa.org/get_involved/advocate.asp}
	\end{enumerate}
\item --- --- --- --- --- --- --- --- --- --- --- --- --- --- --- --- --- --- --- --- --- --- --- --- --- --- --- --- --- --- ---
\item \colorbox{blue}{\bf Professional Organizations for the Musical Artists}
% Professional Organizations for the Musical Artists
\item The Recording Academy: \url{http://www.grammy365.com/join/membership-types}
\end{enumerate}













%%%%%%%%%%%%%%%%%%%%%%%%%%%%%%%%%%%%%%%%%%%
\section{Other Outreach}
\label{otheroutreach}

Other outreach: \vspace{-0.3cm}
\begin{enumerate} \itemsep -4pt
\item The Joy McCann Foundation: \vspace{-0.3cm}
	\begin{enumerate} \itemsep -2pt
	\item The Joy McCann Professorships in Law: \url{http://www.mccannfoundation.org/law.htm}
	\end{enumerate}
\item National Academy of Sciences: \vspace{-0.3cm}
	\begin{enumerate} \itemsep -2pt
	\item {\it Science \& Entertainment Exchange} program: \vspace{-0.2cm}
		\begin{enumerate} \itemsep -2pt
		\item \url{http://www.scienceandentertainmentexchange.org/}
		\item Provide science and engineering knowledge to help professionals in the entertainment industry create engaging storylines involving science and technology.
		\end{enumerate}
	\end{enumerate}
\item U.S. Department of State: \vspace{-0.3cm}
	\begin{enumerate} \itemsep -2pt
	\item Bureau of Educational and Cultural Affairs: \vspace{-0.2cm}
		\begin{enumerate} \itemsep -2pt
		\item Programs: \url{http://exchanges.state.gov/jexchanges/programs.html}
		\item Fulbright Classroom Teacher Exchange Program: \vspace{-0.1cm}
			\begin{enumerate} \itemsep -1pt
			\item \url{http://exchanges.state.gov/globalexchanges/fulbright-teacher-exchange-program.html}
			\item ``The Fulbright Classroom Teacher Exchange provides opportunities for primary and secondary teachers to exchange positions with colleagues in other countries. The participants contribute to mutual understanding by bringing international knowledge and perspectives to the U.S. and foreign classrooms, schools and communities. Full-time U.S. teachers can take part in either a year-long or semester-long direct exchange with a counterpart in another country.''
			\end{enumerate}
		\item FORTUNE/U.S. State Department Global Women's Mentoring Partnership: \vspace{-0.1cm}
			\begin{enumerate} \itemsep -1pt
			\item \url{http://exchanges.state.gov/citizens/professionals/fortunepartnership.html}
			\item ``This public-private partnership places talented, emerging women leaders from all over the world in mentoring programs with FORTUNE's Most Powerful Women Leaders.''
			\end{enumerate}
		\item Edward R. Murrow Program for Journalists: \vspace{-0.1cm}
			\begin{enumerate} \itemsep -1pt
			\item \url{http://exchanges.state.gov/ivlp/murrow.html}
			\item ``The Edward R. Murrow Program for Journalists invites rising international journalists to travel to the United States and examine journalistic principles and practices.''
			\end{enumerate}
		\item International Visitor Leadership Program: \vspace{-0.1cm}
			\begin{enumerate} \itemsep -1pt
			\item \url{http://exchanges.state.gov/ivlp/ivlp.html}
			\item ``These visits reflect the International Visitors' professional interests and support the foreign policy goals of the United States.''
			\item ``International Visitors are current or emerging leaders in government, politics, the media, education, the arts, business and other key fields.''
			\item ``International Visitors travel to the U.S. for carefully designed programs that reflect their professional interests and U.S. foreign policy goals. They travel in a variety of thematic programs, either individually or in groups, for up to three weeks. While in the U.S., International Visitors typically visit Washington, DC and three additional towns or cities that highlight the tremendous diversity of the U.S. They attend professional appointments with their American counterparts, learn about the U.S. system of government at the national, state and local levels, visit American schools, and experience American culture and social life.''
			\item ``There is no application for this program. International Visitors are selected and nominated annually by American Foreign Service Officers at U.S. Embassies around the world.''
			\end{enumerate}
		\item Au Pair: \vspace{-0.1cm}
			\begin{enumerate} \itemsep -1pt
			\item \url{http://exchanges.state.gov/jexchanges/programs/aupair.html}
			\item ``Through the Au Pair program, foreign nationals between 18 and 26 years of age participate in the home life of a host family. Au pairs provide limited childcare services for up to 12 months. An extension of 6, 9, or 12 months may be granted in certain cases.''
			\end{enumerate}
		\item Summer Work Travel: \vspace{-0.1cm}
			\begin{enumerate} \itemsep -1pt
			\item \url{http://exchanges.state.gov/jexchanges/programs/swt.html}
			\item ``In the summer work travel program, post-secondary students may enter the United States to work and travel during their summer vacation. Participants can be admitted to the program more than once. The maximum length of the program is four months.''
			\end{enumerate}
		\item Internship: \vspace{-0.1cm}
			\begin{enumerate} \itemsep -1pt
			\item \url{http://exchanges.state.gov/jexchanges/programs/intern.html}
			\item ``Internship programs are designed to allow foreign professionals to come to the United States to gain exposure to U.S. culture and to receive training in U.S. business practices in their chosen occupational field.  The maximum duration of an internship in any occupational field is 12 months. Upon completion of their exchange programs, participants are expected to return to their home countries.''
			\end{enumerate}
		\item Professional Exchanges Division: \vspace{-0.1cm}
			\begin{enumerate} \itemsep -1pt
			\item \url{http://exchanges.state.gov/citizens/profs.html}
			\item ``The Professional Exchanges division provides grants to U.S. nonprofit organizations to carry out exchange programs that support the professional development of foreign participants. The purpose of each exchange program is to engage with foreign leaders in critical professions, to demonstrate respect for foreign cultures, and to promote mutual understanding between the people of the United States and other countries.''
			\item ``Professional exchanges typically last several years and include internships, study tours or workshops in the United States and in the host country. Participants come from a variety of professions including education administrators, public servants, journalists, labor union officials, entrepreneurs, environmental leaders, jurists, lawyers, and civic leaders.''
			\end{enumerate}
		\end{enumerate}
	\end{enumerate}
\item Teach For All: \url{http://teachforallnetwork.org/}
\item --- --- --- --- --- --- --- --- --- --- --- --- --- --- --- --- --- --- --- --- --- --- --- --- --- --- --- --- --- --- ---
\item \colorbox{blue}{\bf Resources for Artists and Musicians}
% Resources for Artists and Musicians
\item League of American Orchestras and the Association of Performing Arts Presenters: \vspace{-0.3cm}
	\begin{enumerate} \itemsep -2pt
	\item {\it ArtistsfromAbroad.org}: \vspace{-0.2cm}
		\begin{enumerate} \itemsep -2pt
		\item \url{http://www.artistsfromabroad.org/}
		\item ``{\it ArtistsfromAbroad.org} features complete and up-to-date guidance on the visa process and tax treatment for foreign guest artists.''
		\end{enumerate}
	\end{enumerate}
\item Young Concert Artists, Inc. \vspace{-0.3cm}
	\begin{enumerate} \itemsep -2pt
	\item Composer Program (for American composers between 20 and 26 years of age): \url{http://www.yca.org/auditions/}
	\end{enumerate}
\item The John F. Kennedy Center for the Performing Arts: \vspace{-0.3cm}
	\begin{enumerate} \itemsep -2pt
	\item Mary Lou Williams Women in Jazz Emerging Artist Workshop: \vspace{-0.2cm}
		\begin{enumerate} \itemsep -2pt
		\item \url{http://www.kennedy-center.org/programs/jazz/womeninjazz/competition.html}
		\item ``The workshop provides female jazz artists ages 18 to 35 with an opportunity to explore and develop their artistry under the guidance of leading jazz artists and instructors. Each year, the workshop will focus on a specific instrument.''
		\item ``The 2011 Mary Lou Williams Women in Jazz Emerging Artist Workshop is open to advanced female jazz pianists who plan to pursue jazz performance as a career. Eligibility is exclusive to pianists who will be 18-35 years old on May 18, 2011 and have never recorded or been contracted to record as a leader or co-leader on a major label at the time of application. All applicants must be proficient in English.''
		\end{enumerate}
	\end{enumerate}
\item Grantmakers in the Arts (GIA): \vspace{-0.3cm}
	\begin{enumerate} \itemsep -2pt
	\item ``The mission of Grantmakers in the Arts (GIA) is to provide leadership and service to advance the use of philanthropic resources on behalf of arts and culture.''
	\item Arts Funding Topics: \url{http://www.giarts.org/arts-funding-topics}
	\end{enumerate}
\item The Dana Foundation: \vspace{-0.3cm}
	\begin{enumerate} \itemsep -2pt
	\item Arts Education program: \vspace{-0.2cm}
		\begin{enumerate} \itemsep -2pt
		\item Arts Education Grants: \vspace{-0.1cm}
			\begin{enumerate} \itemsep -1pt
			\item \url{http://www.dana.org/grants/BrowseArtsGrants.aspx}
			\item ``In 2001, The Dana Foundation created the Arts Education program with a sole focus of providing grants to support professional development for teaching artists and in-school arts specialists. The first several years of grants were to  programs in New York City, Washington, DC, Los Angeles and to organizations with a 50 mile radius of the three.''
			\item ``The Rural Initiative launched in 2006 with 6 grants awarded to organizations providing professional development in rural areas of the United States.''
			\end{enumerate}
		\end{enumerate}
	\end{enumerate}
\item writing/poetry contests: \vspace{-0.3cm}
	\begin{enumerate} \itemsep -2pt
	\item International 3-Day Novel Contest: \url{http://www.3daynovel.com/about/?contest}
	\end{enumerate}
\end{enumerate}




%%%%%%%%%%%%%%%%%%%%%%%%%%%%%%%%%%%%%%%%%%%
\section{Christian Colleges and Universities}
\label{christianunis}

Christian colleges and universities: \vspace{-0.3cm}
\begin{enumerate} \itemsep -4pt
\item List of Christian colleges and universities: \vspace{-0.3cm}
	\begin{enumerate} \itemsep -2pt
	\item Council for Christian Colleges and Universities (CCCU): \vspace{-0.2cm}
		\begin{enumerate} \itemsep -2pt
		\item \url{http://en.wikipedia.org/wiki/Council_for_Christian_Colleges_and_Universities}
		\item \url{http://www.cccu.org/}
		\end{enumerate}
	\item Christian College Consortium: \vspace{-0.2cm}
		\begin{enumerate} \itemsep -2pt
		\item \url{http://en.wikipedia.org/wiki/Christian_College_Consortium}
		\item \url{http://www.ccconsortium.org/}
		\end{enumerate}
	\end{enumerate}
\item California Baptist University, Riverside
\item Messiah College (Grantham, PA): \vspace{-0.3cm}
	\begin{enumerate} \itemsep -2pt
	\item Department of Engineering: \vspace{-0.2cm}
		\begin{enumerate} \itemsep -2pt
		\item \url{http://www.messiah.edu/departments/engineering/}
		\item B.S. programs in: \vspace{-0.1cm}
			\begin{enumerate} \itemsep -1pt
			\item Biomedical Engineering
			\item Computer Engineering
			\item Electrical Engineering
			\end{enumerate}
		\end{enumerate}
	\item Department of Information and Mathematical Sciences: \vspace{-0.2cm}
		\begin{enumerate} \itemsep -2pt
		\item \url{http://www.messiah.edu/departments/mathsci/index.html}
		\item Offers a B.A. Computer Science program
		\end{enumerate}
	\end{enumerate}
\end{enumerate}
























%%%%%%%%%%%%%%%%%%%%%%%%%%%%%%%%%%%%%%%%%
% Thoughts and Resources for Specific Areas and Topics
%	\input{./have2innovate/outreach}

%	http://sage.math.washington.edu/home/mvngu/network.html
%	http://www1.ccny.cuny.edu/advancement/news/Location-Determines-Social-Network-Influence.cfm
%	http://www.computerworld.com/s/article/9182138/MIT_builds_swimming_oil_eating_robots
%	http://creativecommons.org/licenses/by-sa/3.0/us/
%	http://www.nsf.gov/news/special_reports/science_nation/biologicalclocks.jsp
%	http://ascnetworksnetwork.org/
%	http://webscience.org/WSTNet.html
%	http://www.researcherid.com/Home.action
%	dmi unict.it
%	universit� degli studi di Palermo
%	universit� degli studi di Pavia
%	universit� degli studi di Mediterranea di Reggio Calabria
%	universit� degli studi di Napoli "Parthenope"
%	universit� degli studi di Parma
%	universit� degli studi di Pisa
%	universit� degli studi di Salerno
%	universit� degli studi di Sannio
%	universit� degli studi di Siena
%	universit� degli studi di Salento
%	universit� degli studi di Torino
%	universit� degli studi di Trieste
%	universit� degli studi di Tuscia
%	http://people.bu.edu/dougd/studentInfo.html
%	http://www.ee.ucla.edu/Prospective-home.htm	
%	http://www.youtube.com/user/messengerofChrist
%	http://160.97.10.132/comson/about/contact/comson-members/ciuprina
%	http://www.wordhacker.com/en/article/Barron_gre_list_a.htm
%	Compilers and Operating Systems for Low Power	
%	TU budapest
%	algorithmic-level synthesis
%	clustered voltage scaling / extended clustered voltage scaling
%	http://www.acm.org/globalizationreport/
%	Agnolotti del plin 

%	http://sage.math.washington.edu/home/mvngu/network.html
%	http://www1.ccny.cuny.edu/advancement/news/Location-Determines-Social-Network-Influence.cfm
%	http://www.computerworld.com/s/article/9182138/MIT_builds_swimming_oil_eating_robots
%	http://creativecommons.org/licenses/by-sa/3.0/us/
%	http://www.nsf.gov/news/special_reports/science_nation/biologicalclocks.jsp
%	http://ascnetworksnetwork.org/
%	http://webscience.org/WSTNet.html
%	http://www.researcherid.com/Home.action
%	dmi unict.it
%	universit� degli studi di Palermo
%	universit� degli studi di Pavia
%	universit� degli studi di Mediterranea di Reggio Calabria
%	universit� degli studi di Napoli "Parthenope"
%	universit� degli studi di Parma
%	universit� degli studi di Pisa
%	universit� degli studi di Salerno
%	universit� degli studi di Sannio
%	universit� degli studi di Siena
%	universit� degli studi di Salento
%	universit� degli studi di Torino
%	universit� degli studi di Trieste
%	universit� degli studi di Tuscia
%	http://people.bu.edu/dougd/studentInfo.html
%	http://www.ee.ucla.edu/Prospective-home.htm	
%	http://www.youtube.com/user/messengerofChrist
%	http://160.97.10.132/comson/about/contact/comson-members/ciuprina
%	http://www.wordhacker.com/en/article/Barron_gre_list_a.htm
%	Compilers and Operating Systems for Low Power	
%	TU budapest
%	algorithmic-level synthesis
%	clustered voltage scaling / extended clustered voltage scaling
%	http://www.acm.org/globalizationreport/
%	Agnolotti del plin 

%	http://sage.math.washington.edu/home/mvngu/network.html
%	http://www1.ccny.cuny.edu/advancement/news/Location-Determines-Social-Network-Influence.cfm
%	http://www.computerworld.com/s/article/9182138/MIT_builds_swimming_oil_eating_robots
%	http://creativecommons.org/licenses/by-sa/3.0/us/
%	http://www.nsf.gov/news/special_reports/science_nation/biologicalclocks.jsp
%	http://ascnetworksnetwork.org/
%	http://webscience.org/WSTNet.html
%	http://www.researcherid.com/Home.action
%	dmi unict.it
%	universit� degli studi di Palermo
%	universit� degli studi di Pavia
%	universit� degli studi di Mediterranea di Reggio Calabria
%	universit� degli studi di Napoli "Parthenope"
%	universit� degli studi di Parma
%	universit� degli studi di Pisa
%	universit� degli studi di Salerno
%	universit� degli studi di Sannio
%	universit� degli studi di Siena
%	universit� degli studi di Salento
%	universit� degli studi di Torino
%	universit� degli studi di Trieste
%	universit� degli studi di Tuscia
%	http://people.bu.edu/dougd/studentInfo.html
%	http://www.ee.ucla.edu/Prospective-home.htm	
%	http://www.youtube.com/user/messengerofChrist
%	http://160.97.10.132/comson/about/contact/comson-members/ciuprina
%	http://www.wordhacker.com/en/article/Barron_gre_list_a.htm
%	Compilers and Operating Systems for Low Power	
%	TU budapest
%	algorithmic-level synthesis
%	clustered voltage scaling / extended clustered voltage scaling
%	http://www.acm.org/globalizationreport/
%	Agnolotti del plin 


%	http://www1.ccny.cuny.edu/advancement/news/Location-Determines-Social-Network-Influence.cfm
%	http://www.computerworld.com/s/article/9182138/MIT_builds_swimming_oil_eating_robots
%	http://creativecommons.org/licenses/by-sa/3.0/us/

{\linespread{1}
%\bibliographystyle{IEEEtran}
\bibliographystyle{plain}
%\bibliography{./others/references}
%\bibliography{/data/others/notes/references}
%\bibliography{/data/research/zhiyang_references/references}
\bibliography{/Users/zhiyang/Documents/ricerca/lassi-bibtex/references}

\addcontentsline{toc}{section}{Bibliography}
}

























%%%%%%%%%%%%%%%%%%%%%%%%%%%%%%%%%%%%%%%%%%%%%
\end{document}