\documentclass[a4paper,10pt]{article}
%%%%%%%%%%%%%%%%%%%%%%%%%%%%%%%%%%%%%%%%%%%%%%%%%
\usepackage{graphicx}
\usepackage{amsmath}
\usepackage{array}
\usepackage{amssymb}
\usepackage{setspace}
%\usepackage[margin=1.5cm,vmargin={0pt,1cm},nohead]{geometry}
\usepackage[margin=1in,vmargin={1in,1in}]{geometry}
% Package that has the symbol for ``:=''
\usepackage{txfonts}
% Create fancy headers and footers for this document
\usepackage{fancyhdr}
%\usepackage{cite}
% The ``cite'' package causes the hyperlinks for the in-text references/citations to fail. I believe it is because this package overrides the default package for referencing. Hence, only use the ``cite'' package with the IEEE format.
% Package for ``turnstile'' binary relations, where letters are defined above and below symbols
\usepackage{turnstile}
\usepackage{extarrows}
% Package that provides the cross symbol
\usepackage{ifsym}
\usepackage{marvosym}
% Commands for using the package for hyperlinks - 
\usepackage[pdftex,
	pdftitle={Graphics and Color with LaTeX},
	pdfauthor={Patrick W Daly},
	pdfsubject={Importing images and use of color in LaTeX},
	pdfkeywords={LaTeX, graphics, color},
	pdfpagemode=UseOutlines,bookmarks, bookmarksopen,
	pdfstartview=FitH, colorlinks, linkcolor=blue, citecolor=blue, urlcolor=red,
]{hyperref}
\hypersetup{colorlinks, linkcolor=blue}
% Concatenate references
\usepackage{cite}

% definition of new \LaTeX command for the citation: \cite{Cimatti08} and \cite{Barrett09}
% This allows mathematical/logic symbols to be typeset with the font ``Zapf Chancery'' in ``\LaTeX\ math mode''. To typeset symbols in such font, try: \mathpzc{ABCdef123}
\DeclareMathAlphabet{\mathpzc}{OT1}{pzc}{m}{it}

%\input{../../others/preamble}
%\input{../../others/list_of_colors.def}

%%%%%%%%%%%%%%%%%%%%%%%%%%%%%%%%%%%%%%%%%%%%%
% Start of document
\begin{document}
\title{Notes about Succeeding in Graduate School}
\date{\today}
\author{Zhiyang Ong
\thanks{Email correspondence to: \href{mailto:ongz@acm.org}{ongz@acm.org}}
}
\maketitle

\begin{abstract} 
This article contains information about writing a Ph.D. thesis proposal. It also contains advice for being a graduate student, for grant applications, for teaching assistants, writing a Ph.D. thesis, and taking the Ph.D. oral defense. In addition, it has resources for applying for postdoctoral positions and junior faculty positions.
\end{abstract}


%%%%%%%%%%%%%%%%%%%%%%%%%%%%%%%%%%%%%%%%%%%%%
%%%%%%%%%%%%%%%%%%%%%%%%%%%%%%%%%%%%%%%%%%%%%
% Create the table of contents
\tableofcontents
%%%%%%%%%%%%%%%%%%%%%%%%%%%%%%%%%%%%%%%%%%%%%
%%%%%%%%%%%%%%%%%%%%%%%%%%%%%%%%%%%%%%%%%%%%%


%%%%%%%%%%%%%%%%%%%%%%%%%%%%%%%%%%%%%%%%%%%
\section{\hspace{0.1in} Notes about Succeeding in Graduate School}
\label{notesgradschsuccess}



%%%%%%%%%%%%%%%%%%%%%%%%%%%%%%%%%%%%%%%%%%%
\subsection{\hspace{0.1in} Preliminary Ph.D. Thesis Proposal}
\label{phdthesisproposal}

The full Ph.D. dissertation proposal is approximately 20-50 pages with appropriate tradeoffs between what others have done, what the student has done, and what the student will do. \\
\ \\
Things to be included in my preliminary Ph.D. thesis proposal (preliminary research proposal): \vspace{-0.3cm}
\begin{enumerate} \itemsep -4pt
\item This may also be known as the ``PhD plan''
\item This project proposal shall be about 5-10 pages 
\item This can or should be completed in cooperation with my main supervisor(s)
\item Abstract of the research proposal: include field for keywords to indicate the scope of the project
%%%%%%%%%%%%%
%	Begin description of research proposal
\item Project description (0.5-1 page): \vspace{-0.3cm}
	\begin{enumerate} \itemsep -3pt
%%%%%%%%%%%%%
%	Introduction
	\item {\bf \color{green} Introduction:} \vspace{-0.2cm}
		\begin{enumerate} \itemsep -1pt
		\item provide a general introduction to what the thesis is all about
		\item it is not just a description of the contents of each section
		\item briefly summarize the research problem/question (I shall be stating the research problem/question in detail later)
		\item also, provide a summary for some of the reasons why it is a worthwhile question
		\item {\tt For the Ph.D. thesis ONLY: give an overview of my main results}
		\item this is a bird's-eye view of the answers to the main questions answered in the thesis
		\item set up the basis of the research
		\end{enumerate}
%%%%%%%%%%%%%
%	Background information
	\item {\bf \color{green} Background Information (optional):} \vspace{-0.2cm}
		\begin{enumerate} \itemsep -1pt
		\item a brief section giving background information may be necessary, especially if your work spans two or more traditional fields
		\item that means that your readers may not have any experience with some of the material needed to follow your thesis, so you need to give it to them
		\item a different title than that given above is usually better; e.g., ``A Brief Review of Frammis Algebra.'' 
		\end{enumerate}
%%%%%%%%%%%%%
%	Problem statement, research hypothesis, and objectives
	\item {\bf \color{green} Research Question or Problem Statement:} \vspace{-0.2cm}
		\begin{enumerate} \itemsep -1pt
		\item Problem formulation.
		\item The problem statement should be comprehensible by faculty in the department who are not in the particular specialty of the student and the advisor.
		\item Preliminary considerations: \vspace{-0.2cm}
			\begin{enumerate} \itemsep -2pt
			\item What is the considered scope, or focus of my Ph.D. research?
			\item What's my vision? And, what's my research topic?
			\item Which issues will I be considering?
			\item Is my proposed research topic well-defined and specific?
			\item Is it novel?
			\item If this research topic is competently investigated, is it significant enough to be worthy of a Ph.D.? 
			\end{enumerate}
% Engineering theses tend to refer to a ``problem'' to be solved, whereas other disciplines talk in terms of a ``question'' to be answered.
		\item What problem(s) am I solving?: \vspace{-0.2cm}
			\begin{enumerate} \itemsep -1pt
			\item What is the problem?
			\item provide a clear explanation of the problem(s) to be worked on
			\item Succinctly state the research problem and its significance: \vspace{-0.2cm}
				\begin{itemize} \itemsep -2pt
				\item use a concise statement of the problem that my research will attempt to solve
				\item omit vague terms in my problem statement... for example, do not say ``develop a Zylon algorithm capable of handling very large scale problems in reasonable time''; specify ``large scale'' and ``reasonable time'' by describing what I mean in numerical figures, or numerical scale (e.g. ultra-deep submicron technologies, nanoscale architectures, exabyte data storage, or THz imaging)
				\item definition of the problem space
				\item list keyword to define your topic
				\item state your research topic as a problem
				\item think about the significant terms, concepts, and keywords that describe your topic. 
				\item these terms will become the (search) keys while searching for information about your subject in library catalogs, online databases, and other resources.
				\end{itemize}
			\end{enumerate}
		\item {\bf \color{magenta} What is so interesting about this/(these) problem(s)? } \vspace{-0.2cm}
			\begin{enumerate} \itemsep -2pt
			\item Why is the problem important?
			\item Have I identified a worthwhile problem that has not been answered?
			\item Is this open problem indicated in a number of the papers that I have read?
% In science and mathematics, an open problem or an open question is a known problem that can be accurately stated, and has not yet been solved (no solution for it is known)
			\item for open problems (unresolved issues), provide a justification, by direct reference to a literature review, to indicate that my research problem is unsolved
			\item else, indicate that the problem has not been solved adequately, or its solutions may not be as effective in the near future as technologies improve
			\item alternatively, say... given existing architectural scaling trends (or process technology scaling trends), this problem would remerge since current solutions do not scale well (with respect to architectural scaling trends, or process technology scaling trends). Hence, these solutions may not be able to adequately handle BLAH ISSUES with advanced architectures (or process technologies). In this case, more extensive justification for the selection of this research problem can be given later in the literature review section
			\item analyze the information read during my literature review, and indicate how each class/category of current approaches fails; i.e., current approaches can handle only small problems, and/or takes too much time
			\end{enumerate}
		\item Discussion of why it is worthwhile to solve this problem: \vspace{-0.2cm}
			\begin{enumerate} \itemsep -1pt
			\item What are my motivations for solving this problem?
			\item How would my work extend the cutting edge in this field?
			\item Why is it useful to have a solution to this problem?
			\item Who would care if I have a solution to this problem? R\&D teams in high-tech companies??? University research labs???
			\item What are the wider implications of my research?
			\end{enumerate}
		\item To highlight this section (i.e., Problem Statement), try to include the term ``{\tt Problem Statement}'' in a fairly specific section title. For example, ``The Large-Scale Zylon Algorithm Problem.''
		\end{enumerate}
%%%%%%%%%%%%%
%	Literature review
	\item {\bf \color{green} Review of the State of the Art:} \vspace{-0.2cm}
		\begin{enumerate} \itemsep -1pt
		\item {\bf \color{magenta} Survey of the state of the art (2-5 pages approx.); 5-10 pages in a typical thesis proposal, but is longer in the ``prior work'' section of a Ph.D. dissertation}
		\item this section shall be longer if there exists rich literature for my research topic
		\item make ample references to existing literature in the field
		\item \textcolor{green}{\bf this literature review section can serve as the nucleus for such a chapter in my Ph.D. dissertation; since the chapter in Ph.D. dissertation is longer, it pays to do a extensive literature review early on for my research proposal and keep up to date with advancements in the field}
		\item What has been done so far on the problem?
		\item How does it relate to previous work in the field?
		\item Include a brief introduction to my topic, and introduce key concepts
		\item What is the cutting edge in my field? State-of-the-art solution(s) of the problem.
		\item {\it That is, review the state of the art relevant to my thesis} ... {\color{green} \tt What has been done so far on the problem?}
		\item Again, a different title is probably appropriate; e.g., ``State of the Art in Zylon Algorithms.'' 
		\item Determine a collection of seminal papers (fundamental references) that can be considered real milestones in the field of interest
		\item What are the seminal papers in this field, and for my research topic? Have I read, reviewed, and referenced them? % In psychology, read recent overview chapters, integrative articles, and recent research ... Integrative articles that draw upon diverse areas of the field and relate them to the whole.
		\item For my research topic, are there seminar classes offered by any university that provides a list of papers that I should read?
		\item What other issues are involved in this topic?
		\item Present (/perform critical analysis of) the major ideas in the state of the art right up to, but not including, my own personal brilliant ideas
		\item Organize this section by idea(s), and not by author or by publication.
		\item For example, if there have been three important main approaches to Zylon Algorithms to date, you might organize subsections around these three approaches, if necessary: \vspace{-0.2cm}
			\begin{enumerate} \itemsep -1pt
			\item 3.1 Iterative Approximation of Zylons
			\item 3.2 Statistical Weighting of Zylons
			\item 3.3 Graph-Theoretic Approaches to Zylon Manipulation
			\end{enumerate}
		\item Determine if related research might be published under different keywords
		\item The lit review can be difficult, and I may feel lost and don't know where I'm going. This is because I'm trying to do several things concurrently; I'm trying to learn about this whole field, and get a conceptual framework of how to map out this area of research. So, {Fight On!!!}
		\end{enumerate}
	\item {\bf \color{cyan} A very clear statement of the question is essential to proving that you have made an original and worthwhile contribution to knowledge. To prove the originality and value of your contribution, you must present a thorough review of the existing literature on the subject, and on closely related subjects. Then, by making direct reference to your literature review, you must demonstrate that your question (a) has not been previously answered, and (b) is worth answering. Describing how you answered the question is usually easier to write about, since you have been intimately involved in the details over the course of your graduate work. }
%%%%%%%%%%%%%
%	Proposed approach
	\item {\bf \color{green} How will I be addressing these issues?:} \vspace{-0.2cm}
		\begin{enumerate} \itemsep -1pt
		\item Suggested/proposed approaches, including initial approach
		\item What are my goals/objectives $[$or aims$]$ of this research project?
		\item {\bf \color{magenta} Proposal for innovative research contribution for the thesis work with a clear and detailed description of the research directions and objectives (2-5 pages approx.)}
		\item What would be my contribution(s) of the thesis on the problem?
		\item Would my contribution(s) be original and non-trivial?
		\item How would I position this research project in the international context of research in the field?: \vspace{-0.2cm}
			\begin{enumerate} \itemsep -1pt
			\item Has this ever been done before? Or am I merely extending current work for larger problems (bigger inputs)?
			\item How many people are working on this research topic, or in a similar topic?
			\item What would make this project more significant than theirs?
			\item Where would I rank in comparison to them? Are we the best research team in this topic? if not, top 3? (or top 5?) Top 10?? Top 20??? Insignificant???
			\end{enumerate}
		\item Proposed research methodology/solution (or method): \vspace{-0.2cm}
			\begin{enumerate} \itemsep -1pt
			\item What main techniques, experiments, and trials would I be using?
			\item include detailed descriptions of what I plan to do: \vspace{-0.2cm}
				\begin{itemize} \itemsep -2pt
				\item theorems to formulate and prove
				\item systems to build
				\item experiments to perform
				\end{itemize}
			\item what is my exposition/expos{\'{e}} of the way chosen to reach my goals
			\item for the aforementioned problem(s) to be attacked, what are my approaches for attacking them? \vspace{-0.1cm}
				\begin{itemize} \itemsep -2pt
				\item details for my approaches do not need to be fully worked out; indeed, they shouldn't be at this stage
				\item however, there needs to be enough information for the committee to estimate the likelihood that the approach will work, and to estimate the difficulty of completing the thesis 
				\end{itemize}

			\item {\bf \color{green} What would be my original and useful contribution(s) to the body of knowledge in formal verification, or electronic design automation? }
			\item What is the novelty and groundbreaking character of the proposed research?
			\item Use {\tt divide and conquer} to break the problem into subproblems
			\item For each subproblem that I would be solving, come up with a set of objectives for the subproblem
			\item Is each objective numerically quantifiable? Can I measure properties (or physical quantities, or modal properties -- such as verifiability) of the algorithm, circuit, methodology, or system? If so, how should I quantify them? What metrics would I be using?
% attributes, characteristics, and traits are not necessarily quantifiable/measurable
			\item What is the scientific basis for my evaluation of the experimental results? Or rather, how do I plan to evaluate my experimental results?
			\item Verify if meeting this objectives would solve this problem
			\item Have I convinced the examiners that my proposed solutions can solve the problem that I set for myself? 
			\item Are my solutions effective and efficient? Can I further improve its computational time and space complexity? Are my solutions scalable for larger problems (bigger inputs), and for larger parallel computers (bigger server farm, {\tt Linux} cluster, GPGPU computing platform, or bigger processor (64-bit processors and/or multi- and many-core processors))?
			\item Have I benchmarked my EDA tool $[$for VLSI formal verification$]$? Does it outperform the other EDA tools in the public domain?  
			\item Have I also convinced the examiners that I have foreseen potential roadblocks, such as blind alleys and dead ends? Have I taken appropriate measures to avoid them, and documented this in my proposal? Have I listed contingency plans for potential roadblocks in my proposal? Indicate that I have a reasonable risk/change management process?
			\item Have I shown that my proposed solutions and research directions are relevant to solving the problem?
			\end{enumerate}
		\item Indicate how the techniques and ideas of my reference papers play a role in the directions that I am taking
		\item Accomplished tasks and milestones met: \vspace{-0.2cm}
			\begin{enumerate} \itemsep -2pt
			\item If I have solved some of the subproblems, summarize the results
			\item relevant publications (conference/journal papers, or technical reports) may be attached as appendices
			\end{enumerate}

%%%%%%%%%%%%%%%%%%%%%
%		Deliverables and expected results
		\item Desired/Expected outcomes: \vspace{-0.2cm}
			\begin{enumerate} \itemsep -2pt
			\item Requirements and deliverables of the Ph.D. research. And, evaluation method for the verification of the requirements and deliverables. E.g., ``we will use the XY benchmark and compare to the figures in Z's article.''
			\item What are my expected results in relation to the current international status of research in this field?
			\item What is the impact and potential for promoting scientific innovation, both inside and outside the field?
			\item What are the deliverables?
			\item What is the significance and importance of my work?
			\end{enumerate}
		\end{enumerate}
	\end{enumerate}
\item Who is/(are) my proposed supervisor(s)?
\item Scheduled plan for classes taken in this first 3 semesters
%%%%%%%%%%%%%%%%%%%%%
%	Project timeline, OR working plan / work schedule
\item Project timeline: work schedule (or working plan), with milestones (e.g., literature review, software implementation, software testing, tool analysis, and writing) and targets: \vspace{-0.3cm}
	\begin{enumerate} \itemsep -3pt
	\item A deliverable is different from a project milestone in the sense that a milestone is a way of checking to what extent you are making progress to an outcome. A milestone can be completing the planning phase of the project while a deliverable includes things like design documents, testing scripts, requirement documents etc.
	\item indicate previous coursework and research experience that prepared me for research in this field
	\item indicate what will I do in preparation for Ph.D. research in the program, prior to its commencement
	\item list specific milestones, and a timeline for when I expect to meet each milestone; the timeline should schedule writing up of results along the way, and indicate possible conferences or journals to which the work should be submitted 
	\item Completion of comprehensive literature review of my research problem
	\item Completion of each subproblem that will help me solve my research problem
	\item for each subproblem, include the following milestones/tasks: \vspace{-0.2cm}
		\begin{enumerate} \itemsep -3pt
		\item brief literature review to keep myself to date with contemporary solutions to subproblem
		\item either select one of the best solution to implement, tweak the ``best'' existing solution, or develop a new solution
		\item the development of any solution shall include the following tasks: \vspace{-0.1cm}
			\begin{enumerate} \itemsep -1pt
			\item development of an algorithm or heuristic
			\item if an algorithm is developed, mathematically prove that it works; if necessary, use axioms lemmas ... Also, does this algorithm have any corollary?
			\item if a heuristic is developed, indicate if a metaheuristic is used; if so, which metaheuristic is used
			\end{enumerate}
		\item implementation of the solution
		\item verify, test, and validate the solution
		\item benchmark this solution
		\item analyze this solution and draw conclusions from my experimental results; this should also involve a statistical analysis of my experimental results
		\item write up my solution and findings in a technical report and/or conference paper; this shall form a section or chapter in my Ph.D. dissertation
		\end{enumerate}
	\item When developing and updating this timeline, for each task that I assume will take time {\it t} to complete, plan for a duration of 1.5$\times${\it t} to complete the task
	\item Course credits in {\color{green} \bf ICT doctoral courses} to be completed: $\geq$15 in $1^{st}$ year, \& $\geq$9 in $2^{nd}$ year; $\geq$24 in the first two academic years
	\item Completion/Progress of classes that are required for the degree program
	\item Suggested timeframe for the {\color{green} \bf qualifying examination (Quals)}
	\item Complete a draft of my Ph.D. thesis proposal by the end of my first year, get the draft reviewed before the 3rd week of the second year, make corrections to the proposal in the next fortnight, and repeat review-correct process (2 weeks for review, 2 weeks for correction) no more than thrice ... {\tt Else, it may be an indication that I would have chosen a bad research problem}
	\item Pass my qualifying examination before the $18^{th}$ month mark
	\item Pass my qualifying examination before I commence my first internship. This lifts a burden off me, and brings me closer towards graduation. It can also bring a focus to my research to a specialized (sub-)topic, and help me find an internship that better suits my research interests; if I can use work from my research internship for my thesis, this may allow me to knock of some milestones (solve some subproblems) and bring me closer to graduation.

%%%%%%%%%%%%%%%%%%%%%%%%%%%%%%%%%%%%%%%%%%%%%%
%%%%%%%%%%%%%%%%%%%%%%%%%%%%%%%%%%%%%%%%%%%%%%
%
%	CONFERENCES		CONFERENCES		CONFERENCES
%	
%%%%%%%%%%%%%%%%%%%%%%%%%%%%%%%%%%%%%%%%%%%%%%
%%%%%%%%%%%%%%%%%%%%%%%%%%%%%%%%%%%%%%%%%%%%%%
	\item Include conferences, workshops, and symposiums that I plan to attend: \vspace{-0.2cm}
		\begin{enumerate} \itemsep -2pt
%		\item International Conference on Verification, Model Checking, and Abstract Interpretation (VMCAI) --- Jan % model checking ... program verification $[$Software formal verification???$]$
%		\item ACM SIGACT-SIGPLAN Symposium on Principles of Programming Languages (POPL 201X) -- Jan % 
		\item Asia and South Pacific Design Automation Conference (ASPDAC) -- Jan
%		\item International Conference on High Performance Embedded Architectures \& Compilers (HiPEAC 201X) -- Jan % Processor and multi-core architectures (core, memory system, interconnects,...) ... Heterogeneous multi-cores ... Accelerators (ASICs, GPUs, ASIPs,...) ... Domain/Application specific systems ... Reconfigurable architectures, tools and systems ... Compiler techniques for processors, multi-cores and application/domain-specific architectures ... Programming models for multi-cores and application/domain-specific architectures ... Dynamic, adaptive and continuous optimization and compilation ... Interaction between operating systems and programming models, compilers, architectures ... Tools and techniques for simulation and performance analysis ... Program and workload characterization ... Emerging applications ... Emerging computing platforms (data centers, game consoles, smartphones,...) ... Introduction of novel technologies in current computing systems (3D stacking, photonics, analog VLSI,...) ... Novel architecture and programming paradigms for current or future technologies
%		\item Workshop on Rapid Simulation and Performance Evaluation: Methods and Tools (RAPIDO'1X) -- Jan % Rapid simulation techniques especially those dedicated for new architectures: Multi-cores, FPGA based heterogeneous Multi-cores/MPSoC, 3D-architectures, ..etc ... Variability and power/energy consumption in performance estimation and simulation techniques. ... High-level abstraction modeling, e.g., Transactional Level Modeling (TLM) ... Rapid design space exploration (DSE) for heterogeneous and embedded systems. ... Dynamic binary translation for fast simulation and DSE ... Experience reports using existing simulators ... Benchmarking and simulator validation		....		The focus of the RAPIDO�11 workshop is on methods and tools for rapid simulation and performance evaluation in embedded and high performance systems design. Given continuous advances in chip technology, it is to be expected that future-generation processors will integrate numerous units on a single die, including multiple processor cores, multiple levels of (shared/private) caches or memories, and multiple dedicated accelerators, which will be glued together through a network on-chip (NoC).	...	The design space is huge though:  ... How many cores do we need?   ... Should we have a homogeneous or a heterogeneous design?   ... When dynamic reconfiguration must be performed?  ... How many caches/memories do we need?  ... How to choose the instruction set(s) for these cores?   ... What are the best code optimizations for a given application?  ... How to combine the different metrics (e.g. energy, latency and throughput) into a global search space? 
%		\item Workshop on Computer Architecture and Operating System co-design (CAOS) -- Jan % Architectural and OS support for power and thermal management ... Architectural and OS support for scheduling applications on emerging multi-core systems ... Benchmarking and characterization of OS activity in multi-core architectures ... Architectural and OS support for virtualization ... Architectural and OS support to manage processor resource allocation and heterogeneity for Quality of Service ... Simulation tools for full system simulation
%		\item HiPEAC Workshop on Design for Reliability (DFR'1X) -- Jan % Dependable systems from unreliable components, lifelong reliability ... Fault-Tolerant micro-architectures and system architectures ... Testing and verification strategies for the future ... On-line (dynamic) testing and verification techniques ... Software-based methodologies for fault tolerance and testing ... System validation mechanisms ... Built-in self diagnosis, self-tuning and recovery schemes ... Self-adaptive systems ... System-level design and integration for reliability, verifiability and dependability ... Error modeling, detection, correction, and tolerance for transient and permanent errors ... Reliable on-chip communications ... Energy/reliability/performance tradeoffs ... Aggressive power saving mechanisms ... Compiler/architecture/OS methodologies and strategies for reliability
%		\item Workshop on Interconnection Network Architectures: On-Chip, Multi-Chip (INA-OCMC) -- Jan % Networks-on-Chip (NoC) ... Multi-Chip Interconn. Networks, including Cluster Interconnects ... "Commodity Switches" as general-purpose building blocks ... Switching, buffering, and routing architectures ... Flow control and congestion management in switching fabrics ... Virtualization ... Topology exploration ... Timing, synchronous/asynchronous communication ... Reliability, availability, fault tolerance ... Area/power versus functionality/QoS support in NoC architectures ... Design space exploration ... NoC physical link design ... NoC testing and verification ... Programming models for NoC-centric systems 
%		\item International Workshop on Timing Issues in the Specification and Synthesis of Digital Systems (TAU Workshop) -- Feb / Timing issues in EDA, performance optimization, delay minimization, clock domains, asynchronous VLSI 
%		\item Euromicro International Conference on Parallel, Distributed and Network-Based Computing: Special Session on On-Chip Parallel and Network-Based Systems -- Feb % On-chip network architecture (topology, routing, arbitration) ... Processor allocation and scheduling in CMPs ... Mapping of applications onto NoCs ... Performance and power issues in NoCs ... Multi/many-core workload characterization & evaluation; Modeling and simulation of on-chip parallel and networked systems; Synthesis, verification, debug & test of SoCs; NoC support for memory and cache access; SoC and NoC design methodologies and tools; Network support for SoC quality of service; On-chip systems for FPGAs and structured ASICs; NoC support for CMP/MPSoCs; Floorplan-aware NoC architecture optimization; Application-specific NoC design; Reconfigurable SoCs & NoCs; Memory system design and optimizations for SoCs; SIMD parallel VLSI computing; I/O interconnects and support for SoCs
%		\item ACM SIGPLAN Annual Symposium on Principles and Practice of Parallel Programming (PPoPP 201X) -- Feb % foundational and theoretical aspects, techniques, tools, and practical experiences of parallel programming ... In the context of the symposium, "parallel programming" encompasses work on concurrent and parallel systems (multicore, multithreaded, heterogeneous, clustered systems, distributed systems, and large scale machines) ... Address new parallel workloads, techniques and tools that attempt to improve the productivity of parallel programming, and work towards improved synergy with such emerging architectures ... Parallel programming theory and models ... Formal analysis and verification ... Parallel programming languages ... Compilers and runtime systems ... Task-parallel libraries ... Parallel application frameworks ... Software productivity for parallel programming ... Middleware for parallel systems ... Performance analysis, debugging and optimization ... Development, analysis, or management tools ... Parallel algorithms ... Parallel applications ... Concurrent data structures ... Synchronization and concurrency control ... Software engineering for parallel programs ... Fault tolerance for parallel systems ... Software for heterogeneous architectures ... Programming tools for parallel and heterogeneous systems ... Parallelism in non-scientific workloads: web servers, search, analytics
%		\item International Symposium on High-Performance Computer Architecture (HPCA-1X, or HPCA-2X) -- Feb % Processor, cache, and memory architectures ... Parallel computer architectures ... Multicore and multiprocessor architectures ... Impact of technology on architecture ... Power-efficient architectures and techniques ... Reliable and secure architectures ... High-performance I/O systems ... Embedded, reconfigurable, and heterogeneous architectures ... Interconnect and network interface architectures ... Network processor architectures ... Innovative hardware/software trade-offs ... Impact of compilers and system software on architecture ... Performance modeling, simulation, and projection techniques ... Architectures for emerging technology and applications
%		\item ACM/SIGDA International Symposium on Field-Programmable Gate Arrays (FPGA 201X) -- Feb / FPGA Architecture: Novel logic block architectures, combination of FPGA fabric and system blocks (DSP, processors, memories, etc.), design of routing fabric, I/O interfaces, new commercial architectures and architectural features ... FPGA Circuit Design: Novel FPGA circuits and circuit-level techniques, impact of process and design technologies, methods for analyzing and improving issues with soft-errors, leakage, static and dynamic power, clocking, power grid, yield, manufacturability, reliability, test; studies on future device technologies (e.g. nano-scale, 3D gate) for FPGAs ... CAD for FPGAs: Placement, routing, retiming, logic optimization, technology mapping, system-level partitioning, logic generators, testing and verification, CAD for FPGA-based accelerators, CAD for incremental FPGA design and on-line design mapping and optimization, CAD for modeling, analysis and optimization of timing and power ... High-Level Abstractions and Tools for FPGAs: General-purpose and domain-specific models, languages, tools, and techniques to facilitate the design, development, debugging, verification, and deployment of large-scale and high-performance FPGA-based applications and systems - e.g. DSP, networking or embedded system tools and methodologies ... FPGA-Based and FPGA-like computing engines: Compiled accelerators, reconfigurable computing, adaptive computing devices, systems and software, rapid-prototyping ... Design Studies: Innovative uses of FPGA fabric for computation, exploitation of FPGA features and architectures, optimization of FPGA-based cores (e.g. arithmetic, DSP, security, embedded processors, memory interfaces, or other functions) ... Applications: Implementation of designs on FPGAs to achieve high-performance, low-power, or high-reliability. Novel design algorithms which take advantage of FPGA features. Application-domain studies to analyze or improve FPGA implementation for networking, DSP, embedded, audio/video, automotive, imaging and other relevant areas.
%		\item IEEE International Solid-State Circuits Conference (ISSCC 201X) -- Feb % ENERGY-EFFICIENT DIGITAL: > 1GHz embedded processors, energy-efficient multi-core processors, energy-efficient wide-operating-range processors; digital IC for personal e-health devices, digital IC for automotive applications, energy-efficient sensor system; intelligent power-management methods for digital VLSI; adaptive voltage and frequency scaling, adaptive body bias circuits, on-chip monitoring/sensing circuits for energy-efficient applications; variation-tolerant circuit and architecture, techniques for timing margin reduction; energy-efficient circuit techniques	...	HIGH-PERFORMANCE DIGITAL -- Microprocessors; graphics processors; many-core and thread-rich processors; network processors; high-speed digital circuits; intra-chip communication circuits; soft error, variation, and fault-tolerant circuits; reconfigurable logic arrays; security circuits; high-speed CAMs and register files; clock generation and distribution circuits and architectures; high- performance-logic microarchitectures and circuit techniques; implementation methodologies for high-performance digital VLSI; power and leakage management techniques for high performance processors and graphics; adaptive digital circuits; thermal and wear-out sensors.
%		\item IEEE Latin American Symposium on Circuits and Systems (LASCAS 201X) -- Feb % Nanoelectronics and Gigascale Systems... Cellular Neural Networks and Array Computing... Computer Aided Design... Graph Theory and Computing... VLSI Systems and Applications... Electronic Testing... Fault Tolerant Circuits
%		\item International Conference on Architecture of Computing Systems (ARCS) -- Feb/Mar % computer architecture (multicore processors, memory systems, parallel computing) ... embedded systems architecture, communication, design methodologies, and applications ... Adaptive system architectures such as reconfigurable systems in hardware and software... Organic and Autonomic Computing including both theoretical and practical results on self-organization, self-configuration, self-optimization, self-healing, and self-protection techniques... Energy-awareness, green computing
%		\item Workshop on Dependability and Fault Tolerance -- Mar % formal verification of systems ... testing of hardware and software ... validation and verification
%		\item Workshop on Many Cores -- Mar
%		\item International Workshop on Multi-Core Computing Systems (MuCoCoS) -- Mar % multi-core architectures ... interconnection networks ... multi-core embedded systems ... performance modeling and evaluation of multi-core systems ... design space exploration ... resource usage optimization ... tool-support for multi-core systems
		\item International Symposium on Quality Electronic Design (ISQED) -- Mar %  Design Verification and Design for Testability ...  Design for Manufacturability/Yield & Quality, including formal and simulation based design verification techniques ...  EDA Methodologies, Tools, Flows & IP Cores; Interoperability  and Reuse ... System-level Design, Methodologies & Tools ... Device, substrate, interconnect, circuit , and IP block modeling and simulation techniques ...  Signal integrity analysis: coupling, inductive and charge sharing noise; noise avoidance techniques ... Modeling statistical process variations to improve design margin and robustness, use of statistical circuit simulators ... Power grid design, analysis and optimization; timing analysis and optimization; thermal analysis and design techniques for thermal management.  Power-conscious design methodologies and tools ... Successful applications of TCAD to circuit design. Impacts of process technologies on circuit design and capabilities (e.g. low-Vt transistors versus increased off-state leakages) and the accuracy, use and implementation of SPICE models that faithfully reflect process technologies ... Design of Reliable Circuits and Systems ... ESD  design for digital, mixed signal and RF applications. Exploration of critical factors such as noise, substrate coupling, cross-talk and power supply noise.  Significance and trends in process reliability effects such as gate oxide integrity,  electromigration, ESD, etc.,  and their relation to electronic design ... Physical Design, Methodologies & Tools ... Physical design for manufacturing ... Spare-cell strategies for ECO, decoupling capacitance and antenna rule fixing ... EDA tools, design techniques, and methodologies, dealing with issues such as: timing closure, RLC extraction, ground/Vdd bounce, signal noise/cross-talk /substrate noise, voltage drop, power rail integrity, electromigration, hot carriers, EOS/ESD, plasma induced damage and other yield limiting effects, high frequency effects, thermal effects, power estimation, EMI/EMC, proximity correction & phase shift methods, verification (layout, circuit, function, etc.)
%		\item European Conference on Antennas and Propagation (Eucap) -- Mar / computational electromagnetics for antenna design
%		\item International Test Synthesis Workshop (ITSW) -- Mar % RTL DFT, High-Level/Behavioral Test Synthesis, SoC DFT, Memory and Logic BIST, Test Synthesis for Debug and Diagnosis, DFT for  Mixed-Signal Circuits, Test Resource Partitioning, Functional Verification, DFT for Emerging Technologies, Power and Noise-Aware Test, DFT for At-Speed Test, High-speed I/O test, Reducing the Cost of Test, Design for Manufacturing and Yield, Board and System Test, SER / Reliability, Test Synthesis for Reconfigurable Logic
%		\item IEEE Latin-American Test Workshop (LATW201X) -- Mar % 
%		\item IEEE Workshop on Silicon Errors in Logic - System Effects (SELSE {\it n}), where $n \in \mathbb{N}$ -- Mar / examine system-level effects of errors from a variety of perspectives: architectural, logical and circuit-level, and semiconductor processes ... system-level error management} ... The growing complexity and shrinking geometries of modern device technologies are making high-density, low-voltage devices increasingly susceptible to influences from electrical noise, process variation, and natural radiation interference ... New error mitigation techniques. ... Characterizing the overhead and design complexity of error mitigation techniques ... Case studies describing the engineering tradeoffs necessary to decide what mitigation technique to apply ... System-level models: derating factors and validation of error models. ... Error handling protocols (higher-level protocols for robust system design)
%		\item International Conference on Pervasive and Embedded Computing and Communication Systems (PECCS 201X) -- Mar % Embedded Systems Design::: Software architectures ... Pervasive Embedded Devices ... Networking and Connectivity ... Micro and Nanotechnology ... Real time systems ... RF and Wireless Circuits ... VLSI Design and Implementation ... Low-Power Electronics ... Embedded Robotics ... Instrumentation and Measurement ... Sensors and Sensor Networks
%		\item International Conference on Simulation Tools and Techniques (SIMUTools 201X) -- Mar % Simulation methods: discrete event, parallel and distributed, hybrid, load balancing, partitioning, simulation-interoperation ... Simulation techniques: service-oriented, agent-based, web-based, component-based, symbiotic simulation ... Simulation tools, libraries and frameworks, including specialized tools, such as ns-2/3, OPNET, ATDI ICS, Qualnet, OMNET++, NIIST, e-cell, and other open source tools ... Modeling formalisms: DEVS, Petri Nets, process algebras, state charts, and others ... Simulation verification, validation, accreditation, and analysis: benchmark models, simulation-based verification, workflow, simulation experiment design, optimization, and simulation coercion	...	Applications of simulation in::: Wireless technologies (cellular, vehicular, mesh, ad hoc, wireless sensor networks) ... Network models (mobility models, traffic models, network topology) ... Overlay networks, peer-to-peer networks ... Parallel and distributed systems, high-performance computing systems ... Operating systems ... Fault tolerant systems ... Embedded and real-time systems ... Human behavioral models/representations ... Infrastructure networks (transportation, traffic, electric power, natural gas, etc.) ... Logistics and manufacturing ... Environmental and biological systems ... Security and emergency applications ... Military applications
%		\item ACM Symposium on Applied Computing (SAC 201X) -- Mar % Constraint Solving and Programming	...	Green/Power-Aware Design and Optimization	...	Self-organizing Complex Systems
		\item Design, Automation and Test in Europe (DATE) -- Mar/Apr
%		\item Predictability and Performance in Embedded Systems Workshop (PPES 2011) -- Mar/Apr % The workshop will discuss approaches to achieve improvements of worst-case predictability and of average-case performance on all system layers, including hardware architecture, operating systems, code generation, software architecture and program analysis. It will also discuss the problems arising in industrial practice in trying to achieve one or both of these goals and address proposed tools or standardization efforts.
%		\item International Conference on Tools and Algorithms for the Construction and Analysis of Systems (TACAS 201X) -- Mar/Apr % Specification and verification techniques for finite and infinite-state systems... Software and hardware verification... Theorem-proving and model-checking... System construction and transformation techniques... Static and run-time analysis... Abstraction techniques for modeling and validation... Compositional and refinement-based methodologies... Testing and test-case generation... Analytical techniques for safety, security, or dependability... Analytical techniques for real-time, hybrid, or stochastic systems... Integration of formal methods and static analysis in high-level hardware design or software environments... Tool environments and tool architectures... SAT and SMT solvers
%		\item International Workshop on Numerical Software Verification (NSV-{\it n}), where $n \in \mathbb{N}$ in Roman numbers -- Apr / applying logical and mathematical techniques for reasoning about numerical aspects of software ... Numerical properties of control software ... Models and Abstraction Techniques ... Real-Time Verification ... Reasoning about Automatically Generated Software ... Hardware-Software Interaction ... Validation for avionics, automotive and real-time applications
%		\item International Symposium on Physical Design (ISPD) -- Apr % Floorplanning and interconnect planning ... Partitioning, placement and routing ... Physical design for manufacturability and yield ... Synthesis optimizations within physical design ... Estimation and modeling ... Timing and crosstalk issues in physical design ... Special structures for clocking and power networks ... Physical design for emerging process technologies ... Interactions with behavior-level synthesis flows ... Interactions with logic-level (re-)synthesis flows ... Analysis and management of power dissipation ... Management of design data and constraints ... New physical design methodologies ... New paradigms in physical design ... Circuit performance measurements in a PD context ... Multithreaded/distributed algorithms for physical design
%		\item IEEE Symposium on Low-Power and High-Speed Chips (COOL Chips) -- Apr / Low Power-High Performance Processors for Multimedia, Digital Consumer Electronics, Mobile, Graphics, Encryption, Robotics, Networking and Biometrics ... Novel Architectures and Schemes for Single Core, Multi-Core, Embedded System, Reconfigurable Computing, Grid, Ubiquitous, Dependable Computing and Wireless ... Cool Software including Binary Translations, Compiler Issues and Low Power Techniques. 
%		\item IEEE International Symposium on Performance Analysis of Systems and Software (ISPASS 201X) -- Apr % Performance and power evaluation methodologies::: Analytical modeling ... Statistical approaches ... Tracing and profiling tools ... Simulation techniques ... Hardware (e.g., FPGA) accelerated simulation ... Hardware performance counter architectures ... Techniques for modeling power, reliability, and other modern relevant metrics for computer systems		...	Performance and power analysis ... Performance and power metrics ... Bottleneck identification and analysis ... Visualization		...	Performance and power analysis of commercial and experimental hardware::: General-purpose microprocessors ... Multi-threaded, multi-core and many-core architectures ... Memory and storage systems ... Accelerators and graphics processing units ... Embedded and mobile systems ... Enterprise systems and data centers ... Supercomputers ... Computer networks		...	Performance and power analysis of emerging workloads and software::: Software written in managed languages ... Virtualization and consolidation workloads ... Internet-sector workloads ... Embedded, multimedia, games, telepresence ... Bioinformatics, life sciences, security, biometric		 ...	Application and system code tuning and optimization
%		\item International Symposium on Code Generation and Optimization (CGO 201X) -- Apr % Compilers, back-end code generators, translators, binary optimization tools and runtime environments; static, dynamic, adaptive, or continuous techniques ... New or improved optimization algorithms, including profile-guided and feedback-directed optimization ... Thread extraction and threadlevel speculation, especially for multicore and manycore systems ... Analyses, and optimizations targeting heterogeneous processors and/or GPUs ... Virtualization support for multicore and/or heterogeneous computing ... Phase detection and analysis techniques ... Language features and runtime support for parallelism (including support for transactional semantics, efficient message passing, and dynamic thread creation) ... Program characterization methods targeted at program optimization ... Code transformations to address security, reliability, virtualization, temperature, or energy efficiency ... Architectural support for improved profiling, optimization and code generation ... Experiences with real dynamic optimization and compilation systems on general purpose, embedded system and HPC platforms ... Library and system call support for optimization and code generation ... Solutions that involve crosslayer (HW/OS/VM/SW) design integration ... Efficient profiling and instrumentation techniques ... Memory management, including data distribution, synchronization and garbage collection ... Intermediate representations that enable more powerful or efficient optimization ... Traditional compiler optimizations		...		Code Generation and Optimization::: Techniques for efficient execution of dynamically typed languages ... Techniques for developing or targeting custom or special-purpose targets ... Code generation for emerging programming models ... Code transformations for energy efficiency ... New or improved optimization algorithms, including profile-guided and feedbackdirected optimization ... Techniques for measuring and tuning optimization effectiveness ... Intermediate representations enabling more powerful or efficient optimization		...		Parallelism::: Language features and runtime support for parallelism ... Transformations for heterogeneous or specialized parallel targets, e.g. GPUs ... Data distribution and synchronization ... Virtualization support for multicore and/or heterogeneous computing ... Thread extraction and thread level speculation		 ...		Static and Dynamic Analysis::: Profiling and instrumentation for power, memory, throughput or latency ... Phase detection and analysis techniques ... Efficient profiling and instrumentation techniques ... Program characterization methods targeted at program optimization ... Profile-guided optimization and re-optimization		...	OS, Architecture and Runtime support::: Architectural support for improved profiling, optimization and code generation ... Integrated system design (HW/OS/VM/SW) for improved code generation, including custom or special-purpose processors ... Memory management and garbage collection		...		Security and Reliability::: Code analysis and transformations to address security or reliability concerns	...	Practical Experience::: Real dynamic optimization and compilation systems for general purpose, embedded system and HPC platforms
%		\item IEEE Real-Time and Embedded Technology and Applications Symposium (RTAS 201X, which is part of Cyber-Physical Systems week / CPSWeek) -- Apr % infrastructure, system support, or theoretical foundations for real-time or embedded computing	...	networks of embedded computers ... real-time communication ... real-time resource management and scheduling ... operating system and middleware support for real-time or embedded systems ... energy and temperature management ... QoS management ... multimedia embedded systems ... security, dependability and reliability for real-time embedded systems ... real-time system modeling and analysis ... composability ... control theoretical approaches and performance feedback control ... formal methods, WCET analysis ... software engineering and programming methodologies for real-time embedded systems ... distributed real-time information and database systems	...	Hardware/Software Integration and Co-Design: This track focuses on design methodologies and tools for hardware/software integration and co-design of modern embedded systems for real-time applications. Such systems are increasingly complex and heterogeneous, both in terms of architectures and applications they need to support, so new approaches aimed at their efficient design and optimization are in great demand. General topics relevant to this track include various architecture- and software-related issues of embedded systems design which include, but are not limited to, architecture description languages and tools, WCET analysis, software architectures, design space exploration, synthesis and optimization. Of special interest are SoC design for real-time applications, special purpose functional units, specialized memory structures, multi-core chips and communication aspects, FPGA simulation and prototyping, software simulation and compilation for novel architectures and applications, as well as power, timing and predictability analyses.
%		\item International Conference on Hybrid Systems: Computation and Control (HSCC'1X) -- Apr % embedded reactive systems involving the interplay between symbolic/discrete and continuous dynamical behaviors... Applications and theoretical advancements in the analysis, design, control, optimization, and implementation of hybrid systems (or hybrid dynamical systems)... Models of heterogeneous systems ... Computability and complexity issues ... Real-time computing and control ... Embedded and resource-aware control ... Computation and control over wireless networks ... Mobile robotic networks ... Tools for analysis, verification, control, and design ... Programming languages support and implementation ... Applications, including automotive, communication networks, avionics, energy systems ... transportation networks, biology and other sciences, manufacturing and robotics
%		\item Southern Programmable Logic Conference (SPL201X) -- Apr % Design Methodology: Low-Power Design ... High-speed Techniques ... Physical Design ... Dynamic reconfiguration ... Interconnects and NoCs	...	EDA Tools: Logic and Architectural Synthesis ... Modeling and Simulation Emulation ... CAD for reconfigurable architectures ... Reconfigurable hardware design languages ... System-level design methods ... Testing, verification and benchmarking ... Hardware/software co-design	...	Reliable Embedded Applications: Design verification and validation ... Reliability and fault tolerance ... FIT rates analysis ... High reliability processor cores ... Noise, radiation effects and EMC
%		\item International conference on Design \& Technology of Integrated Systems in nanoscale era (DTIS'1X) -- Apr % Integrated System Design: Low Voltage and Low Power systems ... Synthesis (physical, logic) ... Simulation, Validation and Verification ... 3D integration	...	Integrated System Technology: Device modeling ... Material characterization ... Failure analysis	...	Integrated System Testing: Defect and fault modeling ... Analog and Mixed Signal testing ... MEMS/MOEMS testing ... SOC and SIP testing ... Delay testing ... Memory testing ... Fault Simulation, ATPG ... DFT, BIST and BISR ... On-line testing and fault tolerant systems ... ATE issues ... Alternative test strategies
%		\item International ICST Conference on Practice and Theory of Algorithms in (Computer) Systems (TAPAS 201X) -- Apr % the use, design and evaluation of algorithms for combinatorial optimization problems (either efficient optimal or efficient approximation algorithms) and to real-world applications, engineering and experimental analysis of algorithms	...	Areas of communication and computing that include: coverage, mobility, routing, cooperation, capacity planning, scheduling, and power control	...	Present significant case studies in theoretical and experimental analysis and evaluation of algorithms with specific areas including communications networks, combinatorial optimization and approximation, parallel and distributed computing, computer systems and architecture, economics, game theory, social networks and the world wide web
%		\item International Conference on Thermal, Mechanical and Multiphysics Simulation and Experiments in Micro-Electronics and Micro-Systems (EuroSimE 201X) -- Apr / Multi-physics simulation (thermal, mechanical, thermo-mechanical, coupled thermo-fluidic, coupled electro-mechanics, fluid-structure interactions�) ... Failure analysis and failure mode extraction ... Material characterisation, experiments and modelling ... Validation of simulations by experiments ... Failure criteria and damage-modelling for reliability prediction ... Integrated process modelling ... Advanced numerical and analytical simulation methodologies and tools ... Behavioural modelling ... Simulation-based optimisation, virtual prototyping in product and/or process design ... Compact modeling and model order reduction ... Components and packaging (flip-chip, BGA, CSP, Wafer-Level packages, MCM, Cu/low-k packages) ... Opto-electronic packages ... High temperature and high power packaging ... Piezoelectric components ... Packaging for harsh environments ... ``More than Moore'' applications, such as various SiP, Microsystems, MEMS, sensors actuators, and MOEMS ... Wafer processing, chip design and reliability ... Nano-electronics ... PWB design ad applications ... Bio-electronics, bio-MEMS/NEMS and microfluidics 
%		\item IEEE International Systems Conference 201X (SysCon 201X) -- Apr % Engineering of Complex Systems ... Systems-of-systems ... Systems Engineering ... Systems Integration ... Systems Thinking ... issues and complexities of system-level and system-of-systems applications ... Applications-oriented topics on large-scale systems and system-of-systems in topics noted below ... Systems engineering, education, standards, processes and methodologies for the system-of-systems environment ... Research opportunities and results relating to system-of-systems ... System Architecture and Architectural Frameworks ... Engineering Systems-of-Systems ... Risk Management of Complex Systems Environment ... Systems Reliability ... Engineering Processes for Complex Systems ... Includes Process Improvement and Quality Management ... Product Lifecycle Management Processes and Tools for Systems-of-Systems ... Includes Configuration Management (CM), Requirements management, Data  Management Strategy (CMS) and Integrated Logistics Support ... Service Oriented Architectures ... Comprehensive Cyber Security Approaches for Complex Systems ... Enterprise Systems Engineering ... Agile Development Methods of System-of-Systems ... Modeling and Simulation ... Model-Based Systems Engineering ... Systems Verification and Validation ... Systems Engineering Competency, Education and Training ... Program/Project Management for Complex Systems ... ``Systems thinking'' Benefits ... Technology Transfer Between Academia and Industry ... Societal and Political Impacts of Systems and Systems Design ... Diagnostics, Prognostics, and Enterprise Health Management ... Research in Systems Engineering ... Software Systems Engineering ... Autonomous Systems ... Energy Management and Sustainability, including Renewable Energy ... Space and Communications Systems ... Medical Systems ... Gaming and Entertainment Systems ... Transportation Systems ... Sensors Integration and Application for a Net-centric Environment ... Disaster response ... Global Earth Observation ... Large-Scale Systems Integration (in any application area
%		\item International ICST Conference on Nano-Networks (Nano-Net 201X) -- Apr % Nano-Bio Systems and Applications: New paradigms linking nanotechnology and biology for nano-networks and applications, bio-inspired circuits and architectures, reconfigurable nano-bio systems, nano-mechatronics, nano-robotics, nano-sensors, molecular motors, in-body nanonetworks, nano-bio networks with molecular cells/machines, in-vivo nanosensing.	...	Nano Modeling and Simulations: Physical characterization and modeling of nano-devices, interconnects, statistical mechanics modeling, power/thermal modeling in nano-devices and systems, modeling of nano-bio channels, modeling and simulation of nano/bio networks and systems.		...	Nano-electronics/devices/materials for Communications: Emerging nano-devices and fabrication technologies for nanonetworks, CNTs, nanowires, nanoparticles, GNRs, graphene devices, molecular processing and self-assembly, nano-patterning, emerging 3D-interconnects, nano-materials, coatings and surfaces, dendrimers, energy storage, catalysis, optical modulators and switches, optical and wireless interconnects, quantum electronics, molecular electronics.	...	Theoretical Aspects of Nano-Bio Networks: Information theory for (natural) bio systems, modeling of nano/bio communication channels, capacity bounds and theorems for various nano/bio channels, intercellular communication, transceiver and modulation optimization, reliability and fault tolerance, molecular sensing and sampling, information processing for nano-bio networks, self-organization in nano-bio systems, network calculus and analysis for nano-bio networks.	...	Nano-Bio Computing: Nanoscale processor and memory architectures, new computing paradigms, array processing nano-fabrics, routing and addressing issues in nanonetworks, NoC performance and trade-off analysis, molecular computing, quantum computing.	...	Nano-Bio Networking: Network architectures, topologies, and communication algorithms for nano-bio networks, molecular communication in bio-networks, synchronization, routing/addressing, error control, energy efficiency, nanoscale sensor networks, networks of micro/nanorobotic systems, nanoscale optical, wireless, and, quantum communication networks.
%		\item International Workshop on Formal Methods for Globally Asynchronous Locally Synchronous Design (FMGALS) -- Apr % includes formal verification and design automation
%		\item (Annual) IEEE Symposium on Field-Programmable Custom Computing Machines (FCCM'1X) -- Apr / Architecture of reconfigurable computing devices and systems ... Languages, compilation techniques, tools, and environments for programming and run time support of custom computing machines ... Applications of reconfigurable computing, including the use of reprogrammable logic in scientific computation, mobile communications, medical image processing, data and communication security, network infrastructure and other embedded systems ... Implications and effects of nanotechnology and reconfigurable computing on each other ... Possible forms and system implications of reconfiguration for fault tolerance and avoidance ... Novel use of reconfigurability, including evolvable hardware and adaptive computing ... Spatial computing architectures, languages, compilers, and applications that, like FPGAs and other reconfigurable hardware, use many relatively simple components together for massive parallelism ... Comparisons between reconfigurable implementations and custom multiprocessor implementations (such as NVIDIA�s popular CUDA) ... Hybrid architectures that combine one or more of the above with one or more general-purpose or embedded processors. 
%		\item Annual Electronic Design Process Symposium (EDPS 201X) -- Apr % Includes EDA topics
%		\item IEEE Symposium on Design and Diagnostics of Electronic Circuits and Systems (DDECS) -- Apr % Design Verification/Validation ... Formal Methods in System Design ... Hardware/Software Co-Design ... Logic Synthesis ... Physical Design ... Design and Test in Nano-Technologies ... Reconfigurable Computing ... Network-based Collaborative Design ... Analog, Mixed-Signal, and RF Test ... SoC Test ... Built-in Self-Test and Self-Repair ... Design for Testability and Diagnosis ... Defect/Fault Tolerance and Reliability ... On-line Testing ... Embedded Systems Testing ... Memory, Processor Testing ... MEMS Testing ... ATE Hardware and Software ... Dependable HW / SW Systems
%		\item International Symposium on VLSI Design, Automation and Test (VLSI-DAT) -- Apr % Modeling and simulation; Hardware-software co-design; Logic and architecture synthesis; Physical design and verification; Design for manufacturability; Power estimation and optimization; Design verification; Test generation and fault simulation; BIST and design for testability; RF, analog and mixed-signal test; SOC and system level testing; System level design automation; Electronic System Level Design
%		\item Great Lakes Symposium on VLSI (GLSVLSI) -- Apr/May % hardware/software co-design, logic and behavioral synthesis, logic mapping, simulation and formal verification, layout (partitioning, placement, routing, floorplanning, compaction)
%		\item International Review of Progress in Applied Computational Electromagnetics (ACES) -- Apr / computational electromagnetics and numerical methods
%		\item Annual IEEE/SEMI\textregistered\ Advanced Semiconductor Manufacturing Conference (ASMC) - International Symposium on Semiconductor Manufacturing (ISSM) ... new conference in 2010: (ASMC-ISSM) -- May / Advanced Equipment Processes and Materials ... Defect Inspection ... Cost Effectiveness and Manufacturing Efficiency ... Lithography Advances and DFM ... Factory Automation and Dynamics / Industrial Engineering ... Interactive session ... Virtual Metrology ... Contamination Free Manufacturing ... Advanced Metrology ... Yield Learning ... Advanced Process Control ... Yield Methodologies
%		\item IEEE Workshop on Signal Propagation on Interconnects (SPI 201X) -- May / interconnect modeling, simulation and measurement at chip, board, and package level ... Frequency Domain Measurement Techniques ... Coupling Effects on Interconnects ... Time Domain Measurement Techniques ... Substrate Effects ... Modeling Techniques of Package & On-Chip Interconnects ... Guided Waves on Interconnects ... Macro-Modeling ... Radiation & Interference ... Simulation Techniques for Interconnect Structures ... Electromagnetic Compatibility ... Electromagnetic Field Theory ... Power/Ground-Noise ... Analysis and Modeling of Power Distribution Networks ... Testing & Interconnects ... Propagation Characteristics on Transmission Lines ... Optical Interconnects ... RF and Microwave Interconnects ... Wireless Interconnects
%		\item International Spring Seminar on Electronics Technology (ISSE 201X) -- May / New Materials and Processes ... Innovations in Film Technologies and LTCC ... Advanced Packaging and Interconnection Technologies ... Thermal Characterization, Reliability and Quality ... System Modelling and Simulation ... Bioelectronics, Environmental and Ecology in Electronics Technology ... Nanotechnology, Nanomaterials and Nanoelectronics ... Educational and Information Activities in Electronics Technology
%		\item IEEE VLSI Test Symposium (VTS 201X) -- May
%		\item International Symposium on Embedded Systems Design and Applications (ESDA 201X, in Romania) -- May % Analogue and digital electronic systems. ... Embedded systems, HW and SW: Novel Architectures and Micro-architectures for Embedded Systems; Application-specific and Domain-specific Embedded Systems; Signal processing, HW and SW; Intelligent Control and Fuzzy Systems. ... Networked Embedded Systems: Design Issues for Networked Embedded; Design and Implementation for Networked Embedded Systems; Self Adaptive Networked Entity Sensor Networks: Architectures, Algorithms, Routing, Security; ... Embedded Applications: Intelligent Sensors, Industrial Automation and Controls; Automotive Applications; Industrial Building Automation and Control; Power Station Automation and Control. ... Modeling, Specification and Programming Languages. ... System-on-Chip Design & Testing: Hardware/Software Co-design, SoC Communication and Architectures; Platform-Based Design for SoC Systems; Reconfigurable Platforms. ... Electronics in measurement and control. 
%		\item IEEE Computer Society Annual Symposium on VLSI (ISVLSI) -- May % Test and Verification, Physical design, Architecture-Level Design Solutions, System Level Design, System-on-a-Chip Design, Heat Dissipation, Power Awareness in VLSI Design, Electrical/Packaging Co-Design
%		\item IEEE International Symposium on Circuits and Systems (ISCAS 201X) -- May % Computer?Aided Network Design ... VLSI Systems and Applications
%		\item Reconfigurable Architectures Workshop (RAW 201X) -- May % Run-Time Reconfiguration & Adaptive Computing: Architectures, Algorithms, Technologies ... Run-Time and Dynamic Reconfiguration: reconfiguration handling and speed ... FPGAs and new coarse-/multi-grain devices ... run-time reconfiguration (RTR) ... Models & Architectures: Theoretical Interconnect and Computation Models ... RTR Models and Systems ... RTR Hardware Architectures ... Optical Interconnect Models ... Simulation and Prototyping ... Bounds and Complexity Issues	...	Algorithms & Applications: Algorithmic Techniques ... Mapping Parallel Algorithms ... Distributed Systems & Networks ... Fault Tolerance Issues ... Wireless and Mobile Systems ... Automotive Applications ... Infotainment & Multimedia ... Biology Inspired Applications	...	Design, Technologies & Tools: Configurable Systems-on-Chip ... Energy Efficiency Issues ... Devices and Circuits ... Reconfiguration Techniques ... High Level Design Methods ... System Support ... Adaptive Runtime Systems ... Organic Computing
%		\item International Symposium on Asynchronous Circuits and Systems (ASYNC) -- May % CAD tools for asynchronous design, synthesis, analysis, $[$verification$]$, and optimization ... Physical design of asynchronous logic and pipelines ... Formal methods for correctness, and performance/power analysis ... Design models and methods for asynchronous buses, networks on chip (NoC) and system-on-chip (SoC) interconnects
%		\item ACM Great Lakes Symposium on VLSI (GLVLSI) -- May % Computer-Aided Design (CAD): hardware/software co-design, logic and behavioral synthesis, logic mapping, simulation and formal verification, layout (partitioning, placement, routing, floorplanning, compaction), algorithms and complexity analysis ... Low Power and Power Aware Design: circuits, micro-architectural techniques, thermal estimation and optimization, power estimation methodologies, and CAD tools ... Testing, Reliability, Fault-Tolerance: digital/analog/mixed-signal testing, design for testability and reliability, online testing techniques, static and dynamic defect- and fault-recoverability, and variation-aware design ... Emerging Technologies: nanotechnology, molecular electronics, quantum devices, biologically-inspired computing, CNT, SET, RTD, QCA, VLSI aspects of sensor and sensor network, and CAD tools for emerging technology devices and circuits ... Post-CMOS VLSI: evolutionary computing, optical computing, quantum computing, reversible logic, spin-based computing, biological computation, nanotechnology, molecular electronics, quantum devices, biologically-inspired computing. Emphasis should be on the analysis, novel circuits and architectures, modeling, CAD tools, and design methodologies.
%		\item International Symposium on Networks-on-Chip (NOCS) -- May % Network architecture (topology, routing, arbitration) ... Verification, debug & test of NoC ... Modeling, simulation, and synthesis of NoCs ... Mapping of applications onto NoCs ... Power and energy issues ... Timing, synchronous/asynchronous communication ... Floorplan-aware NoC architecture optimization ... Physical design of interconnect and NoC ... Novel interconnect links/switches/routers ... NoC support for CMP/MPSoCs ... NoCs for FPGAs and structured ASICs ... NoC design methodologies and tools ... Architecture Description Language (ADL)
%		\item International Electrostatic Discharge Workshop (IEW) -- May / TCAD and SPICE to address electrostatic discharge 
%		\item International ICST Conference on Performance Evaluation Methodologies and Tools (Valuetools 201X) -- May % Performance evaluation techniques:::: Advanced simulation tools: Simulation of rare events ... Parallel/distributed simulations ... Variance reduction techniques ... Large deviations ... Hybrid system simulation		...	Control Theory: Optimal control ... Multiobjective optimization and control ... Stochastic control ... Robust control and stabilization ... Model predictive control ... Hybrid and switched systems ... control and performances ... Differential games control	...	Discrete event systems: Petri nets ... Max-plus algebra ... Automata, timed automata ... learning automata ... weighted automata	...	Large system performance analysis: Random matrix theory ... Mean field theory ... Types	...	Learning: Learning automata ... Learning theory ... Machine learning ... ODE approximations, stochastic differential equations	...	Monte Carlo methods: Markov chain Monte Carlo methods ... Sequential Monte Carlo methods	...	Queueing theory: Markovian queues and networks ... Network calculus ... Analytical methods ... Approximation methods ... (In)Sensitivity ... Dynamic Fluid Models ... Diffusion models ... Dam processes ... Perturbation approaches ... Control of Queues	...	stochastic models: Stochastic geometry ... Long-range dependence ... Self-similarity ... Point processes ... Traffic models and measurements		...	Applications:::: Communication networks ... Computer networks (including peer-to-peer systems and traffic control services) ... Computer systems (e.g., grid computing) ... Distributed systems (e.g., distributed control and distributed communication networks) ... Game theory ... Interdisciplinary methodologies (economic, biological and social models) ... Limit performance of detection, estimation, decoding techniques (e.g., Cramer-Rao bounds, Shannon limits) ... Machine learning ... Manufacturing systems and supply chains ... Network information theory ... Optimization ... Resource allocation ... Road traffic and transportation systems ... Secure Networks ... Public utility networks ... Telecommunications
%		\item ACM International Conference on Computing Frontiers (Computing Frontiers 201X) -- May % Innovations in theory, methodologies, technologies, and implementation of advanced computer systems ... Computing paradigms, computational models, application paradigms, computer architecture, development environments, compilers, or operating environments ... Accelerators: manycore, GPU, custom, reconfigurable, embedded, and shybrid ... Active libraries, domain-specific languages and generative techniques ... Defect and variability-tolerant designs ... Power and energy efficiency in architectures, compilers and algorithms ... Compilers and operating systems - adaptive, run-time and autotuning ... Workload characterization of emerging applications ... System management and security ... Computational neuroscience, neuromorphic and biologically-inspired architectures ... Novel frontiers in computational science and scientific data repositories ... Reconfigurable, autonomic, organic and self-organizing computation and systems
%		\item IEEE International Symposium on Multiple-Valued Logic (ISMVL 201X) -- May // \url{http://www.lsi-cad.com/ismvl/} % Algebra and Formal Aspects; ATPG and SAT; Automatic Reasoning; Circuit/Device Implementation; Communication Systems; Computer Arithmetic; Data Mining; Fuzzy Systems and Soft Computing; Image Processing; Logic Design and SwitchingTheory; Logic Programming; Machine Learning and Robotics; Mathematical Fuzzy Logic; Nano Technology; Philosophical Aspects; Quantum Computing; Signal Processing; Spectral Techniques; Verification
%		\item International Workshop on Post-Binary ULSI Systems (ULSI 201X) -- May % New computing paradigms for the post-CMOS and post-digital-logic era
%		\item Reed-Muller 201X Workshop -- May // \url{http://www.lsi-cad.com/RM/index.html} % AND-EXOR based representations that are simpler than standard AND-OR representations... Decision diagrams for synthesis, analyses, and verification... Spectral transformations to detect the properties of logic functions	...	Graph-based representations of logic functions: BDD, MDD, BMD, EVBDD...	Spectral representation of logic functions...	Representations for quantum computing, nanotechnology, and molecular scale computing...	Implementation in silicon (FPLDs and FPGAs)...	Graph functions, bent functions, and cryptographic applications...	EXOR-based logic representations...	Applications, including circuit design, reversible logic, quantum logic, etc. 
%		\item IEEE European Test Symposium (ETS '11) -- May % Automatic Test Generation ... Fault Modeling and Simulation ... Current-Based Test ... Power Issues in Test ... Thermal Test ... Delay and Performance Test ... High-Speed IO/Interconnect Test ... Signal Integrity Test ... Nanometer Technologies Test ... ATE Hardware and Software ... Standards in Testing ... Test(ability) Synthesis ... Built-In Self Test (BIST) ... Design for Test(ability) (DfT) ... Test Data Compression ... On-Line Test ... Self-Repair Methodologies ... Test of Reconfigurable Systems ... Analog, Mixed-Signal, RF Test ... Memory Test and Repair ... Microprocessor Test ... MEMS and Nanotechnology Test ... Failure Analysis ... Diagnosis and Debug ... Design Verification and Validation ... Test Quality and Reliability ... Yield Analysis and Enhancement ... Defect and Fault Tolerance ... Board and System Test ... (Embedded) System Test ... High-Level DfT and TPG ...	 System-in-Package (SiP) Test ... System-on-Chip (SoC) Test
%		\item ACM Symposium on Theory of Computing (STOC 201X) -- May/Jun % algorithms and data structures, computational complexity, cryptography, privacy, computational geometry, algorithmic graph theory and combinatorics, randomness in computing, parallel and distributed computation, machine learning, applications of logic, algorithmic algebra and coding theory, computational biology, computational game theory, quantum computing and other alternative models of computation, and theoretical aspects of areas such as databases, information retrieval, and networks
%		\item ACM Symposium on Parallelism in Algorithms and Architectures (SPAA '1X): -- May/Jun % Parallel and Distributed Algorithms ... Parallel and Distributed Data Structures ... Green Computing and Power-Efficient Architectures ... Management of Massive Data Sets ... Parallel Complexity Theory ... Parallel and Distributed Architectures ... Multi-Core Architectures ... Instruction Level Parallelism and VLSI ... Compilers and Tools for Concurrent Programming ... Supercomputer Architecture and Computing ... Transactional Memory Hardware and Software ... The Internet and the World Wide Web ... Game Theory and Collaborative Learning ... Routing and Information Dissemination ... Resource Management and Awareness ... Peer-to-Peer Systems ... Mobile Ad-Hoc and Sensor Networks ... Robustness, Self-Stabilization and Security ... Synergy of parallelism in algorithms, programming and architecture
%		\item ACM International Conference on Measurement and Modeling of Computer Systems (ACM SIGMETRICS 201X) -- May/Jun % The development and application of analytic, simulation, and measurement-based evaluation techniques. Of particular interest is research that furthers the state-of-the-art in evaluation methods or that creatively applies existing methods to investigate key design tradeoffs in computer or network systems.	...	Methodologically-oriented design and evaluation studies of: Network architectures, protocols, and algorithms ... Wireless, mobile, ad-hoc, and sensor networks ... Computer architectures, memory systems, and storage systems ... Operating systems, file systems, and databases ... Virtualization ... Distributed and cloud computing ... Social networks, Internet servers, multimedia systems, and web services ... Energy-efficient computing systems ... Emerging technologies ... Mobile and personal computing systems ... Real-time systems, fault-tolerant systems, and language systems ... Security systems and network attacks ... Large-scale operational systems	...	Methodologies, evaluation techniques, and algorithms for: Performance, power, and reliability analysis ... Capacity planning, resource allocation, scheduling, QoS, and pricing ... Anomaly detection ... Analytic modeling, model verification, and validation ... System measurement, monitoring, and forecasting ... Workload characterization and benchmarking ... Design of experiments ... Statistical analysis, simulation, and signal processing
%		\item International ACM Symposium on High-Performance Parallel and Distributed Computing (HPDC 201X) -- Jun % clusters, grids, clouds, and parallel and multicore computers ... Applications of parallel and distributed computing. ... Systems, networks, and architectures for high end computing. ... Parallel and multicore issues and opportunities. ... Virtualization of machines, networks, and storage. ... Programming languages and environments. ... I/O, file systems, and data management. ... Data intensive computing. ... Resource management, scheduling, and load-balancing. ... Performance modeling, simulation, and prediction. ... Fault tolerance, reliability and availability. ... Security, configuration, policy, and management issues. ... Models and use cases for utility, grid, and cloud computing.
%		\item Annual ACM SIGACT-SIGOPS Symposium on Principles of Distributed Computing (PODC 201X) -- May/Jun % the theory, design, specification and implementation of distributed systems ... distributed algorithms: design, analysis, and complexity ... communication networks: architectures, services, protocols, applications ... multiprocessor and multi-core architectures and algorithms ... shared and transactional memory, synchronization protocols, concurrent programming ... fault-tolerance, reliability, availability, self organization ... Internet applications, social networks, recommendation systems ... distributed operating systems, middleware platforms, databases ... game-theoretic approaches to distributed computing ... peer-to-peer systems, overlay networks, distributed data management ... high-performance, cluster, cloud and grid computing ... wireless networks, mobile computing, autonomous agents ... context-aware distributed systems ... security in distributed computing, cryptographic protocols ... sensor, mesh, and ad hoc networks ... specification, semantics, verification, and testing of distributed systems
%		\item ACM SIGACT/SIGMOBILE International Workshop on Foundations of Mobile Computing (DIALM-POMC 201X) -- Jun/Sep % Channel assignment and management. ... Energy saving methods and protocols ... Routing, multicast and broadcast ... Gossiping, information diffusion ... Scheduling ... Synchronization and discovery ... Emerging networks: including cognitive, delay-tolerant, ad hoc and sensor networks ... Localization and location tracking, handover/handoff ... Network protocols: design, optimization, and analysis ... Local algorithms, algorithms with incomplete knowledge ... Distributed optimization algorithms ... Dynamic networks, dynamic graph algorithms ... Selfish behavior, incentives, and cooperation ... Modeling ... MAC layer protocols ... Network capacity ... Location- and context-aware distributed systems ... Cryptography and security
%		\item IEEE Conference on Computational Complexity -- Jun % The conference seeks original research papers in all areas of computational complexity theory, studying the absolute and relative power of computational models under resource constraints. Typical models include deterministic, nondeterministic, randomized, and quantum models; uniform and nonuniform models; Boolean, algebraic, and continuous models. Typical resource constraints involve time, space, randomness, program size, input queries, communication, and entanglement; worst-case as well as average case. Other, more specific, topics include: probabilistic and interactive proof systems, inapproximability, proof complexity, descriptive complexity, and complexity-theoretic aspects of cryptography and machine learning.
%		\item ACM SIGPLAN Symposium on Programming Language Design and Implementation (PLDI 201X) -- May/Jun % programming languages: design, implementation, development, and use ... compile-time and runtime technology ... novel language designs and features ... Language designs and extensions ... Static and dynamic analysis of programs ... Domain-specific languages and tools ... Type systems and program logics ... Program transformation and optimization ... Checking or improving the security or correctness of programs ... Memory management ... Parallelism, both implicit and explicit ... Performance analysis, evaluation, and tools ... Novel programming models ... Debugging techniques and tools ... Program understanding ... Interaction of compilers and run-time systems ... with underlying systems
%		\item ACM SIGPLAN Workshop on Programming Languages and Analysis for Security (PLAS 201X) -- May/Jun % use programming language and program analysis techniques to improve the security of software systems. Strongly encouraged are proposals of new, speculative ideas; evaluations of new or known techniques in practical settings; and discussions of emerging threats and important problems. ... Language-based techniques for security ... Verification of security properties in software ... Automated introduction and/or verification of security enforcement mechanisms ... Program analysis techniques for discovering security vulnerabilities ... Compiler-based security mechanisms, such as host-based intrusion detection and in-line reference monitors ... Specifying and enforcing security policies for information flow and access control ... Model-driven approaches to security ... Applications, examples, and implementations of these security techniques
%		\item International Supercomputing Conference (ISC'10) -- May/Jun % Innovative Architectures: GPGPU and accelerator-based systems ... Multicore/Manycore systems	...	Applicability of New Programming Concepts: PGAS ... OpenCL	...	Data Management and Storage Systems: HSM/ILM for HPC ... Tape libraries ... I/O workflow
%		\item CIAO! Doctoral Consortium Workshop -- Jun / for Ph.D. students in the CIAO! initiative; see \url{http://www.ciaonetwork.org/}
%		\item Electronic Components and Technology Conference (ECTC) -- Jun / Advanced Packaging ... Interconnections ... Applied Reliability ... Materials \& Processing ... Assembly and Manufacturing Technology ... Electronic Components \& RF ... Optoelectronics ... Emerging Technologies ... Modeling & Simulation: Electrical, thermal, optical, and mechanical modeling, simulation, and characterization of packaging solutions including system-level applications. Example topics include assembly manufacture modeling, Cu low-K interconnects, drop impact models, embedded passives, equivalent circuit models, fullwave modeling, lead-free solders, macromodeling, measurements, and thermo-mechanical reliability.
%		\item IEEE MTT-S International Microwave Symposium (IMS) -- Jun / numerical methods for computational electromagnetics in the frequency and time domains, while modeling physical behaviors such as those that are electromagnetic, semiconductor, thermal, and mechanical, linear and nonlinear device modeling, parameter extraction of devices, nonlinear circuit analysis and system simulation (harmonic balance, distortion and spurious analysis, and behavior modeling)	...	Frequency-Domain EM Analysis Techniques: Frequency-Domain methods for numerical solution of electromagnetic problems, including field interactions with devices, Circuits and with other physical processes	...	Time-Domain EM Analysis Techniques: Time-Domain methods for numerical modeling of high frequency electronics, including modeling based on physical behaviors (electromagnetic, semiconductor, thermal, mechanical)	...	CAD Algorithms and Techniques: Circuit analysis methods, optimization methods, statistical analysis	...	Linear Device Modeling: Linear models of active and passive devices, models	...	Nonlinear Device Modeling: Large-signal device models, characterization, parameter extraction, validation	...	Nonlinear Circuit and System Simulation: Harmonic balance, simulation techniques, distortion and spurious analysis, system simulations and behavioral modeling
%		\item European Microelectronics and Packaging Conference \& Exhibition (EMPC 201X) --  Jun / Evolution of Current Packaging Technologies: Flip Chip packaging, Embedded, Pop/Sip packages, Small form Factor ... Optoelectronic Packaging ... Materials: Solder Joint Properties and reliability ... Reliability ... Microfluidic, Medical Device Packaging ... Ceramic Packages Including LTCC ... Assembly Technology: Wirebonding ... Advanced Packaging: wafer level Package ... High Power Packaging Technology: Thermal modeling and Management ... Electrical simulation ... Sensor Packaging ... Materials: Adhesives Properties and reliability ... 3D packaging ... Substrates and Interconnects ... MEMS System Packaging
%		\item IEEE Radio Frequency Integrated Circuits Symposium (RFIC 201X) -- Jun % Modeling and CAD: RFIC Modeling, Characterization of Active and Passive Devices
%		\item International Symposium on Memory Management (ISMM 201X) -- Jun % Memory allocation and deallocation ... Garbage collection algorithms and implementations ... Compiler analyses and tools to aid memory management ... Empirical analysis of heap intensive programs ... Formal analysis and verification of heap intensive programs ... Memory system design and analysis ... Verification of memory management algorithms ... Development and evaluation of open source implementations	...	garbage collection, dynamic storage allocation, storage management implementation techniques, plus interactions with languages and operating systems, and empirical studies of programs' memory allocation and referencing behavior
%		\item ARFTG Microwave Measurement Symposium -- Jun \& Nov/Dec // Automatic RF Techniques Group (ARFTG) % 
%		\item Annual IEEE International Mixed-Signals, Sensors, and Systems Test Workshop (IMS3TW'1X) -- Jun % Test & Design for (on/off-line) Test ... Reliability & Design for Reliability ... Fault and Error Modelling & Simulation ... Verification & Design for Verification ... Monitoring/Diagnosis & Design for Debug/Diagnosis ... Fault Tolerance	...	Pertaining to the following systems or underlying technologies: Analog/Mixed-Signal Circuits ... Biomedical Circuits & Systems ... RF & Wirelessly Controlled Devices ... Optoelectronics & Photonics ... Drug Delivery Microsystems ... Lab-on-Chip ... MEMs ... Microfluidics ... Heterogeneous Systems ... Implantable Devices
%		\item International Symposium on Computer Architecture (ISCA 201X) -- Jun % Processor, memory, and storage systems architecture ... Interconnection networks ... Instruction, thread, and data-level parallelism ... Dependable architectures ... Architecture support for security ... Power and energy efficient architectures ... Application specific, reconfigurable, and embedded architectures ... Network processor and router architectures ... Architectures for emerging technologies and applications ... Architecture modeling and performance evaluation
%		\item Mapping Applications to MPSoCs -- Jun / Unfortunately, not much is known about applicable mapping techniques. Mapping could start from task graphs, sequential code or models using other models of computation. Mapping from sequential code requires automatic parallelization techniques. Parallelization techniques designed for high-performance computing are not always applicable, due to the heterogeneity and since memory access times and communication times are substantially different for MPSoCs. 
%		\item UML \& AADL -- Jun / Multi-domain specific modeling languages ... Model transformation and generative approaches ... Model-based Methodologies ... Integration of different formalisms (e.g., Simulink/StateFlow, StateMate and Scade-drive) ... Model Checking of architecture specifications ... ADLs behavioral models simulation, Scheduling analysis and Worst-case execution time prediction ... distributed, real-time and embedded systems (DRE) ... Architecture Analysis and Design Language (AADL)
%		\item International Conference on Autonomic Computing (ICAC 201X) - Jun // \url{http://www.autonomic-conference.org/} // % Applications of autonomic computing in: Enterprise applications ... Clouds and grids ... Internet services ... Data center or large-scale system management ... Embedded and mobile systems ... Energy management ... Sensor networks, especially issues related to autonomous, distributed management ... Internet of things	...	Autonomic computing components and services: protocols, system-level support, services, or application components that enhance aspects of system autonomy, self-management, self-tuning, self-configuration, self-diagnosis, and self-healing, or improve adaptive capabilities. Examples include: Autonomic management of resources, workloads, faults, power/thermal, and other challenges. ... Management of quality of service, including security and dependability ... Self-managing components, such as servers, storage, network protocols, or specific application elements ... Monitoring systems for autonomic computing ... Virtual machine, operating systems, hardware or application support for autonomic computing ... Novel human interfaces for monitoring and controlling autonomic systems ... Management topics, such as specification and modeling of service-level agreements, behavior enforcement and tie-in with IT governance. ... Toolkits, frameworks, principles and architectures, from software engineering practices and experimental methodologies to agent-based techniques and virtualization.	...	Algorithms, theory and foundations of autonomic computing: Analytic foundations are solicited for building efficient autonomic systems, predicting their behavior, quantifying their performance, analyzing their stability, guaranteeing their specifications, or optimizing their efficacy. These include: Decision and analysis techniques and their use, such as machine learning, control theory, predictive methods, emergent behavior, self-organizing networks, rule-based systems and bio-inspired techniques ... Fundamental science and theory of self-managing systems: understanding, controlling or exploiting system behaviors to enforce autonomic properties ... Algorithms, analysis and theory for performance guarantees ... Foundations of self-diagnostic systems
%		\item IEEE Symposium on Computer Arithmetic (ARITH) -- Jun % Arithmetic processor design and implementation ... Arithmetic algorithms and their analysis ... Highly parallel arithmetic units and systems ... New floating-point units and systems ... Low power arithmetic
%		\item International Symposium on Rapid System Prototyping (RSP) -- Jun % FPGAs , SoCs, NoCs and MPSoCs ... Software/Configware/Hardware codesign and tradeoffs ... System verification/validation ... Software and Systems Architecture and Integration ... Applying formal methods to prototyping
%		\item International Symposium on Symbolic and Algebraic Computation (ISSAC 201X) -- Jun % Algorithmic aspects: Exact and symbolic linear, polynomial and differential algebra. Symbolic-numeric, homotopy, and series methods. Computational geometry, group theory, number theory, quantifier elimination and logic. Summation, recurrence equations, integration, ODE & PDE. Symbolic methods in other areas of pure and applied mathematics. Theoretical and practical aspects, including techniques for important special cases and algebraic complexity.		...	Software aspects: Design of packages and systems, data representation. Parallel and distributed algebraic computing, considerations for modern hardware. User-interface issues, and use with systems for, e.g., digital libraries, courseware, simulation and optimization, automated theorem-proving, computer-aided design, and automatic differentiation.	...	Application aspects: Applications that stretch the current limits of computer algebra, use it in new ways, or apply it in situations with broad impact, in particular to the natural sciences, life sciences, engineering, economics and finance, and education.
%		\item International Symposium on Formal Methods (FM201X; e.g., FM2009, FM2010) -- Jun / Nov 
%		\item International Conference on Application and Theory of Petri Nets and Other Models of Concurrency (Petri Nets) -- Jun % System design and verification ... hardware formal verification using Petri Nets
%		\item ACM SIGPLAN/SIGBED Conference on Languages, Compilers, and Tools for Embedded Systems (LCTES 201X) -- Jun % Programming language issues in embedded systems, including language features to exploit multi-core, single-chip SIMD, reconfigurable architecture and other emerging architectures ... Compiler issues in embedded systems, including optimization for low power, low energy, low code and data size, and high (real-time) performance ... Novel embedded architectures: design and implementation of novel embedded architectures; workload analysis and performance evaluation; and architecture support for new language features, new compiler techniques and debugging tools ... Tools for analysis, specification, design and implementation of embedded systems, including: hardware, system software, and application, and their interface; distributed real-time control, media players, reconfigurable architectures and other complex systems; validation and verification, system integration and testing; timing analysis, timing predictability, WCET analysis and real-time scheduling analysis; performance monitoring and tuning; and runtime system support for embedded systems 
%		\item 201X Symposium on VLSI Technology (part of the 201X Symposia on VLSI Technology and Circuits) -- Jun % Advanced device analysis, materials and modeling for VLSIs	...	Design enablement (including technology impacts on circuit design in advanced CMOS nodes)
%		\item 201X Symposium on VLSI Circuits (part of the 201X Symposia on VLSI Technology and Circuits) -- Jun % Circuit design to address challenges of deeply scaled technologies - e.g. DFM, variability, reliability	...	Digital circuit techniques	...	Complex SOC systems describing new architectures and implementations	...	Circuit approaches for clock generation and distribution	...	Power minimization techniques for analog and digital circuits, including novel energy harvesting, battery management and renewable energy topics
%		\item Mapping Applications to MPSoCs 201X -- Jun // \url{http://www.artist-embedded.org/artist/Overview,1821.html}; \url{http://www.artist-embedded.org/artist/Overview,1615.html}; and \url{http://www.artist-embedded.org/artist/Mapping-of-Applications-to-MPSoCs,1590.html} % Recent technological trends have led to the introduction of multi-processor systems on a chip (MPSoCs). It can be expected that the number of processors on such chips will continue to increase. Power efficiency is frequently the driving force having a strong impact on the architectures being used. As a result, heterogeneous architectures incorporating functional units optimized for specific functions are commonly employed	...	This technological trend has dramatic consequences on the design technology. Techniques are required, which map sets of applications onto architectures of MPSoCs. Unfortunately, not much is known about applicable mapping techniques. Mapping could start from task graphs, sequential code or models using other models of computation. Mapping from sequential code requires automatic parallelization techniques	...	Parallelization techniques designed for high-performance computing are not always applicable, due to the heterogeneity and since memory access times and communication times are substantially different for MPSoCs	...	Recent technological trends have led to the introduction of multi-processor systems on a chip (MPSoCs). It can be expected that the number of processors on such chips will continue to increase. Power efficiency is frequently the driving force having a strong impact on the architectures being used. As a result, heterogeneous architectures incorporating functional units optimized for specific functions are commonly employed. This technological trend has dramatic consequences on the design technology. Techniques are required, which map sets of applications onto architectures of MPSoCs. Unfortunately, not much is known about applicable mapping techniques. Mapping could start from task graphs, sequential code or models using other models of computation. Mapping from sequential code requires automatic parallelization techniques. Parallelization techniques designed for high-performance computing are not always applicable, due to the heterogeneity and since memory access times and communication times are substantially different for MPSoCs	...	The aim of the workshop is to be a focus point for research on mapping applications to MPSoCs. Directions for future research should be proposed and evaluated. The scope also includes timing analysis for MPSoCs. The goal is to provide - in a few years - the techniques required for mapping applications efficiently. The development of new products will be seriously constrained if this goal cannot be reached. 
%		\item International Workshop on Software and Compilers for Embedded Systems (SCOPES 201X) -- Jun/Varies // The influence of embedded systems is constantly growing. Increasingly powerful and versatile devices are developed and put on the market at a fast pace. The number of features is increasing, and so are the constraints on the systems concerning size, performance, energy dissipation and timing predictability. Since most systems today use a processor to execute an application program rather than using dedicated hardware, the requirements can not be fulfilled by hardware architects alone: Hardware and software have to work together to meet the tight constraints put on modern devices.	...	One of the key characteristics of embedded software is that it heavily depends on the underlying hardware. The reason of the dependency is that embedded software needs to be designed in an application specific way. To reduce the system design cost, e.g. code size, energy consumption etc., embedded software needs to be optimized exploiting the characteristics of the underlying hardware.	...	SCOPES focuses on the software generation process for modern embedded systems. Topics of interest include all aspects of the compilation process, starting with suitable modeling and specification techniques and programming languages for embedded systems. The emphasis of the workshop lies on code generation techniques for embedded processors. The exploitation of specialized instruction set characteristics is as important as the development of new optimizations for embedded application domains. Cost criteria for the entire code generation and optimization process include runtime, timing predictability, energy dissipation, code size and others. Since today's embedded devices increasingly consist of a multi-processor system-on-chip, the scope of this workshop particularly covers compilation techniques for MPSoC architectures.	...	In addition, this workshop intends to put a spotlight on the interactions between compilers and other components in the embedded system design process. This includes compiler support for e.g. architecture exploration during HW/SW codesign or interactions between operating systems and compilation techniques. Finally, techniques for compiler aided profiling, measurement, debugging and validation of embedded software are also covered by this workshop, because stability of embedded software is mandatory ... code generation techniques for embedded processors ... Cost criteria for the entire code generation and optimization process include runtime, timing predictability, energy dissipation, code size and others ... compiler aided profiling, measurement, debugging and validation of embedded software
%		\item Intel Research Day -- late June
%		\item International Conference Mixed Design of Integrated Circuits and System (MIXDES 201X) -- Jun /  Design of Integrated Circuits and Microsystems: Design methodologies. Digital and analog synthesis. Hardware-software codesign. Reconfigurable hardware. Hardware description languages. Intellectual property-based design. Design reuse... Thermal Issues in Microelectronics: Thermal and electro-thermal modelling, simulation methods and tools. Thermal mapping. Thermal protection circuits... Analysis and Modelling of ICs and Microsystems: Simulation methods and algorithms. Behavioural modelling with VHDL-AMS and other advanced modelling languages. Microsystems modelling. Model reduction. Parameter identification... Microelectronics Technology and Packaging: New microelectronic technologies. Packaging. Sensors and actuators... Testing and Reliability: Design for testability and manufacturability. Measurement instruments and techniques... Power Electronics: Design, manufacturing, and simulation of power semiconductor devices. Hybrid and monolithic Smart Power circuits. Power integration... Signal Processing: Digital and analogue filters, telecommunication circuits. Neural networks. Artificial intelligence. Fuzzy logic. Low voltage and low power solutions... Embedded Systems: Design, verification and applications... Medical Applications: Medical and biotechnology applications. Thermography in medicine.
%		\item International Conference on Computer Aided Verification (CAV) -- Jun/Jul % Algorithms and tools for verifying models and implementations ... Hardware verification techniques ... Hybrid systems and embedded systems verification ... Deductive, compositional, and abstraction techniques for verification ... Program analysis and software verification ... Verification techniques for security ... Testing and runtime analysis based on verification technology ... Verification methods for parallel and concurrent hardware/software systems ... Applications and case studies ... Verification in industrial practice ... Formal methods for biological systems
%		\item Hardware Verification Workshop (HWVW'1X) -- Jun/Jul % model checking technology and system descriptions ... improvements in reasoning engines related to verification ... formal approaches to synthesis, modelling, and verification ... formal verification technology in synthesis ... applications and case studies in verification
%		\item International SPIN Workshop on Model Checking of Software (SPIN) -- Jun/Jul / model checking-based analysis of software systems ... focus of the workshop is on theoretical advances and empirical evaluations related to state-space and path exploration techniques, as implemented in the SPIN model checker and other software verification tools
%		\item Workshop Exploiting Concurrency Efficiently and Correctly (EC2) -- Jun/Jul / parallel or distributed verification tools for concurrent software
%		\item Workshop on Automated Formal Methods (AFM) -- Jun/Jul / focus on SRI suite of tools (PVS, SAL and HybridSAL, and Yices)
%		\item Workshop on Formal Verification of Analog Circuits (FAC) -- Jun/Jul % Formal verification for analog and mixed-signal circuits
%		\item Workshop on Constraints in Formal Verification (CFV) -- Jun/Jul % application of constraint solvers to hardware verification
%		\item Platforms for Analysis, Design and Verification of Embedded Systems (PADVES) -- Jun/Jul % EDA tools for heterogeneous embedded systems (mechanical/hardware/software parts) ... verification and validation ... adaptive, self-organizing, and self-healing systems
%		\item International Conference on Theory and Applications of Satisfiability Testing (SAT) -- Jun/Jul
		\item Design Automation Conference (DAC) -- Jun/Jul
%		\item International Workshop on Logic and Synthesis (IWLS) -- Jun/Jul
%		\item International Workshop on Bio-Design Automation (IWBDA 201X) -- Jun/Jul % Design methodologies for synthetic biology. ... Standardization of biological components. ... Automated assembly techniques. ... Computer-aided modeling and abstraction techniques. ... Engineering methods inspired by biology.
%		\item ACM/IEEE International Conference on Formal Methods and Models for Codesign (MEMOCODE) -- Jun/Jul % System-level modeling and verification, abstraction and refinement between different modeling levels, formal, semi-formal, and specification-driven verification on the system level. Transaction-level modeling ... Design and verification methods for composition of concurrent systems: Multi-core architectures, networks-on-chip ... System-level estimation of performance and power in heterogeneous hardware/software architectures ... Applications and demonstrators of formal design methodologies and case studies of innovative system-level design flows ... formal methods and tools for hardware and software verification including theorem proving and decision procedures
%		\item IEEE International Workshop on Design for Manufacturability and Yield 201X (DFM\&Y) -- Jun/Jul % Analog and Mixed-Signal DFM ... Test-Based Yield Learning ... Electrical, Design-Driven DFM ... Built-in Repair Analysis and Self-Repair ... Random Defectivity and Critical Area ... Adaptive Design Techniques in DFM/DFY ... Embedded Test and Diagnosis ... OPC and RET ... DFM for 3D Integration ... DFM at System/Architecture Level ... Process Monitoring IP ... Statistical Design ... Design-Aware Manufacturing ... Yield Enhancement IP ... Yield Management
%		\item International Workshop on System Level Interconnect Prediction (SLIP) -- Jun/Jul % Interconnect prediction at high-level synthesis, logic synthesis, or physical layout ... Interconnect architecture of multiple cores or network on chips
%		\item IEEE Symposium on Application Specific Processors (SASP) -- Jun/Jul % EDA for MPSoC and NoC ... Tools, techniques, and algorithms for architectural exploration
%		\item IEEE International Conferences on Embedded Software and Systems (ICESS-1X) -- Jun/Jul %  Systems, Models and Algorithms::: Embedded Hardware and Architectures ... Embedded Software and Agents ... Embedded Real-Time Systems ... Models of Physical Systems ... Power Aware Computing ... Distributed Embedded Computing ... Pervasive/Ubiquitous Computing ... Fault Tolerant & Trusted Embedded Systems ... System on Chip (SoC) and Multicore Systems ... Real-Time Operating Systems ... Programming Language Supports/Generation for embedded systems ... Cyber-Physical Systems ... Mobile Computing ... Reconfigurable Computing		...	Design Methodology and Tools::: Hardware/Software Co-Design ... Formal Methods for Embedded Systems ... Embedded Component Technology ... Middleware for Embedded Systems ... Hardware/Software Co-Verification ... Compiler and Debug Techniques ... IDE and Software Tools ... Performance Evaluation Techniques/Tools
%		\item International Conference on Application of Concurrency to System Design (ACSD) -- (Jun/)Jul % Hardware / software co-design, platform-based design, component-based design, refinement techniques, hardware / software abstractions, co-simulation and verification ... Design methods, tools and techniques based on models of computation and concurrency (data-flow models, communicating automata, Petri nets, process algebras, state charts, MSCs, etc.), (performance) analysis, verification, testing and synthesis ... Synchronous and asynchronous design, asynchronous circuits, globally asynchronous locally synchronous systems, interface design, multi-clock systems, functional and timing verification ... Concurrency issues in Systems on Chips, massively parallel architectures, networks on chip, task and communication scheduling, resource, memory and power management, fault-tolerance and Quality of Service issues ... Synthesis and control of concurrent systems, (compositional) modelling and design, (modular) synthesis and analysis, distributed simulation and implementation, (distributed) controller synthesis, adaptive systems, supervisory control
%		\item Euromicro Conference on Real-Time Systems (ECRTS) -- Jul / hardware/software co-design; power-aware and other resource-constrained techniques; systems on chip; time engines and time synchronization ... modelling and formal methods; probabilistic analysis for RT systems; quality of service support; reliability, security and survivability in RT systems; scheduling and schedulability analysis; worst-case execution time (timing) analysis; validation techniques ... functional, structural, and parametric specification and modeling; simulation, emulation, prototyping, and testing at the system, RTL, logic, and physical levels; co-simulation and co-verification ... System, hardware and embedded-software specification, modeling, verification, and test ... CAD for placement, routing, retiming, logic optimization, technology mapping, system-level partitioning, logic generators, testing, and verification of programmable/re-configurable architectures; CAD for modeling, analysis, and optimization of timing and power of programmable/re-configurable architectures ... Emerging technologies, system paradigms and design methodologies ... digital design in 3D layouts; optical, bio, nano and quantum technologies and computing; self-organizing and self-adapting systems ... Logic Synthesis ... System-Level Energy Optimization of HW/SW Embedded Systems (including library and task/resource management by operating system)
%		\item IEEE International Conference on Application-specific Systems, Architectures and Processors (ASAP 201X) -- Jul / theory and practice of application-specific systems, architectures and processors ... arithmetic, cryptography, compression, signal and image processing, application-specific instruction processors ... Bioinformatics and computational biology -- life sciences present a host of interesting problems that can benefit from application-specific solutions ... Computational finance -- the financial community has significant needs for high performance computing ... Architecturally diverse systems -- systems that use varied computing resources such as FPGAs, GPUs, Cell processors
%		\item International Conference on Tests \& Proofs (TAP 201X) -- Jul % Generation of test data, oracles, or preambles by deductive techniques such as theorem proving, model checking, symbolic execution, constraint logic programming, etc.	...	Generation of specifications by deduction ... Verification techniques combining proofs and tests ... Program proving with the aid of testing techniques ... Transfer of concepts from testing to proving (e.g., coverage criteria) ... Automatic bug finding ... Formal frameworks
%		\item International Colloquium on Automata, Languages and Programming (ICALP 201X) -- Jul % Track A - Algorithms, Automata, Complexity and Games: Algorithmic Game Theory, Approximation Algorithms, Automata Theory, Combinatorics in Computer Science, Computational Biology, Computational Complexity, Computational Geometry, Data Structures, Design and Analysis of Algorithms, Internet Algorithmics, Machine Learning, Parallel, Distributed and External Memory Computing, Randomness in Computation, Quantum Computing	...	Track B - Logic, Semantics, and Theory of Programming: Algebraic and Categorical Models, Automata Theory, Formal Languages, Non-standard Approaches to Computability, Databases, Semi-Structured Data and Finite Model Theory, Principles of Programming Languages, Logics, Formal Methods and Model Checking, Models of Concurrent, Distributed, and Mobile Systems, Models of Reactive, Hybrid and Stochastic Systems, Program Analysis and Transformation, Specification, Refinement and Verification, Type Systems and Theory, Typed Calculi		...	Track C - Foundations of Networked Computation: Models, Algorithms and Information Management, Algorithmic Aspects of Networks, Auctions, Computing with Incentives, E-commerce, Privacy, Spam, Formal Methods for Network Information Management, Foundations of Trust and Reputation in Networks, Internet Algorithmics, Mobile and Wireless Networks, Models of Complex Networks, Models and Algorithms for Global Computing, Models of Mobile Computation, Networks Economics, Networks of Low Capability Devices, Overlay Networks and P2P Systems, Social Networks, Specification, Semantics, Synchronization of Networked Systems, Theory of Security in Networks and Distributed Computing, Web Searching, Ranking, Web Mining and Analysis 
%		\item International Symposium on Signals, Circuits and Systems (ISSCS 201X, in Romania) -- Jul % Computer Aided Design
%		\item International Symposium on Software Testing and Analysis (ISSTA 201X) -- Jul % 
%		\item International Conference on Embedded Computer Systems: Architectures, Modeling, and Simulation (IC-SAMOS) -- Jul % Application-specific/Domain-specific Embedded Systems ... Profiling, Measurement and Analysis Techniques for Embedded Systems ... Hardware/Software Co-design ... Design Space Exploration ... System-Level Design, Simulation, and Verification ... Embedded Reconfigurable Processors ... Multimedia and Graphics Architectures ... Energy-Aware and Low-Power Processors ... Design, Specification and Synthesis of Embedded Systems ... Compilers and Mapping Technologies ... Multi-threaded Processors ... Computer Aided DesignEmbedded Parallel Systems ... Processor Simulation Technologies ... Network-on-Chip Platforms ... Multi-Processor Systems-on-Chip ... Digital Signal Processors
%		\item International Workshop on Systems, Architectures, Modeling, and Simulation (SAMOS) -- Jul % System-Level Design, Simulation, and Verification ... Hardware/Software Co-design ... Design Space Exploration ... Profiling, Measurement, and Analysis Techniques for Embedded Systems ... Embedded System Integration and Testing ... Energy-Aware and Low-Power Processors ... Novel Architectures and Micro-architectures for Embedded Systems ... Processor Simulation Technologies ... Compilers and Mapping Technologies ... Computer Aided Design ... Component-based Design and Analysis ... Physical Design ... Network-on-Chip Platforms ... Multi-Processor Systems-on-Chip ... Embedded Parallel Systems ... Embedded System-on-a-Chip Implementations ... Multi-threaded Processors ... High-Level, Architectural, and System Level Synthesis ... Software Defined Radios ... Digital Signal Processors ... Performance Analysis of Embedded Systems ... Memory Management, Smart Caches, and Compiler Controlled Memories ... Embedded Reconfigurable Processors
%		\item Metaheuristics International Conference (MIC 201X) -- Jul % Metaheuristic techniques such as tabu search, simulated annealing, iterated local search, variable neighborhood search, memory-based optimization, dynamic local search, evolutionary algorithms, memetic algorithms, ant colony optimization, variable neighborhood search, particle swarm optimization, scatter search, path relinking, etc. ... Algorithmic techniques that enhance the usability and increase the potential of metaheuristic algorithms such as parallelization of algorithms, reactive search mechanisms for self-tuning, offline metaheuristic algorithm configuration techniques, algorithm portfolios, etc. ... Empirical and theoretical research in metaheuristic including large-scale experimental analyses, algorithm comparisons, new experimental methodologies, engineering methodologies for stochastic local search algorithms, search space analysis, theoretical insights into properties of metaheuristic algorithms, etc. ... High-impact applications of metaheuristics in fields such as bioinformatics, electrical and mechanical engineering, telecommunications, sustainability, business, scheduling and timetabling. Particularly welcome are innovative applications of metaheuristic algorithms that have a potential of widening research frontiers. ... Contributions on the combination of metaheuristic techniques with those from other areas, such as integer programming, constraint programming, machine learning, etc. ... Challenging applications areas such as continuos, mixed discrete-continuous, multi-objective, stochastic, or dynamic problems.
%		\item International Workshop on First-Order Theorem Proving (FTP 201X) -- Jul % The workshop welcomes original contributions on theorem proving in first-order classical, many-valued, modal and description logics, including (but not restricted to): resolution, tableau methods, equational reasoning, term-rewriting, model construction, constraint reasoning, unification, description logics, propositional logic, specialized decision procedures; strategies and complexity of theorem proving procedures; implementation techniques and applications of first-order theorem provers to verification, artificial intelligence, mathematics and education	...	The workshop welcomes original contributions on theorem proving in first-order classical, many-valued, modal and description logics, including (but not restricted to): theorem proving in first-order classical, many-valued, and modal logics, including: satisfiability in propositional logic, satisfiability modulo theories, specialized decision procedures, constraint reasoning, equational reasoning, term rewriting, resolution paramodulation/superposition; strategies and complexity of theorem proving procedures; implementation techniques, applications of first-order theorem provers to: program verification, model checking, artificial intelligence, mathematics, computational linguistics.
%		\item Conference on Automated Deduction (CADE) -- Jul/Aug %  Logics of interest include propositional, first-order, equational, higher-order, classical, description, modal, temporal, many-valued, intuitionistic, other non-classical, meta-logics, logical frameworks, type theory and set theory ... Methods of interest include saturation, resolution, instance-based, tableaux, sequent calculi, natural deduction, term rewriting, decision procedures, model generation, model checking, constraint solving, induction, unification, proof planning, proof checking, proof presentation and explanation ... Applications of interest include program analysis and verification, hardware verification, mathematics, natural language processing, computational linguistics, knowledge representation, ontology reasoning, deductive databases, functional and logic programming, robotics, planning, and other areas of AI
%		\item International Workshop on Satisfiability Modulo Theories (SMT) -- Jul/Aug % decision procedures
%		\item International Joint Conference on Automated Reasoning (IJCAR 201X) -- Jul/Aug % 
%		\item Pragmatics of Decision Procedures in Automated Reasoning (PDPAR 201X) -- Jul/Aug % Decision Procedures ... Automated Reasoning ... IMPORTANT to SMT
%		\item IEEE International Symposium on Electromagnetic Compatibility -- Jul/Aug // Computational Electromagnetics: Computer Modeling; Model Validation; and Statistical Analysis	...	Also, check out the ``Global EMC University,'' which is a summer school on EMC
%		\item IEEE Symposium on Logic in Computer Science (LICS 201X) -- Aug % automata theory, automated deduction, categorical models and logics, concurrency and distributed computation, constraint programming, constructive mathematics, database theory, domain theory, �nite model theory, formal aspects of program analysis, formal methods, higher-order logic, hybrid systems, lambda and combinatory calculi, linear logic, logical aspects of computational complexity, logical frameworks, logics in arti�cial intelligence, logics of programs, logic programming, modal and temporal logics, model checking, probabilistic systems, process calculi, programming language semantics, proof theory, reasoning about security, rewriting, type systems and type theory, and veri�cation
%		\item International Static Analysis Symposium (SAS 201X) -- Aug % (focus in on static analysis for software) abstract domains, abstract interpretation, abstract testing, bug detection, data flow analysis, model checking, new applications, program transformation, program verification, security analysis, theoretical frameworks, type checking
%		\item International Conference on Theorem Proving in Higher Order Logics (TPHOLs) -- Aug % Speci�cation and veri�cation of hardware: microprocessors, memory systems, buses, pipelines, etc; formal semantics of hardware design languages; synthesis; formal design �ows ... Advances in theorem prover technology: proof automation and decision procedures, induction, combination of deductive and algorithmic approaches, incorporation of theorem provers into larger systems, combination of theorem provers with other provers and tools ... Speci�cation and veri�cation of software: program veri�cation, re�nement, and synthesis for functional, declarative and imperative languages; formal semantics of programming languages; compiler and operating system veri�cation; proof carrying code
%		\item International Forum on Application-Specific Multi-Processor SoC (MPSoC) -- Aug
%		\item HOT Chips: A Symposium on High-Performance Chips (HOT Chips) -- Aug / General Purpose Processor Chips: Low-Power; High-Performance; Multi-Core, Multi-Processor Technologies (interconnect, programming models, compiler, runtime systems) ... Other Chips: Novel Technology (quantum computing, nano-structures, micro-arrays); Low-power chips/Dynamic Power Management; Communication/Networking; Chipsets; Wireless LAN/Wireless WAN; FPGAs and FPGA Based Systems; Display Technology ... Application-Specific / Embedded Processors: Systems-on-Chip; Mobile Phone; Digital Signal Processing; Network/Security; Graphics/Multimedia/Game ... Software: Compiler technology; Operating System/Chip Interaction; Performance Evaluation ... Other Technologies: Advanced Packaging Technology, Reliability and Design for Test; Advanced Semiconductor Process Technology 
%		\item IEEE Hot Interconnects Conference: IEEE Symposium on High Performance Interconnects (Hot Interconnects) -- Aug / Novel and innovative interconnect architectures; Multi-core processor interconnects; System-on-Chip Interconnects; Advanced chip-to-chip communication technologies; Optical interconnects; Protocol and interfaces for interprocessor communication; Survivability and fault-tolerance of interconnects; High-speed packet processing engines and network processors; System and storage area network architectures and protocols; High-performance host-network interface architectures; High-bandwidth and low-latency I/O; Tb/s switching and routing technologies; Innovative architectures for supporting collective communication; Novel communication architectures to support grid comp; Requirements driving high-performance interconnects
%		\item IEEE International Midwest Symposium on Circuits and Systems (MWSCAS 201X) -- Aug % Digital Circuits and Computer Arithmetic... Programmable logic, VLSI, CAD and Layout
%		\item International Green Computing Conference -- Aug % Power-aware software ... Code profiling and transformation for power management ... Power-aware middleware ... Power-efficient architectures and chip designs ... Resource management to optimize performance and power ... Runtime systems that assist in power saving ... Models for collective optimization of power and performance ... Monitoring tools for power and performance ... Algorithms for reduced power, energy and heat ... Power-aware applications ... Static and dynamic data allocation for distributed servers ... Efficient circuit design for energy harvesting ... Power efficient cluster and enterprise computing ... Power management at component level, including memory, disk. ... Configurable and renewable energy ... Low power electronics ... Embedded systems, ASICs and FPGSs ... Power leakage and dissipation ... Power implications for portable and mobile computing ... Power aware networking ... Reliability of Power-aware computers ... Use of sensors for climate monitoring ... Smart control for eco-friendly buildings ... Thermal control of data centers ... Energy recycling ... Energy efficient power and cooling infrastructure
%		\item The Work in Progress in Green Computing (WIPGC) -- Aug
%		\item NSF Workshop on the Science of Power Management -- (Apr/)Aug 
%		\item Workshop on Low Power System on Chip (SoC) -- Aug % Low power SoC architecture ... Low power processor design ... I/O design ... Clock routing ... Power efficient circuit design ... Low power memory design ... Low power and high frequency transceiver ... Energy efficient multi-core architectures and Network-on-Chip ... Emerging interconnect technologies, like on-chip photonic, RF and wireless interconnects ... Low power coding methods for SoCs ... Low power SoC test
%		\item EUROMICRO Conference on Software Engineering and Advanced Applications (SEAA) -- Aug % Hardware/Software Do-Design, System-on-Chip Integration ... Optimized Algorithms for Embedded Systems (power management, embedded vision, networks on chip, communication) ... Architectures, Methods, Tools and efficient Processes and Procedures for Design and Development of critical systems (hardware, software) ... Verification and validation of hardware and software, components and systems
%		\item International Symposium on Low Power Electronics and Design (ISLPED) -- Aug %  Architecture, Circuits, and Technology:: 1.1. Technologies and Digital Circuits: emerging logic & memory technologies and applications, low power device and interconnect design, low power low leakage circuits, memory circuits, cooling technologies, battery technologies, variation-tolerant design, temperature-aware and reliable design ... 1.2. Logic and Microarchitecture Design: processor core design, cache and register file design, logic and RTL design, Arithmetic and signal processing circuits, encryption technologies, asynchronous design ... 1.3. Analog, MEMS and Mixed Signal Electronics: RF circuits, Wireless, MEMS circuits, AD/DA converters, I/O circuits, mixed-signal circuits, imaging circuits, analog noise, circuits to support emerging technologies, DC-DC conversion	...	Design Tools, System and Software Design:: 2.1. Design Tools: energy simulation and estimation tools that operate at the circuit/gate level, RT level, behavioral level, and algorithmic level, variation-aware design, physical design and interconnects ... 2.2. System Design and Methodologies: microprocessor, DSP and embedded systems design, FPGA and ASIC designs , emerging applications, server thermal management, system level power management, Behavioral and system level design aids, system-level power- and thermal-aware design, system-level reliability and variability-aware design ... 2.3. Software Design and Optimization: power aware compiler and operating system design (power- and thermal- aware software design, scheduling, and management), application-level optimizations, wireless and sensor networks
%		\item European Conference on Circuit Theory and Design (ECCTD'201X) -- Aug % Circuits ... Systems ... Mathematical methods ... Computational methods ... Signal processing ... Applications
%		\item IEEE International Conference on Embedded and Real-Time Computing Systems and Applications (RTCSA 201X) -- Aug % EMBEDDED SYSTEMS TRACK: Software design and debugging for heterogeneous multi-core embedded platform ... Multi-thread programming for multi-core embedded platform ... System integration/collaboration for systems of embedded systems ... Operating systems and scheduling ... System-on-chip design ... HW/SW co-design ... Power/thermal-aware design issues ... Embedded multimedia applications ... Embedded system design practices ... Networks-on-chip design ... Embedded system architecture ... Design optimization (memory, performance etc.)
%		\item Workshop on Highly Parallel Processing on a Chip (HPPC 201X) -- Aug % for/on highly parallel multi-core systems::: processor core architectures (homogeneous and heterogeneous) ... special purpose processors (accelerators, GPUs) ... on-chip memory and cache (or cache-less) organization, and interconnects ... off-chip memory, I/O, and multi-core interconnects ... overall system design (resource allocation and balancing) ... programming models (e.g. PRAM, BSP, data parallel, vector, transactional) ... parallel programming languages and software libraries ... supporting algorithms and implementation techniques (e.g. multi-threading, work-stealing) ... parallel algorithms and applications ... migration of existing codebase ... teaching of parallel computing
%		\item Euro-Par 2010 -- Aug/Sep %  1. Support tools and environments ...  2. Performance prediction and evaluation ...  3. Scheduling and load balancing ...  4. High performance architectures and compilers ...  5. Parallel and distributed data management ...  6. Grid, cluster and cloud computing ...  7. Peer to peer computing ...  8. Distributed systems and algorithms ...  9. Parallel and distributed programming ... 10. Parallel numerical algorithms ... 11. Multicore and Manycore programming ... 12. Theory and algorithms for parallel computation ... 13. High performance networks ... 14. Mobile and ubiquitous computing
%		\item Euromicro Symposium on Digital System Design: Architectures, Methods and Tools (DSD) -- Aug/Sep % design automation ... System synthesis: high-level, behavioral, RTL, logic, and physical synthesis ... hardware/software co-design; mapping of applications and architectures; algorithm architecture matching; transaction level modeling and higher-level modeling; virtual system prototyping; design space exploration; synthesis of asynchronous and dataflow driven systems ... Systems-on-a-chip and Multiprocessor SoCs ... CMP, SMP, SMT, DSP and VLIW (multi)processor architecture and enhancements; networks on chip ... compiler assisted ASIP and MPSoC generation and configuration ... SoC design environments for embedded systems	...	Programmable/re-configurable architectures/adaptable architectures: programmable/re- configurable/adaptable architectures: design methodologies and tools for reconfigurable computing, run- time, partial and dynamic reconfigurability; CAD for placement, routing, retiming, logic optimization, technology mapping, system-level partitioning, logic generators, testing and verification; CAD for modeling, analysis and optimization of timing and power; high-level models and tools for FPGAs; rapid prototyping systems and platforms.	...	System, hardware and embedded-software specification, modeling, verification and test: design and verification languages; functional, structural and parametric specification and modeling; simulation, emulation, prototyping, and testing at the system, register-transfer, logic and physical levels; co- simulation and co-verification.
%		\item International Conference on Field Programmable Logic and Applications (FPL'1X) -- Aug/Sep / RECONFIGURABLE ARCHITECTURES: Dynamic, partial, run-time reconfiguration, Low power architectures, Defect and fault tolerance, Reconfigurable embedded systems, Field-programmable analog arrays, Interconnects and NoCs ... APPLICATIONS: Communications and networking, Cryptography, Bioinformatics, Application acceleration, Evolvable and bio-inspired applications, Medical solutions, Experiments for High Energy Physics, Astronomy, Aerospace ... DESIGN METHODS AND TOOLS: CAD for reconfigurable architectures, Dynamic, partial, run-time reconfiguration, Logic optimization and technology mapping, Placement and routing algorithms, System-level design methods, Testing, verification and benchmarking, Hardware/software co-design, Compilers and languages, Rapid prototyping, Radiation tolerance and reliability ... SURVEYS, TRENDS AND EDUCATION: Roadmap for reconfigurable computing, Teaching reconfigurable systems, History and surveys of reconfigurable logic, Emerging device technologies
%		\item Symposium on Integrated Circuits and Systems Design (SBCCI) -- Aug/Sep % EDA topics
%		\item International Conference on Concurrency Theory (CONCUR) -- Sep / BASIC MODELS OF CONCURRENCY: such as abstract machines, domain theoretic models, game theoretic models, process algebras, and Petri nets ... LOGICS FOR CONCURRENCY: such as modal logics, probabilistic and stochastic logics, temporal logics, and resource logics ... MODELS OF SPECIALIZED SYSTEMS: such as biology-inspired systems, circuits, hybrid systems, mobile and collaborative systems, multi-core processors, probabilistic systems, real time systems, service oriented computing, and synchronous systems ... VERIFICATION AND ANALYSIS TECHNIQUES FOR CONCURRENT SYSTEMS: such as abstract interpretation, atomicity checking, model checking, race detection, pre-order and equivalence checking, run-time verification, state-space exploration, static analysis, synthesis, testing, theorem proving, and type systems ... RELATED PROGRAMMING MODELS: such as distributed, component-based, object-oriented, and web services.
%		\item IEEE International Behavioral Modeling and Simulation Conference (BMAS) -- Sep / analog synthesis; Behavioral model generation, optimization, or validation and qualification; Statistical behavioral modeling; System on chip modeling simulation methods; Distributed and parallel mixed-signal simulation; Frequency domain modeling and simulation; Compact device modeling lanuages and compilers; Compact device models for emerging technologies and topical issues (nano-devices, distributed thermal effect, leakaging issues, manufacturability, radiation effects, etc); Compilation techniques for simulation optimization; Automated model extraction and reduction; Formal description languages for electronic circuits and systems; Mixed-technology modeling (electro-mechanical, thermal-electronic, etc.); Thermal modeling; Optical modeling; Hydraulics modeling; Mechanical modeling; MEMS modeling; Biological and biochemical modeling; Mixed-level simulation (e.g., mixed finite-element/continuous-time); Mixed-domain modeling (frequency-domain, time-domain); Power electronic modeling; Full-system verification; Stress modeling; Failure effects modeling
%		\item IEEE/IFIP International Conference on VLSI and System-on-Chip (VLSI-SoC 201X) -- Sep % Digital systems and architectures ... 3-D Integration and Physical Design ... Deep Submicron Design and Modeling Issues, New Devices and MEMS ... CAD and Tools, Testability and Design for Test ... Prototyping, Validation, Verification, Modeling and Simulation ... System-on-Chip Design ... Embedded Systems Design and Real-Time Systems ... New Architectures and Compilers, Reconfigurable Systems ... Logic and High-Level Synthesis ... New Applications (biosystems, sensor networks, automotive, security, communications, etc.) ... Low-Power and Thermal-Aware Design ... Green Computing
%		\item International Workshop on Power And Timing Modeling, Optimization and Simulation (PATMOS 201X) -- Sep % Timing and Performance: Methodologies and tools for the analysis, design and verification of timing and performance properties of integrated circuits and systems at all levels of abstraction; ... Variability and statistical timing analysis; ... Design for yield, design for manufacturability; ... Special timing or performance related topics, e.g. crosstalk, synchronization, GALS, side-channel attacks.	...	Power Dissipation: Design techniques for low power circuits and systems at all levels of abstraction; ... Methods and tools for analysis, characterization, design and optimization of the power consumption; ... Low power architectures and libraries; ... Special power related topics, e.g. low voltage, leakage power, power grid, interconnect power, clock tree power, power aware test pattern generation, thermal effects.	...	Design Experience and Case Studies: Examples, test cases, benchmarks or design studies which present innovative solutions for timing, performance or power consumption related design challenges.	...	Power and Timing Issues Addressing Specific Technologies: Reconfigurable Architectures; ... Caches and Memory Devices; ... Low Power Software.
%		\item International Symposium on Embedded Multicore Systems-on-Chip (MCSoC-{\it n}), where $n \in \mathbb{N}$ -- Sep % Programming Models  for Embedded Multicore Architectures ... Architectural Support for Compilers/Programming Models ... Design Methodology for Multicore systems ... Compiling for Low Power Multicore Systems ... Low Power Design Techniques ... Embedded Multicore Systems Hardware (modeling, synthesis, low power simulation and analysis, reliable, performance modeling, security issues) ... Multimedia SoCs ... Multicore Systems for Biomedical Applications ... On Chip Interconnection Networks ... Embedded Memory ... High Speed Signal Interface ... MCSoCs  Optimization and Verification Methodologies ... Packaging Technologies for MCSoCs ... Embedded Multicore RTOS ... Multicore Memory Architectures ... Multicore Validation ... Network-on-chip: NoC architecture, Power and energy issues in NoC , Application specific NoC design, Timing, synchronous /asynchronous communication, RTOS support for NoC, Modeling, simulation, NoC support for MCSoC, NoCs for FPGAs and structured ASICs, NoC design tools ... Multicore Testing: Design-for-test, test synthesis, built-in self-test, embedded test for MCSoC
%		\item Annual Conference of the Italian Operational Research Society (AIRO201X) -- Sep % The conference will address all areas of Operational Research, with special emphasis on models and solution methods for decision support in transportation and logistics.
%		\item IEEE International SOC Conference (SOCC 201X) -- Sep % Embedded Systems, Multi/Many Core Systems and Embedded Memory Technologies ... System Level Design Methodology, EDA and Design Tools for SoC ... Reconfigurable and Programmable Circuits and Systems, System on Programmable Devices (FPGAs) ... Low-Power Circuits, Systems and Design Methodologies ... Signal Integrity, Design for Testability and Design Verification ... Design for Manufacturability, Variation-aware Methodologies ... Network on Chip (NoC) & Interconnects [ See \url{http://www.ieee-socc.org/}. This is another SoC conference of the same name as the SoC conference in November. This is an okay conference. ]
%		\item International Symposium on System-on-Chip 201X (SoC 201X) -- Sep % Alternative computing concepts ... Advanced platform architectures ... Analysis and early estimation techniques ... Configurable and reconfigurable architectures ... Hardware/software reuse techniques ... Low-power hardware techniques ... Low-power software techniques ... Network-on-chip and inter-chip communications ... Processor and compiler generators ... Retargetable compilers ... SoC verification and testing ... Software tools and languages for multiprocessor systems ... System-level design flow and methodology
%		\item International Symposium on Frontiers of Combining Systems (FroCoS) -- Sep % combination of decision procedures, of satis�ability procedures 
%		\item EACSL Annual Conference on Computer Science Logic (CSL) -- Sep % automated deduction and interactive theorem proving ... constructive mathematics and type theory ... equational logic and term rewriting ... automata and games ... modal and temporal logics ... model checking ... logical aspects of computational complexity ... finite model theory ... computational proof theory ... logic programming and constraints ... lambda calculus and combinatory logic ... categorical logic and topological semantics ... domain theory ... database theory ... specification, extraction and transformation of programs ... logical foundations of programming paradigms ... verification and program analysis ... linear logic ... higher-order logic ... nonmonotonic reasoning
%		\item International Conference on Electromagnetics in Advanced Applications (ICEAA) -- Sep / Electromagnetic modeling of devices and circuits; Finite methods; EMC/EMI/EMP
%		\item European Solid-State Circuits Conference (ESSCIRC) -- Sep // AMS and digital VLSI circuit and system design ... MEMS
%		\item European Solid-State Device Research Conference (ESSDERC) -- Sep / numerical, statistical, and analytical modeling of solid-state and optoelectronic devices ... compact modeling for devices and interconnects ... TCAD ... electro-thermal modeling
%		\item IEEE Custom Integrated Circuits Conference (CICC) -- Sep / Compact models for active and passive devices, behavioral modeling, signal-integrity modeling and simulation. System and circuit simulation. Parasitic extraction and reduction. Simulation techniques for analog, RF, and mixed-signal circuits. Package modeling. Process variation, statistical, and reliability modeling. Compact models for extreme environment operation. SOI and multiple gate device modeling.
%		\item European Microwave Integrated Circuits Conference (EuMIC) -- Sep // Part of European Microwave Week, which also includes European Radar Conference (EuRAD 201X); European Wireless Technology (EuWiT) Conference; and European Microwave Conference (EuMC) // Device and noise modelling; Linear and Non-Linear CAD Techniques for Devices, Circuits & Systems (including Thermal and Behavioural Modelling); Mixed-Signal Modeling; circuit simulation	...	Modelling, Simulation and characterisation of Devices and Circuits ... Physics Based Device Modelling and Simulation ... TCAD Device Modelling ... Noise modelling and characterisation ... Linear and Non-Linear CAD Techniques for Devices and Circuits ... Mixed-Signal Modelling ... Modelling of Passive RF, Microwave and mm Wave Components (common topic with EuMC) ... Complex modelling (Transport & Thermal &EM Modelling) ... Linear and Non-Linear device characterisation
%		\item European Microwave Conference (EuMC) -- Sep // Part of European Microwave Week, which also includes European Radar Conference (EuRAD 201X); European Wireless Technology (EuWiT) Conference; and European Microwave Integrated Circuits Conference (EuMIC) // Active Microwave and Millimetre-wave Components, Circuits and Sub-Systems ... Linear and Non-Linear CAD Techniques for Devices, Circuits & Systems (including Thermal and Behavioural Modelling) (common with EuMIC)
%		\item International Symposium on Symbolic and Numeric Algorithms for Scientific Computing (SYNASC 201X, in Romania) -- Sep % Symbolic Computation: computer algebra; symbolic techniques applied to numerics; hybrid symbolic and numeric algorithms; numerics and symbolics for geometry; and programming with constraints, narrowing ... Logic and programming: automatic reasoning; formal system verification; formal verification and synthesis software quality assessment; static analysis; timing analysis ... Artificial Intelligence: hard computational problem solving; intelligent systems (front-ends) for scientific computing; agent-based complex systems modeling and development; knowledge and data intensive systems; soft computing; recommender systems for scientific computing; data mining; web mining; information retrieval ... Numerical computing: iterative approximation of fixed points; solving systems of nonlinear equations; numerical and symbolic algorithms for differential equations; numerical and symbolic algorithms for optimization; parallel algorithms for numeric computing; scientific visualization and image processing ... Parallel computing: parallel, distributed and web computing for symbolics and numerics; grid middleware and applications; agent-based grid computing; grid services; workflow management ... Advances in the Theory of Computing: data structures, algorithms and combinatorial optimization; graphs and combinatorics in computer science; models of computational complexity; logical approaches to complexity; new computational paradigms; automata theory and other formal models
%		\item Electronics System Integration Technology Conference (ESTC 201X) -- Sep / microsystem technology and electronic packaging
%		\item IEEE International Conference on 3D System Integration (3D IC) -- Sep / 3D IC Technology. Through Silicon Vias (TSV), wafer thinning, wafer alignment, wafer bonding, wafer dicing, 3D IC process, monolithic 3D integration, heterogeneous 3D integration, Capacitive coupling, Inductive coupling, multilevel epitaxial growth, etc. ... 3D IC Circuits Technology. 3D SOC, 3D Memory, 3D Processor, 3D DSP, 3D FPGA, 3D RF and microwave/millimeter wave, 3D analog circuits, 3D Biomedical circuits etc. ... 3D Applications - imaging, memory, processors, communications, networking, wireless, biomedical etc. ... 3D Design Methodology. 3D CAD, 3D synthesis, 3D design flows, Signal and power integrity analysis and design in 3D, 3D thermal design and analysis, test and design for test; 3D mechanical stress and reliability design and analysis, etc. ... Test and Reliability of 3D Systems. simulation, modeling and failure analysis, known good die (KGD) test, Testing methodology and testability, Built-in-self test circuit (BIST), Reliabilities (electro-migration/ stress-migration of TSV and microbump), Failure analysis for TSV (X-ray CT scan, etc.), Mechanical stress evaluation method, Heat distribution evaluation method, etc.
%		\item International Conference on Simulation of Semiconductor Process and Devices (SISPAD 201X) -- Sep / All aspects of device simulation, including transport in nano-structures and next generation devices such as Fin/tri-gate, UTSOI, and structures using non-conventional materials, effects of strain on carrier transport, models of device scaling limits, quantum effects, reliability, fluctuations, novel nano-scale devices such as QCA, SET, CNT, and molecular devices ... All aspects of front and back end process simulation, including both continuum and atomistic approaches, models for dopant activation and diffusion, oxidation, silicide growth, interface effects, effects due to stress, nano-scale fabrication, and design of new materials ... Equipment, topography, and lithography simulation ... Virtual fab implementations and algorithms for computational lithography ... Interconnect modeling and algorithms including noise and parasitic effects ... Compact device modeling for circuit simulation, including high frequency and noise modeling ... Integration of circuit, device, process simulation with applications to performance modeling of circuits ... User interfaces and visualization ... High performance computing, advanced numerical methods and algorithms ... Mesh generation and adaptation ... Simulations of new memory structures such as nanocrystal, phase change, MRAM, and devices such as microsensors, microactuators, optoelectronics devices, lasers, and flat panel displays ... Process and device simulation for power generation, control and storage including photovoltaics, power devices, smart power, and other ``green technologies'' ... Benchmarking, calibration, and verification of simulator models.
%		\item IEEE East-West Design \& Test Symposium (EWDTS) -- Sep % Analog, Mixed-Signal and RF Test ... Analysis and Optimization ... ATPG and High-Level TPG ... Built-In Self Test ... Debug and Diagnosis ... Defect/Fault Tolerance and Reliability ... Design for Testability ... Design Verification and Validation ... EDA Tools for Design and Test ... Embedded Software Performance ... Failure Analysis, Defect and Fault ... FPGA Test ... HDL in test and test languages ... High-level Synthesis ... High-Performance Networks and Systems on a Chip ... Low-power Design ... Memory and Processor Test ... Modeling & Fault Simulation ... Network-on-Chip Design & Test ... Modeling and Synthesis of Embedded Systems ... Object-Oriented System Specification and Design ... On-Line Test ... Power Issues in Testing ... Real Time Embedded Systems ... Reliability of Digital Systems ... Scan-Based Techniques ... Self-Repair and Reconfigurable Architectures ... Signal and Information Processing in Radio and Communication Engineering ... System Level Modeling, Simulation & Test Generation ... Using UML for Embedded System Specification ... CAD and EDA Tools, Methods and Algorithms ... Design and Process Engineering ... Logic, Schematic and System Synthesis ... Place and Route ... Thermal, Timing and Electrostatic Analysis of SoCs and Systems on Board ... Wireless Systems Synthesis ... Digital Satellite Television
%		\item Forum on specification \& Design Languages (FDL 201X) -- Sep // \url{http://www.ecsi.me/fdl} // application of specification, design and verification languages to the design, modelling and verification of integrated circuits, complex hardware/software embedded systems, and mixed-technology systems; Assertion Based Design, Verification \& Debug (ABD); Language-Based System Design (LBSD); Embedded Analog and Mixed-Signal System Design (EAMS); UML and MDE for Embedded System Specification \& Design (UMES)
%		\item System, Software, SoC and Silicon Debug (S4D) Conference -- Sep // System Level Debug; System Performance Analysis and Optimization; Embedded Software Debug; ESL Model based System Debug; FPGA based System Debug; Silicon Debug; Design and Synthesis for Debug; DFT Reuse for Debug and Analysis; Debug Architectures and Interfaces; Digital/Analog Debug; Debugging of IP blocks and their Integration; Tools for Debug, Analysis and Optimization; Tools and Equipment Impact; Debug Case Studies; Debug R&D Initiatives; Debug Standards
%		\item Scientific Computing in Electrical Engineering conference (SCEE 201X) -- Sep / Computational Electromagnetics; Circuit Simulation; Device Modeling; Coupled problems; Mathematical and Computational Methods; Model Order Reduction. See \url{http://sites.onera.fr/SCEE2010/node/9}.
%		\item International Test Synthesis Workshop (ITSW 201X) -- Sep
%		\item EMC Europe 201X -- International Symposium on EMC and International Wroclaw Symposium on EMC -- Sep // \url{http://www.emceurope.org/2010/dates.html} 
%		\item European Microwave Conference (EMC) -- Sep/Oct / Linear and Non-Linear CAD Techniques for Devices, Circuits & Systems (including Thermal and Behavioural Modelling)
%		\item Nordic Optimization Symposium (NOS - 201X) -- Sep/Oct % mathematical programming, operational research and optimization 
%		\item NVIDIA GPU Technology Conference, including the {\tt NVIDIA Research Summit}; see \url{http://www.nvidia.com/object/gpu_tech_conf_research_summit.html} -- Sep/Oct % The NVIDIA Research Summit is a cross-disciplinary forum targeting researchers interested in using GPUs in science and engineering. Attendees new to GPU computing will learn how GPU computing can drastically increase computational power and dramatically reduce time-to-discovery; attendees already using GPU computing can showcase their work, network with each other, learn advanced topics in GPU computing, and discuss their work with NVIDIA engineers and researchers.
%		\item International Conference on Parallel Architectures and Compilation Techniques (PACT 201X) -- Sep/Oct % Parallel architectures and computational models ... Compilers and tools for parallel computer systems ... Multicore, multithreaded, superscalar, and VLIW architectures ... Compiler/hardware support for hiding memory latencies ... Support for correctness in hardware and software (esp. with concurrency) ... Reconfigurable computing ... Dynamic translation and optimization ... I/O issues in parallel computing and their relation to applications ... Parallel programming languages, algorithms and applications ... Middleware and run-time system support for parallel computing ... High performance application specific systems ... Applications and experimental systems studies ... Non-traditional computing systems topics
%		\item Sophia-Antipolis Formal Analysis Group (SAFA'201X) Workshop -- Sep/Oct // See \url{http://www-sop.inria.fr/oasis/SAFA/} // focus on formal analysis: Formal languages : definition, toolkits; Model-checking, proves; Simulation, testing; Model-driven engineering; Embedded systems, distributed systems, security-related systems, service-oriented architecture; Industrial and teaching experience
%		\item Sophia Antipolis forum on MicroElectronics (SAME 201X) -- Oct // \url{http://www.same-conference.org/} // System and Design; Integration; Security; Tools and Methodologies; Process and Manufacturing; Prototyping, Verification, Validation, Test
%		\item IEEE Symposium on Foundations of Computer Science (FOCS 201X) -- Oct % theoretical CS
%		\item International Symposium on System-on-Chip (SoC) -- Oct % Advanced platform architectures ... Analysis and early estimation techniques ... Application-specific processors and architectures ... Configurable and reconfigurable architectures ... Hardware/software reuse techniques ... Low-power hardware techniques ... Low-power software techniques ... Operating systems for embedded systems ... Multiprocessor SOC implementations ... Network-on-chip and inter-chip communications ... Processor and compiler generators ... Retargetable compilers ... SoC verification and testing ... Software runtime environments ... Software emulation and simulation ... Software tools and languages for multiprocessor systems ... System-level design flow and methodology [ This is not a good conference; it is average at best. ]
%		\item Haifa Verification Conference (HVC) -- Oct % Simulation-based verification, Formal verification methods, High-level stimuli generation, Model checking, Equivalence checking, SAT-based verification algorithms, Classification of hardware bugs, Static analysis, Design for verifiability, Concurrency testing, Hardware/software co-verification, Debugging, Emulation and acceleration, Defect prevention, CSP-based functional verification, Test-driven development, Hybrid verification methods, Model-based testing, Verification coverage, Developer testing, Formal specification Languages, Review and inspection, Semi-formal verification, Pair testing and first testing, Test automation frameworks, Automatic test generation
%		\item Conference on Electrical Performance of Electronic Packaging and Systems (EPEPS 201X) -- Oct / Current and future issues related to signal interconnections including on-chip and system interconnect structures ... Signal integrity of interconnections ... Power integrity of electronic components and systems ... Design and measurement of high speed links, including jitter and noise management and characterization ... Innovative interconnect packaging structures and their electrical performance ... Measurement and data analysis techniques for system level and on-chip structures ... Signal and power integrity in mixed signal and RF circuits and modules ... RF/microwave packaging structures and their electrical performance ... Electromagnetic modeling and simulation algorithms, tools and design flows ... Macromodeling and model reduction as it applies to electrical analysis ... Advances in transmission-line analysis techniques.
%		\item International Workshop on Thermal Investigations of ICs and Systems (THERMINIC 201X) -- Oct %/ Thermal and Temperature Sensors ... Measurement of Thermal Properties ... Thermal Simulation ... Acquisition and analysis of Thermal data ... Electro-thermal Simulation ... Temperature Mapping ... Thermal Modelling and Investigation of Packages ... Novel and Advanced Cooling Techniques ... Reliability Issues ... Thermal Performance of Interconnects ... High Temperature Electronics ... Heat Transfer Enhancement ... Heat Transfer Education ... Validation of Thermal Software ... Flow Visualisation Techniques ... Coupled (Thermo-mechanical, Thermo-optical) Effects ... Turbulence Modelling in Complex Geometrics ... Defect and failure modelling ... Thermal Stress: Theory and Experiment ... Reliability evolution and prediction ... Thermal Stress Failures: Prediction and Prevention ... Multiphysics simulation ... Nanoengineering issues ... Nanotechnology Applications ... Education ... Noice control
%		\item International Microsystems, Packaging, Assembly and Circuits Technology Conference (IMPACT 201X) -- Oct // Advanced and Emerging Packaging Technology... Materials, Equipment & Process ... Test, Quality, Inspection and Reliability Technology ... Build-up/HDI and Embedded Technology ... Electro Deposition and Electrochemical Processing Technology ... Surface Mounting and Assembly Technology ... Nanotechnology and Interconnection ... Thermal Management ... Advanced Materials, Process and Equipment ... Advanced Sensor and Microsystems Technology ... Modeling, Testing & Design: Electrical, thermal, optical, and mechanical modeling, simulation, and characterization of packaging solutions including assembly manufacture modeling , Cu low-K interconnects, drop impact models, embedded passives, equivalent circuit models, lead-free solders, global local modeling, measurements, thermo-mechanical reliability, optimization, and design of experiment.
%		\item International Conference on Compilers, Architecture, and Synthesis for Embedded Systems (CASES, which is part of Embedded Systems Week / ESWeek) -- Oct % Architectures: multi-core system-on-chip ... on-chip communication architectures ... extensible, customizable ASIPs ... run-time and design time reconfigurable processors and on-chip architectures ... memory architectures: memory management, smart caches and compiler controlled memories ... novel nano-based architectures ... 3-D architectures ... Synthesis: synthesis of hardware software systems ... thermal and power-aware synthesis flows ... synthesis for reliability, low power, performance ... 3-D integration and synthesis ... Embedded Systems: validation, verification, and debugging techniques of embedded software ... analysis techniques for embedded system including design space exploration, co-simulation ... static and dynamic timing analysis ... domain specific embedded applications ... models of computation ... modeling and management for power/thermal, performance and reliability
%		\item International Conference on Hardware/Software Codesign and System Synthesis [or International Conference on Hardware-Software Codesign and System Synthesis, CODES+ISSS, which is part of Embedded Systems Week / ESWeek] -- Oct % High-level, architectural and system-level synthesis - Specification and refinement, design representation, synthesis, partitioning, estimation, design space exploration.	...	Hardware-software co-design - Co-design methodologies, interaction between architecture and software design, HW/SW partitioning, design space exploration, HW/SW interface.	...	Specification languages and models - System-level models and semantics, timing analysis, power, formal properties, heterogeneous systems and components.	...	Simulation and verification - Hardware-software cosimulation, verification methodology, formal verification, HW acceleration, test methodology, design for testability.	...	Power-aware design methodology - Power and performance modeling, analysis and estimation techniques, power management approaches, low-power design methodologies.	...	Embedded systems architecture - Architecture optimization, application-specific architectures, memory and communication architecture exploration, architecture optimization.	...	Embedded software - Compilers, memory management, virtual machines, scheduling, power-aware OS, real-time support and middleware. Multicore and multiprocessor programming models for SoCs and NoCs, profiling techniques and trace generation.	...	Multiprocessors and MPSoC -Multiprocessor architectures, design space exploration, MPSoC.	...	Network-on-chip - On-chip communication architectures and protocols, switching, routers and communications space exploration.	...	Industrial practices and case studies and emerging techniques - Design experiences of high interest to the community. Applications of new state-of-the-art methodologies and tools to real-life problems in various application areas: e.g. wireless, networking, multimedia, automotive, medical systems and sensor networks. New challenges for next generation embedded computing systems, arising from increased heterogeneity, new technologies or new applications.	...	Application-specific architectures and algorithms - Application-specific processor architectures and tools, Hardware accelerators and/or processors for network, media and security applications. Reconfigurable processors.
%		\item International Conference on Embedded Sofware (EMSOFT, which is part of Embedded Systems Week / ESWeek) -- Oct % Design and implementation of embedded software ... Modeling and validation ... Model- and component-based software design and analysis ... Programming languages and compilers ... Software engineering and programming methodologies ... Scheduling and execution time analysis ... Operating systems and middleware ... QoS management and performance analysis ... Hardware-dependent software and interfaces ... Distributed, networked embedded systems and security ... Embedded control and communication ... Software for multiprocessor/multicore embedded systems and systems-on-chip ... Application areas, e.g., automotive, avionics, energy, health care, mobile devices, multimedia
%		\item Workshop on Compiler-Assisted System-On-Chip Assembly 201X (CASA 201X, which is part of Embedded Systems Week) -- Oct % 
%		\item Workshop on Software Synthesis (WSS'1X, which is part of Embedded Systems Week) -- Oct % 
%		\item International Conference on Computer Design (ICCD) -- Oct % Modeling and Performance Analysis ... System Design Methods for Uni-and Parallel-Processors ... Techniques for Low-power, Secure, and Reliable Processor Designs ... Embedded Network, System-on-chip, and Application Specific Processor Design ... High-level, Logic and Physical Synthesis ... Physical Planning, Design, and Early Estimation for Large Circuits ... Automatic Analysis and Optimization of Timing, Power and Noise ... Tools for Multiple-clock Domains, Asynchronous and Mixed-timing Methodologies ... CAD Support for FPGAs, ASSPs, Structured ASICs, Platform-based Design and Networks- on-chip Hardware Description Languages ... Embedded Network, System-on-chip, and Application Specific Processor Design ... Simulation Based and Formal Techniques for Functional Design Verification ... Equivalency Checking, Property Checking, Theorem Proving ... High-level Design Validation ... Design Error Debug and Diagnosis ... Hardware/Software Validation ... Fault Modeling ... Fault Simulation and ATPG ... DFT and BIST ... SoC Testing
%		\item Conference on Design and Architectures for Signal and Image Processing (DASIP) -- Oct // \url{http://www.ecsi.me/dasip} // Design methods and tools: Design verification and fault tolerance; Embedded system security and security validation; System-level design and hardware/software co-design; Communication synthesis, architectural and logic synthesis; Embedded real-time systems and real-time operating systems; Rapid system prototyping, performance analysis and estimation; Formal models, transformations, algorithm transformations and metrics	...	Development platforms, architectures and technologies: Embedded platforms for multimedia and telecom; Many-core and multi-processor systems, SoCs, and NoCs; Reconfigurable ASIPs, FPGAs, and dynamically reconfigurable systems; Asynchronous (self-timed) circuits and analog and mixed-signal circuits; Digital biosignal processing, biologically based and/or inspired systems	...	Use-cases and applications: Ambient intelligence, ubiquitous and wearable computing; Global navigation satellite systems, smart cameras, and PDAs; Security systems, cryptography, object recognition and tracking; Embedded systems for automotive, aerospace, and health applications	...	Smart sensing systems: Sensor networks, environmental and system monitoring; Vision, audio, fingerprint, health monitoring, and biosensors; Structurally-embedded, distributed, and multiplexed sensors; Sensing for active control systems, adaptive and evolutionary sensors	...	Reliable multi-processor scheduling and HW/SW resource management; Reconfigurable computing architectures; Image and signal processing on GPU; Reconfigurable video coding (RVC); Smart image sensors
%		\item IEEE Bipolar / BiCMOS Circuits and Technology Meeting (BCTM) -- Oct / Behavioral modeling techniques; Parameter extraction; RF and thermal simulation techniques; Modeling of passives, interconnect and packages; Statistical modeling; Device, process and circuit simulation
%		\item IEEE Compound Semiconductor IC Symposium (CSICS) -- Oct / circuit simulation
%		\item IEEE International Conference on Tools with Artificial Intelligence (ICTAI) -- Nov / AI algorithms, applications of AI in various fields, Constraint Programming, Evolutionary Computing, Knowledge Representation and Reasoning, Machine Learning, Natural Language Processing, Neural Networks, Ontologies, Planning and Scheduling, Qualitative Reasoning, Search and Heuristics, Semantic Web-Techniques & Technologies, Speech Processing, Vision/Image Processing... Analog IP Design considerations... Analog DFT methods... Parametric Defects and Process Variations... Embedded Test & Diagnostics... Characterization, Ramp, and Production testing of Analog components... Fault models, defect modeling... Yield analysis and recovery
%		\item International Test Conference (ITC 201X) -- Oct / Nov % Test and Design for Manufacturability ... Yield Analysis & Optimization ... Embedded Instruments (BIST & DFT) ... Reliability Screening ... Low-Cost Test ... Diagnosis & Silicon Debug ... High-Speed I/O and RF Test ... Probe-card Design & Simulation ... Self Repair & Fault tolerance ... System Test & False Positives ... Emerging DUTs	...	Adaptive Test ... Board Test ... Defect-based Testing ... Innovative Industrial Test Practice ... Fault Modeling & Simulation & Test ... Power Issues in Test ... Standalone Memory Test & Repair ... ATE Hardware and Software ... Online Test ... Protocol Aware Test ... Design Verification & Validation ... Test Resource Partitioning ... Test Floor Optimizations ... CIS Test ... Automatic Test Generation ... Boundary-Scan ... On-Chip Test Compression ... Economics of Test ... FPGA Test ... Online Test ... IDDQ and Current Test ... Interface Issues ... Microprocessor Test ... Mixed-Signal and Analog Test ... Multisite Test ... System-in-Package and KGD Test ... Test Standards ... Test Quality & Reliability
		\item International Conference on Computer-Aided Design (ICCAD) -- Nov
%		\item IEEE International Workshop on Test and Validation of High Speed Analog Circuits (TVHSAC 201X) -- Nov % Design-for-test, including BIST and loop-back test... Design for characterization and validation, including on-die sensors and test structures... ATE technology for high speed analog measurements that address accuracy, bandwidth and efficiency... Board technology for load-board and probe-card design to address ATE-based test and characterization... Economics of test, test cost and yield optimization
%		\item IEEE International High Level Design Validation and Test Workshop (HLDVT) -- Nov % Simulation-Based Validation ... Formal Veri�cation ... Speci�cation-based Design and Validation ... Error Trace Analysis and Debug ... Hybrid SAT/BDD/ATPG Methods ... Coverage-driven Test Generation ... Design/Synthesis for Test ... Hardware/Software Co-Validation ... Prototyping and Emulation ... Post-silicon Validation and Debug
%		\item Silicon Valley Test Conference (SVTest) -- Nov % Test and Design for Manufacturability ... Yield Analysis and Optimization ... Low Cost Test ... High Speed I/O and RF Test ... Probe-Card Design and Innovation ... Interface Hardware and Simulation ... Design Verification & Validation ... PDN implications for Testing & Validation ... Adaptive Test ... Defect based Testing ... Hi Resolution A/D - D/A testing ... Iddq and current test ... Load Board and Socketing Issues ... Multi-Site Test ... SiP and KGD Test ... Analog & Mixed Signal Test		::::	THIS IS A POOR CONFERENCE
%		\item IEEE International Workshop on Testing Three-Dimensional Stacked Integrated Circuits (3D-TEST 201X) % test of and design-for-test for three-dimensional stacked ICs (3D-SICs), including Systems-in-Package (SiP), Package-on-Package (PoP), and especially 3D-SICs based on Through-Silicon Vias (TSVs). While 3D-SICs offer many attractive advantages with respect to heterogeneous integration, smaller form-factor, higher bandwidth and performance, and lower power dissipation, there are many open issues with respect to testing such products	...	Defects due to Wafer Thinning ... Defects in Intra-Stack Interconnects ... DfT Architectures for 3D-SICs ... EDA Design-to-Test Flow for 3D-SICs ... Failure Analysis for 3D-SICs ... Known-Good Die / Stack Testing ... Pre-Bond and Post-Bond Testing ... Reliability of 3D-SICs ... Standardization for 3D Testing ... System/Board Test Issues for 3D-SICs ... Test Cost Modeling for 3D-SICs ... Test Flow Optimization for 3D-SICs ... Tester Architecture incl. ATE and BIST ... Thermal/Mechanical Stress in 3D-SICs ... TSV Test, Redundancy, and Repair ... Wafer Probing and Probe Damage of 3D-SICs
%		\item Conference on Design of Circuits and Integrated Systems (DCIS 201X) -- Nov % Logic and Achitectural Synthesis ... System Design Methodologies ... Hw-Sw Codesign ... Embedded Design & SoC ... Analog & Mixed-Signal IC Design & Test ... RF IC Design & Test ... Low Power Design ... Modeling, Simulation and Synthesis Techniques ... Testability and Test Techniques ... Failure Analysis and Reliability ... Reconfigurable Computing ... Defect and Fault Tolerance Techniques
%		\item International Workshop on Parallel and Distributed Methods in verifiCation (PDMC) -- Nov % multi-core/distributed model checking ... multi-threaded/distributed equivalence checking ... slicing and distributing the state space ... parallel/distributed satisfiability checking ... parallel/distributed theorem proving ... parallel/distributed constraints solving ... parallel methods in probabilistic model checking ... parallel methods in performance evaluation ... distributed (libraries for) graph algorithms ... distributed state space generation I/O efficient algorithms for verification ... GPU accelerated verification ... GRID vs. clusters vs. SMP (heterogeneity, co-scheduling) ... parallelization for multi-core processors ... load balancing ... scalability experiments of distributed model checking algorithms ... on large Grids and P2P systems ... using verification methods to improve robustness of Grid systems ... applying verification methods to Grid and P2P protocols ... distributed (randomized) data structures and algorithms
%		\item International System-on-Chip (SoC) Conference, Exhibit \& Workshops -- Nov % CAD/EDA Tools related to SoC design [ This seems to be of a low quality conference. See \url{http://www.socconference.com/}. This is another SoC conference of the same name as the SoC conference in September. ]
%		\item International Conference on Formal Methods in Computer-Aided Design (FMCAD) -- Nov
%		\item Norchip Conference (NORCHIP) -- Nov / Digital and analog microelectronics/nanoelectronics ... Embedded systems, FPGA and rapid prototyping ... Measurement, modeling, and simulation of hardware ... CAE, design tools, VLSI design methodology ... Network-on-Chip
%		\item International Conference for High Performance Computing, Networking, Storage and Analysis (SC 1X) -- Nov % Applications, Architecture/Networks, Clouds and Grids, Performance, Storage, and Systems Software ... exploiting multicore systems and GPUs for large-scale computing, scalable systems and storage, and programming models for exasca le computing
%		\item International Conference on Field-Programmable Technology (FPT 201X) -- Dec % Tools and Design techniques for field-programmable technology including placement, routing, synthesis, verification, debugging, run-time support, technology mapping, partitioning, parallelization, timing optimization, design and run-time environments, languages and modelling techniques, provably-correct development, intellectual property core based design, domain-specific development, hardware/software co-design.	...	Architectures for field-programmable technology including field programmable gate arrays, complex programmable logic devices, coarse-grained reconfigurable arrays, field programmable interconnect, field programmable analogue arrays, field programmable arithmetic arrays, memory architectures, interface technologies, low-power techniques, adaptive devices, reconfigurable computing systems, high-performance reconfigurable systems, evolvable hardware and adaptive computing, fault tolerance and avoidance.
%		\item International Workshop on Microprocessor Test and Verification (MTV) -- Dec % Validation of microprocessors and SOCs ... Experiences on test and verification of high performance processors and SOCs ... Test/verification of multimedia processors and SOCs ... Performance testing ... High-level test generation for functional verification ... Emulation techniques ... Silicon debugging ... Formal techniques and their applications ... Verification coverage ... Test generation at the transistor level ... Equivalence checking of custom circuits ... Circuit level verification ... Switch-level circuit modeling ... Timing verification techniques ... Path analysis for verification or test ... Design error models ... Design error diagnosis ... Design for testability or verifiability ... Optimizing SAT procedures for application to testing and formal verification
%		\item IEEE Circuits and Systems Society Forum on Emerging and Selected Topics (CAS-FEST) -- Dec % Variation-Aware Design for Nanoscale VLSI
%		\item IEEE International Electron Devices Meeting (IEDM 201X) -- Dec % CMOS DEVICES & TECHNOLOGY (CDT): circuit/device interaction and co-optimization, CMOS scaling issues, high performance, low power, analog/RF devices, and CMOS platform technology and manufacturing issues, such as DFM and process control	....	MODELING AND SIMULATION (MS): analytical, numerical, and statistical approaches to modeling electronic, optical and multiphysical devices, their isolation and interconnection. Topics include physical and compact models for devices and interconnects, modeling fabrication processes and equipment, simulation algorithms, process characterization, parameter extraction, early compact models for advanced technologies, performance evaluation, design for manufacturing, reliability and technology benchmarking methodology. Papers on the modeling of interactions between process, device and circuit technology are solicited.
%		\item IEEE International Conference on Electronics, Circuits, and Systems (ICECS 201X) -- Dec % Analog/Digital/Mixed Signal Testing ... Computer Aided Network Design ... Design Automation ... System Integration (SoC, Mixed-Signal) ... Device Characterization/ Modeling ... Test and Reliability
%		\item Annual IEEE/ACM International Symposium on Microarchitecture (Micro-4X) -- Dec % ILP and TLP architectures and designs: superscalar, VLIW, multithreaded, multicore, manycore, etc. ... Compiler techniques for instruction-level, thread-level, and memory-level parallelism ... Architectures and compilers for embedded processors, DSPs, GPUs, ASIPs (network processors, multimedia, wireless) ... Dynamic optimization, emulation, and object code translation ... Advanced software/hardware speculation and prediction schemes ... Power, performance, and implementation efficient designs ... Microarchitecture support for reliability, dependability, and security ... Microarchitecture modeling and simulation methodology
%		\item International Workshop on Network on Chip Architectures (NoCArc 201X) -- Dec % NoC Performance Analysis ... Dynamic On-chip Network Reconfiguration ... Topology Selection and Synthesis for NoCs and MPSoCs ... Modeling and Evaluation of On-chip Networks ... Routing Algorithms and Router Micro-architectures ... Design Space Exploration and Tradeoff Analysis ... Guaranteed Throughput and Real Time On-chip Communication ... On-chip Interconnection Network Simulators ... and Emulators ... Mapping of Cores to NoCs ... Validation, Debug and Test of NoCs and MpSoCs ... Power and Energy Issues ... 3D NoC Architectures ... Fault Tolerance and Reliability Issues ... Emerging Technologies and New Design Paradigms ... Memory Architectures for NoC ... Industrial Case Studies of MpSoCs using the NoC Paradigm
%		\item Asia-Pacific Microwave Conference (APMC) -- Dec / Computational electromagnetics; EMC modeling and simulation
%		\item International Conference on Formal Engineering Methods (ICFEM) -- Dec
%		\item Electronics Packaging Technology Conference (EPTC 201X) -- Dec / Advanced Packaging: Wafer level packaging, 3D integration, TSV (through Silicon Via), embedded passives & actives on substrates, flip chip packaging, RF-ID, 3D SiP, Packaging solutions for MEMS, MOEMS, NEMS, Automotive electronics, optoelectronics ... Interconnection Technologies: wire bonding technology, flip chip technology, solder alternatives (ICP, ACP, ACF, NCP), under bump metallurgy, 3D and TSV connections, microbump, substrate technology ... Materials & Processes: Materials and processes for traditional and advanced microelectronic systems, 3D packages, MEMS, solar, green and biomedical packaging that enhance mechanical, thermal, electrical and optical performance as well as cost effectiveness ... Modeling & Simulations: Electrical Modeling & Signal Integrity, Thermal Characterization & Cooling Solutions: Mechanical Modeling & Structural Integrity ... Quality & Reliability: Component, board and system level reliability assessment, interfacial adhesion, accelerated testing and models, advances in reliability test methods and failure analysis ... Emerging Technology: Packaging solutions for solar photovoltaic applications, systems and packaging in the areas of bioelectronics such as biomedical, bioengineering, biosensors and electronics for medical devices; wearable electronics, organic/printable electronics; portable power supplies such as fuel cells; and other novel packaging
%		\item ICFEM09 UML \& FM Workshop -- Dec / Consistent specifications, model transformations (QVT technologies, transformation repositories). Transformations to make models more analyzable so as to make them executable ... Automation of traceability through transformations ... Refinement techniques: developing detailed design from a UML abstract specification ... Refinement of OCL specification ... Formal reasoning on models for code generation ... Technologies for compositional verification of models ... Specification of a formal semantics for the UML. Giving an abstract syntax to UML diagrams ... Formal validation and verification of software
%		\item IEEE Electrical Design of Advanced Packaging \& Systems Symposium (EDAPS) -- Dec / Signal integrity topics including High-speed Digital Signal Integrity Modeling, Design, and Measurement; Power Distribution Network; System in Package (SiP)/System on Package (SoP) Design; High-performance Packaging for System on Chip (SoC); RF/Microwave Packaging for Wireless Communication and Mobile Phone; Interconnect Modeling, Simulation, and Measurement; Embedded Passives Modeling and Measurement; High-speed Channels Modeling and Measurement; EMI/EMC and Electromagnetic Modeling and Measurement; EDA Tools for Chip, Package, and Board Co-design and Simulation.
%		\item IEEE Real-Time Systems Symposium (RTSS 201X) -- Dec % has a special ``Design and Verification of Embedded Real-Time Systems'' track in 2009; see \url{http://www.rtss.org/} $[$due 26 May 2009$]$ % Modeling, evaluation and optimization of non-functional aspects such as timing, memory usage, communication bandwidth, and energy consumption... Design space exploration, performance analysis, and mapping of abstract designs onto target platforms such as time-triggered architectures and MPSoC... Model-based validation techniques ranging from simulation, testing, model-checking, compositional analysis, correctness-by-construction and abstract interpretation... Algorithms and techniques for the implementation of practical and scalable tools for modeling, automated analysis and optimization... Theories, languages and tools supporting coherent design flows spanning software, control, hardware and physical components... Case studies and success stories in industrial applications using existing techniques and tools for system design, analysis and verification.		...		scheduling; databases; observability; composability; security for real-time systems; tools and reduction to practice; control and adaptive RT systems theory; testing and debugging; modeling; formal methods; communications (wireless, wireline, and sensor networks); power, thermal, and energy management; embedded systems; sensor and implantable devices; robustness; fault tolerance and robustness; intelligent behavior; time-sensitive robotics; emergency/disaster management; embedded real-time systems and infrastructures; QoS support; real-time systems middleware, and cyber-physical systems.
%		\item IEEE Pacific Rim International Symposium on Dependable Computing (PRDC'1X) -- Dec % Software and hardware reliability, testing, verification and validation ... Dependability measurement, modeling and evaluation ... Security quantification ... Self-healing, self-protecting and fault-tolerant systems ... Software aging and rejuvenation ... Safety-critical systems and software ... Architecture and system design for dependability ... Fault tolerant algorithms and protocols ... Tools for design and evaluation of dependable systems ... Reliability in Internet and Web systems and applications ... Information security in Internet ... Dependability issues in computer networks and communications ... Dependability issues in distributed and parallel systems ... Dependability issues in real-time systems, database and transaction processing systems ... Dependability issues in autonomic computing ... Dependability issues in embedded systems
		\item \colorbox{yellow}{\bf To learn about more ``Call for Papers,'' see}: \vspace{-0.1cm}
			\begin{itemize} \itemsep -1pt
			\item {\it WikiCFP}: \url{http://www.wikicfp.com/}
			\item {\it Researchr}: \url{http://researchr.org/conferencecalendar/}
			\item {\it Conference Alerts}: \url{http://conferencealerts.com/}
			\item {\it SemreX CFP}: \url{http://grid.hust.edu.cn:8080/call/index.jsp}
			\item Department of Computer Engineering, Tallinn University of Technology: \url{http://ati.ttu.ee/calls_for_papers.php?show=by_deadline}
			\item Jonas Fritzin, ``List of VLSI / IC Conferences and Workshops,'' Division of Electronic Devices, Department of Electrical Engineering (ISY), Link{\"{o}}pings universitet: \url{http://www.ek.isy.liu.se/~fritzin/conf.html}
			\item Mathematical Optimization Society: conferences in mathematical optimization and mathematical programming; \url{http://www.mathprog.org/?nav=meetings}
			\end{itemize}
		\item Also, check out International Association of Science and Technology for Development (IASTED) conferences; {\tt these conferences and symposiums are of lower quality}
		\item See \url{http://www.mos-ak.org/} for events on compact modeling by MOS-AK.
		\item WARNING: Pay attention to rules about simultaneous submission and prior publication. For example, look at ACM's policy at the following reference. ACM, ``Policy on Prior Publication and Simultaneous Submissions,'' ACM, New York, NY, May 21, 2009. Available online at: \url{http://www.acm.org/publications/policies/sim_submissions}; last accessed on September 27, 2010.
		\end{enumerate}
%%%%%%%%%%%%%%%%%%%%%%%%%%%%%%%%%%%%%%%%%%%%%%%
%%%%%%%%%%%%%%%%%%%%%%%%%%%%%%%%%%%%%%%%%%%%%%%
%%%%%%%%%%%%%%%%%%%%%%%%%%%%%%%%%%%%%%%%%%%%%%%
	\item Include programming contests that I plan to participate in: \vspace{-0.2cm}
		\begin{enumerate} \itemsep -2pt
		\item ISPD Programming Contest -- Mar
		\item Memocode 201X HW/SW Co-design contest -- Mar
		\item IEEE Programming Challenge at IWLS -- Apr
		\item {\it Apple Design Awards -- Apr/May; also see WWDC 201X Student Scholarship (applications are due in mid-April)}
		\item Hardware Model Checking Competition (HWMCC) -- Jun/Jul
		\item CADathlon @ ICCAD -- Nov
		\end{enumerate}
	\item Include summer schools/programs or other short-term workshops that I can attend: \vspace{-0.2cm}
% See http://www.hipeac.net/ for more details
		\begin{enumerate} \itemsep -1pt
		\item To attend a Summer School, obtain permission of my advisor and follow the mission procedure presented in the Student Career informative system. 
		\item After having attended the Summer School, follow the procedure regarding Summer Schools credit recognition presented in the Student Career informative system. 
		\item Summer schools that may interest me: \vspace{-0.2cm}
			\begin{enumerate} \itemsep -1pt
			\item EPFL Summer Schools: \vspace{-0.1cm}
				\begin{itemize} \itemsep -1pt
				\item Nanoelectronic Circuits and Tools: \url{http://si.epfl.ch/page12409.html}
				\item Workshop and Nano-scale Systems and Technologies: \url{http://si.epfl.ch/page16490.html}
				\end{itemize}
			\item First International SAT/SMT Solver Summer School 2011: \url{http://people.csail.mit.edu/vganesh/summerschool/index.html}
			\item ARTIST Summer School --- See \url{http://www.artist-embedded.org/artist/Applications,1685.html} --- {\bf \color{green} \Huge APPLY BY MAY 15, 2009}
			\item CompArch Summer School on Parallel Programming and Architectures
			\item HiPEAC International Summer School on Advanced Computer Architecture and Compilation for Embedded Systems (Acaces) --- See \url{http://www.hipeac.net/summerschool/index.php?page=home} --- {\bf \color{green} \Huge APPLY ASAP}
			\item NSF-SIGDA-SRC Design Automation Summer School
			\item COMSON International Summer School on ``Modelling and Optimization in Micro- and Nano- Electronics -- MOMiNE 2009''; see \url{http://momine2009.unical.it/} {\color{green} \bf \Huge ... APPLY BY JUNE 30}
			\item RISC Summer 201X
			\item RISC/SCIEnce Training School in Symbolic Computation
			\item Summer School on Algebraic Analysis and Computer Algebra
			\item CRA-W/CDC Summer Schools: \vspace{-0.2cm}
				\begin{itemize} \itemsep -2pt
				\item CRA-W/CDC Computer Architecture Summer School
				\item See \url{CRA-W/CDC Computer Architecture Summer School} for a list of summer schools covering different topics in computer science. This summer schools are organized/co-organized by the Computer Research Association's Committee on the Status of Women in Computing Research (CRA-W).
				\end{itemize}
			\item Summer School of Information Engineering, Bressanone (BZ), Italia: \url{http://www.dei.unipd.it/ssie}
			\item See \url{http://user.it.uu.se/~bengt/Info/summer-schools.shtml} for a list of summer schools
			\end{enumerate}
		\end{enumerate}
	\item Internship(s): \vspace{-0.2cm}
		\begin{enumerate} \itemsep -2pt
		\item Include details of any (foreseen) collaboration with other departments or research institutions that I am collaborating with for my Ph.D. research and my internships, which is required for the CS Ph.D. at Uni Trento
		\item At the University of Trento's CS Ph.D. program, I have to complete internships overseas for: 2$\times$ 15 course credits. Since I have to complete 60 course credits a year, it means each internship shall last for 3 months. Thus, I have to complete 2$\times$ 3-month internships overseas.
		\end{enumerate}
	\item Suggested timeframe for writing the $[$remainder/bulk$]$ of my Ph.D. dissertation; save about 3-6 months for this
	\item Suggested timeframe for the {\tt Doctoral Thesis examination}; leave about 4-8 weeks to arrange a suitable date for the defense, when all members of the thesis committee can show up, and another 4-8 weeks to make revisions to my dissertation (if necessary)
	\item The Academic year begins on November, 1st and ends on October, 31st.
	\item summary of timeline: \vspace{-0.2cm}
		\begin{enumerate} \itemsep -2pt
		\item $1^{st}$ year Exploration research plan; Literature study
		\item	$2^{nd}$ Execution of research; Publication
		\item	$3^{rd}$ Execution of research; Publication
		\item $4^{th}$ Complete research; PhD Thesis
		\item In my $2^{nd}$ and $3^{rd}$ years, supervise master students and interns; tutorship (TA)
		\item Pass Quals in 3 months; absolute deadline to pass Quals (end of $9^{th}$ month)
		\end{enumerate}
	\end{enumerate}
\item (Explanatory) budget / funding plan: \vspace{-0.3cm}
	\begin{enumerate} \itemsep -4pt
	\item check out scholarships at \url{http://scholarship.bursa-lowongan.com/}
	\item suggestions on how to get funding: \url{http://www.aaai.org/Library/Funding/funding-library.php}
	\end{enumerate}
\item Estimate of required scientific, computing, and material resources: \vspace{-0.2cm}
	\begin{enumerate} \itemsep -2pt
	\item What resources will your research require?
	\end{enumerate}
\item References (max 20 for Uni. Trento)
\item This proposal will be revised through several rounds of iteration with the Ph.D. advisor(s)
% outline of Ph.D. research is to be iteratively revised with advisor
\item Judge my progress based on deliverables and performance/quality of the FV tool
\item Evaluate my research proposal to assess the following: \vspace{-0.2cm}
	\begin{enumerate} \itemsep -2pt
	\item {\bf \color{magenta} Ask myself the following: \vspace{-0.2cm}
		\begin{enumerate} \itemsep -1pt
		\item Have I solved the worthwhile problem, or answered the question?
		\item Has my thesis served as a formal document that proves that I have made an original and useful contribution to knowledge in EDA? That is, has my dissertation served its sole purpose?
		\end{enumerate} }
	\item {\bf \color{green} Does my Ph.D. thesis help the examiners on my thesis committee answer the following questions? } \vspace{-0.2cm}
		\begin{enumerate} \itemsep -1pt
		\item {\bf \color{green} What is this student's research question? }
		\item {\bf \color{green} Is it a good question? (Has it been answered before? Is it a useful question to work on?) }
		\item {\bf \color{green} Did the student convince me that the question was adequately answered? }
		\item {\bf \color{green} Has the student made an adequate contribution to knowledge? }
		\item If your thesis does not provide adequate answers to the few questions listed above, you will likely be faced with a requirement for major revisions or you may fail your thesis defense outright.
		\end{enumerate}
	\item Merit/Quality of the research $[$How innovative is this? Radically innovative???$]$ 
	\item Originality of the research
	\item Applicability of the proposed research: How useful would this research be?
	\item How reasonable is the budget?
	\end{enumerate}
\end{enumerate}
 
 
 
 
 
 
 \vspace{1cm}
 Resources for writing Ph.D. thesis proposals (and other research proposals): \vspace{-0.3cm}
\begin{enumerate} \itemsep -4pt
\item The University of Manchester: \vspace{-0.3cm}
	\begin{enumerate} \itemsep -2pt
	\item The University of Manchester, ``Advice on Writing a Research Proposal,'' School of Computer Science, The University of Manchester. Available at: \url{http://www.cs.manchester.ac.uk/phd/proposal/}; last accessed on September 2, 2010.
	\end{enumerate}
\end{enumerate}
 
 
 







%%%%%%%%%%%%%%%%%%%%%%%%%%%%%%%%%%%%%%%%%%%
\subsubsection{\hspace{0.1in} Notes And Advice For Being A Graduate Student}
\label{adviceforgradstudents}

Notes and advice for being a graduate student: \vspace{-0.3cm}
\begin{enumerate} \itemsep -4pt
\item focus on my research and its originality. And, don't pick a mentor who bogs me down with too many courses, obligations, and prerequisites
\item Find what I am passionate about, and pursue it regardless of what others expect me to do. Don't try to be a clone of my mentor or anyone else, find my own unique path and passions
\item once I identify my passion, and have a job consistent with it, work my butt off and produce, produce, produce.
\item don't be bashful about what I have to offer as a research engineer/scientist -- I've had great training. 
\item remember that I am more than my job; I don't live forever, so have some fun!
\item find a mentor whose commitment is to my professional development, and who will help me achieve what I are interested in
\item have a ``cuddle group'' -- a group of friends, colleagues, and classmates that can support me and who I can talk to about the challenges of being an intern, postdoc, or new faculty member or research engineer/scientist
\item get involved in the community -- volunteer at a domestic violence shelter, or advocate for policies or laws that matter to me ... make a difference!
\item motivation: doing excellent research gives me a better chance to get a research position in industry, or get a tenure-track position in academia immediately after completing my Ph.D.; otherwise, I may have to settle for postdoc positions, and avoid being mired in postdoctoral research for too long (and the constant scramble for grant money) ... postdoc positions that require me to carry out too many duties can also be snares along the way
\item establish a excellent research reputation (quality and quantity of publication record, and winning programming or IC design contests), and grant history to help me get a good research position after graduation
\item When I use \LaTeX\ to typeset any document, such as my dissertation or grant proposal, store each chapter/section/subsection of the document in a new file. This allows me to easily typeset only a chapter, section, or subsection of the document, and disseminate that chapter, section, or subsection to others for review. When people look at an individual chapter, section, or subsection, they may be more wiling to pay greater attention to the details and give me feedback on the document. Use these feedback when I update (correct and extend) my document
\item read my manuscript aloud to peers or have them read it to me -- it's amazing what errors you'll catch
\item attend as many research seminars and colloquiums/colloquia as I can
%%%%%%%%%%%%%%%%%%%%%%%%%%%%%%%%%%%%%%
\item Test my EDA tools with benchmark circuits, and use them to design VLSI circuits and systems. Send the designed VLSI circuits and systems to the following for manufacturing: \vspace{-0.3cm}
	\begin{enumerate} \itemsep -2pt
	\item MOSIS (Metal Oxide Semiconductor Implementation Service, MOSIS Integrated Circuit Fabrication Service): \url{http://www.mosis.com/}
	\item Circuits Multi-Projets\textregistered (or Multi-Project Circuits\textregistered), CMP: \url{http://cmp.imag.fr/}
	\item IMEC: \url{http://www2.imec.be/be_en/home.html}
	\end{enumerate}
\item Determine who the authorities (compare with key opinion leaders / KOLs, or opinion leaders) are in my field, and what are its seminal publications (or seminal/classic papers). That is, who are the true leaders (experts) in my field? Which publications are highly influential in my field? Available online at: \url{http://www.quora.com/How-do-I-find-the-seminal-papers-of-an-academic-field}; last accessed on September 25, 2010.
\item put videos of my talks at \url{http://videolectures.net/} or YouTube; e.g., I can put up videos of my talks at conferences, Quals, and my Ph.D. oral defense
%%%%%%%%%%%%%%%%%%%%%%%%%%%%%%%%%%%%%%%
\item ``DACeZine Profile: Sachin Sapatnekar,'' by Peggy Aycinena, in DACeZine, Volume 4, Issue 3, Dec. 3, 2008: \vspace{-0.2cm}
	\begin{enumerate} \itemsep -2pt
	\item \url{http://www.dac.com/newsletter/shownewsletter.aspx?newsid=64}
	\item For an engineer, the basic skill that needs to be mastered is problem solving. Take a physical problem, abstract it as a mathematical model, solve the abstraction, and map it back to reality. A really good abstraction achieves the necessary balance between the realism of the model and the tractability of the abstracted problem. Of course, this isn�t just an EDA concept -- it�s true of most areas of engineering, and this is what I urge my students to learn. The rest of what should be considered revolves around the technology itself -- something that�s critically important to comprehend, but is ever changing. I emphasize to my students that 10 years from today, they will be working on a completely different area than their current research, and they must learn the skills that make them adaptable.
	\item I�ve found the students who do work in this area to be highly motivated. EDA has always been a rich source of solution techniques that have made a broader impact -- think BDDs, verification, model-order reduction, and so on. Our problems are large in size, challenging in nature, and require great ingenuity to solve. This isn�t going to change anytime soon!
	\end{enumerate}
\item ``Andrew Kahng--Crafting the Future of EDA,'' By Peggy Aycinena, in DACeZine, Volume 4, Issue 2, Oct 16, 2008: \vspace{-0.2cm}
	\begin{enumerate} \itemsep -2pt
	\item \url{http://www.dac.com/newsletter/shownewsletter.aspx?newsid=53}
	\item Initiatives such as OpenAccess, plus the fact that core synthesis, timing, global placement, sizing, etc. technologies are getting long in the tooth, mean that many genies are out of the bottle and easier to match or improve on. So, new EDA companies can become serious players very fast -- for instance, Atoptech in the back end -- and it is easier than ever for customers to switch vendors in and out.
	\item With so much focus on polishing or re-implementing the current scope of ``cash cow'' subflows, not only does the EDA market fail to grow in system-level or die-package or embedded software or analog/mixed-signal design areas, but existing technologies become ever more commoditized and interchangeable.
	\item The big vendors have not figured out how to grab that big brass ring of royalty-based revenue -- although a couple of smaller vendors have -- so EDA remains a rental (and, service) business.
	\item If we�re going to solve the ESL, embedded software, multicore, analog, DFM, etc. challenges, we have to move beyond yet another SPICE parser, yet another shapes database, yet another back-end data model, and so on. Interoperability has proven impossible to incent, so perhaps customers must collectively create their own EDA development entity that generates and supports production-worthy infrastructure, to which EDA vendors can plug in or snap on.
	\item Will EDA companies ever improve their understanding of electronic system design, the chip design process, and software quality, so as to merit ``partner'' rather than ``supplier'' standing in the semiconductor supply chain?
	\item {\bf ``It's an old rant of mine is that tools shouldn't just be gigantic Swiss army knives. They must actively comprehend and define methodologies.''}
	\item most exciting growth areas going forward for young researchers in EDA: \vspace{-0.2cm}
		\begin{enumerate} \itemsep -2pt
		\item ``Multicore! System architectures, how to design them, how to program them, etc. These issues will continue to bring challenges for many, many years to come.  And, any list of buzz-phrases has to include system-level optimization, enablement of embedded software design, and verification.''
		\item ``Plus, we still require a science of design up through the system scale.  EDA has been an incredible m{\'{e}}lange of device physics, graph algorithms, heuristic optimization, computer and network architecture, etc.  Its culture is largely one of incremental improvement of point optimizations, without much global design context. We need better frameworks and we need better experimental and evaluation methodologies!''
		\item non-traditional technologies for computing still demand research on feasibility -- which requires design and optimization -- as well as scalable manufacturing
		\end{enumerate}
	\item advice for new grad students: \vspace{-0.2cm}
		\begin{enumerate} \itemsep -2pt
		\item My generic advice to students is to de-specialize! To be creative and aware, to learn physics, chemistry or biology -- and, optimization -- to learn how to program parallel and heterogeneous systems, and to always understand the context for what you do, and why you do it.
		\item If I were going to grad school today -- beyond depth in some research focus, I would seek breadth, breadth, and more breadth -- not just in CAD and VLSI, but allied areas as well.  
		\item The world will always need better and broader connections between the science or art of design and the underlying technology options and application needs.  Hugo De Man, in his memorable DAC 2000 keynote address, explained to us what a 21st-century design engineer should be able to do. Reference: Hugo De Man, ``System Design Challenges in the Post-PC Era,'' Keynote address, $37^{th}$ ACM/IEEE Design Automation Conference, Los Angeles, June 2000.
		\item Around that time, my friend Ranko Scepanovic was telling me how he was working on ``from C to OPC.'' It would have been great to have perfected such a toolkit during graduate school rather than piecing it together afterward.
		\item Of course, all graduate students need to work on their abilities to: 1) see big pictures and their choke points; 2) create practically relevant abstractions and problem formulations; and 3) innovate and validate impactful solutions to these problems.
		\end{enumerate}
	\item ``Given the massive changes in the geo-political and geo-economic landscape, do you see the geographic focus for cutting-edge CAD tool work remaining here in North America for the foreseeable future? If not, where has it moved to, or will it move to?'' -- Peggy Aycinena: \vspace{-0.2cm}
		\begin{enumerate} \itemsep -2pt
		\item ``I think we will inevitably see multiple foci, colored by regional investment: the EU and system-level design; migration of semiconductor production to Asia; migration of chip implementation services to India and China; etc. We�ll also see other factors at work, such as the macro trends and crossroads we were discussing earlier.'' -- Andrew Kahng
		\item ``One example of a non-U.S. focal point: Taiwan. Taiwan schools are very strong in the CADathlon held at ICCAD each year. They win the placement and global routing contests at ISPD. And, they can be expected to do very strong research in areas such as DFM. On the commercial side, we see the rapidly increasing visibility of SpringSoft.'' -- Andrew Kahng
		\item ``I believe, and I hope, that the U.S. university system and the continued attractiveness of North American schools to students from China, Korea, etc. will help keep the largest focal point here for a long time. But as you and I once discussed in an earlier conversation about `brain drains,' I think we are seeing that flows of engineering talent in general, and CAD tool innovation specifically, are starting to move away from the U.S.'' -- Andrew Kahng
		\end{enumerate}
	\item The world is evolving so quickly in so many ways. We see an unprecedented global economic context, consolidation of technologies and businesses, emergence of new product classes and markets, massive shifts toward cleantech and health/bio -- kind of a hurricane of macrotrends. But at the eye of this hurricane, more than ever before, our economic and societal health depends on technology agility and a continuing stream of innovative semiconductor-based products that bring compelling benefit to mankind (and, compelling value to consumers). To this end, I believe that EDA serves as a nexus that joins the product markets and driving applications, the designers, the semiconductor and packaging technologies, and the mathematical, engineering and algorithmic foundations of IC CAD. And in this context, we need DAC to be a heartbeat not only for the EDA industry and its users, but for the broader design chain as well.
	\end{enumerate}
\item ``If you're proposing a new statistical method, you should absolutely post code on the web for carrying out the technique. This doesn't have to be commercial-level code, and could even be production code with limited comments, but it should be enough that a knowledgeable statistician and programmer can figure out what you did, including all the detailed steps.'' -- Prof. Christopher Paciorek; see \url{http://www.biostat.harvard.edu/~paciorek/research/webPhilosophy/webPhilosophy.html}, last viewed on June 25, 2010.
\item map out my career, and keep updating that map!!!
\item make a list of research labs in my research area where I can intern or be a postdoc; keep the list updated!!!
\item $[$From \url{http://web.cecs.pdx.edu/~daasch/general.cgi?1+2+3#1}, by Prof. W. Robert Daasch$]$ A helpful reality check to track and evaluate the progress of any research project are four practical questions. Each question demands an answer sooner rather than later. To a limited extent these questions address the age-old (and generally unhelpful) discussion about a research effort being pure research or applied research. If the answer to any of the questions goes much beyond 100 words it is likely time to stop and think rather than do.: \vspace{-0.3cm}
	\begin{enumerate} \itemsep -2pt
	\item What is the problem to be solved?
	\item What are the system observables?
	\item Do the current research results describe the past behavior or predict the future behavior of the system?
	\item What is the difference between making progress and going in circles?
	\end{enumerate}
\end{enumerate}






%%%%%%%%%%%%%%%%%%%%%%%%%%%%%%%%%%%%%%%%%%%
\subsubsection{\hspace{0.1in} Notes and Advice For Grant Application}
\label{adviceforgrantapplication}

Notes and advice for grant application: \vspace{-0.3cm}
\begin{enumerate} \itemsep -4pt
\item learn how to write grants, which is an essential skill for researchers
\item the sooner I learn the vital craft of grant-writing, the better off I'll be: Grant money can help finance my education and, indeed, all of my future research
\item grant-writing is integral to graduate students who plan to go into academia when they finish school, but no one teaches you how to do it
\item advice for grant applications: \vspace{-0.2cm}
	\begin{enumerate} \itemsep -3pt
	\item start early: turn my research proposal/plan into a grant application
	\item get educated about the grant-writing process, and seek guidance from others
	\item learn about where can I get grants, how can I learn which grants are available, how difficult or easy are the grants to get, and what the grants will pay for
	\item Involve others: \vspace{-0.2cm}
		\begin{enumerate} \itemsep -2pt
		\item find professors in my area who have written successful grants and garner their advice
		\item get mentors, related experts, and fellow students to read drafts as I proceed
		\item don't hesitate to go outside your department
		\end{enumerate}
	\item Follow your interests: \vspace{-0.2cm}
		\begin{enumerate} \itemsep -2pt
		\item while I might be tempted to seek research dollars in a trendy funding area, stick with my most basic loves
		\item the topics that are of genuine interest to me will be around a lot longer than the research fads of the moment
		\item don't be shy: \vspace{-0.2cm}
			\begin{enumerate} \itemsep -2pt
			\item get advice from my mentor and others on the best people and institutes to contact
			\item don't hesitate to ask them for information and advice
			\item the one exception is that I can't contact reviewers when my application is under review
			\end{enumerate}
		\item follow the rules, such as formatting guidelines
		\item take the advice: \vspace{-0.2cm}
			\begin{enumerate} \itemsep -2pt
			\item if my application is returned with suggestions to revise it, follow reviewers' advice
			\item my reviewers will highlight strengths and areas for improvement, so when I resubmit my grant, make sure the areas for improvement have become strengths
			\item that doesn't mean I have to agree with everything the reviewers say, but I must clearly articulate in my revision why I disagree
			\end{enumerate}
		\item market my plan: \vspace{-0.2cm}
			\begin{enumerate} \itemsep -2pt
			\item make my message very clear in a way that helps them learn why my project is so important
			\item at the same time, don't brag or otherwise sell myself too aggressively
			\item I want to present myself as a thoughtful, serious fellow scientist
			\end{enumerate}
		\item be realistic: \vspace{-0.2cm}
			\begin{enumerate} \itemsep -2pt
			\item reviewers frequently critique young researchers' proposals for being overly ambitious
			\item my research plan should be conceptually tight, be well-linked to the literature, have clearly articulated hypotheses, and allow me to answer the questions the research hypotheses raise
			\item my proposal should be airtight, anticipating any questions the reviewer might have about decisions I made in designing my study
			\item reviewers are looking not just for the quality of research, but also at the quality of the applicant -- the way I think and the way I work
			\end{enumerate}
		\item persist: \vspace{-0.2cm}
			\begin{enumerate} \itemsep -2pt
			\item the advice ``Never give up; never surrender!'' applies to every phase of the grant application process, whether it's enlisting others' help, crafting my grant, or revising my proposal.
			\item do not be discouraged if my application is sent back for a rewrite: It is extremely common for grants to get accepted on the second or third round
			\item many applicants who apply for F31s research grants, and have to revise them end up securing them
			\item if I plan to garner research funds as my career goes on, I'll likely be doing plenty of rewriting and resubmitting
			\end{enumerate}
		\end{enumerate}
	\end{enumerate}
\item resources for writing grant proposals and applying for grants: \vspace{-0.3cm}
	\begin{enumerate} \itemsep -2pt
	\item Alfred P. Sloan Foundation: \url{http://www.sloan.org/apply/page/6}
	\item American Psychological Association: {\it gradPSYCH} Magazine: \vspace{-0.2cm}
		\begin{enumerate} \itemsep -2pt
		\item go to NIH's web page to view the section on writing a topnotch grant application; i.e., see \url{http://www.apa.org/gradpsych/2004/04/grant-resources.aspx} % \url{http://gradpsych.apags.org/apr04/grant-resources.cfm} (OLD link)
		\end{enumerate}
	\item {\it American Association for the Advancement of Science, AAAS}: \url{http://sciencecareers.sciencemag.org/tools_tips/how_to_series/how_to_get_funding} and \url{http://sciencecareers.sciencemag.org/funding} 
	\item The Mathematical Association of America, {\it Tips for Writing Successful Grant Proposals}, The Mathematical Association of America. Available at: \url{http://www.maa.org/programs/grantwriting/}; last accessed on September 2, 2010.
	\item European Science Foundation. {\it Grants, Calls, and Applications}. Available at: \url{http://www.esf.org/research-areas/physical-and-engineering-sciences/grants-calls-and-applications.html}; last accessed on September 2, 2010.
	\item European Commission: \vspace{-0.2cm}
		\begin{enumerate} \itemsep -2pt
		\item \url{http://cordis.europa.eu/fp7/home_en.html}
		\item Find calls at: \vspace{-0.1cm}
			\begin{enumerate} \itemsep -1pt
			\item \url{http://cordis.europa.eu/fp7/dc/index.cfm?fuseaction=UserSite.FP7CallsPage}
			\item \url{http://cordis.europa.eu/fp7/dc/index.cfm}
			\item Information and Communication Technologies (ICT): Information Society Technologies (IST), Future and Emerging Technologies (FET): \vspace{-0.3cm}
				\begin{itemize} \itemsep -2pt
				\item \url{http://cordis.europa.eu/fp7/dc/index.cfm?fuseaction=UserSite.FP7ActivityCallsPage&id_activity=3}
				\item \url{http://cordis.europa.eu/fp7/ict/participating/calls_en.html}
				\item \url{http://cordis.europa.eu/fp7/ict/home_en.html}
				% OLD Links::: \url{http://cordis.europa.eu/ist/fet/cal.htm} and \url{http://cordis.europa.eu/ist/fet/}
				\end{itemize}
			\end{enumerate}
		\end{enumerate}
	\item European Research Council: \url{http://erc.europa.eu/}
	\item UNESCO: \url{http://www.unesco.org/en/venice}
	\item University of Wisconsin-Madison: \vspace{-0.2cm}
		\begin{enumerate} \itemsep -2pt
		\item Grant Proposal Writing: \url{http://researchguides.library.wisc.edu/content.php?pid=16143&sid=108666}
		\end{enumerate}
	\item Montana State University, ``Some Reasons Proposals Fail,'' Office of Sponsored Programs, Montana State University. Available at: \url{http://www.montana.edu/wwwvr/osp/reasons.html}; last accessed on September 2, 2010.
	\item OpenContent Wiki, ``Grant Writing and Project Management 2010''. Available at: \url{http://opencontent.org/wiki/index.php?title=Grant_Writing_and_Project_Management_2010}; last accessed on September 4, 2010. [ Also available as: David Wiley, ``IP\&T 682, Grant Writing and Project Management (Instructional Psychology and Technology 682),'' Department of Instructional Psychology and Technology, David O. McKay School of Education, Brigham Young University.
	\item Semiconductor Research Corporation -- (anticipated) Calls for GRC White Papers: \url{http://www.src.org/compete/}
	\item CATRENE (Cluster for Application and Technology Research in Europe on NanoElectronics): \vspace{-0.2cm}
		\begin{enumerate} \itemsep -2pt
		\item Project Calls -- CATRENE Calls for Project Proposals: \url{http://www.catrene.org/web/calls/local_index.php}
		\item Resources and guidelines for project proposals: \url{http://www.catrene.org/web/calls/prepare_pp.php}
		\end{enumerate}
	\item Association for Computing Machinery (ACM): \vspace{-0.2cm}
		\begin{enumerate} \itemsep -2pt
		\item Sanjeev Arora, Boaz Barak, and Luca Trevisan, ``Funding Opportunities and Tips,'' in {\it Theory Matters Wiki: Theoretical Computer Science (TCS) Advocacy Wiki}, SIGACT Committee for the Advancement of Theoretical Computer Science, ACM Special Interest Group on Algorithms and Computation Theory (SIGACT), Association for Computing Machinery, February 25, 2010. Available at: \url{http://theorymatters.org/pmwiki/pmwiki.php?n=Main.FundingOpportunities}; last accessed on September 15, 2010.
		\end{enumerate}
	\item The Meadows Foundation: \vspace{-0.2cm}
		\begin{enumerate} \itemsep -2pt
		\item \url{http://www.mfi.org/}
		\item Provides grants for educational institutions: \url{http://www.mfi.org/display.asp?link=Z0WR3A}
		\end{enumerate}
	\item Broadcom Foundation: \vspace{-0.2cm}
		\begin{enumerate} \itemsep -2pt
		\item \url{http://www.broadcomfoundation.org/}
		\item Funding areas: \url{http://www.broadcomfoundation.org/about/#funding_areas}
		\item Can seeking funding for research projects to support research in science, technology, engineering, and mathematics (STEM), particularly for underrepresented minorities.
		\item Geographic Focus: \url{http://www.broadcomfoundation.org/about/#geographic_focus}
		\item Funding Guidelines: \url{http://www.broadcomfoundation.org/apply/#funding_guidelines}
		\end{enumerate}
	\item NYSTAR, New York State Foundation for Science, Technology and Innovation: \vspace{-0.2cm}
		\begin{enumerate} \itemsep -2pt
		\item NYSTAR Funding Initiatives: \url{http://www.nystar.state.ny.us/initiatives.htm}
		\end{enumerate}
	\end{enumerate}
\item It's no trivial item to say in my first job interview, ``I already have experience bringing in research dollars''
\item 
\end{enumerate}






%%%%%%%%%%%%%%%%%%%%%%%%%%%%%%%%%%%%%%%%%%%
\subsubsection{\hspace{0.1in} Advice for Teaching Assistants (TAs)}
\label{adviceforteachingassistants}

Advice for graduate teaching assistants (GTAs), or simply TAs: \vspace{-0.3cm}
\begin{enumerate} \itemsep -4pt
\item prepare for the classes with ample time to spare; prepare early, and know my stuff well ... if I don't know my stuff well, my students can/will assume that I do not know what I'm doing, and they'll not pay attention to the discussion session that I am leading ... think of the poor, inept TAs that I had
\item talk to TAs and faculty who have taught the same course that I will teach, and ask them for copies of their syllabi
\item sit in on the class that I'm going to teach
\item find out which faculty in the department are known for being good teachers, and ask them if I can watch them teach
\item attend conference workshops and seminars, and my department's formal GTA programs ... such as a three-day teaching orientation before the school year begins
\item observe other GTAs teaching
\item know my job requirements, and expectations from the course instructor and/or fellow TAs ... For example, do I have to come to class occasionally and record grades? Do I have to grade a hundred term papers and answer students' e-mails?How much time per week am I expected to work?
\item structure weekly or biweekly meetings where I can talk with the professor about student considerations, and how I think the class is responding to the teacher overall
\item budget time into my schedule to complete my teaching tasks
\item new TAs often underestimate how long it takes to design and grade tests, or answer student questions
\item develop an organization system that works for me and stick with it: use \LaTeX to typeset (lecture/study/class) notes, presentation slides, assignments (homework handouts), project descriptions, and examinations.
\end{enumerate}





%%%%%%%%%%%%%%%%%%%%%%%%%%%%%%%%%%%%%%%%%%%
\subsubsection{\hspace{0.1in} Writing a Ph.D. Thesis}
\label{writingaphdthesis}

Notes on writing a Ph.D. Thesis: \vspace{-0.3cm}
\begin{enumerate} \itemsep -4pt
\item Take at least one complete term for writing the thesis, so that I have sufficient time to organize my arguments and results
\item As I formalize my results into a well-organized thesis document, check if my thesis is capable of withstanding the scrutiny of expert examiners. Have I discovered any weaknesses in my thesis? How do I fix these weaknesses? Have I fixed these weaknesses? 
\item Spell difficult new concepts out clearly
\item Choose section titles and wordings to give the examiners clear directions to information they are seeking for in my thesis. Make it easy for my examiners to easily determine my problem statement, my defence of the problem, my answer to the problem, and my conclusions and contributions. Doing these can reduce the need for major revisions, and the number of minor revisions I have to make for my Ph.D. thesis.
\item Remember that a thesis is not a story: it usually doesn't follow the chronology of things that you tried. It's a formal document designed to answer only a few major questions such as: {\tt What is this student�s research question? Is it a good question? (Has it been answered\\ before? Is it a useful question to work on?) Did the student convince me that\\ the question was adequately answered? Has the student made an adequate\\ contribution to knowledge? } 
\item Avoid using phrases like ``Clearly, this is the case...'' or ``Obviously, it follows that ...''; these imply that, if the readers don't understand, then they must be stupid. They might not have understood because you explained it poorly. 
\item Avoid red flags, claims (like ``software is the most important part of a computer system'') that are really only your personal opinion and not substantiated by the literature or the solution you have presented. Examiners like to pick on sentences like that and ask questions like, ``Can you demonstrate that software is the most important part of a computer system?'' 
\item The purpose of your thesis is to clearly document an original contribution to knowledge. You may develop computer programs, prototypes, or other tools as a means of proving your points, but remember, the thesis is not about the tool, it is about the contribution to knowledge. Tools such as computer programs are fine and useful products, but you can't get an advanced degree just for the tool. You must use the tool to demonstrate that you have made an original contribution to knowledge; e.g., through its use, or ideas it embodies.
\item the dissertation can be viewed as a teaching exercise, in which I learn how to conduct, design, and analyze independent research.
\item commit 10 to 20 hours per week for 12 to 18 months to avoid becoming a casualty to the ``All But Dissertation (ABD)'' label
\item set specific work hours and choose a specific place to work
\item communicates my expectations and needs on topics such as deadlines, meeting frequency, authorship credit, and work hours ... For example, if I crave more praise or regular feedback than my advisor dishes out, I can confront the problem directly, such as by saying, ``I am getting discouraged: Tell me three things I am doing right here''
\item beware of these: procrastination, a tendency toward perfectionism, conflicts with advisers, lack of communication with dissertation committees, and failure to finish the dissertation
\item do not push aside unstructured tasks, such as the dissertation, in favor of more immediate tasks ... manage time well for these unstructured tasks
\item advice on how to reach the finish line: \vspace{-0.3cm}
	\begin{enumerate} \itemsep -4pt
	\item set priorities and goals: set daily, monthly, and yearly goals
	\item pick a timeline and stick to it
	\item know what you're getting into
	\item choose a manageable dissertation topic
	\item establish a network within the program
	\item be assertive and set boundaries when needed
	\item relax and have the right attitude
	\end{enumerate}
\item find a ``study-buddy'' to work with ... that way, I would be committed to working on my dissertation for myself and my ``study-buddy''
\item Introduction of a thesis: ``the first part of my thesis lays the groundwork necessary to carry out similar investigations in directions not considered by Cremona'' ... ``In part two of my thesis, I use the well-developed theory of modular forms, combined with the algorithms of part one, to investigate two outstanding conjectures.''
\item The contribution to knowledge of a Master's thesis can be in the nature of an incremental improvement in an area of knowledge, or the application of known techniques in a new area. The Ph.D. must be a substantial and innovative contribution to knowledge. 
\end{enumerate}


%%%%%%%%%%%%%%%%%%%%%%%%%%%%%%%%%%%%%%%%%%%
\subsubsection{\hspace{0.1in} Advice for Ph.D. Oral Defense}
\label{adviceforphdoraldefense}

Advice for the Ph.D. oral exam, oral defense (German: verteidigung/kolloquium), viva voce (/viva), or dissertation defense (/thesis defense) -- how to present a successful dissertation defense: \vspace{-0.3cm}
\begin{enumerate} \itemsep -4pt
\item Note that a shorter version of the oral defense can be included in a ``job talk'' (usually an hour-long lecture) that I would give during an interview for an academic position, except that the ``job talk'' would involve a presentation of my work to date as well as my future research plans ... The job talk shall not focus on the minutia of my past research, while leaving out the larger context of my work. Show attendees of the talk, professors and students (grad students and undergrads) that I have a vision, and I don't have to just stick to one study, or even my own work... Allow my personality to come through
\item don't worry about hardball questions
\item learn what's expected, anticipate the hard-hitting questions, open myself to feedback and, most importantly, remember to relax
\item enjoy myself in the Ph.D. oral defense ... Because it's not every day that I have a roomful of scholars completely interested in what I have to say -- it's something special I should enjoy
\item learn the rules and norms for a defense delivery for the department (or, specifically the Ph.D. program) that I am in ... determine my department's expectations by talking with my dissertation chair or fellow students: \vspace{-0.2cm}
	\begin{enumerate} \itemsep -2pt
	\item Am I expected to bring refreshments, or is that practice discouraged?
	\item Are I allowed to invite friends and family members, or is the defense open only to other graduate students or faculty?
	\item Should my presentation be 10 minutes or 30?
	\item Should I hand a final copy of my dissertation to my committee a month in advance, or is two weeks the norm?
	\item Usually, refreshments are not a requirement and defenses are open, but don't assume that's the norm for my department
	\item students are usually expected to book the room and date for their defense, which can take time ... Give myself a month to do that. It can be challenging to find a time when five busy faculty can meet
	\end{enumerate}
\item communicate with my committee members: \vspace{-0.2cm}
	\begin{enumerate} \itemsep -2pt
	\item talk regularly with my thesis committee about how my research is going
	\item my thesis committee can range from four to five members, depending on the program, who I hand-picked when I began the dissertation process
	\item consult them if I need to alter my methodology or circulate drafts, and to gather advance input if possible
	\item be open to the perspectives of the committee members ... but don't be shy about sharing my expertise or defending a point of view
	\item get as much feedback as time permits in both written commentary and in-person meetings with committee members ... It's a chance for them to ask the tough questions ahead of time
	\item document my progress by having committee members sign off on any major revisions they request at the proposal stage ... Taking that step can prevent confusion among faculty at the defense meeting about why dramatic changes were made
	\item don't go off on my own; if I do, I may fail my Ph.D. oral defense
	\item do what I agreed to do in my proposal, and communicate with my committee about changes
	\item contact with my committee can provide some valuable insight into the types of questions they might ask during the defense -- as can doing a little advance detective work
	\item know my committee members' likes, dislikes, and pet peeves
	\item ask people who have been through a defense with them, read their articles, and surf the Web for more information on their research expertise and specialty areas
	\end{enumerate}
\item practice and prepare: \vspace{-0.2cm}
	\begin{enumerate} \itemsep -2pt
	\item be prepared to present a clear explanation of why I did the study, a brief overview of my methodology and results, and a discussion of the implications of my research, but don't recite the manuscript
	\item the assumption is that my committee has already read my dissertation in detail ... I don't want to bore them by going through it again; I just want to refresh them
	\item at the same time, don't assume that my committee members have memorized my manuscript
	\item if they ask a question that I think I addressed, don't assume they remember that I addressed it; repeat myself patiently
	\item for the question-and-answer portion that follows the presentation, students should be primed to answer questions about their methodology, to defend and explain their choice of analysis, and discuss how their study contributes to the literature, informs theory and where the research might go next
	\item staging a mock defense with fellow graduate students is a great way to practice answering the types of questions I may be asked
	\item join a student-research group that regularly organizes practice defenses
	\item in some cases, the dry run will be more challenging than the defense; sometimes, students ask harder questions than faculty
	\item many students say attending another student's defense helps them prepare and know what to expect
	\item pick a well-prepared peer, since attending a defense that doesn't go well can be anxiety-provoking instead of helpful
	\item practice my talk in the room where I'll eventually defend
	\item know where I will move, look, sit, and take notes
	\item the less I have to react to in the moment, the more focused I can be on the task at hand, which is to demonstrate I have strong knowledge of my project
	\end{enumerate}
\item develop the right attitude: \vspace{-0.2cm}
	\begin{enumerate} \itemsep -2pt
	\item approaching the defense as a critically constructive experience is key
	\item avoid coming off as too protective about my work during the meeting, but to also not be overly compliant about committee members' feedback
	\item students should be open to the perspectives of the committee members, who are committed to helping improve their dissertation, but they shouldn't be shy about sharing their expertise or defending a point of view if they feel their committee may be misinformed
	\item what's more, students shouldn't feel discouraged if their committee asks for minor revisions to their manuscript after the defense
	\item it's not at all uncommon for committee members to suggest a different analysis, some changes in a table, or to rework the discussion section to clarify a certain point ... students sometimes have a couple days work ahead of them to put it in final shape
	\end{enumerate}
\item breathe, then answer: \vspace{-0.2cm}
	\begin{enumerate} \itemsep -2pt
	\item stumped by a question? ... Don't be afraid to take a moment to consider it, paraphrase it back for clarification or ask that it be restated
	\item similarly, if I don't know the answer, it's better to say so and give the best answer I can, rather than digressing for a few minutes
	\item keep in mind that for the most part faculty are just asking questions to see if I can think critically -- they are not trying to be difficult or stump me
	\item staying calm can be one of your greatest assets during the defense
	\item it's normal to be anxious and scared about my defense, but many people before me have passed, and I can too
	\end{enumerate}
\item stage a mock dissertation defense with fellow students to practice answering questions.
\end{enumerate}





%%%%%%%%%%%%%%%%%%%%%%%%%%%%%%%%%%%%%%%%%%%
\subsubsection{\hspace{0.1in} Advice for Applying for Postdoc Positions}
\label{adviceforpostdocapplications}

Advice for securing postdoc positions, and achieving postdoc success: \vspace{-0.3cm}
\begin{enumerate} \itemsep -4pt
\item start my applications early ($\geq$1 year before expected date of Ph.D. oral defense)
\item Set goals: \vspace{-0.2cm}
	\begin{enumerate} \itemsep -2pt
	\item Before I begin my search, decide what I want to get out of the experience
	\item Do I want to learn new skills?
	\item Acquire in-depth training in an area of expertise?
	\item Get great mentoring?
	\item Look ahead to what I think I want to be doing, and work backwards from there
	\end{enumerate}
\item Do your homework: \vspace{-0.2cm}
	\begin{enumerate} \itemsep -2pt
	\item Thoroughly investigate potential positions
	\item Will the position fit with the goals I've set?
	\item Will I get the mentoring and training I need?
	\item Are there health benefits?
	\item Have others been happy there?
	\item Talk with former postdocs and other psychologists to find out
	\end{enumerate}
\item Don't give up: \vspace{-0.2cm}
	\begin{enumerate} \itemsep -2pt
	\item even if it seems too late in the year, keep applying and networking
	\item new funding or postdocs who backed out of a position could open up opportunities
	\end{enumerate}
\item Think ahead: \vspace{-0.2cm}
	\begin{enumerate} \itemsep -3pt
	\item Once I'm in a position, keep my next steps in mind
	\item Where is my first ``regular'' job going to come from?
	\item If I don't get a job right away, what will I do?: \vspace{-0.2cm}
		\begin{enumerate} \itemsep -2pt
		\item engage in open research independently, and with others
		\item publish lots of papers ... publish, publish, publish!!!
		\item engage in open source software development
		\item engage in open source IC design
		\item advice students on their honors (senior capstone) and Masters projects
		\item Where do I want to go after my postdoc?
		\item What happens if that excellent but elusive job isn't available?
		\end{enumerate}
	\end{enumerate}
\item Find a career mentor: \vspace{-0.2cm}
	\begin{enumerate} \itemsep -2pt
	\item Link myself with a mentor who can help me through the job search
	\item a mentor can be an early-career/seasoned professor or research scientist
	\item Someone who can honestly discuss what job opportunities look like, what the appropriate things to ask for, and say, and do are
	\end{enumerate}
\item Spell out my expectations: \vspace{-0.2cm}
	\begin{enumerate} \itemsep -2pt
	\item Discuss my duties and goals with my supervisors before I take the position
	\item Some even advise writing up a contract that includes what the supervision will entail, what skills or training I'm going to get from the fellowship, and how much autonomy will I get in my research projects ... This is helpful for organizations without an established training program for postdocs; get our goals and expectations articulated ... Beware of make-it-up as I go positions
	\end{enumerate}
\item Check my progress: \vspace{-0.2cm}
	\begin{enumerate} \itemsep -2pt
	\item once I start my supervision, regularly discuss my progress toward my goals with my supervisor
	\item write down my fellowship goals at the beginning of the year, and constantly referred back to them to stay on track during my postdoc program(s) ... Determine if I have the flexibility to alter them
	\item ask the right questions, and be proactive about it
	\item seek and earn more degrees of freedom to try to get some additional training
	\end{enumerate}
\item regard pursuing a postdoc research fellowship as taking that step toward what I really want to be doing as an independent investigator; it may take more than one postdoc to make that transition
\item Mentor fit: \vspace{-0.2cm}
	\begin{enumerate} \itemsep -2pt
	\item think about the qualities of my Ph.D. advisor that are good for me and that help me work well, and look for those qualities in a postdoctoral mentor
	\item research gets done better and more quickly when I really click with the person I'm working with
	\item Has the researcher provided good mentoring to other fellows?
	\item Have fellows published papers with the mentor?
	\item What was the authorship order of those papers?
	\item Talk with current and former postdocs, interns, and staff to get a feel for how things work
	\item think about it as I am hiring this mentor to work for me ... do the vetting of the prospective mentor before I start talking to him/her about writing a grant together
	\item consider whether the mentor will allow me the level of independence I need ... if I am transitioning to a new research area, I'm not ready for complete independence; however, when I'm ready to work autonomously, I need to work in a lab that gives me the freedom that I need ... In this case, a different lab may be a better fit
	\end{enumerate}
\item Funding: \vspace{-0.2cm}
	\begin{enumerate} \itemsep -2pt
	\item How will my research be funded?
	\item If I'll be on a grant, how long will the grant last? Is that enough time to complete my work?
	\item I may need to begin applying for the next year's funding shortly after I begin my fellowship.
	\item Two common funding approaches are to get written into an investigator's research grant or, to show I am competitive with others, to write my own research grant $[$such as those for NSF/SRC in the US$]$
	\end{enumerate}
\item Goals and expectations: \vspace{-0.2cm}
	\begin{enumerate} \itemsep -2pt
	\item Talk over my fellowship objectives with my mentor and seek agreement, perhaps writing a contract that seals it.
	\item Once in my fellowship, continually evaluate my progress based on those goals
	\item However, that doesn't mean my objectives can't shift
	\item fellowships often represent the last time I have such freedom to explore my various research interests as a researcher
	\item In terms of my intellectual development, nothing beats it
	\item I don't have faculty meetings, don't have to teach. I can just focus on my research.
	\end{enumerate}
\end{enumerate}



%%%%%%%%%%%%%%%%%%%%%%%%%%%%%%%%%%%%%%%%%%%
\subsubsection{\hspace{0.1in} Resources for Postdoc Positions}
\label{resourcesforpostdocpositions}

Resources for postdoc positions: \vspace{-0.3cm}
\begin{enumerate} \itemsep -4pt
\item Inside Higher Ed: \vspace{-0.3cm}
	\begin{enumerate} \itemsep -2pt
	\item \url{http://www.insidehighered.com/}
	\item Has information concerning the application for faculty positions, and advice for career paths in higher education institutions. 
	\item Kerry Ann Rockquemore, ``Winning Tenure Without Losing Your Soul: Stop Talking, Start Walking,'' Inside Higher Ed: Career Advice, Inside Higher Ed, January 25, 2010. Available at: \url{http://www.insidehighered.com/advice/winning/winning2}; last accessed on September 7, 2010.
	\end{enumerate}
\item Zoe Smith and Ariana Sutton-Grier, ``Making the Most of Your Postdoc,'' The Chronicle of Higher Education: Advice: Do Your Job Better, July 15, 2010. Available at: \url{http://chronicle.com/article/Making-the-Most-of-Your/66265/}; last accessed on September 6, 2010.
\item National Postdoctoral Association: \vspace{-0.3cm}
	\begin{enumerate} \itemsep -2pt
	\item Resources on Becoming a Postdoc: \url{http://www.nationalpostdoc.org/graduate-students}
	\item Faculty \& Administrators: \url{http://www.nationalpostdoc.org/faculty-administrators}
	\item Diversity Programs \& Resources: \url{http://www.nationalpostdoc.org/diversity-issues}
	\item International Postdocs: \url{http://www.nationalpostdoc.org/international-issues}
	\item The NPA Postdoctoral Core Competencies Toolkit (NPA Core Competencies): \url{http://www.nationalpostdoc.org/competencies}
	\end{enumerate}
\item Research Foundation -- Flanders (FWO), or ``Fonds voor Wetenschappelijk Onderzoek -- Vlaanderen'' (FWO): \vspace{-0.2cm}
	\begin{enumerate} \itemsep -2pt
	\item See \url{http://www.fwo.be/en/index.aspx}
	\item This is sponsored by the National Fund for Scientific Research (Belgium) 
	\item Research Foundation - Flanders (FWO): five-yearly prizes: \vspace{-0.2cm}
		\begin{enumerate} \itemsep -2pt
		\item Two Dr. A. De Leeuw-Damry-Bourlart prizes (100,000 Euro): For exact and applied sciences.
		\item Two Dr. Joseph Maisin prizes: For fundamental biomedical sciences and clinical biomedical sciences.
		\item The Ernest John Solvay prize: For humanities and social sciences.
		\end{enumerate}
	\item The Research Foundation � Flanders (FWO) pays a researcher�s income for the following categories: \vspace{-0.2cm}
		\begin{enumerate} \itemsep -2pt
		\item Ph.D. student
		\item Postdoctoral researcher
		\item Clinical fellowship
		\end{enumerate}
	\end{enumerate}
\item {\it PhDjobs.com}: \url{http://www.phdjobs.com/}
\item {\it innovation report}: \vspace{-0.3cm}
	\begin{enumerate} \itemsep -2pt
	\item Jobs (in industry and academia): \url{http://www.innovations-report.com/jobs/jobs.php}
	\end{enumerate}
\item Princeton University: \vspace{-0.3cm}
	\begin{enumerate} \itemsep -2pt
	\item Department of Computer Science, School of Engineering and Applied Science (SEAS): \vspace{-0.2cm}
		\begin{enumerate} \itemsep -2pt
		\item Jeff Erickson and Boaz Barak, {\it TCS opportunities: Postdocs and other positions in theoretical computer science}, Center for Computational Intractability, Department of Computer Science, School of Engineering and Applied Science (SEAS), Princeton University. Available at: \url{http://intractability.princeton.edu/jobs/}; last accessed on September 15, 2010.
		\end{enumerate}
	\end{enumerate}
\item Simons Foundation: Simons Postdoctoral Fellows Program, \url{https://simonsfoundation.org/funding-guidelines/simons-postdoctoral-fellows-program}
\item IBM Postdoctoral Fellowships: \vspace{-0.3cm}
	\begin{enumerate} \itemsep -2pt
	\item IBM Herman Goldstine Postdoctoral Fellowship in Mathematical and Computer Sciences; see \url{http://domino.research.ibm.com/comm/research_projects.nsf/pages/goldstine.index.html}
	\item Josef Raviv Memorial Postdoctoral Fellowship; see \url{http://domino.research.ibm.com/comm/research.nsf/pages/d.compsci.josef.raviv.general.info.html}, \url{http://domino.research.ibm.com/comm/research.nsf/pages/d.compsci.raviv.winner.html}, and \url{http://domino.research.ibm.com/comm/research.nsf/pages/d.compsci.raviv.winner2008.html}
	\end{enumerate}
\item Computing Innovation Fellows (CIFellows); post my profile on \url{http://cifellows.org/profiles/}; also see \url{http://www.cifellows.org/}
\end{enumerate}






%%%%%%%%%%%%%%%%%%%%%%%%%%%%%%%%%%%%%%%%%%%
\subsubsection{\hspace{0.1in} Resources Concerning the Application for Junior Faculty Positions}
\label{resourcesforjnrprof}

Resources concerning the application for junior faculty positions: \vspace{-0.3cm}
\begin{enumerate} \itemsep -4pt
\item Mailing lists of the Computer Research Association's Committee on the Status of Women in Computing Research (CRA-W): \vspace{-0.3cm}
	\begin{enumerate} \itemsep -2pt
	\item Subscribe to these and receive announcements concerning openings for junior faculty positions.
	\item \url{http://www.cra-w.org/mailinglists}
	\item \url{http://www.cra-w.org/PhdjobhuntHers}
	\end{enumerate}
\item Computing Research Association (CRA): \vspace{-0.3cm}
	\begin{enumerate} \itemsep -2pt
	\item \url{http://www.cra.org/for-students/}
	\item CRA's Job Announcements: \url{http://www.cra.org/ads/}
	\item Computing Postdoc Job Opportunities: \url{http://cifellows.org/opportunities}
	\item Computing Postdoc Profiles: \url{http://cifellows.org/profiles}
	\end{enumerate}
\item Inside Higher Ed: \url{http://www.insidehighered.com/career/seekers}
\item American Association of University Professors: \vspace{-0.3cm}
	\begin{enumerate} \itemsep -2pt
	\item Career Center: \url{http://careercenter.aaup.org/search.cfm}
	\item Issues in Higher Education: \url{http://www.aaup.org/AAUP/issues/}
	\end{enumerate}
\item The Chronicle of Higher Education: \vspace{-0.3cm}
	\begin{enumerate} \itemsep -2pt
	\item Global Jobs: \url{http://chronicle.com/section/Global-Jobs/434/}
	\item Great Colleges to Work For: \vspace{-0.2cm}
		\begin{enumerate} \itemsep -2pt
		\item \url{http://chronicle.com/section/Great-Colleges-to-Work-For/156/}
		\item \url{http://chronicle.com/section/The-Academic-Workplace/156}
		\item \url{http://chronicle.com/article/Great-Colleges-to-Work-For/65724/}
		\end{enumerate}
	\item CV Doctor: \url{http://chronicle.com/article/The-CV-Doctor-Is-Back/49086/}
	\end{enumerate}
\item Times Higher Education (THE): \vspace{-0.3cm}
	\begin{enumerate} \itemsep -2pt
	\item Advanced Job Search: \url{http://www.timeshighereducation.co.uk/jobs_home.asp?navCode=84}
	\item Career: \url{http://www.timeshighereducation.co.uk/section.asp?navcode=96}
	\end{enumerate}
\item {\it Mathematical Association of America} (MAA): \vspace{-0.3cm}
	\begin{enumerate} \itemsep -2pt
	\item Information for new Ph.D.s seeking academic careers
	\item \url{http://www.maa.org/careers/}
	\item MAA Math Classifieds: \url{http://www.mathclassifieds.org/home/index.cfm?site_id=1925}
	\end{enumerate}
\item American Mathematical Society: \vspace{-0.3cm}
	\begin{enumerate} \itemsep -2pt
	\item Information on applying for positions in academia: \url{http://www.ams.org/programs/students/programs/students/gradinfo/gradinfo}
	\item \url{http://www.ams.org/profession/career-info/new-phds/new-phds}
	\end{enumerate}
\item Association for Women in Science (AWIS) career library: \url{http://www.awis.affiniscape.com/displaycommon.cfm?an=1&subarticlenbr=249}
\item American Psychological Association's resources for grad students and postdocs: \url{http://www.apa.org/education/grad/index.aspx}
\item {\it HigherEdJobs.com}: \url{http://www.higheredjobs.com/}
\item IEEE Real World Engineering Projects (RWEP): \vspace{-0.3cm}
	\begin{enumerate} \itemsep -2pt
	\item resources to help plan and run projects in EECS and related fields that will benefit humanity
	\item \url{http://www.realworldengineering.org/}
	\end{enumerate}
\item iBerry (collection of job lists): \url{http://iberry.com/cms/jobs.htm}
\item Common CV Network (for research and academic positions in Canada): \url{http://www.cvcommun.net/index_e.html}
\item University of Pennsylvania: \vspace{-0.3cm}
	\begin{enumerate} \itemsep -2pt
	\item Stephanie Weirich, {\it Computer Science Faculty Job Search Resources}, Department of Computer and Information Science, School of Engineering and Applied Science, University of Pennsylvania. Available at: \url{http://www.seas.upenn.edu/~sweirich/resources.htm}; last accessed on September 5, 2010.
	\end{enumerate}
\item {\tt Blue Lab Coats}, {\it Academic Job Applications}. Available at: \url{http://bluelabcoats.wordpress.com/application-pkgs/}; last accessed on September 10, 2010.
\item Advice on giving the job talk: \vspace{-0.3cm}
	\begin{enumerate} \itemsep -2pt
	\item John Farrell, ``What to Say in a Good Research Talk,'' Department of Computer Science, James Cook University, May 1994. Available at: \url{http://www.computersciencestudent.com/SS/HowTo/ResearchTalkJohnFarrell.html}; last accessed on August 25, 2010.
	\end{enumerate}
\item residential education: \vspace{-0.3cm}
	\begin{enumerate} \itemsep -2pt
	\item Telluride Association: \vspace{-0.2cm}
		\begin{enumerate} \itemsep -2pt
		\item Information about how to become a Faculty Fellow at the Cornell Branch and the Michigan Branch of Telluride Association, which are ``residential colleges'': \url{http://www.tellurideassociation.org/programs/college_faculty.html}
		\end{enumerate}
	\end{enumerate}
\item Awards, grants, and fellowships to look out for: \vspace{-0.3cm}
	\begin{enumerate} \itemsep -2pt
	\item NSF Career Grant / NSF Career Award
	\item Heinz Family Philanthropies, The Heinz Awards: \url{http://www.heinzawards.net/awards}
	\item Sloan Research Fellowships: \vspace{-0.2cm}
		\begin{enumerate} \itemsep -2pt
		\item Applicants must ``hold a Ph.D. (or equivalent) in chemistry, physics, mathematics, computer science, economics, neuroscience or computational and evolutionary molecular biology, or in a related interdisciplinary field''
		\item Applicants must be tenure-track junior faculty in a North American college or university
		\item \url{http://www.sloan.org/fellowships}
		\end{enumerate}
	\item Santa Fe Institute's Miller Distinguished Scholarship (for excellent interdisciplinary researchers): \url{http://www.santafe.edu/research/miller-scholars/}
	\item Lemelson-MIT Prize (for mid-career innovators): \url{http://web.mit.edu/invent/a-prize.html}
	\item Research Corporation for Science Advancement: \vspace{-0.2cm}
		\begin{enumerate} \itemsep -2pt
		\item Cottrell College Science Awards (for science researchers at predominantly undergraduate US colleges and universities): \url{http://www.rescorp.org/cottrell-college-science-awards}
		\item Cottrell Scholar Awards (for science researchers at US colleges and universities): \url{http://www.rescorp.org/cottrell-scholar-awards/}
		\end{enumerate}
	\item {\it Semiconductor Industry Association} University Researcher Awards: \url{http://www.sia-online.org/cs/papers_publications/press_release_detail?pressrelease.id=1722}
	\item Packard Fellowships for Science and Engineering (for professors at selected US universities): \url{http://www.packard.org/genericDetails.aspx?RootCatID=3&CategoryID=152}
	\item {\it American Society for Engineering Education} awards: \url{http://www.asee.org/activities/awards/division.cfm}
	\item Hellman Fellowship in Science and Technology Policy (for early-career professionals with training in science or engineering who are interested in transitioning to a career in public policy and administration): \url{http://www.amacad.org/hellman.aspx}
	\item Simons Foundation's {\it Collaboration Grants for Mathematicians}: \url{https://simonsfoundation.org/funding-guidelines/current-funding-opportunities/collaboration-grants-for-mathematicians}
	\item The National Science Foundation: \vspace{-0.2cm}
		\begin{enumerate} \itemsep -2pt
		\item Alan T. Waterman Award (for junior faculty in the US): \url{http://www.nsf.gov/od/waterman/waterman.jsp} 
		\end{enumerate}
	\item Computing Research Association (CRA): A. Nico Habermann Award, \url{http://www.cra.org/awards/habermann-current/}
	\item Microsoft New Faculty Fellowship Award: \url{http://research.microsoft.com/en-us/collaboration/awards/msrff_all.aspx#2005}
	\end{enumerate}
\item Underrepresented minority outreach: \vspace{-0.3cm}
	\begin{enumerate} \itemsep -2pt
	\item The Mathematical Association of America: \vspace{-0.2cm}
		\begin{enumerate} \itemsep -2pt
		\item Pre-College Programs: \url{http://www.maa.org/funding/pre_college.html}
		\item Programs for Undergraduate Students: \url{http://www.maa.org/funding/undergraduate.html}
		\item Special Interest Group of the MAA on Research in Undergraduate Mathematics Education (SIGMAA RUME): \url{http://sigmaa.maa.org/rume/}. Also, see \url{http://sigmaa.maa.org/rume/highlights.html}.
		\end{enumerate}
	\end{enumerate}
\end{enumerate}






%%%%%%%%%%%%%%%%%%%%%%%%%%%%%%%%%%%%%%%%%%%%%
\end{document}

